\documentclass{report}
\usepackage{Custom_Latex/Summary_Notes/notes}
\addtolength{\oddsidemargin}{-.5in}
\addtolength{\evensidemargin}{-.5in}
\addtolength{\textwidth}{1in}
\usepackage{pgf}

\graphicspath{{./Images/}}


\begin{document}
\title{IADS - CW 2}
\author{Maksymilian Mozolewski}
\maketitle
\pagebreak
\section{Algorithm}
\subsection{Introduction}
The algorithm I want to present, is a mix of innovation and old heuristics. I decided to create a heuristic which required setting parameters, and so naturally ended up experimenting with different values and comparing my custom heuristic to the others presented.
\subsection{Explanation}
The heuristic is a mix of greedy, 2-opt and a nearest insertion heuristic \cite{nearest}. Instead of the classic greedy, I used "Greedy with lookahead" which given l - the lookahead parameter, a set of visited nodes, and a tour, will look for the shortest sub-tour starting at the end of the permutation with length l. the nearest insertion heuristic, is similar to greedy, but instead of looking at just the last node in the permutation, it finds the nearest unvisited node to any of the nodes present in the tour already, and looks for an edge into which the node can be inserted to minimize the total cost of the tour. The custom heuristic performs both greedy lookahead and nearest insertion, and commits to the tour which results in the smaller increase in cost, while taking into acount a "bias" parameter which can sway its choice - I could have picked the one which minimizes cost per new node, however decided not to as I intended for smaller lookahead values to be preferred - after each lengthening of the tour, the heuristic performs a number of 2-opt optimisations on the tour for as long as they improve the cost or untill we hit the maximum number of optimisations: which is another parameter.
\subsection{Pseudocode}

\definecolor{codegreen}{rgb}{0,0.6,0}
\definecolor{codegray}{rgb}{0.3,0.3,0.3}
\definecolor{codepurple}{rgb}{0.2,0,0.82}
\definecolor{b}{rgb}{0.58,0,0.82}
\definecolor{d}{rgb}{0.58,0,0.82}
\definecolor{back}{rgb}{0.95,0.95,0.95}
\lstdefinestyle{mystyle}{
    language = none,
    backgroundcolor=\color{back},
    commentstyle=\color{codegray},
    keywordstyle=\color{codepurple},
    numberstyle=\tiny\color{black},
    stringstyle=\color{codegreen},
    basicstyle=\ttfamily\footnotesize,
    breakatwhitespace=false,
    breaklines=true,
    captionpos=b,
    keepspaces=true,
    numbers=left,
    numbersep=2pt,
    showspaces=false,
    showstringspaces=false,
    showtabs=false,
    tabsize=2,
    numberfirstline=true,
    literate=
        {=}{$\leftarrow{}$}{1}
        {==}{$={}$}{1}
        {leq}{$\leq{}$}{1}
        {geq}{$\geq{}$}{1}
        {cup}{$\cup{}$}{1},
    morekeywords={then,Algorithm,Function,while,if,else},
}
\lstset{style=mystyle}

\begin{lstlisting}
Algorithm Custom (o,l,b)
    perm = [0] 
    visited = {0}
    n = number of nodes in the graph

    while perm.length < n
        
        if bias is 1 or 0, only carry out nearest,greedy and update tour,
        otherwise:
            subTour,costDelta1 = GreedyLookahead(perm,visited,l)
            insertionIdx,insertValue,costDelta2 = NearestInsertion(perm,visited)
            
            if costDelta1*bias < costDelta2*(1-bias) 
                perm = perm1,subTour
            else
                perm = perm.insert(insertionId,insertValue)

        opts = 0
        better = True
        while better and opts < o
            better = False 
            for j from 0 to perm.length -1
                for i from 0 to j
                    better = tryReverse(i,j)
            opts = opts + 1
            
\end{lstlisting}

\subsection{Runtime Analysis}
Let m be the length of current subtour, n the number of nodes in the graph, and o,l,b the maximum 2-opt runs, lookahead and bias.\\\\
With nearest insertion we first find a node closest to one of the nodes in the current subtour, this in the worst case is $O(nm)$.
We then check the m edges to find the one which minimizes the cost, which is at worst $O(m)$.
This totals a worst case performance of $O(nm)$
\\\\
Greedy with lookahead is a little more complicated, but essentially it's described by the recurrence $T(n,l,m) = (n-m-1)T(n,l-1,m+1) + O(n)$ when $l \neq 0 \And m < n$ or $O(1)$ otherwise. We can calculate the total running time by noting that the work on each level h can be captured with $W(h) = \frac{(n-m-1)!}{(n-m-1-h)!}O(n)$, which summed over the worst-case height of the tree: l is:
\begin{equation*}
    \sum_{i = 0}^{l}\frac{(n-m-1)!}{(n-m-1-i)!}O(n) = O(n) + (n-m-1)O(n)+ (n-m-1)(n-m-2)O(n)+\hdots = O(n) + O(n^{2}) + O(n^{3})+\hdots = O(n^{l})
\end{equation*}
\\\\
With 2-opt, we check if reversing any part of the permutation improves it, and if it does, we commit to the reversing. We only use indices up to the diagonal, so we have $m^{2}$ possible reversals, and each reversal, if committed to would take $\theta(m)$ time, so a single optimization round of 2-opt in the worst case will take $O(m^{2} + m) = O(m^{2})$ operations, and so the complexity here at worst is $O(om^{2})$.
\\\\
So the worst case running time of the whole algorithm is given by: 
\begin{equation*}
    \sum^{n}_{m = 1}(O(n^{l}) + O(om^{2}) + O(nm) + O(1)) = O(n^{l+1} +o(\frac{n(n+1)}{2})^{2} + \frac{n^{2}(n+1)}{2} + n) = O(n^{l+1} + on^{4} + n^{3})
\end{equation*}
Which is clearly polynomial since l and o are constants with $l \geq 1, o \geq 0$. And when the bias is set to 0 or 1, the running time becomes: respectively $O(n^{l+1} + n^{3})$,$O(on^{4} + n^{3})$
\section{Experiments}
\subsection{Testing set up}
The heuristics were tested on random graphs in 3 categories: euclidian-metric,non-euclidian-metric,non-euclidian-non-metric (ER,MR,NMR).\\\\
The ER graphs were generated by selecting random n points with a spread proportional to the number of nodes. MR graphs were generated by picking a value x proportional to the number of nodes, and for each edge a random value in the range [x,2x], which meant that the graphs were not necessarily Euclidian, but each set of 3 edges satisfied the triangle inequality. NMR graphs were generated by just chosing random numbers for the distances, proportional to the number of nodes in the graphs. I decided to keep the spread proportional to number of nodes since for random uniform tests, altering the spread would likely not have any meaningful impact, apart from "scaling" the output values uniformly anyway, and very small values in distances would not be ideal.\\\\
In total I had 500 graphs where every 5 graphs the number of nodes increased by 1, giving 100 different sized graphs in each problem type.
I then measured the Average Tour Value across all graphs, the Average Tour Value across tests with the same size, The run-times and also recorded the resulting permutations for all heuristics. After collecting all the data (around 70 different files/28Mb), I graphed some of it. The default values used for parameters were 10,1,0.5 for opts,lookahead and bias respectively, and were varied for certain experiments.

\subsection{Hypothesis}
My hypothesis was that since my heuristic incorporated parts of Greedy and two-opt, that it would find a solution somewhat close to either of them, and that for non-metric graphs it would fare much better if run with a higher lookahead value, since in non-metric graphs, a path might end up being shorter if you remove one of its nodes, and so I thought that looking ahead more than 1 node unlike Greedy, would yield great results. I also expected the bias to have quite a big effect on the tours.
\subsection{Results}
The custom heuristic turned out to be consistently better than pure greedy, see \bd{Figure 1}, which was not surprising since custom was a modified version of greedy with a layer of optimisations. I did wonder if the fact that 2-opt changed the last element of the permutation with some probability on each iteration would have a negative impact on the quality of the tour, but it turns out that wasn't the case. I figure that the optimisation each iteration had a "restarting effect" on the tour, where greedy/insertion were performed on a better or identical tour, and so custom could not perform worse than just greedy\\\\
\bd{Figure 1} also shows that the swap heuristic performs much better in MR graphs than any other, and in those problem spaces performs reasonably well. This is quite interesting and surprising, especially since I'd expect that if a heuristic performs well on euclidian graphs then it should perform quite similarly on metric graphs.\\\\
It turned out that the custom heuristic was sometimes better than 2-opt by itself, however on average 2-opt was still slightly better. The big surprise to me was that even though 2-opt has a much worse time complexity (some papers state $O(n^{10})$ )\cite{2-opt} it has been much faster than my algorithm as seen in \bd{Figure 6} - you can see that the different problem spaces didn't really change runtimes, and that custom had the steepest runtime curve, while 2-opt was 2nd worse. One explanation is that 2-opt optimisation runs faster on smaller inputs up to some size which I have not recorded, or that on this domain of problems 2-opt optimisation has a different worst case runtime - the cited study generated graphs on the unit square.\\\\
an interesting fact is that my heuristic seems to suffer the same problems as greedy - it generates bad paths whenever the closest points form a path in such a way that there are leftover points that the heuristic needs to travel back to. However this weakness is softened by the usage of 2-opt which smooths out the tour. It also seems to beat 2-opt in situations where there aren't too many points "left behind", and where greedy would have failed badly, the optimisations can recover the tour into a much more efficient one. See \bd{Figures 3 and 4}.
\\\\
A big surprise to me, was how little the nearest insertion heuristic helped, the additional information really did not seem to impact the tour values positively at all, in fact using the custom heuristic with bias set for just greedy, showed slightly better results, and varying the bias showed no significant impact. This is probably due to 2-opt optimisations run at each iteration "converging" the tours to similar local optima even with the different bias parameters\\\\
It's interesting to see that the bias and lookahead did not have a serious effect on the average tour costs, which was quite unexpected, I only show \bd{Figure 5}, and leave out the graphs inspecting bias, as they show the same pattern and are uninteresting. I think that this could be due to the fact that the a higher lookahead value just ended up returning similar paths to those which greedy would take, and 2-opt "converged" the differences to the same local optima; or that higher lookahead meant that nearest insertion was preferred, due to higher cost deltas involved in greedy lookaheads, which meant that the results ended up being similar to the runs involving a bias towards insertions - and changing the bias showed no impact on the averages, meaning that the lookahead parameter was not really significant and only increased the runtime by a factor of $n^{l}$.
\\\\
\subsection{Conclusions}
The custom heuristic works better than greedy if running time is not very important, if one needs a quick and relatively good heuristic, just using greedy or 2-opt seems to be the best choice out of the studied heuristics. Custom works best with bias set to 0, lookahead set to 1 and optimisations set to 1. The Swap heuristic becomes a reasonable choice (not necessarily a good one) as a heuristic but only for MR graphs.
\subsection{Data}
\begin{figure}[p]
    \caption{Results from performing each of the heuristics on 500 RE,MR,NMR graphs with increasing numbers of nodes, and widths proportional to their size. The graph shows average tour values for each graph size}
    \centering
    \scalebox{0.5}{\input{Years/Year2/IADS/CW3/Images/Heuristics.pgf}}
\end{figure}
\begin{figure}[p]
    \caption{Results from performing custom on 500 RE graphs with varying values for the optimisations parameter.}
    \centering
    \scalebox{0.5}{%% Creator: Matplotlib, PGF backend
%%
%% To include the figure in your LaTeX document, write
%%   \input{<filename>.pgf}
%%
%% Make sure the required packages are loaded in your preamble
%%   \usepackage{pgf}
%%
%% and, on pdftex
%%   \usepackage[utf8]{inputenc}\DeclareUnicodeCharacter{2212}{-}
%%
%% or, on luatex and xetex
%%   \usepackage{unicode-math}
%%
%% Figures using additional raster images can only be included by \input if
%% they are in the same directory as the main LaTeX file. For loading figures
%% from other directories you can use the `import` package
%%   \usepackage{import}
%%
%% and then include the figures with
%%   \import{<path to file>}{<filename>.pgf}
%%
%% Matplotlib used the following preamble
%%   \usepackage{fontspec}
%%   \setmainfont{DejaVuSerif.ttf}[Path=/home/maks/.local/share/virtualenvs/CW3-zMJxnm_q/lib/python3.7/site-packages/matplotlib/mpl-data/fonts/ttf/]
%%   \setsansfont{DejaVuSans.ttf}[Path=/home/maks/.local/share/virtualenvs/CW3-zMJxnm_q/lib/python3.7/site-packages/matplotlib/mpl-data/fonts/ttf/]
%%   \setmonofont{DejaVuSansMono.ttf}[Path=/home/maks/.local/share/virtualenvs/CW3-zMJxnm_q/lib/python3.7/site-packages/matplotlib/mpl-data/fonts/ttf/]
%%
\begingroup%
\makeatletter%
\begin{pgfpicture}%
\pgfpathrectangle{\pgfpointorigin}{\pgfqpoint{13.660000in}{7.340000in}}%
\pgfusepath{use as bounding box, clip}%
\begin{pgfscope}%
\pgfsetbuttcap%
\pgfsetmiterjoin%
\definecolor{currentfill}{rgb}{1.000000,1.000000,1.000000}%
\pgfsetfillcolor{currentfill}%
\pgfsetlinewidth{0.000000pt}%
\definecolor{currentstroke}{rgb}{1.000000,1.000000,1.000000}%
\pgfsetstrokecolor{currentstroke}%
\pgfsetdash{}{0pt}%
\pgfpathmoveto{\pgfqpoint{0.000000in}{0.000000in}}%
\pgfpathlineto{\pgfqpoint{13.660000in}{0.000000in}}%
\pgfpathlineto{\pgfqpoint{13.660000in}{7.340000in}}%
\pgfpathlineto{\pgfqpoint{0.000000in}{7.340000in}}%
\pgfpathclose%
\pgfusepath{fill}%
\end{pgfscope}%
\begin{pgfscope}%
\pgfsetbuttcap%
\pgfsetmiterjoin%
\definecolor{currentfill}{rgb}{1.000000,1.000000,1.000000}%
\pgfsetfillcolor{currentfill}%
\pgfsetlinewidth{0.000000pt}%
\definecolor{currentstroke}{rgb}{0.000000,0.000000,0.000000}%
\pgfsetstrokecolor{currentstroke}%
\pgfsetstrokeopacity{0.000000}%
\pgfsetdash{}{0pt}%
\pgfpathmoveto{\pgfqpoint{1.499429in}{0.839159in}}%
\pgfpathlineto{\pgfqpoint{13.339844in}{0.839159in}}%
\pgfpathlineto{\pgfqpoint{13.339844in}{6.806869in}}%
\pgfpathlineto{\pgfqpoint{1.499429in}{6.806869in}}%
\pgfpathclose%
\pgfusepath{fill}%
\end{pgfscope}%
\begin{pgfscope}%
\pgfsetbuttcap%
\pgfsetroundjoin%
\definecolor{currentfill}{rgb}{0.000000,0.000000,0.000000}%
\pgfsetfillcolor{currentfill}%
\pgfsetlinewidth{0.803000pt}%
\definecolor{currentstroke}{rgb}{0.000000,0.000000,0.000000}%
\pgfsetstrokecolor{currentstroke}%
\pgfsetdash{}{0pt}%
\pgfsys@defobject{currentmarker}{\pgfqpoint{0.000000in}{-0.048611in}}{\pgfqpoint{0.000000in}{0.000000in}}{%
\pgfpathmoveto{\pgfqpoint{0.000000in}{0.000000in}}%
\pgfpathlineto{\pgfqpoint{0.000000in}{-0.048611in}}%
\pgfusepath{stroke,fill}%
}%
\begin{pgfscope}%
\pgfsys@transformshift{2.037629in}{0.839159in}%
\pgfsys@useobject{currentmarker}{}%
\end{pgfscope}%
\end{pgfscope}%
\begin{pgfscope}%
\definecolor{textcolor}{rgb}{0.000000,0.000000,0.000000}%
\pgfsetstrokecolor{textcolor}%
\pgfsetfillcolor{textcolor}%
\pgftext[x=2.037629in,y=0.741937in,,top]{\color{textcolor}\rmfamily\fontsize{10.000000}{12.000000}\selectfont \(\displaystyle 0\)}%
\end{pgfscope}%
\begin{pgfscope}%
\pgfsetbuttcap%
\pgfsetroundjoin%
\definecolor{currentfill}{rgb}{0.000000,0.000000,0.000000}%
\pgfsetfillcolor{currentfill}%
\pgfsetlinewidth{0.803000pt}%
\definecolor{currentstroke}{rgb}{0.000000,0.000000,0.000000}%
\pgfsetstrokecolor{currentstroke}%
\pgfsetdash{}{0pt}%
\pgfsys@defobject{currentmarker}{\pgfqpoint{0.000000in}{-0.048611in}}{\pgfqpoint{0.000000in}{0.000000in}}{%
\pgfpathmoveto{\pgfqpoint{0.000000in}{0.000000in}}%
\pgfpathlineto{\pgfqpoint{0.000000in}{-0.048611in}}%
\pgfusepath{stroke,fill}%
}%
\begin{pgfscope}%
\pgfsys@transformshift{4.212178in}{0.839159in}%
\pgfsys@useobject{currentmarker}{}%
\end{pgfscope}%
\end{pgfscope}%
\begin{pgfscope}%
\definecolor{textcolor}{rgb}{0.000000,0.000000,0.000000}%
\pgfsetstrokecolor{textcolor}%
\pgfsetfillcolor{textcolor}%
\pgftext[x=4.212178in,y=0.741937in,,top]{\color{textcolor}\rmfamily\fontsize{10.000000}{12.000000}\selectfont \(\displaystyle 20\)}%
\end{pgfscope}%
\begin{pgfscope}%
\pgfsetbuttcap%
\pgfsetroundjoin%
\definecolor{currentfill}{rgb}{0.000000,0.000000,0.000000}%
\pgfsetfillcolor{currentfill}%
\pgfsetlinewidth{0.803000pt}%
\definecolor{currentstroke}{rgb}{0.000000,0.000000,0.000000}%
\pgfsetstrokecolor{currentstroke}%
\pgfsetdash{}{0pt}%
\pgfsys@defobject{currentmarker}{\pgfqpoint{0.000000in}{-0.048611in}}{\pgfqpoint{0.000000in}{0.000000in}}{%
\pgfpathmoveto{\pgfqpoint{0.000000in}{0.000000in}}%
\pgfpathlineto{\pgfqpoint{0.000000in}{-0.048611in}}%
\pgfusepath{stroke,fill}%
}%
\begin{pgfscope}%
\pgfsys@transformshift{6.386726in}{0.839159in}%
\pgfsys@useobject{currentmarker}{}%
\end{pgfscope}%
\end{pgfscope}%
\begin{pgfscope}%
\definecolor{textcolor}{rgb}{0.000000,0.000000,0.000000}%
\pgfsetstrokecolor{textcolor}%
\pgfsetfillcolor{textcolor}%
\pgftext[x=6.386726in,y=0.741937in,,top]{\color{textcolor}\rmfamily\fontsize{10.000000}{12.000000}\selectfont \(\displaystyle 40\)}%
\end{pgfscope}%
\begin{pgfscope}%
\pgfsetbuttcap%
\pgfsetroundjoin%
\definecolor{currentfill}{rgb}{0.000000,0.000000,0.000000}%
\pgfsetfillcolor{currentfill}%
\pgfsetlinewidth{0.803000pt}%
\definecolor{currentstroke}{rgb}{0.000000,0.000000,0.000000}%
\pgfsetstrokecolor{currentstroke}%
\pgfsetdash{}{0pt}%
\pgfsys@defobject{currentmarker}{\pgfqpoint{0.000000in}{-0.048611in}}{\pgfqpoint{0.000000in}{0.000000in}}{%
\pgfpathmoveto{\pgfqpoint{0.000000in}{0.000000in}}%
\pgfpathlineto{\pgfqpoint{0.000000in}{-0.048611in}}%
\pgfusepath{stroke,fill}%
}%
\begin{pgfscope}%
\pgfsys@transformshift{8.561274in}{0.839159in}%
\pgfsys@useobject{currentmarker}{}%
\end{pgfscope}%
\end{pgfscope}%
\begin{pgfscope}%
\definecolor{textcolor}{rgb}{0.000000,0.000000,0.000000}%
\pgfsetstrokecolor{textcolor}%
\pgfsetfillcolor{textcolor}%
\pgftext[x=8.561274in,y=0.741937in,,top]{\color{textcolor}\rmfamily\fontsize{10.000000}{12.000000}\selectfont \(\displaystyle 60\)}%
\end{pgfscope}%
\begin{pgfscope}%
\pgfsetbuttcap%
\pgfsetroundjoin%
\definecolor{currentfill}{rgb}{0.000000,0.000000,0.000000}%
\pgfsetfillcolor{currentfill}%
\pgfsetlinewidth{0.803000pt}%
\definecolor{currentstroke}{rgb}{0.000000,0.000000,0.000000}%
\pgfsetstrokecolor{currentstroke}%
\pgfsetdash{}{0pt}%
\pgfsys@defobject{currentmarker}{\pgfqpoint{0.000000in}{-0.048611in}}{\pgfqpoint{0.000000in}{0.000000in}}{%
\pgfpathmoveto{\pgfqpoint{0.000000in}{0.000000in}}%
\pgfpathlineto{\pgfqpoint{0.000000in}{-0.048611in}}%
\pgfusepath{stroke,fill}%
}%
\begin{pgfscope}%
\pgfsys@transformshift{10.735822in}{0.839159in}%
\pgfsys@useobject{currentmarker}{}%
\end{pgfscope}%
\end{pgfscope}%
\begin{pgfscope}%
\definecolor{textcolor}{rgb}{0.000000,0.000000,0.000000}%
\pgfsetstrokecolor{textcolor}%
\pgfsetfillcolor{textcolor}%
\pgftext[x=10.735822in,y=0.741937in,,top]{\color{textcolor}\rmfamily\fontsize{10.000000}{12.000000}\selectfont \(\displaystyle 80\)}%
\end{pgfscope}%
\begin{pgfscope}%
\pgfsetbuttcap%
\pgfsetroundjoin%
\definecolor{currentfill}{rgb}{0.000000,0.000000,0.000000}%
\pgfsetfillcolor{currentfill}%
\pgfsetlinewidth{0.803000pt}%
\definecolor{currentstroke}{rgb}{0.000000,0.000000,0.000000}%
\pgfsetstrokecolor{currentstroke}%
\pgfsetdash{}{0pt}%
\pgfsys@defobject{currentmarker}{\pgfqpoint{0.000000in}{-0.048611in}}{\pgfqpoint{0.000000in}{0.000000in}}{%
\pgfpathmoveto{\pgfqpoint{0.000000in}{0.000000in}}%
\pgfpathlineto{\pgfqpoint{0.000000in}{-0.048611in}}%
\pgfusepath{stroke,fill}%
}%
\begin{pgfscope}%
\pgfsys@transformshift{12.910370in}{0.839159in}%
\pgfsys@useobject{currentmarker}{}%
\end{pgfscope}%
\end{pgfscope}%
\begin{pgfscope}%
\definecolor{textcolor}{rgb}{0.000000,0.000000,0.000000}%
\pgfsetstrokecolor{textcolor}%
\pgfsetfillcolor{textcolor}%
\pgftext[x=12.910370in,y=0.741937in,,top]{\color{textcolor}\rmfamily\fontsize{10.000000}{12.000000}\selectfont \(\displaystyle 100\)}%
\end{pgfscope}%
\begin{pgfscope}%
\definecolor{textcolor}{rgb}{0.000000,0.000000,0.000000}%
\pgfsetstrokecolor{textcolor}%
\pgfsetfillcolor{textcolor}%
\pgftext[x=7.419636in,y=0.551968in,,top]{\color{textcolor}\rmfamily\fontsize{10.000000}{12.000000}\selectfont Number of nodes}%
\end{pgfscope}%
\begin{pgfscope}%
\pgfsetbuttcap%
\pgfsetroundjoin%
\definecolor{currentfill}{rgb}{0.000000,0.000000,0.000000}%
\pgfsetfillcolor{currentfill}%
\pgfsetlinewidth{0.803000pt}%
\definecolor{currentstroke}{rgb}{0.000000,0.000000,0.000000}%
\pgfsetstrokecolor{currentstroke}%
\pgfsetdash{}{0pt}%
\pgfsys@defobject{currentmarker}{\pgfqpoint{-0.048611in}{0.000000in}}{\pgfqpoint{0.000000in}{0.000000in}}{%
\pgfpathmoveto{\pgfqpoint{0.000000in}{0.000000in}}%
\pgfpathlineto{\pgfqpoint{-0.048611in}{0.000000in}}%
\pgfusepath{stroke,fill}%
}%
\begin{pgfscope}%
\pgfsys@transformshift{1.499429in}{1.110419in}%
\pgfsys@useobject{currentmarker}{}%
\end{pgfscope}%
\end{pgfscope}%
\begin{pgfscope}%
\definecolor{textcolor}{rgb}{0.000000,0.000000,0.000000}%
\pgfsetstrokecolor{textcolor}%
\pgfsetfillcolor{textcolor}%
\pgftext[x=1.332762in, y=1.057657in, left, base]{\color{textcolor}\rmfamily\fontsize{10.000000}{12.000000}\selectfont \(\displaystyle 0\)}%
\end{pgfscope}%
\begin{pgfscope}%
\pgfsetbuttcap%
\pgfsetroundjoin%
\definecolor{currentfill}{rgb}{0.000000,0.000000,0.000000}%
\pgfsetfillcolor{currentfill}%
\pgfsetlinewidth{0.803000pt}%
\definecolor{currentstroke}{rgb}{0.000000,0.000000,0.000000}%
\pgfsetstrokecolor{currentstroke}%
\pgfsetdash{}{0pt}%
\pgfsys@defobject{currentmarker}{\pgfqpoint{-0.048611in}{0.000000in}}{\pgfqpoint{0.000000in}{0.000000in}}{%
\pgfpathmoveto{\pgfqpoint{0.000000in}{0.000000in}}%
\pgfpathlineto{\pgfqpoint{-0.048611in}{0.000000in}}%
\pgfusepath{stroke,fill}%
}%
\begin{pgfscope}%
\pgfsys@transformshift{1.499429in}{2.031626in}%
\pgfsys@useobject{currentmarker}{}%
\end{pgfscope}%
\end{pgfscope}%
\begin{pgfscope}%
\definecolor{textcolor}{rgb}{0.000000,0.000000,0.000000}%
\pgfsetstrokecolor{textcolor}%
\pgfsetfillcolor{textcolor}%
\pgftext[x=1.124428in, y=1.978864in, left, base]{\color{textcolor}\rmfamily\fontsize{10.000000}{12.000000}\selectfont \(\displaystyle 1000\)}%
\end{pgfscope}%
\begin{pgfscope}%
\pgfsetbuttcap%
\pgfsetroundjoin%
\definecolor{currentfill}{rgb}{0.000000,0.000000,0.000000}%
\pgfsetfillcolor{currentfill}%
\pgfsetlinewidth{0.803000pt}%
\definecolor{currentstroke}{rgb}{0.000000,0.000000,0.000000}%
\pgfsetstrokecolor{currentstroke}%
\pgfsetdash{}{0pt}%
\pgfsys@defobject{currentmarker}{\pgfqpoint{-0.048611in}{0.000000in}}{\pgfqpoint{0.000000in}{0.000000in}}{%
\pgfpathmoveto{\pgfqpoint{0.000000in}{0.000000in}}%
\pgfpathlineto{\pgfqpoint{-0.048611in}{0.000000in}}%
\pgfusepath{stroke,fill}%
}%
\begin{pgfscope}%
\pgfsys@transformshift{1.499429in}{2.952833in}%
\pgfsys@useobject{currentmarker}{}%
\end{pgfscope}%
\end{pgfscope}%
\begin{pgfscope}%
\definecolor{textcolor}{rgb}{0.000000,0.000000,0.000000}%
\pgfsetstrokecolor{textcolor}%
\pgfsetfillcolor{textcolor}%
\pgftext[x=1.124428in, y=2.900071in, left, base]{\color{textcolor}\rmfamily\fontsize{10.000000}{12.000000}\selectfont \(\displaystyle 2000\)}%
\end{pgfscope}%
\begin{pgfscope}%
\pgfsetbuttcap%
\pgfsetroundjoin%
\definecolor{currentfill}{rgb}{0.000000,0.000000,0.000000}%
\pgfsetfillcolor{currentfill}%
\pgfsetlinewidth{0.803000pt}%
\definecolor{currentstroke}{rgb}{0.000000,0.000000,0.000000}%
\pgfsetstrokecolor{currentstroke}%
\pgfsetdash{}{0pt}%
\pgfsys@defobject{currentmarker}{\pgfqpoint{-0.048611in}{0.000000in}}{\pgfqpoint{0.000000in}{0.000000in}}{%
\pgfpathmoveto{\pgfqpoint{0.000000in}{0.000000in}}%
\pgfpathlineto{\pgfqpoint{-0.048611in}{0.000000in}}%
\pgfusepath{stroke,fill}%
}%
\begin{pgfscope}%
\pgfsys@transformshift{1.499429in}{3.874040in}%
\pgfsys@useobject{currentmarker}{}%
\end{pgfscope}%
\end{pgfscope}%
\begin{pgfscope}%
\definecolor{textcolor}{rgb}{0.000000,0.000000,0.000000}%
\pgfsetstrokecolor{textcolor}%
\pgfsetfillcolor{textcolor}%
\pgftext[x=1.124428in, y=3.821278in, left, base]{\color{textcolor}\rmfamily\fontsize{10.000000}{12.000000}\selectfont \(\displaystyle 3000\)}%
\end{pgfscope}%
\begin{pgfscope}%
\pgfsetbuttcap%
\pgfsetroundjoin%
\definecolor{currentfill}{rgb}{0.000000,0.000000,0.000000}%
\pgfsetfillcolor{currentfill}%
\pgfsetlinewidth{0.803000pt}%
\definecolor{currentstroke}{rgb}{0.000000,0.000000,0.000000}%
\pgfsetstrokecolor{currentstroke}%
\pgfsetdash{}{0pt}%
\pgfsys@defobject{currentmarker}{\pgfqpoint{-0.048611in}{0.000000in}}{\pgfqpoint{0.000000in}{0.000000in}}{%
\pgfpathmoveto{\pgfqpoint{0.000000in}{0.000000in}}%
\pgfpathlineto{\pgfqpoint{-0.048611in}{0.000000in}}%
\pgfusepath{stroke,fill}%
}%
\begin{pgfscope}%
\pgfsys@transformshift{1.499429in}{4.795247in}%
\pgfsys@useobject{currentmarker}{}%
\end{pgfscope}%
\end{pgfscope}%
\begin{pgfscope}%
\definecolor{textcolor}{rgb}{0.000000,0.000000,0.000000}%
\pgfsetstrokecolor{textcolor}%
\pgfsetfillcolor{textcolor}%
\pgftext[x=1.124428in, y=4.742485in, left, base]{\color{textcolor}\rmfamily\fontsize{10.000000}{12.000000}\selectfont \(\displaystyle 4000\)}%
\end{pgfscope}%
\begin{pgfscope}%
\pgfsetbuttcap%
\pgfsetroundjoin%
\definecolor{currentfill}{rgb}{0.000000,0.000000,0.000000}%
\pgfsetfillcolor{currentfill}%
\pgfsetlinewidth{0.803000pt}%
\definecolor{currentstroke}{rgb}{0.000000,0.000000,0.000000}%
\pgfsetstrokecolor{currentstroke}%
\pgfsetdash{}{0pt}%
\pgfsys@defobject{currentmarker}{\pgfqpoint{-0.048611in}{0.000000in}}{\pgfqpoint{0.000000in}{0.000000in}}{%
\pgfpathmoveto{\pgfqpoint{0.000000in}{0.000000in}}%
\pgfpathlineto{\pgfqpoint{-0.048611in}{0.000000in}}%
\pgfusepath{stroke,fill}%
}%
\begin{pgfscope}%
\pgfsys@transformshift{1.499429in}{5.716454in}%
\pgfsys@useobject{currentmarker}{}%
\end{pgfscope}%
\end{pgfscope}%
\begin{pgfscope}%
\definecolor{textcolor}{rgb}{0.000000,0.000000,0.000000}%
\pgfsetstrokecolor{textcolor}%
\pgfsetfillcolor{textcolor}%
\pgftext[x=1.124428in, y=5.663693in, left, base]{\color{textcolor}\rmfamily\fontsize{10.000000}{12.000000}\selectfont \(\displaystyle 5000\)}%
\end{pgfscope}%
\begin{pgfscope}%
\pgfsetbuttcap%
\pgfsetroundjoin%
\definecolor{currentfill}{rgb}{0.000000,0.000000,0.000000}%
\pgfsetfillcolor{currentfill}%
\pgfsetlinewidth{0.803000pt}%
\definecolor{currentstroke}{rgb}{0.000000,0.000000,0.000000}%
\pgfsetstrokecolor{currentstroke}%
\pgfsetdash{}{0pt}%
\pgfsys@defobject{currentmarker}{\pgfqpoint{-0.048611in}{0.000000in}}{\pgfqpoint{0.000000in}{0.000000in}}{%
\pgfpathmoveto{\pgfqpoint{0.000000in}{0.000000in}}%
\pgfpathlineto{\pgfqpoint{-0.048611in}{0.000000in}}%
\pgfusepath{stroke,fill}%
}%
\begin{pgfscope}%
\pgfsys@transformshift{1.499429in}{6.637661in}%
\pgfsys@useobject{currentmarker}{}%
\end{pgfscope}%
\end{pgfscope}%
\begin{pgfscope}%
\definecolor{textcolor}{rgb}{0.000000,0.000000,0.000000}%
\pgfsetstrokecolor{textcolor}%
\pgfsetfillcolor{textcolor}%
\pgftext[x=1.124428in, y=6.584900in, left, base]{\color{textcolor}\rmfamily\fontsize{10.000000}{12.000000}\selectfont \(\displaystyle 6000\)}%
\end{pgfscope}%
\begin{pgfscope}%
\definecolor{textcolor}{rgb}{0.000000,0.000000,0.000000}%
\pgfsetstrokecolor{textcolor}%
\pgfsetfillcolor{textcolor}%
\pgftext[x=1.068872in,y=3.823014in,,bottom,rotate=90.000000]{\color{textcolor}\rmfamily\fontsize{10.000000}{12.000000}\selectfont Average Tour Cost}%
\end{pgfscope}%
\begin{pgfscope}%
\pgfpathrectangle{\pgfqpoint{1.499429in}{0.839159in}}{\pgfqpoint{11.840415in}{5.967710in}}%
\pgfusepath{clip}%
\pgfsetrectcap%
\pgfsetroundjoin%
\pgfsetlinewidth{1.505625pt}%
\definecolor{currentstroke}{rgb}{0.121569,0.466667,0.705882}%
\pgfsetstrokecolor{currentstroke}%
\pgfsetdash{}{0pt}%
\pgfpathmoveto{\pgfqpoint{2.037629in}{1.110419in}}%
\pgfpathlineto{\pgfqpoint{2.146357in}{1.115889in}}%
\pgfpathlineto{\pgfqpoint{2.255084in}{1.120673in}}%
\pgfpathlineto{\pgfqpoint{2.363812in}{1.123588in}}%
\pgfpathlineto{\pgfqpoint{2.472539in}{1.135380in}}%
\pgfpathlineto{\pgfqpoint{2.581266in}{1.135174in}}%
\pgfpathlineto{\pgfqpoint{2.689994in}{1.144408in}}%
\pgfpathlineto{\pgfqpoint{2.798721in}{1.159319in}}%
\pgfpathlineto{\pgfqpoint{2.907449in}{1.161956in}}%
\pgfpathlineto{\pgfqpoint{3.016176in}{1.196211in}}%
\pgfpathlineto{\pgfqpoint{3.124903in}{1.200751in}}%
\pgfpathlineto{\pgfqpoint{3.233631in}{1.233851in}}%
\pgfpathlineto{\pgfqpoint{3.342358in}{1.236529in}}%
\pgfpathlineto{\pgfqpoint{3.451086in}{1.245578in}}%
\pgfpathlineto{\pgfqpoint{3.559813in}{1.288423in}}%
\pgfpathlineto{\pgfqpoint{3.668540in}{1.287464in}}%
\pgfpathlineto{\pgfqpoint{3.777268in}{1.308841in}}%
\pgfpathlineto{\pgfqpoint{3.885995in}{1.302969in}}%
\pgfpathlineto{\pgfqpoint{3.994723in}{1.334158in}}%
\pgfpathlineto{\pgfqpoint{4.103450in}{1.413659in}}%
\pgfpathlineto{\pgfqpoint{4.212178in}{1.385622in}}%
\pgfpathlineto{\pgfqpoint{4.320905in}{1.438726in}}%
\pgfpathlineto{\pgfqpoint{4.429632in}{1.468915in}}%
\pgfpathlineto{\pgfqpoint{4.538360in}{1.512802in}}%
\pgfpathlineto{\pgfqpoint{4.647087in}{1.574554in}}%
\pgfpathlineto{\pgfqpoint{4.755815in}{1.528917in}}%
\pgfpathlineto{\pgfqpoint{4.864542in}{1.604547in}}%
\pgfpathlineto{\pgfqpoint{4.973269in}{1.609777in}}%
\pgfpathlineto{\pgfqpoint{5.081997in}{1.692259in}}%
\pgfpathlineto{\pgfqpoint{5.190724in}{1.755416in}}%
\pgfpathlineto{\pgfqpoint{5.299452in}{1.677856in}}%
\pgfpathlineto{\pgfqpoint{5.408179in}{1.823243in}}%
\pgfpathlineto{\pgfqpoint{5.516906in}{1.838209in}}%
\pgfpathlineto{\pgfqpoint{5.625634in}{1.770471in}}%
\pgfpathlineto{\pgfqpoint{5.734361in}{1.831193in}}%
\pgfpathlineto{\pgfqpoint{5.843089in}{1.803988in}}%
\pgfpathlineto{\pgfqpoint{5.951816in}{2.040797in}}%
\pgfpathlineto{\pgfqpoint{6.060544in}{1.942201in}}%
\pgfpathlineto{\pgfqpoint{6.169271in}{2.039087in}}%
\pgfpathlineto{\pgfqpoint{6.277998in}{2.098018in}}%
\pgfpathlineto{\pgfqpoint{6.386726in}{2.142526in}}%
\pgfpathlineto{\pgfqpoint{6.495453in}{2.250646in}}%
\pgfpathlineto{\pgfqpoint{6.604181in}{2.325695in}}%
\pgfpathlineto{\pgfqpoint{6.712908in}{2.311104in}}%
\pgfpathlineto{\pgfqpoint{6.821635in}{2.539052in}}%
\pgfpathlineto{\pgfqpoint{6.930363in}{2.449230in}}%
\pgfpathlineto{\pgfqpoint{7.039090in}{2.524741in}}%
\pgfpathlineto{\pgfqpoint{7.147818in}{2.536194in}}%
\pgfpathlineto{\pgfqpoint{7.256545in}{2.671444in}}%
\pgfpathlineto{\pgfqpoint{7.365272in}{2.546016in}}%
\pgfpathlineto{\pgfqpoint{7.474000in}{2.657267in}}%
\pgfpathlineto{\pgfqpoint{7.582727in}{2.505623in}}%
\pgfpathlineto{\pgfqpoint{7.691455in}{2.838457in}}%
\pgfpathlineto{\pgfqpoint{7.800182in}{2.777607in}}%
\pgfpathlineto{\pgfqpoint{7.908910in}{3.192274in}}%
\pgfpathlineto{\pgfqpoint{8.017637in}{3.257825in}}%
\pgfpathlineto{\pgfqpoint{8.126364in}{3.444480in}}%
\pgfpathlineto{\pgfqpoint{8.235092in}{2.843705in}}%
\pgfpathlineto{\pgfqpoint{8.343819in}{3.058484in}}%
\pgfpathlineto{\pgfqpoint{8.452547in}{3.188618in}}%
\pgfpathlineto{\pgfqpoint{8.561274in}{3.280994in}}%
\pgfpathlineto{\pgfqpoint{8.670001in}{3.353280in}}%
\pgfpathlineto{\pgfqpoint{8.778729in}{3.439552in}}%
\pgfpathlineto{\pgfqpoint{8.887456in}{3.383383in}}%
\pgfpathlineto{\pgfqpoint{8.996184in}{3.426801in}}%
\pgfpathlineto{\pgfqpoint{9.104911in}{3.871925in}}%
\pgfpathlineto{\pgfqpoint{9.213638in}{3.775174in}}%
\pgfpathlineto{\pgfqpoint{9.322366in}{3.392219in}}%
\pgfpathlineto{\pgfqpoint{9.431093in}{3.724000in}}%
\pgfpathlineto{\pgfqpoint{9.539821in}{4.026028in}}%
\pgfpathlineto{\pgfqpoint{9.648548in}{3.916551in}}%
\pgfpathlineto{\pgfqpoint{9.757276in}{4.221390in}}%
\pgfpathlineto{\pgfqpoint{9.866003in}{4.142849in}}%
\pgfpathlineto{\pgfqpoint{9.974730in}{4.122238in}}%
\pgfpathlineto{\pgfqpoint{10.083458in}{4.306673in}}%
\pgfpathlineto{\pgfqpoint{10.192185in}{4.175399in}}%
\pgfpathlineto{\pgfqpoint{10.300913in}{4.288667in}}%
\pgfpathlineto{\pgfqpoint{10.409640in}{4.543637in}}%
\pgfpathlineto{\pgfqpoint{10.518367in}{4.601915in}}%
\pgfpathlineto{\pgfqpoint{10.627095in}{4.546287in}}%
\pgfpathlineto{\pgfqpoint{10.735822in}{4.874528in}}%
\pgfpathlineto{\pgfqpoint{10.844550in}{4.528256in}}%
\pgfpathlineto{\pgfqpoint{10.953277in}{5.282096in}}%
\pgfpathlineto{\pgfqpoint{11.062004in}{5.144018in}}%
\pgfpathlineto{\pgfqpoint{11.170732in}{5.305564in}}%
\pgfpathlineto{\pgfqpoint{11.279459in}{5.107425in}}%
\pgfpathlineto{\pgfqpoint{11.388187in}{5.615144in}}%
\pgfpathlineto{\pgfqpoint{11.496914in}{5.235350in}}%
\pgfpathlineto{\pgfqpoint{11.605642in}{5.687593in}}%
\pgfpathlineto{\pgfqpoint{11.714369in}{5.770830in}}%
\pgfpathlineto{\pgfqpoint{11.823096in}{5.793498in}}%
\pgfpathlineto{\pgfqpoint{11.931824in}{5.434733in}}%
\pgfpathlineto{\pgfqpoint{12.040551in}{6.382235in}}%
\pgfpathlineto{\pgfqpoint{12.149279in}{5.839237in}}%
\pgfpathlineto{\pgfqpoint{12.258006in}{5.802353in}}%
\pgfpathlineto{\pgfqpoint{12.366733in}{6.535610in}}%
\pgfpathlineto{\pgfqpoint{12.475461in}{5.771503in}}%
\pgfpathlineto{\pgfqpoint{12.584188in}{5.912581in}}%
\pgfpathlineto{\pgfqpoint{12.692916in}{6.168973in}}%
\pgfpathlineto{\pgfqpoint{12.801643in}{6.119266in}}%
\pgfusepath{stroke}%
\end{pgfscope}%
\begin{pgfscope}%
\pgfpathrectangle{\pgfqpoint{1.499429in}{0.839159in}}{\pgfqpoint{11.840415in}{5.967710in}}%
\pgfusepath{clip}%
\pgfsetrectcap%
\pgfsetroundjoin%
\pgfsetlinewidth{1.505625pt}%
\definecolor{currentstroke}{rgb}{1.000000,0.498039,0.054902}%
\pgfsetstrokecolor{currentstroke}%
\pgfsetdash{}{0pt}%
\pgfpathmoveto{\pgfqpoint{2.037629in}{1.110419in}}%
\pgfpathlineto{\pgfqpoint{2.146357in}{1.115889in}}%
\pgfpathlineto{\pgfqpoint{2.255084in}{1.120673in}}%
\pgfpathlineto{\pgfqpoint{2.363812in}{1.123201in}}%
\pgfpathlineto{\pgfqpoint{2.472539in}{1.132486in}}%
\pgfpathlineto{\pgfqpoint{2.581266in}{1.134362in}}%
\pgfpathlineto{\pgfqpoint{2.689994in}{1.140054in}}%
\pgfpathlineto{\pgfqpoint{2.798721in}{1.150043in}}%
\pgfpathlineto{\pgfqpoint{2.907449in}{1.157407in}}%
\pgfpathlineto{\pgfqpoint{3.016176in}{1.163022in}}%
\pgfpathlineto{\pgfqpoint{3.124903in}{1.169294in}}%
\pgfpathlineto{\pgfqpoint{3.233631in}{1.182811in}}%
\pgfpathlineto{\pgfqpoint{3.342358in}{1.188934in}}%
\pgfpathlineto{\pgfqpoint{3.451086in}{1.202318in}}%
\pgfpathlineto{\pgfqpoint{3.559813in}{1.215993in}}%
\pgfpathlineto{\pgfqpoint{3.668540in}{1.215244in}}%
\pgfpathlineto{\pgfqpoint{3.777268in}{1.225092in}}%
\pgfpathlineto{\pgfqpoint{3.885995in}{1.232544in}}%
\pgfpathlineto{\pgfqpoint{3.994723in}{1.252724in}}%
\pgfpathlineto{\pgfqpoint{4.103450in}{1.269448in}}%
\pgfpathlineto{\pgfqpoint{4.212178in}{1.268946in}}%
\pgfpathlineto{\pgfqpoint{4.320905in}{1.283565in}}%
\pgfpathlineto{\pgfqpoint{4.429632in}{1.305494in}}%
\pgfpathlineto{\pgfqpoint{4.538360in}{1.304530in}}%
\pgfpathlineto{\pgfqpoint{4.647087in}{1.311953in}}%
\pgfpathlineto{\pgfqpoint{4.755815in}{1.323517in}}%
\pgfpathlineto{\pgfqpoint{4.864542in}{1.330057in}}%
\pgfpathlineto{\pgfqpoint{4.973269in}{1.364330in}}%
\pgfpathlineto{\pgfqpoint{5.081997in}{1.356538in}}%
\pgfpathlineto{\pgfqpoint{5.190724in}{1.381539in}}%
\pgfpathlineto{\pgfqpoint{5.299452in}{1.405584in}}%
\pgfpathlineto{\pgfqpoint{5.408179in}{1.418366in}}%
\pgfpathlineto{\pgfqpoint{5.516906in}{1.395030in}}%
\pgfpathlineto{\pgfqpoint{5.625634in}{1.427143in}}%
\pgfpathlineto{\pgfqpoint{5.734361in}{1.442134in}}%
\pgfpathlineto{\pgfqpoint{5.843089in}{1.459037in}}%
\pgfpathlineto{\pgfqpoint{5.951816in}{1.479671in}}%
\pgfpathlineto{\pgfqpoint{6.060544in}{1.492611in}}%
\pgfpathlineto{\pgfqpoint{6.169271in}{1.515005in}}%
\pgfpathlineto{\pgfqpoint{6.277998in}{1.516325in}}%
\pgfpathlineto{\pgfqpoint{6.386726in}{1.539357in}}%
\pgfpathlineto{\pgfqpoint{6.495453in}{1.558233in}}%
\pgfpathlineto{\pgfqpoint{6.604181in}{1.563996in}}%
\pgfpathlineto{\pgfqpoint{6.712908in}{1.586579in}}%
\pgfpathlineto{\pgfqpoint{6.821635in}{1.610264in}}%
\pgfpathlineto{\pgfqpoint{6.930363in}{1.636827in}}%
\pgfpathlineto{\pgfqpoint{7.039090in}{1.647080in}}%
\pgfpathlineto{\pgfqpoint{7.147818in}{1.660785in}}%
\pgfpathlineto{\pgfqpoint{7.256545in}{1.681602in}}%
\pgfpathlineto{\pgfqpoint{7.365272in}{1.696427in}}%
\pgfpathlineto{\pgfqpoint{7.474000in}{1.676069in}}%
\pgfpathlineto{\pgfqpoint{7.582727in}{1.732337in}}%
\pgfpathlineto{\pgfqpoint{7.691455in}{1.746033in}}%
\pgfpathlineto{\pgfqpoint{7.800182in}{1.735077in}}%
\pgfpathlineto{\pgfqpoint{7.908910in}{1.784200in}}%
\pgfpathlineto{\pgfqpoint{8.017637in}{1.784483in}}%
\pgfpathlineto{\pgfqpoint{8.126364in}{1.821528in}}%
\pgfpathlineto{\pgfqpoint{8.235092in}{1.808226in}}%
\pgfpathlineto{\pgfqpoint{8.343819in}{1.831355in}}%
\pgfpathlineto{\pgfqpoint{8.452547in}{1.872413in}}%
\pgfpathlineto{\pgfqpoint{8.561274in}{1.895608in}}%
\pgfpathlineto{\pgfqpoint{8.670001in}{1.878396in}}%
\pgfpathlineto{\pgfqpoint{8.778729in}{1.908576in}}%
\pgfpathlineto{\pgfqpoint{8.887456in}{1.953215in}}%
\pgfpathlineto{\pgfqpoint{8.996184in}{1.946054in}}%
\pgfpathlineto{\pgfqpoint{9.104911in}{1.983979in}}%
\pgfpathlineto{\pgfqpoint{9.213638in}{1.965524in}}%
\pgfpathlineto{\pgfqpoint{9.322366in}{1.992375in}}%
\pgfpathlineto{\pgfqpoint{9.431093in}{2.032923in}}%
\pgfpathlineto{\pgfqpoint{9.539821in}{2.027052in}}%
\pgfpathlineto{\pgfqpoint{9.648548in}{2.054563in}}%
\pgfpathlineto{\pgfqpoint{9.757276in}{2.106382in}}%
\pgfpathlineto{\pgfqpoint{9.866003in}{2.122814in}}%
\pgfpathlineto{\pgfqpoint{9.974730in}{2.090025in}}%
\pgfpathlineto{\pgfqpoint{10.083458in}{2.165872in}}%
\pgfpathlineto{\pgfqpoint{10.192185in}{2.205091in}}%
\pgfpathlineto{\pgfqpoint{10.300913in}{2.156049in}}%
\pgfpathlineto{\pgfqpoint{10.409640in}{2.253394in}}%
\pgfpathlineto{\pgfqpoint{10.518367in}{2.230473in}}%
\pgfpathlineto{\pgfqpoint{10.627095in}{2.279015in}}%
\pgfpathlineto{\pgfqpoint{10.735822in}{2.237747in}}%
\pgfpathlineto{\pgfqpoint{10.844550in}{2.279069in}}%
\pgfpathlineto{\pgfqpoint{10.953277in}{2.315734in}}%
\pgfpathlineto{\pgfqpoint{11.062004in}{2.334110in}}%
\pgfpathlineto{\pgfqpoint{11.170732in}{2.353333in}}%
\pgfpathlineto{\pgfqpoint{11.279459in}{2.398736in}}%
\pgfpathlineto{\pgfqpoint{11.388187in}{2.433268in}}%
\pgfpathlineto{\pgfqpoint{11.496914in}{2.419663in}}%
\pgfpathlineto{\pgfqpoint{11.605642in}{2.463400in}}%
\pgfpathlineto{\pgfqpoint{11.714369in}{2.460139in}}%
\pgfpathlineto{\pgfqpoint{11.823096in}{2.486513in}}%
\pgfpathlineto{\pgfqpoint{11.931824in}{2.536326in}}%
\pgfpathlineto{\pgfqpoint{12.040551in}{2.513389in}}%
\pgfpathlineto{\pgfqpoint{12.149279in}{2.558663in}}%
\pgfpathlineto{\pgfqpoint{12.258006in}{2.601231in}}%
\pgfpathlineto{\pgfqpoint{12.366733in}{2.666320in}}%
\pgfpathlineto{\pgfqpoint{12.475461in}{2.617461in}}%
\pgfpathlineto{\pgfqpoint{12.584188in}{2.597438in}}%
\pgfpathlineto{\pgfqpoint{12.692916in}{2.627561in}}%
\pgfpathlineto{\pgfqpoint{12.801643in}{2.695512in}}%
\pgfusepath{stroke}%
\end{pgfscope}%
\begin{pgfscope}%
\pgfpathrectangle{\pgfqpoint{1.499429in}{0.839159in}}{\pgfqpoint{11.840415in}{5.967710in}}%
\pgfusepath{clip}%
\pgfsetrectcap%
\pgfsetroundjoin%
\pgfsetlinewidth{1.505625pt}%
\definecolor{currentstroke}{rgb}{0.172549,0.627451,0.172549}%
\pgfsetstrokecolor{currentstroke}%
\pgfsetdash{}{0pt}%
\pgfpathmoveto{\pgfqpoint{2.037629in}{1.110419in}}%
\pgfpathlineto{\pgfqpoint{2.146357in}{1.115889in}}%
\pgfpathlineto{\pgfqpoint{2.255084in}{1.120673in}}%
\pgfpathlineto{\pgfqpoint{2.363812in}{1.123201in}}%
\pgfpathlineto{\pgfqpoint{2.472539in}{1.132486in}}%
\pgfpathlineto{\pgfqpoint{2.581266in}{1.134362in}}%
\pgfpathlineto{\pgfqpoint{2.689994in}{1.140054in}}%
\pgfpathlineto{\pgfqpoint{2.798721in}{1.150043in}}%
\pgfpathlineto{\pgfqpoint{2.907449in}{1.157407in}}%
\pgfpathlineto{\pgfqpoint{3.016176in}{1.163022in}}%
\pgfpathlineto{\pgfqpoint{3.124903in}{1.169294in}}%
\pgfpathlineto{\pgfqpoint{3.233631in}{1.182811in}}%
\pgfpathlineto{\pgfqpoint{3.342358in}{1.188934in}}%
\pgfpathlineto{\pgfqpoint{3.451086in}{1.202318in}}%
\pgfpathlineto{\pgfqpoint{3.559813in}{1.215347in}}%
\pgfpathlineto{\pgfqpoint{3.668540in}{1.215244in}}%
\pgfpathlineto{\pgfqpoint{3.777268in}{1.224094in}}%
\pgfpathlineto{\pgfqpoint{3.885995in}{1.231964in}}%
\pgfpathlineto{\pgfqpoint{3.994723in}{1.252724in}}%
\pgfpathlineto{\pgfqpoint{4.103450in}{1.269448in}}%
\pgfpathlineto{\pgfqpoint{4.212178in}{1.268946in}}%
\pgfpathlineto{\pgfqpoint{4.320905in}{1.279989in}}%
\pgfpathlineto{\pgfqpoint{4.429632in}{1.305494in}}%
\pgfpathlineto{\pgfqpoint{4.538360in}{1.304185in}}%
\pgfpathlineto{\pgfqpoint{4.647087in}{1.310638in}}%
\pgfpathlineto{\pgfqpoint{4.755815in}{1.323517in}}%
\pgfpathlineto{\pgfqpoint{4.864542in}{1.330057in}}%
\pgfpathlineto{\pgfqpoint{4.973269in}{1.364330in}}%
\pgfpathlineto{\pgfqpoint{5.081997in}{1.356538in}}%
\pgfpathlineto{\pgfqpoint{5.190724in}{1.381539in}}%
\pgfpathlineto{\pgfqpoint{5.299452in}{1.407176in}}%
\pgfpathlineto{\pgfqpoint{5.408179in}{1.418366in}}%
\pgfpathlineto{\pgfqpoint{5.516906in}{1.399699in}}%
\pgfpathlineto{\pgfqpoint{5.625634in}{1.427143in}}%
\pgfpathlineto{\pgfqpoint{5.734361in}{1.438986in}}%
\pgfpathlineto{\pgfqpoint{5.843089in}{1.459037in}}%
\pgfpathlineto{\pgfqpoint{5.951816in}{1.470527in}}%
\pgfpathlineto{\pgfqpoint{6.060544in}{1.491828in}}%
\pgfpathlineto{\pgfqpoint{6.169271in}{1.515005in}}%
\pgfpathlineto{\pgfqpoint{6.277998in}{1.515974in}}%
\pgfpathlineto{\pgfqpoint{6.386726in}{1.539357in}}%
\pgfpathlineto{\pgfqpoint{6.495453in}{1.556823in}}%
\pgfpathlineto{\pgfqpoint{6.604181in}{1.563677in}}%
\pgfpathlineto{\pgfqpoint{6.712908in}{1.586511in}}%
\pgfpathlineto{\pgfqpoint{6.821635in}{1.609083in}}%
\pgfpathlineto{\pgfqpoint{6.930363in}{1.628351in}}%
\pgfpathlineto{\pgfqpoint{7.039090in}{1.647080in}}%
\pgfpathlineto{\pgfqpoint{7.147818in}{1.660785in}}%
\pgfpathlineto{\pgfqpoint{7.256545in}{1.681602in}}%
\pgfpathlineto{\pgfqpoint{7.365272in}{1.683373in}}%
\pgfpathlineto{\pgfqpoint{7.474000in}{1.685498in}}%
\pgfpathlineto{\pgfqpoint{7.582727in}{1.738039in}}%
\pgfpathlineto{\pgfqpoint{7.691455in}{1.746033in}}%
\pgfpathlineto{\pgfqpoint{7.800182in}{1.737433in}}%
\pgfpathlineto{\pgfqpoint{7.908910in}{1.776218in}}%
\pgfpathlineto{\pgfqpoint{8.017637in}{1.781912in}}%
\pgfpathlineto{\pgfqpoint{8.126364in}{1.828251in}}%
\pgfpathlineto{\pgfqpoint{8.235092in}{1.796029in}}%
\pgfpathlineto{\pgfqpoint{8.343819in}{1.836473in}}%
\pgfpathlineto{\pgfqpoint{8.452547in}{1.869478in}}%
\pgfpathlineto{\pgfqpoint{8.561274in}{1.895608in}}%
\pgfpathlineto{\pgfqpoint{8.670001in}{1.881887in}}%
\pgfpathlineto{\pgfqpoint{8.778729in}{1.906463in}}%
\pgfpathlineto{\pgfqpoint{8.887456in}{1.944551in}}%
\pgfpathlineto{\pgfqpoint{8.996184in}{1.949979in}}%
\pgfpathlineto{\pgfqpoint{9.104911in}{1.963237in}}%
\pgfpathlineto{\pgfqpoint{9.213638in}{1.962399in}}%
\pgfpathlineto{\pgfqpoint{9.322366in}{1.970141in}}%
\pgfpathlineto{\pgfqpoint{9.431093in}{2.032923in}}%
\pgfpathlineto{\pgfqpoint{9.539821in}{2.033288in}}%
\pgfpathlineto{\pgfqpoint{9.648548in}{2.053473in}}%
\pgfpathlineto{\pgfqpoint{9.757276in}{2.106395in}}%
\pgfpathlineto{\pgfqpoint{9.866003in}{2.128646in}}%
\pgfpathlineto{\pgfqpoint{9.974730in}{2.085071in}}%
\pgfpathlineto{\pgfqpoint{10.083458in}{2.165466in}}%
\pgfpathlineto{\pgfqpoint{10.192185in}{2.222389in}}%
\pgfpathlineto{\pgfqpoint{10.300913in}{2.153166in}}%
\pgfpathlineto{\pgfqpoint{10.409640in}{2.257554in}}%
\pgfpathlineto{\pgfqpoint{10.518367in}{2.230473in}}%
\pgfpathlineto{\pgfqpoint{10.627095in}{2.297160in}}%
\pgfpathlineto{\pgfqpoint{10.735822in}{2.239038in}}%
\pgfpathlineto{\pgfqpoint{10.844550in}{2.272613in}}%
\pgfpathlineto{\pgfqpoint{10.953277in}{2.315734in}}%
\pgfpathlineto{\pgfqpoint{11.062004in}{2.322542in}}%
\pgfpathlineto{\pgfqpoint{11.170732in}{2.340929in}}%
\pgfpathlineto{\pgfqpoint{11.279459in}{2.416897in}}%
\pgfpathlineto{\pgfqpoint{11.388187in}{2.431093in}}%
\pgfpathlineto{\pgfqpoint{11.496914in}{2.419663in}}%
\pgfpathlineto{\pgfqpoint{11.605642in}{2.492533in}}%
\pgfpathlineto{\pgfqpoint{11.714369in}{2.472584in}}%
\pgfpathlineto{\pgfqpoint{11.823096in}{2.500157in}}%
\pgfpathlineto{\pgfqpoint{11.931824in}{2.536366in}}%
\pgfpathlineto{\pgfqpoint{12.040551in}{2.540966in}}%
\pgfpathlineto{\pgfqpoint{12.149279in}{2.560282in}}%
\pgfpathlineto{\pgfqpoint{12.258006in}{2.595266in}}%
\pgfpathlineto{\pgfqpoint{12.366733in}{2.669568in}}%
\pgfpathlineto{\pgfqpoint{12.475461in}{2.623664in}}%
\pgfpathlineto{\pgfqpoint{12.584188in}{2.629433in}}%
\pgfpathlineto{\pgfqpoint{12.692916in}{2.615912in}}%
\pgfpathlineto{\pgfqpoint{12.801643in}{2.714156in}}%
\pgfusepath{stroke}%
\end{pgfscope}%
\begin{pgfscope}%
\pgfpathrectangle{\pgfqpoint{1.499429in}{0.839159in}}{\pgfqpoint{11.840415in}{5.967710in}}%
\pgfusepath{clip}%
\pgfsetrectcap%
\pgfsetroundjoin%
\pgfsetlinewidth{1.505625pt}%
\definecolor{currentstroke}{rgb}{0.839216,0.152941,0.156863}%
\pgfsetstrokecolor{currentstroke}%
\pgfsetdash{}{0pt}%
\pgfpathmoveto{\pgfqpoint{2.037629in}{1.110419in}}%
\pgfpathlineto{\pgfqpoint{2.146357in}{1.115889in}}%
\pgfpathlineto{\pgfqpoint{2.255084in}{1.120673in}}%
\pgfpathlineto{\pgfqpoint{2.363812in}{1.123201in}}%
\pgfpathlineto{\pgfqpoint{2.472539in}{1.132486in}}%
\pgfpathlineto{\pgfqpoint{2.581266in}{1.134362in}}%
\pgfpathlineto{\pgfqpoint{2.689994in}{1.140054in}}%
\pgfpathlineto{\pgfqpoint{2.798721in}{1.150043in}}%
\pgfpathlineto{\pgfqpoint{2.907449in}{1.157407in}}%
\pgfpathlineto{\pgfqpoint{3.016176in}{1.163022in}}%
\pgfpathlineto{\pgfqpoint{3.124903in}{1.169294in}}%
\pgfpathlineto{\pgfqpoint{3.233631in}{1.182811in}}%
\pgfpathlineto{\pgfqpoint{3.342358in}{1.188934in}}%
\pgfpathlineto{\pgfqpoint{3.451086in}{1.202318in}}%
\pgfpathlineto{\pgfqpoint{3.559813in}{1.215347in}}%
\pgfpathlineto{\pgfqpoint{3.668540in}{1.215244in}}%
\pgfpathlineto{\pgfqpoint{3.777268in}{1.224094in}}%
\pgfpathlineto{\pgfqpoint{3.885995in}{1.231964in}}%
\pgfpathlineto{\pgfqpoint{3.994723in}{1.252724in}}%
\pgfpathlineto{\pgfqpoint{4.103450in}{1.269448in}}%
\pgfpathlineto{\pgfqpoint{4.212178in}{1.268946in}}%
\pgfpathlineto{\pgfqpoint{4.320905in}{1.279989in}}%
\pgfpathlineto{\pgfqpoint{4.429632in}{1.305494in}}%
\pgfpathlineto{\pgfqpoint{4.538360in}{1.304185in}}%
\pgfpathlineto{\pgfqpoint{4.647087in}{1.310638in}}%
\pgfpathlineto{\pgfqpoint{4.755815in}{1.323517in}}%
\pgfpathlineto{\pgfqpoint{4.864542in}{1.330057in}}%
\pgfpathlineto{\pgfqpoint{4.973269in}{1.364330in}}%
\pgfpathlineto{\pgfqpoint{5.081997in}{1.356538in}}%
\pgfpathlineto{\pgfqpoint{5.190724in}{1.381539in}}%
\pgfpathlineto{\pgfqpoint{5.299452in}{1.407176in}}%
\pgfpathlineto{\pgfqpoint{5.408179in}{1.418366in}}%
\pgfpathlineto{\pgfqpoint{5.516906in}{1.399699in}}%
\pgfpathlineto{\pgfqpoint{5.625634in}{1.427143in}}%
\pgfpathlineto{\pgfqpoint{5.734361in}{1.438986in}}%
\pgfpathlineto{\pgfqpoint{5.843089in}{1.459037in}}%
\pgfpathlineto{\pgfqpoint{5.951816in}{1.470527in}}%
\pgfpathlineto{\pgfqpoint{6.060544in}{1.491828in}}%
\pgfpathlineto{\pgfqpoint{6.169271in}{1.515005in}}%
\pgfpathlineto{\pgfqpoint{6.277998in}{1.515974in}}%
\pgfpathlineto{\pgfqpoint{6.386726in}{1.539357in}}%
\pgfpathlineto{\pgfqpoint{6.495453in}{1.556823in}}%
\pgfpathlineto{\pgfqpoint{6.604181in}{1.563677in}}%
\pgfpathlineto{\pgfqpoint{6.712908in}{1.586511in}}%
\pgfpathlineto{\pgfqpoint{6.821635in}{1.609083in}}%
\pgfpathlineto{\pgfqpoint{6.930363in}{1.628351in}}%
\pgfpathlineto{\pgfqpoint{7.039090in}{1.647080in}}%
\pgfpathlineto{\pgfqpoint{7.147818in}{1.660785in}}%
\pgfpathlineto{\pgfqpoint{7.256545in}{1.681602in}}%
\pgfpathlineto{\pgfqpoint{7.365272in}{1.683373in}}%
\pgfpathlineto{\pgfqpoint{7.474000in}{1.685498in}}%
\pgfpathlineto{\pgfqpoint{7.582727in}{1.738039in}}%
\pgfpathlineto{\pgfqpoint{7.691455in}{1.746033in}}%
\pgfpathlineto{\pgfqpoint{7.800182in}{1.737433in}}%
\pgfpathlineto{\pgfqpoint{7.908910in}{1.776218in}}%
\pgfpathlineto{\pgfqpoint{8.017637in}{1.781912in}}%
\pgfpathlineto{\pgfqpoint{8.126364in}{1.828251in}}%
\pgfpathlineto{\pgfqpoint{8.235092in}{1.796029in}}%
\pgfpathlineto{\pgfqpoint{8.343819in}{1.836473in}}%
\pgfpathlineto{\pgfqpoint{8.452547in}{1.869478in}}%
\pgfpathlineto{\pgfqpoint{8.561274in}{1.895608in}}%
\pgfpathlineto{\pgfqpoint{8.670001in}{1.881887in}}%
\pgfpathlineto{\pgfqpoint{8.778729in}{1.906060in}}%
\pgfpathlineto{\pgfqpoint{8.887456in}{1.944551in}}%
\pgfpathlineto{\pgfqpoint{8.996184in}{1.949979in}}%
\pgfpathlineto{\pgfqpoint{9.104911in}{1.963237in}}%
\pgfpathlineto{\pgfqpoint{9.213638in}{1.962399in}}%
\pgfpathlineto{\pgfqpoint{9.322366in}{1.970141in}}%
\pgfpathlineto{\pgfqpoint{9.431093in}{2.032923in}}%
\pgfpathlineto{\pgfqpoint{9.539821in}{2.033288in}}%
\pgfpathlineto{\pgfqpoint{9.648548in}{2.053473in}}%
\pgfpathlineto{\pgfqpoint{9.757276in}{2.106395in}}%
\pgfpathlineto{\pgfqpoint{9.866003in}{2.128646in}}%
\pgfpathlineto{\pgfqpoint{9.974730in}{2.085071in}}%
\pgfpathlineto{\pgfqpoint{10.083458in}{2.165466in}}%
\pgfpathlineto{\pgfqpoint{10.192185in}{2.222389in}}%
\pgfpathlineto{\pgfqpoint{10.300913in}{2.153166in}}%
\pgfpathlineto{\pgfqpoint{10.409640in}{2.257554in}}%
\pgfpathlineto{\pgfqpoint{10.518367in}{2.230473in}}%
\pgfpathlineto{\pgfqpoint{10.627095in}{2.297160in}}%
\pgfpathlineto{\pgfqpoint{10.735822in}{2.239038in}}%
\pgfpathlineto{\pgfqpoint{10.844550in}{2.272613in}}%
\pgfpathlineto{\pgfqpoint{10.953277in}{2.315734in}}%
\pgfpathlineto{\pgfqpoint{11.062004in}{2.322542in}}%
\pgfpathlineto{\pgfqpoint{11.170732in}{2.340929in}}%
\pgfpathlineto{\pgfqpoint{11.279459in}{2.416897in}}%
\pgfpathlineto{\pgfqpoint{11.388187in}{2.431093in}}%
\pgfpathlineto{\pgfqpoint{11.496914in}{2.419663in}}%
\pgfpathlineto{\pgfqpoint{11.605642in}{2.492533in}}%
\pgfpathlineto{\pgfqpoint{11.714369in}{2.472584in}}%
\pgfpathlineto{\pgfqpoint{11.823096in}{2.500157in}}%
\pgfpathlineto{\pgfqpoint{11.931824in}{2.536366in}}%
\pgfpathlineto{\pgfqpoint{12.040551in}{2.540966in}}%
\pgfpathlineto{\pgfqpoint{12.149279in}{2.560282in}}%
\pgfpathlineto{\pgfqpoint{12.258006in}{2.586793in}}%
\pgfpathlineto{\pgfqpoint{12.366733in}{2.669568in}}%
\pgfpathlineto{\pgfqpoint{12.475461in}{2.623664in}}%
\pgfpathlineto{\pgfqpoint{12.584188in}{2.629433in}}%
\pgfpathlineto{\pgfqpoint{12.692916in}{2.615912in}}%
\pgfpathlineto{\pgfqpoint{12.801643in}{2.714156in}}%
\pgfusepath{stroke}%
\end{pgfscope}%
\begin{pgfscope}%
\pgfpathrectangle{\pgfqpoint{1.499429in}{0.839159in}}{\pgfqpoint{11.840415in}{5.967710in}}%
\pgfusepath{clip}%
\pgfsetrectcap%
\pgfsetroundjoin%
\pgfsetlinewidth{1.505625pt}%
\definecolor{currentstroke}{rgb}{0.580392,0.403922,0.741176}%
\pgfsetstrokecolor{currentstroke}%
\pgfsetdash{}{0pt}%
\pgfpathmoveto{\pgfqpoint{2.037629in}{1.110419in}}%
\pgfpathlineto{\pgfqpoint{2.146357in}{1.115889in}}%
\pgfpathlineto{\pgfqpoint{2.255084in}{1.120673in}}%
\pgfpathlineto{\pgfqpoint{2.363812in}{1.123201in}}%
\pgfpathlineto{\pgfqpoint{2.472539in}{1.132486in}}%
\pgfpathlineto{\pgfqpoint{2.581266in}{1.134362in}}%
\pgfpathlineto{\pgfqpoint{2.689994in}{1.140054in}}%
\pgfpathlineto{\pgfqpoint{2.798721in}{1.150043in}}%
\pgfpathlineto{\pgfqpoint{2.907449in}{1.157407in}}%
\pgfpathlineto{\pgfqpoint{3.016176in}{1.163022in}}%
\pgfpathlineto{\pgfqpoint{3.124903in}{1.169294in}}%
\pgfpathlineto{\pgfqpoint{3.233631in}{1.182811in}}%
\pgfpathlineto{\pgfqpoint{3.342358in}{1.188934in}}%
\pgfpathlineto{\pgfqpoint{3.451086in}{1.202318in}}%
\pgfpathlineto{\pgfqpoint{3.559813in}{1.215347in}}%
\pgfpathlineto{\pgfqpoint{3.668540in}{1.215244in}}%
\pgfpathlineto{\pgfqpoint{3.777268in}{1.224094in}}%
\pgfpathlineto{\pgfqpoint{3.885995in}{1.231964in}}%
\pgfpathlineto{\pgfqpoint{3.994723in}{1.252724in}}%
\pgfpathlineto{\pgfqpoint{4.103450in}{1.269448in}}%
\pgfpathlineto{\pgfqpoint{4.212178in}{1.268946in}}%
\pgfpathlineto{\pgfqpoint{4.320905in}{1.279989in}}%
\pgfpathlineto{\pgfqpoint{4.429632in}{1.305494in}}%
\pgfpathlineto{\pgfqpoint{4.538360in}{1.304185in}}%
\pgfpathlineto{\pgfqpoint{4.647087in}{1.310638in}}%
\pgfpathlineto{\pgfqpoint{4.755815in}{1.323517in}}%
\pgfpathlineto{\pgfqpoint{4.864542in}{1.330057in}}%
\pgfpathlineto{\pgfqpoint{4.973269in}{1.364330in}}%
\pgfpathlineto{\pgfqpoint{5.081997in}{1.356538in}}%
\pgfpathlineto{\pgfqpoint{5.190724in}{1.381539in}}%
\pgfpathlineto{\pgfqpoint{5.299452in}{1.407176in}}%
\pgfpathlineto{\pgfqpoint{5.408179in}{1.418366in}}%
\pgfpathlineto{\pgfqpoint{5.516906in}{1.399699in}}%
\pgfpathlineto{\pgfqpoint{5.625634in}{1.427143in}}%
\pgfpathlineto{\pgfqpoint{5.734361in}{1.438986in}}%
\pgfpathlineto{\pgfqpoint{5.843089in}{1.459037in}}%
\pgfpathlineto{\pgfqpoint{5.951816in}{1.470527in}}%
\pgfpathlineto{\pgfqpoint{6.060544in}{1.491828in}}%
\pgfpathlineto{\pgfqpoint{6.169271in}{1.515005in}}%
\pgfpathlineto{\pgfqpoint{6.277998in}{1.515974in}}%
\pgfpathlineto{\pgfqpoint{6.386726in}{1.539357in}}%
\pgfpathlineto{\pgfqpoint{6.495453in}{1.556823in}}%
\pgfpathlineto{\pgfqpoint{6.604181in}{1.563677in}}%
\pgfpathlineto{\pgfqpoint{6.712908in}{1.586511in}}%
\pgfpathlineto{\pgfqpoint{6.821635in}{1.609083in}}%
\pgfpathlineto{\pgfqpoint{6.930363in}{1.628351in}}%
\pgfpathlineto{\pgfqpoint{7.039090in}{1.647080in}}%
\pgfpathlineto{\pgfqpoint{7.147818in}{1.660785in}}%
\pgfpathlineto{\pgfqpoint{7.256545in}{1.681602in}}%
\pgfpathlineto{\pgfqpoint{7.365272in}{1.683373in}}%
\pgfpathlineto{\pgfqpoint{7.474000in}{1.685498in}}%
\pgfpathlineto{\pgfqpoint{7.582727in}{1.738039in}}%
\pgfpathlineto{\pgfqpoint{7.691455in}{1.746033in}}%
\pgfpathlineto{\pgfqpoint{7.800182in}{1.737433in}}%
\pgfpathlineto{\pgfqpoint{7.908910in}{1.776218in}}%
\pgfpathlineto{\pgfqpoint{8.017637in}{1.781912in}}%
\pgfpathlineto{\pgfqpoint{8.126364in}{1.828251in}}%
\pgfpathlineto{\pgfqpoint{8.235092in}{1.796029in}}%
\pgfpathlineto{\pgfqpoint{8.343819in}{1.836473in}}%
\pgfpathlineto{\pgfqpoint{8.452547in}{1.869478in}}%
\pgfpathlineto{\pgfqpoint{8.561274in}{1.895608in}}%
\pgfpathlineto{\pgfqpoint{8.670001in}{1.881887in}}%
\pgfpathlineto{\pgfqpoint{8.778729in}{1.906060in}}%
\pgfpathlineto{\pgfqpoint{8.887456in}{1.944551in}}%
\pgfpathlineto{\pgfqpoint{8.996184in}{1.949979in}}%
\pgfpathlineto{\pgfqpoint{9.104911in}{1.963237in}}%
\pgfpathlineto{\pgfqpoint{9.213638in}{1.962399in}}%
\pgfpathlineto{\pgfqpoint{9.322366in}{1.970141in}}%
\pgfpathlineto{\pgfqpoint{9.431093in}{2.032923in}}%
\pgfpathlineto{\pgfqpoint{9.539821in}{2.033288in}}%
\pgfpathlineto{\pgfqpoint{9.648548in}{2.053473in}}%
\pgfpathlineto{\pgfqpoint{9.757276in}{2.106395in}}%
\pgfpathlineto{\pgfqpoint{9.866003in}{2.128646in}}%
\pgfpathlineto{\pgfqpoint{9.974730in}{2.085071in}}%
\pgfpathlineto{\pgfqpoint{10.083458in}{2.165466in}}%
\pgfpathlineto{\pgfqpoint{10.192185in}{2.222389in}}%
\pgfpathlineto{\pgfqpoint{10.300913in}{2.153166in}}%
\pgfpathlineto{\pgfqpoint{10.409640in}{2.257554in}}%
\pgfpathlineto{\pgfqpoint{10.518367in}{2.230473in}}%
\pgfpathlineto{\pgfqpoint{10.627095in}{2.297160in}}%
\pgfpathlineto{\pgfqpoint{10.735822in}{2.239038in}}%
\pgfpathlineto{\pgfqpoint{10.844550in}{2.272613in}}%
\pgfpathlineto{\pgfqpoint{10.953277in}{2.315734in}}%
\pgfpathlineto{\pgfqpoint{11.062004in}{2.322542in}}%
\pgfpathlineto{\pgfqpoint{11.170732in}{2.340929in}}%
\pgfpathlineto{\pgfqpoint{11.279459in}{2.416897in}}%
\pgfpathlineto{\pgfqpoint{11.388187in}{2.431093in}}%
\pgfpathlineto{\pgfqpoint{11.496914in}{2.419663in}}%
\pgfpathlineto{\pgfqpoint{11.605642in}{2.492533in}}%
\pgfpathlineto{\pgfqpoint{11.714369in}{2.472584in}}%
\pgfpathlineto{\pgfqpoint{11.823096in}{2.500157in}}%
\pgfpathlineto{\pgfqpoint{11.931824in}{2.536366in}}%
\pgfpathlineto{\pgfqpoint{12.040551in}{2.540966in}}%
\pgfpathlineto{\pgfqpoint{12.149279in}{2.560282in}}%
\pgfpathlineto{\pgfqpoint{12.258006in}{2.586793in}}%
\pgfpathlineto{\pgfqpoint{12.366733in}{2.669568in}}%
\pgfpathlineto{\pgfqpoint{12.475461in}{2.623664in}}%
\pgfpathlineto{\pgfqpoint{12.584188in}{2.629433in}}%
\pgfpathlineto{\pgfqpoint{12.692916in}{2.615912in}}%
\pgfpathlineto{\pgfqpoint{12.801643in}{2.714156in}}%
\pgfusepath{stroke}%
\end{pgfscope}%
\begin{pgfscope}%
\pgfpathrectangle{\pgfqpoint{1.499429in}{0.839159in}}{\pgfqpoint{11.840415in}{5.967710in}}%
\pgfusepath{clip}%
\pgfsetrectcap%
\pgfsetroundjoin%
\pgfsetlinewidth{1.505625pt}%
\definecolor{currentstroke}{rgb}{0.549020,0.337255,0.294118}%
\pgfsetstrokecolor{currentstroke}%
\pgfsetdash{}{0pt}%
\pgfpathmoveto{\pgfqpoint{2.037629in}{1.110419in}}%
\pgfpathlineto{\pgfqpoint{2.146357in}{1.115889in}}%
\pgfpathlineto{\pgfqpoint{2.255084in}{1.120673in}}%
\pgfpathlineto{\pgfqpoint{2.363812in}{1.123201in}}%
\pgfpathlineto{\pgfqpoint{2.472539in}{1.132486in}}%
\pgfpathlineto{\pgfqpoint{2.581266in}{1.134362in}}%
\pgfpathlineto{\pgfqpoint{2.689994in}{1.140054in}}%
\pgfpathlineto{\pgfqpoint{2.798721in}{1.150043in}}%
\pgfpathlineto{\pgfqpoint{2.907449in}{1.157407in}}%
\pgfpathlineto{\pgfqpoint{3.016176in}{1.163022in}}%
\pgfpathlineto{\pgfqpoint{3.124903in}{1.169294in}}%
\pgfpathlineto{\pgfqpoint{3.233631in}{1.182811in}}%
\pgfpathlineto{\pgfqpoint{3.342358in}{1.188934in}}%
\pgfpathlineto{\pgfqpoint{3.451086in}{1.202318in}}%
\pgfpathlineto{\pgfqpoint{3.559813in}{1.215347in}}%
\pgfpathlineto{\pgfqpoint{3.668540in}{1.215244in}}%
\pgfpathlineto{\pgfqpoint{3.777268in}{1.224094in}}%
\pgfpathlineto{\pgfqpoint{3.885995in}{1.231964in}}%
\pgfpathlineto{\pgfqpoint{3.994723in}{1.252724in}}%
\pgfpathlineto{\pgfqpoint{4.103450in}{1.269448in}}%
\pgfpathlineto{\pgfqpoint{4.212178in}{1.268946in}}%
\pgfpathlineto{\pgfqpoint{4.320905in}{1.279989in}}%
\pgfpathlineto{\pgfqpoint{4.429632in}{1.305494in}}%
\pgfpathlineto{\pgfqpoint{4.538360in}{1.304185in}}%
\pgfpathlineto{\pgfqpoint{4.647087in}{1.310638in}}%
\pgfpathlineto{\pgfqpoint{4.755815in}{1.323517in}}%
\pgfpathlineto{\pgfqpoint{4.864542in}{1.330057in}}%
\pgfpathlineto{\pgfqpoint{4.973269in}{1.364330in}}%
\pgfpathlineto{\pgfqpoint{5.081997in}{1.356538in}}%
\pgfpathlineto{\pgfqpoint{5.190724in}{1.381539in}}%
\pgfpathlineto{\pgfqpoint{5.299452in}{1.407176in}}%
\pgfpathlineto{\pgfqpoint{5.408179in}{1.418366in}}%
\pgfpathlineto{\pgfqpoint{5.516906in}{1.399699in}}%
\pgfpathlineto{\pgfqpoint{5.625634in}{1.427143in}}%
\pgfpathlineto{\pgfqpoint{5.734361in}{1.438986in}}%
\pgfpathlineto{\pgfqpoint{5.843089in}{1.459037in}}%
\pgfpathlineto{\pgfqpoint{5.951816in}{1.470527in}}%
\pgfpathlineto{\pgfqpoint{6.060544in}{1.491828in}}%
\pgfpathlineto{\pgfqpoint{6.169271in}{1.515005in}}%
\pgfpathlineto{\pgfqpoint{6.277998in}{1.515974in}}%
\pgfpathlineto{\pgfqpoint{6.386726in}{1.539357in}}%
\pgfpathlineto{\pgfqpoint{6.495453in}{1.556823in}}%
\pgfpathlineto{\pgfqpoint{6.604181in}{1.563677in}}%
\pgfpathlineto{\pgfqpoint{6.712908in}{1.586511in}}%
\pgfpathlineto{\pgfqpoint{6.821635in}{1.609083in}}%
\pgfpathlineto{\pgfqpoint{6.930363in}{1.628351in}}%
\pgfpathlineto{\pgfqpoint{7.039090in}{1.647080in}}%
\pgfpathlineto{\pgfqpoint{7.147818in}{1.660785in}}%
\pgfpathlineto{\pgfqpoint{7.256545in}{1.681602in}}%
\pgfpathlineto{\pgfqpoint{7.365272in}{1.683373in}}%
\pgfpathlineto{\pgfqpoint{7.474000in}{1.685498in}}%
\pgfpathlineto{\pgfqpoint{7.582727in}{1.738039in}}%
\pgfpathlineto{\pgfqpoint{7.691455in}{1.746033in}}%
\pgfpathlineto{\pgfqpoint{7.800182in}{1.737433in}}%
\pgfpathlineto{\pgfqpoint{7.908910in}{1.776218in}}%
\pgfpathlineto{\pgfqpoint{8.017637in}{1.781912in}}%
\pgfpathlineto{\pgfqpoint{8.126364in}{1.828251in}}%
\pgfpathlineto{\pgfqpoint{8.235092in}{1.796029in}}%
\pgfpathlineto{\pgfqpoint{8.343819in}{1.836473in}}%
\pgfpathlineto{\pgfqpoint{8.452547in}{1.869478in}}%
\pgfpathlineto{\pgfqpoint{8.561274in}{1.895608in}}%
\pgfpathlineto{\pgfqpoint{8.670001in}{1.881887in}}%
\pgfpathlineto{\pgfqpoint{8.778729in}{1.906060in}}%
\pgfpathlineto{\pgfqpoint{8.887456in}{1.944551in}}%
\pgfpathlineto{\pgfqpoint{8.996184in}{1.949979in}}%
\pgfpathlineto{\pgfqpoint{9.104911in}{1.963237in}}%
\pgfpathlineto{\pgfqpoint{9.213638in}{1.962399in}}%
\pgfpathlineto{\pgfqpoint{9.322366in}{1.970141in}}%
\pgfpathlineto{\pgfqpoint{9.431093in}{2.032923in}}%
\pgfpathlineto{\pgfqpoint{9.539821in}{2.033288in}}%
\pgfpathlineto{\pgfqpoint{9.648548in}{2.053473in}}%
\pgfpathlineto{\pgfqpoint{9.757276in}{2.106395in}}%
\pgfpathlineto{\pgfqpoint{9.866003in}{2.128646in}}%
\pgfpathlineto{\pgfqpoint{9.974730in}{2.085071in}}%
\pgfpathlineto{\pgfqpoint{10.083458in}{2.165466in}}%
\pgfpathlineto{\pgfqpoint{10.192185in}{2.222389in}}%
\pgfpathlineto{\pgfqpoint{10.300913in}{2.153166in}}%
\pgfpathlineto{\pgfqpoint{10.409640in}{2.257554in}}%
\pgfpathlineto{\pgfqpoint{10.518367in}{2.230473in}}%
\pgfpathlineto{\pgfqpoint{10.627095in}{2.297160in}}%
\pgfpathlineto{\pgfqpoint{10.735822in}{2.239038in}}%
\pgfpathlineto{\pgfqpoint{10.844550in}{2.272613in}}%
\pgfpathlineto{\pgfqpoint{10.953277in}{2.315734in}}%
\pgfpathlineto{\pgfqpoint{11.062004in}{2.322542in}}%
\pgfpathlineto{\pgfqpoint{11.170732in}{2.340929in}}%
\pgfpathlineto{\pgfqpoint{11.279459in}{2.416897in}}%
\pgfpathlineto{\pgfqpoint{11.388187in}{2.431093in}}%
\pgfpathlineto{\pgfqpoint{11.496914in}{2.419663in}}%
\pgfpathlineto{\pgfqpoint{11.605642in}{2.492533in}}%
\pgfpathlineto{\pgfqpoint{11.714369in}{2.472584in}}%
\pgfpathlineto{\pgfqpoint{11.823096in}{2.500157in}}%
\pgfpathlineto{\pgfqpoint{11.931824in}{2.536366in}}%
\pgfpathlineto{\pgfqpoint{12.040551in}{2.540966in}}%
\pgfpathlineto{\pgfqpoint{12.149279in}{2.560282in}}%
\pgfpathlineto{\pgfqpoint{12.258006in}{2.586793in}}%
\pgfpathlineto{\pgfqpoint{12.366733in}{2.669568in}}%
\pgfpathlineto{\pgfqpoint{12.475461in}{2.623664in}}%
\pgfpathlineto{\pgfqpoint{12.584188in}{2.629433in}}%
\pgfpathlineto{\pgfqpoint{12.692916in}{2.615912in}}%
\pgfpathlineto{\pgfqpoint{12.801643in}{2.714156in}}%
\pgfusepath{stroke}%
\end{pgfscope}%
\begin{pgfscope}%
\pgfpathrectangle{\pgfqpoint{1.499429in}{0.839159in}}{\pgfqpoint{11.840415in}{5.967710in}}%
\pgfusepath{clip}%
\pgfsetrectcap%
\pgfsetroundjoin%
\pgfsetlinewidth{1.505625pt}%
\definecolor{currentstroke}{rgb}{0.890196,0.466667,0.760784}%
\pgfsetstrokecolor{currentstroke}%
\pgfsetdash{}{0pt}%
\pgfpathmoveto{\pgfqpoint{2.037629in}{1.110419in}}%
\pgfpathlineto{\pgfqpoint{2.146357in}{1.115889in}}%
\pgfpathlineto{\pgfqpoint{2.255084in}{1.120673in}}%
\pgfpathlineto{\pgfqpoint{2.363812in}{1.123201in}}%
\pgfpathlineto{\pgfqpoint{2.472539in}{1.132486in}}%
\pgfpathlineto{\pgfqpoint{2.581266in}{1.134362in}}%
\pgfpathlineto{\pgfqpoint{2.689994in}{1.140054in}}%
\pgfpathlineto{\pgfqpoint{2.798721in}{1.150043in}}%
\pgfpathlineto{\pgfqpoint{2.907449in}{1.157407in}}%
\pgfpathlineto{\pgfqpoint{3.016176in}{1.163022in}}%
\pgfpathlineto{\pgfqpoint{3.124903in}{1.169294in}}%
\pgfpathlineto{\pgfqpoint{3.233631in}{1.182811in}}%
\pgfpathlineto{\pgfqpoint{3.342358in}{1.188934in}}%
\pgfpathlineto{\pgfqpoint{3.451086in}{1.202318in}}%
\pgfpathlineto{\pgfqpoint{3.559813in}{1.215347in}}%
\pgfpathlineto{\pgfqpoint{3.668540in}{1.215244in}}%
\pgfpathlineto{\pgfqpoint{3.777268in}{1.224094in}}%
\pgfpathlineto{\pgfqpoint{3.885995in}{1.231964in}}%
\pgfpathlineto{\pgfqpoint{3.994723in}{1.252724in}}%
\pgfpathlineto{\pgfqpoint{4.103450in}{1.269448in}}%
\pgfpathlineto{\pgfqpoint{4.212178in}{1.268946in}}%
\pgfpathlineto{\pgfqpoint{4.320905in}{1.279989in}}%
\pgfpathlineto{\pgfqpoint{4.429632in}{1.305494in}}%
\pgfpathlineto{\pgfqpoint{4.538360in}{1.304185in}}%
\pgfpathlineto{\pgfqpoint{4.647087in}{1.310638in}}%
\pgfpathlineto{\pgfqpoint{4.755815in}{1.323517in}}%
\pgfpathlineto{\pgfqpoint{4.864542in}{1.330057in}}%
\pgfpathlineto{\pgfqpoint{4.973269in}{1.364330in}}%
\pgfpathlineto{\pgfqpoint{5.081997in}{1.356538in}}%
\pgfpathlineto{\pgfqpoint{5.190724in}{1.381539in}}%
\pgfpathlineto{\pgfqpoint{5.299452in}{1.407176in}}%
\pgfpathlineto{\pgfqpoint{5.408179in}{1.418366in}}%
\pgfpathlineto{\pgfqpoint{5.516906in}{1.399699in}}%
\pgfpathlineto{\pgfqpoint{5.625634in}{1.427143in}}%
\pgfpathlineto{\pgfqpoint{5.734361in}{1.438986in}}%
\pgfpathlineto{\pgfqpoint{5.843089in}{1.459037in}}%
\pgfpathlineto{\pgfqpoint{5.951816in}{1.470527in}}%
\pgfpathlineto{\pgfqpoint{6.060544in}{1.491828in}}%
\pgfpathlineto{\pgfqpoint{6.169271in}{1.515005in}}%
\pgfpathlineto{\pgfqpoint{6.277998in}{1.515974in}}%
\pgfpathlineto{\pgfqpoint{6.386726in}{1.539357in}}%
\pgfpathlineto{\pgfqpoint{6.495453in}{1.556823in}}%
\pgfpathlineto{\pgfqpoint{6.604181in}{1.563677in}}%
\pgfpathlineto{\pgfqpoint{6.712908in}{1.586511in}}%
\pgfpathlineto{\pgfqpoint{6.821635in}{1.609083in}}%
\pgfpathlineto{\pgfqpoint{6.930363in}{1.628351in}}%
\pgfpathlineto{\pgfqpoint{7.039090in}{1.647080in}}%
\pgfpathlineto{\pgfqpoint{7.147818in}{1.660785in}}%
\pgfpathlineto{\pgfqpoint{7.256545in}{1.681602in}}%
\pgfpathlineto{\pgfqpoint{7.365272in}{1.683373in}}%
\pgfpathlineto{\pgfqpoint{7.474000in}{1.685498in}}%
\pgfpathlineto{\pgfqpoint{7.582727in}{1.738039in}}%
\pgfpathlineto{\pgfqpoint{7.691455in}{1.746033in}}%
\pgfpathlineto{\pgfqpoint{7.800182in}{1.737433in}}%
\pgfpathlineto{\pgfqpoint{7.908910in}{1.776218in}}%
\pgfpathlineto{\pgfqpoint{8.017637in}{1.781912in}}%
\pgfpathlineto{\pgfqpoint{8.126364in}{1.828251in}}%
\pgfpathlineto{\pgfqpoint{8.235092in}{1.796029in}}%
\pgfpathlineto{\pgfqpoint{8.343819in}{1.836473in}}%
\pgfpathlineto{\pgfqpoint{8.452547in}{1.869478in}}%
\pgfpathlineto{\pgfqpoint{8.561274in}{1.895608in}}%
\pgfpathlineto{\pgfqpoint{8.670001in}{1.881887in}}%
\pgfpathlineto{\pgfqpoint{8.778729in}{1.906060in}}%
\pgfpathlineto{\pgfqpoint{8.887456in}{1.944551in}}%
\pgfpathlineto{\pgfqpoint{8.996184in}{1.949979in}}%
\pgfpathlineto{\pgfqpoint{9.104911in}{1.963237in}}%
\pgfpathlineto{\pgfqpoint{9.213638in}{1.962399in}}%
\pgfpathlineto{\pgfqpoint{9.322366in}{1.970141in}}%
\pgfpathlineto{\pgfqpoint{9.431093in}{2.032923in}}%
\pgfpathlineto{\pgfqpoint{9.539821in}{2.033288in}}%
\pgfpathlineto{\pgfqpoint{9.648548in}{2.053473in}}%
\pgfpathlineto{\pgfqpoint{9.757276in}{2.106395in}}%
\pgfpathlineto{\pgfqpoint{9.866003in}{2.128646in}}%
\pgfpathlineto{\pgfqpoint{9.974730in}{2.085071in}}%
\pgfpathlineto{\pgfqpoint{10.083458in}{2.165466in}}%
\pgfpathlineto{\pgfqpoint{10.192185in}{2.222389in}}%
\pgfpathlineto{\pgfqpoint{10.300913in}{2.153166in}}%
\pgfpathlineto{\pgfqpoint{10.409640in}{2.257554in}}%
\pgfpathlineto{\pgfqpoint{10.518367in}{2.230473in}}%
\pgfpathlineto{\pgfqpoint{10.627095in}{2.297160in}}%
\pgfpathlineto{\pgfqpoint{10.735822in}{2.239038in}}%
\pgfpathlineto{\pgfqpoint{10.844550in}{2.272613in}}%
\pgfpathlineto{\pgfqpoint{10.953277in}{2.315734in}}%
\pgfpathlineto{\pgfqpoint{11.062004in}{2.322542in}}%
\pgfpathlineto{\pgfqpoint{11.170732in}{2.340929in}}%
\pgfpathlineto{\pgfqpoint{11.279459in}{2.416897in}}%
\pgfpathlineto{\pgfqpoint{11.388187in}{2.431093in}}%
\pgfpathlineto{\pgfqpoint{11.496914in}{2.419663in}}%
\pgfpathlineto{\pgfqpoint{11.605642in}{2.492533in}}%
\pgfpathlineto{\pgfqpoint{11.714369in}{2.472584in}}%
\pgfpathlineto{\pgfqpoint{11.823096in}{2.500157in}}%
\pgfpathlineto{\pgfqpoint{11.931824in}{2.536366in}}%
\pgfpathlineto{\pgfqpoint{12.040551in}{2.540966in}}%
\pgfpathlineto{\pgfqpoint{12.149279in}{2.560282in}}%
\pgfpathlineto{\pgfqpoint{12.258006in}{2.586793in}}%
\pgfpathlineto{\pgfqpoint{12.366733in}{2.669568in}}%
\pgfpathlineto{\pgfqpoint{12.475461in}{2.623664in}}%
\pgfpathlineto{\pgfqpoint{12.584188in}{2.629433in}}%
\pgfpathlineto{\pgfqpoint{12.692916in}{2.615912in}}%
\pgfpathlineto{\pgfqpoint{12.801643in}{2.714156in}}%
\pgfusepath{stroke}%
\end{pgfscope}%
\begin{pgfscope}%
\pgfpathrectangle{\pgfqpoint{1.499429in}{0.839159in}}{\pgfqpoint{11.840415in}{5.967710in}}%
\pgfusepath{clip}%
\pgfsetrectcap%
\pgfsetroundjoin%
\pgfsetlinewidth{1.505625pt}%
\definecolor{currentstroke}{rgb}{0.498039,0.498039,0.498039}%
\pgfsetstrokecolor{currentstroke}%
\pgfsetdash{}{0pt}%
\pgfpathmoveto{\pgfqpoint{2.037629in}{1.110419in}}%
\pgfpathlineto{\pgfqpoint{2.146357in}{1.115889in}}%
\pgfpathlineto{\pgfqpoint{2.255084in}{1.120673in}}%
\pgfpathlineto{\pgfqpoint{2.363812in}{1.123201in}}%
\pgfpathlineto{\pgfqpoint{2.472539in}{1.132486in}}%
\pgfpathlineto{\pgfqpoint{2.581266in}{1.134362in}}%
\pgfpathlineto{\pgfqpoint{2.689994in}{1.140054in}}%
\pgfpathlineto{\pgfqpoint{2.798721in}{1.150043in}}%
\pgfpathlineto{\pgfqpoint{2.907449in}{1.157407in}}%
\pgfpathlineto{\pgfqpoint{3.016176in}{1.163022in}}%
\pgfpathlineto{\pgfqpoint{3.124903in}{1.169294in}}%
\pgfpathlineto{\pgfqpoint{3.233631in}{1.182811in}}%
\pgfpathlineto{\pgfqpoint{3.342358in}{1.188934in}}%
\pgfpathlineto{\pgfqpoint{3.451086in}{1.202318in}}%
\pgfpathlineto{\pgfqpoint{3.559813in}{1.215347in}}%
\pgfpathlineto{\pgfqpoint{3.668540in}{1.215244in}}%
\pgfpathlineto{\pgfqpoint{3.777268in}{1.224094in}}%
\pgfpathlineto{\pgfqpoint{3.885995in}{1.231964in}}%
\pgfpathlineto{\pgfqpoint{3.994723in}{1.252724in}}%
\pgfpathlineto{\pgfqpoint{4.103450in}{1.269448in}}%
\pgfpathlineto{\pgfqpoint{4.212178in}{1.268946in}}%
\pgfpathlineto{\pgfqpoint{4.320905in}{1.279989in}}%
\pgfpathlineto{\pgfqpoint{4.429632in}{1.305494in}}%
\pgfpathlineto{\pgfqpoint{4.538360in}{1.304185in}}%
\pgfpathlineto{\pgfqpoint{4.647087in}{1.310638in}}%
\pgfpathlineto{\pgfqpoint{4.755815in}{1.323517in}}%
\pgfpathlineto{\pgfqpoint{4.864542in}{1.330057in}}%
\pgfpathlineto{\pgfqpoint{4.973269in}{1.364330in}}%
\pgfpathlineto{\pgfqpoint{5.081997in}{1.356538in}}%
\pgfpathlineto{\pgfqpoint{5.190724in}{1.381539in}}%
\pgfpathlineto{\pgfqpoint{5.299452in}{1.407176in}}%
\pgfpathlineto{\pgfqpoint{5.408179in}{1.418366in}}%
\pgfpathlineto{\pgfqpoint{5.516906in}{1.399699in}}%
\pgfpathlineto{\pgfqpoint{5.625634in}{1.427143in}}%
\pgfpathlineto{\pgfqpoint{5.734361in}{1.438986in}}%
\pgfpathlineto{\pgfqpoint{5.843089in}{1.459037in}}%
\pgfpathlineto{\pgfqpoint{5.951816in}{1.470527in}}%
\pgfpathlineto{\pgfqpoint{6.060544in}{1.491828in}}%
\pgfpathlineto{\pgfqpoint{6.169271in}{1.515005in}}%
\pgfpathlineto{\pgfqpoint{6.277998in}{1.515974in}}%
\pgfpathlineto{\pgfqpoint{6.386726in}{1.539357in}}%
\pgfpathlineto{\pgfqpoint{6.495453in}{1.556823in}}%
\pgfpathlineto{\pgfqpoint{6.604181in}{1.563677in}}%
\pgfpathlineto{\pgfqpoint{6.712908in}{1.586511in}}%
\pgfpathlineto{\pgfqpoint{6.821635in}{1.609083in}}%
\pgfpathlineto{\pgfqpoint{6.930363in}{1.628351in}}%
\pgfpathlineto{\pgfqpoint{7.039090in}{1.647080in}}%
\pgfpathlineto{\pgfqpoint{7.147818in}{1.660785in}}%
\pgfpathlineto{\pgfqpoint{7.256545in}{1.681602in}}%
\pgfpathlineto{\pgfqpoint{7.365272in}{1.683373in}}%
\pgfpathlineto{\pgfqpoint{7.474000in}{1.685498in}}%
\pgfpathlineto{\pgfqpoint{7.582727in}{1.738039in}}%
\pgfpathlineto{\pgfqpoint{7.691455in}{1.746033in}}%
\pgfpathlineto{\pgfqpoint{7.800182in}{1.737433in}}%
\pgfpathlineto{\pgfqpoint{7.908910in}{1.776218in}}%
\pgfpathlineto{\pgfqpoint{8.017637in}{1.781912in}}%
\pgfpathlineto{\pgfqpoint{8.126364in}{1.828251in}}%
\pgfpathlineto{\pgfqpoint{8.235092in}{1.796029in}}%
\pgfpathlineto{\pgfqpoint{8.343819in}{1.836473in}}%
\pgfpathlineto{\pgfqpoint{8.452547in}{1.869478in}}%
\pgfpathlineto{\pgfqpoint{8.561274in}{1.895608in}}%
\pgfpathlineto{\pgfqpoint{8.670001in}{1.881887in}}%
\pgfpathlineto{\pgfqpoint{8.778729in}{1.906060in}}%
\pgfpathlineto{\pgfqpoint{8.887456in}{1.944551in}}%
\pgfpathlineto{\pgfqpoint{8.996184in}{1.949979in}}%
\pgfpathlineto{\pgfqpoint{9.104911in}{1.963237in}}%
\pgfpathlineto{\pgfqpoint{9.213638in}{1.962399in}}%
\pgfpathlineto{\pgfqpoint{9.322366in}{1.970141in}}%
\pgfpathlineto{\pgfqpoint{9.431093in}{2.032923in}}%
\pgfpathlineto{\pgfqpoint{9.539821in}{2.033288in}}%
\pgfpathlineto{\pgfqpoint{9.648548in}{2.053473in}}%
\pgfpathlineto{\pgfqpoint{9.757276in}{2.106395in}}%
\pgfpathlineto{\pgfqpoint{9.866003in}{2.128646in}}%
\pgfpathlineto{\pgfqpoint{9.974730in}{2.085071in}}%
\pgfpathlineto{\pgfqpoint{10.083458in}{2.165466in}}%
\pgfpathlineto{\pgfqpoint{10.192185in}{2.222389in}}%
\pgfpathlineto{\pgfqpoint{10.300913in}{2.153166in}}%
\pgfpathlineto{\pgfqpoint{10.409640in}{2.257554in}}%
\pgfpathlineto{\pgfqpoint{10.518367in}{2.230473in}}%
\pgfpathlineto{\pgfqpoint{10.627095in}{2.297160in}}%
\pgfpathlineto{\pgfqpoint{10.735822in}{2.239038in}}%
\pgfpathlineto{\pgfqpoint{10.844550in}{2.272613in}}%
\pgfpathlineto{\pgfqpoint{10.953277in}{2.315734in}}%
\pgfpathlineto{\pgfqpoint{11.062004in}{2.322542in}}%
\pgfpathlineto{\pgfqpoint{11.170732in}{2.340929in}}%
\pgfpathlineto{\pgfqpoint{11.279459in}{2.416897in}}%
\pgfpathlineto{\pgfqpoint{11.388187in}{2.431093in}}%
\pgfpathlineto{\pgfqpoint{11.496914in}{2.419663in}}%
\pgfpathlineto{\pgfqpoint{11.605642in}{2.492533in}}%
\pgfpathlineto{\pgfqpoint{11.714369in}{2.472584in}}%
\pgfpathlineto{\pgfqpoint{11.823096in}{2.500157in}}%
\pgfpathlineto{\pgfqpoint{11.931824in}{2.536366in}}%
\pgfpathlineto{\pgfqpoint{12.040551in}{2.540966in}}%
\pgfpathlineto{\pgfqpoint{12.149279in}{2.560282in}}%
\pgfpathlineto{\pgfqpoint{12.258006in}{2.586793in}}%
\pgfpathlineto{\pgfqpoint{12.366733in}{2.669568in}}%
\pgfpathlineto{\pgfqpoint{12.475461in}{2.623664in}}%
\pgfpathlineto{\pgfqpoint{12.584188in}{2.629433in}}%
\pgfpathlineto{\pgfqpoint{12.692916in}{2.615912in}}%
\pgfpathlineto{\pgfqpoint{12.801643in}{2.714156in}}%
\pgfusepath{stroke}%
\end{pgfscope}%
\begin{pgfscope}%
\pgfpathrectangle{\pgfqpoint{1.499429in}{0.839159in}}{\pgfqpoint{11.840415in}{5.967710in}}%
\pgfusepath{clip}%
\pgfsetrectcap%
\pgfsetroundjoin%
\pgfsetlinewidth{1.505625pt}%
\definecolor{currentstroke}{rgb}{0.737255,0.741176,0.133333}%
\pgfsetstrokecolor{currentstroke}%
\pgfsetdash{}{0pt}%
\pgfpathmoveto{\pgfqpoint{2.037629in}{1.110419in}}%
\pgfpathlineto{\pgfqpoint{2.146357in}{1.115889in}}%
\pgfpathlineto{\pgfqpoint{2.255084in}{1.120673in}}%
\pgfpathlineto{\pgfqpoint{2.363812in}{1.123201in}}%
\pgfpathlineto{\pgfqpoint{2.472539in}{1.132486in}}%
\pgfpathlineto{\pgfqpoint{2.581266in}{1.134362in}}%
\pgfpathlineto{\pgfqpoint{2.689994in}{1.140054in}}%
\pgfpathlineto{\pgfqpoint{2.798721in}{1.150043in}}%
\pgfpathlineto{\pgfqpoint{2.907449in}{1.157407in}}%
\pgfpathlineto{\pgfqpoint{3.016176in}{1.163022in}}%
\pgfpathlineto{\pgfqpoint{3.124903in}{1.169294in}}%
\pgfpathlineto{\pgfqpoint{3.233631in}{1.182811in}}%
\pgfpathlineto{\pgfqpoint{3.342358in}{1.188934in}}%
\pgfpathlineto{\pgfqpoint{3.451086in}{1.202318in}}%
\pgfpathlineto{\pgfqpoint{3.559813in}{1.215347in}}%
\pgfpathlineto{\pgfqpoint{3.668540in}{1.215244in}}%
\pgfpathlineto{\pgfqpoint{3.777268in}{1.224094in}}%
\pgfpathlineto{\pgfqpoint{3.885995in}{1.231964in}}%
\pgfpathlineto{\pgfqpoint{3.994723in}{1.252724in}}%
\pgfpathlineto{\pgfqpoint{4.103450in}{1.269448in}}%
\pgfpathlineto{\pgfqpoint{4.212178in}{1.268946in}}%
\pgfpathlineto{\pgfqpoint{4.320905in}{1.279989in}}%
\pgfpathlineto{\pgfqpoint{4.429632in}{1.305494in}}%
\pgfpathlineto{\pgfqpoint{4.538360in}{1.304185in}}%
\pgfpathlineto{\pgfqpoint{4.647087in}{1.310638in}}%
\pgfpathlineto{\pgfqpoint{4.755815in}{1.323517in}}%
\pgfpathlineto{\pgfqpoint{4.864542in}{1.330057in}}%
\pgfpathlineto{\pgfqpoint{4.973269in}{1.364330in}}%
\pgfpathlineto{\pgfqpoint{5.081997in}{1.356538in}}%
\pgfpathlineto{\pgfqpoint{5.190724in}{1.381539in}}%
\pgfpathlineto{\pgfqpoint{5.299452in}{1.407176in}}%
\pgfpathlineto{\pgfqpoint{5.408179in}{1.418366in}}%
\pgfpathlineto{\pgfqpoint{5.516906in}{1.399699in}}%
\pgfpathlineto{\pgfqpoint{5.625634in}{1.427143in}}%
\pgfpathlineto{\pgfqpoint{5.734361in}{1.438986in}}%
\pgfpathlineto{\pgfqpoint{5.843089in}{1.459037in}}%
\pgfpathlineto{\pgfqpoint{5.951816in}{1.470527in}}%
\pgfpathlineto{\pgfqpoint{6.060544in}{1.491828in}}%
\pgfpathlineto{\pgfqpoint{6.169271in}{1.515005in}}%
\pgfpathlineto{\pgfqpoint{6.277998in}{1.515974in}}%
\pgfpathlineto{\pgfqpoint{6.386726in}{1.539357in}}%
\pgfpathlineto{\pgfqpoint{6.495453in}{1.556823in}}%
\pgfpathlineto{\pgfqpoint{6.604181in}{1.563677in}}%
\pgfpathlineto{\pgfqpoint{6.712908in}{1.586511in}}%
\pgfpathlineto{\pgfqpoint{6.821635in}{1.609083in}}%
\pgfpathlineto{\pgfqpoint{6.930363in}{1.628351in}}%
\pgfpathlineto{\pgfqpoint{7.039090in}{1.647080in}}%
\pgfpathlineto{\pgfqpoint{7.147818in}{1.660785in}}%
\pgfpathlineto{\pgfqpoint{7.256545in}{1.681602in}}%
\pgfpathlineto{\pgfqpoint{7.365272in}{1.683373in}}%
\pgfpathlineto{\pgfqpoint{7.474000in}{1.685498in}}%
\pgfpathlineto{\pgfqpoint{7.582727in}{1.738039in}}%
\pgfpathlineto{\pgfqpoint{7.691455in}{1.746033in}}%
\pgfpathlineto{\pgfqpoint{7.800182in}{1.737433in}}%
\pgfpathlineto{\pgfqpoint{7.908910in}{1.776218in}}%
\pgfpathlineto{\pgfqpoint{8.017637in}{1.781912in}}%
\pgfpathlineto{\pgfqpoint{8.126364in}{1.828251in}}%
\pgfpathlineto{\pgfqpoint{8.235092in}{1.796029in}}%
\pgfpathlineto{\pgfqpoint{8.343819in}{1.836473in}}%
\pgfpathlineto{\pgfqpoint{8.452547in}{1.869478in}}%
\pgfpathlineto{\pgfqpoint{8.561274in}{1.895608in}}%
\pgfpathlineto{\pgfqpoint{8.670001in}{1.881887in}}%
\pgfpathlineto{\pgfqpoint{8.778729in}{1.906060in}}%
\pgfpathlineto{\pgfqpoint{8.887456in}{1.944551in}}%
\pgfpathlineto{\pgfqpoint{8.996184in}{1.949979in}}%
\pgfpathlineto{\pgfqpoint{9.104911in}{1.963237in}}%
\pgfpathlineto{\pgfqpoint{9.213638in}{1.962399in}}%
\pgfpathlineto{\pgfqpoint{9.322366in}{1.970141in}}%
\pgfpathlineto{\pgfqpoint{9.431093in}{2.032923in}}%
\pgfpathlineto{\pgfqpoint{9.539821in}{2.033288in}}%
\pgfpathlineto{\pgfqpoint{9.648548in}{2.053473in}}%
\pgfpathlineto{\pgfqpoint{9.757276in}{2.106395in}}%
\pgfpathlineto{\pgfqpoint{9.866003in}{2.128646in}}%
\pgfpathlineto{\pgfqpoint{9.974730in}{2.085071in}}%
\pgfpathlineto{\pgfqpoint{10.083458in}{2.165466in}}%
\pgfpathlineto{\pgfqpoint{10.192185in}{2.222389in}}%
\pgfpathlineto{\pgfqpoint{10.300913in}{2.153166in}}%
\pgfpathlineto{\pgfqpoint{10.409640in}{2.257554in}}%
\pgfpathlineto{\pgfqpoint{10.518367in}{2.230473in}}%
\pgfpathlineto{\pgfqpoint{10.627095in}{2.297160in}}%
\pgfpathlineto{\pgfqpoint{10.735822in}{2.239038in}}%
\pgfpathlineto{\pgfqpoint{10.844550in}{2.272613in}}%
\pgfpathlineto{\pgfqpoint{10.953277in}{2.315734in}}%
\pgfpathlineto{\pgfqpoint{11.062004in}{2.322542in}}%
\pgfpathlineto{\pgfqpoint{11.170732in}{2.340929in}}%
\pgfpathlineto{\pgfqpoint{11.279459in}{2.416897in}}%
\pgfpathlineto{\pgfqpoint{11.388187in}{2.431093in}}%
\pgfpathlineto{\pgfqpoint{11.496914in}{2.419663in}}%
\pgfpathlineto{\pgfqpoint{11.605642in}{2.492533in}}%
\pgfpathlineto{\pgfqpoint{11.714369in}{2.472584in}}%
\pgfpathlineto{\pgfqpoint{11.823096in}{2.500157in}}%
\pgfpathlineto{\pgfqpoint{11.931824in}{2.536366in}}%
\pgfpathlineto{\pgfqpoint{12.040551in}{2.540966in}}%
\pgfpathlineto{\pgfqpoint{12.149279in}{2.560282in}}%
\pgfpathlineto{\pgfqpoint{12.258006in}{2.586793in}}%
\pgfpathlineto{\pgfqpoint{12.366733in}{2.669568in}}%
\pgfpathlineto{\pgfqpoint{12.475461in}{2.623664in}}%
\pgfpathlineto{\pgfqpoint{12.584188in}{2.629433in}}%
\pgfpathlineto{\pgfqpoint{12.692916in}{2.615912in}}%
\pgfpathlineto{\pgfqpoint{12.801643in}{2.714156in}}%
\pgfusepath{stroke}%
\end{pgfscope}%
\begin{pgfscope}%
\pgfpathrectangle{\pgfqpoint{1.499429in}{0.839159in}}{\pgfqpoint{11.840415in}{5.967710in}}%
\pgfusepath{clip}%
\pgfsetrectcap%
\pgfsetroundjoin%
\pgfsetlinewidth{1.505625pt}%
\definecolor{currentstroke}{rgb}{0.090196,0.745098,0.811765}%
\pgfsetstrokecolor{currentstroke}%
\pgfsetdash{}{0pt}%
\pgfpathmoveto{\pgfqpoint{2.037629in}{1.110419in}}%
\pgfpathlineto{\pgfqpoint{2.146357in}{1.115889in}}%
\pgfpathlineto{\pgfqpoint{2.255084in}{1.120673in}}%
\pgfpathlineto{\pgfqpoint{2.363812in}{1.123201in}}%
\pgfpathlineto{\pgfqpoint{2.472539in}{1.132486in}}%
\pgfpathlineto{\pgfqpoint{2.581266in}{1.134362in}}%
\pgfpathlineto{\pgfqpoint{2.689994in}{1.140054in}}%
\pgfpathlineto{\pgfqpoint{2.798721in}{1.150043in}}%
\pgfpathlineto{\pgfqpoint{2.907449in}{1.157407in}}%
\pgfpathlineto{\pgfqpoint{3.016176in}{1.163022in}}%
\pgfpathlineto{\pgfqpoint{3.124903in}{1.169294in}}%
\pgfpathlineto{\pgfqpoint{3.233631in}{1.182811in}}%
\pgfpathlineto{\pgfqpoint{3.342358in}{1.188934in}}%
\pgfpathlineto{\pgfqpoint{3.451086in}{1.202318in}}%
\pgfpathlineto{\pgfqpoint{3.559813in}{1.215347in}}%
\pgfpathlineto{\pgfqpoint{3.668540in}{1.215244in}}%
\pgfpathlineto{\pgfqpoint{3.777268in}{1.224094in}}%
\pgfpathlineto{\pgfqpoint{3.885995in}{1.231964in}}%
\pgfpathlineto{\pgfqpoint{3.994723in}{1.252724in}}%
\pgfpathlineto{\pgfqpoint{4.103450in}{1.269448in}}%
\pgfpathlineto{\pgfqpoint{4.212178in}{1.268946in}}%
\pgfpathlineto{\pgfqpoint{4.320905in}{1.279989in}}%
\pgfpathlineto{\pgfqpoint{4.429632in}{1.305494in}}%
\pgfpathlineto{\pgfqpoint{4.538360in}{1.304185in}}%
\pgfpathlineto{\pgfqpoint{4.647087in}{1.310638in}}%
\pgfpathlineto{\pgfqpoint{4.755815in}{1.323517in}}%
\pgfpathlineto{\pgfqpoint{4.864542in}{1.330057in}}%
\pgfpathlineto{\pgfqpoint{4.973269in}{1.364330in}}%
\pgfpathlineto{\pgfqpoint{5.081997in}{1.356538in}}%
\pgfpathlineto{\pgfqpoint{5.190724in}{1.381539in}}%
\pgfpathlineto{\pgfqpoint{5.299452in}{1.407176in}}%
\pgfpathlineto{\pgfqpoint{5.408179in}{1.418366in}}%
\pgfpathlineto{\pgfqpoint{5.516906in}{1.399699in}}%
\pgfpathlineto{\pgfqpoint{5.625634in}{1.427143in}}%
\pgfpathlineto{\pgfqpoint{5.734361in}{1.438986in}}%
\pgfpathlineto{\pgfqpoint{5.843089in}{1.459037in}}%
\pgfpathlineto{\pgfqpoint{5.951816in}{1.470527in}}%
\pgfpathlineto{\pgfqpoint{6.060544in}{1.491828in}}%
\pgfpathlineto{\pgfqpoint{6.169271in}{1.515005in}}%
\pgfpathlineto{\pgfqpoint{6.277998in}{1.515974in}}%
\pgfpathlineto{\pgfqpoint{6.386726in}{1.539357in}}%
\pgfpathlineto{\pgfqpoint{6.495453in}{1.556823in}}%
\pgfpathlineto{\pgfqpoint{6.604181in}{1.563677in}}%
\pgfpathlineto{\pgfqpoint{6.712908in}{1.586511in}}%
\pgfpathlineto{\pgfqpoint{6.821635in}{1.609083in}}%
\pgfpathlineto{\pgfqpoint{6.930363in}{1.628351in}}%
\pgfpathlineto{\pgfqpoint{7.039090in}{1.647080in}}%
\pgfpathlineto{\pgfqpoint{7.147818in}{1.660785in}}%
\pgfpathlineto{\pgfqpoint{7.256545in}{1.681602in}}%
\pgfpathlineto{\pgfqpoint{7.365272in}{1.683373in}}%
\pgfpathlineto{\pgfqpoint{7.474000in}{1.685498in}}%
\pgfpathlineto{\pgfqpoint{7.582727in}{1.738039in}}%
\pgfpathlineto{\pgfqpoint{7.691455in}{1.746033in}}%
\pgfpathlineto{\pgfqpoint{7.800182in}{1.737433in}}%
\pgfpathlineto{\pgfqpoint{7.908910in}{1.776218in}}%
\pgfpathlineto{\pgfqpoint{8.017637in}{1.781912in}}%
\pgfpathlineto{\pgfqpoint{8.126364in}{1.828251in}}%
\pgfpathlineto{\pgfqpoint{8.235092in}{1.796029in}}%
\pgfpathlineto{\pgfqpoint{8.343819in}{1.836473in}}%
\pgfpathlineto{\pgfqpoint{8.452547in}{1.869478in}}%
\pgfpathlineto{\pgfqpoint{8.561274in}{1.895608in}}%
\pgfpathlineto{\pgfqpoint{8.670001in}{1.881887in}}%
\pgfpathlineto{\pgfqpoint{8.778729in}{1.906060in}}%
\pgfpathlineto{\pgfqpoint{8.887456in}{1.944551in}}%
\pgfpathlineto{\pgfqpoint{8.996184in}{1.949979in}}%
\pgfpathlineto{\pgfqpoint{9.104911in}{1.963237in}}%
\pgfpathlineto{\pgfqpoint{9.213638in}{1.962399in}}%
\pgfpathlineto{\pgfqpoint{9.322366in}{1.970141in}}%
\pgfpathlineto{\pgfqpoint{9.431093in}{2.032923in}}%
\pgfpathlineto{\pgfqpoint{9.539821in}{2.033288in}}%
\pgfpathlineto{\pgfqpoint{9.648548in}{2.053473in}}%
\pgfpathlineto{\pgfqpoint{9.757276in}{2.106395in}}%
\pgfpathlineto{\pgfqpoint{9.866003in}{2.128646in}}%
\pgfpathlineto{\pgfqpoint{9.974730in}{2.085071in}}%
\pgfpathlineto{\pgfqpoint{10.083458in}{2.165466in}}%
\pgfpathlineto{\pgfqpoint{10.192185in}{2.222389in}}%
\pgfpathlineto{\pgfqpoint{10.300913in}{2.153166in}}%
\pgfpathlineto{\pgfqpoint{10.409640in}{2.257554in}}%
\pgfpathlineto{\pgfqpoint{10.518367in}{2.230473in}}%
\pgfpathlineto{\pgfqpoint{10.627095in}{2.297160in}}%
\pgfpathlineto{\pgfqpoint{10.735822in}{2.239038in}}%
\pgfpathlineto{\pgfqpoint{10.844550in}{2.272613in}}%
\pgfpathlineto{\pgfqpoint{10.953277in}{2.315734in}}%
\pgfpathlineto{\pgfqpoint{11.062004in}{2.322542in}}%
\pgfpathlineto{\pgfqpoint{11.170732in}{2.340929in}}%
\pgfpathlineto{\pgfqpoint{11.279459in}{2.416897in}}%
\pgfpathlineto{\pgfqpoint{11.388187in}{2.431093in}}%
\pgfpathlineto{\pgfqpoint{11.496914in}{2.419663in}}%
\pgfpathlineto{\pgfqpoint{11.605642in}{2.492533in}}%
\pgfpathlineto{\pgfqpoint{11.714369in}{2.472584in}}%
\pgfpathlineto{\pgfqpoint{11.823096in}{2.500157in}}%
\pgfpathlineto{\pgfqpoint{11.931824in}{2.536366in}}%
\pgfpathlineto{\pgfqpoint{12.040551in}{2.540966in}}%
\pgfpathlineto{\pgfqpoint{12.149279in}{2.560282in}}%
\pgfpathlineto{\pgfqpoint{12.258006in}{2.586793in}}%
\pgfpathlineto{\pgfqpoint{12.366733in}{2.669568in}}%
\pgfpathlineto{\pgfqpoint{12.475461in}{2.623664in}}%
\pgfpathlineto{\pgfqpoint{12.584188in}{2.629433in}}%
\pgfpathlineto{\pgfqpoint{12.692916in}{2.615912in}}%
\pgfpathlineto{\pgfqpoint{12.801643in}{2.714156in}}%
\pgfusepath{stroke}%
\end{pgfscope}%
\begin{pgfscope}%
\pgfsetrectcap%
\pgfsetmiterjoin%
\pgfsetlinewidth{0.803000pt}%
\definecolor{currentstroke}{rgb}{0.000000,0.000000,0.000000}%
\pgfsetstrokecolor{currentstroke}%
\pgfsetdash{}{0pt}%
\pgfpathmoveto{\pgfqpoint{1.499429in}{0.839159in}}%
\pgfpathlineto{\pgfqpoint{1.499429in}{6.806869in}}%
\pgfusepath{stroke}%
\end{pgfscope}%
\begin{pgfscope}%
\pgfsetrectcap%
\pgfsetmiterjoin%
\pgfsetlinewidth{0.803000pt}%
\definecolor{currentstroke}{rgb}{0.000000,0.000000,0.000000}%
\pgfsetstrokecolor{currentstroke}%
\pgfsetdash{}{0pt}%
\pgfpathmoveto{\pgfqpoint{13.339844in}{0.839159in}}%
\pgfpathlineto{\pgfqpoint{13.339844in}{6.806869in}}%
\pgfusepath{stroke}%
\end{pgfscope}%
\begin{pgfscope}%
\pgfsetrectcap%
\pgfsetmiterjoin%
\pgfsetlinewidth{0.803000pt}%
\definecolor{currentstroke}{rgb}{0.000000,0.000000,0.000000}%
\pgfsetstrokecolor{currentstroke}%
\pgfsetdash{}{0pt}%
\pgfpathmoveto{\pgfqpoint{1.499429in}{0.839159in}}%
\pgfpathlineto{\pgfqpoint{13.339844in}{0.839159in}}%
\pgfusepath{stroke}%
\end{pgfscope}%
\begin{pgfscope}%
\pgfsetrectcap%
\pgfsetmiterjoin%
\pgfsetlinewidth{0.803000pt}%
\definecolor{currentstroke}{rgb}{0.000000,0.000000,0.000000}%
\pgfsetstrokecolor{currentstroke}%
\pgfsetdash{}{0pt}%
\pgfpathmoveto{\pgfqpoint{1.499429in}{6.806869in}}%
\pgfpathlineto{\pgfqpoint{13.339844in}{6.806869in}}%
\pgfusepath{stroke}%
\end{pgfscope}%
\begin{pgfscope}%
\definecolor{textcolor}{rgb}{0.000000,0.000000,0.000000}%
\pgfsetstrokecolor{textcolor}%
\pgfsetfillcolor{textcolor}%
\pgftext[x=7.419636in,y=6.890203in,,base]{\color{textcolor}\rmfamily\fontsize{12.000000}{14.400000}\selectfont Effects of changing number of 2-opt runs}%
\end{pgfscope}%
\begin{pgfscope}%
\pgfsetbuttcap%
\pgfsetmiterjoin%
\definecolor{currentfill}{rgb}{1.000000,1.000000,1.000000}%
\pgfsetfillcolor{currentfill}%
\pgfsetfillopacity{0.800000}%
\pgfsetlinewidth{1.003750pt}%
\definecolor{currentstroke}{rgb}{0.800000,0.800000,0.800000}%
\pgfsetstrokecolor{currentstroke}%
\pgfsetstrokeopacity{0.800000}%
\pgfsetdash{}{0pt}%
\pgfpathmoveto{\pgfqpoint{1.596651in}{4.657186in}}%
\pgfpathlineto{\pgfqpoint{3.292248in}{4.657186in}}%
\pgfpathquadraticcurveto{\pgfqpoint{3.320026in}{4.657186in}}{\pgfqpoint{3.320026in}{4.684964in}}%
\pgfpathlineto{\pgfqpoint{3.320026in}{6.709647in}}%
\pgfpathquadraticcurveto{\pgfqpoint{3.320026in}{6.737425in}}{\pgfqpoint{3.292248in}{6.737425in}}%
\pgfpathlineto{\pgfqpoint{1.596651in}{6.737425in}}%
\pgfpathquadraticcurveto{\pgfqpoint{1.568873in}{6.737425in}}{\pgfqpoint{1.568873in}{6.709647in}}%
\pgfpathlineto{\pgfqpoint{1.568873in}{4.684964in}}%
\pgfpathquadraticcurveto{\pgfqpoint{1.568873in}{4.657186in}}{\pgfqpoint{1.596651in}{4.657186in}}%
\pgfpathclose%
\pgfusepath{stroke,fill}%
\end{pgfscope}%
\begin{pgfscope}%
\pgfsetrectcap%
\pgfsetroundjoin%
\pgfsetlinewidth{1.505625pt}%
\definecolor{currentstroke}{rgb}{0.121569,0.466667,0.705882}%
\pgfsetstrokecolor{currentstroke}%
\pgfsetdash{}{0pt}%
\pgfpathmoveto{\pgfqpoint{1.624429in}{6.624957in}}%
\pgfpathlineto{\pgfqpoint{1.902206in}{6.624957in}}%
\pgfusepath{stroke}%
\end{pgfscope}%
\begin{pgfscope}%
\definecolor{textcolor}{rgb}{0.000000,0.000000,0.000000}%
\pgfsetstrokecolor{textcolor}%
\pgfsetfillcolor{textcolor}%
\pgftext[x=2.013317in,y=6.576346in,left,base]{\color{textcolor}\rmfamily\fontsize{10.000000}{12.000000}\selectfont ER:0-custom}%
\end{pgfscope}%
\begin{pgfscope}%
\pgfsetrectcap%
\pgfsetroundjoin%
\pgfsetlinewidth{1.505625pt}%
\definecolor{currentstroke}{rgb}{1.000000,0.498039,0.054902}%
\pgfsetstrokecolor{currentstroke}%
\pgfsetdash{}{0pt}%
\pgfpathmoveto{\pgfqpoint{1.624429in}{6.421100in}}%
\pgfpathlineto{\pgfqpoint{1.902206in}{6.421100in}}%
\pgfusepath{stroke}%
\end{pgfscope}%
\begin{pgfscope}%
\definecolor{textcolor}{rgb}{0.000000,0.000000,0.000000}%
\pgfsetstrokecolor{textcolor}%
\pgfsetfillcolor{textcolor}%
\pgftext[x=2.013317in,y=6.372489in,left,base]{\color{textcolor}\rmfamily\fontsize{10.000000}{12.000000}\selectfont ER:1-custom}%
\end{pgfscope}%
\begin{pgfscope}%
\pgfsetrectcap%
\pgfsetroundjoin%
\pgfsetlinewidth{1.505625pt}%
\definecolor{currentstroke}{rgb}{0.172549,0.627451,0.172549}%
\pgfsetstrokecolor{currentstroke}%
\pgfsetdash{}{0pt}%
\pgfpathmoveto{\pgfqpoint{1.624429in}{6.217243in}}%
\pgfpathlineto{\pgfqpoint{1.902206in}{6.217243in}}%
\pgfusepath{stroke}%
\end{pgfscope}%
\begin{pgfscope}%
\definecolor{textcolor}{rgb}{0.000000,0.000000,0.000000}%
\pgfsetstrokecolor{textcolor}%
\pgfsetfillcolor{textcolor}%
\pgftext[x=2.013317in,y=6.168632in,left,base]{\color{textcolor}\rmfamily\fontsize{10.000000}{12.000000}\selectfont ER:2-custom}%
\end{pgfscope}%
\begin{pgfscope}%
\pgfsetrectcap%
\pgfsetroundjoin%
\pgfsetlinewidth{1.505625pt}%
\definecolor{currentstroke}{rgb}{0.839216,0.152941,0.156863}%
\pgfsetstrokecolor{currentstroke}%
\pgfsetdash{}{0pt}%
\pgfpathmoveto{\pgfqpoint{1.624429in}{6.013386in}}%
\pgfpathlineto{\pgfqpoint{1.902206in}{6.013386in}}%
\pgfusepath{stroke}%
\end{pgfscope}%
\begin{pgfscope}%
\definecolor{textcolor}{rgb}{0.000000,0.000000,0.000000}%
\pgfsetstrokecolor{textcolor}%
\pgfsetfillcolor{textcolor}%
\pgftext[x=2.013317in,y=5.964774in,left,base]{\color{textcolor}\rmfamily\fontsize{10.000000}{12.000000}\selectfont ER:3-custom}%
\end{pgfscope}%
\begin{pgfscope}%
\pgfsetrectcap%
\pgfsetroundjoin%
\pgfsetlinewidth{1.505625pt}%
\definecolor{currentstroke}{rgb}{0.580392,0.403922,0.741176}%
\pgfsetstrokecolor{currentstroke}%
\pgfsetdash{}{0pt}%
\pgfpathmoveto{\pgfqpoint{1.624429in}{5.809528in}}%
\pgfpathlineto{\pgfqpoint{1.902206in}{5.809528in}}%
\pgfusepath{stroke}%
\end{pgfscope}%
\begin{pgfscope}%
\definecolor{textcolor}{rgb}{0.000000,0.000000,0.000000}%
\pgfsetstrokecolor{textcolor}%
\pgfsetfillcolor{textcolor}%
\pgftext[x=2.013317in,y=5.760917in,left,base]{\color{textcolor}\rmfamily\fontsize{10.000000}{12.000000}\selectfont ER:4-custom}%
\end{pgfscope}%
\begin{pgfscope}%
\pgfsetrectcap%
\pgfsetroundjoin%
\pgfsetlinewidth{1.505625pt}%
\definecolor{currentstroke}{rgb}{0.549020,0.337255,0.294118}%
\pgfsetstrokecolor{currentstroke}%
\pgfsetdash{}{0pt}%
\pgfpathmoveto{\pgfqpoint{1.624429in}{5.605671in}}%
\pgfpathlineto{\pgfqpoint{1.902206in}{5.605671in}}%
\pgfusepath{stroke}%
\end{pgfscope}%
\begin{pgfscope}%
\definecolor{textcolor}{rgb}{0.000000,0.000000,0.000000}%
\pgfsetstrokecolor{textcolor}%
\pgfsetfillcolor{textcolor}%
\pgftext[x=2.013317in,y=5.557060in,left,base]{\color{textcolor}\rmfamily\fontsize{10.000000}{12.000000}\selectfont ER:5-custom}%
\end{pgfscope}%
\begin{pgfscope}%
\pgfsetrectcap%
\pgfsetroundjoin%
\pgfsetlinewidth{1.505625pt}%
\definecolor{currentstroke}{rgb}{0.890196,0.466667,0.760784}%
\pgfsetstrokecolor{currentstroke}%
\pgfsetdash{}{0pt}%
\pgfpathmoveto{\pgfqpoint{1.624429in}{5.401814in}}%
\pgfpathlineto{\pgfqpoint{1.902206in}{5.401814in}}%
\pgfusepath{stroke}%
\end{pgfscope}%
\begin{pgfscope}%
\definecolor{textcolor}{rgb}{0.000000,0.000000,0.000000}%
\pgfsetstrokecolor{textcolor}%
\pgfsetfillcolor{textcolor}%
\pgftext[x=2.013317in,y=5.353203in,left,base]{\color{textcolor}\rmfamily\fontsize{10.000000}{12.000000}\selectfont ER:10-custom}%
\end{pgfscope}%
\begin{pgfscope}%
\pgfsetrectcap%
\pgfsetroundjoin%
\pgfsetlinewidth{1.505625pt}%
\definecolor{currentstroke}{rgb}{0.498039,0.498039,0.498039}%
\pgfsetstrokecolor{currentstroke}%
\pgfsetdash{}{0pt}%
\pgfpathmoveto{\pgfqpoint{1.624429in}{5.197957in}}%
\pgfpathlineto{\pgfqpoint{1.902206in}{5.197957in}}%
\pgfusepath{stroke}%
\end{pgfscope}%
\begin{pgfscope}%
\definecolor{textcolor}{rgb}{0.000000,0.000000,0.000000}%
\pgfsetstrokecolor{textcolor}%
\pgfsetfillcolor{textcolor}%
\pgftext[x=2.013317in,y=5.149346in,left,base]{\color{textcolor}\rmfamily\fontsize{10.000000}{12.000000}\selectfont ER:100-custom}%
\end{pgfscope}%
\begin{pgfscope}%
\pgfsetrectcap%
\pgfsetroundjoin%
\pgfsetlinewidth{1.505625pt}%
\definecolor{currentstroke}{rgb}{0.737255,0.741176,0.133333}%
\pgfsetstrokecolor{currentstroke}%
\pgfsetdash{}{0pt}%
\pgfpathmoveto{\pgfqpoint{1.624429in}{4.994099in}}%
\pgfpathlineto{\pgfqpoint{1.902206in}{4.994099in}}%
\pgfusepath{stroke}%
\end{pgfscope}%
\begin{pgfscope}%
\definecolor{textcolor}{rgb}{0.000000,0.000000,0.000000}%
\pgfsetstrokecolor{textcolor}%
\pgfsetfillcolor{textcolor}%
\pgftext[x=2.013317in,y=4.945488in,left,base]{\color{textcolor}\rmfamily\fontsize{10.000000}{12.000000}\selectfont ER:1000-custom}%
\end{pgfscope}%
\begin{pgfscope}%
\pgfsetrectcap%
\pgfsetroundjoin%
\pgfsetlinewidth{1.505625pt}%
\definecolor{currentstroke}{rgb}{0.090196,0.745098,0.811765}%
\pgfsetstrokecolor{currentstroke}%
\pgfsetdash{}{0pt}%
\pgfpathmoveto{\pgfqpoint{1.624429in}{4.790242in}}%
\pgfpathlineto{\pgfqpoint{1.902206in}{4.790242in}}%
\pgfusepath{stroke}%
\end{pgfscope}%
\begin{pgfscope}%
\definecolor{textcolor}{rgb}{0.000000,0.000000,0.000000}%
\pgfsetstrokecolor{textcolor}%
\pgfsetfillcolor{textcolor}%
\pgftext[x=2.013317in,y=4.741631in,left,base]{\color{textcolor}\rmfamily\fontsize{10.000000}{12.000000}\selectfont ER:10000-custom}%
\end{pgfscope}%
\end{pgfpicture}%
\makeatother%
\endgroup%
}
\end{figure}
\begin{figure}[p]
    \caption{Example Euclidian TSP problem where custom fares very well}
    \centering
    \scalebox{0.5}{%% Creator: Matplotlib, PGF backend
%%
%% To include the figure in your LaTeX document, write
%%   \input{<filename>.pgf}
%%
%% Make sure the required packages are loaded in your preamble
%%   \usepackage{pgf}
%%
%% and, on pdftex
%%   \usepackage[utf8]{inputenc}\DeclareUnicodeCharacter{2212}{-}
%%
%% or, on luatex and xetex
%%   \usepackage{unicode-math}
%%
%% Figures using additional raster images can only be included by \input if
%% they are in the same directory as the main LaTeX file. For loading figures
%% from other directories you can use the `import` package
%%   \usepackage{import}
%%
%% and then include the figures with
%%   \import{<path to file>}{<filename>.pgf}
%%
%% Matplotlib used the following preamble
%%   \usepackage{fontspec}
%%   \setmainfont{DejaVuSerif.ttf}[Path=/home/maks/.local/share/virtualenvs/CW3-zMJxnm_q/lib/python3.7/site-packages/matplotlib/mpl-data/fonts/ttf/]
%%   \setsansfont{DejaVuSans.ttf}[Path=/home/maks/.local/share/virtualenvs/CW3-zMJxnm_q/lib/python3.7/site-packages/matplotlib/mpl-data/fonts/ttf/]
%%   \setmonofont{DejaVuSansMono.ttf}[Path=/home/maks/.local/share/virtualenvs/CW3-zMJxnm_q/lib/python3.7/site-packages/matplotlib/mpl-data/fonts/ttf/]
%%
\begingroup%
\makeatletter%
\begin{pgfpicture}%
\pgfpathrectangle{\pgfpointorigin}{\pgfqpoint{13.660000in}{7.340000in}}%
\pgfusepath{use as bounding box, clip}%
\begin{pgfscope}%
\pgfsetbuttcap%
\pgfsetmiterjoin%
\definecolor{currentfill}{rgb}{1.000000,1.000000,1.000000}%
\pgfsetfillcolor{currentfill}%
\pgfsetlinewidth{0.000000pt}%
\definecolor{currentstroke}{rgb}{1.000000,1.000000,1.000000}%
\pgfsetstrokecolor{currentstroke}%
\pgfsetdash{}{0pt}%
\pgfpathmoveto{\pgfqpoint{0.000000in}{0.000000in}}%
\pgfpathlineto{\pgfqpoint{13.660000in}{0.000000in}}%
\pgfpathlineto{\pgfqpoint{13.660000in}{7.340000in}}%
\pgfpathlineto{\pgfqpoint{0.000000in}{7.340000in}}%
\pgfpathclose%
\pgfusepath{fill}%
\end{pgfscope}%
\begin{pgfscope}%
\pgfsetbuttcap%
\pgfsetmiterjoin%
\definecolor{currentfill}{rgb}{1.000000,1.000000,1.000000}%
\pgfsetfillcolor{currentfill}%
\pgfsetlinewidth{0.000000pt}%
\definecolor{currentstroke}{rgb}{0.000000,0.000000,0.000000}%
\pgfsetstrokecolor{currentstroke}%
\pgfsetstrokeopacity{0.000000}%
\pgfsetdash{}{0pt}%
\pgfpathmoveto{\pgfqpoint{0.970666in}{4.121437in}}%
\pgfpathlineto{\pgfqpoint{6.669922in}{4.121437in}}%
\pgfpathlineto{\pgfqpoint{6.669922in}{6.806869in}}%
\pgfpathlineto{\pgfqpoint{0.970666in}{6.806869in}}%
\pgfpathclose%
\pgfusepath{fill}%
\end{pgfscope}%
\begin{pgfscope}%
\pgfpathrectangle{\pgfqpoint{0.970666in}{4.121437in}}{\pgfqpoint{5.699255in}{2.685432in}}%
\pgfusepath{clip}%
\pgfsetrectcap%
\pgfsetroundjoin%
\pgfsetlinewidth{1.505625pt}%
\definecolor{currentstroke}{rgb}{0.000000,0.000000,0.000000}%
\pgfsetstrokecolor{currentstroke}%
\pgfsetdash{}{0pt}%
\pgfpathmoveto{\pgfqpoint{5.262612in}{4.543769in}}%
\pgfpathlineto{\pgfqpoint{1.901872in}{4.830981in}}%
\pgfpathlineto{\pgfqpoint{5.990772in}{6.267041in}}%
\pgfpathlineto{\pgfqpoint{4.702489in}{6.488978in}}%
\pgfpathlineto{\pgfqpoint{1.873865in}{4.948477in}}%
\pgfpathlineto{\pgfqpoint{1.817853in}{5.405405in}}%
\pgfpathlineto{\pgfqpoint{2.910094in}{5.810113in}}%
\pgfpathlineto{\pgfqpoint{4.562458in}{5.274854in}}%
\pgfpathlineto{\pgfqpoint{6.326846in}{5.823168in}}%
\pgfpathlineto{\pgfqpoint{5.318624in}{6.397592in}}%
\pgfpathlineto{\pgfqpoint{4.842519in}{6.554253in}}%
\pgfpathlineto{\pgfqpoint{1.845859in}{6.280096in}}%
\pgfpathlineto{\pgfqpoint{2.686044in}{5.640397in}}%
\pgfpathlineto{\pgfqpoint{6.102797in}{6.410647in}}%
\pgfpathlineto{\pgfqpoint{5.570680in}{6.058160in}}%
\pgfpathlineto{\pgfqpoint{4.226384in}{5.914554in}}%
\pgfpathlineto{\pgfqpoint{1.509785in}{4.909312in}}%
\pgfpathlineto{\pgfqpoint{4.030340in}{4.387108in}}%
\pgfpathlineto{\pgfqpoint{4.982550in}{6.502033in}}%
\pgfpathlineto{\pgfqpoint{5.906754in}{6.358427in}}%
\pgfpathlineto{\pgfqpoint{5.234606in}{4.530714in}}%
\pgfpathlineto{\pgfqpoint{2.377976in}{6.253986in}}%
\pgfpathlineto{\pgfqpoint{1.369754in}{4.948477in}}%
\pgfpathlineto{\pgfqpoint{2.405983in}{4.648210in}}%
\pgfpathlineto{\pgfqpoint{4.310402in}{5.849278in}}%
\pgfpathlineto{\pgfqpoint{3.526229in}{6.384537in}}%
\pgfpathlineto{\pgfqpoint{1.873865in}{5.209579in}}%
\pgfpathlineto{\pgfqpoint{6.214822in}{5.222634in}}%
\pgfpathlineto{\pgfqpoint{5.318624in}{4.582935in}}%
\pgfpathlineto{\pgfqpoint{4.898532in}{4.478494in}}%
\pgfpathlineto{\pgfqpoint{2.938100in}{5.405405in}}%
\pgfpathlineto{\pgfqpoint{2.602026in}{6.306206in}}%
\pgfpathlineto{\pgfqpoint{2.461995in}{6.671749in}}%
\pgfpathlineto{\pgfqpoint{1.369754in}{6.410647in}}%
\pgfpathlineto{\pgfqpoint{4.058347in}{5.105138in}}%
\pgfpathlineto{\pgfqpoint{5.794729in}{5.562066in}}%
\pgfpathlineto{\pgfqpoint{3.554236in}{6.449812in}}%
\pgfpathlineto{\pgfqpoint{2.405983in}{5.588177in}}%
\pgfpathlineto{\pgfqpoint{4.982550in}{5.065973in}}%
\pgfpathlineto{\pgfqpoint{5.626692in}{4.635155in}}%
\pgfpathlineto{\pgfqpoint{4.086353in}{4.439329in}}%
\pgfpathlineto{\pgfqpoint{1.621810in}{4.739596in}}%
\pgfpathlineto{\pgfqpoint{4.142365in}{5.118193in}}%
\pgfpathlineto{\pgfqpoint{3.806291in}{5.418460in}}%
\pgfpathlineto{\pgfqpoint{6.242828in}{6.671749in}}%
\pgfpathlineto{\pgfqpoint{2.910094in}{5.940664in}}%
\pgfpathlineto{\pgfqpoint{2.714051in}{5.666507in}}%
\pgfpathlineto{\pgfqpoint{2.602026in}{5.562066in}}%
\pgfpathlineto{\pgfqpoint{2.630032in}{4.830981in}}%
\pgfpathlineto{\pgfqpoint{4.730495in}{4.987642in}}%
\pgfpathlineto{\pgfqpoint{3.890310in}{4.452384in}}%
\pgfpathlineto{\pgfqpoint{1.929878in}{4.543769in}}%
\pgfpathlineto{\pgfqpoint{3.442211in}{5.810113in}}%
\pgfpathlineto{\pgfqpoint{5.318624in}{5.888444in}}%
\pgfpathlineto{\pgfqpoint{3.946322in}{6.632584in}}%
\pgfpathlineto{\pgfqpoint{3.162149in}{6.475923in}}%
\pgfpathlineto{\pgfqpoint{2.153927in}{6.032050in}}%
\pgfpathlineto{\pgfqpoint{6.410865in}{5.979829in}}%
\pgfpathlineto{\pgfqpoint{4.590464in}{4.478494in}}%
\pgfpathlineto{\pgfqpoint{1.593804in}{4.387108in}}%
\pgfpathlineto{\pgfqpoint{2.658038in}{4.857091in}}%
\pgfpathlineto{\pgfqpoint{1.985890in}{5.862333in}}%
\pgfpathlineto{\pgfqpoint{3.078131in}{6.162601in}}%
\pgfpathlineto{\pgfqpoint{2.910094in}{5.340130in}}%
\pgfpathlineto{\pgfqpoint{4.898532in}{4.491549in}}%
\pgfpathlineto{\pgfqpoint{3.946322in}{6.005939in}}%
\pgfpathlineto{\pgfqpoint{1.285736in}{6.267041in}}%
\pgfpathlineto{\pgfqpoint{3.190155in}{5.470681in}}%
\pgfpathlineto{\pgfqpoint{1.845859in}{4.269612in}}%
\pgfpathlineto{\pgfqpoint{1.397761in}{5.979829in}}%
\pgfpathlineto{\pgfqpoint{5.710711in}{6.684804in}}%
\pgfpathlineto{\pgfqpoint{3.806291in}{5.157359in}}%
\pgfpathlineto{\pgfqpoint{5.262612in}{4.635155in}}%
\pgfpathlineto{\pgfqpoint{5.514667in}{4.778761in}}%
\pgfpathlineto{\pgfqpoint{4.506445in}{5.065973in}}%
\pgfpathlineto{\pgfqpoint{3.470217in}{4.987642in}}%
\pgfpathlineto{\pgfqpoint{2.377976in}{4.739596in}}%
\pgfpathlineto{\pgfqpoint{2.209939in}{6.175656in}}%
\pgfpathlineto{\pgfqpoint{5.430649in}{5.797058in}}%
\pgfpathlineto{\pgfqpoint{3.694266in}{5.092083in}}%
\pgfpathlineto{\pgfqpoint{1.733835in}{4.243502in}}%
\pgfpathlineto{\pgfqpoint{2.377976in}{6.423702in}}%
\pgfpathlineto{\pgfqpoint{5.234606in}{6.423702in}}%
\pgfpathlineto{\pgfqpoint{3.890310in}{6.293151in}}%
\pgfpathlineto{\pgfqpoint{1.733835in}{6.397592in}}%
\pgfpathlineto{\pgfqpoint{1.229724in}{6.332317in}}%
\pgfpathlineto{\pgfqpoint{1.901872in}{5.052918in}}%
\pgfpathlineto{\pgfqpoint{3.078131in}{4.400163in}}%
\pgfpathlineto{\pgfqpoint{5.990772in}{4.439329in}}%
\pgfpathlineto{\pgfqpoint{3.414205in}{5.940664in}}%
\pgfpathlineto{\pgfqpoint{1.677822in}{5.979829in}}%
\pgfpathlineto{\pgfqpoint{2.405983in}{4.491549in}}%
\pgfpathlineto{\pgfqpoint{5.486661in}{6.018995in}}%
\pgfpathlineto{\pgfqpoint{5.906754in}{5.927609in}}%
\pgfpathlineto{\pgfqpoint{5.402643in}{5.013753in}}%
\pgfpathlineto{\pgfqpoint{5.262612in}{4.543769in}}%
\pgfusepath{stroke}%
\end{pgfscope}%
\begin{pgfscope}%
\pgfpathrectangle{\pgfqpoint{0.970666in}{4.121437in}}{\pgfqpoint{5.699255in}{2.685432in}}%
\pgfusepath{clip}%
\pgfsetbuttcap%
\pgfsetroundjoin%
\definecolor{currentfill}{rgb}{1.000000,0.000000,0.000000}%
\pgfsetfillcolor{currentfill}%
\pgfsetlinewidth{1.003750pt}%
\definecolor{currentstroke}{rgb}{1.000000,0.000000,0.000000}%
\pgfsetstrokecolor{currentstroke}%
\pgfsetdash{}{0pt}%
\pgfpathmoveto{\pgfqpoint{5.262612in}{4.502103in}}%
\pgfpathcurveto{\pgfqpoint{5.273662in}{4.502103in}}{\pgfqpoint{5.284261in}{4.506493in}}{\pgfqpoint{5.292075in}{4.514307in}}%
\pgfpathcurveto{\pgfqpoint{5.299888in}{4.522120in}}{\pgfqpoint{5.304279in}{4.532719in}}{\pgfqpoint{5.304279in}{4.543769in}}%
\pgfpathcurveto{\pgfqpoint{5.304279in}{4.554819in}}{\pgfqpoint{5.299888in}{4.565418in}}{\pgfqpoint{5.292075in}{4.573232in}}%
\pgfpathcurveto{\pgfqpoint{5.284261in}{4.581046in}}{\pgfqpoint{5.273662in}{4.585436in}}{\pgfqpoint{5.262612in}{4.585436in}}%
\pgfpathcurveto{\pgfqpoint{5.251562in}{4.585436in}}{\pgfqpoint{5.240963in}{4.581046in}}{\pgfqpoint{5.233149in}{4.573232in}}%
\pgfpathcurveto{\pgfqpoint{5.225335in}{4.565418in}}{\pgfqpoint{5.220945in}{4.554819in}}{\pgfqpoint{5.220945in}{4.543769in}}%
\pgfpathcurveto{\pgfqpoint{5.220945in}{4.532719in}}{\pgfqpoint{5.225335in}{4.522120in}}{\pgfqpoint{5.233149in}{4.514307in}}%
\pgfpathcurveto{\pgfqpoint{5.240963in}{4.506493in}}{\pgfqpoint{5.251562in}{4.502103in}}{\pgfqpoint{5.262612in}{4.502103in}}%
\pgfpathclose%
\pgfusepath{stroke,fill}%
\end{pgfscope}%
\begin{pgfscope}%
\pgfpathrectangle{\pgfqpoint{0.970666in}{4.121437in}}{\pgfqpoint{5.699255in}{2.685432in}}%
\pgfusepath{clip}%
\pgfsetbuttcap%
\pgfsetroundjoin%
\definecolor{currentfill}{rgb}{0.750000,0.750000,0.000000}%
\pgfsetfillcolor{currentfill}%
\pgfsetlinewidth{1.003750pt}%
\definecolor{currentstroke}{rgb}{0.750000,0.750000,0.000000}%
\pgfsetstrokecolor{currentstroke}%
\pgfsetdash{}{0pt}%
\pgfpathmoveto{\pgfqpoint{1.901872in}{4.789315in}}%
\pgfpathcurveto{\pgfqpoint{1.912922in}{4.789315in}}{\pgfqpoint{1.923521in}{4.793705in}}{\pgfqpoint{1.931334in}{4.801518in}}%
\pgfpathcurveto{\pgfqpoint{1.939148in}{4.809332in}}{\pgfqpoint{1.943538in}{4.819931in}}{\pgfqpoint{1.943538in}{4.830981in}}%
\pgfpathcurveto{\pgfqpoint{1.943538in}{4.842031in}}{\pgfqpoint{1.939148in}{4.852630in}}{\pgfqpoint{1.931334in}{4.860444in}}%
\pgfpathcurveto{\pgfqpoint{1.923521in}{4.868258in}}{\pgfqpoint{1.912922in}{4.872648in}}{\pgfqpoint{1.901872in}{4.872648in}}%
\pgfpathcurveto{\pgfqpoint{1.890821in}{4.872648in}}{\pgfqpoint{1.880222in}{4.868258in}}{\pgfqpoint{1.872409in}{4.860444in}}%
\pgfpathcurveto{\pgfqpoint{1.864595in}{4.852630in}}{\pgfqpoint{1.860205in}{4.842031in}}{\pgfqpoint{1.860205in}{4.830981in}}%
\pgfpathcurveto{\pgfqpoint{1.860205in}{4.819931in}}{\pgfqpoint{1.864595in}{4.809332in}}{\pgfqpoint{1.872409in}{4.801518in}}%
\pgfpathcurveto{\pgfqpoint{1.880222in}{4.793705in}}{\pgfqpoint{1.890821in}{4.789315in}}{\pgfqpoint{1.901872in}{4.789315in}}%
\pgfpathclose%
\pgfusepath{stroke,fill}%
\end{pgfscope}%
\begin{pgfscope}%
\pgfpathrectangle{\pgfqpoint{0.970666in}{4.121437in}}{\pgfqpoint{5.699255in}{2.685432in}}%
\pgfusepath{clip}%
\pgfsetbuttcap%
\pgfsetroundjoin%
\definecolor{currentfill}{rgb}{0.000000,0.000000,0.000000}%
\pgfsetfillcolor{currentfill}%
\pgfsetlinewidth{1.003750pt}%
\definecolor{currentstroke}{rgb}{0.000000,0.000000,0.000000}%
\pgfsetstrokecolor{currentstroke}%
\pgfsetdash{}{0pt}%
\pgfpathmoveto{\pgfqpoint{5.990772in}{6.225375in}}%
\pgfpathcurveto{\pgfqpoint{6.001822in}{6.225375in}}{\pgfqpoint{6.012421in}{6.229765in}}{\pgfqpoint{6.020235in}{6.237578in}}%
\pgfpathcurveto{\pgfqpoint{6.028049in}{6.245392in}}{\pgfqpoint{6.032439in}{6.255991in}}{\pgfqpoint{6.032439in}{6.267041in}}%
\pgfpathcurveto{\pgfqpoint{6.032439in}{6.278091in}}{\pgfqpoint{6.028049in}{6.288690in}}{\pgfqpoint{6.020235in}{6.296504in}}%
\pgfpathcurveto{\pgfqpoint{6.012421in}{6.304318in}}{\pgfqpoint{6.001822in}{6.308708in}}{\pgfqpoint{5.990772in}{6.308708in}}%
\pgfpathcurveto{\pgfqpoint{5.979722in}{6.308708in}}{\pgfqpoint{5.969123in}{6.304318in}}{\pgfqpoint{5.961309in}{6.296504in}}%
\pgfpathcurveto{\pgfqpoint{5.953496in}{6.288690in}}{\pgfqpoint{5.949106in}{6.278091in}}{\pgfqpoint{5.949106in}{6.267041in}}%
\pgfpathcurveto{\pgfqpoint{5.949106in}{6.255991in}}{\pgfqpoint{5.953496in}{6.245392in}}{\pgfqpoint{5.961309in}{6.237578in}}%
\pgfpathcurveto{\pgfqpoint{5.969123in}{6.229765in}}{\pgfqpoint{5.979722in}{6.225375in}}{\pgfqpoint{5.990772in}{6.225375in}}%
\pgfpathclose%
\pgfusepath{stroke,fill}%
\end{pgfscope}%
\begin{pgfscope}%
\pgfpathrectangle{\pgfqpoint{0.970666in}{4.121437in}}{\pgfqpoint{5.699255in}{2.685432in}}%
\pgfusepath{clip}%
\pgfsetbuttcap%
\pgfsetroundjoin%
\definecolor{currentfill}{rgb}{0.000000,0.000000,0.000000}%
\pgfsetfillcolor{currentfill}%
\pgfsetlinewidth{1.003750pt}%
\definecolor{currentstroke}{rgb}{0.000000,0.000000,0.000000}%
\pgfsetstrokecolor{currentstroke}%
\pgfsetdash{}{0pt}%
\pgfpathmoveto{\pgfqpoint{4.702489in}{6.447311in}}%
\pgfpathcurveto{\pgfqpoint{4.713539in}{6.447311in}}{\pgfqpoint{4.724138in}{6.451701in}}{\pgfqpoint{4.731951in}{6.459515in}}%
\pgfpathcurveto{\pgfqpoint{4.739765in}{6.467329in}}{\pgfqpoint{4.744155in}{6.477928in}}{\pgfqpoint{4.744155in}{6.488978in}}%
\pgfpathcurveto{\pgfqpoint{4.744155in}{6.500028in}}{\pgfqpoint{4.739765in}{6.510627in}}{\pgfqpoint{4.731951in}{6.518441in}}%
\pgfpathcurveto{\pgfqpoint{4.724138in}{6.526254in}}{\pgfqpoint{4.713539in}{6.530644in}}{\pgfqpoint{4.702489in}{6.530644in}}%
\pgfpathcurveto{\pgfqpoint{4.691438in}{6.530644in}}{\pgfqpoint{4.680839in}{6.526254in}}{\pgfqpoint{4.673026in}{6.518441in}}%
\pgfpathcurveto{\pgfqpoint{4.665212in}{6.510627in}}{\pgfqpoint{4.660822in}{6.500028in}}{\pgfqpoint{4.660822in}{6.488978in}}%
\pgfpathcurveto{\pgfqpoint{4.660822in}{6.477928in}}{\pgfqpoint{4.665212in}{6.467329in}}{\pgfqpoint{4.673026in}{6.459515in}}%
\pgfpathcurveto{\pgfqpoint{4.680839in}{6.451701in}}{\pgfqpoint{4.691438in}{6.447311in}}{\pgfqpoint{4.702489in}{6.447311in}}%
\pgfpathclose%
\pgfusepath{stroke,fill}%
\end{pgfscope}%
\begin{pgfscope}%
\pgfpathrectangle{\pgfqpoint{0.970666in}{4.121437in}}{\pgfqpoint{5.699255in}{2.685432in}}%
\pgfusepath{clip}%
\pgfsetbuttcap%
\pgfsetroundjoin%
\definecolor{currentfill}{rgb}{0.000000,0.000000,0.000000}%
\pgfsetfillcolor{currentfill}%
\pgfsetlinewidth{1.003750pt}%
\definecolor{currentstroke}{rgb}{0.000000,0.000000,0.000000}%
\pgfsetstrokecolor{currentstroke}%
\pgfsetdash{}{0pt}%
\pgfpathmoveto{\pgfqpoint{1.873865in}{4.906810in}}%
\pgfpathcurveto{\pgfqpoint{1.884916in}{4.906810in}}{\pgfqpoint{1.895515in}{4.911201in}}{\pgfqpoint{1.903328in}{4.919014in}}%
\pgfpathcurveto{\pgfqpoint{1.911142in}{4.926828in}}{\pgfqpoint{1.915532in}{4.937427in}}{\pgfqpoint{1.915532in}{4.948477in}}%
\pgfpathcurveto{\pgfqpoint{1.915532in}{4.959527in}}{\pgfqpoint{1.911142in}{4.970126in}}{\pgfqpoint{1.903328in}{4.977940in}}%
\pgfpathcurveto{\pgfqpoint{1.895515in}{4.985753in}}{\pgfqpoint{1.884916in}{4.990144in}}{\pgfqpoint{1.873865in}{4.990144in}}%
\pgfpathcurveto{\pgfqpoint{1.862815in}{4.990144in}}{\pgfqpoint{1.852216in}{4.985753in}}{\pgfqpoint{1.844403in}{4.977940in}}%
\pgfpathcurveto{\pgfqpoint{1.836589in}{4.970126in}}{\pgfqpoint{1.832199in}{4.959527in}}{\pgfqpoint{1.832199in}{4.948477in}}%
\pgfpathcurveto{\pgfqpoint{1.832199in}{4.937427in}}{\pgfqpoint{1.836589in}{4.926828in}}{\pgfqpoint{1.844403in}{4.919014in}}%
\pgfpathcurveto{\pgfqpoint{1.852216in}{4.911201in}}{\pgfqpoint{1.862815in}{4.906810in}}{\pgfqpoint{1.873865in}{4.906810in}}%
\pgfpathclose%
\pgfusepath{stroke,fill}%
\end{pgfscope}%
\begin{pgfscope}%
\pgfpathrectangle{\pgfqpoint{0.970666in}{4.121437in}}{\pgfqpoint{5.699255in}{2.685432in}}%
\pgfusepath{clip}%
\pgfsetbuttcap%
\pgfsetroundjoin%
\definecolor{currentfill}{rgb}{0.000000,0.000000,0.000000}%
\pgfsetfillcolor{currentfill}%
\pgfsetlinewidth{1.003750pt}%
\definecolor{currentstroke}{rgb}{0.000000,0.000000,0.000000}%
\pgfsetstrokecolor{currentstroke}%
\pgfsetdash{}{0pt}%
\pgfpathmoveto{\pgfqpoint{1.817853in}{5.363739in}}%
\pgfpathcurveto{\pgfqpoint{1.828903in}{5.363739in}}{\pgfqpoint{1.839502in}{5.368129in}}{\pgfqpoint{1.847316in}{5.375942in}}%
\pgfpathcurveto{\pgfqpoint{1.855130in}{5.383756in}}{\pgfqpoint{1.859520in}{5.394355in}}{\pgfqpoint{1.859520in}{5.405405in}}%
\pgfpathcurveto{\pgfqpoint{1.859520in}{5.416455in}}{\pgfqpoint{1.855130in}{5.427054in}}{\pgfqpoint{1.847316in}{5.434868in}}%
\pgfpathcurveto{\pgfqpoint{1.839502in}{5.442682in}}{\pgfqpoint{1.828903in}{5.447072in}}{\pgfqpoint{1.817853in}{5.447072in}}%
\pgfpathcurveto{\pgfqpoint{1.806803in}{5.447072in}}{\pgfqpoint{1.796204in}{5.442682in}}{\pgfqpoint{1.788390in}{5.434868in}}%
\pgfpathcurveto{\pgfqpoint{1.780577in}{5.427054in}}{\pgfqpoint{1.776186in}{5.416455in}}{\pgfqpoint{1.776186in}{5.405405in}}%
\pgfpathcurveto{\pgfqpoint{1.776186in}{5.394355in}}{\pgfqpoint{1.780577in}{5.383756in}}{\pgfqpoint{1.788390in}{5.375942in}}%
\pgfpathcurveto{\pgfqpoint{1.796204in}{5.368129in}}{\pgfqpoint{1.806803in}{5.363739in}}{\pgfqpoint{1.817853in}{5.363739in}}%
\pgfpathclose%
\pgfusepath{stroke,fill}%
\end{pgfscope}%
\begin{pgfscope}%
\pgfpathrectangle{\pgfqpoint{0.970666in}{4.121437in}}{\pgfqpoint{5.699255in}{2.685432in}}%
\pgfusepath{clip}%
\pgfsetbuttcap%
\pgfsetroundjoin%
\definecolor{currentfill}{rgb}{0.000000,0.000000,0.000000}%
\pgfsetfillcolor{currentfill}%
\pgfsetlinewidth{1.003750pt}%
\definecolor{currentstroke}{rgb}{0.000000,0.000000,0.000000}%
\pgfsetstrokecolor{currentstroke}%
\pgfsetdash{}{0pt}%
\pgfpathmoveto{\pgfqpoint{2.910094in}{5.768446in}}%
\pgfpathcurveto{\pgfqpoint{2.921144in}{5.768446in}}{\pgfqpoint{2.931743in}{5.772837in}}{\pgfqpoint{2.939556in}{5.780650in}}%
\pgfpathcurveto{\pgfqpoint{2.947370in}{5.788464in}}{\pgfqpoint{2.951760in}{5.799063in}}{\pgfqpoint{2.951760in}{5.810113in}}%
\pgfpathcurveto{\pgfqpoint{2.951760in}{5.821163in}}{\pgfqpoint{2.947370in}{5.831762in}}{\pgfqpoint{2.939556in}{5.839576in}}%
\pgfpathcurveto{\pgfqpoint{2.931743in}{5.847389in}}{\pgfqpoint{2.921144in}{5.851780in}}{\pgfqpoint{2.910094in}{5.851780in}}%
\pgfpathcurveto{\pgfqpoint{2.899044in}{5.851780in}}{\pgfqpoint{2.888445in}{5.847389in}}{\pgfqpoint{2.880631in}{5.839576in}}%
\pgfpathcurveto{\pgfqpoint{2.872817in}{5.831762in}}{\pgfqpoint{2.868427in}{5.821163in}}{\pgfqpoint{2.868427in}{5.810113in}}%
\pgfpathcurveto{\pgfqpoint{2.868427in}{5.799063in}}{\pgfqpoint{2.872817in}{5.788464in}}{\pgfqpoint{2.880631in}{5.780650in}}%
\pgfpathcurveto{\pgfqpoint{2.888445in}{5.772837in}}{\pgfqpoint{2.899044in}{5.768446in}}{\pgfqpoint{2.910094in}{5.768446in}}%
\pgfpathclose%
\pgfusepath{stroke,fill}%
\end{pgfscope}%
\begin{pgfscope}%
\pgfpathrectangle{\pgfqpoint{0.970666in}{4.121437in}}{\pgfqpoint{5.699255in}{2.685432in}}%
\pgfusepath{clip}%
\pgfsetbuttcap%
\pgfsetroundjoin%
\definecolor{currentfill}{rgb}{0.000000,0.000000,0.000000}%
\pgfsetfillcolor{currentfill}%
\pgfsetlinewidth{1.003750pt}%
\definecolor{currentstroke}{rgb}{0.000000,0.000000,0.000000}%
\pgfsetstrokecolor{currentstroke}%
\pgfsetdash{}{0pt}%
\pgfpathmoveto{\pgfqpoint{4.562458in}{5.233188in}}%
\pgfpathcurveto{\pgfqpoint{4.573508in}{5.233188in}}{\pgfqpoint{4.584107in}{5.237578in}}{\pgfqpoint{4.591920in}{5.245392in}}%
\pgfpathcurveto{\pgfqpoint{4.599734in}{5.253205in}}{\pgfqpoint{4.604124in}{5.263804in}}{\pgfqpoint{4.604124in}{5.274854in}}%
\pgfpathcurveto{\pgfqpoint{4.604124in}{5.285904in}}{\pgfqpoint{4.599734in}{5.296504in}}{\pgfqpoint{4.591920in}{5.304317in}}%
\pgfpathcurveto{\pgfqpoint{4.584107in}{5.312131in}}{\pgfqpoint{4.573508in}{5.316521in}}{\pgfqpoint{4.562458in}{5.316521in}}%
\pgfpathcurveto{\pgfqpoint{4.551408in}{5.316521in}}{\pgfqpoint{4.540808in}{5.312131in}}{\pgfqpoint{4.532995in}{5.304317in}}%
\pgfpathcurveto{\pgfqpoint{4.525181in}{5.296504in}}{\pgfqpoint{4.520791in}{5.285904in}}{\pgfqpoint{4.520791in}{5.274854in}}%
\pgfpathcurveto{\pgfqpoint{4.520791in}{5.263804in}}{\pgfqpoint{4.525181in}{5.253205in}}{\pgfqpoint{4.532995in}{5.245392in}}%
\pgfpathcurveto{\pgfqpoint{4.540808in}{5.237578in}}{\pgfqpoint{4.551408in}{5.233188in}}{\pgfqpoint{4.562458in}{5.233188in}}%
\pgfpathclose%
\pgfusepath{stroke,fill}%
\end{pgfscope}%
\begin{pgfscope}%
\pgfpathrectangle{\pgfqpoint{0.970666in}{4.121437in}}{\pgfqpoint{5.699255in}{2.685432in}}%
\pgfusepath{clip}%
\pgfsetbuttcap%
\pgfsetroundjoin%
\definecolor{currentfill}{rgb}{0.000000,0.000000,0.000000}%
\pgfsetfillcolor{currentfill}%
\pgfsetlinewidth{1.003750pt}%
\definecolor{currentstroke}{rgb}{0.000000,0.000000,0.000000}%
\pgfsetstrokecolor{currentstroke}%
\pgfsetdash{}{0pt}%
\pgfpathmoveto{\pgfqpoint{6.326846in}{5.781501in}}%
\pgfpathcurveto{\pgfqpoint{6.337896in}{5.781501in}}{\pgfqpoint{6.348495in}{5.785892in}}{\pgfqpoint{6.356309in}{5.793705in}}%
\pgfpathcurveto{\pgfqpoint{6.364123in}{5.801519in}}{\pgfqpoint{6.368513in}{5.812118in}}{\pgfqpoint{6.368513in}{5.823168in}}%
\pgfpathcurveto{\pgfqpoint{6.368513in}{5.834218in}}{\pgfqpoint{6.364123in}{5.844817in}}{\pgfqpoint{6.356309in}{5.852631in}}%
\pgfpathcurveto{\pgfqpoint{6.348495in}{5.860445in}}{\pgfqpoint{6.337896in}{5.864835in}}{\pgfqpoint{6.326846in}{5.864835in}}%
\pgfpathcurveto{\pgfqpoint{6.315796in}{5.864835in}}{\pgfqpoint{6.305197in}{5.860445in}}{\pgfqpoint{6.297384in}{5.852631in}}%
\pgfpathcurveto{\pgfqpoint{6.289570in}{5.844817in}}{\pgfqpoint{6.285180in}{5.834218in}}{\pgfqpoint{6.285180in}{5.823168in}}%
\pgfpathcurveto{\pgfqpoint{6.285180in}{5.812118in}}{\pgfqpoint{6.289570in}{5.801519in}}{\pgfqpoint{6.297384in}{5.793705in}}%
\pgfpathcurveto{\pgfqpoint{6.305197in}{5.785892in}}{\pgfqpoint{6.315796in}{5.781501in}}{\pgfqpoint{6.326846in}{5.781501in}}%
\pgfpathclose%
\pgfusepath{stroke,fill}%
\end{pgfscope}%
\begin{pgfscope}%
\pgfpathrectangle{\pgfqpoint{0.970666in}{4.121437in}}{\pgfqpoint{5.699255in}{2.685432in}}%
\pgfusepath{clip}%
\pgfsetbuttcap%
\pgfsetroundjoin%
\definecolor{currentfill}{rgb}{0.000000,0.000000,0.000000}%
\pgfsetfillcolor{currentfill}%
\pgfsetlinewidth{1.003750pt}%
\definecolor{currentstroke}{rgb}{0.000000,0.000000,0.000000}%
\pgfsetstrokecolor{currentstroke}%
\pgfsetdash{}{0pt}%
\pgfpathmoveto{\pgfqpoint{5.318624in}{6.355925in}}%
\pgfpathcurveto{\pgfqpoint{5.329674in}{6.355925in}}{\pgfqpoint{5.340273in}{6.360316in}}{\pgfqpoint{5.348087in}{6.368129in}}%
\pgfpathcurveto{\pgfqpoint{5.355901in}{6.375943in}}{\pgfqpoint{5.360291in}{6.386542in}}{\pgfqpoint{5.360291in}{6.397592in}}%
\pgfpathcurveto{\pgfqpoint{5.360291in}{6.408642in}}{\pgfqpoint{5.355901in}{6.419241in}}{\pgfqpoint{5.348087in}{6.427055in}}%
\pgfpathcurveto{\pgfqpoint{5.340273in}{6.434869in}}{\pgfqpoint{5.329674in}{6.439259in}}{\pgfqpoint{5.318624in}{6.439259in}}%
\pgfpathcurveto{\pgfqpoint{5.307574in}{6.439259in}}{\pgfqpoint{5.296975in}{6.434869in}}{\pgfqpoint{5.289161in}{6.427055in}}%
\pgfpathcurveto{\pgfqpoint{5.281348in}{6.419241in}}{\pgfqpoint{5.276958in}{6.408642in}}{\pgfqpoint{5.276958in}{6.397592in}}%
\pgfpathcurveto{\pgfqpoint{5.276958in}{6.386542in}}{\pgfqpoint{5.281348in}{6.375943in}}{\pgfqpoint{5.289161in}{6.368129in}}%
\pgfpathcurveto{\pgfqpoint{5.296975in}{6.360316in}}{\pgfqpoint{5.307574in}{6.355925in}}{\pgfqpoint{5.318624in}{6.355925in}}%
\pgfpathclose%
\pgfusepath{stroke,fill}%
\end{pgfscope}%
\begin{pgfscope}%
\pgfpathrectangle{\pgfqpoint{0.970666in}{4.121437in}}{\pgfqpoint{5.699255in}{2.685432in}}%
\pgfusepath{clip}%
\pgfsetbuttcap%
\pgfsetroundjoin%
\definecolor{currentfill}{rgb}{0.000000,0.000000,0.000000}%
\pgfsetfillcolor{currentfill}%
\pgfsetlinewidth{1.003750pt}%
\definecolor{currentstroke}{rgb}{0.000000,0.000000,0.000000}%
\pgfsetstrokecolor{currentstroke}%
\pgfsetdash{}{0pt}%
\pgfpathmoveto{\pgfqpoint{4.842519in}{6.512587in}}%
\pgfpathcurveto{\pgfqpoint{4.853569in}{6.512587in}}{\pgfqpoint{4.864169in}{6.516977in}}{\pgfqpoint{4.871982in}{6.524790in}}%
\pgfpathcurveto{\pgfqpoint{4.879796in}{6.532604in}}{\pgfqpoint{4.884186in}{6.543203in}}{\pgfqpoint{4.884186in}{6.554253in}}%
\pgfpathcurveto{\pgfqpoint{4.884186in}{6.565303in}}{\pgfqpoint{4.879796in}{6.575902in}}{\pgfqpoint{4.871982in}{6.583716in}}%
\pgfpathcurveto{\pgfqpoint{4.864169in}{6.591530in}}{\pgfqpoint{4.853569in}{6.595920in}}{\pgfqpoint{4.842519in}{6.595920in}}%
\pgfpathcurveto{\pgfqpoint{4.831469in}{6.595920in}}{\pgfqpoint{4.820870in}{6.591530in}}{\pgfqpoint{4.813057in}{6.583716in}}%
\pgfpathcurveto{\pgfqpoint{4.805243in}{6.575902in}}{\pgfqpoint{4.800853in}{6.565303in}}{\pgfqpoint{4.800853in}{6.554253in}}%
\pgfpathcurveto{\pgfqpoint{4.800853in}{6.543203in}}{\pgfqpoint{4.805243in}{6.532604in}}{\pgfqpoint{4.813057in}{6.524790in}}%
\pgfpathcurveto{\pgfqpoint{4.820870in}{6.516977in}}{\pgfqpoint{4.831469in}{6.512587in}}{\pgfqpoint{4.842519in}{6.512587in}}%
\pgfpathclose%
\pgfusepath{stroke,fill}%
\end{pgfscope}%
\begin{pgfscope}%
\pgfpathrectangle{\pgfqpoint{0.970666in}{4.121437in}}{\pgfqpoint{5.699255in}{2.685432in}}%
\pgfusepath{clip}%
\pgfsetbuttcap%
\pgfsetroundjoin%
\definecolor{currentfill}{rgb}{0.000000,0.000000,0.000000}%
\pgfsetfillcolor{currentfill}%
\pgfsetlinewidth{1.003750pt}%
\definecolor{currentstroke}{rgb}{0.000000,0.000000,0.000000}%
\pgfsetstrokecolor{currentstroke}%
\pgfsetdash{}{0pt}%
\pgfpathmoveto{\pgfqpoint{1.845859in}{6.238430in}}%
\pgfpathcurveto{\pgfqpoint{1.856909in}{6.238430in}}{\pgfqpoint{1.867508in}{6.242820in}}{\pgfqpoint{1.875322in}{6.250634in}}%
\pgfpathcurveto{\pgfqpoint{1.883136in}{6.258447in}}{\pgfqpoint{1.887526in}{6.269046in}}{\pgfqpoint{1.887526in}{6.280096in}}%
\pgfpathcurveto{\pgfqpoint{1.887526in}{6.291146in}}{\pgfqpoint{1.883136in}{6.301745in}}{\pgfqpoint{1.875322in}{6.309559in}}%
\pgfpathcurveto{\pgfqpoint{1.867508in}{6.317373in}}{\pgfqpoint{1.856909in}{6.321763in}}{\pgfqpoint{1.845859in}{6.321763in}}%
\pgfpathcurveto{\pgfqpoint{1.834809in}{6.321763in}}{\pgfqpoint{1.824210in}{6.317373in}}{\pgfqpoint{1.816396in}{6.309559in}}%
\pgfpathcurveto{\pgfqpoint{1.808583in}{6.301745in}}{\pgfqpoint{1.804193in}{6.291146in}}{\pgfqpoint{1.804193in}{6.280096in}}%
\pgfpathcurveto{\pgfqpoint{1.804193in}{6.269046in}}{\pgfqpoint{1.808583in}{6.258447in}}{\pgfqpoint{1.816396in}{6.250634in}}%
\pgfpathcurveto{\pgfqpoint{1.824210in}{6.242820in}}{\pgfqpoint{1.834809in}{6.238430in}}{\pgfqpoint{1.845859in}{6.238430in}}%
\pgfpathclose%
\pgfusepath{stroke,fill}%
\end{pgfscope}%
\begin{pgfscope}%
\pgfpathrectangle{\pgfqpoint{0.970666in}{4.121437in}}{\pgfqpoint{5.699255in}{2.685432in}}%
\pgfusepath{clip}%
\pgfsetbuttcap%
\pgfsetroundjoin%
\definecolor{currentfill}{rgb}{0.000000,0.000000,0.000000}%
\pgfsetfillcolor{currentfill}%
\pgfsetlinewidth{1.003750pt}%
\definecolor{currentstroke}{rgb}{0.000000,0.000000,0.000000}%
\pgfsetstrokecolor{currentstroke}%
\pgfsetdash{}{0pt}%
\pgfpathmoveto{\pgfqpoint{2.686044in}{5.598730in}}%
\pgfpathcurveto{\pgfqpoint{2.697094in}{5.598730in}}{\pgfqpoint{2.707694in}{5.603120in}}{\pgfqpoint{2.715507in}{5.610934in}}%
\pgfpathcurveto{\pgfqpoint{2.723321in}{5.618748in}}{\pgfqpoint{2.727711in}{5.629347in}}{\pgfqpoint{2.727711in}{5.640397in}}%
\pgfpathcurveto{\pgfqpoint{2.727711in}{5.651447in}}{\pgfqpoint{2.723321in}{5.662046in}}{\pgfqpoint{2.715507in}{5.669860in}}%
\pgfpathcurveto{\pgfqpoint{2.707694in}{5.677673in}}{\pgfqpoint{2.697094in}{5.682064in}}{\pgfqpoint{2.686044in}{5.682064in}}%
\pgfpathcurveto{\pgfqpoint{2.674994in}{5.682064in}}{\pgfqpoint{2.664395in}{5.677673in}}{\pgfqpoint{2.656582in}{5.669860in}}%
\pgfpathcurveto{\pgfqpoint{2.648768in}{5.662046in}}{\pgfqpoint{2.644378in}{5.651447in}}{\pgfqpoint{2.644378in}{5.640397in}}%
\pgfpathcurveto{\pgfqpoint{2.644378in}{5.629347in}}{\pgfqpoint{2.648768in}{5.618748in}}{\pgfqpoint{2.656582in}{5.610934in}}%
\pgfpathcurveto{\pgfqpoint{2.664395in}{5.603120in}}{\pgfqpoint{2.674994in}{5.598730in}}{\pgfqpoint{2.686044in}{5.598730in}}%
\pgfpathclose%
\pgfusepath{stroke,fill}%
\end{pgfscope}%
\begin{pgfscope}%
\pgfpathrectangle{\pgfqpoint{0.970666in}{4.121437in}}{\pgfqpoint{5.699255in}{2.685432in}}%
\pgfusepath{clip}%
\pgfsetbuttcap%
\pgfsetroundjoin%
\definecolor{currentfill}{rgb}{0.000000,0.000000,0.000000}%
\pgfsetfillcolor{currentfill}%
\pgfsetlinewidth{1.003750pt}%
\definecolor{currentstroke}{rgb}{0.000000,0.000000,0.000000}%
\pgfsetstrokecolor{currentstroke}%
\pgfsetdash{}{0pt}%
\pgfpathmoveto{\pgfqpoint{6.102797in}{6.368981in}}%
\pgfpathcurveto{\pgfqpoint{6.113847in}{6.368981in}}{\pgfqpoint{6.124446in}{6.373371in}}{\pgfqpoint{6.132260in}{6.381184in}}%
\pgfpathcurveto{\pgfqpoint{6.140073in}{6.388998in}}{\pgfqpoint{6.144464in}{6.399597in}}{\pgfqpoint{6.144464in}{6.410647in}}%
\pgfpathcurveto{\pgfqpoint{6.144464in}{6.421697in}}{\pgfqpoint{6.140073in}{6.432296in}}{\pgfqpoint{6.132260in}{6.440110in}}%
\pgfpathcurveto{\pgfqpoint{6.124446in}{6.447924in}}{\pgfqpoint{6.113847in}{6.452314in}}{\pgfqpoint{6.102797in}{6.452314in}}%
\pgfpathcurveto{\pgfqpoint{6.091747in}{6.452314in}}{\pgfqpoint{6.081148in}{6.447924in}}{\pgfqpoint{6.073334in}{6.440110in}}%
\pgfpathcurveto{\pgfqpoint{6.065521in}{6.432296in}}{\pgfqpoint{6.061130in}{6.421697in}}{\pgfqpoint{6.061130in}{6.410647in}}%
\pgfpathcurveto{\pgfqpoint{6.061130in}{6.399597in}}{\pgfqpoint{6.065521in}{6.388998in}}{\pgfqpoint{6.073334in}{6.381184in}}%
\pgfpathcurveto{\pgfqpoint{6.081148in}{6.373371in}}{\pgfqpoint{6.091747in}{6.368981in}}{\pgfqpoint{6.102797in}{6.368981in}}%
\pgfpathclose%
\pgfusepath{stroke,fill}%
\end{pgfscope}%
\begin{pgfscope}%
\pgfpathrectangle{\pgfqpoint{0.970666in}{4.121437in}}{\pgfqpoint{5.699255in}{2.685432in}}%
\pgfusepath{clip}%
\pgfsetbuttcap%
\pgfsetroundjoin%
\definecolor{currentfill}{rgb}{0.000000,0.000000,0.000000}%
\pgfsetfillcolor{currentfill}%
\pgfsetlinewidth{1.003750pt}%
\definecolor{currentstroke}{rgb}{0.000000,0.000000,0.000000}%
\pgfsetstrokecolor{currentstroke}%
\pgfsetdash{}{0pt}%
\pgfpathmoveto{\pgfqpoint{5.570680in}{6.016493in}}%
\pgfpathcurveto{\pgfqpoint{5.581730in}{6.016493in}}{\pgfqpoint{5.592329in}{6.020883in}}{\pgfqpoint{5.600143in}{6.028697in}}%
\pgfpathcurveto{\pgfqpoint{5.607956in}{6.036511in}}{\pgfqpoint{5.612346in}{6.047110in}}{\pgfqpoint{5.612346in}{6.058160in}}%
\pgfpathcurveto{\pgfqpoint{5.612346in}{6.069210in}}{\pgfqpoint{5.607956in}{6.079809in}}{\pgfqpoint{5.600143in}{6.087623in}}%
\pgfpathcurveto{\pgfqpoint{5.592329in}{6.095436in}}{\pgfqpoint{5.581730in}{6.099826in}}{\pgfqpoint{5.570680in}{6.099826in}}%
\pgfpathcurveto{\pgfqpoint{5.559630in}{6.099826in}}{\pgfqpoint{5.549031in}{6.095436in}}{\pgfqpoint{5.541217in}{6.087623in}}%
\pgfpathcurveto{\pgfqpoint{5.533403in}{6.079809in}}{\pgfqpoint{5.529013in}{6.069210in}}{\pgfqpoint{5.529013in}{6.058160in}}%
\pgfpathcurveto{\pgfqpoint{5.529013in}{6.047110in}}{\pgfqpoint{5.533403in}{6.036511in}}{\pgfqpoint{5.541217in}{6.028697in}}%
\pgfpathcurveto{\pgfqpoint{5.549031in}{6.020883in}}{\pgfqpoint{5.559630in}{6.016493in}}{\pgfqpoint{5.570680in}{6.016493in}}%
\pgfpathclose%
\pgfusepath{stroke,fill}%
\end{pgfscope}%
\begin{pgfscope}%
\pgfpathrectangle{\pgfqpoint{0.970666in}{4.121437in}}{\pgfqpoint{5.699255in}{2.685432in}}%
\pgfusepath{clip}%
\pgfsetbuttcap%
\pgfsetroundjoin%
\definecolor{currentfill}{rgb}{0.000000,0.000000,0.000000}%
\pgfsetfillcolor{currentfill}%
\pgfsetlinewidth{1.003750pt}%
\definecolor{currentstroke}{rgb}{0.000000,0.000000,0.000000}%
\pgfsetstrokecolor{currentstroke}%
\pgfsetdash{}{0pt}%
\pgfpathmoveto{\pgfqpoint{4.226384in}{5.872887in}}%
\pgfpathcurveto{\pgfqpoint{4.237434in}{5.872887in}}{\pgfqpoint{4.248033in}{5.877277in}}{\pgfqpoint{4.255846in}{5.885091in}}%
\pgfpathcurveto{\pgfqpoint{4.263660in}{5.892905in}}{\pgfqpoint{4.268050in}{5.903504in}}{\pgfqpoint{4.268050in}{5.914554in}}%
\pgfpathcurveto{\pgfqpoint{4.268050in}{5.925604in}}{\pgfqpoint{4.263660in}{5.936203in}}{\pgfqpoint{4.255846in}{5.944017in}}%
\pgfpathcurveto{\pgfqpoint{4.248033in}{5.951830in}}{\pgfqpoint{4.237434in}{5.956220in}}{\pgfqpoint{4.226384in}{5.956220in}}%
\pgfpathcurveto{\pgfqpoint{4.215334in}{5.956220in}}{\pgfqpoint{4.204734in}{5.951830in}}{\pgfqpoint{4.196921in}{5.944017in}}%
\pgfpathcurveto{\pgfqpoint{4.189107in}{5.936203in}}{\pgfqpoint{4.184717in}{5.925604in}}{\pgfqpoint{4.184717in}{5.914554in}}%
\pgfpathcurveto{\pgfqpoint{4.184717in}{5.903504in}}{\pgfqpoint{4.189107in}{5.892905in}}{\pgfqpoint{4.196921in}{5.885091in}}%
\pgfpathcurveto{\pgfqpoint{4.204734in}{5.877277in}}{\pgfqpoint{4.215334in}{5.872887in}}{\pgfqpoint{4.226384in}{5.872887in}}%
\pgfpathclose%
\pgfusepath{stroke,fill}%
\end{pgfscope}%
\begin{pgfscope}%
\pgfpathrectangle{\pgfqpoint{0.970666in}{4.121437in}}{\pgfqpoint{5.699255in}{2.685432in}}%
\pgfusepath{clip}%
\pgfsetbuttcap%
\pgfsetroundjoin%
\definecolor{currentfill}{rgb}{0.000000,0.000000,0.000000}%
\pgfsetfillcolor{currentfill}%
\pgfsetlinewidth{1.003750pt}%
\definecolor{currentstroke}{rgb}{0.000000,0.000000,0.000000}%
\pgfsetstrokecolor{currentstroke}%
\pgfsetdash{}{0pt}%
\pgfpathmoveto{\pgfqpoint{1.509785in}{4.867645in}}%
\pgfpathcurveto{\pgfqpoint{1.520835in}{4.867645in}}{\pgfqpoint{1.531434in}{4.872035in}}{\pgfqpoint{1.539248in}{4.879849in}}%
\pgfpathcurveto{\pgfqpoint{1.547062in}{4.887663in}}{\pgfqpoint{1.551452in}{4.898262in}}{\pgfqpoint{1.551452in}{4.909312in}}%
\pgfpathcurveto{\pgfqpoint{1.551452in}{4.920362in}}{\pgfqpoint{1.547062in}{4.930961in}}{\pgfqpoint{1.539248in}{4.938775in}}%
\pgfpathcurveto{\pgfqpoint{1.531434in}{4.946588in}}{\pgfqpoint{1.520835in}{4.950978in}}{\pgfqpoint{1.509785in}{4.950978in}}%
\pgfpathcurveto{\pgfqpoint{1.498735in}{4.950978in}}{\pgfqpoint{1.488136in}{4.946588in}}{\pgfqpoint{1.480322in}{4.938775in}}%
\pgfpathcurveto{\pgfqpoint{1.472509in}{4.930961in}}{\pgfqpoint{1.468119in}{4.920362in}}{\pgfqpoint{1.468119in}{4.909312in}}%
\pgfpathcurveto{\pgfqpoint{1.468119in}{4.898262in}}{\pgfqpoint{1.472509in}{4.887663in}}{\pgfqpoint{1.480322in}{4.879849in}}%
\pgfpathcurveto{\pgfqpoint{1.488136in}{4.872035in}}{\pgfqpoint{1.498735in}{4.867645in}}{\pgfqpoint{1.509785in}{4.867645in}}%
\pgfpathclose%
\pgfusepath{stroke,fill}%
\end{pgfscope}%
\begin{pgfscope}%
\pgfpathrectangle{\pgfqpoint{0.970666in}{4.121437in}}{\pgfqpoint{5.699255in}{2.685432in}}%
\pgfusepath{clip}%
\pgfsetbuttcap%
\pgfsetroundjoin%
\definecolor{currentfill}{rgb}{0.000000,0.000000,0.000000}%
\pgfsetfillcolor{currentfill}%
\pgfsetlinewidth{1.003750pt}%
\definecolor{currentstroke}{rgb}{0.000000,0.000000,0.000000}%
\pgfsetstrokecolor{currentstroke}%
\pgfsetdash{}{0pt}%
\pgfpathmoveto{\pgfqpoint{4.030340in}{4.345442in}}%
\pgfpathcurveto{\pgfqpoint{4.041391in}{4.345442in}}{\pgfqpoint{4.051990in}{4.349832in}}{\pgfqpoint{4.059803in}{4.357645in}}%
\pgfpathcurveto{\pgfqpoint{4.067617in}{4.365459in}}{\pgfqpoint{4.072007in}{4.376058in}}{\pgfqpoint{4.072007in}{4.387108in}}%
\pgfpathcurveto{\pgfqpoint{4.072007in}{4.398158in}}{\pgfqpoint{4.067617in}{4.408757in}}{\pgfqpoint{4.059803in}{4.416571in}}%
\pgfpathcurveto{\pgfqpoint{4.051990in}{4.424385in}}{\pgfqpoint{4.041391in}{4.428775in}}{\pgfqpoint{4.030340in}{4.428775in}}%
\pgfpathcurveto{\pgfqpoint{4.019290in}{4.428775in}}{\pgfqpoint{4.008691in}{4.424385in}}{\pgfqpoint{4.000878in}{4.416571in}}%
\pgfpathcurveto{\pgfqpoint{3.993064in}{4.408757in}}{\pgfqpoint{3.988674in}{4.398158in}}{\pgfqpoint{3.988674in}{4.387108in}}%
\pgfpathcurveto{\pgfqpoint{3.988674in}{4.376058in}}{\pgfqpoint{3.993064in}{4.365459in}}{\pgfqpoint{4.000878in}{4.357645in}}%
\pgfpathcurveto{\pgfqpoint{4.008691in}{4.349832in}}{\pgfqpoint{4.019290in}{4.345442in}}{\pgfqpoint{4.030340in}{4.345442in}}%
\pgfpathclose%
\pgfusepath{stroke,fill}%
\end{pgfscope}%
\begin{pgfscope}%
\pgfpathrectangle{\pgfqpoint{0.970666in}{4.121437in}}{\pgfqpoint{5.699255in}{2.685432in}}%
\pgfusepath{clip}%
\pgfsetbuttcap%
\pgfsetroundjoin%
\definecolor{currentfill}{rgb}{0.000000,0.000000,0.000000}%
\pgfsetfillcolor{currentfill}%
\pgfsetlinewidth{1.003750pt}%
\definecolor{currentstroke}{rgb}{0.000000,0.000000,0.000000}%
\pgfsetstrokecolor{currentstroke}%
\pgfsetdash{}{0pt}%
\pgfpathmoveto{\pgfqpoint{4.982550in}{6.460366in}}%
\pgfpathcurveto{\pgfqpoint{4.993600in}{6.460366in}}{\pgfqpoint{5.004199in}{6.464756in}}{\pgfqpoint{5.012013in}{6.472570in}}%
\pgfpathcurveto{\pgfqpoint{5.019827in}{6.480384in}}{\pgfqpoint{5.024217in}{6.490983in}}{\pgfqpoint{5.024217in}{6.502033in}}%
\pgfpathcurveto{\pgfqpoint{5.024217in}{6.513083in}}{\pgfqpoint{5.019827in}{6.523682in}}{\pgfqpoint{5.012013in}{6.531496in}}%
\pgfpathcurveto{\pgfqpoint{5.004199in}{6.539309in}}{\pgfqpoint{4.993600in}{6.543700in}}{\pgfqpoint{4.982550in}{6.543700in}}%
\pgfpathcurveto{\pgfqpoint{4.971500in}{6.543700in}}{\pgfqpoint{4.960901in}{6.539309in}}{\pgfqpoint{4.953087in}{6.531496in}}%
\pgfpathcurveto{\pgfqpoint{4.945274in}{6.523682in}}{\pgfqpoint{4.940884in}{6.513083in}}{\pgfqpoint{4.940884in}{6.502033in}}%
\pgfpathcurveto{\pgfqpoint{4.940884in}{6.490983in}}{\pgfqpoint{4.945274in}{6.480384in}}{\pgfqpoint{4.953087in}{6.472570in}}%
\pgfpathcurveto{\pgfqpoint{4.960901in}{6.464756in}}{\pgfqpoint{4.971500in}{6.460366in}}{\pgfqpoint{4.982550in}{6.460366in}}%
\pgfpathclose%
\pgfusepath{stroke,fill}%
\end{pgfscope}%
\begin{pgfscope}%
\pgfpathrectangle{\pgfqpoint{0.970666in}{4.121437in}}{\pgfqpoint{5.699255in}{2.685432in}}%
\pgfusepath{clip}%
\pgfsetbuttcap%
\pgfsetroundjoin%
\definecolor{currentfill}{rgb}{0.000000,0.000000,0.000000}%
\pgfsetfillcolor{currentfill}%
\pgfsetlinewidth{1.003750pt}%
\definecolor{currentstroke}{rgb}{0.000000,0.000000,0.000000}%
\pgfsetstrokecolor{currentstroke}%
\pgfsetdash{}{0pt}%
\pgfpathmoveto{\pgfqpoint{5.906754in}{6.316760in}}%
\pgfpathcurveto{\pgfqpoint{5.917804in}{6.316760in}}{\pgfqpoint{5.928403in}{6.321150in}}{\pgfqpoint{5.936217in}{6.328964in}}%
\pgfpathcurveto{\pgfqpoint{5.944030in}{6.336778in}}{\pgfqpoint{5.948420in}{6.347377in}}{\pgfqpoint{5.948420in}{6.358427in}}%
\pgfpathcurveto{\pgfqpoint{5.948420in}{6.369477in}}{\pgfqpoint{5.944030in}{6.380076in}}{\pgfqpoint{5.936217in}{6.387890in}}%
\pgfpathcurveto{\pgfqpoint{5.928403in}{6.395703in}}{\pgfqpoint{5.917804in}{6.400094in}}{\pgfqpoint{5.906754in}{6.400094in}}%
\pgfpathcurveto{\pgfqpoint{5.895704in}{6.400094in}}{\pgfqpoint{5.885105in}{6.395703in}}{\pgfqpoint{5.877291in}{6.387890in}}%
\pgfpathcurveto{\pgfqpoint{5.869477in}{6.380076in}}{\pgfqpoint{5.865087in}{6.369477in}}{\pgfqpoint{5.865087in}{6.358427in}}%
\pgfpathcurveto{\pgfqpoint{5.865087in}{6.347377in}}{\pgfqpoint{5.869477in}{6.336778in}}{\pgfqpoint{5.877291in}{6.328964in}}%
\pgfpathcurveto{\pgfqpoint{5.885105in}{6.321150in}}{\pgfqpoint{5.895704in}{6.316760in}}{\pgfqpoint{5.906754in}{6.316760in}}%
\pgfpathclose%
\pgfusepath{stroke,fill}%
\end{pgfscope}%
\begin{pgfscope}%
\pgfpathrectangle{\pgfqpoint{0.970666in}{4.121437in}}{\pgfqpoint{5.699255in}{2.685432in}}%
\pgfusepath{clip}%
\pgfsetbuttcap%
\pgfsetroundjoin%
\definecolor{currentfill}{rgb}{0.000000,0.000000,0.000000}%
\pgfsetfillcolor{currentfill}%
\pgfsetlinewidth{1.003750pt}%
\definecolor{currentstroke}{rgb}{0.000000,0.000000,0.000000}%
\pgfsetstrokecolor{currentstroke}%
\pgfsetdash{}{0pt}%
\pgfpathmoveto{\pgfqpoint{5.234606in}{4.489048in}}%
\pgfpathcurveto{\pgfqpoint{5.245656in}{4.489048in}}{\pgfqpoint{5.256255in}{4.493438in}}{\pgfqpoint{5.264068in}{4.501251in}}%
\pgfpathcurveto{\pgfqpoint{5.271882in}{4.509065in}}{\pgfqpoint{5.276272in}{4.519664in}}{\pgfqpoint{5.276272in}{4.530714in}}%
\pgfpathcurveto{\pgfqpoint{5.276272in}{4.541764in}}{\pgfqpoint{5.271882in}{4.552363in}}{\pgfqpoint{5.264068in}{4.560177in}}%
\pgfpathcurveto{\pgfqpoint{5.256255in}{4.567991in}}{\pgfqpoint{5.245656in}{4.572381in}}{\pgfqpoint{5.234606in}{4.572381in}}%
\pgfpathcurveto{\pgfqpoint{5.223556in}{4.572381in}}{\pgfqpoint{5.212957in}{4.567991in}}{\pgfqpoint{5.205143in}{4.560177in}}%
\pgfpathcurveto{\pgfqpoint{5.197329in}{4.552363in}}{\pgfqpoint{5.192939in}{4.541764in}}{\pgfqpoint{5.192939in}{4.530714in}}%
\pgfpathcurveto{\pgfqpoint{5.192939in}{4.519664in}}{\pgfqpoint{5.197329in}{4.509065in}}{\pgfqpoint{5.205143in}{4.501251in}}%
\pgfpathcurveto{\pgfqpoint{5.212957in}{4.493438in}}{\pgfqpoint{5.223556in}{4.489048in}}{\pgfqpoint{5.234606in}{4.489048in}}%
\pgfpathclose%
\pgfusepath{stroke,fill}%
\end{pgfscope}%
\begin{pgfscope}%
\pgfpathrectangle{\pgfqpoint{0.970666in}{4.121437in}}{\pgfqpoint{5.699255in}{2.685432in}}%
\pgfusepath{clip}%
\pgfsetbuttcap%
\pgfsetroundjoin%
\definecolor{currentfill}{rgb}{0.000000,0.000000,0.000000}%
\pgfsetfillcolor{currentfill}%
\pgfsetlinewidth{1.003750pt}%
\definecolor{currentstroke}{rgb}{0.000000,0.000000,0.000000}%
\pgfsetstrokecolor{currentstroke}%
\pgfsetdash{}{0pt}%
\pgfpathmoveto{\pgfqpoint{2.377976in}{6.212319in}}%
\pgfpathcurveto{\pgfqpoint{2.389027in}{6.212319in}}{\pgfqpoint{2.399626in}{6.216710in}}{\pgfqpoint{2.407439in}{6.224523in}}%
\pgfpathcurveto{\pgfqpoint{2.415253in}{6.232337in}}{\pgfqpoint{2.419643in}{6.242936in}}{\pgfqpoint{2.419643in}{6.253986in}}%
\pgfpathcurveto{\pgfqpoint{2.419643in}{6.265036in}}{\pgfqpoint{2.415253in}{6.275635in}}{\pgfqpoint{2.407439in}{6.283449in}}%
\pgfpathcurveto{\pgfqpoint{2.399626in}{6.291263in}}{\pgfqpoint{2.389027in}{6.295653in}}{\pgfqpoint{2.377976in}{6.295653in}}%
\pgfpathcurveto{\pgfqpoint{2.366926in}{6.295653in}}{\pgfqpoint{2.356327in}{6.291263in}}{\pgfqpoint{2.348514in}{6.283449in}}%
\pgfpathcurveto{\pgfqpoint{2.340700in}{6.275635in}}{\pgfqpoint{2.336310in}{6.265036in}}{\pgfqpoint{2.336310in}{6.253986in}}%
\pgfpathcurveto{\pgfqpoint{2.336310in}{6.242936in}}{\pgfqpoint{2.340700in}{6.232337in}}{\pgfqpoint{2.348514in}{6.224523in}}%
\pgfpathcurveto{\pgfqpoint{2.356327in}{6.216710in}}{\pgfqpoint{2.366926in}{6.212319in}}{\pgfqpoint{2.377976in}{6.212319in}}%
\pgfpathclose%
\pgfusepath{stroke,fill}%
\end{pgfscope}%
\begin{pgfscope}%
\pgfpathrectangle{\pgfqpoint{0.970666in}{4.121437in}}{\pgfqpoint{5.699255in}{2.685432in}}%
\pgfusepath{clip}%
\pgfsetbuttcap%
\pgfsetroundjoin%
\definecolor{currentfill}{rgb}{0.000000,0.000000,0.000000}%
\pgfsetfillcolor{currentfill}%
\pgfsetlinewidth{1.003750pt}%
\definecolor{currentstroke}{rgb}{0.000000,0.000000,0.000000}%
\pgfsetstrokecolor{currentstroke}%
\pgfsetdash{}{0pt}%
\pgfpathmoveto{\pgfqpoint{1.369754in}{4.906810in}}%
\pgfpathcurveto{\pgfqpoint{1.380805in}{4.906810in}}{\pgfqpoint{1.391404in}{4.911201in}}{\pgfqpoint{1.399217in}{4.919014in}}%
\pgfpathcurveto{\pgfqpoint{1.407031in}{4.926828in}}{\pgfqpoint{1.411421in}{4.937427in}}{\pgfqpoint{1.411421in}{4.948477in}}%
\pgfpathcurveto{\pgfqpoint{1.411421in}{4.959527in}}{\pgfqpoint{1.407031in}{4.970126in}}{\pgfqpoint{1.399217in}{4.977940in}}%
\pgfpathcurveto{\pgfqpoint{1.391404in}{4.985753in}}{\pgfqpoint{1.380805in}{4.990144in}}{\pgfqpoint{1.369754in}{4.990144in}}%
\pgfpathcurveto{\pgfqpoint{1.358704in}{4.990144in}}{\pgfqpoint{1.348105in}{4.985753in}}{\pgfqpoint{1.340292in}{4.977940in}}%
\pgfpathcurveto{\pgfqpoint{1.332478in}{4.970126in}}{\pgfqpoint{1.328088in}{4.959527in}}{\pgfqpoint{1.328088in}{4.948477in}}%
\pgfpathcurveto{\pgfqpoint{1.328088in}{4.937427in}}{\pgfqpoint{1.332478in}{4.926828in}}{\pgfqpoint{1.340292in}{4.919014in}}%
\pgfpathcurveto{\pgfqpoint{1.348105in}{4.911201in}}{\pgfqpoint{1.358704in}{4.906810in}}{\pgfqpoint{1.369754in}{4.906810in}}%
\pgfpathclose%
\pgfusepath{stroke,fill}%
\end{pgfscope}%
\begin{pgfscope}%
\pgfpathrectangle{\pgfqpoint{0.970666in}{4.121437in}}{\pgfqpoint{5.699255in}{2.685432in}}%
\pgfusepath{clip}%
\pgfsetbuttcap%
\pgfsetroundjoin%
\definecolor{currentfill}{rgb}{0.000000,0.000000,0.000000}%
\pgfsetfillcolor{currentfill}%
\pgfsetlinewidth{1.003750pt}%
\definecolor{currentstroke}{rgb}{0.000000,0.000000,0.000000}%
\pgfsetstrokecolor{currentstroke}%
\pgfsetdash{}{0pt}%
\pgfpathmoveto{\pgfqpoint{2.405983in}{4.606543in}}%
\pgfpathcurveto{\pgfqpoint{2.417033in}{4.606543in}}{\pgfqpoint{2.427632in}{4.610934in}}{\pgfqpoint{2.435445in}{4.618747in}}%
\pgfpathcurveto{\pgfqpoint{2.443259in}{4.626561in}}{\pgfqpoint{2.447649in}{4.637160in}}{\pgfqpoint{2.447649in}{4.648210in}}%
\pgfpathcurveto{\pgfqpoint{2.447649in}{4.659260in}}{\pgfqpoint{2.443259in}{4.669859in}}{\pgfqpoint{2.435445in}{4.677673in}}%
\pgfpathcurveto{\pgfqpoint{2.427632in}{4.685486in}}{\pgfqpoint{2.417033in}{4.689877in}}{\pgfqpoint{2.405983in}{4.689877in}}%
\pgfpathcurveto{\pgfqpoint{2.394933in}{4.689877in}}{\pgfqpoint{2.384333in}{4.685486in}}{\pgfqpoint{2.376520in}{4.677673in}}%
\pgfpathcurveto{\pgfqpoint{2.368706in}{4.669859in}}{\pgfqpoint{2.364316in}{4.659260in}}{\pgfqpoint{2.364316in}{4.648210in}}%
\pgfpathcurveto{\pgfqpoint{2.364316in}{4.637160in}}{\pgfqpoint{2.368706in}{4.626561in}}{\pgfqpoint{2.376520in}{4.618747in}}%
\pgfpathcurveto{\pgfqpoint{2.384333in}{4.610934in}}{\pgfqpoint{2.394933in}{4.606543in}}{\pgfqpoint{2.405983in}{4.606543in}}%
\pgfpathclose%
\pgfusepath{stroke,fill}%
\end{pgfscope}%
\begin{pgfscope}%
\pgfpathrectangle{\pgfqpoint{0.970666in}{4.121437in}}{\pgfqpoint{5.699255in}{2.685432in}}%
\pgfusepath{clip}%
\pgfsetbuttcap%
\pgfsetroundjoin%
\definecolor{currentfill}{rgb}{0.000000,0.000000,0.000000}%
\pgfsetfillcolor{currentfill}%
\pgfsetlinewidth{1.003750pt}%
\definecolor{currentstroke}{rgb}{0.000000,0.000000,0.000000}%
\pgfsetstrokecolor{currentstroke}%
\pgfsetdash{}{0pt}%
\pgfpathmoveto{\pgfqpoint{4.310402in}{5.807612in}}%
\pgfpathcurveto{\pgfqpoint{4.321452in}{5.807612in}}{\pgfqpoint{4.332051in}{5.812002in}}{\pgfqpoint{4.339865in}{5.819816in}}%
\pgfpathcurveto{\pgfqpoint{4.347679in}{5.827629in}}{\pgfqpoint{4.352069in}{5.838228in}}{\pgfqpoint{4.352069in}{5.849278in}}%
\pgfpathcurveto{\pgfqpoint{4.352069in}{5.860328in}}{\pgfqpoint{4.347679in}{5.870927in}}{\pgfqpoint{4.339865in}{5.878741in}}%
\pgfpathcurveto{\pgfqpoint{4.332051in}{5.886555in}}{\pgfqpoint{4.321452in}{5.890945in}}{\pgfqpoint{4.310402in}{5.890945in}}%
\pgfpathcurveto{\pgfqpoint{4.299352in}{5.890945in}}{\pgfqpoint{4.288753in}{5.886555in}}{\pgfqpoint{4.280939in}{5.878741in}}%
\pgfpathcurveto{\pgfqpoint{4.273126in}{5.870927in}}{\pgfqpoint{4.268735in}{5.860328in}}{\pgfqpoint{4.268735in}{5.849278in}}%
\pgfpathcurveto{\pgfqpoint{4.268735in}{5.838228in}}{\pgfqpoint{4.273126in}{5.827629in}}{\pgfqpoint{4.280939in}{5.819816in}}%
\pgfpathcurveto{\pgfqpoint{4.288753in}{5.812002in}}{\pgfqpoint{4.299352in}{5.807612in}}{\pgfqpoint{4.310402in}{5.807612in}}%
\pgfpathclose%
\pgfusepath{stroke,fill}%
\end{pgfscope}%
\begin{pgfscope}%
\pgfpathrectangle{\pgfqpoint{0.970666in}{4.121437in}}{\pgfqpoint{5.699255in}{2.685432in}}%
\pgfusepath{clip}%
\pgfsetbuttcap%
\pgfsetroundjoin%
\definecolor{currentfill}{rgb}{0.000000,0.000000,0.000000}%
\pgfsetfillcolor{currentfill}%
\pgfsetlinewidth{1.003750pt}%
\definecolor{currentstroke}{rgb}{0.000000,0.000000,0.000000}%
\pgfsetstrokecolor{currentstroke}%
\pgfsetdash{}{0pt}%
\pgfpathmoveto{\pgfqpoint{3.526229in}{6.342870in}}%
\pgfpathcurveto{\pgfqpoint{3.537280in}{6.342870in}}{\pgfqpoint{3.547879in}{6.347261in}}{\pgfqpoint{3.555692in}{6.355074in}}%
\pgfpathcurveto{\pgfqpoint{3.563506in}{6.362888in}}{\pgfqpoint{3.567896in}{6.373487in}}{\pgfqpoint{3.567896in}{6.384537in}}%
\pgfpathcurveto{\pgfqpoint{3.567896in}{6.395587in}}{\pgfqpoint{3.563506in}{6.406186in}}{\pgfqpoint{3.555692in}{6.414000in}}%
\pgfpathcurveto{\pgfqpoint{3.547879in}{6.421813in}}{\pgfqpoint{3.537280in}{6.426204in}}{\pgfqpoint{3.526229in}{6.426204in}}%
\pgfpathcurveto{\pgfqpoint{3.515179in}{6.426204in}}{\pgfqpoint{3.504580in}{6.421813in}}{\pgfqpoint{3.496767in}{6.414000in}}%
\pgfpathcurveto{\pgfqpoint{3.488953in}{6.406186in}}{\pgfqpoint{3.484563in}{6.395587in}}{\pgfqpoint{3.484563in}{6.384537in}}%
\pgfpathcurveto{\pgfqpoint{3.484563in}{6.373487in}}{\pgfqpoint{3.488953in}{6.362888in}}{\pgfqpoint{3.496767in}{6.355074in}}%
\pgfpathcurveto{\pgfqpoint{3.504580in}{6.347261in}}{\pgfqpoint{3.515179in}{6.342870in}}{\pgfqpoint{3.526229in}{6.342870in}}%
\pgfpathclose%
\pgfusepath{stroke,fill}%
\end{pgfscope}%
\begin{pgfscope}%
\pgfpathrectangle{\pgfqpoint{0.970666in}{4.121437in}}{\pgfqpoint{5.699255in}{2.685432in}}%
\pgfusepath{clip}%
\pgfsetbuttcap%
\pgfsetroundjoin%
\definecolor{currentfill}{rgb}{0.000000,0.000000,0.000000}%
\pgfsetfillcolor{currentfill}%
\pgfsetlinewidth{1.003750pt}%
\definecolor{currentstroke}{rgb}{0.000000,0.000000,0.000000}%
\pgfsetstrokecolor{currentstroke}%
\pgfsetdash{}{0pt}%
\pgfpathmoveto{\pgfqpoint{1.873865in}{5.167912in}}%
\pgfpathcurveto{\pgfqpoint{1.884916in}{5.167912in}}{\pgfqpoint{1.895515in}{5.172302in}}{\pgfqpoint{1.903328in}{5.180116in}}%
\pgfpathcurveto{\pgfqpoint{1.911142in}{5.187930in}}{\pgfqpoint{1.915532in}{5.198529in}}{\pgfqpoint{1.915532in}{5.209579in}}%
\pgfpathcurveto{\pgfqpoint{1.915532in}{5.220629in}}{\pgfqpoint{1.911142in}{5.231228in}}{\pgfqpoint{1.903328in}{5.239042in}}%
\pgfpathcurveto{\pgfqpoint{1.895515in}{5.246855in}}{\pgfqpoint{1.884916in}{5.251246in}}{\pgfqpoint{1.873865in}{5.251246in}}%
\pgfpathcurveto{\pgfqpoint{1.862815in}{5.251246in}}{\pgfqpoint{1.852216in}{5.246855in}}{\pgfqpoint{1.844403in}{5.239042in}}%
\pgfpathcurveto{\pgfqpoint{1.836589in}{5.231228in}}{\pgfqpoint{1.832199in}{5.220629in}}{\pgfqpoint{1.832199in}{5.209579in}}%
\pgfpathcurveto{\pgfqpoint{1.832199in}{5.198529in}}{\pgfqpoint{1.836589in}{5.187930in}}{\pgfqpoint{1.844403in}{5.180116in}}%
\pgfpathcurveto{\pgfqpoint{1.852216in}{5.172302in}}{\pgfqpoint{1.862815in}{5.167912in}}{\pgfqpoint{1.873865in}{5.167912in}}%
\pgfpathclose%
\pgfusepath{stroke,fill}%
\end{pgfscope}%
\begin{pgfscope}%
\pgfpathrectangle{\pgfqpoint{0.970666in}{4.121437in}}{\pgfqpoint{5.699255in}{2.685432in}}%
\pgfusepath{clip}%
\pgfsetbuttcap%
\pgfsetroundjoin%
\definecolor{currentfill}{rgb}{0.000000,0.000000,0.000000}%
\pgfsetfillcolor{currentfill}%
\pgfsetlinewidth{1.003750pt}%
\definecolor{currentstroke}{rgb}{0.000000,0.000000,0.000000}%
\pgfsetstrokecolor{currentstroke}%
\pgfsetdash{}{0pt}%
\pgfpathmoveto{\pgfqpoint{6.214822in}{5.180967in}}%
\pgfpathcurveto{\pgfqpoint{6.225872in}{5.180967in}}{\pgfqpoint{6.236471in}{5.185358in}}{\pgfqpoint{6.244284in}{5.193171in}}%
\pgfpathcurveto{\pgfqpoint{6.252098in}{5.200985in}}{\pgfqpoint{6.256488in}{5.211584in}}{\pgfqpoint{6.256488in}{5.222634in}}%
\pgfpathcurveto{\pgfqpoint{6.256488in}{5.233684in}}{\pgfqpoint{6.252098in}{5.244283in}}{\pgfqpoint{6.244284in}{5.252097in}}%
\pgfpathcurveto{\pgfqpoint{6.236471in}{5.259910in}}{\pgfqpoint{6.225872in}{5.264301in}}{\pgfqpoint{6.214822in}{5.264301in}}%
\pgfpathcurveto{\pgfqpoint{6.203772in}{5.264301in}}{\pgfqpoint{6.193172in}{5.259910in}}{\pgfqpoint{6.185359in}{5.252097in}}%
\pgfpathcurveto{\pgfqpoint{6.177545in}{5.244283in}}{\pgfqpoint{6.173155in}{5.233684in}}{\pgfqpoint{6.173155in}{5.222634in}}%
\pgfpathcurveto{\pgfqpoint{6.173155in}{5.211584in}}{\pgfqpoint{6.177545in}{5.200985in}}{\pgfqpoint{6.185359in}{5.193171in}}%
\pgfpathcurveto{\pgfqpoint{6.193172in}{5.185358in}}{\pgfqpoint{6.203772in}{5.180967in}}{\pgfqpoint{6.214822in}{5.180967in}}%
\pgfpathclose%
\pgfusepath{stroke,fill}%
\end{pgfscope}%
\begin{pgfscope}%
\pgfpathrectangle{\pgfqpoint{0.970666in}{4.121437in}}{\pgfqpoint{5.699255in}{2.685432in}}%
\pgfusepath{clip}%
\pgfsetbuttcap%
\pgfsetroundjoin%
\definecolor{currentfill}{rgb}{0.000000,0.000000,0.000000}%
\pgfsetfillcolor{currentfill}%
\pgfsetlinewidth{1.003750pt}%
\definecolor{currentstroke}{rgb}{0.000000,0.000000,0.000000}%
\pgfsetstrokecolor{currentstroke}%
\pgfsetdash{}{0pt}%
\pgfpathmoveto{\pgfqpoint{5.318624in}{4.541268in}}%
\pgfpathcurveto{\pgfqpoint{5.329674in}{4.541268in}}{\pgfqpoint{5.340273in}{4.545658in}}{\pgfqpoint{5.348087in}{4.553472in}}%
\pgfpathcurveto{\pgfqpoint{5.355901in}{4.561285in}}{\pgfqpoint{5.360291in}{4.571884in}}{\pgfqpoint{5.360291in}{4.582935in}}%
\pgfpathcurveto{\pgfqpoint{5.360291in}{4.593985in}}{\pgfqpoint{5.355901in}{4.604584in}}{\pgfqpoint{5.348087in}{4.612397in}}%
\pgfpathcurveto{\pgfqpoint{5.340273in}{4.620211in}}{\pgfqpoint{5.329674in}{4.624601in}}{\pgfqpoint{5.318624in}{4.624601in}}%
\pgfpathcurveto{\pgfqpoint{5.307574in}{4.624601in}}{\pgfqpoint{5.296975in}{4.620211in}}{\pgfqpoint{5.289161in}{4.612397in}}%
\pgfpathcurveto{\pgfqpoint{5.281348in}{4.604584in}}{\pgfqpoint{5.276958in}{4.593985in}}{\pgfqpoint{5.276958in}{4.582935in}}%
\pgfpathcurveto{\pgfqpoint{5.276958in}{4.571884in}}{\pgfqpoint{5.281348in}{4.561285in}}{\pgfqpoint{5.289161in}{4.553472in}}%
\pgfpathcurveto{\pgfqpoint{5.296975in}{4.545658in}}{\pgfqpoint{5.307574in}{4.541268in}}{\pgfqpoint{5.318624in}{4.541268in}}%
\pgfpathclose%
\pgfusepath{stroke,fill}%
\end{pgfscope}%
\begin{pgfscope}%
\pgfpathrectangle{\pgfqpoint{0.970666in}{4.121437in}}{\pgfqpoint{5.699255in}{2.685432in}}%
\pgfusepath{clip}%
\pgfsetbuttcap%
\pgfsetroundjoin%
\definecolor{currentfill}{rgb}{0.000000,0.000000,0.000000}%
\pgfsetfillcolor{currentfill}%
\pgfsetlinewidth{1.003750pt}%
\definecolor{currentstroke}{rgb}{0.000000,0.000000,0.000000}%
\pgfsetstrokecolor{currentstroke}%
\pgfsetdash{}{0pt}%
\pgfpathmoveto{\pgfqpoint{4.898532in}{4.436827in}}%
\pgfpathcurveto{\pgfqpoint{4.909582in}{4.436827in}}{\pgfqpoint{4.920181in}{4.441217in}}{\pgfqpoint{4.927994in}{4.449031in}}%
\pgfpathcurveto{\pgfqpoint{4.935808in}{4.456845in}}{\pgfqpoint{4.940198in}{4.467444in}}{\pgfqpoint{4.940198in}{4.478494in}}%
\pgfpathcurveto{\pgfqpoint{4.940198in}{4.489544in}}{\pgfqpoint{4.935808in}{4.500143in}}{\pgfqpoint{4.927994in}{4.507957in}}%
\pgfpathcurveto{\pgfqpoint{4.920181in}{4.515770in}}{\pgfqpoint{4.909582in}{4.520161in}}{\pgfqpoint{4.898532in}{4.520161in}}%
\pgfpathcurveto{\pgfqpoint{4.887482in}{4.520161in}}{\pgfqpoint{4.876883in}{4.515770in}}{\pgfqpoint{4.869069in}{4.507957in}}%
\pgfpathcurveto{\pgfqpoint{4.861255in}{4.500143in}}{\pgfqpoint{4.856865in}{4.489544in}}{\pgfqpoint{4.856865in}{4.478494in}}%
\pgfpathcurveto{\pgfqpoint{4.856865in}{4.467444in}}{\pgfqpoint{4.861255in}{4.456845in}}{\pgfqpoint{4.869069in}{4.449031in}}%
\pgfpathcurveto{\pgfqpoint{4.876883in}{4.441217in}}{\pgfqpoint{4.887482in}{4.436827in}}{\pgfqpoint{4.898532in}{4.436827in}}%
\pgfpathclose%
\pgfusepath{stroke,fill}%
\end{pgfscope}%
\begin{pgfscope}%
\pgfpathrectangle{\pgfqpoint{0.970666in}{4.121437in}}{\pgfqpoint{5.699255in}{2.685432in}}%
\pgfusepath{clip}%
\pgfsetbuttcap%
\pgfsetroundjoin%
\definecolor{currentfill}{rgb}{0.000000,0.000000,0.000000}%
\pgfsetfillcolor{currentfill}%
\pgfsetlinewidth{1.003750pt}%
\definecolor{currentstroke}{rgb}{0.000000,0.000000,0.000000}%
\pgfsetstrokecolor{currentstroke}%
\pgfsetdash{}{0pt}%
\pgfpathmoveto{\pgfqpoint{2.938100in}{5.363739in}}%
\pgfpathcurveto{\pgfqpoint{2.949150in}{5.363739in}}{\pgfqpoint{2.959749in}{5.368129in}}{\pgfqpoint{2.967563in}{5.375942in}}%
\pgfpathcurveto{\pgfqpoint{2.975376in}{5.383756in}}{\pgfqpoint{2.979767in}{5.394355in}}{\pgfqpoint{2.979767in}{5.405405in}}%
\pgfpathcurveto{\pgfqpoint{2.979767in}{5.416455in}}{\pgfqpoint{2.975376in}{5.427054in}}{\pgfqpoint{2.967563in}{5.434868in}}%
\pgfpathcurveto{\pgfqpoint{2.959749in}{5.442682in}}{\pgfqpoint{2.949150in}{5.447072in}}{\pgfqpoint{2.938100in}{5.447072in}}%
\pgfpathcurveto{\pgfqpoint{2.927050in}{5.447072in}}{\pgfqpoint{2.916451in}{5.442682in}}{\pgfqpoint{2.908637in}{5.434868in}}%
\pgfpathcurveto{\pgfqpoint{2.900823in}{5.427054in}}{\pgfqpoint{2.896433in}{5.416455in}}{\pgfqpoint{2.896433in}{5.405405in}}%
\pgfpathcurveto{\pgfqpoint{2.896433in}{5.394355in}}{\pgfqpoint{2.900823in}{5.383756in}}{\pgfqpoint{2.908637in}{5.375942in}}%
\pgfpathcurveto{\pgfqpoint{2.916451in}{5.368129in}}{\pgfqpoint{2.927050in}{5.363739in}}{\pgfqpoint{2.938100in}{5.363739in}}%
\pgfpathclose%
\pgfusepath{stroke,fill}%
\end{pgfscope}%
\begin{pgfscope}%
\pgfpathrectangle{\pgfqpoint{0.970666in}{4.121437in}}{\pgfqpoint{5.699255in}{2.685432in}}%
\pgfusepath{clip}%
\pgfsetbuttcap%
\pgfsetroundjoin%
\definecolor{currentfill}{rgb}{0.000000,0.000000,0.000000}%
\pgfsetfillcolor{currentfill}%
\pgfsetlinewidth{1.003750pt}%
\definecolor{currentstroke}{rgb}{0.000000,0.000000,0.000000}%
\pgfsetstrokecolor{currentstroke}%
\pgfsetdash{}{0pt}%
\pgfpathmoveto{\pgfqpoint{2.602026in}{6.264540in}}%
\pgfpathcurveto{\pgfqpoint{2.613076in}{6.264540in}}{\pgfqpoint{2.623675in}{6.268930in}}{\pgfqpoint{2.631489in}{6.276744in}}%
\pgfpathcurveto{\pgfqpoint{2.639302in}{6.284557in}}{\pgfqpoint{2.643692in}{6.295156in}}{\pgfqpoint{2.643692in}{6.306206in}}%
\pgfpathcurveto{\pgfqpoint{2.643692in}{6.317257in}}{\pgfqpoint{2.639302in}{6.327856in}}{\pgfqpoint{2.631489in}{6.335669in}}%
\pgfpathcurveto{\pgfqpoint{2.623675in}{6.343483in}}{\pgfqpoint{2.613076in}{6.347873in}}{\pgfqpoint{2.602026in}{6.347873in}}%
\pgfpathcurveto{\pgfqpoint{2.590976in}{6.347873in}}{\pgfqpoint{2.580377in}{6.343483in}}{\pgfqpoint{2.572563in}{6.335669in}}%
\pgfpathcurveto{\pgfqpoint{2.564749in}{6.327856in}}{\pgfqpoint{2.560359in}{6.317257in}}{\pgfqpoint{2.560359in}{6.306206in}}%
\pgfpathcurveto{\pgfqpoint{2.560359in}{6.295156in}}{\pgfqpoint{2.564749in}{6.284557in}}{\pgfqpoint{2.572563in}{6.276744in}}%
\pgfpathcurveto{\pgfqpoint{2.580377in}{6.268930in}}{\pgfqpoint{2.590976in}{6.264540in}}{\pgfqpoint{2.602026in}{6.264540in}}%
\pgfpathclose%
\pgfusepath{stroke,fill}%
\end{pgfscope}%
\begin{pgfscope}%
\pgfpathrectangle{\pgfqpoint{0.970666in}{4.121437in}}{\pgfqpoint{5.699255in}{2.685432in}}%
\pgfusepath{clip}%
\pgfsetbuttcap%
\pgfsetroundjoin%
\definecolor{currentfill}{rgb}{0.000000,0.000000,0.000000}%
\pgfsetfillcolor{currentfill}%
\pgfsetlinewidth{1.003750pt}%
\definecolor{currentstroke}{rgb}{0.000000,0.000000,0.000000}%
\pgfsetstrokecolor{currentstroke}%
\pgfsetdash{}{0pt}%
\pgfpathmoveto{\pgfqpoint{2.461995in}{6.630082in}}%
\pgfpathcurveto{\pgfqpoint{2.473045in}{6.630082in}}{\pgfqpoint{2.483644in}{6.634473in}}{\pgfqpoint{2.491458in}{6.642286in}}%
\pgfpathcurveto{\pgfqpoint{2.499271in}{6.650100in}}{\pgfqpoint{2.503662in}{6.660699in}}{\pgfqpoint{2.503662in}{6.671749in}}%
\pgfpathcurveto{\pgfqpoint{2.503662in}{6.682799in}}{\pgfqpoint{2.499271in}{6.693398in}}{\pgfqpoint{2.491458in}{6.701212in}}%
\pgfpathcurveto{\pgfqpoint{2.483644in}{6.709025in}}{\pgfqpoint{2.473045in}{6.713416in}}{\pgfqpoint{2.461995in}{6.713416in}}%
\pgfpathcurveto{\pgfqpoint{2.450945in}{6.713416in}}{\pgfqpoint{2.440346in}{6.709025in}}{\pgfqpoint{2.432532in}{6.701212in}}%
\pgfpathcurveto{\pgfqpoint{2.424719in}{6.693398in}}{\pgfqpoint{2.420328in}{6.682799in}}{\pgfqpoint{2.420328in}{6.671749in}}%
\pgfpathcurveto{\pgfqpoint{2.420328in}{6.660699in}}{\pgfqpoint{2.424719in}{6.650100in}}{\pgfqpoint{2.432532in}{6.642286in}}%
\pgfpathcurveto{\pgfqpoint{2.440346in}{6.634473in}}{\pgfqpoint{2.450945in}{6.630082in}}{\pgfqpoint{2.461995in}{6.630082in}}%
\pgfpathclose%
\pgfusepath{stroke,fill}%
\end{pgfscope}%
\begin{pgfscope}%
\pgfpathrectangle{\pgfqpoint{0.970666in}{4.121437in}}{\pgfqpoint{5.699255in}{2.685432in}}%
\pgfusepath{clip}%
\pgfsetbuttcap%
\pgfsetroundjoin%
\definecolor{currentfill}{rgb}{0.000000,0.000000,0.000000}%
\pgfsetfillcolor{currentfill}%
\pgfsetlinewidth{1.003750pt}%
\definecolor{currentstroke}{rgb}{0.000000,0.000000,0.000000}%
\pgfsetstrokecolor{currentstroke}%
\pgfsetdash{}{0pt}%
\pgfpathmoveto{\pgfqpoint{1.369754in}{6.368981in}}%
\pgfpathcurveto{\pgfqpoint{1.380805in}{6.368981in}}{\pgfqpoint{1.391404in}{6.373371in}}{\pgfqpoint{1.399217in}{6.381184in}}%
\pgfpathcurveto{\pgfqpoint{1.407031in}{6.388998in}}{\pgfqpoint{1.411421in}{6.399597in}}{\pgfqpoint{1.411421in}{6.410647in}}%
\pgfpathcurveto{\pgfqpoint{1.411421in}{6.421697in}}{\pgfqpoint{1.407031in}{6.432296in}}{\pgfqpoint{1.399217in}{6.440110in}}%
\pgfpathcurveto{\pgfqpoint{1.391404in}{6.447924in}}{\pgfqpoint{1.380805in}{6.452314in}}{\pgfqpoint{1.369754in}{6.452314in}}%
\pgfpathcurveto{\pgfqpoint{1.358704in}{6.452314in}}{\pgfqpoint{1.348105in}{6.447924in}}{\pgfqpoint{1.340292in}{6.440110in}}%
\pgfpathcurveto{\pgfqpoint{1.332478in}{6.432296in}}{\pgfqpoint{1.328088in}{6.421697in}}{\pgfqpoint{1.328088in}{6.410647in}}%
\pgfpathcurveto{\pgfqpoint{1.328088in}{6.399597in}}{\pgfqpoint{1.332478in}{6.388998in}}{\pgfqpoint{1.340292in}{6.381184in}}%
\pgfpathcurveto{\pgfqpoint{1.348105in}{6.373371in}}{\pgfqpoint{1.358704in}{6.368981in}}{\pgfqpoint{1.369754in}{6.368981in}}%
\pgfpathclose%
\pgfusepath{stroke,fill}%
\end{pgfscope}%
\begin{pgfscope}%
\pgfpathrectangle{\pgfqpoint{0.970666in}{4.121437in}}{\pgfqpoint{5.699255in}{2.685432in}}%
\pgfusepath{clip}%
\pgfsetbuttcap%
\pgfsetroundjoin%
\definecolor{currentfill}{rgb}{0.000000,0.000000,0.000000}%
\pgfsetfillcolor{currentfill}%
\pgfsetlinewidth{1.003750pt}%
\definecolor{currentstroke}{rgb}{0.000000,0.000000,0.000000}%
\pgfsetstrokecolor{currentstroke}%
\pgfsetdash{}{0pt}%
\pgfpathmoveto{\pgfqpoint{4.058347in}{5.063472in}}%
\pgfpathcurveto{\pgfqpoint{4.069397in}{5.063472in}}{\pgfqpoint{4.079996in}{5.067862in}}{\pgfqpoint{4.087809in}{5.075675in}}%
\pgfpathcurveto{\pgfqpoint{4.095623in}{5.083489in}}{\pgfqpoint{4.100013in}{5.094088in}}{\pgfqpoint{4.100013in}{5.105138in}}%
\pgfpathcurveto{\pgfqpoint{4.100013in}{5.116188in}}{\pgfqpoint{4.095623in}{5.126787in}}{\pgfqpoint{4.087809in}{5.134601in}}%
\pgfpathcurveto{\pgfqpoint{4.079996in}{5.142415in}}{\pgfqpoint{4.069397in}{5.146805in}}{\pgfqpoint{4.058347in}{5.146805in}}%
\pgfpathcurveto{\pgfqpoint{4.047296in}{5.146805in}}{\pgfqpoint{4.036697in}{5.142415in}}{\pgfqpoint{4.028884in}{5.134601in}}%
\pgfpathcurveto{\pgfqpoint{4.021070in}{5.126787in}}{\pgfqpoint{4.016680in}{5.116188in}}{\pgfqpoint{4.016680in}{5.105138in}}%
\pgfpathcurveto{\pgfqpoint{4.016680in}{5.094088in}}{\pgfqpoint{4.021070in}{5.083489in}}{\pgfqpoint{4.028884in}{5.075675in}}%
\pgfpathcurveto{\pgfqpoint{4.036697in}{5.067862in}}{\pgfqpoint{4.047296in}{5.063472in}}{\pgfqpoint{4.058347in}{5.063472in}}%
\pgfpathclose%
\pgfusepath{stroke,fill}%
\end{pgfscope}%
\begin{pgfscope}%
\pgfpathrectangle{\pgfqpoint{0.970666in}{4.121437in}}{\pgfqpoint{5.699255in}{2.685432in}}%
\pgfusepath{clip}%
\pgfsetbuttcap%
\pgfsetroundjoin%
\definecolor{currentfill}{rgb}{0.000000,0.000000,0.000000}%
\pgfsetfillcolor{currentfill}%
\pgfsetlinewidth{1.003750pt}%
\definecolor{currentstroke}{rgb}{0.000000,0.000000,0.000000}%
\pgfsetstrokecolor{currentstroke}%
\pgfsetdash{}{0pt}%
\pgfpathmoveto{\pgfqpoint{5.794729in}{5.520400in}}%
\pgfpathcurveto{\pgfqpoint{5.805779in}{5.520400in}}{\pgfqpoint{5.816378in}{5.524790in}}{\pgfqpoint{5.824192in}{5.532604in}}%
\pgfpathcurveto{\pgfqpoint{5.832005in}{5.540417in}}{\pgfqpoint{5.836396in}{5.551016in}}{\pgfqpoint{5.836396in}{5.562066in}}%
\pgfpathcurveto{\pgfqpoint{5.836396in}{5.573116in}}{\pgfqpoint{5.832005in}{5.583716in}}{\pgfqpoint{5.824192in}{5.591529in}}%
\pgfpathcurveto{\pgfqpoint{5.816378in}{5.599343in}}{\pgfqpoint{5.805779in}{5.603733in}}{\pgfqpoint{5.794729in}{5.603733in}}%
\pgfpathcurveto{\pgfqpoint{5.783679in}{5.603733in}}{\pgfqpoint{5.773080in}{5.599343in}}{\pgfqpoint{5.765266in}{5.591529in}}%
\pgfpathcurveto{\pgfqpoint{5.757453in}{5.583716in}}{\pgfqpoint{5.753062in}{5.573116in}}{\pgfqpoint{5.753062in}{5.562066in}}%
\pgfpathcurveto{\pgfqpoint{5.753062in}{5.551016in}}{\pgfqpoint{5.757453in}{5.540417in}}{\pgfqpoint{5.765266in}{5.532604in}}%
\pgfpathcurveto{\pgfqpoint{5.773080in}{5.524790in}}{\pgfqpoint{5.783679in}{5.520400in}}{\pgfqpoint{5.794729in}{5.520400in}}%
\pgfpathclose%
\pgfusepath{stroke,fill}%
\end{pgfscope}%
\begin{pgfscope}%
\pgfpathrectangle{\pgfqpoint{0.970666in}{4.121437in}}{\pgfqpoint{5.699255in}{2.685432in}}%
\pgfusepath{clip}%
\pgfsetbuttcap%
\pgfsetroundjoin%
\definecolor{currentfill}{rgb}{0.000000,0.000000,0.000000}%
\pgfsetfillcolor{currentfill}%
\pgfsetlinewidth{1.003750pt}%
\definecolor{currentstroke}{rgb}{0.000000,0.000000,0.000000}%
\pgfsetstrokecolor{currentstroke}%
\pgfsetdash{}{0pt}%
\pgfpathmoveto{\pgfqpoint{3.554236in}{6.408146in}}%
\pgfpathcurveto{\pgfqpoint{3.565286in}{6.408146in}}{\pgfqpoint{3.575885in}{6.412536in}}{\pgfqpoint{3.583698in}{6.420350in}}%
\pgfpathcurveto{\pgfqpoint{3.591512in}{6.428163in}}{\pgfqpoint{3.595902in}{6.438762in}}{\pgfqpoint{3.595902in}{6.449812in}}%
\pgfpathcurveto{\pgfqpoint{3.595902in}{6.460863in}}{\pgfqpoint{3.591512in}{6.471462in}}{\pgfqpoint{3.583698in}{6.479275in}}%
\pgfpathcurveto{\pgfqpoint{3.575885in}{6.487089in}}{\pgfqpoint{3.565286in}{6.491479in}}{\pgfqpoint{3.554236in}{6.491479in}}%
\pgfpathcurveto{\pgfqpoint{3.543185in}{6.491479in}}{\pgfqpoint{3.532586in}{6.487089in}}{\pgfqpoint{3.524773in}{6.479275in}}%
\pgfpathcurveto{\pgfqpoint{3.516959in}{6.471462in}}{\pgfqpoint{3.512569in}{6.460863in}}{\pgfqpoint{3.512569in}{6.449812in}}%
\pgfpathcurveto{\pgfqpoint{3.512569in}{6.438762in}}{\pgfqpoint{3.516959in}{6.428163in}}{\pgfqpoint{3.524773in}{6.420350in}}%
\pgfpathcurveto{\pgfqpoint{3.532586in}{6.412536in}}{\pgfqpoint{3.543185in}{6.408146in}}{\pgfqpoint{3.554236in}{6.408146in}}%
\pgfpathclose%
\pgfusepath{stroke,fill}%
\end{pgfscope}%
\begin{pgfscope}%
\pgfpathrectangle{\pgfqpoint{0.970666in}{4.121437in}}{\pgfqpoint{5.699255in}{2.685432in}}%
\pgfusepath{clip}%
\pgfsetbuttcap%
\pgfsetroundjoin%
\definecolor{currentfill}{rgb}{0.000000,0.000000,0.000000}%
\pgfsetfillcolor{currentfill}%
\pgfsetlinewidth{1.003750pt}%
\definecolor{currentstroke}{rgb}{0.000000,0.000000,0.000000}%
\pgfsetstrokecolor{currentstroke}%
\pgfsetdash{}{0pt}%
\pgfpathmoveto{\pgfqpoint{2.405983in}{5.546510in}}%
\pgfpathcurveto{\pgfqpoint{2.417033in}{5.546510in}}{\pgfqpoint{2.427632in}{5.550900in}}{\pgfqpoint{2.435445in}{5.558714in}}%
\pgfpathcurveto{\pgfqpoint{2.443259in}{5.566527in}}{\pgfqpoint{2.447649in}{5.577126in}}{\pgfqpoint{2.447649in}{5.588177in}}%
\pgfpathcurveto{\pgfqpoint{2.447649in}{5.599227in}}{\pgfqpoint{2.443259in}{5.609826in}}{\pgfqpoint{2.435445in}{5.617639in}}%
\pgfpathcurveto{\pgfqpoint{2.427632in}{5.625453in}}{\pgfqpoint{2.417033in}{5.629843in}}{\pgfqpoint{2.405983in}{5.629843in}}%
\pgfpathcurveto{\pgfqpoint{2.394933in}{5.629843in}}{\pgfqpoint{2.384333in}{5.625453in}}{\pgfqpoint{2.376520in}{5.617639in}}%
\pgfpathcurveto{\pgfqpoint{2.368706in}{5.609826in}}{\pgfqpoint{2.364316in}{5.599227in}}{\pgfqpoint{2.364316in}{5.588177in}}%
\pgfpathcurveto{\pgfqpoint{2.364316in}{5.577126in}}{\pgfqpoint{2.368706in}{5.566527in}}{\pgfqpoint{2.376520in}{5.558714in}}%
\pgfpathcurveto{\pgfqpoint{2.384333in}{5.550900in}}{\pgfqpoint{2.394933in}{5.546510in}}{\pgfqpoint{2.405983in}{5.546510in}}%
\pgfpathclose%
\pgfusepath{stroke,fill}%
\end{pgfscope}%
\begin{pgfscope}%
\pgfpathrectangle{\pgfqpoint{0.970666in}{4.121437in}}{\pgfqpoint{5.699255in}{2.685432in}}%
\pgfusepath{clip}%
\pgfsetbuttcap%
\pgfsetroundjoin%
\definecolor{currentfill}{rgb}{0.000000,0.000000,0.000000}%
\pgfsetfillcolor{currentfill}%
\pgfsetlinewidth{1.003750pt}%
\definecolor{currentstroke}{rgb}{0.000000,0.000000,0.000000}%
\pgfsetstrokecolor{currentstroke}%
\pgfsetdash{}{0pt}%
\pgfpathmoveto{\pgfqpoint{4.982550in}{5.024306in}}%
\pgfpathcurveto{\pgfqpoint{4.993600in}{5.024306in}}{\pgfqpoint{5.004199in}{5.028697in}}{\pgfqpoint{5.012013in}{5.036510in}}%
\pgfpathcurveto{\pgfqpoint{5.019827in}{5.044324in}}{\pgfqpoint{5.024217in}{5.054923in}}{\pgfqpoint{5.024217in}{5.065973in}}%
\pgfpathcurveto{\pgfqpoint{5.024217in}{5.077023in}}{\pgfqpoint{5.019827in}{5.087622in}}{\pgfqpoint{5.012013in}{5.095436in}}%
\pgfpathcurveto{\pgfqpoint{5.004199in}{5.103249in}}{\pgfqpoint{4.993600in}{5.107640in}}{\pgfqpoint{4.982550in}{5.107640in}}%
\pgfpathcurveto{\pgfqpoint{4.971500in}{5.107640in}}{\pgfqpoint{4.960901in}{5.103249in}}{\pgfqpoint{4.953087in}{5.095436in}}%
\pgfpathcurveto{\pgfqpoint{4.945274in}{5.087622in}}{\pgfqpoint{4.940884in}{5.077023in}}{\pgfqpoint{4.940884in}{5.065973in}}%
\pgfpathcurveto{\pgfqpoint{4.940884in}{5.054923in}}{\pgfqpoint{4.945274in}{5.044324in}}{\pgfqpoint{4.953087in}{5.036510in}}%
\pgfpathcurveto{\pgfqpoint{4.960901in}{5.028697in}}{\pgfqpoint{4.971500in}{5.024306in}}{\pgfqpoint{4.982550in}{5.024306in}}%
\pgfpathclose%
\pgfusepath{stroke,fill}%
\end{pgfscope}%
\begin{pgfscope}%
\pgfpathrectangle{\pgfqpoint{0.970666in}{4.121437in}}{\pgfqpoint{5.699255in}{2.685432in}}%
\pgfusepath{clip}%
\pgfsetbuttcap%
\pgfsetroundjoin%
\definecolor{currentfill}{rgb}{0.000000,0.000000,0.000000}%
\pgfsetfillcolor{currentfill}%
\pgfsetlinewidth{1.003750pt}%
\definecolor{currentstroke}{rgb}{0.000000,0.000000,0.000000}%
\pgfsetstrokecolor{currentstroke}%
\pgfsetdash{}{0pt}%
\pgfpathmoveto{\pgfqpoint{5.626692in}{4.593488in}}%
\pgfpathcurveto{\pgfqpoint{5.637742in}{4.593488in}}{\pgfqpoint{5.648341in}{4.597879in}}{\pgfqpoint{5.656155in}{4.605692in}}%
\pgfpathcurveto{\pgfqpoint{5.663968in}{4.613506in}}{\pgfqpoint{5.668359in}{4.624105in}}{\pgfqpoint{5.668359in}{4.635155in}}%
\pgfpathcurveto{\pgfqpoint{5.668359in}{4.646205in}}{\pgfqpoint{5.663968in}{4.656804in}}{\pgfqpoint{5.656155in}{4.664618in}}%
\pgfpathcurveto{\pgfqpoint{5.648341in}{4.672431in}}{\pgfqpoint{5.637742in}{4.676822in}}{\pgfqpoint{5.626692in}{4.676822in}}%
\pgfpathcurveto{\pgfqpoint{5.615642in}{4.676822in}}{\pgfqpoint{5.605043in}{4.672431in}}{\pgfqpoint{5.597229in}{4.664618in}}%
\pgfpathcurveto{\pgfqpoint{5.589416in}{4.656804in}}{\pgfqpoint{5.585025in}{4.646205in}}{\pgfqpoint{5.585025in}{4.635155in}}%
\pgfpathcurveto{\pgfqpoint{5.585025in}{4.624105in}}{\pgfqpoint{5.589416in}{4.613506in}}{\pgfqpoint{5.597229in}{4.605692in}}%
\pgfpathcurveto{\pgfqpoint{5.605043in}{4.597879in}}{\pgfqpoint{5.615642in}{4.593488in}}{\pgfqpoint{5.626692in}{4.593488in}}%
\pgfpathclose%
\pgfusepath{stroke,fill}%
\end{pgfscope}%
\begin{pgfscope}%
\pgfpathrectangle{\pgfqpoint{0.970666in}{4.121437in}}{\pgfqpoint{5.699255in}{2.685432in}}%
\pgfusepath{clip}%
\pgfsetbuttcap%
\pgfsetroundjoin%
\definecolor{currentfill}{rgb}{0.000000,0.000000,0.000000}%
\pgfsetfillcolor{currentfill}%
\pgfsetlinewidth{1.003750pt}%
\definecolor{currentstroke}{rgb}{0.000000,0.000000,0.000000}%
\pgfsetstrokecolor{currentstroke}%
\pgfsetdash{}{0pt}%
\pgfpathmoveto{\pgfqpoint{4.086353in}{4.397662in}}%
\pgfpathcurveto{\pgfqpoint{4.097403in}{4.397662in}}{\pgfqpoint{4.108002in}{4.402052in}}{\pgfqpoint{4.115816in}{4.409866in}}%
\pgfpathcurveto{\pgfqpoint{4.123629in}{4.417679in}}{\pgfqpoint{4.128019in}{4.428278in}}{\pgfqpoint{4.128019in}{4.439329in}}%
\pgfpathcurveto{\pgfqpoint{4.128019in}{4.450379in}}{\pgfqpoint{4.123629in}{4.460978in}}{\pgfqpoint{4.115816in}{4.468791in}}%
\pgfpathcurveto{\pgfqpoint{4.108002in}{4.476605in}}{\pgfqpoint{4.097403in}{4.480995in}}{\pgfqpoint{4.086353in}{4.480995in}}%
\pgfpathcurveto{\pgfqpoint{4.075303in}{4.480995in}}{\pgfqpoint{4.064704in}{4.476605in}}{\pgfqpoint{4.056890in}{4.468791in}}%
\pgfpathcurveto{\pgfqpoint{4.049076in}{4.460978in}}{\pgfqpoint{4.044686in}{4.450379in}}{\pgfqpoint{4.044686in}{4.439329in}}%
\pgfpathcurveto{\pgfqpoint{4.044686in}{4.428278in}}{\pgfqpoint{4.049076in}{4.417679in}}{\pgfqpoint{4.056890in}{4.409866in}}%
\pgfpathcurveto{\pgfqpoint{4.064704in}{4.402052in}}{\pgfqpoint{4.075303in}{4.397662in}}{\pgfqpoint{4.086353in}{4.397662in}}%
\pgfpathclose%
\pgfusepath{stroke,fill}%
\end{pgfscope}%
\begin{pgfscope}%
\pgfpathrectangle{\pgfqpoint{0.970666in}{4.121437in}}{\pgfqpoint{5.699255in}{2.685432in}}%
\pgfusepath{clip}%
\pgfsetbuttcap%
\pgfsetroundjoin%
\definecolor{currentfill}{rgb}{0.000000,0.000000,0.000000}%
\pgfsetfillcolor{currentfill}%
\pgfsetlinewidth{1.003750pt}%
\definecolor{currentstroke}{rgb}{0.000000,0.000000,0.000000}%
\pgfsetstrokecolor{currentstroke}%
\pgfsetdash{}{0pt}%
\pgfpathmoveto{\pgfqpoint{1.621810in}{4.697929in}}%
\pgfpathcurveto{\pgfqpoint{1.632860in}{4.697929in}}{\pgfqpoint{1.643459in}{4.702319in}}{\pgfqpoint{1.651273in}{4.710133in}}%
\pgfpathcurveto{\pgfqpoint{1.659086in}{4.717946in}}{\pgfqpoint{1.663477in}{4.728546in}}{\pgfqpoint{1.663477in}{4.739596in}}%
\pgfpathcurveto{\pgfqpoint{1.663477in}{4.750646in}}{\pgfqpoint{1.659086in}{4.761245in}}{\pgfqpoint{1.651273in}{4.769058in}}%
\pgfpathcurveto{\pgfqpoint{1.643459in}{4.776872in}}{\pgfqpoint{1.632860in}{4.781262in}}{\pgfqpoint{1.621810in}{4.781262in}}%
\pgfpathcurveto{\pgfqpoint{1.610760in}{4.781262in}}{\pgfqpoint{1.600161in}{4.776872in}}{\pgfqpoint{1.592347in}{4.769058in}}%
\pgfpathcurveto{\pgfqpoint{1.584534in}{4.761245in}}{\pgfqpoint{1.580143in}{4.750646in}}{\pgfqpoint{1.580143in}{4.739596in}}%
\pgfpathcurveto{\pgfqpoint{1.580143in}{4.728546in}}{\pgfqpoint{1.584534in}{4.717946in}}{\pgfqpoint{1.592347in}{4.710133in}}%
\pgfpathcurveto{\pgfqpoint{1.600161in}{4.702319in}}{\pgfqpoint{1.610760in}{4.697929in}}{\pgfqpoint{1.621810in}{4.697929in}}%
\pgfpathclose%
\pgfusepath{stroke,fill}%
\end{pgfscope}%
\begin{pgfscope}%
\pgfpathrectangle{\pgfqpoint{0.970666in}{4.121437in}}{\pgfqpoint{5.699255in}{2.685432in}}%
\pgfusepath{clip}%
\pgfsetbuttcap%
\pgfsetroundjoin%
\definecolor{currentfill}{rgb}{0.000000,0.000000,0.000000}%
\pgfsetfillcolor{currentfill}%
\pgfsetlinewidth{1.003750pt}%
\definecolor{currentstroke}{rgb}{0.000000,0.000000,0.000000}%
\pgfsetstrokecolor{currentstroke}%
\pgfsetdash{}{0pt}%
\pgfpathmoveto{\pgfqpoint{4.142365in}{5.076527in}}%
\pgfpathcurveto{\pgfqpoint{4.153415in}{5.076527in}}{\pgfqpoint{4.164014in}{5.080917in}}{\pgfqpoint{4.171828in}{5.088730in}}%
\pgfpathcurveto{\pgfqpoint{4.179642in}{5.096544in}}{\pgfqpoint{4.184032in}{5.107143in}}{\pgfqpoint{4.184032in}{5.118193in}}%
\pgfpathcurveto{\pgfqpoint{4.184032in}{5.129243in}}{\pgfqpoint{4.179642in}{5.139842in}}{\pgfqpoint{4.171828in}{5.147656in}}%
\pgfpathcurveto{\pgfqpoint{4.164014in}{5.155470in}}{\pgfqpoint{4.153415in}{5.159860in}}{\pgfqpoint{4.142365in}{5.159860in}}%
\pgfpathcurveto{\pgfqpoint{4.131315in}{5.159860in}}{\pgfqpoint{4.120716in}{5.155470in}}{\pgfqpoint{4.112902in}{5.147656in}}%
\pgfpathcurveto{\pgfqpoint{4.105089in}{5.139842in}}{\pgfqpoint{4.100698in}{5.129243in}}{\pgfqpoint{4.100698in}{5.118193in}}%
\pgfpathcurveto{\pgfqpoint{4.100698in}{5.107143in}}{\pgfqpoint{4.105089in}{5.096544in}}{\pgfqpoint{4.112902in}{5.088730in}}%
\pgfpathcurveto{\pgfqpoint{4.120716in}{5.080917in}}{\pgfqpoint{4.131315in}{5.076527in}}{\pgfqpoint{4.142365in}{5.076527in}}%
\pgfpathclose%
\pgfusepath{stroke,fill}%
\end{pgfscope}%
\begin{pgfscope}%
\pgfpathrectangle{\pgfqpoint{0.970666in}{4.121437in}}{\pgfqpoint{5.699255in}{2.685432in}}%
\pgfusepath{clip}%
\pgfsetbuttcap%
\pgfsetroundjoin%
\definecolor{currentfill}{rgb}{0.000000,0.000000,0.000000}%
\pgfsetfillcolor{currentfill}%
\pgfsetlinewidth{1.003750pt}%
\definecolor{currentstroke}{rgb}{0.000000,0.000000,0.000000}%
\pgfsetstrokecolor{currentstroke}%
\pgfsetdash{}{0pt}%
\pgfpathmoveto{\pgfqpoint{3.806291in}{5.376794in}}%
\pgfpathcurveto{\pgfqpoint{3.817341in}{5.376794in}}{\pgfqpoint{3.827940in}{5.381184in}}{\pgfqpoint{3.835754in}{5.388998in}}%
\pgfpathcurveto{\pgfqpoint{3.843568in}{5.396811in}}{\pgfqpoint{3.847958in}{5.407410in}}{\pgfqpoint{3.847958in}{5.418460in}}%
\pgfpathcurveto{\pgfqpoint{3.847958in}{5.429510in}}{\pgfqpoint{3.843568in}{5.440110in}}{\pgfqpoint{3.835754in}{5.447923in}}%
\pgfpathcurveto{\pgfqpoint{3.827940in}{5.455737in}}{\pgfqpoint{3.817341in}{5.460127in}}{\pgfqpoint{3.806291in}{5.460127in}}%
\pgfpathcurveto{\pgfqpoint{3.795241in}{5.460127in}}{\pgfqpoint{3.784642in}{5.455737in}}{\pgfqpoint{3.776828in}{5.447923in}}%
\pgfpathcurveto{\pgfqpoint{3.769015in}{5.440110in}}{\pgfqpoint{3.764624in}{5.429510in}}{\pgfqpoint{3.764624in}{5.418460in}}%
\pgfpathcurveto{\pgfqpoint{3.764624in}{5.407410in}}{\pgfqpoint{3.769015in}{5.396811in}}{\pgfqpoint{3.776828in}{5.388998in}}%
\pgfpathcurveto{\pgfqpoint{3.784642in}{5.381184in}}{\pgfqpoint{3.795241in}{5.376794in}}{\pgfqpoint{3.806291in}{5.376794in}}%
\pgfpathclose%
\pgfusepath{stroke,fill}%
\end{pgfscope}%
\begin{pgfscope}%
\pgfpathrectangle{\pgfqpoint{0.970666in}{4.121437in}}{\pgfqpoint{5.699255in}{2.685432in}}%
\pgfusepath{clip}%
\pgfsetbuttcap%
\pgfsetroundjoin%
\definecolor{currentfill}{rgb}{0.000000,0.000000,0.000000}%
\pgfsetfillcolor{currentfill}%
\pgfsetlinewidth{1.003750pt}%
\definecolor{currentstroke}{rgb}{0.000000,0.000000,0.000000}%
\pgfsetstrokecolor{currentstroke}%
\pgfsetdash{}{0pt}%
\pgfpathmoveto{\pgfqpoint{6.242828in}{6.630082in}}%
\pgfpathcurveto{\pgfqpoint{6.253878in}{6.630082in}}{\pgfqpoint{6.264477in}{6.634473in}}{\pgfqpoint{6.272291in}{6.642286in}}%
\pgfpathcurveto{\pgfqpoint{6.280104in}{6.650100in}}{\pgfqpoint{6.284494in}{6.660699in}}{\pgfqpoint{6.284494in}{6.671749in}}%
\pgfpathcurveto{\pgfqpoint{6.284494in}{6.682799in}}{\pgfqpoint{6.280104in}{6.693398in}}{\pgfqpoint{6.272291in}{6.701212in}}%
\pgfpathcurveto{\pgfqpoint{6.264477in}{6.709025in}}{\pgfqpoint{6.253878in}{6.713416in}}{\pgfqpoint{6.242828in}{6.713416in}}%
\pgfpathcurveto{\pgfqpoint{6.231778in}{6.713416in}}{\pgfqpoint{6.221179in}{6.709025in}}{\pgfqpoint{6.213365in}{6.701212in}}%
\pgfpathcurveto{\pgfqpoint{6.205551in}{6.693398in}}{\pgfqpoint{6.201161in}{6.682799in}}{\pgfqpoint{6.201161in}{6.671749in}}%
\pgfpathcurveto{\pgfqpoint{6.201161in}{6.660699in}}{\pgfqpoint{6.205551in}{6.650100in}}{\pgfqpoint{6.213365in}{6.642286in}}%
\pgfpathcurveto{\pgfqpoint{6.221179in}{6.634473in}}{\pgfqpoint{6.231778in}{6.630082in}}{\pgfqpoint{6.242828in}{6.630082in}}%
\pgfpathclose%
\pgfusepath{stroke,fill}%
\end{pgfscope}%
\begin{pgfscope}%
\pgfpathrectangle{\pgfqpoint{0.970666in}{4.121437in}}{\pgfqpoint{5.699255in}{2.685432in}}%
\pgfusepath{clip}%
\pgfsetbuttcap%
\pgfsetroundjoin%
\definecolor{currentfill}{rgb}{0.000000,0.000000,0.000000}%
\pgfsetfillcolor{currentfill}%
\pgfsetlinewidth{1.003750pt}%
\definecolor{currentstroke}{rgb}{0.000000,0.000000,0.000000}%
\pgfsetstrokecolor{currentstroke}%
\pgfsetdash{}{0pt}%
\pgfpathmoveto{\pgfqpoint{2.910094in}{5.898997in}}%
\pgfpathcurveto{\pgfqpoint{2.921144in}{5.898997in}}{\pgfqpoint{2.931743in}{5.903388in}}{\pgfqpoint{2.939556in}{5.911201in}}%
\pgfpathcurveto{\pgfqpoint{2.947370in}{5.919015in}}{\pgfqpoint{2.951760in}{5.929614in}}{\pgfqpoint{2.951760in}{5.940664in}}%
\pgfpathcurveto{\pgfqpoint{2.951760in}{5.951714in}}{\pgfqpoint{2.947370in}{5.962313in}}{\pgfqpoint{2.939556in}{5.970127in}}%
\pgfpathcurveto{\pgfqpoint{2.931743in}{5.977940in}}{\pgfqpoint{2.921144in}{5.982331in}}{\pgfqpoint{2.910094in}{5.982331in}}%
\pgfpathcurveto{\pgfqpoint{2.899044in}{5.982331in}}{\pgfqpoint{2.888445in}{5.977940in}}{\pgfqpoint{2.880631in}{5.970127in}}%
\pgfpathcurveto{\pgfqpoint{2.872817in}{5.962313in}}{\pgfqpoint{2.868427in}{5.951714in}}{\pgfqpoint{2.868427in}{5.940664in}}%
\pgfpathcurveto{\pgfqpoint{2.868427in}{5.929614in}}{\pgfqpoint{2.872817in}{5.919015in}}{\pgfqpoint{2.880631in}{5.911201in}}%
\pgfpathcurveto{\pgfqpoint{2.888445in}{5.903388in}}{\pgfqpoint{2.899044in}{5.898997in}}{\pgfqpoint{2.910094in}{5.898997in}}%
\pgfpathclose%
\pgfusepath{stroke,fill}%
\end{pgfscope}%
\begin{pgfscope}%
\pgfpathrectangle{\pgfqpoint{0.970666in}{4.121437in}}{\pgfqpoint{5.699255in}{2.685432in}}%
\pgfusepath{clip}%
\pgfsetbuttcap%
\pgfsetroundjoin%
\definecolor{currentfill}{rgb}{0.000000,0.000000,0.000000}%
\pgfsetfillcolor{currentfill}%
\pgfsetlinewidth{1.003750pt}%
\definecolor{currentstroke}{rgb}{0.000000,0.000000,0.000000}%
\pgfsetstrokecolor{currentstroke}%
\pgfsetdash{}{0pt}%
\pgfpathmoveto{\pgfqpoint{2.714051in}{5.624840in}}%
\pgfpathcurveto{\pgfqpoint{2.725101in}{5.624840in}}{\pgfqpoint{2.735700in}{5.629231in}}{\pgfqpoint{2.743513in}{5.637044in}}%
\pgfpathcurveto{\pgfqpoint{2.751327in}{5.644858in}}{\pgfqpoint{2.755717in}{5.655457in}}{\pgfqpoint{2.755717in}{5.666507in}}%
\pgfpathcurveto{\pgfqpoint{2.755717in}{5.677557in}}{\pgfqpoint{2.751327in}{5.688156in}}{\pgfqpoint{2.743513in}{5.695970in}}%
\pgfpathcurveto{\pgfqpoint{2.735700in}{5.703783in}}{\pgfqpoint{2.725101in}{5.708174in}}{\pgfqpoint{2.714051in}{5.708174in}}%
\pgfpathcurveto{\pgfqpoint{2.703000in}{5.708174in}}{\pgfqpoint{2.692401in}{5.703783in}}{\pgfqpoint{2.684588in}{5.695970in}}%
\pgfpathcurveto{\pgfqpoint{2.676774in}{5.688156in}}{\pgfqpoint{2.672384in}{5.677557in}}{\pgfqpoint{2.672384in}{5.666507in}}%
\pgfpathcurveto{\pgfqpoint{2.672384in}{5.655457in}}{\pgfqpoint{2.676774in}{5.644858in}}{\pgfqpoint{2.684588in}{5.637044in}}%
\pgfpathcurveto{\pgfqpoint{2.692401in}{5.629231in}}{\pgfqpoint{2.703000in}{5.624840in}}{\pgfqpoint{2.714051in}{5.624840in}}%
\pgfpathclose%
\pgfusepath{stroke,fill}%
\end{pgfscope}%
\begin{pgfscope}%
\pgfpathrectangle{\pgfqpoint{0.970666in}{4.121437in}}{\pgfqpoint{5.699255in}{2.685432in}}%
\pgfusepath{clip}%
\pgfsetbuttcap%
\pgfsetroundjoin%
\definecolor{currentfill}{rgb}{0.000000,0.000000,0.000000}%
\pgfsetfillcolor{currentfill}%
\pgfsetlinewidth{1.003750pt}%
\definecolor{currentstroke}{rgb}{0.000000,0.000000,0.000000}%
\pgfsetstrokecolor{currentstroke}%
\pgfsetdash{}{0pt}%
\pgfpathmoveto{\pgfqpoint{2.602026in}{5.520400in}}%
\pgfpathcurveto{\pgfqpoint{2.613076in}{5.520400in}}{\pgfqpoint{2.623675in}{5.524790in}}{\pgfqpoint{2.631489in}{5.532604in}}%
\pgfpathcurveto{\pgfqpoint{2.639302in}{5.540417in}}{\pgfqpoint{2.643692in}{5.551016in}}{\pgfqpoint{2.643692in}{5.562066in}}%
\pgfpathcurveto{\pgfqpoint{2.643692in}{5.573116in}}{\pgfqpoint{2.639302in}{5.583716in}}{\pgfqpoint{2.631489in}{5.591529in}}%
\pgfpathcurveto{\pgfqpoint{2.623675in}{5.599343in}}{\pgfqpoint{2.613076in}{5.603733in}}{\pgfqpoint{2.602026in}{5.603733in}}%
\pgfpathcurveto{\pgfqpoint{2.590976in}{5.603733in}}{\pgfqpoint{2.580377in}{5.599343in}}{\pgfqpoint{2.572563in}{5.591529in}}%
\pgfpathcurveto{\pgfqpoint{2.564749in}{5.583716in}}{\pgfqpoint{2.560359in}{5.573116in}}{\pgfqpoint{2.560359in}{5.562066in}}%
\pgfpathcurveto{\pgfqpoint{2.560359in}{5.551016in}}{\pgfqpoint{2.564749in}{5.540417in}}{\pgfqpoint{2.572563in}{5.532604in}}%
\pgfpathcurveto{\pgfqpoint{2.580377in}{5.524790in}}{\pgfqpoint{2.590976in}{5.520400in}}{\pgfqpoint{2.602026in}{5.520400in}}%
\pgfpathclose%
\pgfusepath{stroke,fill}%
\end{pgfscope}%
\begin{pgfscope}%
\pgfpathrectangle{\pgfqpoint{0.970666in}{4.121437in}}{\pgfqpoint{5.699255in}{2.685432in}}%
\pgfusepath{clip}%
\pgfsetbuttcap%
\pgfsetroundjoin%
\definecolor{currentfill}{rgb}{0.000000,0.000000,0.000000}%
\pgfsetfillcolor{currentfill}%
\pgfsetlinewidth{1.003750pt}%
\definecolor{currentstroke}{rgb}{0.000000,0.000000,0.000000}%
\pgfsetstrokecolor{currentstroke}%
\pgfsetdash{}{0pt}%
\pgfpathmoveto{\pgfqpoint{2.630032in}{4.789315in}}%
\pgfpathcurveto{\pgfqpoint{2.641082in}{4.789315in}}{\pgfqpoint{2.651681in}{4.793705in}}{\pgfqpoint{2.659495in}{4.801518in}}%
\pgfpathcurveto{\pgfqpoint{2.667308in}{4.809332in}}{\pgfqpoint{2.671699in}{4.819931in}}{\pgfqpoint{2.671699in}{4.830981in}}%
\pgfpathcurveto{\pgfqpoint{2.671699in}{4.842031in}}{\pgfqpoint{2.667308in}{4.852630in}}{\pgfqpoint{2.659495in}{4.860444in}}%
\pgfpathcurveto{\pgfqpoint{2.651681in}{4.868258in}}{\pgfqpoint{2.641082in}{4.872648in}}{\pgfqpoint{2.630032in}{4.872648in}}%
\pgfpathcurveto{\pgfqpoint{2.618982in}{4.872648in}}{\pgfqpoint{2.608383in}{4.868258in}}{\pgfqpoint{2.600569in}{4.860444in}}%
\pgfpathcurveto{\pgfqpoint{2.592756in}{4.852630in}}{\pgfqpoint{2.588365in}{4.842031in}}{\pgfqpoint{2.588365in}{4.830981in}}%
\pgfpathcurveto{\pgfqpoint{2.588365in}{4.819931in}}{\pgfqpoint{2.592756in}{4.809332in}}{\pgfqpoint{2.600569in}{4.801518in}}%
\pgfpathcurveto{\pgfqpoint{2.608383in}{4.793705in}}{\pgfqpoint{2.618982in}{4.789315in}}{\pgfqpoint{2.630032in}{4.789315in}}%
\pgfpathclose%
\pgfusepath{stroke,fill}%
\end{pgfscope}%
\begin{pgfscope}%
\pgfpathrectangle{\pgfqpoint{0.970666in}{4.121437in}}{\pgfqpoint{5.699255in}{2.685432in}}%
\pgfusepath{clip}%
\pgfsetbuttcap%
\pgfsetroundjoin%
\definecolor{currentfill}{rgb}{0.000000,0.000000,0.000000}%
\pgfsetfillcolor{currentfill}%
\pgfsetlinewidth{1.003750pt}%
\definecolor{currentstroke}{rgb}{0.000000,0.000000,0.000000}%
\pgfsetstrokecolor{currentstroke}%
\pgfsetdash{}{0pt}%
\pgfpathmoveto{\pgfqpoint{4.730495in}{4.945976in}}%
\pgfpathcurveto{\pgfqpoint{4.741545in}{4.945976in}}{\pgfqpoint{4.752144in}{4.950366in}}{\pgfqpoint{4.759957in}{4.958180in}}%
\pgfpathcurveto{\pgfqpoint{4.767771in}{4.965993in}}{\pgfqpoint{4.772161in}{4.976592in}}{\pgfqpoint{4.772161in}{4.987642in}}%
\pgfpathcurveto{\pgfqpoint{4.772161in}{4.998692in}}{\pgfqpoint{4.767771in}{5.009292in}}{\pgfqpoint{4.759957in}{5.017105in}}%
\pgfpathcurveto{\pgfqpoint{4.752144in}{5.024919in}}{\pgfqpoint{4.741545in}{5.029309in}}{\pgfqpoint{4.730495in}{5.029309in}}%
\pgfpathcurveto{\pgfqpoint{4.719445in}{5.029309in}}{\pgfqpoint{4.708846in}{5.024919in}}{\pgfqpoint{4.701032in}{5.017105in}}%
\pgfpathcurveto{\pgfqpoint{4.693218in}{5.009292in}}{\pgfqpoint{4.688828in}{4.998692in}}{\pgfqpoint{4.688828in}{4.987642in}}%
\pgfpathcurveto{\pgfqpoint{4.688828in}{4.976592in}}{\pgfqpoint{4.693218in}{4.965993in}}{\pgfqpoint{4.701032in}{4.958180in}}%
\pgfpathcurveto{\pgfqpoint{4.708846in}{4.950366in}}{\pgfqpoint{4.719445in}{4.945976in}}{\pgfqpoint{4.730495in}{4.945976in}}%
\pgfpathclose%
\pgfusepath{stroke,fill}%
\end{pgfscope}%
\begin{pgfscope}%
\pgfpathrectangle{\pgfqpoint{0.970666in}{4.121437in}}{\pgfqpoint{5.699255in}{2.685432in}}%
\pgfusepath{clip}%
\pgfsetbuttcap%
\pgfsetroundjoin%
\definecolor{currentfill}{rgb}{0.000000,0.000000,0.000000}%
\pgfsetfillcolor{currentfill}%
\pgfsetlinewidth{1.003750pt}%
\definecolor{currentstroke}{rgb}{0.000000,0.000000,0.000000}%
\pgfsetstrokecolor{currentstroke}%
\pgfsetdash{}{0pt}%
\pgfpathmoveto{\pgfqpoint{3.890310in}{4.410717in}}%
\pgfpathcurveto{\pgfqpoint{3.901360in}{4.410717in}}{\pgfqpoint{3.911959in}{4.415107in}}{\pgfqpoint{3.919772in}{4.422921in}}%
\pgfpathcurveto{\pgfqpoint{3.927586in}{4.430734in}}{\pgfqpoint{3.931976in}{4.441334in}}{\pgfqpoint{3.931976in}{4.452384in}}%
\pgfpathcurveto{\pgfqpoint{3.931976in}{4.463434in}}{\pgfqpoint{3.927586in}{4.474033in}}{\pgfqpoint{3.919772in}{4.481846in}}%
\pgfpathcurveto{\pgfqpoint{3.911959in}{4.489660in}}{\pgfqpoint{3.901360in}{4.494050in}}{\pgfqpoint{3.890310in}{4.494050in}}%
\pgfpathcurveto{\pgfqpoint{3.879259in}{4.494050in}}{\pgfqpoint{3.868660in}{4.489660in}}{\pgfqpoint{3.860847in}{4.481846in}}%
\pgfpathcurveto{\pgfqpoint{3.853033in}{4.474033in}}{\pgfqpoint{3.848643in}{4.463434in}}{\pgfqpoint{3.848643in}{4.452384in}}%
\pgfpathcurveto{\pgfqpoint{3.848643in}{4.441334in}}{\pgfqpoint{3.853033in}{4.430734in}}{\pgfqpoint{3.860847in}{4.422921in}}%
\pgfpathcurveto{\pgfqpoint{3.868660in}{4.415107in}}{\pgfqpoint{3.879259in}{4.410717in}}{\pgfqpoint{3.890310in}{4.410717in}}%
\pgfpathclose%
\pgfusepath{stroke,fill}%
\end{pgfscope}%
\begin{pgfscope}%
\pgfpathrectangle{\pgfqpoint{0.970666in}{4.121437in}}{\pgfqpoint{5.699255in}{2.685432in}}%
\pgfusepath{clip}%
\pgfsetbuttcap%
\pgfsetroundjoin%
\definecolor{currentfill}{rgb}{0.000000,0.000000,0.000000}%
\pgfsetfillcolor{currentfill}%
\pgfsetlinewidth{1.003750pt}%
\definecolor{currentstroke}{rgb}{0.000000,0.000000,0.000000}%
\pgfsetstrokecolor{currentstroke}%
\pgfsetdash{}{0pt}%
\pgfpathmoveto{\pgfqpoint{1.929878in}{4.502103in}}%
\pgfpathcurveto{\pgfqpoint{1.940928in}{4.502103in}}{\pgfqpoint{1.951527in}{4.506493in}}{\pgfqpoint{1.959341in}{4.514307in}}%
\pgfpathcurveto{\pgfqpoint{1.967154in}{4.522120in}}{\pgfqpoint{1.971544in}{4.532719in}}{\pgfqpoint{1.971544in}{4.543769in}}%
\pgfpathcurveto{\pgfqpoint{1.971544in}{4.554819in}}{\pgfqpoint{1.967154in}{4.565418in}}{\pgfqpoint{1.959341in}{4.573232in}}%
\pgfpathcurveto{\pgfqpoint{1.951527in}{4.581046in}}{\pgfqpoint{1.940928in}{4.585436in}}{\pgfqpoint{1.929878in}{4.585436in}}%
\pgfpathcurveto{\pgfqpoint{1.918828in}{4.585436in}}{\pgfqpoint{1.908229in}{4.581046in}}{\pgfqpoint{1.900415in}{4.573232in}}%
\pgfpathcurveto{\pgfqpoint{1.892601in}{4.565418in}}{\pgfqpoint{1.888211in}{4.554819in}}{\pgfqpoint{1.888211in}{4.543769in}}%
\pgfpathcurveto{\pgfqpoint{1.888211in}{4.532719in}}{\pgfqpoint{1.892601in}{4.522120in}}{\pgfqpoint{1.900415in}{4.514307in}}%
\pgfpathcurveto{\pgfqpoint{1.908229in}{4.506493in}}{\pgfqpoint{1.918828in}{4.502103in}}{\pgfqpoint{1.929878in}{4.502103in}}%
\pgfpathclose%
\pgfusepath{stroke,fill}%
\end{pgfscope}%
\begin{pgfscope}%
\pgfpathrectangle{\pgfqpoint{0.970666in}{4.121437in}}{\pgfqpoint{5.699255in}{2.685432in}}%
\pgfusepath{clip}%
\pgfsetbuttcap%
\pgfsetroundjoin%
\definecolor{currentfill}{rgb}{0.000000,0.000000,0.000000}%
\pgfsetfillcolor{currentfill}%
\pgfsetlinewidth{1.003750pt}%
\definecolor{currentstroke}{rgb}{0.000000,0.000000,0.000000}%
\pgfsetstrokecolor{currentstroke}%
\pgfsetdash{}{0pt}%
\pgfpathmoveto{\pgfqpoint{3.442211in}{5.768446in}}%
\pgfpathcurveto{\pgfqpoint{3.453261in}{5.768446in}}{\pgfqpoint{3.463860in}{5.772837in}}{\pgfqpoint{3.471674in}{5.780650in}}%
\pgfpathcurveto{\pgfqpoint{3.479487in}{5.788464in}}{\pgfqpoint{3.483878in}{5.799063in}}{\pgfqpoint{3.483878in}{5.810113in}}%
\pgfpathcurveto{\pgfqpoint{3.483878in}{5.821163in}}{\pgfqpoint{3.479487in}{5.831762in}}{\pgfqpoint{3.471674in}{5.839576in}}%
\pgfpathcurveto{\pgfqpoint{3.463860in}{5.847389in}}{\pgfqpoint{3.453261in}{5.851780in}}{\pgfqpoint{3.442211in}{5.851780in}}%
\pgfpathcurveto{\pgfqpoint{3.431161in}{5.851780in}}{\pgfqpoint{3.420562in}{5.847389in}}{\pgfqpoint{3.412748in}{5.839576in}}%
\pgfpathcurveto{\pgfqpoint{3.404934in}{5.831762in}}{\pgfqpoint{3.400544in}{5.821163in}}{\pgfqpoint{3.400544in}{5.810113in}}%
\pgfpathcurveto{\pgfqpoint{3.400544in}{5.799063in}}{\pgfqpoint{3.404934in}{5.788464in}}{\pgfqpoint{3.412748in}{5.780650in}}%
\pgfpathcurveto{\pgfqpoint{3.420562in}{5.772837in}}{\pgfqpoint{3.431161in}{5.768446in}}{\pgfqpoint{3.442211in}{5.768446in}}%
\pgfpathclose%
\pgfusepath{stroke,fill}%
\end{pgfscope}%
\begin{pgfscope}%
\pgfpathrectangle{\pgfqpoint{0.970666in}{4.121437in}}{\pgfqpoint{5.699255in}{2.685432in}}%
\pgfusepath{clip}%
\pgfsetbuttcap%
\pgfsetroundjoin%
\definecolor{currentfill}{rgb}{0.000000,0.000000,0.000000}%
\pgfsetfillcolor{currentfill}%
\pgfsetlinewidth{1.003750pt}%
\definecolor{currentstroke}{rgb}{0.000000,0.000000,0.000000}%
\pgfsetstrokecolor{currentstroke}%
\pgfsetdash{}{0pt}%
\pgfpathmoveto{\pgfqpoint{5.318624in}{5.846777in}}%
\pgfpathcurveto{\pgfqpoint{5.329674in}{5.846777in}}{\pgfqpoint{5.340273in}{5.851167in}}{\pgfqpoint{5.348087in}{5.858981in}}%
\pgfpathcurveto{\pgfqpoint{5.355901in}{5.866794in}}{\pgfqpoint{5.360291in}{5.877393in}}{\pgfqpoint{5.360291in}{5.888444in}}%
\pgfpathcurveto{\pgfqpoint{5.360291in}{5.899494in}}{\pgfqpoint{5.355901in}{5.910093in}}{\pgfqpoint{5.348087in}{5.917906in}}%
\pgfpathcurveto{\pgfqpoint{5.340273in}{5.925720in}}{\pgfqpoint{5.329674in}{5.930110in}}{\pgfqpoint{5.318624in}{5.930110in}}%
\pgfpathcurveto{\pgfqpoint{5.307574in}{5.930110in}}{\pgfqpoint{5.296975in}{5.925720in}}{\pgfqpoint{5.289161in}{5.917906in}}%
\pgfpathcurveto{\pgfqpoint{5.281348in}{5.910093in}}{\pgfqpoint{5.276958in}{5.899494in}}{\pgfqpoint{5.276958in}{5.888444in}}%
\pgfpathcurveto{\pgfqpoint{5.276958in}{5.877393in}}{\pgfqpoint{5.281348in}{5.866794in}}{\pgfqpoint{5.289161in}{5.858981in}}%
\pgfpathcurveto{\pgfqpoint{5.296975in}{5.851167in}}{\pgfqpoint{5.307574in}{5.846777in}}{\pgfqpoint{5.318624in}{5.846777in}}%
\pgfpathclose%
\pgfusepath{stroke,fill}%
\end{pgfscope}%
\begin{pgfscope}%
\pgfpathrectangle{\pgfqpoint{0.970666in}{4.121437in}}{\pgfqpoint{5.699255in}{2.685432in}}%
\pgfusepath{clip}%
\pgfsetbuttcap%
\pgfsetroundjoin%
\definecolor{currentfill}{rgb}{0.000000,0.000000,0.000000}%
\pgfsetfillcolor{currentfill}%
\pgfsetlinewidth{1.003750pt}%
\definecolor{currentstroke}{rgb}{0.000000,0.000000,0.000000}%
\pgfsetstrokecolor{currentstroke}%
\pgfsetdash{}{0pt}%
\pgfpathmoveto{\pgfqpoint{3.946322in}{6.590917in}}%
\pgfpathcurveto{\pgfqpoint{3.957372in}{6.590917in}}{\pgfqpoint{3.967971in}{6.595307in}}{\pgfqpoint{3.975785in}{6.603121in}}%
\pgfpathcurveto{\pgfqpoint{3.983598in}{6.610935in}}{\pgfqpoint{3.987989in}{6.621534in}}{\pgfqpoint{3.987989in}{6.632584in}}%
\pgfpathcurveto{\pgfqpoint{3.987989in}{6.643634in}}{\pgfqpoint{3.983598in}{6.654233in}}{\pgfqpoint{3.975785in}{6.662047in}}%
\pgfpathcurveto{\pgfqpoint{3.967971in}{6.669860in}}{\pgfqpoint{3.957372in}{6.674250in}}{\pgfqpoint{3.946322in}{6.674250in}}%
\pgfpathcurveto{\pgfqpoint{3.935272in}{6.674250in}}{\pgfqpoint{3.924673in}{6.669860in}}{\pgfqpoint{3.916859in}{6.662047in}}%
\pgfpathcurveto{\pgfqpoint{3.909046in}{6.654233in}}{\pgfqpoint{3.904655in}{6.643634in}}{\pgfqpoint{3.904655in}{6.632584in}}%
\pgfpathcurveto{\pgfqpoint{3.904655in}{6.621534in}}{\pgfqpoint{3.909046in}{6.610935in}}{\pgfqpoint{3.916859in}{6.603121in}}%
\pgfpathcurveto{\pgfqpoint{3.924673in}{6.595307in}}{\pgfqpoint{3.935272in}{6.590917in}}{\pgfqpoint{3.946322in}{6.590917in}}%
\pgfpathclose%
\pgfusepath{stroke,fill}%
\end{pgfscope}%
\begin{pgfscope}%
\pgfpathrectangle{\pgfqpoint{0.970666in}{4.121437in}}{\pgfqpoint{5.699255in}{2.685432in}}%
\pgfusepath{clip}%
\pgfsetbuttcap%
\pgfsetroundjoin%
\definecolor{currentfill}{rgb}{0.000000,0.000000,0.000000}%
\pgfsetfillcolor{currentfill}%
\pgfsetlinewidth{1.003750pt}%
\definecolor{currentstroke}{rgb}{0.000000,0.000000,0.000000}%
\pgfsetstrokecolor{currentstroke}%
\pgfsetdash{}{0pt}%
\pgfpathmoveto{\pgfqpoint{3.162149in}{6.434256in}}%
\pgfpathcurveto{\pgfqpoint{3.173199in}{6.434256in}}{\pgfqpoint{3.183798in}{6.438646in}}{\pgfqpoint{3.191612in}{6.446460in}}%
\pgfpathcurveto{\pgfqpoint{3.199426in}{6.454274in}}{\pgfqpoint{3.203816in}{6.464873in}}{\pgfqpoint{3.203816in}{6.475923in}}%
\pgfpathcurveto{\pgfqpoint{3.203816in}{6.486973in}}{\pgfqpoint{3.199426in}{6.497572in}}{\pgfqpoint{3.191612in}{6.505385in}}%
\pgfpathcurveto{\pgfqpoint{3.183798in}{6.513199in}}{\pgfqpoint{3.173199in}{6.517589in}}{\pgfqpoint{3.162149in}{6.517589in}}%
\pgfpathcurveto{\pgfqpoint{3.151099in}{6.517589in}}{\pgfqpoint{3.140500in}{6.513199in}}{\pgfqpoint{3.132686in}{6.505385in}}%
\pgfpathcurveto{\pgfqpoint{3.124873in}{6.497572in}}{\pgfqpoint{3.120483in}{6.486973in}}{\pgfqpoint{3.120483in}{6.475923in}}%
\pgfpathcurveto{\pgfqpoint{3.120483in}{6.464873in}}{\pgfqpoint{3.124873in}{6.454274in}}{\pgfqpoint{3.132686in}{6.446460in}}%
\pgfpathcurveto{\pgfqpoint{3.140500in}{6.438646in}}{\pgfqpoint{3.151099in}{6.434256in}}{\pgfqpoint{3.162149in}{6.434256in}}%
\pgfpathclose%
\pgfusepath{stroke,fill}%
\end{pgfscope}%
\begin{pgfscope}%
\pgfpathrectangle{\pgfqpoint{0.970666in}{4.121437in}}{\pgfqpoint{5.699255in}{2.685432in}}%
\pgfusepath{clip}%
\pgfsetbuttcap%
\pgfsetroundjoin%
\definecolor{currentfill}{rgb}{0.000000,0.000000,0.000000}%
\pgfsetfillcolor{currentfill}%
\pgfsetlinewidth{1.003750pt}%
\definecolor{currentstroke}{rgb}{0.000000,0.000000,0.000000}%
\pgfsetstrokecolor{currentstroke}%
\pgfsetdash{}{0pt}%
\pgfpathmoveto{\pgfqpoint{2.153927in}{5.990383in}}%
\pgfpathcurveto{\pgfqpoint{2.164977in}{5.990383in}}{\pgfqpoint{2.175576in}{5.994773in}}{\pgfqpoint{2.183390in}{6.002587in}}%
\pgfpathcurveto{\pgfqpoint{2.191204in}{6.010400in}}{\pgfqpoint{2.195594in}{6.020999in}}{\pgfqpoint{2.195594in}{6.032050in}}%
\pgfpathcurveto{\pgfqpoint{2.195594in}{6.043100in}}{\pgfqpoint{2.191204in}{6.053699in}}{\pgfqpoint{2.183390in}{6.061512in}}%
\pgfpathcurveto{\pgfqpoint{2.175576in}{6.069326in}}{\pgfqpoint{2.164977in}{6.073716in}}{\pgfqpoint{2.153927in}{6.073716in}}%
\pgfpathcurveto{\pgfqpoint{2.142877in}{6.073716in}}{\pgfqpoint{2.132278in}{6.069326in}}{\pgfqpoint{2.124464in}{6.061512in}}%
\pgfpathcurveto{\pgfqpoint{2.116651in}{6.053699in}}{\pgfqpoint{2.112260in}{6.043100in}}{\pgfqpoint{2.112260in}{6.032050in}}%
\pgfpathcurveto{\pgfqpoint{2.112260in}{6.020999in}}{\pgfqpoint{2.116651in}{6.010400in}}{\pgfqpoint{2.124464in}{6.002587in}}%
\pgfpathcurveto{\pgfqpoint{2.132278in}{5.994773in}}{\pgfqpoint{2.142877in}{5.990383in}}{\pgfqpoint{2.153927in}{5.990383in}}%
\pgfpathclose%
\pgfusepath{stroke,fill}%
\end{pgfscope}%
\begin{pgfscope}%
\pgfpathrectangle{\pgfqpoint{0.970666in}{4.121437in}}{\pgfqpoint{5.699255in}{2.685432in}}%
\pgfusepath{clip}%
\pgfsetbuttcap%
\pgfsetroundjoin%
\definecolor{currentfill}{rgb}{0.000000,0.000000,0.000000}%
\pgfsetfillcolor{currentfill}%
\pgfsetlinewidth{1.003750pt}%
\definecolor{currentstroke}{rgb}{0.000000,0.000000,0.000000}%
\pgfsetstrokecolor{currentstroke}%
\pgfsetdash{}{0pt}%
\pgfpathmoveto{\pgfqpoint{6.410865in}{5.938163in}}%
\pgfpathcurveto{\pgfqpoint{6.421915in}{5.938163in}}{\pgfqpoint{6.432514in}{5.942553in}}{\pgfqpoint{6.440328in}{5.950366in}}%
\pgfpathcurveto{\pgfqpoint{6.448141in}{5.958180in}}{\pgfqpoint{6.452531in}{5.968779in}}{\pgfqpoint{6.452531in}{5.979829in}}%
\pgfpathcurveto{\pgfqpoint{6.452531in}{5.990879in}}{\pgfqpoint{6.448141in}{6.001478in}}{\pgfqpoint{6.440328in}{6.009292in}}%
\pgfpathcurveto{\pgfqpoint{6.432514in}{6.017106in}}{\pgfqpoint{6.421915in}{6.021496in}}{\pgfqpoint{6.410865in}{6.021496in}}%
\pgfpathcurveto{\pgfqpoint{6.399815in}{6.021496in}}{\pgfqpoint{6.389216in}{6.017106in}}{\pgfqpoint{6.381402in}{6.009292in}}%
\pgfpathcurveto{\pgfqpoint{6.373588in}{6.001478in}}{\pgfqpoint{6.369198in}{5.990879in}}{\pgfqpoint{6.369198in}{5.979829in}}%
\pgfpathcurveto{\pgfqpoint{6.369198in}{5.968779in}}{\pgfqpoint{6.373588in}{5.958180in}}{\pgfqpoint{6.381402in}{5.950366in}}%
\pgfpathcurveto{\pgfqpoint{6.389216in}{5.942553in}}{\pgfqpoint{6.399815in}{5.938163in}}{\pgfqpoint{6.410865in}{5.938163in}}%
\pgfpathclose%
\pgfusepath{stroke,fill}%
\end{pgfscope}%
\begin{pgfscope}%
\pgfpathrectangle{\pgfqpoint{0.970666in}{4.121437in}}{\pgfqpoint{5.699255in}{2.685432in}}%
\pgfusepath{clip}%
\pgfsetbuttcap%
\pgfsetroundjoin%
\definecolor{currentfill}{rgb}{0.000000,0.000000,0.000000}%
\pgfsetfillcolor{currentfill}%
\pgfsetlinewidth{1.003750pt}%
\definecolor{currentstroke}{rgb}{0.000000,0.000000,0.000000}%
\pgfsetstrokecolor{currentstroke}%
\pgfsetdash{}{0pt}%
\pgfpathmoveto{\pgfqpoint{4.590464in}{4.436827in}}%
\pgfpathcurveto{\pgfqpoint{4.601514in}{4.436827in}}{\pgfqpoint{4.612113in}{4.441217in}}{\pgfqpoint{4.619927in}{4.449031in}}%
\pgfpathcurveto{\pgfqpoint{4.627740in}{4.456845in}}{\pgfqpoint{4.632130in}{4.467444in}}{\pgfqpoint{4.632130in}{4.478494in}}%
\pgfpathcurveto{\pgfqpoint{4.632130in}{4.489544in}}{\pgfqpoint{4.627740in}{4.500143in}}{\pgfqpoint{4.619927in}{4.507957in}}%
\pgfpathcurveto{\pgfqpoint{4.612113in}{4.515770in}}{\pgfqpoint{4.601514in}{4.520161in}}{\pgfqpoint{4.590464in}{4.520161in}}%
\pgfpathcurveto{\pgfqpoint{4.579414in}{4.520161in}}{\pgfqpoint{4.568815in}{4.515770in}}{\pgfqpoint{4.561001in}{4.507957in}}%
\pgfpathcurveto{\pgfqpoint{4.553187in}{4.500143in}}{\pgfqpoint{4.548797in}{4.489544in}}{\pgfqpoint{4.548797in}{4.478494in}}%
\pgfpathcurveto{\pgfqpoint{4.548797in}{4.467444in}}{\pgfqpoint{4.553187in}{4.456845in}}{\pgfqpoint{4.561001in}{4.449031in}}%
\pgfpathcurveto{\pgfqpoint{4.568815in}{4.441217in}}{\pgfqpoint{4.579414in}{4.436827in}}{\pgfqpoint{4.590464in}{4.436827in}}%
\pgfpathclose%
\pgfusepath{stroke,fill}%
\end{pgfscope}%
\begin{pgfscope}%
\pgfpathrectangle{\pgfqpoint{0.970666in}{4.121437in}}{\pgfqpoint{5.699255in}{2.685432in}}%
\pgfusepath{clip}%
\pgfsetbuttcap%
\pgfsetroundjoin%
\definecolor{currentfill}{rgb}{0.000000,0.000000,0.000000}%
\pgfsetfillcolor{currentfill}%
\pgfsetlinewidth{1.003750pt}%
\definecolor{currentstroke}{rgb}{0.000000,0.000000,0.000000}%
\pgfsetstrokecolor{currentstroke}%
\pgfsetdash{}{0pt}%
\pgfpathmoveto{\pgfqpoint{1.593804in}{4.345442in}}%
\pgfpathcurveto{\pgfqpoint{1.604854in}{4.345442in}}{\pgfqpoint{1.615453in}{4.349832in}}{\pgfqpoint{1.623267in}{4.357645in}}%
\pgfpathcurveto{\pgfqpoint{1.631080in}{4.365459in}}{\pgfqpoint{1.635470in}{4.376058in}}{\pgfqpoint{1.635470in}{4.387108in}}%
\pgfpathcurveto{\pgfqpoint{1.635470in}{4.398158in}}{\pgfqpoint{1.631080in}{4.408757in}}{\pgfqpoint{1.623267in}{4.416571in}}%
\pgfpathcurveto{\pgfqpoint{1.615453in}{4.424385in}}{\pgfqpoint{1.604854in}{4.428775in}}{\pgfqpoint{1.593804in}{4.428775in}}%
\pgfpathcurveto{\pgfqpoint{1.582754in}{4.428775in}}{\pgfqpoint{1.572155in}{4.424385in}}{\pgfqpoint{1.564341in}{4.416571in}}%
\pgfpathcurveto{\pgfqpoint{1.556527in}{4.408757in}}{\pgfqpoint{1.552137in}{4.398158in}}{\pgfqpoint{1.552137in}{4.387108in}}%
\pgfpathcurveto{\pgfqpoint{1.552137in}{4.376058in}}{\pgfqpoint{1.556527in}{4.365459in}}{\pgfqpoint{1.564341in}{4.357645in}}%
\pgfpathcurveto{\pgfqpoint{1.572155in}{4.349832in}}{\pgfqpoint{1.582754in}{4.345442in}}{\pgfqpoint{1.593804in}{4.345442in}}%
\pgfpathclose%
\pgfusepath{stroke,fill}%
\end{pgfscope}%
\begin{pgfscope}%
\pgfpathrectangle{\pgfqpoint{0.970666in}{4.121437in}}{\pgfqpoint{5.699255in}{2.685432in}}%
\pgfusepath{clip}%
\pgfsetbuttcap%
\pgfsetroundjoin%
\definecolor{currentfill}{rgb}{0.000000,0.000000,0.000000}%
\pgfsetfillcolor{currentfill}%
\pgfsetlinewidth{1.003750pt}%
\definecolor{currentstroke}{rgb}{0.000000,0.000000,0.000000}%
\pgfsetstrokecolor{currentstroke}%
\pgfsetdash{}{0pt}%
\pgfpathmoveto{\pgfqpoint{2.658038in}{4.815425in}}%
\pgfpathcurveto{\pgfqpoint{2.669088in}{4.815425in}}{\pgfqpoint{2.679687in}{4.819815in}}{\pgfqpoint{2.687501in}{4.827629in}}%
\pgfpathcurveto{\pgfqpoint{2.695315in}{4.835442in}}{\pgfqpoint{2.699705in}{4.846041in}}{\pgfqpoint{2.699705in}{4.857091in}}%
\pgfpathcurveto{\pgfqpoint{2.699705in}{4.868142in}}{\pgfqpoint{2.695315in}{4.878741in}}{\pgfqpoint{2.687501in}{4.886554in}}%
\pgfpathcurveto{\pgfqpoint{2.679687in}{4.894368in}}{\pgfqpoint{2.669088in}{4.898758in}}{\pgfqpoint{2.658038in}{4.898758in}}%
\pgfpathcurveto{\pgfqpoint{2.646988in}{4.898758in}}{\pgfqpoint{2.636389in}{4.894368in}}{\pgfqpoint{2.628575in}{4.886554in}}%
\pgfpathcurveto{\pgfqpoint{2.620762in}{4.878741in}}{\pgfqpoint{2.616372in}{4.868142in}}{\pgfqpoint{2.616372in}{4.857091in}}%
\pgfpathcurveto{\pgfqpoint{2.616372in}{4.846041in}}{\pgfqpoint{2.620762in}{4.835442in}}{\pgfqpoint{2.628575in}{4.827629in}}%
\pgfpathcurveto{\pgfqpoint{2.636389in}{4.819815in}}{\pgfqpoint{2.646988in}{4.815425in}}{\pgfqpoint{2.658038in}{4.815425in}}%
\pgfpathclose%
\pgfusepath{stroke,fill}%
\end{pgfscope}%
\begin{pgfscope}%
\pgfpathrectangle{\pgfqpoint{0.970666in}{4.121437in}}{\pgfqpoint{5.699255in}{2.685432in}}%
\pgfusepath{clip}%
\pgfsetbuttcap%
\pgfsetroundjoin%
\definecolor{currentfill}{rgb}{0.000000,0.000000,0.000000}%
\pgfsetfillcolor{currentfill}%
\pgfsetlinewidth{1.003750pt}%
\definecolor{currentstroke}{rgb}{0.000000,0.000000,0.000000}%
\pgfsetstrokecolor{currentstroke}%
\pgfsetdash{}{0pt}%
\pgfpathmoveto{\pgfqpoint{1.985890in}{5.820667in}}%
\pgfpathcurveto{\pgfqpoint{1.996940in}{5.820667in}}{\pgfqpoint{2.007539in}{5.825057in}}{\pgfqpoint{2.015353in}{5.832871in}}%
\pgfpathcurveto{\pgfqpoint{2.023167in}{5.840684in}}{\pgfqpoint{2.027557in}{5.851283in}}{\pgfqpoint{2.027557in}{5.862333in}}%
\pgfpathcurveto{\pgfqpoint{2.027557in}{5.873384in}}{\pgfqpoint{2.023167in}{5.883983in}}{\pgfqpoint{2.015353in}{5.891796in}}%
\pgfpathcurveto{\pgfqpoint{2.007539in}{5.899610in}}{\pgfqpoint{1.996940in}{5.904000in}}{\pgfqpoint{1.985890in}{5.904000in}}%
\pgfpathcurveto{\pgfqpoint{1.974840in}{5.904000in}}{\pgfqpoint{1.964241in}{5.899610in}}{\pgfqpoint{1.956427in}{5.891796in}}%
\pgfpathcurveto{\pgfqpoint{1.948614in}{5.883983in}}{\pgfqpoint{1.944223in}{5.873384in}}{\pgfqpoint{1.944223in}{5.862333in}}%
\pgfpathcurveto{\pgfqpoint{1.944223in}{5.851283in}}{\pgfqpoint{1.948614in}{5.840684in}}{\pgfqpoint{1.956427in}{5.832871in}}%
\pgfpathcurveto{\pgfqpoint{1.964241in}{5.825057in}}{\pgfqpoint{1.974840in}{5.820667in}}{\pgfqpoint{1.985890in}{5.820667in}}%
\pgfpathclose%
\pgfusepath{stroke,fill}%
\end{pgfscope}%
\begin{pgfscope}%
\pgfpathrectangle{\pgfqpoint{0.970666in}{4.121437in}}{\pgfqpoint{5.699255in}{2.685432in}}%
\pgfusepath{clip}%
\pgfsetbuttcap%
\pgfsetroundjoin%
\definecolor{currentfill}{rgb}{0.000000,0.000000,0.000000}%
\pgfsetfillcolor{currentfill}%
\pgfsetlinewidth{1.003750pt}%
\definecolor{currentstroke}{rgb}{0.000000,0.000000,0.000000}%
\pgfsetstrokecolor{currentstroke}%
\pgfsetdash{}{0pt}%
\pgfpathmoveto{\pgfqpoint{3.078131in}{6.120934in}}%
\pgfpathcurveto{\pgfqpoint{3.089181in}{6.120934in}}{\pgfqpoint{3.099780in}{6.125324in}}{\pgfqpoint{3.107593in}{6.133138in}}%
\pgfpathcurveto{\pgfqpoint{3.115407in}{6.140951in}}{\pgfqpoint{3.119797in}{6.151550in}}{\pgfqpoint{3.119797in}{6.162601in}}%
\pgfpathcurveto{\pgfqpoint{3.119797in}{6.173651in}}{\pgfqpoint{3.115407in}{6.184250in}}{\pgfqpoint{3.107593in}{6.192063in}}%
\pgfpathcurveto{\pgfqpoint{3.099780in}{6.199877in}}{\pgfqpoint{3.089181in}{6.204267in}}{\pgfqpoint{3.078131in}{6.204267in}}%
\pgfpathcurveto{\pgfqpoint{3.067081in}{6.204267in}}{\pgfqpoint{3.056482in}{6.199877in}}{\pgfqpoint{3.048668in}{6.192063in}}%
\pgfpathcurveto{\pgfqpoint{3.040854in}{6.184250in}}{\pgfqpoint{3.036464in}{6.173651in}}{\pgfqpoint{3.036464in}{6.162601in}}%
\pgfpathcurveto{\pgfqpoint{3.036464in}{6.151550in}}{\pgfqpoint{3.040854in}{6.140951in}}{\pgfqpoint{3.048668in}{6.133138in}}%
\pgfpathcurveto{\pgfqpoint{3.056482in}{6.125324in}}{\pgfqpoint{3.067081in}{6.120934in}}{\pgfqpoint{3.078131in}{6.120934in}}%
\pgfpathclose%
\pgfusepath{stroke,fill}%
\end{pgfscope}%
\begin{pgfscope}%
\pgfpathrectangle{\pgfqpoint{0.970666in}{4.121437in}}{\pgfqpoint{5.699255in}{2.685432in}}%
\pgfusepath{clip}%
\pgfsetbuttcap%
\pgfsetroundjoin%
\definecolor{currentfill}{rgb}{0.000000,0.000000,0.000000}%
\pgfsetfillcolor{currentfill}%
\pgfsetlinewidth{1.003750pt}%
\definecolor{currentstroke}{rgb}{0.000000,0.000000,0.000000}%
\pgfsetstrokecolor{currentstroke}%
\pgfsetdash{}{0pt}%
\pgfpathmoveto{\pgfqpoint{2.910094in}{5.298463in}}%
\pgfpathcurveto{\pgfqpoint{2.921144in}{5.298463in}}{\pgfqpoint{2.931743in}{5.302853in}}{\pgfqpoint{2.939556in}{5.310667in}}%
\pgfpathcurveto{\pgfqpoint{2.947370in}{5.318481in}}{\pgfqpoint{2.951760in}{5.329080in}}{\pgfqpoint{2.951760in}{5.340130in}}%
\pgfpathcurveto{\pgfqpoint{2.951760in}{5.351180in}}{\pgfqpoint{2.947370in}{5.361779in}}{\pgfqpoint{2.939556in}{5.369593in}}%
\pgfpathcurveto{\pgfqpoint{2.931743in}{5.377406in}}{\pgfqpoint{2.921144in}{5.381796in}}{\pgfqpoint{2.910094in}{5.381796in}}%
\pgfpathcurveto{\pgfqpoint{2.899044in}{5.381796in}}{\pgfqpoint{2.888445in}{5.377406in}}{\pgfqpoint{2.880631in}{5.369593in}}%
\pgfpathcurveto{\pgfqpoint{2.872817in}{5.361779in}}{\pgfqpoint{2.868427in}{5.351180in}}{\pgfqpoint{2.868427in}{5.340130in}}%
\pgfpathcurveto{\pgfqpoint{2.868427in}{5.329080in}}{\pgfqpoint{2.872817in}{5.318481in}}{\pgfqpoint{2.880631in}{5.310667in}}%
\pgfpathcurveto{\pgfqpoint{2.888445in}{5.302853in}}{\pgfqpoint{2.899044in}{5.298463in}}{\pgfqpoint{2.910094in}{5.298463in}}%
\pgfpathclose%
\pgfusepath{stroke,fill}%
\end{pgfscope}%
\begin{pgfscope}%
\pgfpathrectangle{\pgfqpoint{0.970666in}{4.121437in}}{\pgfqpoint{5.699255in}{2.685432in}}%
\pgfusepath{clip}%
\pgfsetbuttcap%
\pgfsetroundjoin%
\definecolor{currentfill}{rgb}{0.000000,0.000000,0.000000}%
\pgfsetfillcolor{currentfill}%
\pgfsetlinewidth{1.003750pt}%
\definecolor{currentstroke}{rgb}{0.000000,0.000000,0.000000}%
\pgfsetstrokecolor{currentstroke}%
\pgfsetdash{}{0pt}%
\pgfpathmoveto{\pgfqpoint{4.898532in}{4.449882in}}%
\pgfpathcurveto{\pgfqpoint{4.909582in}{4.449882in}}{\pgfqpoint{4.920181in}{4.454273in}}{\pgfqpoint{4.927994in}{4.462086in}}%
\pgfpathcurveto{\pgfqpoint{4.935808in}{4.469900in}}{\pgfqpoint{4.940198in}{4.480499in}}{\pgfqpoint{4.940198in}{4.491549in}}%
\pgfpathcurveto{\pgfqpoint{4.940198in}{4.502599in}}{\pgfqpoint{4.935808in}{4.513198in}}{\pgfqpoint{4.927994in}{4.521012in}}%
\pgfpathcurveto{\pgfqpoint{4.920181in}{4.528825in}}{\pgfqpoint{4.909582in}{4.533216in}}{\pgfqpoint{4.898532in}{4.533216in}}%
\pgfpathcurveto{\pgfqpoint{4.887482in}{4.533216in}}{\pgfqpoint{4.876883in}{4.528825in}}{\pgfqpoint{4.869069in}{4.521012in}}%
\pgfpathcurveto{\pgfqpoint{4.861255in}{4.513198in}}{\pgfqpoint{4.856865in}{4.502599in}}{\pgfqpoint{4.856865in}{4.491549in}}%
\pgfpathcurveto{\pgfqpoint{4.856865in}{4.480499in}}{\pgfqpoint{4.861255in}{4.469900in}}{\pgfqpoint{4.869069in}{4.462086in}}%
\pgfpathcurveto{\pgfqpoint{4.876883in}{4.454273in}}{\pgfqpoint{4.887482in}{4.449882in}}{\pgfqpoint{4.898532in}{4.449882in}}%
\pgfpathclose%
\pgfusepath{stroke,fill}%
\end{pgfscope}%
\begin{pgfscope}%
\pgfpathrectangle{\pgfqpoint{0.970666in}{4.121437in}}{\pgfqpoint{5.699255in}{2.685432in}}%
\pgfusepath{clip}%
\pgfsetbuttcap%
\pgfsetroundjoin%
\definecolor{currentfill}{rgb}{0.000000,0.000000,0.000000}%
\pgfsetfillcolor{currentfill}%
\pgfsetlinewidth{1.003750pt}%
\definecolor{currentstroke}{rgb}{0.000000,0.000000,0.000000}%
\pgfsetstrokecolor{currentstroke}%
\pgfsetdash{}{0pt}%
\pgfpathmoveto{\pgfqpoint{3.946322in}{5.964273in}}%
\pgfpathcurveto{\pgfqpoint{3.957372in}{5.964273in}}{\pgfqpoint{3.967971in}{5.968663in}}{\pgfqpoint{3.975785in}{5.976477in}}%
\pgfpathcurveto{\pgfqpoint{3.983598in}{5.984290in}}{\pgfqpoint{3.987989in}{5.994889in}}{\pgfqpoint{3.987989in}{6.005939in}}%
\pgfpathcurveto{\pgfqpoint{3.987989in}{6.016990in}}{\pgfqpoint{3.983598in}{6.027589in}}{\pgfqpoint{3.975785in}{6.035402in}}%
\pgfpathcurveto{\pgfqpoint{3.967971in}{6.043216in}}{\pgfqpoint{3.957372in}{6.047606in}}{\pgfqpoint{3.946322in}{6.047606in}}%
\pgfpathcurveto{\pgfqpoint{3.935272in}{6.047606in}}{\pgfqpoint{3.924673in}{6.043216in}}{\pgfqpoint{3.916859in}{6.035402in}}%
\pgfpathcurveto{\pgfqpoint{3.909046in}{6.027589in}}{\pgfqpoint{3.904655in}{6.016990in}}{\pgfqpoint{3.904655in}{6.005939in}}%
\pgfpathcurveto{\pgfqpoint{3.904655in}{5.994889in}}{\pgfqpoint{3.909046in}{5.984290in}}{\pgfqpoint{3.916859in}{5.976477in}}%
\pgfpathcurveto{\pgfqpoint{3.924673in}{5.968663in}}{\pgfqpoint{3.935272in}{5.964273in}}{\pgfqpoint{3.946322in}{5.964273in}}%
\pgfpathclose%
\pgfusepath{stroke,fill}%
\end{pgfscope}%
\begin{pgfscope}%
\pgfpathrectangle{\pgfqpoint{0.970666in}{4.121437in}}{\pgfqpoint{5.699255in}{2.685432in}}%
\pgfusepath{clip}%
\pgfsetbuttcap%
\pgfsetroundjoin%
\definecolor{currentfill}{rgb}{0.000000,0.000000,0.000000}%
\pgfsetfillcolor{currentfill}%
\pgfsetlinewidth{1.003750pt}%
\definecolor{currentstroke}{rgb}{0.000000,0.000000,0.000000}%
\pgfsetstrokecolor{currentstroke}%
\pgfsetdash{}{0pt}%
\pgfpathmoveto{\pgfqpoint{1.285736in}{6.225375in}}%
\pgfpathcurveto{\pgfqpoint{1.296786in}{6.225375in}}{\pgfqpoint{1.307385in}{6.229765in}}{\pgfqpoint{1.315199in}{6.237578in}}%
\pgfpathcurveto{\pgfqpoint{1.323012in}{6.245392in}}{\pgfqpoint{1.327403in}{6.255991in}}{\pgfqpoint{1.327403in}{6.267041in}}%
\pgfpathcurveto{\pgfqpoint{1.327403in}{6.278091in}}{\pgfqpoint{1.323012in}{6.288690in}}{\pgfqpoint{1.315199in}{6.296504in}}%
\pgfpathcurveto{\pgfqpoint{1.307385in}{6.304318in}}{\pgfqpoint{1.296786in}{6.308708in}}{\pgfqpoint{1.285736in}{6.308708in}}%
\pgfpathcurveto{\pgfqpoint{1.274686in}{6.308708in}}{\pgfqpoint{1.264087in}{6.304318in}}{\pgfqpoint{1.256273in}{6.296504in}}%
\pgfpathcurveto{\pgfqpoint{1.248459in}{6.288690in}}{\pgfqpoint{1.244069in}{6.278091in}}{\pgfqpoint{1.244069in}{6.267041in}}%
\pgfpathcurveto{\pgfqpoint{1.244069in}{6.255991in}}{\pgfqpoint{1.248459in}{6.245392in}}{\pgfqpoint{1.256273in}{6.237578in}}%
\pgfpathcurveto{\pgfqpoint{1.264087in}{6.229765in}}{\pgfqpoint{1.274686in}{6.225375in}}{\pgfqpoint{1.285736in}{6.225375in}}%
\pgfpathclose%
\pgfusepath{stroke,fill}%
\end{pgfscope}%
\begin{pgfscope}%
\pgfpathrectangle{\pgfqpoint{0.970666in}{4.121437in}}{\pgfqpoint{5.699255in}{2.685432in}}%
\pgfusepath{clip}%
\pgfsetbuttcap%
\pgfsetroundjoin%
\definecolor{currentfill}{rgb}{0.000000,0.000000,0.000000}%
\pgfsetfillcolor{currentfill}%
\pgfsetlinewidth{1.003750pt}%
\definecolor{currentstroke}{rgb}{0.000000,0.000000,0.000000}%
\pgfsetstrokecolor{currentstroke}%
\pgfsetdash{}{0pt}%
\pgfpathmoveto{\pgfqpoint{3.190155in}{5.429014in}}%
\pgfpathcurveto{\pgfqpoint{3.201206in}{5.429014in}}{\pgfqpoint{3.211805in}{5.433404in}}{\pgfqpoint{3.219618in}{5.441218in}}%
\pgfpathcurveto{\pgfqpoint{3.227432in}{5.449032in}}{\pgfqpoint{3.231822in}{5.459631in}}{\pgfqpoint{3.231822in}{5.470681in}}%
\pgfpathcurveto{\pgfqpoint{3.231822in}{5.481731in}}{\pgfqpoint{3.227432in}{5.492330in}}{\pgfqpoint{3.219618in}{5.500143in}}%
\pgfpathcurveto{\pgfqpoint{3.211805in}{5.507957in}}{\pgfqpoint{3.201206in}{5.512347in}}{\pgfqpoint{3.190155in}{5.512347in}}%
\pgfpathcurveto{\pgfqpoint{3.179105in}{5.512347in}}{\pgfqpoint{3.168506in}{5.507957in}}{\pgfqpoint{3.160693in}{5.500143in}}%
\pgfpathcurveto{\pgfqpoint{3.152879in}{5.492330in}}{\pgfqpoint{3.148489in}{5.481731in}}{\pgfqpoint{3.148489in}{5.470681in}}%
\pgfpathcurveto{\pgfqpoint{3.148489in}{5.459631in}}{\pgfqpoint{3.152879in}{5.449032in}}{\pgfqpoint{3.160693in}{5.441218in}}%
\pgfpathcurveto{\pgfqpoint{3.168506in}{5.433404in}}{\pgfqpoint{3.179105in}{5.429014in}}{\pgfqpoint{3.190155in}{5.429014in}}%
\pgfpathclose%
\pgfusepath{stroke,fill}%
\end{pgfscope}%
\begin{pgfscope}%
\pgfpathrectangle{\pgfqpoint{0.970666in}{4.121437in}}{\pgfqpoint{5.699255in}{2.685432in}}%
\pgfusepath{clip}%
\pgfsetbuttcap%
\pgfsetroundjoin%
\definecolor{currentfill}{rgb}{0.000000,0.000000,0.000000}%
\pgfsetfillcolor{currentfill}%
\pgfsetlinewidth{1.003750pt}%
\definecolor{currentstroke}{rgb}{0.000000,0.000000,0.000000}%
\pgfsetstrokecolor{currentstroke}%
\pgfsetdash{}{0pt}%
\pgfpathmoveto{\pgfqpoint{1.845859in}{4.227946in}}%
\pgfpathcurveto{\pgfqpoint{1.856909in}{4.227946in}}{\pgfqpoint{1.867508in}{4.232336in}}{\pgfqpoint{1.875322in}{4.240150in}}%
\pgfpathcurveto{\pgfqpoint{1.883136in}{4.247963in}}{\pgfqpoint{1.887526in}{4.258562in}}{\pgfqpoint{1.887526in}{4.269612in}}%
\pgfpathcurveto{\pgfqpoint{1.887526in}{4.280663in}}{\pgfqpoint{1.883136in}{4.291262in}}{\pgfqpoint{1.875322in}{4.299075in}}%
\pgfpathcurveto{\pgfqpoint{1.867508in}{4.306889in}}{\pgfqpoint{1.856909in}{4.311279in}}{\pgfqpoint{1.845859in}{4.311279in}}%
\pgfpathcurveto{\pgfqpoint{1.834809in}{4.311279in}}{\pgfqpoint{1.824210in}{4.306889in}}{\pgfqpoint{1.816396in}{4.299075in}}%
\pgfpathcurveto{\pgfqpoint{1.808583in}{4.291262in}}{\pgfqpoint{1.804193in}{4.280663in}}{\pgfqpoint{1.804193in}{4.269612in}}%
\pgfpathcurveto{\pgfqpoint{1.804193in}{4.258562in}}{\pgfqpoint{1.808583in}{4.247963in}}{\pgfqpoint{1.816396in}{4.240150in}}%
\pgfpathcurveto{\pgfqpoint{1.824210in}{4.232336in}}{\pgfqpoint{1.834809in}{4.227946in}}{\pgfqpoint{1.845859in}{4.227946in}}%
\pgfpathclose%
\pgfusepath{stroke,fill}%
\end{pgfscope}%
\begin{pgfscope}%
\pgfpathrectangle{\pgfqpoint{0.970666in}{4.121437in}}{\pgfqpoint{5.699255in}{2.685432in}}%
\pgfusepath{clip}%
\pgfsetbuttcap%
\pgfsetroundjoin%
\definecolor{currentfill}{rgb}{0.000000,0.000000,0.000000}%
\pgfsetfillcolor{currentfill}%
\pgfsetlinewidth{1.003750pt}%
\definecolor{currentstroke}{rgb}{0.000000,0.000000,0.000000}%
\pgfsetstrokecolor{currentstroke}%
\pgfsetdash{}{0pt}%
\pgfpathmoveto{\pgfqpoint{1.397761in}{5.938163in}}%
\pgfpathcurveto{\pgfqpoint{1.408811in}{5.938163in}}{\pgfqpoint{1.419410in}{5.942553in}}{\pgfqpoint{1.427223in}{5.950366in}}%
\pgfpathcurveto{\pgfqpoint{1.435037in}{5.958180in}}{\pgfqpoint{1.439427in}{5.968779in}}{\pgfqpoint{1.439427in}{5.979829in}}%
\pgfpathcurveto{\pgfqpoint{1.439427in}{5.990879in}}{\pgfqpoint{1.435037in}{6.001478in}}{\pgfqpoint{1.427223in}{6.009292in}}%
\pgfpathcurveto{\pgfqpoint{1.419410in}{6.017106in}}{\pgfqpoint{1.408811in}{6.021496in}}{\pgfqpoint{1.397761in}{6.021496in}}%
\pgfpathcurveto{\pgfqpoint{1.386710in}{6.021496in}}{\pgfqpoint{1.376111in}{6.017106in}}{\pgfqpoint{1.368298in}{6.009292in}}%
\pgfpathcurveto{\pgfqpoint{1.360484in}{6.001478in}}{\pgfqpoint{1.356094in}{5.990879in}}{\pgfqpoint{1.356094in}{5.979829in}}%
\pgfpathcurveto{\pgfqpoint{1.356094in}{5.968779in}}{\pgfqpoint{1.360484in}{5.958180in}}{\pgfqpoint{1.368298in}{5.950366in}}%
\pgfpathcurveto{\pgfqpoint{1.376111in}{5.942553in}}{\pgfqpoint{1.386710in}{5.938163in}}{\pgfqpoint{1.397761in}{5.938163in}}%
\pgfpathclose%
\pgfusepath{stroke,fill}%
\end{pgfscope}%
\begin{pgfscope}%
\pgfpathrectangle{\pgfqpoint{0.970666in}{4.121437in}}{\pgfqpoint{5.699255in}{2.685432in}}%
\pgfusepath{clip}%
\pgfsetbuttcap%
\pgfsetroundjoin%
\definecolor{currentfill}{rgb}{0.000000,0.000000,0.000000}%
\pgfsetfillcolor{currentfill}%
\pgfsetlinewidth{1.003750pt}%
\definecolor{currentstroke}{rgb}{0.000000,0.000000,0.000000}%
\pgfsetstrokecolor{currentstroke}%
\pgfsetdash{}{0pt}%
\pgfpathmoveto{\pgfqpoint{5.710711in}{6.643137in}}%
\pgfpathcurveto{\pgfqpoint{5.721761in}{6.643137in}}{\pgfqpoint{5.732360in}{6.647528in}}{\pgfqpoint{5.740173in}{6.655341in}}%
\pgfpathcurveto{\pgfqpoint{5.747987in}{6.663155in}}{\pgfqpoint{5.752377in}{6.673754in}}{\pgfqpoint{5.752377in}{6.684804in}}%
\pgfpathcurveto{\pgfqpoint{5.752377in}{6.695854in}}{\pgfqpoint{5.747987in}{6.706453in}}{\pgfqpoint{5.740173in}{6.714267in}}%
\pgfpathcurveto{\pgfqpoint{5.732360in}{6.722081in}}{\pgfqpoint{5.721761in}{6.726471in}}{\pgfqpoint{5.710711in}{6.726471in}}%
\pgfpathcurveto{\pgfqpoint{5.699660in}{6.726471in}}{\pgfqpoint{5.689061in}{6.722081in}}{\pgfqpoint{5.681248in}{6.714267in}}%
\pgfpathcurveto{\pgfqpoint{5.673434in}{6.706453in}}{\pgfqpoint{5.669044in}{6.695854in}}{\pgfqpoint{5.669044in}{6.684804in}}%
\pgfpathcurveto{\pgfqpoint{5.669044in}{6.673754in}}{\pgfqpoint{5.673434in}{6.663155in}}{\pgfqpoint{5.681248in}{6.655341in}}%
\pgfpathcurveto{\pgfqpoint{5.689061in}{6.647528in}}{\pgfqpoint{5.699660in}{6.643137in}}{\pgfqpoint{5.710711in}{6.643137in}}%
\pgfpathclose%
\pgfusepath{stroke,fill}%
\end{pgfscope}%
\begin{pgfscope}%
\pgfpathrectangle{\pgfqpoint{0.970666in}{4.121437in}}{\pgfqpoint{5.699255in}{2.685432in}}%
\pgfusepath{clip}%
\pgfsetbuttcap%
\pgfsetroundjoin%
\definecolor{currentfill}{rgb}{0.000000,0.000000,0.000000}%
\pgfsetfillcolor{currentfill}%
\pgfsetlinewidth{1.003750pt}%
\definecolor{currentstroke}{rgb}{0.000000,0.000000,0.000000}%
\pgfsetstrokecolor{currentstroke}%
\pgfsetdash{}{0pt}%
\pgfpathmoveto{\pgfqpoint{3.806291in}{5.115692in}}%
\pgfpathcurveto{\pgfqpoint{3.817341in}{5.115692in}}{\pgfqpoint{3.827940in}{5.120082in}}{\pgfqpoint{3.835754in}{5.127896in}}%
\pgfpathcurveto{\pgfqpoint{3.843568in}{5.135709in}}{\pgfqpoint{3.847958in}{5.146308in}}{\pgfqpoint{3.847958in}{5.157359in}}%
\pgfpathcurveto{\pgfqpoint{3.847958in}{5.168409in}}{\pgfqpoint{3.843568in}{5.179008in}}{\pgfqpoint{3.835754in}{5.186821in}}%
\pgfpathcurveto{\pgfqpoint{3.827940in}{5.194635in}}{\pgfqpoint{3.817341in}{5.199025in}}{\pgfqpoint{3.806291in}{5.199025in}}%
\pgfpathcurveto{\pgfqpoint{3.795241in}{5.199025in}}{\pgfqpoint{3.784642in}{5.194635in}}{\pgfqpoint{3.776828in}{5.186821in}}%
\pgfpathcurveto{\pgfqpoint{3.769015in}{5.179008in}}{\pgfqpoint{3.764624in}{5.168409in}}{\pgfqpoint{3.764624in}{5.157359in}}%
\pgfpathcurveto{\pgfqpoint{3.764624in}{5.146308in}}{\pgfqpoint{3.769015in}{5.135709in}}{\pgfqpoint{3.776828in}{5.127896in}}%
\pgfpathcurveto{\pgfqpoint{3.784642in}{5.120082in}}{\pgfqpoint{3.795241in}{5.115692in}}{\pgfqpoint{3.806291in}{5.115692in}}%
\pgfpathclose%
\pgfusepath{stroke,fill}%
\end{pgfscope}%
\begin{pgfscope}%
\pgfpathrectangle{\pgfqpoint{0.970666in}{4.121437in}}{\pgfqpoint{5.699255in}{2.685432in}}%
\pgfusepath{clip}%
\pgfsetbuttcap%
\pgfsetroundjoin%
\definecolor{currentfill}{rgb}{0.000000,0.000000,0.000000}%
\pgfsetfillcolor{currentfill}%
\pgfsetlinewidth{1.003750pt}%
\definecolor{currentstroke}{rgb}{0.000000,0.000000,0.000000}%
\pgfsetstrokecolor{currentstroke}%
\pgfsetdash{}{0pt}%
\pgfpathmoveto{\pgfqpoint{5.262612in}{4.593488in}}%
\pgfpathcurveto{\pgfqpoint{5.273662in}{4.593488in}}{\pgfqpoint{5.284261in}{4.597879in}}{\pgfqpoint{5.292075in}{4.605692in}}%
\pgfpathcurveto{\pgfqpoint{5.299888in}{4.613506in}}{\pgfqpoint{5.304279in}{4.624105in}}{\pgfqpoint{5.304279in}{4.635155in}}%
\pgfpathcurveto{\pgfqpoint{5.304279in}{4.646205in}}{\pgfqpoint{5.299888in}{4.656804in}}{\pgfqpoint{5.292075in}{4.664618in}}%
\pgfpathcurveto{\pgfqpoint{5.284261in}{4.672431in}}{\pgfqpoint{5.273662in}{4.676822in}}{\pgfqpoint{5.262612in}{4.676822in}}%
\pgfpathcurveto{\pgfqpoint{5.251562in}{4.676822in}}{\pgfqpoint{5.240963in}{4.672431in}}{\pgfqpoint{5.233149in}{4.664618in}}%
\pgfpathcurveto{\pgfqpoint{5.225335in}{4.656804in}}{\pgfqpoint{5.220945in}{4.646205in}}{\pgfqpoint{5.220945in}{4.635155in}}%
\pgfpathcurveto{\pgfqpoint{5.220945in}{4.624105in}}{\pgfqpoint{5.225335in}{4.613506in}}{\pgfqpoint{5.233149in}{4.605692in}}%
\pgfpathcurveto{\pgfqpoint{5.240963in}{4.597879in}}{\pgfqpoint{5.251562in}{4.593488in}}{\pgfqpoint{5.262612in}{4.593488in}}%
\pgfpathclose%
\pgfusepath{stroke,fill}%
\end{pgfscope}%
\begin{pgfscope}%
\pgfpathrectangle{\pgfqpoint{0.970666in}{4.121437in}}{\pgfqpoint{5.699255in}{2.685432in}}%
\pgfusepath{clip}%
\pgfsetbuttcap%
\pgfsetroundjoin%
\definecolor{currentfill}{rgb}{0.000000,0.000000,0.000000}%
\pgfsetfillcolor{currentfill}%
\pgfsetlinewidth{1.003750pt}%
\definecolor{currentstroke}{rgb}{0.000000,0.000000,0.000000}%
\pgfsetstrokecolor{currentstroke}%
\pgfsetdash{}{0pt}%
\pgfpathmoveto{\pgfqpoint{5.514667in}{4.737094in}}%
\pgfpathcurveto{\pgfqpoint{5.525718in}{4.737094in}}{\pgfqpoint{5.536317in}{4.741485in}}{\pgfqpoint{5.544130in}{4.749298in}}%
\pgfpathcurveto{\pgfqpoint{5.551944in}{4.757112in}}{\pgfqpoint{5.556334in}{4.767711in}}{\pgfqpoint{5.556334in}{4.778761in}}%
\pgfpathcurveto{\pgfqpoint{5.556334in}{4.789811in}}{\pgfqpoint{5.551944in}{4.800410in}}{\pgfqpoint{5.544130in}{4.808224in}}%
\pgfpathcurveto{\pgfqpoint{5.536317in}{4.816037in}}{\pgfqpoint{5.525718in}{4.820428in}}{\pgfqpoint{5.514667in}{4.820428in}}%
\pgfpathcurveto{\pgfqpoint{5.503617in}{4.820428in}}{\pgfqpoint{5.493018in}{4.816037in}}{\pgfqpoint{5.485205in}{4.808224in}}%
\pgfpathcurveto{\pgfqpoint{5.477391in}{4.800410in}}{\pgfqpoint{5.473001in}{4.789811in}}{\pgfqpoint{5.473001in}{4.778761in}}%
\pgfpathcurveto{\pgfqpoint{5.473001in}{4.767711in}}{\pgfqpoint{5.477391in}{4.757112in}}{\pgfqpoint{5.485205in}{4.749298in}}%
\pgfpathcurveto{\pgfqpoint{5.493018in}{4.741485in}}{\pgfqpoint{5.503617in}{4.737094in}}{\pgfqpoint{5.514667in}{4.737094in}}%
\pgfpathclose%
\pgfusepath{stroke,fill}%
\end{pgfscope}%
\begin{pgfscope}%
\pgfpathrectangle{\pgfqpoint{0.970666in}{4.121437in}}{\pgfqpoint{5.699255in}{2.685432in}}%
\pgfusepath{clip}%
\pgfsetbuttcap%
\pgfsetroundjoin%
\definecolor{currentfill}{rgb}{0.000000,0.000000,0.000000}%
\pgfsetfillcolor{currentfill}%
\pgfsetlinewidth{1.003750pt}%
\definecolor{currentstroke}{rgb}{0.000000,0.000000,0.000000}%
\pgfsetstrokecolor{currentstroke}%
\pgfsetdash{}{0pt}%
\pgfpathmoveto{\pgfqpoint{4.506445in}{5.024306in}}%
\pgfpathcurveto{\pgfqpoint{4.517495in}{5.024306in}}{\pgfqpoint{4.528094in}{5.028697in}}{\pgfqpoint{4.535908in}{5.036510in}}%
\pgfpathcurveto{\pgfqpoint{4.543722in}{5.044324in}}{\pgfqpoint{4.548112in}{5.054923in}}{\pgfqpoint{4.548112in}{5.065973in}}%
\pgfpathcurveto{\pgfqpoint{4.548112in}{5.077023in}}{\pgfqpoint{4.543722in}{5.087622in}}{\pgfqpoint{4.535908in}{5.095436in}}%
\pgfpathcurveto{\pgfqpoint{4.528094in}{5.103249in}}{\pgfqpoint{4.517495in}{5.107640in}}{\pgfqpoint{4.506445in}{5.107640in}}%
\pgfpathcurveto{\pgfqpoint{4.495395in}{5.107640in}}{\pgfqpoint{4.484796in}{5.103249in}}{\pgfqpoint{4.476983in}{5.095436in}}%
\pgfpathcurveto{\pgfqpoint{4.469169in}{5.087622in}}{\pgfqpoint{4.464779in}{5.077023in}}{\pgfqpoint{4.464779in}{5.065973in}}%
\pgfpathcurveto{\pgfqpoint{4.464779in}{5.054923in}}{\pgfqpoint{4.469169in}{5.044324in}}{\pgfqpoint{4.476983in}{5.036510in}}%
\pgfpathcurveto{\pgfqpoint{4.484796in}{5.028697in}}{\pgfqpoint{4.495395in}{5.024306in}}{\pgfqpoint{4.506445in}{5.024306in}}%
\pgfpathclose%
\pgfusepath{stroke,fill}%
\end{pgfscope}%
\begin{pgfscope}%
\pgfpathrectangle{\pgfqpoint{0.970666in}{4.121437in}}{\pgfqpoint{5.699255in}{2.685432in}}%
\pgfusepath{clip}%
\pgfsetbuttcap%
\pgfsetroundjoin%
\definecolor{currentfill}{rgb}{0.000000,0.000000,0.000000}%
\pgfsetfillcolor{currentfill}%
\pgfsetlinewidth{1.003750pt}%
\definecolor{currentstroke}{rgb}{0.000000,0.000000,0.000000}%
\pgfsetstrokecolor{currentstroke}%
\pgfsetdash{}{0pt}%
\pgfpathmoveto{\pgfqpoint{3.470217in}{4.945976in}}%
\pgfpathcurveto{\pgfqpoint{3.481267in}{4.945976in}}{\pgfqpoint{3.491866in}{4.950366in}}{\pgfqpoint{3.499680in}{4.958180in}}%
\pgfpathcurveto{\pgfqpoint{3.507493in}{4.965993in}}{\pgfqpoint{3.511884in}{4.976592in}}{\pgfqpoint{3.511884in}{4.987642in}}%
\pgfpathcurveto{\pgfqpoint{3.511884in}{4.998692in}}{\pgfqpoint{3.507493in}{5.009292in}}{\pgfqpoint{3.499680in}{5.017105in}}%
\pgfpathcurveto{\pgfqpoint{3.491866in}{5.024919in}}{\pgfqpoint{3.481267in}{5.029309in}}{\pgfqpoint{3.470217in}{5.029309in}}%
\pgfpathcurveto{\pgfqpoint{3.459167in}{5.029309in}}{\pgfqpoint{3.448568in}{5.024919in}}{\pgfqpoint{3.440754in}{5.017105in}}%
\pgfpathcurveto{\pgfqpoint{3.432941in}{5.009292in}}{\pgfqpoint{3.428550in}{4.998692in}}{\pgfqpoint{3.428550in}{4.987642in}}%
\pgfpathcurveto{\pgfqpoint{3.428550in}{4.976592in}}{\pgfqpoint{3.432941in}{4.965993in}}{\pgfqpoint{3.440754in}{4.958180in}}%
\pgfpathcurveto{\pgfqpoint{3.448568in}{4.950366in}}{\pgfqpoint{3.459167in}{4.945976in}}{\pgfqpoint{3.470217in}{4.945976in}}%
\pgfpathclose%
\pgfusepath{stroke,fill}%
\end{pgfscope}%
\begin{pgfscope}%
\pgfpathrectangle{\pgfqpoint{0.970666in}{4.121437in}}{\pgfqpoint{5.699255in}{2.685432in}}%
\pgfusepath{clip}%
\pgfsetbuttcap%
\pgfsetroundjoin%
\definecolor{currentfill}{rgb}{0.000000,0.000000,0.000000}%
\pgfsetfillcolor{currentfill}%
\pgfsetlinewidth{1.003750pt}%
\definecolor{currentstroke}{rgb}{0.000000,0.000000,0.000000}%
\pgfsetstrokecolor{currentstroke}%
\pgfsetdash{}{0pt}%
\pgfpathmoveto{\pgfqpoint{2.377976in}{4.697929in}}%
\pgfpathcurveto{\pgfqpoint{2.389027in}{4.697929in}}{\pgfqpoint{2.399626in}{4.702319in}}{\pgfqpoint{2.407439in}{4.710133in}}%
\pgfpathcurveto{\pgfqpoint{2.415253in}{4.717946in}}{\pgfqpoint{2.419643in}{4.728546in}}{\pgfqpoint{2.419643in}{4.739596in}}%
\pgfpathcurveto{\pgfqpoint{2.419643in}{4.750646in}}{\pgfqpoint{2.415253in}{4.761245in}}{\pgfqpoint{2.407439in}{4.769058in}}%
\pgfpathcurveto{\pgfqpoint{2.399626in}{4.776872in}}{\pgfqpoint{2.389027in}{4.781262in}}{\pgfqpoint{2.377976in}{4.781262in}}%
\pgfpathcurveto{\pgfqpoint{2.366926in}{4.781262in}}{\pgfqpoint{2.356327in}{4.776872in}}{\pgfqpoint{2.348514in}{4.769058in}}%
\pgfpathcurveto{\pgfqpoint{2.340700in}{4.761245in}}{\pgfqpoint{2.336310in}{4.750646in}}{\pgfqpoint{2.336310in}{4.739596in}}%
\pgfpathcurveto{\pgfqpoint{2.336310in}{4.728546in}}{\pgfqpoint{2.340700in}{4.717946in}}{\pgfqpoint{2.348514in}{4.710133in}}%
\pgfpathcurveto{\pgfqpoint{2.356327in}{4.702319in}}{\pgfqpoint{2.366926in}{4.697929in}}{\pgfqpoint{2.377976in}{4.697929in}}%
\pgfpathclose%
\pgfusepath{stroke,fill}%
\end{pgfscope}%
\begin{pgfscope}%
\pgfpathrectangle{\pgfqpoint{0.970666in}{4.121437in}}{\pgfqpoint{5.699255in}{2.685432in}}%
\pgfusepath{clip}%
\pgfsetbuttcap%
\pgfsetroundjoin%
\definecolor{currentfill}{rgb}{0.000000,0.000000,0.000000}%
\pgfsetfillcolor{currentfill}%
\pgfsetlinewidth{1.003750pt}%
\definecolor{currentstroke}{rgb}{0.000000,0.000000,0.000000}%
\pgfsetstrokecolor{currentstroke}%
\pgfsetdash{}{0pt}%
\pgfpathmoveto{\pgfqpoint{2.209939in}{6.133989in}}%
\pgfpathcurveto{\pgfqpoint{2.220990in}{6.133989in}}{\pgfqpoint{2.231589in}{6.138379in}}{\pgfqpoint{2.239402in}{6.146193in}}%
\pgfpathcurveto{\pgfqpoint{2.247216in}{6.154006in}}{\pgfqpoint{2.251606in}{6.164605in}}{\pgfqpoint{2.251606in}{6.175656in}}%
\pgfpathcurveto{\pgfqpoint{2.251606in}{6.186706in}}{\pgfqpoint{2.247216in}{6.197305in}}{\pgfqpoint{2.239402in}{6.205118in}}%
\pgfpathcurveto{\pgfqpoint{2.231589in}{6.212932in}}{\pgfqpoint{2.220990in}{6.217322in}}{\pgfqpoint{2.209939in}{6.217322in}}%
\pgfpathcurveto{\pgfqpoint{2.198889in}{6.217322in}}{\pgfqpoint{2.188290in}{6.212932in}}{\pgfqpoint{2.180477in}{6.205118in}}%
\pgfpathcurveto{\pgfqpoint{2.172663in}{6.197305in}}{\pgfqpoint{2.168273in}{6.186706in}}{\pgfqpoint{2.168273in}{6.175656in}}%
\pgfpathcurveto{\pgfqpoint{2.168273in}{6.164605in}}{\pgfqpoint{2.172663in}{6.154006in}}{\pgfqpoint{2.180477in}{6.146193in}}%
\pgfpathcurveto{\pgfqpoint{2.188290in}{6.138379in}}{\pgfqpoint{2.198889in}{6.133989in}}{\pgfqpoint{2.209939in}{6.133989in}}%
\pgfpathclose%
\pgfusepath{stroke,fill}%
\end{pgfscope}%
\begin{pgfscope}%
\pgfpathrectangle{\pgfqpoint{0.970666in}{4.121437in}}{\pgfqpoint{5.699255in}{2.685432in}}%
\pgfusepath{clip}%
\pgfsetbuttcap%
\pgfsetroundjoin%
\definecolor{currentfill}{rgb}{0.000000,0.000000,0.000000}%
\pgfsetfillcolor{currentfill}%
\pgfsetlinewidth{1.003750pt}%
\definecolor{currentstroke}{rgb}{0.000000,0.000000,0.000000}%
\pgfsetstrokecolor{currentstroke}%
\pgfsetdash{}{0pt}%
\pgfpathmoveto{\pgfqpoint{5.430649in}{5.755391in}}%
\pgfpathcurveto{\pgfqpoint{5.441699in}{5.755391in}}{\pgfqpoint{5.452298in}{5.759782in}}{\pgfqpoint{5.460112in}{5.767595in}}%
\pgfpathcurveto{\pgfqpoint{5.467925in}{5.775409in}}{\pgfqpoint{5.472316in}{5.786008in}}{\pgfqpoint{5.472316in}{5.797058in}}%
\pgfpathcurveto{\pgfqpoint{5.472316in}{5.808108in}}{\pgfqpoint{5.467925in}{5.818707in}}{\pgfqpoint{5.460112in}{5.826521in}}%
\pgfpathcurveto{\pgfqpoint{5.452298in}{5.834334in}}{\pgfqpoint{5.441699in}{5.838725in}}{\pgfqpoint{5.430649in}{5.838725in}}%
\pgfpathcurveto{\pgfqpoint{5.419599in}{5.838725in}}{\pgfqpoint{5.409000in}{5.834334in}}{\pgfqpoint{5.401186in}{5.826521in}}%
\pgfpathcurveto{\pgfqpoint{5.393372in}{5.818707in}}{\pgfqpoint{5.388982in}{5.808108in}}{\pgfqpoint{5.388982in}{5.797058in}}%
\pgfpathcurveto{\pgfqpoint{5.388982in}{5.786008in}}{\pgfqpoint{5.393372in}{5.775409in}}{\pgfqpoint{5.401186in}{5.767595in}}%
\pgfpathcurveto{\pgfqpoint{5.409000in}{5.759782in}}{\pgfqpoint{5.419599in}{5.755391in}}{\pgfqpoint{5.430649in}{5.755391in}}%
\pgfpathclose%
\pgfusepath{stroke,fill}%
\end{pgfscope}%
\begin{pgfscope}%
\pgfpathrectangle{\pgfqpoint{0.970666in}{4.121437in}}{\pgfqpoint{5.699255in}{2.685432in}}%
\pgfusepath{clip}%
\pgfsetbuttcap%
\pgfsetroundjoin%
\definecolor{currentfill}{rgb}{0.000000,0.000000,0.000000}%
\pgfsetfillcolor{currentfill}%
\pgfsetlinewidth{1.003750pt}%
\definecolor{currentstroke}{rgb}{0.000000,0.000000,0.000000}%
\pgfsetstrokecolor{currentstroke}%
\pgfsetdash{}{0pt}%
\pgfpathmoveto{\pgfqpoint{3.694266in}{5.050416in}}%
\pgfpathcurveto{\pgfqpoint{3.705317in}{5.050416in}}{\pgfqpoint{3.715916in}{5.054807in}}{\pgfqpoint{3.723729in}{5.062620in}}%
\pgfpathcurveto{\pgfqpoint{3.731543in}{5.070434in}}{\pgfqpoint{3.735933in}{5.081033in}}{\pgfqpoint{3.735933in}{5.092083in}}%
\pgfpathcurveto{\pgfqpoint{3.735933in}{5.103133in}}{\pgfqpoint{3.731543in}{5.113732in}}{\pgfqpoint{3.723729in}{5.121546in}}%
\pgfpathcurveto{\pgfqpoint{3.715916in}{5.129359in}}{\pgfqpoint{3.705317in}{5.133750in}}{\pgfqpoint{3.694266in}{5.133750in}}%
\pgfpathcurveto{\pgfqpoint{3.683216in}{5.133750in}}{\pgfqpoint{3.672617in}{5.129359in}}{\pgfqpoint{3.664804in}{5.121546in}}%
\pgfpathcurveto{\pgfqpoint{3.656990in}{5.113732in}}{\pgfqpoint{3.652600in}{5.103133in}}{\pgfqpoint{3.652600in}{5.092083in}}%
\pgfpathcurveto{\pgfqpoint{3.652600in}{5.081033in}}{\pgfqpoint{3.656990in}{5.070434in}}{\pgfqpoint{3.664804in}{5.062620in}}%
\pgfpathcurveto{\pgfqpoint{3.672617in}{5.054807in}}{\pgfqpoint{3.683216in}{5.050416in}}{\pgfqpoint{3.694266in}{5.050416in}}%
\pgfpathclose%
\pgfusepath{stroke,fill}%
\end{pgfscope}%
\begin{pgfscope}%
\pgfpathrectangle{\pgfqpoint{0.970666in}{4.121437in}}{\pgfqpoint{5.699255in}{2.685432in}}%
\pgfusepath{clip}%
\pgfsetbuttcap%
\pgfsetroundjoin%
\definecolor{currentfill}{rgb}{0.000000,0.000000,0.000000}%
\pgfsetfillcolor{currentfill}%
\pgfsetlinewidth{1.003750pt}%
\definecolor{currentstroke}{rgb}{0.000000,0.000000,0.000000}%
\pgfsetstrokecolor{currentstroke}%
\pgfsetdash{}{0pt}%
\pgfpathmoveto{\pgfqpoint{1.733835in}{4.201836in}}%
\pgfpathcurveto{\pgfqpoint{1.744885in}{4.201836in}}{\pgfqpoint{1.755484in}{4.206226in}}{\pgfqpoint{1.763297in}{4.214039in}}%
\pgfpathcurveto{\pgfqpoint{1.771111in}{4.221853in}}{\pgfqpoint{1.775501in}{4.232452in}}{\pgfqpoint{1.775501in}{4.243502in}}%
\pgfpathcurveto{\pgfqpoint{1.775501in}{4.254552in}}{\pgfqpoint{1.771111in}{4.265151in}}{\pgfqpoint{1.763297in}{4.272965in}}%
\pgfpathcurveto{\pgfqpoint{1.755484in}{4.280779in}}{\pgfqpoint{1.744885in}{4.285169in}}{\pgfqpoint{1.733835in}{4.285169in}}%
\pgfpathcurveto{\pgfqpoint{1.722784in}{4.285169in}}{\pgfqpoint{1.712185in}{4.280779in}}{\pgfqpoint{1.704372in}{4.272965in}}%
\pgfpathcurveto{\pgfqpoint{1.696558in}{4.265151in}}{\pgfqpoint{1.692168in}{4.254552in}}{\pgfqpoint{1.692168in}{4.243502in}}%
\pgfpathcurveto{\pgfqpoint{1.692168in}{4.232452in}}{\pgfqpoint{1.696558in}{4.221853in}}{\pgfqpoint{1.704372in}{4.214039in}}%
\pgfpathcurveto{\pgfqpoint{1.712185in}{4.206226in}}{\pgfqpoint{1.722784in}{4.201836in}}{\pgfqpoint{1.733835in}{4.201836in}}%
\pgfpathclose%
\pgfusepath{stroke,fill}%
\end{pgfscope}%
\begin{pgfscope}%
\pgfpathrectangle{\pgfqpoint{0.970666in}{4.121437in}}{\pgfqpoint{5.699255in}{2.685432in}}%
\pgfusepath{clip}%
\pgfsetbuttcap%
\pgfsetroundjoin%
\definecolor{currentfill}{rgb}{0.000000,0.000000,0.000000}%
\pgfsetfillcolor{currentfill}%
\pgfsetlinewidth{1.003750pt}%
\definecolor{currentstroke}{rgb}{0.000000,0.000000,0.000000}%
\pgfsetstrokecolor{currentstroke}%
\pgfsetdash{}{0pt}%
\pgfpathmoveto{\pgfqpoint{2.377976in}{6.382036in}}%
\pgfpathcurveto{\pgfqpoint{2.389027in}{6.382036in}}{\pgfqpoint{2.399626in}{6.386426in}}{\pgfqpoint{2.407439in}{6.394240in}}%
\pgfpathcurveto{\pgfqpoint{2.415253in}{6.402053in}}{\pgfqpoint{2.419643in}{6.412652in}}{\pgfqpoint{2.419643in}{6.423702in}}%
\pgfpathcurveto{\pgfqpoint{2.419643in}{6.434752in}}{\pgfqpoint{2.415253in}{6.445351in}}{\pgfqpoint{2.407439in}{6.453165in}}%
\pgfpathcurveto{\pgfqpoint{2.399626in}{6.460979in}}{\pgfqpoint{2.389027in}{6.465369in}}{\pgfqpoint{2.377976in}{6.465369in}}%
\pgfpathcurveto{\pgfqpoint{2.366926in}{6.465369in}}{\pgfqpoint{2.356327in}{6.460979in}}{\pgfqpoint{2.348514in}{6.453165in}}%
\pgfpathcurveto{\pgfqpoint{2.340700in}{6.445351in}}{\pgfqpoint{2.336310in}{6.434752in}}{\pgfqpoint{2.336310in}{6.423702in}}%
\pgfpathcurveto{\pgfqpoint{2.336310in}{6.412652in}}{\pgfqpoint{2.340700in}{6.402053in}}{\pgfqpoint{2.348514in}{6.394240in}}%
\pgfpathcurveto{\pgfqpoint{2.356327in}{6.386426in}}{\pgfqpoint{2.366926in}{6.382036in}}{\pgfqpoint{2.377976in}{6.382036in}}%
\pgfpathclose%
\pgfusepath{stroke,fill}%
\end{pgfscope}%
\begin{pgfscope}%
\pgfpathrectangle{\pgfqpoint{0.970666in}{4.121437in}}{\pgfqpoint{5.699255in}{2.685432in}}%
\pgfusepath{clip}%
\pgfsetbuttcap%
\pgfsetroundjoin%
\definecolor{currentfill}{rgb}{0.000000,0.000000,0.000000}%
\pgfsetfillcolor{currentfill}%
\pgfsetlinewidth{1.003750pt}%
\definecolor{currentstroke}{rgb}{0.000000,0.000000,0.000000}%
\pgfsetstrokecolor{currentstroke}%
\pgfsetdash{}{0pt}%
\pgfpathmoveto{\pgfqpoint{5.234606in}{6.382036in}}%
\pgfpathcurveto{\pgfqpoint{5.245656in}{6.382036in}}{\pgfqpoint{5.256255in}{6.386426in}}{\pgfqpoint{5.264068in}{6.394240in}}%
\pgfpathcurveto{\pgfqpoint{5.271882in}{6.402053in}}{\pgfqpoint{5.276272in}{6.412652in}}{\pgfqpoint{5.276272in}{6.423702in}}%
\pgfpathcurveto{\pgfqpoint{5.276272in}{6.434752in}}{\pgfqpoint{5.271882in}{6.445351in}}{\pgfqpoint{5.264068in}{6.453165in}}%
\pgfpathcurveto{\pgfqpoint{5.256255in}{6.460979in}}{\pgfqpoint{5.245656in}{6.465369in}}{\pgfqpoint{5.234606in}{6.465369in}}%
\pgfpathcurveto{\pgfqpoint{5.223556in}{6.465369in}}{\pgfqpoint{5.212957in}{6.460979in}}{\pgfqpoint{5.205143in}{6.453165in}}%
\pgfpathcurveto{\pgfqpoint{5.197329in}{6.445351in}}{\pgfqpoint{5.192939in}{6.434752in}}{\pgfqpoint{5.192939in}{6.423702in}}%
\pgfpathcurveto{\pgfqpoint{5.192939in}{6.412652in}}{\pgfqpoint{5.197329in}{6.402053in}}{\pgfqpoint{5.205143in}{6.394240in}}%
\pgfpathcurveto{\pgfqpoint{5.212957in}{6.386426in}}{\pgfqpoint{5.223556in}{6.382036in}}{\pgfqpoint{5.234606in}{6.382036in}}%
\pgfpathclose%
\pgfusepath{stroke,fill}%
\end{pgfscope}%
\begin{pgfscope}%
\pgfpathrectangle{\pgfqpoint{0.970666in}{4.121437in}}{\pgfqpoint{5.699255in}{2.685432in}}%
\pgfusepath{clip}%
\pgfsetbuttcap%
\pgfsetroundjoin%
\definecolor{currentfill}{rgb}{0.000000,0.000000,0.000000}%
\pgfsetfillcolor{currentfill}%
\pgfsetlinewidth{1.003750pt}%
\definecolor{currentstroke}{rgb}{0.000000,0.000000,0.000000}%
\pgfsetstrokecolor{currentstroke}%
\pgfsetdash{}{0pt}%
\pgfpathmoveto{\pgfqpoint{3.890310in}{6.251485in}}%
\pgfpathcurveto{\pgfqpoint{3.901360in}{6.251485in}}{\pgfqpoint{3.911959in}{6.255875in}}{\pgfqpoint{3.919772in}{6.263689in}}%
\pgfpathcurveto{\pgfqpoint{3.927586in}{6.271502in}}{\pgfqpoint{3.931976in}{6.282101in}}{\pgfqpoint{3.931976in}{6.293151in}}%
\pgfpathcurveto{\pgfqpoint{3.931976in}{6.304202in}}{\pgfqpoint{3.927586in}{6.314801in}}{\pgfqpoint{3.919772in}{6.322614in}}%
\pgfpathcurveto{\pgfqpoint{3.911959in}{6.330428in}}{\pgfqpoint{3.901360in}{6.334818in}}{\pgfqpoint{3.890310in}{6.334818in}}%
\pgfpathcurveto{\pgfqpoint{3.879259in}{6.334818in}}{\pgfqpoint{3.868660in}{6.330428in}}{\pgfqpoint{3.860847in}{6.322614in}}%
\pgfpathcurveto{\pgfqpoint{3.853033in}{6.314801in}}{\pgfqpoint{3.848643in}{6.304202in}}{\pgfqpoint{3.848643in}{6.293151in}}%
\pgfpathcurveto{\pgfqpoint{3.848643in}{6.282101in}}{\pgfqpoint{3.853033in}{6.271502in}}{\pgfqpoint{3.860847in}{6.263689in}}%
\pgfpathcurveto{\pgfqpoint{3.868660in}{6.255875in}}{\pgfqpoint{3.879259in}{6.251485in}}{\pgfqpoint{3.890310in}{6.251485in}}%
\pgfpathclose%
\pgfusepath{stroke,fill}%
\end{pgfscope}%
\begin{pgfscope}%
\pgfpathrectangle{\pgfqpoint{0.970666in}{4.121437in}}{\pgfqpoint{5.699255in}{2.685432in}}%
\pgfusepath{clip}%
\pgfsetbuttcap%
\pgfsetroundjoin%
\definecolor{currentfill}{rgb}{0.000000,0.000000,0.000000}%
\pgfsetfillcolor{currentfill}%
\pgfsetlinewidth{1.003750pt}%
\definecolor{currentstroke}{rgb}{0.000000,0.000000,0.000000}%
\pgfsetstrokecolor{currentstroke}%
\pgfsetdash{}{0pt}%
\pgfpathmoveto{\pgfqpoint{1.733835in}{6.355925in}}%
\pgfpathcurveto{\pgfqpoint{1.744885in}{6.355925in}}{\pgfqpoint{1.755484in}{6.360316in}}{\pgfqpoint{1.763297in}{6.368129in}}%
\pgfpathcurveto{\pgfqpoint{1.771111in}{6.375943in}}{\pgfqpoint{1.775501in}{6.386542in}}{\pgfqpoint{1.775501in}{6.397592in}}%
\pgfpathcurveto{\pgfqpoint{1.775501in}{6.408642in}}{\pgfqpoint{1.771111in}{6.419241in}}{\pgfqpoint{1.763297in}{6.427055in}}%
\pgfpathcurveto{\pgfqpoint{1.755484in}{6.434869in}}{\pgfqpoint{1.744885in}{6.439259in}}{\pgfqpoint{1.733835in}{6.439259in}}%
\pgfpathcurveto{\pgfqpoint{1.722784in}{6.439259in}}{\pgfqpoint{1.712185in}{6.434869in}}{\pgfqpoint{1.704372in}{6.427055in}}%
\pgfpathcurveto{\pgfqpoint{1.696558in}{6.419241in}}{\pgfqpoint{1.692168in}{6.408642in}}{\pgfqpoint{1.692168in}{6.397592in}}%
\pgfpathcurveto{\pgfqpoint{1.692168in}{6.386542in}}{\pgfqpoint{1.696558in}{6.375943in}}{\pgfqpoint{1.704372in}{6.368129in}}%
\pgfpathcurveto{\pgfqpoint{1.712185in}{6.360316in}}{\pgfqpoint{1.722784in}{6.355925in}}{\pgfqpoint{1.733835in}{6.355925in}}%
\pgfpathclose%
\pgfusepath{stroke,fill}%
\end{pgfscope}%
\begin{pgfscope}%
\pgfpathrectangle{\pgfqpoint{0.970666in}{4.121437in}}{\pgfqpoint{5.699255in}{2.685432in}}%
\pgfusepath{clip}%
\pgfsetbuttcap%
\pgfsetroundjoin%
\definecolor{currentfill}{rgb}{0.000000,0.000000,0.000000}%
\pgfsetfillcolor{currentfill}%
\pgfsetlinewidth{1.003750pt}%
\definecolor{currentstroke}{rgb}{0.000000,0.000000,0.000000}%
\pgfsetstrokecolor{currentstroke}%
\pgfsetdash{}{0pt}%
\pgfpathmoveto{\pgfqpoint{1.229724in}{6.290650in}}%
\pgfpathcurveto{\pgfqpoint{1.240774in}{6.290650in}}{\pgfqpoint{1.251373in}{6.295040in}}{\pgfqpoint{1.259186in}{6.302854in}}%
\pgfpathcurveto{\pgfqpoint{1.267000in}{6.310668in}}{\pgfqpoint{1.271390in}{6.321267in}}{\pgfqpoint{1.271390in}{6.332317in}}%
\pgfpathcurveto{\pgfqpoint{1.271390in}{6.343367in}}{\pgfqpoint{1.267000in}{6.353966in}}{\pgfqpoint{1.259186in}{6.361779in}}%
\pgfpathcurveto{\pgfqpoint{1.251373in}{6.369593in}}{\pgfqpoint{1.240774in}{6.373983in}}{\pgfqpoint{1.229724in}{6.373983in}}%
\pgfpathcurveto{\pgfqpoint{1.218673in}{6.373983in}}{\pgfqpoint{1.208074in}{6.369593in}}{\pgfqpoint{1.200261in}{6.361779in}}%
\pgfpathcurveto{\pgfqpoint{1.192447in}{6.353966in}}{\pgfqpoint{1.188057in}{6.343367in}}{\pgfqpoint{1.188057in}{6.332317in}}%
\pgfpathcurveto{\pgfqpoint{1.188057in}{6.321267in}}{\pgfqpoint{1.192447in}{6.310668in}}{\pgfqpoint{1.200261in}{6.302854in}}%
\pgfpathcurveto{\pgfqpoint{1.208074in}{6.295040in}}{\pgfqpoint{1.218673in}{6.290650in}}{\pgfqpoint{1.229724in}{6.290650in}}%
\pgfpathclose%
\pgfusepath{stroke,fill}%
\end{pgfscope}%
\begin{pgfscope}%
\pgfpathrectangle{\pgfqpoint{0.970666in}{4.121437in}}{\pgfqpoint{5.699255in}{2.685432in}}%
\pgfusepath{clip}%
\pgfsetbuttcap%
\pgfsetroundjoin%
\definecolor{currentfill}{rgb}{0.000000,0.000000,0.000000}%
\pgfsetfillcolor{currentfill}%
\pgfsetlinewidth{1.003750pt}%
\definecolor{currentstroke}{rgb}{0.000000,0.000000,0.000000}%
\pgfsetstrokecolor{currentstroke}%
\pgfsetdash{}{0pt}%
\pgfpathmoveto{\pgfqpoint{1.901872in}{5.011251in}}%
\pgfpathcurveto{\pgfqpoint{1.912922in}{5.011251in}}{\pgfqpoint{1.923521in}{5.015641in}}{\pgfqpoint{1.931334in}{5.023455in}}%
\pgfpathcurveto{\pgfqpoint{1.939148in}{5.031269in}}{\pgfqpoint{1.943538in}{5.041868in}}{\pgfqpoint{1.943538in}{5.052918in}}%
\pgfpathcurveto{\pgfqpoint{1.943538in}{5.063968in}}{\pgfqpoint{1.939148in}{5.074567in}}{\pgfqpoint{1.931334in}{5.082381in}}%
\pgfpathcurveto{\pgfqpoint{1.923521in}{5.090194in}}{\pgfqpoint{1.912922in}{5.094584in}}{\pgfqpoint{1.901872in}{5.094584in}}%
\pgfpathcurveto{\pgfqpoint{1.890821in}{5.094584in}}{\pgfqpoint{1.880222in}{5.090194in}}{\pgfqpoint{1.872409in}{5.082381in}}%
\pgfpathcurveto{\pgfqpoint{1.864595in}{5.074567in}}{\pgfqpoint{1.860205in}{5.063968in}}{\pgfqpoint{1.860205in}{5.052918in}}%
\pgfpathcurveto{\pgfqpoint{1.860205in}{5.041868in}}{\pgfqpoint{1.864595in}{5.031269in}}{\pgfqpoint{1.872409in}{5.023455in}}%
\pgfpathcurveto{\pgfqpoint{1.880222in}{5.015641in}}{\pgfqpoint{1.890821in}{5.011251in}}{\pgfqpoint{1.901872in}{5.011251in}}%
\pgfpathclose%
\pgfusepath{stroke,fill}%
\end{pgfscope}%
\begin{pgfscope}%
\pgfpathrectangle{\pgfqpoint{0.970666in}{4.121437in}}{\pgfqpoint{5.699255in}{2.685432in}}%
\pgfusepath{clip}%
\pgfsetbuttcap%
\pgfsetroundjoin%
\definecolor{currentfill}{rgb}{0.000000,0.000000,0.000000}%
\pgfsetfillcolor{currentfill}%
\pgfsetlinewidth{1.003750pt}%
\definecolor{currentstroke}{rgb}{0.000000,0.000000,0.000000}%
\pgfsetstrokecolor{currentstroke}%
\pgfsetdash{}{0pt}%
\pgfpathmoveto{\pgfqpoint{3.078131in}{4.358497in}}%
\pgfpathcurveto{\pgfqpoint{3.089181in}{4.358497in}}{\pgfqpoint{3.099780in}{4.362887in}}{\pgfqpoint{3.107593in}{4.370701in}}%
\pgfpathcurveto{\pgfqpoint{3.115407in}{4.378514in}}{\pgfqpoint{3.119797in}{4.389113in}}{\pgfqpoint{3.119797in}{4.400163in}}%
\pgfpathcurveto{\pgfqpoint{3.119797in}{4.411213in}}{\pgfqpoint{3.115407in}{4.421812in}}{\pgfqpoint{3.107593in}{4.429626in}}%
\pgfpathcurveto{\pgfqpoint{3.099780in}{4.437440in}}{\pgfqpoint{3.089181in}{4.441830in}}{\pgfqpoint{3.078131in}{4.441830in}}%
\pgfpathcurveto{\pgfqpoint{3.067081in}{4.441830in}}{\pgfqpoint{3.056482in}{4.437440in}}{\pgfqpoint{3.048668in}{4.429626in}}%
\pgfpathcurveto{\pgfqpoint{3.040854in}{4.421812in}}{\pgfqpoint{3.036464in}{4.411213in}}{\pgfqpoint{3.036464in}{4.400163in}}%
\pgfpathcurveto{\pgfqpoint{3.036464in}{4.389113in}}{\pgfqpoint{3.040854in}{4.378514in}}{\pgfqpoint{3.048668in}{4.370701in}}%
\pgfpathcurveto{\pgfqpoint{3.056482in}{4.362887in}}{\pgfqpoint{3.067081in}{4.358497in}}{\pgfqpoint{3.078131in}{4.358497in}}%
\pgfpathclose%
\pgfusepath{stroke,fill}%
\end{pgfscope}%
\begin{pgfscope}%
\pgfpathrectangle{\pgfqpoint{0.970666in}{4.121437in}}{\pgfqpoint{5.699255in}{2.685432in}}%
\pgfusepath{clip}%
\pgfsetbuttcap%
\pgfsetroundjoin%
\definecolor{currentfill}{rgb}{0.000000,0.000000,0.000000}%
\pgfsetfillcolor{currentfill}%
\pgfsetlinewidth{1.003750pt}%
\definecolor{currentstroke}{rgb}{0.000000,0.000000,0.000000}%
\pgfsetstrokecolor{currentstroke}%
\pgfsetdash{}{0pt}%
\pgfpathmoveto{\pgfqpoint{5.990772in}{4.397662in}}%
\pgfpathcurveto{\pgfqpoint{6.001822in}{4.397662in}}{\pgfqpoint{6.012421in}{4.402052in}}{\pgfqpoint{6.020235in}{4.409866in}}%
\pgfpathcurveto{\pgfqpoint{6.028049in}{4.417679in}}{\pgfqpoint{6.032439in}{4.428278in}}{\pgfqpoint{6.032439in}{4.439329in}}%
\pgfpathcurveto{\pgfqpoint{6.032439in}{4.450379in}}{\pgfqpoint{6.028049in}{4.460978in}}{\pgfqpoint{6.020235in}{4.468791in}}%
\pgfpathcurveto{\pgfqpoint{6.012421in}{4.476605in}}{\pgfqpoint{6.001822in}{4.480995in}}{\pgfqpoint{5.990772in}{4.480995in}}%
\pgfpathcurveto{\pgfqpoint{5.979722in}{4.480995in}}{\pgfqpoint{5.969123in}{4.476605in}}{\pgfqpoint{5.961309in}{4.468791in}}%
\pgfpathcurveto{\pgfqpoint{5.953496in}{4.460978in}}{\pgfqpoint{5.949106in}{4.450379in}}{\pgfqpoint{5.949106in}{4.439329in}}%
\pgfpathcurveto{\pgfqpoint{5.949106in}{4.428278in}}{\pgfqpoint{5.953496in}{4.417679in}}{\pgfqpoint{5.961309in}{4.409866in}}%
\pgfpathcurveto{\pgfqpoint{5.969123in}{4.402052in}}{\pgfqpoint{5.979722in}{4.397662in}}{\pgfqpoint{5.990772in}{4.397662in}}%
\pgfpathclose%
\pgfusepath{stroke,fill}%
\end{pgfscope}%
\begin{pgfscope}%
\pgfpathrectangle{\pgfqpoint{0.970666in}{4.121437in}}{\pgfqpoint{5.699255in}{2.685432in}}%
\pgfusepath{clip}%
\pgfsetbuttcap%
\pgfsetroundjoin%
\definecolor{currentfill}{rgb}{0.000000,0.000000,0.000000}%
\pgfsetfillcolor{currentfill}%
\pgfsetlinewidth{1.003750pt}%
\definecolor{currentstroke}{rgb}{0.000000,0.000000,0.000000}%
\pgfsetstrokecolor{currentstroke}%
\pgfsetdash{}{0pt}%
\pgfpathmoveto{\pgfqpoint{3.414205in}{5.898997in}}%
\pgfpathcurveto{\pgfqpoint{3.425255in}{5.898997in}}{\pgfqpoint{3.435854in}{5.903388in}}{\pgfqpoint{3.443668in}{5.911201in}}%
\pgfpathcurveto{\pgfqpoint{3.451481in}{5.919015in}}{\pgfqpoint{3.455871in}{5.929614in}}{\pgfqpoint{3.455871in}{5.940664in}}%
\pgfpathcurveto{\pgfqpoint{3.455871in}{5.951714in}}{\pgfqpoint{3.451481in}{5.962313in}}{\pgfqpoint{3.443668in}{5.970127in}}%
\pgfpathcurveto{\pgfqpoint{3.435854in}{5.977940in}}{\pgfqpoint{3.425255in}{5.982331in}}{\pgfqpoint{3.414205in}{5.982331in}}%
\pgfpathcurveto{\pgfqpoint{3.403155in}{5.982331in}}{\pgfqpoint{3.392556in}{5.977940in}}{\pgfqpoint{3.384742in}{5.970127in}}%
\pgfpathcurveto{\pgfqpoint{3.376928in}{5.962313in}}{\pgfqpoint{3.372538in}{5.951714in}}{\pgfqpoint{3.372538in}{5.940664in}}%
\pgfpathcurveto{\pgfqpoint{3.372538in}{5.929614in}}{\pgfqpoint{3.376928in}{5.919015in}}{\pgfqpoint{3.384742in}{5.911201in}}%
\pgfpathcurveto{\pgfqpoint{3.392556in}{5.903388in}}{\pgfqpoint{3.403155in}{5.898997in}}{\pgfqpoint{3.414205in}{5.898997in}}%
\pgfpathclose%
\pgfusepath{stroke,fill}%
\end{pgfscope}%
\begin{pgfscope}%
\pgfpathrectangle{\pgfqpoint{0.970666in}{4.121437in}}{\pgfqpoint{5.699255in}{2.685432in}}%
\pgfusepath{clip}%
\pgfsetbuttcap%
\pgfsetroundjoin%
\definecolor{currentfill}{rgb}{0.000000,0.000000,0.000000}%
\pgfsetfillcolor{currentfill}%
\pgfsetlinewidth{1.003750pt}%
\definecolor{currentstroke}{rgb}{0.000000,0.000000,0.000000}%
\pgfsetstrokecolor{currentstroke}%
\pgfsetdash{}{0pt}%
\pgfpathmoveto{\pgfqpoint{1.677822in}{5.938163in}}%
\pgfpathcurveto{\pgfqpoint{1.688872in}{5.938163in}}{\pgfqpoint{1.699471in}{5.942553in}}{\pgfqpoint{1.707285in}{5.950366in}}%
\pgfpathcurveto{\pgfqpoint{1.715099in}{5.958180in}}{\pgfqpoint{1.719489in}{5.968779in}}{\pgfqpoint{1.719489in}{5.979829in}}%
\pgfpathcurveto{\pgfqpoint{1.719489in}{5.990879in}}{\pgfqpoint{1.715099in}{6.001478in}}{\pgfqpoint{1.707285in}{6.009292in}}%
\pgfpathcurveto{\pgfqpoint{1.699471in}{6.017106in}}{\pgfqpoint{1.688872in}{6.021496in}}{\pgfqpoint{1.677822in}{6.021496in}}%
\pgfpathcurveto{\pgfqpoint{1.666772in}{6.021496in}}{\pgfqpoint{1.656173in}{6.017106in}}{\pgfqpoint{1.648359in}{6.009292in}}%
\pgfpathcurveto{\pgfqpoint{1.640546in}{6.001478in}}{\pgfqpoint{1.636156in}{5.990879in}}{\pgfqpoint{1.636156in}{5.979829in}}%
\pgfpathcurveto{\pgfqpoint{1.636156in}{5.968779in}}{\pgfqpoint{1.640546in}{5.958180in}}{\pgfqpoint{1.648359in}{5.950366in}}%
\pgfpathcurveto{\pgfqpoint{1.656173in}{5.942553in}}{\pgfqpoint{1.666772in}{5.938163in}}{\pgfqpoint{1.677822in}{5.938163in}}%
\pgfpathclose%
\pgfusepath{stroke,fill}%
\end{pgfscope}%
\begin{pgfscope}%
\pgfpathrectangle{\pgfqpoint{0.970666in}{4.121437in}}{\pgfqpoint{5.699255in}{2.685432in}}%
\pgfusepath{clip}%
\pgfsetbuttcap%
\pgfsetroundjoin%
\definecolor{currentfill}{rgb}{0.000000,0.000000,0.000000}%
\pgfsetfillcolor{currentfill}%
\pgfsetlinewidth{1.003750pt}%
\definecolor{currentstroke}{rgb}{0.000000,0.000000,0.000000}%
\pgfsetstrokecolor{currentstroke}%
\pgfsetdash{}{0pt}%
\pgfpathmoveto{\pgfqpoint{2.405983in}{4.449882in}}%
\pgfpathcurveto{\pgfqpoint{2.417033in}{4.449882in}}{\pgfqpoint{2.427632in}{4.454273in}}{\pgfqpoint{2.435445in}{4.462086in}}%
\pgfpathcurveto{\pgfqpoint{2.443259in}{4.469900in}}{\pgfqpoint{2.447649in}{4.480499in}}{\pgfqpoint{2.447649in}{4.491549in}}%
\pgfpathcurveto{\pgfqpoint{2.447649in}{4.502599in}}{\pgfqpoint{2.443259in}{4.513198in}}{\pgfqpoint{2.435445in}{4.521012in}}%
\pgfpathcurveto{\pgfqpoint{2.427632in}{4.528825in}}{\pgfqpoint{2.417033in}{4.533216in}}{\pgfqpoint{2.405983in}{4.533216in}}%
\pgfpathcurveto{\pgfqpoint{2.394933in}{4.533216in}}{\pgfqpoint{2.384333in}{4.528825in}}{\pgfqpoint{2.376520in}{4.521012in}}%
\pgfpathcurveto{\pgfqpoint{2.368706in}{4.513198in}}{\pgfqpoint{2.364316in}{4.502599in}}{\pgfqpoint{2.364316in}{4.491549in}}%
\pgfpathcurveto{\pgfqpoint{2.364316in}{4.480499in}}{\pgfqpoint{2.368706in}{4.469900in}}{\pgfqpoint{2.376520in}{4.462086in}}%
\pgfpathcurveto{\pgfqpoint{2.384333in}{4.454273in}}{\pgfqpoint{2.394933in}{4.449882in}}{\pgfqpoint{2.405983in}{4.449882in}}%
\pgfpathclose%
\pgfusepath{stroke,fill}%
\end{pgfscope}%
\begin{pgfscope}%
\pgfpathrectangle{\pgfqpoint{0.970666in}{4.121437in}}{\pgfqpoint{5.699255in}{2.685432in}}%
\pgfusepath{clip}%
\pgfsetbuttcap%
\pgfsetroundjoin%
\definecolor{currentfill}{rgb}{0.000000,0.000000,0.000000}%
\pgfsetfillcolor{currentfill}%
\pgfsetlinewidth{1.003750pt}%
\definecolor{currentstroke}{rgb}{0.000000,0.000000,0.000000}%
\pgfsetstrokecolor{currentstroke}%
\pgfsetdash{}{0pt}%
\pgfpathmoveto{\pgfqpoint{5.486661in}{5.977328in}}%
\pgfpathcurveto{\pgfqpoint{5.497711in}{5.977328in}}{\pgfqpoint{5.508310in}{5.981718in}}{\pgfqpoint{5.516124in}{5.989532in}}%
\pgfpathcurveto{\pgfqpoint{5.523938in}{5.997345in}}{\pgfqpoint{5.528328in}{6.007944in}}{\pgfqpoint{5.528328in}{6.018995in}}%
\pgfpathcurveto{\pgfqpoint{5.528328in}{6.030045in}}{\pgfqpoint{5.523938in}{6.040644in}}{\pgfqpoint{5.516124in}{6.048457in}}%
\pgfpathcurveto{\pgfqpoint{5.508310in}{6.056271in}}{\pgfqpoint{5.497711in}{6.060661in}}{\pgfqpoint{5.486661in}{6.060661in}}%
\pgfpathcurveto{\pgfqpoint{5.475611in}{6.060661in}}{\pgfqpoint{5.465012in}{6.056271in}}{\pgfqpoint{5.457198in}{6.048457in}}%
\pgfpathcurveto{\pgfqpoint{5.449385in}{6.040644in}}{\pgfqpoint{5.444995in}{6.030045in}}{\pgfqpoint{5.444995in}{6.018995in}}%
\pgfpathcurveto{\pgfqpoint{5.444995in}{6.007944in}}{\pgfqpoint{5.449385in}{5.997345in}}{\pgfqpoint{5.457198in}{5.989532in}}%
\pgfpathcurveto{\pgfqpoint{5.465012in}{5.981718in}}{\pgfqpoint{5.475611in}{5.977328in}}{\pgfqpoint{5.486661in}{5.977328in}}%
\pgfpathclose%
\pgfusepath{stroke,fill}%
\end{pgfscope}%
\begin{pgfscope}%
\pgfpathrectangle{\pgfqpoint{0.970666in}{4.121437in}}{\pgfqpoint{5.699255in}{2.685432in}}%
\pgfusepath{clip}%
\pgfsetbuttcap%
\pgfsetroundjoin%
\definecolor{currentfill}{rgb}{0.000000,0.000000,0.000000}%
\pgfsetfillcolor{currentfill}%
\pgfsetlinewidth{1.003750pt}%
\definecolor{currentstroke}{rgb}{0.000000,0.000000,0.000000}%
\pgfsetstrokecolor{currentstroke}%
\pgfsetdash{}{0pt}%
\pgfpathmoveto{\pgfqpoint{5.906754in}{5.885942in}}%
\pgfpathcurveto{\pgfqpoint{5.917804in}{5.885942in}}{\pgfqpoint{5.928403in}{5.890332in}}{\pgfqpoint{5.936217in}{5.898146in}}%
\pgfpathcurveto{\pgfqpoint{5.944030in}{5.905960in}}{\pgfqpoint{5.948420in}{5.916559in}}{\pgfqpoint{5.948420in}{5.927609in}}%
\pgfpathcurveto{\pgfqpoint{5.948420in}{5.938659in}}{\pgfqpoint{5.944030in}{5.949258in}}{\pgfqpoint{5.936217in}{5.957072in}}%
\pgfpathcurveto{\pgfqpoint{5.928403in}{5.964885in}}{\pgfqpoint{5.917804in}{5.969276in}}{\pgfqpoint{5.906754in}{5.969276in}}%
\pgfpathcurveto{\pgfqpoint{5.895704in}{5.969276in}}{\pgfqpoint{5.885105in}{5.964885in}}{\pgfqpoint{5.877291in}{5.957072in}}%
\pgfpathcurveto{\pgfqpoint{5.869477in}{5.949258in}}{\pgfqpoint{5.865087in}{5.938659in}}{\pgfqpoint{5.865087in}{5.927609in}}%
\pgfpathcurveto{\pgfqpoint{5.865087in}{5.916559in}}{\pgfqpoint{5.869477in}{5.905960in}}{\pgfqpoint{5.877291in}{5.898146in}}%
\pgfpathcurveto{\pgfqpoint{5.885105in}{5.890332in}}{\pgfqpoint{5.895704in}{5.885942in}}{\pgfqpoint{5.906754in}{5.885942in}}%
\pgfpathclose%
\pgfusepath{stroke,fill}%
\end{pgfscope}%
\begin{pgfscope}%
\pgfpathrectangle{\pgfqpoint{0.970666in}{4.121437in}}{\pgfqpoint{5.699255in}{2.685432in}}%
\pgfusepath{clip}%
\pgfsetbuttcap%
\pgfsetroundjoin%
\definecolor{currentfill}{rgb}{0.000000,0.000000,1.000000}%
\pgfsetfillcolor{currentfill}%
\pgfsetlinewidth{1.003750pt}%
\definecolor{currentstroke}{rgb}{0.000000,0.000000,1.000000}%
\pgfsetstrokecolor{currentstroke}%
\pgfsetdash{}{0pt}%
\pgfpathmoveto{\pgfqpoint{5.402643in}{4.972086in}}%
\pgfpathcurveto{\pgfqpoint{5.413693in}{4.972086in}}{\pgfqpoint{5.424292in}{4.976476in}}{\pgfqpoint{5.432106in}{4.984290in}}%
\pgfpathcurveto{\pgfqpoint{5.439919in}{4.992103in}}{\pgfqpoint{5.444309in}{5.002702in}}{\pgfqpoint{5.444309in}{5.013753in}}%
\pgfpathcurveto{\pgfqpoint{5.444309in}{5.024803in}}{\pgfqpoint{5.439919in}{5.035402in}}{\pgfqpoint{5.432106in}{5.043215in}}%
\pgfpathcurveto{\pgfqpoint{5.424292in}{5.051029in}}{\pgfqpoint{5.413693in}{5.055419in}}{\pgfqpoint{5.402643in}{5.055419in}}%
\pgfpathcurveto{\pgfqpoint{5.391593in}{5.055419in}}{\pgfqpoint{5.380994in}{5.051029in}}{\pgfqpoint{5.373180in}{5.043215in}}%
\pgfpathcurveto{\pgfqpoint{5.365366in}{5.035402in}}{\pgfqpoint{5.360976in}{5.024803in}}{\pgfqpoint{5.360976in}{5.013753in}}%
\pgfpathcurveto{\pgfqpoint{5.360976in}{5.002702in}}{\pgfqpoint{5.365366in}{4.992103in}}{\pgfqpoint{5.373180in}{4.984290in}}%
\pgfpathcurveto{\pgfqpoint{5.380994in}{4.976476in}}{\pgfqpoint{5.391593in}{4.972086in}}{\pgfqpoint{5.402643in}{4.972086in}}%
\pgfpathclose%
\pgfusepath{stroke,fill}%
\end{pgfscope}%
\begin{pgfscope}%
\pgfsetbuttcap%
\pgfsetroundjoin%
\definecolor{currentfill}{rgb}{0.000000,0.000000,0.000000}%
\pgfsetfillcolor{currentfill}%
\pgfsetlinewidth{0.803000pt}%
\definecolor{currentstroke}{rgb}{0.000000,0.000000,0.000000}%
\pgfsetstrokecolor{currentstroke}%
\pgfsetdash{}{0pt}%
\pgfsys@defobject{currentmarker}{\pgfqpoint{0.000000in}{-0.048611in}}{\pgfqpoint{0.000000in}{0.000000in}}{%
\pgfpathmoveto{\pgfqpoint{0.000000in}{0.000000in}}%
\pgfpathlineto{\pgfqpoint{0.000000in}{-0.048611in}}%
\pgfusepath{stroke,fill}%
}%
\begin{pgfscope}%
\pgfsys@transformshift{1.145705in}{4.121437in}%
\pgfsys@useobject{currentmarker}{}%
\end{pgfscope}%
\end{pgfscope}%
\begin{pgfscope}%
\definecolor{textcolor}{rgb}{0.000000,0.000000,0.000000}%
\pgfsetstrokecolor{textcolor}%
\pgfsetfillcolor{textcolor}%
\pgftext[x=1.145705in,y=4.024215in,,top]{\color{textcolor}\rmfamily\fontsize{10.000000}{12.000000}\selectfont \(\displaystyle 0\)}%
\end{pgfscope}%
\begin{pgfscope}%
\pgfsetbuttcap%
\pgfsetroundjoin%
\definecolor{currentfill}{rgb}{0.000000,0.000000,0.000000}%
\pgfsetfillcolor{currentfill}%
\pgfsetlinewidth{0.803000pt}%
\definecolor{currentstroke}{rgb}{0.000000,0.000000,0.000000}%
\pgfsetstrokecolor{currentstroke}%
\pgfsetdash{}{0pt}%
\pgfsys@defobject{currentmarker}{\pgfqpoint{0.000000in}{-0.048611in}}{\pgfqpoint{0.000000in}{0.000000in}}{%
\pgfpathmoveto{\pgfqpoint{0.000000in}{0.000000in}}%
\pgfpathlineto{\pgfqpoint{0.000000in}{-0.048611in}}%
\pgfusepath{stroke,fill}%
}%
\begin{pgfscope}%
\pgfsys@transformshift{1.845859in}{4.121437in}%
\pgfsys@useobject{currentmarker}{}%
\end{pgfscope}%
\end{pgfscope}%
\begin{pgfscope}%
\definecolor{textcolor}{rgb}{0.000000,0.000000,0.000000}%
\pgfsetstrokecolor{textcolor}%
\pgfsetfillcolor{textcolor}%
\pgftext[x=1.845859in,y=4.024215in,,top]{\color{textcolor}\rmfamily\fontsize{10.000000}{12.000000}\selectfont \(\displaystyle 25\)}%
\end{pgfscope}%
\begin{pgfscope}%
\pgfsetbuttcap%
\pgfsetroundjoin%
\definecolor{currentfill}{rgb}{0.000000,0.000000,0.000000}%
\pgfsetfillcolor{currentfill}%
\pgfsetlinewidth{0.803000pt}%
\definecolor{currentstroke}{rgb}{0.000000,0.000000,0.000000}%
\pgfsetstrokecolor{currentstroke}%
\pgfsetdash{}{0pt}%
\pgfsys@defobject{currentmarker}{\pgfqpoint{0.000000in}{-0.048611in}}{\pgfqpoint{0.000000in}{0.000000in}}{%
\pgfpathmoveto{\pgfqpoint{0.000000in}{0.000000in}}%
\pgfpathlineto{\pgfqpoint{0.000000in}{-0.048611in}}%
\pgfusepath{stroke,fill}%
}%
\begin{pgfscope}%
\pgfsys@transformshift{2.546013in}{4.121437in}%
\pgfsys@useobject{currentmarker}{}%
\end{pgfscope}%
\end{pgfscope}%
\begin{pgfscope}%
\definecolor{textcolor}{rgb}{0.000000,0.000000,0.000000}%
\pgfsetstrokecolor{textcolor}%
\pgfsetfillcolor{textcolor}%
\pgftext[x=2.546013in,y=4.024215in,,top]{\color{textcolor}\rmfamily\fontsize{10.000000}{12.000000}\selectfont \(\displaystyle 50\)}%
\end{pgfscope}%
\begin{pgfscope}%
\pgfsetbuttcap%
\pgfsetroundjoin%
\definecolor{currentfill}{rgb}{0.000000,0.000000,0.000000}%
\pgfsetfillcolor{currentfill}%
\pgfsetlinewidth{0.803000pt}%
\definecolor{currentstroke}{rgb}{0.000000,0.000000,0.000000}%
\pgfsetstrokecolor{currentstroke}%
\pgfsetdash{}{0pt}%
\pgfsys@defobject{currentmarker}{\pgfqpoint{0.000000in}{-0.048611in}}{\pgfqpoint{0.000000in}{0.000000in}}{%
\pgfpathmoveto{\pgfqpoint{0.000000in}{0.000000in}}%
\pgfpathlineto{\pgfqpoint{0.000000in}{-0.048611in}}%
\pgfusepath{stroke,fill}%
}%
\begin{pgfscope}%
\pgfsys@transformshift{3.246168in}{4.121437in}%
\pgfsys@useobject{currentmarker}{}%
\end{pgfscope}%
\end{pgfscope}%
\begin{pgfscope}%
\definecolor{textcolor}{rgb}{0.000000,0.000000,0.000000}%
\pgfsetstrokecolor{textcolor}%
\pgfsetfillcolor{textcolor}%
\pgftext[x=3.246168in,y=4.024215in,,top]{\color{textcolor}\rmfamily\fontsize{10.000000}{12.000000}\selectfont \(\displaystyle 75\)}%
\end{pgfscope}%
\begin{pgfscope}%
\pgfsetbuttcap%
\pgfsetroundjoin%
\definecolor{currentfill}{rgb}{0.000000,0.000000,0.000000}%
\pgfsetfillcolor{currentfill}%
\pgfsetlinewidth{0.803000pt}%
\definecolor{currentstroke}{rgb}{0.000000,0.000000,0.000000}%
\pgfsetstrokecolor{currentstroke}%
\pgfsetdash{}{0pt}%
\pgfsys@defobject{currentmarker}{\pgfqpoint{0.000000in}{-0.048611in}}{\pgfqpoint{0.000000in}{0.000000in}}{%
\pgfpathmoveto{\pgfqpoint{0.000000in}{0.000000in}}%
\pgfpathlineto{\pgfqpoint{0.000000in}{-0.048611in}}%
\pgfusepath{stroke,fill}%
}%
\begin{pgfscope}%
\pgfsys@transformshift{3.946322in}{4.121437in}%
\pgfsys@useobject{currentmarker}{}%
\end{pgfscope}%
\end{pgfscope}%
\begin{pgfscope}%
\definecolor{textcolor}{rgb}{0.000000,0.000000,0.000000}%
\pgfsetstrokecolor{textcolor}%
\pgfsetfillcolor{textcolor}%
\pgftext[x=3.946322in,y=4.024215in,,top]{\color{textcolor}\rmfamily\fontsize{10.000000}{12.000000}\selectfont \(\displaystyle 100\)}%
\end{pgfscope}%
\begin{pgfscope}%
\pgfsetbuttcap%
\pgfsetroundjoin%
\definecolor{currentfill}{rgb}{0.000000,0.000000,0.000000}%
\pgfsetfillcolor{currentfill}%
\pgfsetlinewidth{0.803000pt}%
\definecolor{currentstroke}{rgb}{0.000000,0.000000,0.000000}%
\pgfsetstrokecolor{currentstroke}%
\pgfsetdash{}{0pt}%
\pgfsys@defobject{currentmarker}{\pgfqpoint{0.000000in}{-0.048611in}}{\pgfqpoint{0.000000in}{0.000000in}}{%
\pgfpathmoveto{\pgfqpoint{0.000000in}{0.000000in}}%
\pgfpathlineto{\pgfqpoint{0.000000in}{-0.048611in}}%
\pgfusepath{stroke,fill}%
}%
\begin{pgfscope}%
\pgfsys@transformshift{4.646476in}{4.121437in}%
\pgfsys@useobject{currentmarker}{}%
\end{pgfscope}%
\end{pgfscope}%
\begin{pgfscope}%
\definecolor{textcolor}{rgb}{0.000000,0.000000,0.000000}%
\pgfsetstrokecolor{textcolor}%
\pgfsetfillcolor{textcolor}%
\pgftext[x=4.646476in,y=4.024215in,,top]{\color{textcolor}\rmfamily\fontsize{10.000000}{12.000000}\selectfont \(\displaystyle 125\)}%
\end{pgfscope}%
\begin{pgfscope}%
\pgfsetbuttcap%
\pgfsetroundjoin%
\definecolor{currentfill}{rgb}{0.000000,0.000000,0.000000}%
\pgfsetfillcolor{currentfill}%
\pgfsetlinewidth{0.803000pt}%
\definecolor{currentstroke}{rgb}{0.000000,0.000000,0.000000}%
\pgfsetstrokecolor{currentstroke}%
\pgfsetdash{}{0pt}%
\pgfsys@defobject{currentmarker}{\pgfqpoint{0.000000in}{-0.048611in}}{\pgfqpoint{0.000000in}{0.000000in}}{%
\pgfpathmoveto{\pgfqpoint{0.000000in}{0.000000in}}%
\pgfpathlineto{\pgfqpoint{0.000000in}{-0.048611in}}%
\pgfusepath{stroke,fill}%
}%
\begin{pgfscope}%
\pgfsys@transformshift{5.346630in}{4.121437in}%
\pgfsys@useobject{currentmarker}{}%
\end{pgfscope}%
\end{pgfscope}%
\begin{pgfscope}%
\definecolor{textcolor}{rgb}{0.000000,0.000000,0.000000}%
\pgfsetstrokecolor{textcolor}%
\pgfsetfillcolor{textcolor}%
\pgftext[x=5.346630in,y=4.024215in,,top]{\color{textcolor}\rmfamily\fontsize{10.000000}{12.000000}\selectfont \(\displaystyle 150\)}%
\end{pgfscope}%
\begin{pgfscope}%
\pgfsetbuttcap%
\pgfsetroundjoin%
\definecolor{currentfill}{rgb}{0.000000,0.000000,0.000000}%
\pgfsetfillcolor{currentfill}%
\pgfsetlinewidth{0.803000pt}%
\definecolor{currentstroke}{rgb}{0.000000,0.000000,0.000000}%
\pgfsetstrokecolor{currentstroke}%
\pgfsetdash{}{0pt}%
\pgfsys@defobject{currentmarker}{\pgfqpoint{0.000000in}{-0.048611in}}{\pgfqpoint{0.000000in}{0.000000in}}{%
\pgfpathmoveto{\pgfqpoint{0.000000in}{0.000000in}}%
\pgfpathlineto{\pgfqpoint{0.000000in}{-0.048611in}}%
\pgfusepath{stroke,fill}%
}%
\begin{pgfscope}%
\pgfsys@transformshift{6.046785in}{4.121437in}%
\pgfsys@useobject{currentmarker}{}%
\end{pgfscope}%
\end{pgfscope}%
\begin{pgfscope}%
\definecolor{textcolor}{rgb}{0.000000,0.000000,0.000000}%
\pgfsetstrokecolor{textcolor}%
\pgfsetfillcolor{textcolor}%
\pgftext[x=6.046785in,y=4.024215in,,top]{\color{textcolor}\rmfamily\fontsize{10.000000}{12.000000}\selectfont \(\displaystyle 175\)}%
\end{pgfscope}%
\begin{pgfscope}%
\pgfsetbuttcap%
\pgfsetroundjoin%
\definecolor{currentfill}{rgb}{0.000000,0.000000,0.000000}%
\pgfsetfillcolor{currentfill}%
\pgfsetlinewidth{0.803000pt}%
\definecolor{currentstroke}{rgb}{0.000000,0.000000,0.000000}%
\pgfsetstrokecolor{currentstroke}%
\pgfsetdash{}{0pt}%
\pgfsys@defobject{currentmarker}{\pgfqpoint{-0.048611in}{0.000000in}}{\pgfqpoint{0.000000in}{0.000000in}}{%
\pgfpathmoveto{\pgfqpoint{0.000000in}{0.000000in}}%
\pgfpathlineto{\pgfqpoint{-0.048611in}{0.000000in}}%
\pgfusepath{stroke,fill}%
}%
\begin{pgfscope}%
\pgfsys@transformshift{0.970666in}{4.243502in}%
\pgfsys@useobject{currentmarker}{}%
\end{pgfscope}%
\end{pgfscope}%
\begin{pgfscope}%
\definecolor{textcolor}{rgb}{0.000000,0.000000,0.000000}%
\pgfsetstrokecolor{textcolor}%
\pgfsetfillcolor{textcolor}%
\pgftext[x=0.804000in, y=4.190741in, left, base]{\color{textcolor}\rmfamily\fontsize{10.000000}{12.000000}\selectfont \(\displaystyle 0\)}%
\end{pgfscope}%
\begin{pgfscope}%
\pgfsetbuttcap%
\pgfsetroundjoin%
\definecolor{currentfill}{rgb}{0.000000,0.000000,0.000000}%
\pgfsetfillcolor{currentfill}%
\pgfsetlinewidth{0.803000pt}%
\definecolor{currentstroke}{rgb}{0.000000,0.000000,0.000000}%
\pgfsetstrokecolor{currentstroke}%
\pgfsetdash{}{0pt}%
\pgfsys@defobject{currentmarker}{\pgfqpoint{-0.048611in}{0.000000in}}{\pgfqpoint{0.000000in}{0.000000in}}{%
\pgfpathmoveto{\pgfqpoint{0.000000in}{0.000000in}}%
\pgfpathlineto{\pgfqpoint{-0.048611in}{0.000000in}}%
\pgfusepath{stroke,fill}%
}%
\begin{pgfscope}%
\pgfsys@transformshift{0.970666in}{4.569879in}%
\pgfsys@useobject{currentmarker}{}%
\end{pgfscope}%
\end{pgfscope}%
\begin{pgfscope}%
\definecolor{textcolor}{rgb}{0.000000,0.000000,0.000000}%
\pgfsetstrokecolor{textcolor}%
\pgfsetfillcolor{textcolor}%
\pgftext[x=0.734555in, y=4.517118in, left, base]{\color{textcolor}\rmfamily\fontsize{10.000000}{12.000000}\selectfont \(\displaystyle 25\)}%
\end{pgfscope}%
\begin{pgfscope}%
\pgfsetbuttcap%
\pgfsetroundjoin%
\definecolor{currentfill}{rgb}{0.000000,0.000000,0.000000}%
\pgfsetfillcolor{currentfill}%
\pgfsetlinewidth{0.803000pt}%
\definecolor{currentstroke}{rgb}{0.000000,0.000000,0.000000}%
\pgfsetstrokecolor{currentstroke}%
\pgfsetdash{}{0pt}%
\pgfsys@defobject{currentmarker}{\pgfqpoint{-0.048611in}{0.000000in}}{\pgfqpoint{0.000000in}{0.000000in}}{%
\pgfpathmoveto{\pgfqpoint{0.000000in}{0.000000in}}%
\pgfpathlineto{\pgfqpoint{-0.048611in}{0.000000in}}%
\pgfusepath{stroke,fill}%
}%
\begin{pgfscope}%
\pgfsys@transformshift{0.970666in}{4.896257in}%
\pgfsys@useobject{currentmarker}{}%
\end{pgfscope}%
\end{pgfscope}%
\begin{pgfscope}%
\definecolor{textcolor}{rgb}{0.000000,0.000000,0.000000}%
\pgfsetstrokecolor{textcolor}%
\pgfsetfillcolor{textcolor}%
\pgftext[x=0.734555in, y=4.843495in, left, base]{\color{textcolor}\rmfamily\fontsize{10.000000}{12.000000}\selectfont \(\displaystyle 50\)}%
\end{pgfscope}%
\begin{pgfscope}%
\pgfsetbuttcap%
\pgfsetroundjoin%
\definecolor{currentfill}{rgb}{0.000000,0.000000,0.000000}%
\pgfsetfillcolor{currentfill}%
\pgfsetlinewidth{0.803000pt}%
\definecolor{currentstroke}{rgb}{0.000000,0.000000,0.000000}%
\pgfsetstrokecolor{currentstroke}%
\pgfsetdash{}{0pt}%
\pgfsys@defobject{currentmarker}{\pgfqpoint{-0.048611in}{0.000000in}}{\pgfqpoint{0.000000in}{0.000000in}}{%
\pgfpathmoveto{\pgfqpoint{0.000000in}{0.000000in}}%
\pgfpathlineto{\pgfqpoint{-0.048611in}{0.000000in}}%
\pgfusepath{stroke,fill}%
}%
\begin{pgfscope}%
\pgfsys@transformshift{0.970666in}{5.222634in}%
\pgfsys@useobject{currentmarker}{}%
\end{pgfscope}%
\end{pgfscope}%
\begin{pgfscope}%
\definecolor{textcolor}{rgb}{0.000000,0.000000,0.000000}%
\pgfsetstrokecolor{textcolor}%
\pgfsetfillcolor{textcolor}%
\pgftext[x=0.734555in, y=5.169872in, left, base]{\color{textcolor}\rmfamily\fontsize{10.000000}{12.000000}\selectfont \(\displaystyle 75\)}%
\end{pgfscope}%
\begin{pgfscope}%
\pgfsetbuttcap%
\pgfsetroundjoin%
\definecolor{currentfill}{rgb}{0.000000,0.000000,0.000000}%
\pgfsetfillcolor{currentfill}%
\pgfsetlinewidth{0.803000pt}%
\definecolor{currentstroke}{rgb}{0.000000,0.000000,0.000000}%
\pgfsetstrokecolor{currentstroke}%
\pgfsetdash{}{0pt}%
\pgfsys@defobject{currentmarker}{\pgfqpoint{-0.048611in}{0.000000in}}{\pgfqpoint{0.000000in}{0.000000in}}{%
\pgfpathmoveto{\pgfqpoint{0.000000in}{0.000000in}}%
\pgfpathlineto{\pgfqpoint{-0.048611in}{0.000000in}}%
\pgfusepath{stroke,fill}%
}%
\begin{pgfscope}%
\pgfsys@transformshift{0.970666in}{5.549011in}%
\pgfsys@useobject{currentmarker}{}%
\end{pgfscope}%
\end{pgfscope}%
\begin{pgfscope}%
\definecolor{textcolor}{rgb}{0.000000,0.000000,0.000000}%
\pgfsetstrokecolor{textcolor}%
\pgfsetfillcolor{textcolor}%
\pgftext[x=0.665110in, y=5.496250in, left, base]{\color{textcolor}\rmfamily\fontsize{10.000000}{12.000000}\selectfont \(\displaystyle 100\)}%
\end{pgfscope}%
\begin{pgfscope}%
\pgfsetbuttcap%
\pgfsetroundjoin%
\definecolor{currentfill}{rgb}{0.000000,0.000000,0.000000}%
\pgfsetfillcolor{currentfill}%
\pgfsetlinewidth{0.803000pt}%
\definecolor{currentstroke}{rgb}{0.000000,0.000000,0.000000}%
\pgfsetstrokecolor{currentstroke}%
\pgfsetdash{}{0pt}%
\pgfsys@defobject{currentmarker}{\pgfqpoint{-0.048611in}{0.000000in}}{\pgfqpoint{0.000000in}{0.000000in}}{%
\pgfpathmoveto{\pgfqpoint{0.000000in}{0.000000in}}%
\pgfpathlineto{\pgfqpoint{-0.048611in}{0.000000in}}%
\pgfusepath{stroke,fill}%
}%
\begin{pgfscope}%
\pgfsys@transformshift{0.970666in}{5.875389in}%
\pgfsys@useobject{currentmarker}{}%
\end{pgfscope}%
\end{pgfscope}%
\begin{pgfscope}%
\definecolor{textcolor}{rgb}{0.000000,0.000000,0.000000}%
\pgfsetstrokecolor{textcolor}%
\pgfsetfillcolor{textcolor}%
\pgftext[x=0.665110in, y=5.822627in, left, base]{\color{textcolor}\rmfamily\fontsize{10.000000}{12.000000}\selectfont \(\displaystyle 125\)}%
\end{pgfscope}%
\begin{pgfscope}%
\pgfsetbuttcap%
\pgfsetroundjoin%
\definecolor{currentfill}{rgb}{0.000000,0.000000,0.000000}%
\pgfsetfillcolor{currentfill}%
\pgfsetlinewidth{0.803000pt}%
\definecolor{currentstroke}{rgb}{0.000000,0.000000,0.000000}%
\pgfsetstrokecolor{currentstroke}%
\pgfsetdash{}{0pt}%
\pgfsys@defobject{currentmarker}{\pgfqpoint{-0.048611in}{0.000000in}}{\pgfqpoint{0.000000in}{0.000000in}}{%
\pgfpathmoveto{\pgfqpoint{0.000000in}{0.000000in}}%
\pgfpathlineto{\pgfqpoint{-0.048611in}{0.000000in}}%
\pgfusepath{stroke,fill}%
}%
\begin{pgfscope}%
\pgfsys@transformshift{0.970666in}{6.201766in}%
\pgfsys@useobject{currentmarker}{}%
\end{pgfscope}%
\end{pgfscope}%
\begin{pgfscope}%
\definecolor{textcolor}{rgb}{0.000000,0.000000,0.000000}%
\pgfsetstrokecolor{textcolor}%
\pgfsetfillcolor{textcolor}%
\pgftext[x=0.665110in, y=6.149004in, left, base]{\color{textcolor}\rmfamily\fontsize{10.000000}{12.000000}\selectfont \(\displaystyle 150\)}%
\end{pgfscope}%
\begin{pgfscope}%
\pgfsetbuttcap%
\pgfsetroundjoin%
\definecolor{currentfill}{rgb}{0.000000,0.000000,0.000000}%
\pgfsetfillcolor{currentfill}%
\pgfsetlinewidth{0.803000pt}%
\definecolor{currentstroke}{rgb}{0.000000,0.000000,0.000000}%
\pgfsetstrokecolor{currentstroke}%
\pgfsetdash{}{0pt}%
\pgfsys@defobject{currentmarker}{\pgfqpoint{-0.048611in}{0.000000in}}{\pgfqpoint{0.000000in}{0.000000in}}{%
\pgfpathmoveto{\pgfqpoint{0.000000in}{0.000000in}}%
\pgfpathlineto{\pgfqpoint{-0.048611in}{0.000000in}}%
\pgfusepath{stroke,fill}%
}%
\begin{pgfscope}%
\pgfsys@transformshift{0.970666in}{6.528143in}%
\pgfsys@useobject{currentmarker}{}%
\end{pgfscope}%
\end{pgfscope}%
\begin{pgfscope}%
\definecolor{textcolor}{rgb}{0.000000,0.000000,0.000000}%
\pgfsetstrokecolor{textcolor}%
\pgfsetfillcolor{textcolor}%
\pgftext[x=0.665110in, y=6.475382in, left, base]{\color{textcolor}\rmfamily\fontsize{10.000000}{12.000000}\selectfont \(\displaystyle 175\)}%
\end{pgfscope}%
\begin{pgfscope}%
\pgfsetrectcap%
\pgfsetmiterjoin%
\pgfsetlinewidth{0.803000pt}%
\definecolor{currentstroke}{rgb}{0.000000,0.000000,0.000000}%
\pgfsetstrokecolor{currentstroke}%
\pgfsetdash{}{0pt}%
\pgfpathmoveto{\pgfqpoint{0.970666in}{4.121437in}}%
\pgfpathlineto{\pgfqpoint{0.970666in}{6.806869in}}%
\pgfusepath{stroke}%
\end{pgfscope}%
\begin{pgfscope}%
\pgfsetrectcap%
\pgfsetmiterjoin%
\pgfsetlinewidth{0.803000pt}%
\definecolor{currentstroke}{rgb}{0.000000,0.000000,0.000000}%
\pgfsetstrokecolor{currentstroke}%
\pgfsetdash{}{0pt}%
\pgfpathmoveto{\pgfqpoint{6.669922in}{4.121437in}}%
\pgfpathlineto{\pgfqpoint{6.669922in}{6.806869in}}%
\pgfusepath{stroke}%
\end{pgfscope}%
\begin{pgfscope}%
\pgfsetrectcap%
\pgfsetmiterjoin%
\pgfsetlinewidth{0.803000pt}%
\definecolor{currentstroke}{rgb}{0.000000,0.000000,0.000000}%
\pgfsetstrokecolor{currentstroke}%
\pgfsetdash{}{0pt}%
\pgfpathmoveto{\pgfqpoint{0.970666in}{4.121437in}}%
\pgfpathlineto{\pgfqpoint{6.669922in}{4.121437in}}%
\pgfusepath{stroke}%
\end{pgfscope}%
\begin{pgfscope}%
\pgfsetrectcap%
\pgfsetmiterjoin%
\pgfsetlinewidth{0.803000pt}%
\definecolor{currentstroke}{rgb}{0.000000,0.000000,0.000000}%
\pgfsetstrokecolor{currentstroke}%
\pgfsetdash{}{0pt}%
\pgfpathmoveto{\pgfqpoint{0.970666in}{6.806869in}}%
\pgfpathlineto{\pgfqpoint{6.669922in}{6.806869in}}%
\pgfusepath{stroke}%
\end{pgfscope}%
\begin{pgfscope}%
\definecolor{textcolor}{rgb}{0.000000,0.000000,0.000000}%
\pgfsetstrokecolor{textcolor}%
\pgfsetfillcolor{textcolor}%
\pgftext[x=3.820294in,y=6.890203in,,base]{\color{textcolor}\rmfamily\fontsize{12.000000}{14.400000}\selectfont swap 7989.5}%
\end{pgfscope}%
\begin{pgfscope}%
\pgfsetbuttcap%
\pgfsetmiterjoin%
\definecolor{currentfill}{rgb}{1.000000,1.000000,1.000000}%
\pgfsetfillcolor{currentfill}%
\pgfsetlinewidth{0.000000pt}%
\definecolor{currentstroke}{rgb}{0.000000,0.000000,0.000000}%
\pgfsetstrokecolor{currentstroke}%
\pgfsetstrokeopacity{0.000000}%
\pgfsetdash{}{0pt}%
\pgfpathmoveto{\pgfqpoint{7.640588in}{4.121437in}}%
\pgfpathlineto{\pgfqpoint{13.339844in}{4.121437in}}%
\pgfpathlineto{\pgfqpoint{13.339844in}{6.806869in}}%
\pgfpathlineto{\pgfqpoint{7.640588in}{6.806869in}}%
\pgfpathclose%
\pgfusepath{fill}%
\end{pgfscope}%
\begin{pgfscope}%
\pgfpathrectangle{\pgfqpoint{7.640588in}{4.121437in}}{\pgfqpoint{5.699255in}{2.685432in}}%
\pgfusepath{clip}%
\pgfsetrectcap%
\pgfsetroundjoin%
\pgfsetlinewidth{1.505625pt}%
\definecolor{currentstroke}{rgb}{0.000000,0.000000,0.000000}%
\pgfsetstrokecolor{currentstroke}%
\pgfsetdash{}{0pt}%
\pgfpathmoveto{\pgfqpoint{11.932534in}{4.635155in}}%
\pgfpathlineto{\pgfqpoint{11.988546in}{4.582935in}}%
\pgfpathlineto{\pgfqpoint{12.660694in}{4.439329in}}%
\pgfpathlineto{\pgfqpoint{12.296614in}{4.635155in}}%
\pgfpathlineto{\pgfqpoint{12.184589in}{4.778761in}}%
\pgfpathlineto{\pgfqpoint{12.884744in}{5.222634in}}%
\pgfpathlineto{\pgfqpoint{12.996768in}{5.823168in}}%
\pgfpathlineto{\pgfqpoint{13.080787in}{5.979829in}}%
\pgfpathlineto{\pgfqpoint{12.576676in}{5.927609in}}%
\pgfpathlineto{\pgfqpoint{12.240602in}{6.058160in}}%
\pgfpathlineto{\pgfqpoint{12.156583in}{6.018995in}}%
\pgfpathlineto{\pgfqpoint{11.988546in}{5.888444in}}%
\pgfpathlineto{\pgfqpoint{12.100571in}{5.797058in}}%
\pgfpathlineto{\pgfqpoint{12.464651in}{5.562066in}}%
\pgfpathlineto{\pgfqpoint{12.072565in}{5.013753in}}%
\pgfpathlineto{\pgfqpoint{11.652472in}{5.065973in}}%
\pgfpathlineto{\pgfqpoint{11.400417in}{4.987642in}}%
\pgfpathlineto{\pgfqpoint{11.176367in}{5.065973in}}%
\pgfpathlineto{\pgfqpoint{11.232380in}{5.274854in}}%
\pgfpathlineto{\pgfqpoint{10.812287in}{5.118193in}}%
\pgfpathlineto{\pgfqpoint{10.728268in}{5.105138in}}%
\pgfpathlineto{\pgfqpoint{10.476213in}{5.418460in}}%
\pgfpathlineto{\pgfqpoint{10.980324in}{5.849278in}}%
\pgfpathlineto{\pgfqpoint{10.896306in}{5.914554in}}%
\pgfpathlineto{\pgfqpoint{10.616244in}{6.005939in}}%
\pgfpathlineto{\pgfqpoint{11.904528in}{6.423702in}}%
\pgfpathlineto{\pgfqpoint{11.988546in}{6.397592in}}%
\pgfpathlineto{\pgfqpoint{12.576676in}{6.358427in}}%
\pgfpathlineto{\pgfqpoint{12.660694in}{6.267041in}}%
\pgfpathlineto{\pgfqpoint{12.772719in}{6.410647in}}%
\pgfpathlineto{\pgfqpoint{12.912750in}{6.671749in}}%
\pgfpathlineto{\pgfqpoint{12.380632in}{6.684804in}}%
\pgfpathlineto{\pgfqpoint{11.652472in}{6.502033in}}%
\pgfpathlineto{\pgfqpoint{11.512441in}{6.554253in}}%
\pgfpathlineto{\pgfqpoint{11.372410in}{6.488978in}}%
\pgfpathlineto{\pgfqpoint{10.616244in}{6.632584in}}%
\pgfpathlineto{\pgfqpoint{9.832071in}{6.475923in}}%
\pgfpathlineto{\pgfqpoint{9.131917in}{6.671749in}}%
\pgfpathlineto{\pgfqpoint{9.047898in}{6.423702in}}%
\pgfpathlineto{\pgfqpoint{8.515781in}{6.280096in}}%
\pgfpathlineto{\pgfqpoint{8.403756in}{6.397592in}}%
\pgfpathlineto{\pgfqpoint{8.039676in}{6.410647in}}%
\pgfpathlineto{\pgfqpoint{7.899645in}{6.332317in}}%
\pgfpathlineto{\pgfqpoint{7.955658in}{6.267041in}}%
\pgfpathlineto{\pgfqpoint{8.067682in}{5.979829in}}%
\pgfpathlineto{\pgfqpoint{8.347744in}{5.979829in}}%
\pgfpathlineto{\pgfqpoint{8.879861in}{6.175656in}}%
\pgfpathlineto{\pgfqpoint{9.047898in}{6.253986in}}%
\pgfpathlineto{\pgfqpoint{9.271948in}{6.306206in}}%
\pgfpathlineto{\pgfqpoint{9.748053in}{6.162601in}}%
\pgfpathlineto{\pgfqpoint{10.196151in}{6.384537in}}%
\pgfpathlineto{\pgfqpoint{10.224157in}{6.449812in}}%
\pgfpathlineto{\pgfqpoint{10.560231in}{6.293151in}}%
\pgfpathlineto{\pgfqpoint{10.084127in}{5.940664in}}%
\pgfpathlineto{\pgfqpoint{10.112133in}{5.810113in}}%
\pgfpathlineto{\pgfqpoint{9.580016in}{5.810113in}}%
\pgfpathlineto{\pgfqpoint{9.580016in}{5.940664in}}%
\pgfpathlineto{\pgfqpoint{8.823849in}{6.032050in}}%
\pgfpathlineto{\pgfqpoint{8.655812in}{5.862333in}}%
\pgfpathlineto{\pgfqpoint{9.075905in}{5.588177in}}%
\pgfpathlineto{\pgfqpoint{9.271948in}{5.562066in}}%
\pgfpathlineto{\pgfqpoint{9.355966in}{5.640397in}}%
\pgfpathlineto{\pgfqpoint{9.383972in}{5.666507in}}%
\pgfpathlineto{\pgfqpoint{9.860077in}{5.470681in}}%
\pgfpathlineto{\pgfqpoint{9.608022in}{5.405405in}}%
\pgfpathlineto{\pgfqpoint{9.580016in}{5.340130in}}%
\pgfpathlineto{\pgfqpoint{8.487775in}{5.405405in}}%
\pgfpathlineto{\pgfqpoint{8.543787in}{5.209579in}}%
\pgfpathlineto{\pgfqpoint{8.571793in}{5.052918in}}%
\pgfpathlineto{\pgfqpoint{8.543787in}{4.948477in}}%
\pgfpathlineto{\pgfqpoint{8.039676in}{4.948477in}}%
\pgfpathlineto{\pgfqpoint{8.179707in}{4.909312in}}%
\pgfpathlineto{\pgfqpoint{8.571793in}{4.830981in}}%
\pgfpathlineto{\pgfqpoint{9.327960in}{4.857091in}}%
\pgfpathlineto{\pgfqpoint{9.299954in}{4.830981in}}%
\pgfpathlineto{\pgfqpoint{9.047898in}{4.739596in}}%
\pgfpathlineto{\pgfqpoint{9.075905in}{4.648210in}}%
\pgfpathlineto{\pgfqpoint{8.599800in}{4.543769in}}%
\pgfpathlineto{\pgfqpoint{8.291732in}{4.739596in}}%
\pgfpathlineto{\pgfqpoint{8.263726in}{4.387108in}}%
\pgfpathlineto{\pgfqpoint{8.403756in}{4.243502in}}%
\pgfpathlineto{\pgfqpoint{8.515781in}{4.269612in}}%
\pgfpathlineto{\pgfqpoint{9.075905in}{4.491549in}}%
\pgfpathlineto{\pgfqpoint{9.748053in}{4.400163in}}%
\pgfpathlineto{\pgfqpoint{10.140139in}{4.987642in}}%
\pgfpathlineto{\pgfqpoint{10.364188in}{5.092083in}}%
\pgfpathlineto{\pgfqpoint{10.476213in}{5.157359in}}%
\pgfpathlineto{\pgfqpoint{10.560231in}{4.452384in}}%
\pgfpathlineto{\pgfqpoint{10.700262in}{4.387108in}}%
\pgfpathlineto{\pgfqpoint{10.756275in}{4.439329in}}%
\pgfpathlineto{\pgfqpoint{11.260386in}{4.478494in}}%
\pgfpathlineto{\pgfqpoint{11.568454in}{4.478494in}}%
\pgfpathlineto{\pgfqpoint{11.568454in}{4.491549in}}%
\pgfpathlineto{\pgfqpoint{11.904528in}{4.530714in}}%
\pgfpathlineto{\pgfqpoint{11.932534in}{4.543769in}}%
\pgfpathlineto{\pgfqpoint{11.932534in}{4.635155in}}%
\pgfusepath{stroke}%
\end{pgfscope}%
\begin{pgfscope}%
\pgfpathrectangle{\pgfqpoint{7.640588in}{4.121437in}}{\pgfqpoint{5.699255in}{2.685432in}}%
\pgfusepath{clip}%
\pgfsetbuttcap%
\pgfsetroundjoin%
\definecolor{currentfill}{rgb}{1.000000,0.000000,0.000000}%
\pgfsetfillcolor{currentfill}%
\pgfsetlinewidth{1.003750pt}%
\definecolor{currentstroke}{rgb}{1.000000,0.000000,0.000000}%
\pgfsetstrokecolor{currentstroke}%
\pgfsetdash{}{0pt}%
\pgfpathmoveto{\pgfqpoint{11.932534in}{4.593488in}}%
\pgfpathcurveto{\pgfqpoint{11.943584in}{4.593488in}}{\pgfqpoint{11.954183in}{4.597879in}}{\pgfqpoint{11.961997in}{4.605692in}}%
\pgfpathcurveto{\pgfqpoint{11.969810in}{4.613506in}}{\pgfqpoint{11.974200in}{4.624105in}}{\pgfqpoint{11.974200in}{4.635155in}}%
\pgfpathcurveto{\pgfqpoint{11.974200in}{4.646205in}}{\pgfqpoint{11.969810in}{4.656804in}}{\pgfqpoint{11.961997in}{4.664618in}}%
\pgfpathcurveto{\pgfqpoint{11.954183in}{4.672431in}}{\pgfqpoint{11.943584in}{4.676822in}}{\pgfqpoint{11.932534in}{4.676822in}}%
\pgfpathcurveto{\pgfqpoint{11.921484in}{4.676822in}}{\pgfqpoint{11.910885in}{4.672431in}}{\pgfqpoint{11.903071in}{4.664618in}}%
\pgfpathcurveto{\pgfqpoint{11.895257in}{4.656804in}}{\pgfqpoint{11.890867in}{4.646205in}}{\pgfqpoint{11.890867in}{4.635155in}}%
\pgfpathcurveto{\pgfqpoint{11.890867in}{4.624105in}}{\pgfqpoint{11.895257in}{4.613506in}}{\pgfqpoint{11.903071in}{4.605692in}}%
\pgfpathcurveto{\pgfqpoint{11.910885in}{4.597879in}}{\pgfqpoint{11.921484in}{4.593488in}}{\pgfqpoint{11.932534in}{4.593488in}}%
\pgfpathclose%
\pgfusepath{stroke,fill}%
\end{pgfscope}%
\begin{pgfscope}%
\pgfpathrectangle{\pgfqpoint{7.640588in}{4.121437in}}{\pgfqpoint{5.699255in}{2.685432in}}%
\pgfusepath{clip}%
\pgfsetbuttcap%
\pgfsetroundjoin%
\definecolor{currentfill}{rgb}{0.750000,0.750000,0.000000}%
\pgfsetfillcolor{currentfill}%
\pgfsetlinewidth{1.003750pt}%
\definecolor{currentstroke}{rgb}{0.750000,0.750000,0.000000}%
\pgfsetstrokecolor{currentstroke}%
\pgfsetdash{}{0pt}%
\pgfpathmoveto{\pgfqpoint{11.988546in}{4.541268in}}%
\pgfpathcurveto{\pgfqpoint{11.999596in}{4.541268in}}{\pgfqpoint{12.010195in}{4.545658in}}{\pgfqpoint{12.018009in}{4.553472in}}%
\pgfpathcurveto{\pgfqpoint{12.025823in}{4.561285in}}{\pgfqpoint{12.030213in}{4.571884in}}{\pgfqpoint{12.030213in}{4.582935in}}%
\pgfpathcurveto{\pgfqpoint{12.030213in}{4.593985in}}{\pgfqpoint{12.025823in}{4.604584in}}{\pgfqpoint{12.018009in}{4.612397in}}%
\pgfpathcurveto{\pgfqpoint{12.010195in}{4.620211in}}{\pgfqpoint{11.999596in}{4.624601in}}{\pgfqpoint{11.988546in}{4.624601in}}%
\pgfpathcurveto{\pgfqpoint{11.977496in}{4.624601in}}{\pgfqpoint{11.966897in}{4.620211in}}{\pgfqpoint{11.959083in}{4.612397in}}%
\pgfpathcurveto{\pgfqpoint{11.951270in}{4.604584in}}{\pgfqpoint{11.946879in}{4.593985in}}{\pgfqpoint{11.946879in}{4.582935in}}%
\pgfpathcurveto{\pgfqpoint{11.946879in}{4.571884in}}{\pgfqpoint{11.951270in}{4.561285in}}{\pgfqpoint{11.959083in}{4.553472in}}%
\pgfpathcurveto{\pgfqpoint{11.966897in}{4.545658in}}{\pgfqpoint{11.977496in}{4.541268in}}{\pgfqpoint{11.988546in}{4.541268in}}%
\pgfpathclose%
\pgfusepath{stroke,fill}%
\end{pgfscope}%
\begin{pgfscope}%
\pgfpathrectangle{\pgfqpoint{7.640588in}{4.121437in}}{\pgfqpoint{5.699255in}{2.685432in}}%
\pgfusepath{clip}%
\pgfsetbuttcap%
\pgfsetroundjoin%
\definecolor{currentfill}{rgb}{0.000000,0.000000,0.000000}%
\pgfsetfillcolor{currentfill}%
\pgfsetlinewidth{1.003750pt}%
\definecolor{currentstroke}{rgb}{0.000000,0.000000,0.000000}%
\pgfsetstrokecolor{currentstroke}%
\pgfsetdash{}{0pt}%
\pgfpathmoveto{\pgfqpoint{12.660694in}{4.397662in}}%
\pgfpathcurveto{\pgfqpoint{12.671744in}{4.397662in}}{\pgfqpoint{12.682343in}{4.402052in}}{\pgfqpoint{12.690157in}{4.409866in}}%
\pgfpathcurveto{\pgfqpoint{12.697971in}{4.417679in}}{\pgfqpoint{12.702361in}{4.428278in}}{\pgfqpoint{12.702361in}{4.439329in}}%
\pgfpathcurveto{\pgfqpoint{12.702361in}{4.450379in}}{\pgfqpoint{12.697971in}{4.460978in}}{\pgfqpoint{12.690157in}{4.468791in}}%
\pgfpathcurveto{\pgfqpoint{12.682343in}{4.476605in}}{\pgfqpoint{12.671744in}{4.480995in}}{\pgfqpoint{12.660694in}{4.480995in}}%
\pgfpathcurveto{\pgfqpoint{12.649644in}{4.480995in}}{\pgfqpoint{12.639045in}{4.476605in}}{\pgfqpoint{12.631231in}{4.468791in}}%
\pgfpathcurveto{\pgfqpoint{12.623418in}{4.460978in}}{\pgfqpoint{12.619027in}{4.450379in}}{\pgfqpoint{12.619027in}{4.439329in}}%
\pgfpathcurveto{\pgfqpoint{12.619027in}{4.428278in}}{\pgfqpoint{12.623418in}{4.417679in}}{\pgfqpoint{12.631231in}{4.409866in}}%
\pgfpathcurveto{\pgfqpoint{12.639045in}{4.402052in}}{\pgfqpoint{12.649644in}{4.397662in}}{\pgfqpoint{12.660694in}{4.397662in}}%
\pgfpathclose%
\pgfusepath{stroke,fill}%
\end{pgfscope}%
\begin{pgfscope}%
\pgfpathrectangle{\pgfqpoint{7.640588in}{4.121437in}}{\pgfqpoint{5.699255in}{2.685432in}}%
\pgfusepath{clip}%
\pgfsetbuttcap%
\pgfsetroundjoin%
\definecolor{currentfill}{rgb}{0.000000,0.000000,0.000000}%
\pgfsetfillcolor{currentfill}%
\pgfsetlinewidth{1.003750pt}%
\definecolor{currentstroke}{rgb}{0.000000,0.000000,0.000000}%
\pgfsetstrokecolor{currentstroke}%
\pgfsetdash{}{0pt}%
\pgfpathmoveto{\pgfqpoint{12.296614in}{4.593488in}}%
\pgfpathcurveto{\pgfqpoint{12.307664in}{4.593488in}}{\pgfqpoint{12.318263in}{4.597879in}}{\pgfqpoint{12.326077in}{4.605692in}}%
\pgfpathcurveto{\pgfqpoint{12.333890in}{4.613506in}}{\pgfqpoint{12.338281in}{4.624105in}}{\pgfqpoint{12.338281in}{4.635155in}}%
\pgfpathcurveto{\pgfqpoint{12.338281in}{4.646205in}}{\pgfqpoint{12.333890in}{4.656804in}}{\pgfqpoint{12.326077in}{4.664618in}}%
\pgfpathcurveto{\pgfqpoint{12.318263in}{4.672431in}}{\pgfqpoint{12.307664in}{4.676822in}}{\pgfqpoint{12.296614in}{4.676822in}}%
\pgfpathcurveto{\pgfqpoint{12.285564in}{4.676822in}}{\pgfqpoint{12.274965in}{4.672431in}}{\pgfqpoint{12.267151in}{4.664618in}}%
\pgfpathcurveto{\pgfqpoint{12.259338in}{4.656804in}}{\pgfqpoint{12.254947in}{4.646205in}}{\pgfqpoint{12.254947in}{4.635155in}}%
\pgfpathcurveto{\pgfqpoint{12.254947in}{4.624105in}}{\pgfqpoint{12.259338in}{4.613506in}}{\pgfqpoint{12.267151in}{4.605692in}}%
\pgfpathcurveto{\pgfqpoint{12.274965in}{4.597879in}}{\pgfqpoint{12.285564in}{4.593488in}}{\pgfqpoint{12.296614in}{4.593488in}}%
\pgfpathclose%
\pgfusepath{stroke,fill}%
\end{pgfscope}%
\begin{pgfscope}%
\pgfpathrectangle{\pgfqpoint{7.640588in}{4.121437in}}{\pgfqpoint{5.699255in}{2.685432in}}%
\pgfusepath{clip}%
\pgfsetbuttcap%
\pgfsetroundjoin%
\definecolor{currentfill}{rgb}{0.000000,0.000000,0.000000}%
\pgfsetfillcolor{currentfill}%
\pgfsetlinewidth{1.003750pt}%
\definecolor{currentstroke}{rgb}{0.000000,0.000000,0.000000}%
\pgfsetstrokecolor{currentstroke}%
\pgfsetdash{}{0pt}%
\pgfpathmoveto{\pgfqpoint{12.184589in}{4.737094in}}%
\pgfpathcurveto{\pgfqpoint{12.195639in}{4.737094in}}{\pgfqpoint{12.206238in}{4.741485in}}{\pgfqpoint{12.214052in}{4.749298in}}%
\pgfpathcurveto{\pgfqpoint{12.221866in}{4.757112in}}{\pgfqpoint{12.226256in}{4.767711in}}{\pgfqpoint{12.226256in}{4.778761in}}%
\pgfpathcurveto{\pgfqpoint{12.226256in}{4.789811in}}{\pgfqpoint{12.221866in}{4.800410in}}{\pgfqpoint{12.214052in}{4.808224in}}%
\pgfpathcurveto{\pgfqpoint{12.206238in}{4.816037in}}{\pgfqpoint{12.195639in}{4.820428in}}{\pgfqpoint{12.184589in}{4.820428in}}%
\pgfpathcurveto{\pgfqpoint{12.173539in}{4.820428in}}{\pgfqpoint{12.162940in}{4.816037in}}{\pgfqpoint{12.155126in}{4.808224in}}%
\pgfpathcurveto{\pgfqpoint{12.147313in}{4.800410in}}{\pgfqpoint{12.142923in}{4.789811in}}{\pgfqpoint{12.142923in}{4.778761in}}%
\pgfpathcurveto{\pgfqpoint{12.142923in}{4.767711in}}{\pgfqpoint{12.147313in}{4.757112in}}{\pgfqpoint{12.155126in}{4.749298in}}%
\pgfpathcurveto{\pgfqpoint{12.162940in}{4.741485in}}{\pgfqpoint{12.173539in}{4.737094in}}{\pgfqpoint{12.184589in}{4.737094in}}%
\pgfpathclose%
\pgfusepath{stroke,fill}%
\end{pgfscope}%
\begin{pgfscope}%
\pgfpathrectangle{\pgfqpoint{7.640588in}{4.121437in}}{\pgfqpoint{5.699255in}{2.685432in}}%
\pgfusepath{clip}%
\pgfsetbuttcap%
\pgfsetroundjoin%
\definecolor{currentfill}{rgb}{0.000000,0.000000,0.000000}%
\pgfsetfillcolor{currentfill}%
\pgfsetlinewidth{1.003750pt}%
\definecolor{currentstroke}{rgb}{0.000000,0.000000,0.000000}%
\pgfsetstrokecolor{currentstroke}%
\pgfsetdash{}{0pt}%
\pgfpathmoveto{\pgfqpoint{12.884744in}{5.180967in}}%
\pgfpathcurveto{\pgfqpoint{12.895794in}{5.180967in}}{\pgfqpoint{12.906393in}{5.185358in}}{\pgfqpoint{12.914206in}{5.193171in}}%
\pgfpathcurveto{\pgfqpoint{12.922020in}{5.200985in}}{\pgfqpoint{12.926410in}{5.211584in}}{\pgfqpoint{12.926410in}{5.222634in}}%
\pgfpathcurveto{\pgfqpoint{12.926410in}{5.233684in}}{\pgfqpoint{12.922020in}{5.244283in}}{\pgfqpoint{12.914206in}{5.252097in}}%
\pgfpathcurveto{\pgfqpoint{12.906393in}{5.259910in}}{\pgfqpoint{12.895794in}{5.264301in}}{\pgfqpoint{12.884744in}{5.264301in}}%
\pgfpathcurveto{\pgfqpoint{12.873693in}{5.264301in}}{\pgfqpoint{12.863094in}{5.259910in}}{\pgfqpoint{12.855281in}{5.252097in}}%
\pgfpathcurveto{\pgfqpoint{12.847467in}{5.244283in}}{\pgfqpoint{12.843077in}{5.233684in}}{\pgfqpoint{12.843077in}{5.222634in}}%
\pgfpathcurveto{\pgfqpoint{12.843077in}{5.211584in}}{\pgfqpoint{12.847467in}{5.200985in}}{\pgfqpoint{12.855281in}{5.193171in}}%
\pgfpathcurveto{\pgfqpoint{12.863094in}{5.185358in}}{\pgfqpoint{12.873693in}{5.180967in}}{\pgfqpoint{12.884744in}{5.180967in}}%
\pgfpathclose%
\pgfusepath{stroke,fill}%
\end{pgfscope}%
\begin{pgfscope}%
\pgfpathrectangle{\pgfqpoint{7.640588in}{4.121437in}}{\pgfqpoint{5.699255in}{2.685432in}}%
\pgfusepath{clip}%
\pgfsetbuttcap%
\pgfsetroundjoin%
\definecolor{currentfill}{rgb}{0.000000,0.000000,0.000000}%
\pgfsetfillcolor{currentfill}%
\pgfsetlinewidth{1.003750pt}%
\definecolor{currentstroke}{rgb}{0.000000,0.000000,0.000000}%
\pgfsetstrokecolor{currentstroke}%
\pgfsetdash{}{0pt}%
\pgfpathmoveto{\pgfqpoint{12.996768in}{5.781501in}}%
\pgfpathcurveto{\pgfqpoint{13.007818in}{5.781501in}}{\pgfqpoint{13.018417in}{5.785892in}}{\pgfqpoint{13.026231in}{5.793705in}}%
\pgfpathcurveto{\pgfqpoint{13.034045in}{5.801519in}}{\pgfqpoint{13.038435in}{5.812118in}}{\pgfqpoint{13.038435in}{5.823168in}}%
\pgfpathcurveto{\pgfqpoint{13.038435in}{5.834218in}}{\pgfqpoint{13.034045in}{5.844817in}}{\pgfqpoint{13.026231in}{5.852631in}}%
\pgfpathcurveto{\pgfqpoint{13.018417in}{5.860445in}}{\pgfqpoint{13.007818in}{5.864835in}}{\pgfqpoint{12.996768in}{5.864835in}}%
\pgfpathcurveto{\pgfqpoint{12.985718in}{5.864835in}}{\pgfqpoint{12.975119in}{5.860445in}}{\pgfqpoint{12.967305in}{5.852631in}}%
\pgfpathcurveto{\pgfqpoint{12.959492in}{5.844817in}}{\pgfqpoint{12.955102in}{5.834218in}}{\pgfqpoint{12.955102in}{5.823168in}}%
\pgfpathcurveto{\pgfqpoint{12.955102in}{5.812118in}}{\pgfqpoint{12.959492in}{5.801519in}}{\pgfqpoint{12.967305in}{5.793705in}}%
\pgfpathcurveto{\pgfqpoint{12.975119in}{5.785892in}}{\pgfqpoint{12.985718in}{5.781501in}}{\pgfqpoint{12.996768in}{5.781501in}}%
\pgfpathclose%
\pgfusepath{stroke,fill}%
\end{pgfscope}%
\begin{pgfscope}%
\pgfpathrectangle{\pgfqpoint{7.640588in}{4.121437in}}{\pgfqpoint{5.699255in}{2.685432in}}%
\pgfusepath{clip}%
\pgfsetbuttcap%
\pgfsetroundjoin%
\definecolor{currentfill}{rgb}{0.000000,0.000000,0.000000}%
\pgfsetfillcolor{currentfill}%
\pgfsetlinewidth{1.003750pt}%
\definecolor{currentstroke}{rgb}{0.000000,0.000000,0.000000}%
\pgfsetstrokecolor{currentstroke}%
\pgfsetdash{}{0pt}%
\pgfpathmoveto{\pgfqpoint{13.080787in}{5.938163in}}%
\pgfpathcurveto{\pgfqpoint{13.091837in}{5.938163in}}{\pgfqpoint{13.102436in}{5.942553in}}{\pgfqpoint{13.110249in}{5.950366in}}%
\pgfpathcurveto{\pgfqpoint{13.118063in}{5.958180in}}{\pgfqpoint{13.122453in}{5.968779in}}{\pgfqpoint{13.122453in}{5.979829in}}%
\pgfpathcurveto{\pgfqpoint{13.122453in}{5.990879in}}{\pgfqpoint{13.118063in}{6.001478in}}{\pgfqpoint{13.110249in}{6.009292in}}%
\pgfpathcurveto{\pgfqpoint{13.102436in}{6.017106in}}{\pgfqpoint{13.091837in}{6.021496in}}{\pgfqpoint{13.080787in}{6.021496in}}%
\pgfpathcurveto{\pgfqpoint{13.069737in}{6.021496in}}{\pgfqpoint{13.059138in}{6.017106in}}{\pgfqpoint{13.051324in}{6.009292in}}%
\pgfpathcurveto{\pgfqpoint{13.043510in}{6.001478in}}{\pgfqpoint{13.039120in}{5.990879in}}{\pgfqpoint{13.039120in}{5.979829in}}%
\pgfpathcurveto{\pgfqpoint{13.039120in}{5.968779in}}{\pgfqpoint{13.043510in}{5.958180in}}{\pgfqpoint{13.051324in}{5.950366in}}%
\pgfpathcurveto{\pgfqpoint{13.059138in}{5.942553in}}{\pgfqpoint{13.069737in}{5.938163in}}{\pgfqpoint{13.080787in}{5.938163in}}%
\pgfpathclose%
\pgfusepath{stroke,fill}%
\end{pgfscope}%
\begin{pgfscope}%
\pgfpathrectangle{\pgfqpoint{7.640588in}{4.121437in}}{\pgfqpoint{5.699255in}{2.685432in}}%
\pgfusepath{clip}%
\pgfsetbuttcap%
\pgfsetroundjoin%
\definecolor{currentfill}{rgb}{0.000000,0.000000,0.000000}%
\pgfsetfillcolor{currentfill}%
\pgfsetlinewidth{1.003750pt}%
\definecolor{currentstroke}{rgb}{0.000000,0.000000,0.000000}%
\pgfsetstrokecolor{currentstroke}%
\pgfsetdash{}{0pt}%
\pgfpathmoveto{\pgfqpoint{12.576676in}{5.885942in}}%
\pgfpathcurveto{\pgfqpoint{12.587726in}{5.885942in}}{\pgfqpoint{12.598325in}{5.890332in}}{\pgfqpoint{12.606138in}{5.898146in}}%
\pgfpathcurveto{\pgfqpoint{12.613952in}{5.905960in}}{\pgfqpoint{12.618342in}{5.916559in}}{\pgfqpoint{12.618342in}{5.927609in}}%
\pgfpathcurveto{\pgfqpoint{12.618342in}{5.938659in}}{\pgfqpoint{12.613952in}{5.949258in}}{\pgfqpoint{12.606138in}{5.957072in}}%
\pgfpathcurveto{\pgfqpoint{12.598325in}{5.964885in}}{\pgfqpoint{12.587726in}{5.969276in}}{\pgfqpoint{12.576676in}{5.969276in}}%
\pgfpathcurveto{\pgfqpoint{12.565626in}{5.969276in}}{\pgfqpoint{12.555026in}{5.964885in}}{\pgfqpoint{12.547213in}{5.957072in}}%
\pgfpathcurveto{\pgfqpoint{12.539399in}{5.949258in}}{\pgfqpoint{12.535009in}{5.938659in}}{\pgfqpoint{12.535009in}{5.927609in}}%
\pgfpathcurveto{\pgfqpoint{12.535009in}{5.916559in}}{\pgfqpoint{12.539399in}{5.905960in}}{\pgfqpoint{12.547213in}{5.898146in}}%
\pgfpathcurveto{\pgfqpoint{12.555026in}{5.890332in}}{\pgfqpoint{12.565626in}{5.885942in}}{\pgfqpoint{12.576676in}{5.885942in}}%
\pgfpathclose%
\pgfusepath{stroke,fill}%
\end{pgfscope}%
\begin{pgfscope}%
\pgfpathrectangle{\pgfqpoint{7.640588in}{4.121437in}}{\pgfqpoint{5.699255in}{2.685432in}}%
\pgfusepath{clip}%
\pgfsetbuttcap%
\pgfsetroundjoin%
\definecolor{currentfill}{rgb}{0.000000,0.000000,0.000000}%
\pgfsetfillcolor{currentfill}%
\pgfsetlinewidth{1.003750pt}%
\definecolor{currentstroke}{rgb}{0.000000,0.000000,0.000000}%
\pgfsetstrokecolor{currentstroke}%
\pgfsetdash{}{0pt}%
\pgfpathmoveto{\pgfqpoint{12.240602in}{6.016493in}}%
\pgfpathcurveto{\pgfqpoint{12.251652in}{6.016493in}}{\pgfqpoint{12.262251in}{6.020883in}}{\pgfqpoint{12.270064in}{6.028697in}}%
\pgfpathcurveto{\pgfqpoint{12.277878in}{6.036511in}}{\pgfqpoint{12.282268in}{6.047110in}}{\pgfqpoint{12.282268in}{6.058160in}}%
\pgfpathcurveto{\pgfqpoint{12.282268in}{6.069210in}}{\pgfqpoint{12.277878in}{6.079809in}}{\pgfqpoint{12.270064in}{6.087623in}}%
\pgfpathcurveto{\pgfqpoint{12.262251in}{6.095436in}}{\pgfqpoint{12.251652in}{6.099826in}}{\pgfqpoint{12.240602in}{6.099826in}}%
\pgfpathcurveto{\pgfqpoint{12.229551in}{6.099826in}}{\pgfqpoint{12.218952in}{6.095436in}}{\pgfqpoint{12.211139in}{6.087623in}}%
\pgfpathcurveto{\pgfqpoint{12.203325in}{6.079809in}}{\pgfqpoint{12.198935in}{6.069210in}}{\pgfqpoint{12.198935in}{6.058160in}}%
\pgfpathcurveto{\pgfqpoint{12.198935in}{6.047110in}}{\pgfqpoint{12.203325in}{6.036511in}}{\pgfqpoint{12.211139in}{6.028697in}}%
\pgfpathcurveto{\pgfqpoint{12.218952in}{6.020883in}}{\pgfqpoint{12.229551in}{6.016493in}}{\pgfqpoint{12.240602in}{6.016493in}}%
\pgfpathclose%
\pgfusepath{stroke,fill}%
\end{pgfscope}%
\begin{pgfscope}%
\pgfpathrectangle{\pgfqpoint{7.640588in}{4.121437in}}{\pgfqpoint{5.699255in}{2.685432in}}%
\pgfusepath{clip}%
\pgfsetbuttcap%
\pgfsetroundjoin%
\definecolor{currentfill}{rgb}{0.000000,0.000000,0.000000}%
\pgfsetfillcolor{currentfill}%
\pgfsetlinewidth{1.003750pt}%
\definecolor{currentstroke}{rgb}{0.000000,0.000000,0.000000}%
\pgfsetstrokecolor{currentstroke}%
\pgfsetdash{}{0pt}%
\pgfpathmoveto{\pgfqpoint{12.156583in}{5.977328in}}%
\pgfpathcurveto{\pgfqpoint{12.167633in}{5.977328in}}{\pgfqpoint{12.178232in}{5.981718in}}{\pgfqpoint{12.186046in}{5.989532in}}%
\pgfpathcurveto{\pgfqpoint{12.193860in}{5.997345in}}{\pgfqpoint{12.198250in}{6.007944in}}{\pgfqpoint{12.198250in}{6.018995in}}%
\pgfpathcurveto{\pgfqpoint{12.198250in}{6.030045in}}{\pgfqpoint{12.193860in}{6.040644in}}{\pgfqpoint{12.186046in}{6.048457in}}%
\pgfpathcurveto{\pgfqpoint{12.178232in}{6.056271in}}{\pgfqpoint{12.167633in}{6.060661in}}{\pgfqpoint{12.156583in}{6.060661in}}%
\pgfpathcurveto{\pgfqpoint{12.145533in}{6.060661in}}{\pgfqpoint{12.134934in}{6.056271in}}{\pgfqpoint{12.127120in}{6.048457in}}%
\pgfpathcurveto{\pgfqpoint{12.119307in}{6.040644in}}{\pgfqpoint{12.114916in}{6.030045in}}{\pgfqpoint{12.114916in}{6.018995in}}%
\pgfpathcurveto{\pgfqpoint{12.114916in}{6.007944in}}{\pgfqpoint{12.119307in}{5.997345in}}{\pgfqpoint{12.127120in}{5.989532in}}%
\pgfpathcurveto{\pgfqpoint{12.134934in}{5.981718in}}{\pgfqpoint{12.145533in}{5.977328in}}{\pgfqpoint{12.156583in}{5.977328in}}%
\pgfpathclose%
\pgfusepath{stroke,fill}%
\end{pgfscope}%
\begin{pgfscope}%
\pgfpathrectangle{\pgfqpoint{7.640588in}{4.121437in}}{\pgfqpoint{5.699255in}{2.685432in}}%
\pgfusepath{clip}%
\pgfsetbuttcap%
\pgfsetroundjoin%
\definecolor{currentfill}{rgb}{0.000000,0.000000,0.000000}%
\pgfsetfillcolor{currentfill}%
\pgfsetlinewidth{1.003750pt}%
\definecolor{currentstroke}{rgb}{0.000000,0.000000,0.000000}%
\pgfsetstrokecolor{currentstroke}%
\pgfsetdash{}{0pt}%
\pgfpathmoveto{\pgfqpoint{11.988546in}{5.846777in}}%
\pgfpathcurveto{\pgfqpoint{11.999596in}{5.846777in}}{\pgfqpoint{12.010195in}{5.851167in}}{\pgfqpoint{12.018009in}{5.858981in}}%
\pgfpathcurveto{\pgfqpoint{12.025823in}{5.866794in}}{\pgfqpoint{12.030213in}{5.877393in}}{\pgfqpoint{12.030213in}{5.888444in}}%
\pgfpathcurveto{\pgfqpoint{12.030213in}{5.899494in}}{\pgfqpoint{12.025823in}{5.910093in}}{\pgfqpoint{12.018009in}{5.917906in}}%
\pgfpathcurveto{\pgfqpoint{12.010195in}{5.925720in}}{\pgfqpoint{11.999596in}{5.930110in}}{\pgfqpoint{11.988546in}{5.930110in}}%
\pgfpathcurveto{\pgfqpoint{11.977496in}{5.930110in}}{\pgfqpoint{11.966897in}{5.925720in}}{\pgfqpoint{11.959083in}{5.917906in}}%
\pgfpathcurveto{\pgfqpoint{11.951270in}{5.910093in}}{\pgfqpoint{11.946879in}{5.899494in}}{\pgfqpoint{11.946879in}{5.888444in}}%
\pgfpathcurveto{\pgfqpoint{11.946879in}{5.877393in}}{\pgfqpoint{11.951270in}{5.866794in}}{\pgfqpoint{11.959083in}{5.858981in}}%
\pgfpathcurveto{\pgfqpoint{11.966897in}{5.851167in}}{\pgfqpoint{11.977496in}{5.846777in}}{\pgfqpoint{11.988546in}{5.846777in}}%
\pgfpathclose%
\pgfusepath{stroke,fill}%
\end{pgfscope}%
\begin{pgfscope}%
\pgfpathrectangle{\pgfqpoint{7.640588in}{4.121437in}}{\pgfqpoint{5.699255in}{2.685432in}}%
\pgfusepath{clip}%
\pgfsetbuttcap%
\pgfsetroundjoin%
\definecolor{currentfill}{rgb}{0.000000,0.000000,0.000000}%
\pgfsetfillcolor{currentfill}%
\pgfsetlinewidth{1.003750pt}%
\definecolor{currentstroke}{rgb}{0.000000,0.000000,0.000000}%
\pgfsetstrokecolor{currentstroke}%
\pgfsetdash{}{0pt}%
\pgfpathmoveto{\pgfqpoint{12.100571in}{5.755391in}}%
\pgfpathcurveto{\pgfqpoint{12.111621in}{5.755391in}}{\pgfqpoint{12.122220in}{5.759782in}}{\pgfqpoint{12.130034in}{5.767595in}}%
\pgfpathcurveto{\pgfqpoint{12.137847in}{5.775409in}}{\pgfqpoint{12.142237in}{5.786008in}}{\pgfqpoint{12.142237in}{5.797058in}}%
\pgfpathcurveto{\pgfqpoint{12.142237in}{5.808108in}}{\pgfqpoint{12.137847in}{5.818707in}}{\pgfqpoint{12.130034in}{5.826521in}}%
\pgfpathcurveto{\pgfqpoint{12.122220in}{5.834334in}}{\pgfqpoint{12.111621in}{5.838725in}}{\pgfqpoint{12.100571in}{5.838725in}}%
\pgfpathcurveto{\pgfqpoint{12.089521in}{5.838725in}}{\pgfqpoint{12.078922in}{5.834334in}}{\pgfqpoint{12.071108in}{5.826521in}}%
\pgfpathcurveto{\pgfqpoint{12.063294in}{5.818707in}}{\pgfqpoint{12.058904in}{5.808108in}}{\pgfqpoint{12.058904in}{5.797058in}}%
\pgfpathcurveto{\pgfqpoint{12.058904in}{5.786008in}}{\pgfqpoint{12.063294in}{5.775409in}}{\pgfqpoint{12.071108in}{5.767595in}}%
\pgfpathcurveto{\pgfqpoint{12.078922in}{5.759782in}}{\pgfqpoint{12.089521in}{5.755391in}}{\pgfqpoint{12.100571in}{5.755391in}}%
\pgfpathclose%
\pgfusepath{stroke,fill}%
\end{pgfscope}%
\begin{pgfscope}%
\pgfpathrectangle{\pgfqpoint{7.640588in}{4.121437in}}{\pgfqpoint{5.699255in}{2.685432in}}%
\pgfusepath{clip}%
\pgfsetbuttcap%
\pgfsetroundjoin%
\definecolor{currentfill}{rgb}{0.000000,0.000000,0.000000}%
\pgfsetfillcolor{currentfill}%
\pgfsetlinewidth{1.003750pt}%
\definecolor{currentstroke}{rgb}{0.000000,0.000000,0.000000}%
\pgfsetstrokecolor{currentstroke}%
\pgfsetdash{}{0pt}%
\pgfpathmoveto{\pgfqpoint{12.464651in}{5.520400in}}%
\pgfpathcurveto{\pgfqpoint{12.475701in}{5.520400in}}{\pgfqpoint{12.486300in}{5.524790in}}{\pgfqpoint{12.494114in}{5.532604in}}%
\pgfpathcurveto{\pgfqpoint{12.501927in}{5.540417in}}{\pgfqpoint{12.506318in}{5.551016in}}{\pgfqpoint{12.506318in}{5.562066in}}%
\pgfpathcurveto{\pgfqpoint{12.506318in}{5.573116in}}{\pgfqpoint{12.501927in}{5.583716in}}{\pgfqpoint{12.494114in}{5.591529in}}%
\pgfpathcurveto{\pgfqpoint{12.486300in}{5.599343in}}{\pgfqpoint{12.475701in}{5.603733in}}{\pgfqpoint{12.464651in}{5.603733in}}%
\pgfpathcurveto{\pgfqpoint{12.453601in}{5.603733in}}{\pgfqpoint{12.443002in}{5.599343in}}{\pgfqpoint{12.435188in}{5.591529in}}%
\pgfpathcurveto{\pgfqpoint{12.427375in}{5.583716in}}{\pgfqpoint{12.422984in}{5.573116in}}{\pgfqpoint{12.422984in}{5.562066in}}%
\pgfpathcurveto{\pgfqpoint{12.422984in}{5.551016in}}{\pgfqpoint{12.427375in}{5.540417in}}{\pgfqpoint{12.435188in}{5.532604in}}%
\pgfpathcurveto{\pgfqpoint{12.443002in}{5.524790in}}{\pgfqpoint{12.453601in}{5.520400in}}{\pgfqpoint{12.464651in}{5.520400in}}%
\pgfpathclose%
\pgfusepath{stroke,fill}%
\end{pgfscope}%
\begin{pgfscope}%
\pgfpathrectangle{\pgfqpoint{7.640588in}{4.121437in}}{\pgfqpoint{5.699255in}{2.685432in}}%
\pgfusepath{clip}%
\pgfsetbuttcap%
\pgfsetroundjoin%
\definecolor{currentfill}{rgb}{0.000000,0.000000,0.000000}%
\pgfsetfillcolor{currentfill}%
\pgfsetlinewidth{1.003750pt}%
\definecolor{currentstroke}{rgb}{0.000000,0.000000,0.000000}%
\pgfsetstrokecolor{currentstroke}%
\pgfsetdash{}{0pt}%
\pgfpathmoveto{\pgfqpoint{12.072565in}{4.972086in}}%
\pgfpathcurveto{\pgfqpoint{12.083615in}{4.972086in}}{\pgfqpoint{12.094214in}{4.976476in}}{\pgfqpoint{12.102027in}{4.984290in}}%
\pgfpathcurveto{\pgfqpoint{12.109841in}{4.992103in}}{\pgfqpoint{12.114231in}{5.002702in}}{\pgfqpoint{12.114231in}{5.013753in}}%
\pgfpathcurveto{\pgfqpoint{12.114231in}{5.024803in}}{\pgfqpoint{12.109841in}{5.035402in}}{\pgfqpoint{12.102027in}{5.043215in}}%
\pgfpathcurveto{\pgfqpoint{12.094214in}{5.051029in}}{\pgfqpoint{12.083615in}{5.055419in}}{\pgfqpoint{12.072565in}{5.055419in}}%
\pgfpathcurveto{\pgfqpoint{12.061514in}{5.055419in}}{\pgfqpoint{12.050915in}{5.051029in}}{\pgfqpoint{12.043102in}{5.043215in}}%
\pgfpathcurveto{\pgfqpoint{12.035288in}{5.035402in}}{\pgfqpoint{12.030898in}{5.024803in}}{\pgfqpoint{12.030898in}{5.013753in}}%
\pgfpathcurveto{\pgfqpoint{12.030898in}{5.002702in}}{\pgfqpoint{12.035288in}{4.992103in}}{\pgfqpoint{12.043102in}{4.984290in}}%
\pgfpathcurveto{\pgfqpoint{12.050915in}{4.976476in}}{\pgfqpoint{12.061514in}{4.972086in}}{\pgfqpoint{12.072565in}{4.972086in}}%
\pgfpathclose%
\pgfusepath{stroke,fill}%
\end{pgfscope}%
\begin{pgfscope}%
\pgfpathrectangle{\pgfqpoint{7.640588in}{4.121437in}}{\pgfqpoint{5.699255in}{2.685432in}}%
\pgfusepath{clip}%
\pgfsetbuttcap%
\pgfsetroundjoin%
\definecolor{currentfill}{rgb}{0.000000,0.000000,0.000000}%
\pgfsetfillcolor{currentfill}%
\pgfsetlinewidth{1.003750pt}%
\definecolor{currentstroke}{rgb}{0.000000,0.000000,0.000000}%
\pgfsetstrokecolor{currentstroke}%
\pgfsetdash{}{0pt}%
\pgfpathmoveto{\pgfqpoint{11.652472in}{5.024306in}}%
\pgfpathcurveto{\pgfqpoint{11.663522in}{5.024306in}}{\pgfqpoint{11.674121in}{5.028697in}}{\pgfqpoint{11.681935in}{5.036510in}}%
\pgfpathcurveto{\pgfqpoint{11.689748in}{5.044324in}}{\pgfqpoint{11.694139in}{5.054923in}}{\pgfqpoint{11.694139in}{5.065973in}}%
\pgfpathcurveto{\pgfqpoint{11.694139in}{5.077023in}}{\pgfqpoint{11.689748in}{5.087622in}}{\pgfqpoint{11.681935in}{5.095436in}}%
\pgfpathcurveto{\pgfqpoint{11.674121in}{5.103249in}}{\pgfqpoint{11.663522in}{5.107640in}}{\pgfqpoint{11.652472in}{5.107640in}}%
\pgfpathcurveto{\pgfqpoint{11.641422in}{5.107640in}}{\pgfqpoint{11.630823in}{5.103249in}}{\pgfqpoint{11.623009in}{5.095436in}}%
\pgfpathcurveto{\pgfqpoint{11.615196in}{5.087622in}}{\pgfqpoint{11.610805in}{5.077023in}}{\pgfqpoint{11.610805in}{5.065973in}}%
\pgfpathcurveto{\pgfqpoint{11.610805in}{5.054923in}}{\pgfqpoint{11.615196in}{5.044324in}}{\pgfqpoint{11.623009in}{5.036510in}}%
\pgfpathcurveto{\pgfqpoint{11.630823in}{5.028697in}}{\pgfqpoint{11.641422in}{5.024306in}}{\pgfqpoint{11.652472in}{5.024306in}}%
\pgfpathclose%
\pgfusepath{stroke,fill}%
\end{pgfscope}%
\begin{pgfscope}%
\pgfpathrectangle{\pgfqpoint{7.640588in}{4.121437in}}{\pgfqpoint{5.699255in}{2.685432in}}%
\pgfusepath{clip}%
\pgfsetbuttcap%
\pgfsetroundjoin%
\definecolor{currentfill}{rgb}{0.000000,0.000000,0.000000}%
\pgfsetfillcolor{currentfill}%
\pgfsetlinewidth{1.003750pt}%
\definecolor{currentstroke}{rgb}{0.000000,0.000000,0.000000}%
\pgfsetstrokecolor{currentstroke}%
\pgfsetdash{}{0pt}%
\pgfpathmoveto{\pgfqpoint{11.400417in}{4.945976in}}%
\pgfpathcurveto{\pgfqpoint{11.411467in}{4.945976in}}{\pgfqpoint{11.422066in}{4.950366in}}{\pgfqpoint{11.429879in}{4.958180in}}%
\pgfpathcurveto{\pgfqpoint{11.437693in}{4.965993in}}{\pgfqpoint{11.442083in}{4.976592in}}{\pgfqpoint{11.442083in}{4.987642in}}%
\pgfpathcurveto{\pgfqpoint{11.442083in}{4.998692in}}{\pgfqpoint{11.437693in}{5.009292in}}{\pgfqpoint{11.429879in}{5.017105in}}%
\pgfpathcurveto{\pgfqpoint{11.422066in}{5.024919in}}{\pgfqpoint{11.411467in}{5.029309in}}{\pgfqpoint{11.400417in}{5.029309in}}%
\pgfpathcurveto{\pgfqpoint{11.389366in}{5.029309in}}{\pgfqpoint{11.378767in}{5.024919in}}{\pgfqpoint{11.370954in}{5.017105in}}%
\pgfpathcurveto{\pgfqpoint{11.363140in}{5.009292in}}{\pgfqpoint{11.358750in}{4.998692in}}{\pgfqpoint{11.358750in}{4.987642in}}%
\pgfpathcurveto{\pgfqpoint{11.358750in}{4.976592in}}{\pgfqpoint{11.363140in}{4.965993in}}{\pgfqpoint{11.370954in}{4.958180in}}%
\pgfpathcurveto{\pgfqpoint{11.378767in}{4.950366in}}{\pgfqpoint{11.389366in}{4.945976in}}{\pgfqpoint{11.400417in}{4.945976in}}%
\pgfpathclose%
\pgfusepath{stroke,fill}%
\end{pgfscope}%
\begin{pgfscope}%
\pgfpathrectangle{\pgfqpoint{7.640588in}{4.121437in}}{\pgfqpoint{5.699255in}{2.685432in}}%
\pgfusepath{clip}%
\pgfsetbuttcap%
\pgfsetroundjoin%
\definecolor{currentfill}{rgb}{0.000000,0.000000,0.000000}%
\pgfsetfillcolor{currentfill}%
\pgfsetlinewidth{1.003750pt}%
\definecolor{currentstroke}{rgb}{0.000000,0.000000,0.000000}%
\pgfsetstrokecolor{currentstroke}%
\pgfsetdash{}{0pt}%
\pgfpathmoveto{\pgfqpoint{11.176367in}{5.024306in}}%
\pgfpathcurveto{\pgfqpoint{11.187417in}{5.024306in}}{\pgfqpoint{11.198016in}{5.028697in}}{\pgfqpoint{11.205830in}{5.036510in}}%
\pgfpathcurveto{\pgfqpoint{11.213644in}{5.044324in}}{\pgfqpoint{11.218034in}{5.054923in}}{\pgfqpoint{11.218034in}{5.065973in}}%
\pgfpathcurveto{\pgfqpoint{11.218034in}{5.077023in}}{\pgfqpoint{11.213644in}{5.087622in}}{\pgfqpoint{11.205830in}{5.095436in}}%
\pgfpathcurveto{\pgfqpoint{11.198016in}{5.103249in}}{\pgfqpoint{11.187417in}{5.107640in}}{\pgfqpoint{11.176367in}{5.107640in}}%
\pgfpathcurveto{\pgfqpoint{11.165317in}{5.107640in}}{\pgfqpoint{11.154718in}{5.103249in}}{\pgfqpoint{11.146904in}{5.095436in}}%
\pgfpathcurveto{\pgfqpoint{11.139091in}{5.087622in}}{\pgfqpoint{11.134701in}{5.077023in}}{\pgfqpoint{11.134701in}{5.065973in}}%
\pgfpathcurveto{\pgfqpoint{11.134701in}{5.054923in}}{\pgfqpoint{11.139091in}{5.044324in}}{\pgfqpoint{11.146904in}{5.036510in}}%
\pgfpathcurveto{\pgfqpoint{11.154718in}{5.028697in}}{\pgfqpoint{11.165317in}{5.024306in}}{\pgfqpoint{11.176367in}{5.024306in}}%
\pgfpathclose%
\pgfusepath{stroke,fill}%
\end{pgfscope}%
\begin{pgfscope}%
\pgfpathrectangle{\pgfqpoint{7.640588in}{4.121437in}}{\pgfqpoint{5.699255in}{2.685432in}}%
\pgfusepath{clip}%
\pgfsetbuttcap%
\pgfsetroundjoin%
\definecolor{currentfill}{rgb}{0.000000,0.000000,0.000000}%
\pgfsetfillcolor{currentfill}%
\pgfsetlinewidth{1.003750pt}%
\definecolor{currentstroke}{rgb}{0.000000,0.000000,0.000000}%
\pgfsetstrokecolor{currentstroke}%
\pgfsetdash{}{0pt}%
\pgfpathmoveto{\pgfqpoint{11.232380in}{5.233188in}}%
\pgfpathcurveto{\pgfqpoint{11.243430in}{5.233188in}}{\pgfqpoint{11.254029in}{5.237578in}}{\pgfqpoint{11.261842in}{5.245392in}}%
\pgfpathcurveto{\pgfqpoint{11.269656in}{5.253205in}}{\pgfqpoint{11.274046in}{5.263804in}}{\pgfqpoint{11.274046in}{5.274854in}}%
\pgfpathcurveto{\pgfqpoint{11.274046in}{5.285904in}}{\pgfqpoint{11.269656in}{5.296504in}}{\pgfqpoint{11.261842in}{5.304317in}}%
\pgfpathcurveto{\pgfqpoint{11.254029in}{5.312131in}}{\pgfqpoint{11.243430in}{5.316521in}}{\pgfqpoint{11.232380in}{5.316521in}}%
\pgfpathcurveto{\pgfqpoint{11.221329in}{5.316521in}}{\pgfqpoint{11.210730in}{5.312131in}}{\pgfqpoint{11.202917in}{5.304317in}}%
\pgfpathcurveto{\pgfqpoint{11.195103in}{5.296504in}}{\pgfqpoint{11.190713in}{5.285904in}}{\pgfqpoint{11.190713in}{5.274854in}}%
\pgfpathcurveto{\pgfqpoint{11.190713in}{5.263804in}}{\pgfqpoint{11.195103in}{5.253205in}}{\pgfqpoint{11.202917in}{5.245392in}}%
\pgfpathcurveto{\pgfqpoint{11.210730in}{5.237578in}}{\pgfqpoint{11.221329in}{5.233188in}}{\pgfqpoint{11.232380in}{5.233188in}}%
\pgfpathclose%
\pgfusepath{stroke,fill}%
\end{pgfscope}%
\begin{pgfscope}%
\pgfpathrectangle{\pgfqpoint{7.640588in}{4.121437in}}{\pgfqpoint{5.699255in}{2.685432in}}%
\pgfusepath{clip}%
\pgfsetbuttcap%
\pgfsetroundjoin%
\definecolor{currentfill}{rgb}{0.000000,0.000000,0.000000}%
\pgfsetfillcolor{currentfill}%
\pgfsetlinewidth{1.003750pt}%
\definecolor{currentstroke}{rgb}{0.000000,0.000000,0.000000}%
\pgfsetstrokecolor{currentstroke}%
\pgfsetdash{}{0pt}%
\pgfpathmoveto{\pgfqpoint{10.812287in}{5.076527in}}%
\pgfpathcurveto{\pgfqpoint{10.823337in}{5.076527in}}{\pgfqpoint{10.833936in}{5.080917in}}{\pgfqpoint{10.841750in}{5.088730in}}%
\pgfpathcurveto{\pgfqpoint{10.849563in}{5.096544in}}{\pgfqpoint{10.853954in}{5.107143in}}{\pgfqpoint{10.853954in}{5.118193in}}%
\pgfpathcurveto{\pgfqpoint{10.853954in}{5.129243in}}{\pgfqpoint{10.849563in}{5.139842in}}{\pgfqpoint{10.841750in}{5.147656in}}%
\pgfpathcurveto{\pgfqpoint{10.833936in}{5.155470in}}{\pgfqpoint{10.823337in}{5.159860in}}{\pgfqpoint{10.812287in}{5.159860in}}%
\pgfpathcurveto{\pgfqpoint{10.801237in}{5.159860in}}{\pgfqpoint{10.790638in}{5.155470in}}{\pgfqpoint{10.782824in}{5.147656in}}%
\pgfpathcurveto{\pgfqpoint{10.775011in}{5.139842in}}{\pgfqpoint{10.770620in}{5.129243in}}{\pgfqpoint{10.770620in}{5.118193in}}%
\pgfpathcurveto{\pgfqpoint{10.770620in}{5.107143in}}{\pgfqpoint{10.775011in}{5.096544in}}{\pgfqpoint{10.782824in}{5.088730in}}%
\pgfpathcurveto{\pgfqpoint{10.790638in}{5.080917in}}{\pgfqpoint{10.801237in}{5.076527in}}{\pgfqpoint{10.812287in}{5.076527in}}%
\pgfpathclose%
\pgfusepath{stroke,fill}%
\end{pgfscope}%
\begin{pgfscope}%
\pgfpathrectangle{\pgfqpoint{7.640588in}{4.121437in}}{\pgfqpoint{5.699255in}{2.685432in}}%
\pgfusepath{clip}%
\pgfsetbuttcap%
\pgfsetroundjoin%
\definecolor{currentfill}{rgb}{0.000000,0.000000,0.000000}%
\pgfsetfillcolor{currentfill}%
\pgfsetlinewidth{1.003750pt}%
\definecolor{currentstroke}{rgb}{0.000000,0.000000,0.000000}%
\pgfsetstrokecolor{currentstroke}%
\pgfsetdash{}{0pt}%
\pgfpathmoveto{\pgfqpoint{10.728268in}{5.063472in}}%
\pgfpathcurveto{\pgfqpoint{10.739319in}{5.063472in}}{\pgfqpoint{10.749918in}{5.067862in}}{\pgfqpoint{10.757731in}{5.075675in}}%
\pgfpathcurveto{\pgfqpoint{10.765545in}{5.083489in}}{\pgfqpoint{10.769935in}{5.094088in}}{\pgfqpoint{10.769935in}{5.105138in}}%
\pgfpathcurveto{\pgfqpoint{10.769935in}{5.116188in}}{\pgfqpoint{10.765545in}{5.126787in}}{\pgfqpoint{10.757731in}{5.134601in}}%
\pgfpathcurveto{\pgfqpoint{10.749918in}{5.142415in}}{\pgfqpoint{10.739319in}{5.146805in}}{\pgfqpoint{10.728268in}{5.146805in}}%
\pgfpathcurveto{\pgfqpoint{10.717218in}{5.146805in}}{\pgfqpoint{10.706619in}{5.142415in}}{\pgfqpoint{10.698806in}{5.134601in}}%
\pgfpathcurveto{\pgfqpoint{10.690992in}{5.126787in}}{\pgfqpoint{10.686602in}{5.116188in}}{\pgfqpoint{10.686602in}{5.105138in}}%
\pgfpathcurveto{\pgfqpoint{10.686602in}{5.094088in}}{\pgfqpoint{10.690992in}{5.083489in}}{\pgfqpoint{10.698806in}{5.075675in}}%
\pgfpathcurveto{\pgfqpoint{10.706619in}{5.067862in}}{\pgfqpoint{10.717218in}{5.063472in}}{\pgfqpoint{10.728268in}{5.063472in}}%
\pgfpathclose%
\pgfusepath{stroke,fill}%
\end{pgfscope}%
\begin{pgfscope}%
\pgfpathrectangle{\pgfqpoint{7.640588in}{4.121437in}}{\pgfqpoint{5.699255in}{2.685432in}}%
\pgfusepath{clip}%
\pgfsetbuttcap%
\pgfsetroundjoin%
\definecolor{currentfill}{rgb}{0.000000,0.000000,0.000000}%
\pgfsetfillcolor{currentfill}%
\pgfsetlinewidth{1.003750pt}%
\definecolor{currentstroke}{rgb}{0.000000,0.000000,0.000000}%
\pgfsetstrokecolor{currentstroke}%
\pgfsetdash{}{0pt}%
\pgfpathmoveto{\pgfqpoint{10.476213in}{5.376794in}}%
\pgfpathcurveto{\pgfqpoint{10.487263in}{5.376794in}}{\pgfqpoint{10.497862in}{5.381184in}}{\pgfqpoint{10.505676in}{5.388998in}}%
\pgfpathcurveto{\pgfqpoint{10.513489in}{5.396811in}}{\pgfqpoint{10.517880in}{5.407410in}}{\pgfqpoint{10.517880in}{5.418460in}}%
\pgfpathcurveto{\pgfqpoint{10.517880in}{5.429510in}}{\pgfqpoint{10.513489in}{5.440110in}}{\pgfqpoint{10.505676in}{5.447923in}}%
\pgfpathcurveto{\pgfqpoint{10.497862in}{5.455737in}}{\pgfqpoint{10.487263in}{5.460127in}}{\pgfqpoint{10.476213in}{5.460127in}}%
\pgfpathcurveto{\pgfqpoint{10.465163in}{5.460127in}}{\pgfqpoint{10.454564in}{5.455737in}}{\pgfqpoint{10.446750in}{5.447923in}}%
\pgfpathcurveto{\pgfqpoint{10.438937in}{5.440110in}}{\pgfqpoint{10.434546in}{5.429510in}}{\pgfqpoint{10.434546in}{5.418460in}}%
\pgfpathcurveto{\pgfqpoint{10.434546in}{5.407410in}}{\pgfqpoint{10.438937in}{5.396811in}}{\pgfqpoint{10.446750in}{5.388998in}}%
\pgfpathcurveto{\pgfqpoint{10.454564in}{5.381184in}}{\pgfqpoint{10.465163in}{5.376794in}}{\pgfqpoint{10.476213in}{5.376794in}}%
\pgfpathclose%
\pgfusepath{stroke,fill}%
\end{pgfscope}%
\begin{pgfscope}%
\pgfpathrectangle{\pgfqpoint{7.640588in}{4.121437in}}{\pgfqpoint{5.699255in}{2.685432in}}%
\pgfusepath{clip}%
\pgfsetbuttcap%
\pgfsetroundjoin%
\definecolor{currentfill}{rgb}{0.000000,0.000000,0.000000}%
\pgfsetfillcolor{currentfill}%
\pgfsetlinewidth{1.003750pt}%
\definecolor{currentstroke}{rgb}{0.000000,0.000000,0.000000}%
\pgfsetstrokecolor{currentstroke}%
\pgfsetdash{}{0pt}%
\pgfpathmoveto{\pgfqpoint{10.980324in}{5.807612in}}%
\pgfpathcurveto{\pgfqpoint{10.991374in}{5.807612in}}{\pgfqpoint{11.001973in}{5.812002in}}{\pgfqpoint{11.009787in}{5.819816in}}%
\pgfpathcurveto{\pgfqpoint{11.017600in}{5.827629in}}{\pgfqpoint{11.021991in}{5.838228in}}{\pgfqpoint{11.021991in}{5.849278in}}%
\pgfpathcurveto{\pgfqpoint{11.021991in}{5.860328in}}{\pgfqpoint{11.017600in}{5.870927in}}{\pgfqpoint{11.009787in}{5.878741in}}%
\pgfpathcurveto{\pgfqpoint{11.001973in}{5.886555in}}{\pgfqpoint{10.991374in}{5.890945in}}{\pgfqpoint{10.980324in}{5.890945in}}%
\pgfpathcurveto{\pgfqpoint{10.969274in}{5.890945in}}{\pgfqpoint{10.958675in}{5.886555in}}{\pgfqpoint{10.950861in}{5.878741in}}%
\pgfpathcurveto{\pgfqpoint{10.943048in}{5.870927in}}{\pgfqpoint{10.938657in}{5.860328in}}{\pgfqpoint{10.938657in}{5.849278in}}%
\pgfpathcurveto{\pgfqpoint{10.938657in}{5.838228in}}{\pgfqpoint{10.943048in}{5.827629in}}{\pgfqpoint{10.950861in}{5.819816in}}%
\pgfpathcurveto{\pgfqpoint{10.958675in}{5.812002in}}{\pgfqpoint{10.969274in}{5.807612in}}{\pgfqpoint{10.980324in}{5.807612in}}%
\pgfpathclose%
\pgfusepath{stroke,fill}%
\end{pgfscope}%
\begin{pgfscope}%
\pgfpathrectangle{\pgfqpoint{7.640588in}{4.121437in}}{\pgfqpoint{5.699255in}{2.685432in}}%
\pgfusepath{clip}%
\pgfsetbuttcap%
\pgfsetroundjoin%
\definecolor{currentfill}{rgb}{0.000000,0.000000,0.000000}%
\pgfsetfillcolor{currentfill}%
\pgfsetlinewidth{1.003750pt}%
\definecolor{currentstroke}{rgb}{0.000000,0.000000,0.000000}%
\pgfsetstrokecolor{currentstroke}%
\pgfsetdash{}{0pt}%
\pgfpathmoveto{\pgfqpoint{10.896306in}{5.872887in}}%
\pgfpathcurveto{\pgfqpoint{10.907356in}{5.872887in}}{\pgfqpoint{10.917955in}{5.877277in}}{\pgfqpoint{10.925768in}{5.885091in}}%
\pgfpathcurveto{\pgfqpoint{10.933582in}{5.892905in}}{\pgfqpoint{10.937972in}{5.903504in}}{\pgfqpoint{10.937972in}{5.914554in}}%
\pgfpathcurveto{\pgfqpoint{10.937972in}{5.925604in}}{\pgfqpoint{10.933582in}{5.936203in}}{\pgfqpoint{10.925768in}{5.944017in}}%
\pgfpathcurveto{\pgfqpoint{10.917955in}{5.951830in}}{\pgfqpoint{10.907356in}{5.956220in}}{\pgfqpoint{10.896306in}{5.956220in}}%
\pgfpathcurveto{\pgfqpoint{10.885255in}{5.956220in}}{\pgfqpoint{10.874656in}{5.951830in}}{\pgfqpoint{10.866843in}{5.944017in}}%
\pgfpathcurveto{\pgfqpoint{10.859029in}{5.936203in}}{\pgfqpoint{10.854639in}{5.925604in}}{\pgfqpoint{10.854639in}{5.914554in}}%
\pgfpathcurveto{\pgfqpoint{10.854639in}{5.903504in}}{\pgfqpoint{10.859029in}{5.892905in}}{\pgfqpoint{10.866843in}{5.885091in}}%
\pgfpathcurveto{\pgfqpoint{10.874656in}{5.877277in}}{\pgfqpoint{10.885255in}{5.872887in}}{\pgfqpoint{10.896306in}{5.872887in}}%
\pgfpathclose%
\pgfusepath{stroke,fill}%
\end{pgfscope}%
\begin{pgfscope}%
\pgfpathrectangle{\pgfqpoint{7.640588in}{4.121437in}}{\pgfqpoint{5.699255in}{2.685432in}}%
\pgfusepath{clip}%
\pgfsetbuttcap%
\pgfsetroundjoin%
\definecolor{currentfill}{rgb}{0.000000,0.000000,0.000000}%
\pgfsetfillcolor{currentfill}%
\pgfsetlinewidth{1.003750pt}%
\definecolor{currentstroke}{rgb}{0.000000,0.000000,0.000000}%
\pgfsetstrokecolor{currentstroke}%
\pgfsetdash{}{0pt}%
\pgfpathmoveto{\pgfqpoint{10.616244in}{5.964273in}}%
\pgfpathcurveto{\pgfqpoint{10.627294in}{5.964273in}}{\pgfqpoint{10.637893in}{5.968663in}}{\pgfqpoint{10.645707in}{5.976477in}}%
\pgfpathcurveto{\pgfqpoint{10.653520in}{5.984290in}}{\pgfqpoint{10.657910in}{5.994889in}}{\pgfqpoint{10.657910in}{6.005939in}}%
\pgfpathcurveto{\pgfqpoint{10.657910in}{6.016990in}}{\pgfqpoint{10.653520in}{6.027589in}}{\pgfqpoint{10.645707in}{6.035402in}}%
\pgfpathcurveto{\pgfqpoint{10.637893in}{6.043216in}}{\pgfqpoint{10.627294in}{6.047606in}}{\pgfqpoint{10.616244in}{6.047606in}}%
\pgfpathcurveto{\pgfqpoint{10.605194in}{6.047606in}}{\pgfqpoint{10.594595in}{6.043216in}}{\pgfqpoint{10.586781in}{6.035402in}}%
\pgfpathcurveto{\pgfqpoint{10.578967in}{6.027589in}}{\pgfqpoint{10.574577in}{6.016990in}}{\pgfqpoint{10.574577in}{6.005939in}}%
\pgfpathcurveto{\pgfqpoint{10.574577in}{5.994889in}}{\pgfqpoint{10.578967in}{5.984290in}}{\pgfqpoint{10.586781in}{5.976477in}}%
\pgfpathcurveto{\pgfqpoint{10.594595in}{5.968663in}}{\pgfqpoint{10.605194in}{5.964273in}}{\pgfqpoint{10.616244in}{5.964273in}}%
\pgfpathclose%
\pgfusepath{stroke,fill}%
\end{pgfscope}%
\begin{pgfscope}%
\pgfpathrectangle{\pgfqpoint{7.640588in}{4.121437in}}{\pgfqpoint{5.699255in}{2.685432in}}%
\pgfusepath{clip}%
\pgfsetbuttcap%
\pgfsetroundjoin%
\definecolor{currentfill}{rgb}{0.000000,0.000000,0.000000}%
\pgfsetfillcolor{currentfill}%
\pgfsetlinewidth{1.003750pt}%
\definecolor{currentstroke}{rgb}{0.000000,0.000000,0.000000}%
\pgfsetstrokecolor{currentstroke}%
\pgfsetdash{}{0pt}%
\pgfpathmoveto{\pgfqpoint{11.904528in}{6.382036in}}%
\pgfpathcurveto{\pgfqpoint{11.915578in}{6.382036in}}{\pgfqpoint{11.926177in}{6.386426in}}{\pgfqpoint{11.933990in}{6.394240in}}%
\pgfpathcurveto{\pgfqpoint{11.941804in}{6.402053in}}{\pgfqpoint{11.946194in}{6.412652in}}{\pgfqpoint{11.946194in}{6.423702in}}%
\pgfpathcurveto{\pgfqpoint{11.946194in}{6.434752in}}{\pgfqpoint{11.941804in}{6.445351in}}{\pgfqpoint{11.933990in}{6.453165in}}%
\pgfpathcurveto{\pgfqpoint{11.926177in}{6.460979in}}{\pgfqpoint{11.915578in}{6.465369in}}{\pgfqpoint{11.904528in}{6.465369in}}%
\pgfpathcurveto{\pgfqpoint{11.893477in}{6.465369in}}{\pgfqpoint{11.882878in}{6.460979in}}{\pgfqpoint{11.875065in}{6.453165in}}%
\pgfpathcurveto{\pgfqpoint{11.867251in}{6.445351in}}{\pgfqpoint{11.862861in}{6.434752in}}{\pgfqpoint{11.862861in}{6.423702in}}%
\pgfpathcurveto{\pgfqpoint{11.862861in}{6.412652in}}{\pgfqpoint{11.867251in}{6.402053in}}{\pgfqpoint{11.875065in}{6.394240in}}%
\pgfpathcurveto{\pgfqpoint{11.882878in}{6.386426in}}{\pgfqpoint{11.893477in}{6.382036in}}{\pgfqpoint{11.904528in}{6.382036in}}%
\pgfpathclose%
\pgfusepath{stroke,fill}%
\end{pgfscope}%
\begin{pgfscope}%
\pgfpathrectangle{\pgfqpoint{7.640588in}{4.121437in}}{\pgfqpoint{5.699255in}{2.685432in}}%
\pgfusepath{clip}%
\pgfsetbuttcap%
\pgfsetroundjoin%
\definecolor{currentfill}{rgb}{0.000000,0.000000,0.000000}%
\pgfsetfillcolor{currentfill}%
\pgfsetlinewidth{1.003750pt}%
\definecolor{currentstroke}{rgb}{0.000000,0.000000,0.000000}%
\pgfsetstrokecolor{currentstroke}%
\pgfsetdash{}{0pt}%
\pgfpathmoveto{\pgfqpoint{11.988546in}{6.355925in}}%
\pgfpathcurveto{\pgfqpoint{11.999596in}{6.355925in}}{\pgfqpoint{12.010195in}{6.360316in}}{\pgfqpoint{12.018009in}{6.368129in}}%
\pgfpathcurveto{\pgfqpoint{12.025823in}{6.375943in}}{\pgfqpoint{12.030213in}{6.386542in}}{\pgfqpoint{12.030213in}{6.397592in}}%
\pgfpathcurveto{\pgfqpoint{12.030213in}{6.408642in}}{\pgfqpoint{12.025823in}{6.419241in}}{\pgfqpoint{12.018009in}{6.427055in}}%
\pgfpathcurveto{\pgfqpoint{12.010195in}{6.434869in}}{\pgfqpoint{11.999596in}{6.439259in}}{\pgfqpoint{11.988546in}{6.439259in}}%
\pgfpathcurveto{\pgfqpoint{11.977496in}{6.439259in}}{\pgfqpoint{11.966897in}{6.434869in}}{\pgfqpoint{11.959083in}{6.427055in}}%
\pgfpathcurveto{\pgfqpoint{11.951270in}{6.419241in}}{\pgfqpoint{11.946879in}{6.408642in}}{\pgfqpoint{11.946879in}{6.397592in}}%
\pgfpathcurveto{\pgfqpoint{11.946879in}{6.386542in}}{\pgfqpoint{11.951270in}{6.375943in}}{\pgfqpoint{11.959083in}{6.368129in}}%
\pgfpathcurveto{\pgfqpoint{11.966897in}{6.360316in}}{\pgfqpoint{11.977496in}{6.355925in}}{\pgfqpoint{11.988546in}{6.355925in}}%
\pgfpathclose%
\pgfusepath{stroke,fill}%
\end{pgfscope}%
\begin{pgfscope}%
\pgfpathrectangle{\pgfqpoint{7.640588in}{4.121437in}}{\pgfqpoint{5.699255in}{2.685432in}}%
\pgfusepath{clip}%
\pgfsetbuttcap%
\pgfsetroundjoin%
\definecolor{currentfill}{rgb}{0.000000,0.000000,0.000000}%
\pgfsetfillcolor{currentfill}%
\pgfsetlinewidth{1.003750pt}%
\definecolor{currentstroke}{rgb}{0.000000,0.000000,0.000000}%
\pgfsetstrokecolor{currentstroke}%
\pgfsetdash{}{0pt}%
\pgfpathmoveto{\pgfqpoint{12.576676in}{6.316760in}}%
\pgfpathcurveto{\pgfqpoint{12.587726in}{6.316760in}}{\pgfqpoint{12.598325in}{6.321150in}}{\pgfqpoint{12.606138in}{6.328964in}}%
\pgfpathcurveto{\pgfqpoint{12.613952in}{6.336778in}}{\pgfqpoint{12.618342in}{6.347377in}}{\pgfqpoint{12.618342in}{6.358427in}}%
\pgfpathcurveto{\pgfqpoint{12.618342in}{6.369477in}}{\pgfqpoint{12.613952in}{6.380076in}}{\pgfqpoint{12.606138in}{6.387890in}}%
\pgfpathcurveto{\pgfqpoint{12.598325in}{6.395703in}}{\pgfqpoint{12.587726in}{6.400094in}}{\pgfqpoint{12.576676in}{6.400094in}}%
\pgfpathcurveto{\pgfqpoint{12.565626in}{6.400094in}}{\pgfqpoint{12.555026in}{6.395703in}}{\pgfqpoint{12.547213in}{6.387890in}}%
\pgfpathcurveto{\pgfqpoint{12.539399in}{6.380076in}}{\pgfqpoint{12.535009in}{6.369477in}}{\pgfqpoint{12.535009in}{6.358427in}}%
\pgfpathcurveto{\pgfqpoint{12.535009in}{6.347377in}}{\pgfqpoint{12.539399in}{6.336778in}}{\pgfqpoint{12.547213in}{6.328964in}}%
\pgfpathcurveto{\pgfqpoint{12.555026in}{6.321150in}}{\pgfqpoint{12.565626in}{6.316760in}}{\pgfqpoint{12.576676in}{6.316760in}}%
\pgfpathclose%
\pgfusepath{stroke,fill}%
\end{pgfscope}%
\begin{pgfscope}%
\pgfpathrectangle{\pgfqpoint{7.640588in}{4.121437in}}{\pgfqpoint{5.699255in}{2.685432in}}%
\pgfusepath{clip}%
\pgfsetbuttcap%
\pgfsetroundjoin%
\definecolor{currentfill}{rgb}{0.000000,0.000000,0.000000}%
\pgfsetfillcolor{currentfill}%
\pgfsetlinewidth{1.003750pt}%
\definecolor{currentstroke}{rgb}{0.000000,0.000000,0.000000}%
\pgfsetstrokecolor{currentstroke}%
\pgfsetdash{}{0pt}%
\pgfpathmoveto{\pgfqpoint{12.660694in}{6.225375in}}%
\pgfpathcurveto{\pgfqpoint{12.671744in}{6.225375in}}{\pgfqpoint{12.682343in}{6.229765in}}{\pgfqpoint{12.690157in}{6.237578in}}%
\pgfpathcurveto{\pgfqpoint{12.697971in}{6.245392in}}{\pgfqpoint{12.702361in}{6.255991in}}{\pgfqpoint{12.702361in}{6.267041in}}%
\pgfpathcurveto{\pgfqpoint{12.702361in}{6.278091in}}{\pgfqpoint{12.697971in}{6.288690in}}{\pgfqpoint{12.690157in}{6.296504in}}%
\pgfpathcurveto{\pgfqpoint{12.682343in}{6.304318in}}{\pgfqpoint{12.671744in}{6.308708in}}{\pgfqpoint{12.660694in}{6.308708in}}%
\pgfpathcurveto{\pgfqpoint{12.649644in}{6.308708in}}{\pgfqpoint{12.639045in}{6.304318in}}{\pgfqpoint{12.631231in}{6.296504in}}%
\pgfpathcurveto{\pgfqpoint{12.623418in}{6.288690in}}{\pgfqpoint{12.619027in}{6.278091in}}{\pgfqpoint{12.619027in}{6.267041in}}%
\pgfpathcurveto{\pgfqpoint{12.619027in}{6.255991in}}{\pgfqpoint{12.623418in}{6.245392in}}{\pgfqpoint{12.631231in}{6.237578in}}%
\pgfpathcurveto{\pgfqpoint{12.639045in}{6.229765in}}{\pgfqpoint{12.649644in}{6.225375in}}{\pgfqpoint{12.660694in}{6.225375in}}%
\pgfpathclose%
\pgfusepath{stroke,fill}%
\end{pgfscope}%
\begin{pgfscope}%
\pgfpathrectangle{\pgfqpoint{7.640588in}{4.121437in}}{\pgfqpoint{5.699255in}{2.685432in}}%
\pgfusepath{clip}%
\pgfsetbuttcap%
\pgfsetroundjoin%
\definecolor{currentfill}{rgb}{0.000000,0.000000,0.000000}%
\pgfsetfillcolor{currentfill}%
\pgfsetlinewidth{1.003750pt}%
\definecolor{currentstroke}{rgb}{0.000000,0.000000,0.000000}%
\pgfsetstrokecolor{currentstroke}%
\pgfsetdash{}{0pt}%
\pgfpathmoveto{\pgfqpoint{12.772719in}{6.368981in}}%
\pgfpathcurveto{\pgfqpoint{12.783769in}{6.368981in}}{\pgfqpoint{12.794368in}{6.373371in}}{\pgfqpoint{12.802182in}{6.381184in}}%
\pgfpathcurveto{\pgfqpoint{12.809995in}{6.388998in}}{\pgfqpoint{12.814385in}{6.399597in}}{\pgfqpoint{12.814385in}{6.410647in}}%
\pgfpathcurveto{\pgfqpoint{12.814385in}{6.421697in}}{\pgfqpoint{12.809995in}{6.432296in}}{\pgfqpoint{12.802182in}{6.440110in}}%
\pgfpathcurveto{\pgfqpoint{12.794368in}{6.447924in}}{\pgfqpoint{12.783769in}{6.452314in}}{\pgfqpoint{12.772719in}{6.452314in}}%
\pgfpathcurveto{\pgfqpoint{12.761669in}{6.452314in}}{\pgfqpoint{12.751070in}{6.447924in}}{\pgfqpoint{12.743256in}{6.440110in}}%
\pgfpathcurveto{\pgfqpoint{12.735442in}{6.432296in}}{\pgfqpoint{12.731052in}{6.421697in}}{\pgfqpoint{12.731052in}{6.410647in}}%
\pgfpathcurveto{\pgfqpoint{12.731052in}{6.399597in}}{\pgfqpoint{12.735442in}{6.388998in}}{\pgfqpoint{12.743256in}{6.381184in}}%
\pgfpathcurveto{\pgfqpoint{12.751070in}{6.373371in}}{\pgfqpoint{12.761669in}{6.368981in}}{\pgfqpoint{12.772719in}{6.368981in}}%
\pgfpathclose%
\pgfusepath{stroke,fill}%
\end{pgfscope}%
\begin{pgfscope}%
\pgfpathrectangle{\pgfqpoint{7.640588in}{4.121437in}}{\pgfqpoint{5.699255in}{2.685432in}}%
\pgfusepath{clip}%
\pgfsetbuttcap%
\pgfsetroundjoin%
\definecolor{currentfill}{rgb}{0.000000,0.000000,0.000000}%
\pgfsetfillcolor{currentfill}%
\pgfsetlinewidth{1.003750pt}%
\definecolor{currentstroke}{rgb}{0.000000,0.000000,0.000000}%
\pgfsetstrokecolor{currentstroke}%
\pgfsetdash{}{0pt}%
\pgfpathmoveto{\pgfqpoint{12.912750in}{6.630082in}}%
\pgfpathcurveto{\pgfqpoint{12.923800in}{6.630082in}}{\pgfqpoint{12.934399in}{6.634473in}}{\pgfqpoint{12.942212in}{6.642286in}}%
\pgfpathcurveto{\pgfqpoint{12.950026in}{6.650100in}}{\pgfqpoint{12.954416in}{6.660699in}}{\pgfqpoint{12.954416in}{6.671749in}}%
\pgfpathcurveto{\pgfqpoint{12.954416in}{6.682799in}}{\pgfqpoint{12.950026in}{6.693398in}}{\pgfqpoint{12.942212in}{6.701212in}}%
\pgfpathcurveto{\pgfqpoint{12.934399in}{6.709025in}}{\pgfqpoint{12.923800in}{6.713416in}}{\pgfqpoint{12.912750in}{6.713416in}}%
\pgfpathcurveto{\pgfqpoint{12.901700in}{6.713416in}}{\pgfqpoint{12.891101in}{6.709025in}}{\pgfqpoint{12.883287in}{6.701212in}}%
\pgfpathcurveto{\pgfqpoint{12.875473in}{6.693398in}}{\pgfqpoint{12.871083in}{6.682799in}}{\pgfqpoint{12.871083in}{6.671749in}}%
\pgfpathcurveto{\pgfqpoint{12.871083in}{6.660699in}}{\pgfqpoint{12.875473in}{6.650100in}}{\pgfqpoint{12.883287in}{6.642286in}}%
\pgfpathcurveto{\pgfqpoint{12.891101in}{6.634473in}}{\pgfqpoint{12.901700in}{6.630082in}}{\pgfqpoint{12.912750in}{6.630082in}}%
\pgfpathclose%
\pgfusepath{stroke,fill}%
\end{pgfscope}%
\begin{pgfscope}%
\pgfpathrectangle{\pgfqpoint{7.640588in}{4.121437in}}{\pgfqpoint{5.699255in}{2.685432in}}%
\pgfusepath{clip}%
\pgfsetbuttcap%
\pgfsetroundjoin%
\definecolor{currentfill}{rgb}{0.000000,0.000000,0.000000}%
\pgfsetfillcolor{currentfill}%
\pgfsetlinewidth{1.003750pt}%
\definecolor{currentstroke}{rgb}{0.000000,0.000000,0.000000}%
\pgfsetstrokecolor{currentstroke}%
\pgfsetdash{}{0pt}%
\pgfpathmoveto{\pgfqpoint{12.380632in}{6.643137in}}%
\pgfpathcurveto{\pgfqpoint{12.391683in}{6.643137in}}{\pgfqpoint{12.402282in}{6.647528in}}{\pgfqpoint{12.410095in}{6.655341in}}%
\pgfpathcurveto{\pgfqpoint{12.417909in}{6.663155in}}{\pgfqpoint{12.422299in}{6.673754in}}{\pgfqpoint{12.422299in}{6.684804in}}%
\pgfpathcurveto{\pgfqpoint{12.422299in}{6.695854in}}{\pgfqpoint{12.417909in}{6.706453in}}{\pgfqpoint{12.410095in}{6.714267in}}%
\pgfpathcurveto{\pgfqpoint{12.402282in}{6.722081in}}{\pgfqpoint{12.391683in}{6.726471in}}{\pgfqpoint{12.380632in}{6.726471in}}%
\pgfpathcurveto{\pgfqpoint{12.369582in}{6.726471in}}{\pgfqpoint{12.358983in}{6.722081in}}{\pgfqpoint{12.351170in}{6.714267in}}%
\pgfpathcurveto{\pgfqpoint{12.343356in}{6.706453in}}{\pgfqpoint{12.338966in}{6.695854in}}{\pgfqpoint{12.338966in}{6.684804in}}%
\pgfpathcurveto{\pgfqpoint{12.338966in}{6.673754in}}{\pgfqpoint{12.343356in}{6.663155in}}{\pgfqpoint{12.351170in}{6.655341in}}%
\pgfpathcurveto{\pgfqpoint{12.358983in}{6.647528in}}{\pgfqpoint{12.369582in}{6.643137in}}{\pgfqpoint{12.380632in}{6.643137in}}%
\pgfpathclose%
\pgfusepath{stroke,fill}%
\end{pgfscope}%
\begin{pgfscope}%
\pgfpathrectangle{\pgfqpoint{7.640588in}{4.121437in}}{\pgfqpoint{5.699255in}{2.685432in}}%
\pgfusepath{clip}%
\pgfsetbuttcap%
\pgfsetroundjoin%
\definecolor{currentfill}{rgb}{0.000000,0.000000,0.000000}%
\pgfsetfillcolor{currentfill}%
\pgfsetlinewidth{1.003750pt}%
\definecolor{currentstroke}{rgb}{0.000000,0.000000,0.000000}%
\pgfsetstrokecolor{currentstroke}%
\pgfsetdash{}{0pt}%
\pgfpathmoveto{\pgfqpoint{11.652472in}{6.460366in}}%
\pgfpathcurveto{\pgfqpoint{11.663522in}{6.460366in}}{\pgfqpoint{11.674121in}{6.464756in}}{\pgfqpoint{11.681935in}{6.472570in}}%
\pgfpathcurveto{\pgfqpoint{11.689748in}{6.480384in}}{\pgfqpoint{11.694139in}{6.490983in}}{\pgfqpoint{11.694139in}{6.502033in}}%
\pgfpathcurveto{\pgfqpoint{11.694139in}{6.513083in}}{\pgfqpoint{11.689748in}{6.523682in}}{\pgfqpoint{11.681935in}{6.531496in}}%
\pgfpathcurveto{\pgfqpoint{11.674121in}{6.539309in}}{\pgfqpoint{11.663522in}{6.543700in}}{\pgfqpoint{11.652472in}{6.543700in}}%
\pgfpathcurveto{\pgfqpoint{11.641422in}{6.543700in}}{\pgfqpoint{11.630823in}{6.539309in}}{\pgfqpoint{11.623009in}{6.531496in}}%
\pgfpathcurveto{\pgfqpoint{11.615196in}{6.523682in}}{\pgfqpoint{11.610805in}{6.513083in}}{\pgfqpoint{11.610805in}{6.502033in}}%
\pgfpathcurveto{\pgfqpoint{11.610805in}{6.490983in}}{\pgfqpoint{11.615196in}{6.480384in}}{\pgfqpoint{11.623009in}{6.472570in}}%
\pgfpathcurveto{\pgfqpoint{11.630823in}{6.464756in}}{\pgfqpoint{11.641422in}{6.460366in}}{\pgfqpoint{11.652472in}{6.460366in}}%
\pgfpathclose%
\pgfusepath{stroke,fill}%
\end{pgfscope}%
\begin{pgfscope}%
\pgfpathrectangle{\pgfqpoint{7.640588in}{4.121437in}}{\pgfqpoint{5.699255in}{2.685432in}}%
\pgfusepath{clip}%
\pgfsetbuttcap%
\pgfsetroundjoin%
\definecolor{currentfill}{rgb}{0.000000,0.000000,0.000000}%
\pgfsetfillcolor{currentfill}%
\pgfsetlinewidth{1.003750pt}%
\definecolor{currentstroke}{rgb}{0.000000,0.000000,0.000000}%
\pgfsetstrokecolor{currentstroke}%
\pgfsetdash{}{0pt}%
\pgfpathmoveto{\pgfqpoint{11.512441in}{6.512587in}}%
\pgfpathcurveto{\pgfqpoint{11.523491in}{6.512587in}}{\pgfqpoint{11.534090in}{6.516977in}}{\pgfqpoint{11.541904in}{6.524790in}}%
\pgfpathcurveto{\pgfqpoint{11.549718in}{6.532604in}}{\pgfqpoint{11.554108in}{6.543203in}}{\pgfqpoint{11.554108in}{6.554253in}}%
\pgfpathcurveto{\pgfqpoint{11.554108in}{6.565303in}}{\pgfqpoint{11.549718in}{6.575902in}}{\pgfqpoint{11.541904in}{6.583716in}}%
\pgfpathcurveto{\pgfqpoint{11.534090in}{6.591530in}}{\pgfqpoint{11.523491in}{6.595920in}}{\pgfqpoint{11.512441in}{6.595920in}}%
\pgfpathcurveto{\pgfqpoint{11.501391in}{6.595920in}}{\pgfqpoint{11.490792in}{6.591530in}}{\pgfqpoint{11.482978in}{6.583716in}}%
\pgfpathcurveto{\pgfqpoint{11.475165in}{6.575902in}}{\pgfqpoint{11.470775in}{6.565303in}}{\pgfqpoint{11.470775in}{6.554253in}}%
\pgfpathcurveto{\pgfqpoint{11.470775in}{6.543203in}}{\pgfqpoint{11.475165in}{6.532604in}}{\pgfqpoint{11.482978in}{6.524790in}}%
\pgfpathcurveto{\pgfqpoint{11.490792in}{6.516977in}}{\pgfqpoint{11.501391in}{6.512587in}}{\pgfqpoint{11.512441in}{6.512587in}}%
\pgfpathclose%
\pgfusepath{stroke,fill}%
\end{pgfscope}%
\begin{pgfscope}%
\pgfpathrectangle{\pgfqpoint{7.640588in}{4.121437in}}{\pgfqpoint{5.699255in}{2.685432in}}%
\pgfusepath{clip}%
\pgfsetbuttcap%
\pgfsetroundjoin%
\definecolor{currentfill}{rgb}{0.000000,0.000000,0.000000}%
\pgfsetfillcolor{currentfill}%
\pgfsetlinewidth{1.003750pt}%
\definecolor{currentstroke}{rgb}{0.000000,0.000000,0.000000}%
\pgfsetstrokecolor{currentstroke}%
\pgfsetdash{}{0pt}%
\pgfpathmoveto{\pgfqpoint{11.372410in}{6.447311in}}%
\pgfpathcurveto{\pgfqpoint{11.383461in}{6.447311in}}{\pgfqpoint{11.394060in}{6.451701in}}{\pgfqpoint{11.401873in}{6.459515in}}%
\pgfpathcurveto{\pgfqpoint{11.409687in}{6.467329in}}{\pgfqpoint{11.414077in}{6.477928in}}{\pgfqpoint{11.414077in}{6.488978in}}%
\pgfpathcurveto{\pgfqpoint{11.414077in}{6.500028in}}{\pgfqpoint{11.409687in}{6.510627in}}{\pgfqpoint{11.401873in}{6.518441in}}%
\pgfpathcurveto{\pgfqpoint{11.394060in}{6.526254in}}{\pgfqpoint{11.383461in}{6.530644in}}{\pgfqpoint{11.372410in}{6.530644in}}%
\pgfpathcurveto{\pgfqpoint{11.361360in}{6.530644in}}{\pgfqpoint{11.350761in}{6.526254in}}{\pgfqpoint{11.342948in}{6.518441in}}%
\pgfpathcurveto{\pgfqpoint{11.335134in}{6.510627in}}{\pgfqpoint{11.330744in}{6.500028in}}{\pgfqpoint{11.330744in}{6.488978in}}%
\pgfpathcurveto{\pgfqpoint{11.330744in}{6.477928in}}{\pgfqpoint{11.335134in}{6.467329in}}{\pgfqpoint{11.342948in}{6.459515in}}%
\pgfpathcurveto{\pgfqpoint{11.350761in}{6.451701in}}{\pgfqpoint{11.361360in}{6.447311in}}{\pgfqpoint{11.372410in}{6.447311in}}%
\pgfpathclose%
\pgfusepath{stroke,fill}%
\end{pgfscope}%
\begin{pgfscope}%
\pgfpathrectangle{\pgfqpoint{7.640588in}{4.121437in}}{\pgfqpoint{5.699255in}{2.685432in}}%
\pgfusepath{clip}%
\pgfsetbuttcap%
\pgfsetroundjoin%
\definecolor{currentfill}{rgb}{0.000000,0.000000,0.000000}%
\pgfsetfillcolor{currentfill}%
\pgfsetlinewidth{1.003750pt}%
\definecolor{currentstroke}{rgb}{0.000000,0.000000,0.000000}%
\pgfsetstrokecolor{currentstroke}%
\pgfsetdash{}{0pt}%
\pgfpathmoveto{\pgfqpoint{10.616244in}{6.590917in}}%
\pgfpathcurveto{\pgfqpoint{10.627294in}{6.590917in}}{\pgfqpoint{10.637893in}{6.595307in}}{\pgfqpoint{10.645707in}{6.603121in}}%
\pgfpathcurveto{\pgfqpoint{10.653520in}{6.610935in}}{\pgfqpoint{10.657910in}{6.621534in}}{\pgfqpoint{10.657910in}{6.632584in}}%
\pgfpathcurveto{\pgfqpoint{10.657910in}{6.643634in}}{\pgfqpoint{10.653520in}{6.654233in}}{\pgfqpoint{10.645707in}{6.662047in}}%
\pgfpathcurveto{\pgfqpoint{10.637893in}{6.669860in}}{\pgfqpoint{10.627294in}{6.674250in}}{\pgfqpoint{10.616244in}{6.674250in}}%
\pgfpathcurveto{\pgfqpoint{10.605194in}{6.674250in}}{\pgfqpoint{10.594595in}{6.669860in}}{\pgfqpoint{10.586781in}{6.662047in}}%
\pgfpathcurveto{\pgfqpoint{10.578967in}{6.654233in}}{\pgfqpoint{10.574577in}{6.643634in}}{\pgfqpoint{10.574577in}{6.632584in}}%
\pgfpathcurveto{\pgfqpoint{10.574577in}{6.621534in}}{\pgfqpoint{10.578967in}{6.610935in}}{\pgfqpoint{10.586781in}{6.603121in}}%
\pgfpathcurveto{\pgfqpoint{10.594595in}{6.595307in}}{\pgfqpoint{10.605194in}{6.590917in}}{\pgfqpoint{10.616244in}{6.590917in}}%
\pgfpathclose%
\pgfusepath{stroke,fill}%
\end{pgfscope}%
\begin{pgfscope}%
\pgfpathrectangle{\pgfqpoint{7.640588in}{4.121437in}}{\pgfqpoint{5.699255in}{2.685432in}}%
\pgfusepath{clip}%
\pgfsetbuttcap%
\pgfsetroundjoin%
\definecolor{currentfill}{rgb}{0.000000,0.000000,0.000000}%
\pgfsetfillcolor{currentfill}%
\pgfsetlinewidth{1.003750pt}%
\definecolor{currentstroke}{rgb}{0.000000,0.000000,0.000000}%
\pgfsetstrokecolor{currentstroke}%
\pgfsetdash{}{0pt}%
\pgfpathmoveto{\pgfqpoint{9.832071in}{6.434256in}}%
\pgfpathcurveto{\pgfqpoint{9.843121in}{6.434256in}}{\pgfqpoint{9.853720in}{6.438646in}}{\pgfqpoint{9.861534in}{6.446460in}}%
\pgfpathcurveto{\pgfqpoint{9.869347in}{6.454274in}}{\pgfqpoint{9.873738in}{6.464873in}}{\pgfqpoint{9.873738in}{6.475923in}}%
\pgfpathcurveto{\pgfqpoint{9.873738in}{6.486973in}}{\pgfqpoint{9.869347in}{6.497572in}}{\pgfqpoint{9.861534in}{6.505385in}}%
\pgfpathcurveto{\pgfqpoint{9.853720in}{6.513199in}}{\pgfqpoint{9.843121in}{6.517589in}}{\pgfqpoint{9.832071in}{6.517589in}}%
\pgfpathcurveto{\pgfqpoint{9.821021in}{6.517589in}}{\pgfqpoint{9.810422in}{6.513199in}}{\pgfqpoint{9.802608in}{6.505385in}}%
\pgfpathcurveto{\pgfqpoint{9.794795in}{6.497572in}}{\pgfqpoint{9.790404in}{6.486973in}}{\pgfqpoint{9.790404in}{6.475923in}}%
\pgfpathcurveto{\pgfqpoint{9.790404in}{6.464873in}}{\pgfqpoint{9.794795in}{6.454274in}}{\pgfqpoint{9.802608in}{6.446460in}}%
\pgfpathcurveto{\pgfqpoint{9.810422in}{6.438646in}}{\pgfqpoint{9.821021in}{6.434256in}}{\pgfqpoint{9.832071in}{6.434256in}}%
\pgfpathclose%
\pgfusepath{stroke,fill}%
\end{pgfscope}%
\begin{pgfscope}%
\pgfpathrectangle{\pgfqpoint{7.640588in}{4.121437in}}{\pgfqpoint{5.699255in}{2.685432in}}%
\pgfusepath{clip}%
\pgfsetbuttcap%
\pgfsetroundjoin%
\definecolor{currentfill}{rgb}{0.000000,0.000000,0.000000}%
\pgfsetfillcolor{currentfill}%
\pgfsetlinewidth{1.003750pt}%
\definecolor{currentstroke}{rgb}{0.000000,0.000000,0.000000}%
\pgfsetstrokecolor{currentstroke}%
\pgfsetdash{}{0pt}%
\pgfpathmoveto{\pgfqpoint{9.131917in}{6.630082in}}%
\pgfpathcurveto{\pgfqpoint{9.142967in}{6.630082in}}{\pgfqpoint{9.153566in}{6.634473in}}{\pgfqpoint{9.161380in}{6.642286in}}%
\pgfpathcurveto{\pgfqpoint{9.169193in}{6.650100in}}{\pgfqpoint{9.173584in}{6.660699in}}{\pgfqpoint{9.173584in}{6.671749in}}%
\pgfpathcurveto{\pgfqpoint{9.173584in}{6.682799in}}{\pgfqpoint{9.169193in}{6.693398in}}{\pgfqpoint{9.161380in}{6.701212in}}%
\pgfpathcurveto{\pgfqpoint{9.153566in}{6.709025in}}{\pgfqpoint{9.142967in}{6.713416in}}{\pgfqpoint{9.131917in}{6.713416in}}%
\pgfpathcurveto{\pgfqpoint{9.120867in}{6.713416in}}{\pgfqpoint{9.110268in}{6.709025in}}{\pgfqpoint{9.102454in}{6.701212in}}%
\pgfpathcurveto{\pgfqpoint{9.094640in}{6.693398in}}{\pgfqpoint{9.090250in}{6.682799in}}{\pgfqpoint{9.090250in}{6.671749in}}%
\pgfpathcurveto{\pgfqpoint{9.090250in}{6.660699in}}{\pgfqpoint{9.094640in}{6.650100in}}{\pgfqpoint{9.102454in}{6.642286in}}%
\pgfpathcurveto{\pgfqpoint{9.110268in}{6.634473in}}{\pgfqpoint{9.120867in}{6.630082in}}{\pgfqpoint{9.131917in}{6.630082in}}%
\pgfpathclose%
\pgfusepath{stroke,fill}%
\end{pgfscope}%
\begin{pgfscope}%
\pgfpathrectangle{\pgfqpoint{7.640588in}{4.121437in}}{\pgfqpoint{5.699255in}{2.685432in}}%
\pgfusepath{clip}%
\pgfsetbuttcap%
\pgfsetroundjoin%
\definecolor{currentfill}{rgb}{0.000000,0.000000,0.000000}%
\pgfsetfillcolor{currentfill}%
\pgfsetlinewidth{1.003750pt}%
\definecolor{currentstroke}{rgb}{0.000000,0.000000,0.000000}%
\pgfsetstrokecolor{currentstroke}%
\pgfsetdash{}{0pt}%
\pgfpathmoveto{\pgfqpoint{9.047898in}{6.382036in}}%
\pgfpathcurveto{\pgfqpoint{9.058948in}{6.382036in}}{\pgfqpoint{9.069548in}{6.386426in}}{\pgfqpoint{9.077361in}{6.394240in}}%
\pgfpathcurveto{\pgfqpoint{9.085175in}{6.402053in}}{\pgfqpoint{9.089565in}{6.412652in}}{\pgfqpoint{9.089565in}{6.423702in}}%
\pgfpathcurveto{\pgfqpoint{9.089565in}{6.434752in}}{\pgfqpoint{9.085175in}{6.445351in}}{\pgfqpoint{9.077361in}{6.453165in}}%
\pgfpathcurveto{\pgfqpoint{9.069548in}{6.460979in}}{\pgfqpoint{9.058948in}{6.465369in}}{\pgfqpoint{9.047898in}{6.465369in}}%
\pgfpathcurveto{\pgfqpoint{9.036848in}{6.465369in}}{\pgfqpoint{9.026249in}{6.460979in}}{\pgfqpoint{9.018436in}{6.453165in}}%
\pgfpathcurveto{\pgfqpoint{9.010622in}{6.445351in}}{\pgfqpoint{9.006232in}{6.434752in}}{\pgfqpoint{9.006232in}{6.423702in}}%
\pgfpathcurveto{\pgfqpoint{9.006232in}{6.412652in}}{\pgfqpoint{9.010622in}{6.402053in}}{\pgfqpoint{9.018436in}{6.394240in}}%
\pgfpathcurveto{\pgfqpoint{9.026249in}{6.386426in}}{\pgfqpoint{9.036848in}{6.382036in}}{\pgfqpoint{9.047898in}{6.382036in}}%
\pgfpathclose%
\pgfusepath{stroke,fill}%
\end{pgfscope}%
\begin{pgfscope}%
\pgfpathrectangle{\pgfqpoint{7.640588in}{4.121437in}}{\pgfqpoint{5.699255in}{2.685432in}}%
\pgfusepath{clip}%
\pgfsetbuttcap%
\pgfsetroundjoin%
\definecolor{currentfill}{rgb}{0.000000,0.000000,0.000000}%
\pgfsetfillcolor{currentfill}%
\pgfsetlinewidth{1.003750pt}%
\definecolor{currentstroke}{rgb}{0.000000,0.000000,0.000000}%
\pgfsetstrokecolor{currentstroke}%
\pgfsetdash{}{0pt}%
\pgfpathmoveto{\pgfqpoint{8.515781in}{6.238430in}}%
\pgfpathcurveto{\pgfqpoint{8.526831in}{6.238430in}}{\pgfqpoint{8.537430in}{6.242820in}}{\pgfqpoint{8.545244in}{6.250634in}}%
\pgfpathcurveto{\pgfqpoint{8.553058in}{6.258447in}}{\pgfqpoint{8.557448in}{6.269046in}}{\pgfqpoint{8.557448in}{6.280096in}}%
\pgfpathcurveto{\pgfqpoint{8.557448in}{6.291146in}}{\pgfqpoint{8.553058in}{6.301745in}}{\pgfqpoint{8.545244in}{6.309559in}}%
\pgfpathcurveto{\pgfqpoint{8.537430in}{6.317373in}}{\pgfqpoint{8.526831in}{6.321763in}}{\pgfqpoint{8.515781in}{6.321763in}}%
\pgfpathcurveto{\pgfqpoint{8.504731in}{6.321763in}}{\pgfqpoint{8.494132in}{6.317373in}}{\pgfqpoint{8.486318in}{6.309559in}}%
\pgfpathcurveto{\pgfqpoint{8.478505in}{6.301745in}}{\pgfqpoint{8.474114in}{6.291146in}}{\pgfqpoint{8.474114in}{6.280096in}}%
\pgfpathcurveto{\pgfqpoint{8.474114in}{6.269046in}}{\pgfqpoint{8.478505in}{6.258447in}}{\pgfqpoint{8.486318in}{6.250634in}}%
\pgfpathcurveto{\pgfqpoint{8.494132in}{6.242820in}}{\pgfqpoint{8.504731in}{6.238430in}}{\pgfqpoint{8.515781in}{6.238430in}}%
\pgfpathclose%
\pgfusepath{stroke,fill}%
\end{pgfscope}%
\begin{pgfscope}%
\pgfpathrectangle{\pgfqpoint{7.640588in}{4.121437in}}{\pgfqpoint{5.699255in}{2.685432in}}%
\pgfusepath{clip}%
\pgfsetbuttcap%
\pgfsetroundjoin%
\definecolor{currentfill}{rgb}{0.000000,0.000000,0.000000}%
\pgfsetfillcolor{currentfill}%
\pgfsetlinewidth{1.003750pt}%
\definecolor{currentstroke}{rgb}{0.000000,0.000000,0.000000}%
\pgfsetstrokecolor{currentstroke}%
\pgfsetdash{}{0pt}%
\pgfpathmoveto{\pgfqpoint{8.403756in}{6.355925in}}%
\pgfpathcurveto{\pgfqpoint{8.414807in}{6.355925in}}{\pgfqpoint{8.425406in}{6.360316in}}{\pgfqpoint{8.433219in}{6.368129in}}%
\pgfpathcurveto{\pgfqpoint{8.441033in}{6.375943in}}{\pgfqpoint{8.445423in}{6.386542in}}{\pgfqpoint{8.445423in}{6.397592in}}%
\pgfpathcurveto{\pgfqpoint{8.445423in}{6.408642in}}{\pgfqpoint{8.441033in}{6.419241in}}{\pgfqpoint{8.433219in}{6.427055in}}%
\pgfpathcurveto{\pgfqpoint{8.425406in}{6.434869in}}{\pgfqpoint{8.414807in}{6.439259in}}{\pgfqpoint{8.403756in}{6.439259in}}%
\pgfpathcurveto{\pgfqpoint{8.392706in}{6.439259in}}{\pgfqpoint{8.382107in}{6.434869in}}{\pgfqpoint{8.374294in}{6.427055in}}%
\pgfpathcurveto{\pgfqpoint{8.366480in}{6.419241in}}{\pgfqpoint{8.362090in}{6.408642in}}{\pgfqpoint{8.362090in}{6.397592in}}%
\pgfpathcurveto{\pgfqpoint{8.362090in}{6.386542in}}{\pgfqpoint{8.366480in}{6.375943in}}{\pgfqpoint{8.374294in}{6.368129in}}%
\pgfpathcurveto{\pgfqpoint{8.382107in}{6.360316in}}{\pgfqpoint{8.392706in}{6.355925in}}{\pgfqpoint{8.403756in}{6.355925in}}%
\pgfpathclose%
\pgfusepath{stroke,fill}%
\end{pgfscope}%
\begin{pgfscope}%
\pgfpathrectangle{\pgfqpoint{7.640588in}{4.121437in}}{\pgfqpoint{5.699255in}{2.685432in}}%
\pgfusepath{clip}%
\pgfsetbuttcap%
\pgfsetroundjoin%
\definecolor{currentfill}{rgb}{0.000000,0.000000,0.000000}%
\pgfsetfillcolor{currentfill}%
\pgfsetlinewidth{1.003750pt}%
\definecolor{currentstroke}{rgb}{0.000000,0.000000,0.000000}%
\pgfsetstrokecolor{currentstroke}%
\pgfsetdash{}{0pt}%
\pgfpathmoveto{\pgfqpoint{8.039676in}{6.368981in}}%
\pgfpathcurveto{\pgfqpoint{8.050726in}{6.368981in}}{\pgfqpoint{8.061325in}{6.373371in}}{\pgfqpoint{8.069139in}{6.381184in}}%
\pgfpathcurveto{\pgfqpoint{8.076953in}{6.388998in}}{\pgfqpoint{8.081343in}{6.399597in}}{\pgfqpoint{8.081343in}{6.410647in}}%
\pgfpathcurveto{\pgfqpoint{8.081343in}{6.421697in}}{\pgfqpoint{8.076953in}{6.432296in}}{\pgfqpoint{8.069139in}{6.440110in}}%
\pgfpathcurveto{\pgfqpoint{8.061325in}{6.447924in}}{\pgfqpoint{8.050726in}{6.452314in}}{\pgfqpoint{8.039676in}{6.452314in}}%
\pgfpathcurveto{\pgfqpoint{8.028626in}{6.452314in}}{\pgfqpoint{8.018027in}{6.447924in}}{\pgfqpoint{8.010213in}{6.440110in}}%
\pgfpathcurveto{\pgfqpoint{8.002400in}{6.432296in}}{\pgfqpoint{7.998010in}{6.421697in}}{\pgfqpoint{7.998010in}{6.410647in}}%
\pgfpathcurveto{\pgfqpoint{7.998010in}{6.399597in}}{\pgfqpoint{8.002400in}{6.388998in}}{\pgfqpoint{8.010213in}{6.381184in}}%
\pgfpathcurveto{\pgfqpoint{8.018027in}{6.373371in}}{\pgfqpoint{8.028626in}{6.368981in}}{\pgfqpoint{8.039676in}{6.368981in}}%
\pgfpathclose%
\pgfusepath{stroke,fill}%
\end{pgfscope}%
\begin{pgfscope}%
\pgfpathrectangle{\pgfqpoint{7.640588in}{4.121437in}}{\pgfqpoint{5.699255in}{2.685432in}}%
\pgfusepath{clip}%
\pgfsetbuttcap%
\pgfsetroundjoin%
\definecolor{currentfill}{rgb}{0.000000,0.000000,0.000000}%
\pgfsetfillcolor{currentfill}%
\pgfsetlinewidth{1.003750pt}%
\definecolor{currentstroke}{rgb}{0.000000,0.000000,0.000000}%
\pgfsetstrokecolor{currentstroke}%
\pgfsetdash{}{0pt}%
\pgfpathmoveto{\pgfqpoint{7.899645in}{6.290650in}}%
\pgfpathcurveto{\pgfqpoint{7.910696in}{6.290650in}}{\pgfqpoint{7.921295in}{6.295040in}}{\pgfqpoint{7.929108in}{6.302854in}}%
\pgfpathcurveto{\pgfqpoint{7.936922in}{6.310668in}}{\pgfqpoint{7.941312in}{6.321267in}}{\pgfqpoint{7.941312in}{6.332317in}}%
\pgfpathcurveto{\pgfqpoint{7.941312in}{6.343367in}}{\pgfqpoint{7.936922in}{6.353966in}}{\pgfqpoint{7.929108in}{6.361779in}}%
\pgfpathcurveto{\pgfqpoint{7.921295in}{6.369593in}}{\pgfqpoint{7.910696in}{6.373983in}}{\pgfqpoint{7.899645in}{6.373983in}}%
\pgfpathcurveto{\pgfqpoint{7.888595in}{6.373983in}}{\pgfqpoint{7.877996in}{6.369593in}}{\pgfqpoint{7.870183in}{6.361779in}}%
\pgfpathcurveto{\pgfqpoint{7.862369in}{6.353966in}}{\pgfqpoint{7.857979in}{6.343367in}}{\pgfqpoint{7.857979in}{6.332317in}}%
\pgfpathcurveto{\pgfqpoint{7.857979in}{6.321267in}}{\pgfqpoint{7.862369in}{6.310668in}}{\pgfqpoint{7.870183in}{6.302854in}}%
\pgfpathcurveto{\pgfqpoint{7.877996in}{6.295040in}}{\pgfqpoint{7.888595in}{6.290650in}}{\pgfqpoint{7.899645in}{6.290650in}}%
\pgfpathclose%
\pgfusepath{stroke,fill}%
\end{pgfscope}%
\begin{pgfscope}%
\pgfpathrectangle{\pgfqpoint{7.640588in}{4.121437in}}{\pgfqpoint{5.699255in}{2.685432in}}%
\pgfusepath{clip}%
\pgfsetbuttcap%
\pgfsetroundjoin%
\definecolor{currentfill}{rgb}{0.000000,0.000000,0.000000}%
\pgfsetfillcolor{currentfill}%
\pgfsetlinewidth{1.003750pt}%
\definecolor{currentstroke}{rgb}{0.000000,0.000000,0.000000}%
\pgfsetstrokecolor{currentstroke}%
\pgfsetdash{}{0pt}%
\pgfpathmoveto{\pgfqpoint{7.955658in}{6.225375in}}%
\pgfpathcurveto{\pgfqpoint{7.966708in}{6.225375in}}{\pgfqpoint{7.977307in}{6.229765in}}{\pgfqpoint{7.985121in}{6.237578in}}%
\pgfpathcurveto{\pgfqpoint{7.992934in}{6.245392in}}{\pgfqpoint{7.997324in}{6.255991in}}{\pgfqpoint{7.997324in}{6.267041in}}%
\pgfpathcurveto{\pgfqpoint{7.997324in}{6.278091in}}{\pgfqpoint{7.992934in}{6.288690in}}{\pgfqpoint{7.985121in}{6.296504in}}%
\pgfpathcurveto{\pgfqpoint{7.977307in}{6.304318in}}{\pgfqpoint{7.966708in}{6.308708in}}{\pgfqpoint{7.955658in}{6.308708in}}%
\pgfpathcurveto{\pgfqpoint{7.944608in}{6.308708in}}{\pgfqpoint{7.934009in}{6.304318in}}{\pgfqpoint{7.926195in}{6.296504in}}%
\pgfpathcurveto{\pgfqpoint{7.918381in}{6.288690in}}{\pgfqpoint{7.913991in}{6.278091in}}{\pgfqpoint{7.913991in}{6.267041in}}%
\pgfpathcurveto{\pgfqpoint{7.913991in}{6.255991in}}{\pgfqpoint{7.918381in}{6.245392in}}{\pgfqpoint{7.926195in}{6.237578in}}%
\pgfpathcurveto{\pgfqpoint{7.934009in}{6.229765in}}{\pgfqpoint{7.944608in}{6.225375in}}{\pgfqpoint{7.955658in}{6.225375in}}%
\pgfpathclose%
\pgfusepath{stroke,fill}%
\end{pgfscope}%
\begin{pgfscope}%
\pgfpathrectangle{\pgfqpoint{7.640588in}{4.121437in}}{\pgfqpoint{5.699255in}{2.685432in}}%
\pgfusepath{clip}%
\pgfsetbuttcap%
\pgfsetroundjoin%
\definecolor{currentfill}{rgb}{0.000000,0.000000,0.000000}%
\pgfsetfillcolor{currentfill}%
\pgfsetlinewidth{1.003750pt}%
\definecolor{currentstroke}{rgb}{0.000000,0.000000,0.000000}%
\pgfsetstrokecolor{currentstroke}%
\pgfsetdash{}{0pt}%
\pgfpathmoveto{\pgfqpoint{8.067682in}{5.938163in}}%
\pgfpathcurveto{\pgfqpoint{8.078733in}{5.938163in}}{\pgfqpoint{8.089332in}{5.942553in}}{\pgfqpoint{8.097145in}{5.950366in}}%
\pgfpathcurveto{\pgfqpoint{8.104959in}{5.958180in}}{\pgfqpoint{8.109349in}{5.968779in}}{\pgfqpoint{8.109349in}{5.979829in}}%
\pgfpathcurveto{\pgfqpoint{8.109349in}{5.990879in}}{\pgfqpoint{8.104959in}{6.001478in}}{\pgfqpoint{8.097145in}{6.009292in}}%
\pgfpathcurveto{\pgfqpoint{8.089332in}{6.017106in}}{\pgfqpoint{8.078733in}{6.021496in}}{\pgfqpoint{8.067682in}{6.021496in}}%
\pgfpathcurveto{\pgfqpoint{8.056632in}{6.021496in}}{\pgfqpoint{8.046033in}{6.017106in}}{\pgfqpoint{8.038220in}{6.009292in}}%
\pgfpathcurveto{\pgfqpoint{8.030406in}{6.001478in}}{\pgfqpoint{8.026016in}{5.990879in}}{\pgfqpoint{8.026016in}{5.979829in}}%
\pgfpathcurveto{\pgfqpoint{8.026016in}{5.968779in}}{\pgfqpoint{8.030406in}{5.958180in}}{\pgfqpoint{8.038220in}{5.950366in}}%
\pgfpathcurveto{\pgfqpoint{8.046033in}{5.942553in}}{\pgfqpoint{8.056632in}{5.938163in}}{\pgfqpoint{8.067682in}{5.938163in}}%
\pgfpathclose%
\pgfusepath{stroke,fill}%
\end{pgfscope}%
\begin{pgfscope}%
\pgfpathrectangle{\pgfqpoint{7.640588in}{4.121437in}}{\pgfqpoint{5.699255in}{2.685432in}}%
\pgfusepath{clip}%
\pgfsetbuttcap%
\pgfsetroundjoin%
\definecolor{currentfill}{rgb}{0.000000,0.000000,0.000000}%
\pgfsetfillcolor{currentfill}%
\pgfsetlinewidth{1.003750pt}%
\definecolor{currentstroke}{rgb}{0.000000,0.000000,0.000000}%
\pgfsetstrokecolor{currentstroke}%
\pgfsetdash{}{0pt}%
\pgfpathmoveto{\pgfqpoint{8.347744in}{5.938163in}}%
\pgfpathcurveto{\pgfqpoint{8.358794in}{5.938163in}}{\pgfqpoint{8.369393in}{5.942553in}}{\pgfqpoint{8.377207in}{5.950366in}}%
\pgfpathcurveto{\pgfqpoint{8.385021in}{5.958180in}}{\pgfqpoint{8.389411in}{5.968779in}}{\pgfqpoint{8.389411in}{5.979829in}}%
\pgfpathcurveto{\pgfqpoint{8.389411in}{5.990879in}}{\pgfqpoint{8.385021in}{6.001478in}}{\pgfqpoint{8.377207in}{6.009292in}}%
\pgfpathcurveto{\pgfqpoint{8.369393in}{6.017106in}}{\pgfqpoint{8.358794in}{6.021496in}}{\pgfqpoint{8.347744in}{6.021496in}}%
\pgfpathcurveto{\pgfqpoint{8.336694in}{6.021496in}}{\pgfqpoint{8.326095in}{6.017106in}}{\pgfqpoint{8.318281in}{6.009292in}}%
\pgfpathcurveto{\pgfqpoint{8.310468in}{6.001478in}}{\pgfqpoint{8.306077in}{5.990879in}}{\pgfqpoint{8.306077in}{5.979829in}}%
\pgfpathcurveto{\pgfqpoint{8.306077in}{5.968779in}}{\pgfqpoint{8.310468in}{5.958180in}}{\pgfqpoint{8.318281in}{5.950366in}}%
\pgfpathcurveto{\pgfqpoint{8.326095in}{5.942553in}}{\pgfqpoint{8.336694in}{5.938163in}}{\pgfqpoint{8.347744in}{5.938163in}}%
\pgfpathclose%
\pgfusepath{stroke,fill}%
\end{pgfscope}%
\begin{pgfscope}%
\pgfpathrectangle{\pgfqpoint{7.640588in}{4.121437in}}{\pgfqpoint{5.699255in}{2.685432in}}%
\pgfusepath{clip}%
\pgfsetbuttcap%
\pgfsetroundjoin%
\definecolor{currentfill}{rgb}{0.000000,0.000000,0.000000}%
\pgfsetfillcolor{currentfill}%
\pgfsetlinewidth{1.003750pt}%
\definecolor{currentstroke}{rgb}{0.000000,0.000000,0.000000}%
\pgfsetstrokecolor{currentstroke}%
\pgfsetdash{}{0pt}%
\pgfpathmoveto{\pgfqpoint{8.879861in}{6.133989in}}%
\pgfpathcurveto{\pgfqpoint{8.890911in}{6.133989in}}{\pgfqpoint{8.901511in}{6.138379in}}{\pgfqpoint{8.909324in}{6.146193in}}%
\pgfpathcurveto{\pgfqpoint{8.917138in}{6.154006in}}{\pgfqpoint{8.921528in}{6.164605in}}{\pgfqpoint{8.921528in}{6.175656in}}%
\pgfpathcurveto{\pgfqpoint{8.921528in}{6.186706in}}{\pgfqpoint{8.917138in}{6.197305in}}{\pgfqpoint{8.909324in}{6.205118in}}%
\pgfpathcurveto{\pgfqpoint{8.901511in}{6.212932in}}{\pgfqpoint{8.890911in}{6.217322in}}{\pgfqpoint{8.879861in}{6.217322in}}%
\pgfpathcurveto{\pgfqpoint{8.868811in}{6.217322in}}{\pgfqpoint{8.858212in}{6.212932in}}{\pgfqpoint{8.850399in}{6.205118in}}%
\pgfpathcurveto{\pgfqpoint{8.842585in}{6.197305in}}{\pgfqpoint{8.838195in}{6.186706in}}{\pgfqpoint{8.838195in}{6.175656in}}%
\pgfpathcurveto{\pgfqpoint{8.838195in}{6.164605in}}{\pgfqpoint{8.842585in}{6.154006in}}{\pgfqpoint{8.850399in}{6.146193in}}%
\pgfpathcurveto{\pgfqpoint{8.858212in}{6.138379in}}{\pgfqpoint{8.868811in}{6.133989in}}{\pgfqpoint{8.879861in}{6.133989in}}%
\pgfpathclose%
\pgfusepath{stroke,fill}%
\end{pgfscope}%
\begin{pgfscope}%
\pgfpathrectangle{\pgfqpoint{7.640588in}{4.121437in}}{\pgfqpoint{5.699255in}{2.685432in}}%
\pgfusepath{clip}%
\pgfsetbuttcap%
\pgfsetroundjoin%
\definecolor{currentfill}{rgb}{0.000000,0.000000,0.000000}%
\pgfsetfillcolor{currentfill}%
\pgfsetlinewidth{1.003750pt}%
\definecolor{currentstroke}{rgb}{0.000000,0.000000,0.000000}%
\pgfsetstrokecolor{currentstroke}%
\pgfsetdash{}{0pt}%
\pgfpathmoveto{\pgfqpoint{9.047898in}{6.212319in}}%
\pgfpathcurveto{\pgfqpoint{9.058948in}{6.212319in}}{\pgfqpoint{9.069548in}{6.216710in}}{\pgfqpoint{9.077361in}{6.224523in}}%
\pgfpathcurveto{\pgfqpoint{9.085175in}{6.232337in}}{\pgfqpoint{9.089565in}{6.242936in}}{\pgfqpoint{9.089565in}{6.253986in}}%
\pgfpathcurveto{\pgfqpoint{9.089565in}{6.265036in}}{\pgfqpoint{9.085175in}{6.275635in}}{\pgfqpoint{9.077361in}{6.283449in}}%
\pgfpathcurveto{\pgfqpoint{9.069548in}{6.291263in}}{\pgfqpoint{9.058948in}{6.295653in}}{\pgfqpoint{9.047898in}{6.295653in}}%
\pgfpathcurveto{\pgfqpoint{9.036848in}{6.295653in}}{\pgfqpoint{9.026249in}{6.291263in}}{\pgfqpoint{9.018436in}{6.283449in}}%
\pgfpathcurveto{\pgfqpoint{9.010622in}{6.275635in}}{\pgfqpoint{9.006232in}{6.265036in}}{\pgfqpoint{9.006232in}{6.253986in}}%
\pgfpathcurveto{\pgfqpoint{9.006232in}{6.242936in}}{\pgfqpoint{9.010622in}{6.232337in}}{\pgfqpoint{9.018436in}{6.224523in}}%
\pgfpathcurveto{\pgfqpoint{9.026249in}{6.216710in}}{\pgfqpoint{9.036848in}{6.212319in}}{\pgfqpoint{9.047898in}{6.212319in}}%
\pgfpathclose%
\pgfusepath{stroke,fill}%
\end{pgfscope}%
\begin{pgfscope}%
\pgfpathrectangle{\pgfqpoint{7.640588in}{4.121437in}}{\pgfqpoint{5.699255in}{2.685432in}}%
\pgfusepath{clip}%
\pgfsetbuttcap%
\pgfsetroundjoin%
\definecolor{currentfill}{rgb}{0.000000,0.000000,0.000000}%
\pgfsetfillcolor{currentfill}%
\pgfsetlinewidth{1.003750pt}%
\definecolor{currentstroke}{rgb}{0.000000,0.000000,0.000000}%
\pgfsetstrokecolor{currentstroke}%
\pgfsetdash{}{0pt}%
\pgfpathmoveto{\pgfqpoint{9.271948in}{6.264540in}}%
\pgfpathcurveto{\pgfqpoint{9.282998in}{6.264540in}}{\pgfqpoint{9.293597in}{6.268930in}}{\pgfqpoint{9.301410in}{6.276744in}}%
\pgfpathcurveto{\pgfqpoint{9.309224in}{6.284557in}}{\pgfqpoint{9.313614in}{6.295156in}}{\pgfqpoint{9.313614in}{6.306206in}}%
\pgfpathcurveto{\pgfqpoint{9.313614in}{6.317257in}}{\pgfqpoint{9.309224in}{6.327856in}}{\pgfqpoint{9.301410in}{6.335669in}}%
\pgfpathcurveto{\pgfqpoint{9.293597in}{6.343483in}}{\pgfqpoint{9.282998in}{6.347873in}}{\pgfqpoint{9.271948in}{6.347873in}}%
\pgfpathcurveto{\pgfqpoint{9.260898in}{6.347873in}}{\pgfqpoint{9.250299in}{6.343483in}}{\pgfqpoint{9.242485in}{6.335669in}}%
\pgfpathcurveto{\pgfqpoint{9.234671in}{6.327856in}}{\pgfqpoint{9.230281in}{6.317257in}}{\pgfqpoint{9.230281in}{6.306206in}}%
\pgfpathcurveto{\pgfqpoint{9.230281in}{6.295156in}}{\pgfqpoint{9.234671in}{6.284557in}}{\pgfqpoint{9.242485in}{6.276744in}}%
\pgfpathcurveto{\pgfqpoint{9.250299in}{6.268930in}}{\pgfqpoint{9.260898in}{6.264540in}}{\pgfqpoint{9.271948in}{6.264540in}}%
\pgfpathclose%
\pgfusepath{stroke,fill}%
\end{pgfscope}%
\begin{pgfscope}%
\pgfpathrectangle{\pgfqpoint{7.640588in}{4.121437in}}{\pgfqpoint{5.699255in}{2.685432in}}%
\pgfusepath{clip}%
\pgfsetbuttcap%
\pgfsetroundjoin%
\definecolor{currentfill}{rgb}{0.000000,0.000000,0.000000}%
\pgfsetfillcolor{currentfill}%
\pgfsetlinewidth{1.003750pt}%
\definecolor{currentstroke}{rgb}{0.000000,0.000000,0.000000}%
\pgfsetstrokecolor{currentstroke}%
\pgfsetdash{}{0pt}%
\pgfpathmoveto{\pgfqpoint{9.748053in}{6.120934in}}%
\pgfpathcurveto{\pgfqpoint{9.759103in}{6.120934in}}{\pgfqpoint{9.769702in}{6.125324in}}{\pgfqpoint{9.777515in}{6.133138in}}%
\pgfpathcurveto{\pgfqpoint{9.785329in}{6.140951in}}{\pgfqpoint{9.789719in}{6.151550in}}{\pgfqpoint{9.789719in}{6.162601in}}%
\pgfpathcurveto{\pgfqpoint{9.789719in}{6.173651in}}{\pgfqpoint{9.785329in}{6.184250in}}{\pgfqpoint{9.777515in}{6.192063in}}%
\pgfpathcurveto{\pgfqpoint{9.769702in}{6.199877in}}{\pgfqpoint{9.759103in}{6.204267in}}{\pgfqpoint{9.748053in}{6.204267in}}%
\pgfpathcurveto{\pgfqpoint{9.737002in}{6.204267in}}{\pgfqpoint{9.726403in}{6.199877in}}{\pgfqpoint{9.718590in}{6.192063in}}%
\pgfpathcurveto{\pgfqpoint{9.710776in}{6.184250in}}{\pgfqpoint{9.706386in}{6.173651in}}{\pgfqpoint{9.706386in}{6.162601in}}%
\pgfpathcurveto{\pgfqpoint{9.706386in}{6.151550in}}{\pgfqpoint{9.710776in}{6.140951in}}{\pgfqpoint{9.718590in}{6.133138in}}%
\pgfpathcurveto{\pgfqpoint{9.726403in}{6.125324in}}{\pgfqpoint{9.737002in}{6.120934in}}{\pgfqpoint{9.748053in}{6.120934in}}%
\pgfpathclose%
\pgfusepath{stroke,fill}%
\end{pgfscope}%
\begin{pgfscope}%
\pgfpathrectangle{\pgfqpoint{7.640588in}{4.121437in}}{\pgfqpoint{5.699255in}{2.685432in}}%
\pgfusepath{clip}%
\pgfsetbuttcap%
\pgfsetroundjoin%
\definecolor{currentfill}{rgb}{0.000000,0.000000,0.000000}%
\pgfsetfillcolor{currentfill}%
\pgfsetlinewidth{1.003750pt}%
\definecolor{currentstroke}{rgb}{0.000000,0.000000,0.000000}%
\pgfsetstrokecolor{currentstroke}%
\pgfsetdash{}{0pt}%
\pgfpathmoveto{\pgfqpoint{10.196151in}{6.342870in}}%
\pgfpathcurveto{\pgfqpoint{10.207201in}{6.342870in}}{\pgfqpoint{10.217800in}{6.347261in}}{\pgfqpoint{10.225614in}{6.355074in}}%
\pgfpathcurveto{\pgfqpoint{10.233428in}{6.362888in}}{\pgfqpoint{10.237818in}{6.373487in}}{\pgfqpoint{10.237818in}{6.384537in}}%
\pgfpathcurveto{\pgfqpoint{10.237818in}{6.395587in}}{\pgfqpoint{10.233428in}{6.406186in}}{\pgfqpoint{10.225614in}{6.414000in}}%
\pgfpathcurveto{\pgfqpoint{10.217800in}{6.421813in}}{\pgfqpoint{10.207201in}{6.426204in}}{\pgfqpoint{10.196151in}{6.426204in}}%
\pgfpathcurveto{\pgfqpoint{10.185101in}{6.426204in}}{\pgfqpoint{10.174502in}{6.421813in}}{\pgfqpoint{10.166689in}{6.414000in}}%
\pgfpathcurveto{\pgfqpoint{10.158875in}{6.406186in}}{\pgfqpoint{10.154485in}{6.395587in}}{\pgfqpoint{10.154485in}{6.384537in}}%
\pgfpathcurveto{\pgfqpoint{10.154485in}{6.373487in}}{\pgfqpoint{10.158875in}{6.362888in}}{\pgfqpoint{10.166689in}{6.355074in}}%
\pgfpathcurveto{\pgfqpoint{10.174502in}{6.347261in}}{\pgfqpoint{10.185101in}{6.342870in}}{\pgfqpoint{10.196151in}{6.342870in}}%
\pgfpathclose%
\pgfusepath{stroke,fill}%
\end{pgfscope}%
\begin{pgfscope}%
\pgfpathrectangle{\pgfqpoint{7.640588in}{4.121437in}}{\pgfqpoint{5.699255in}{2.685432in}}%
\pgfusepath{clip}%
\pgfsetbuttcap%
\pgfsetroundjoin%
\definecolor{currentfill}{rgb}{0.000000,0.000000,0.000000}%
\pgfsetfillcolor{currentfill}%
\pgfsetlinewidth{1.003750pt}%
\definecolor{currentstroke}{rgb}{0.000000,0.000000,0.000000}%
\pgfsetstrokecolor{currentstroke}%
\pgfsetdash{}{0pt}%
\pgfpathmoveto{\pgfqpoint{10.224157in}{6.408146in}}%
\pgfpathcurveto{\pgfqpoint{10.235208in}{6.408146in}}{\pgfqpoint{10.245807in}{6.412536in}}{\pgfqpoint{10.253620in}{6.420350in}}%
\pgfpathcurveto{\pgfqpoint{10.261434in}{6.428163in}}{\pgfqpoint{10.265824in}{6.438762in}}{\pgfqpoint{10.265824in}{6.449812in}}%
\pgfpathcurveto{\pgfqpoint{10.265824in}{6.460863in}}{\pgfqpoint{10.261434in}{6.471462in}}{\pgfqpoint{10.253620in}{6.479275in}}%
\pgfpathcurveto{\pgfqpoint{10.245807in}{6.487089in}}{\pgfqpoint{10.235208in}{6.491479in}}{\pgfqpoint{10.224157in}{6.491479in}}%
\pgfpathcurveto{\pgfqpoint{10.213107in}{6.491479in}}{\pgfqpoint{10.202508in}{6.487089in}}{\pgfqpoint{10.194695in}{6.479275in}}%
\pgfpathcurveto{\pgfqpoint{10.186881in}{6.471462in}}{\pgfqpoint{10.182491in}{6.460863in}}{\pgfqpoint{10.182491in}{6.449812in}}%
\pgfpathcurveto{\pgfqpoint{10.182491in}{6.438762in}}{\pgfqpoint{10.186881in}{6.428163in}}{\pgfqpoint{10.194695in}{6.420350in}}%
\pgfpathcurveto{\pgfqpoint{10.202508in}{6.412536in}}{\pgfqpoint{10.213107in}{6.408146in}}{\pgfqpoint{10.224157in}{6.408146in}}%
\pgfpathclose%
\pgfusepath{stroke,fill}%
\end{pgfscope}%
\begin{pgfscope}%
\pgfpathrectangle{\pgfqpoint{7.640588in}{4.121437in}}{\pgfqpoint{5.699255in}{2.685432in}}%
\pgfusepath{clip}%
\pgfsetbuttcap%
\pgfsetroundjoin%
\definecolor{currentfill}{rgb}{0.000000,0.000000,0.000000}%
\pgfsetfillcolor{currentfill}%
\pgfsetlinewidth{1.003750pt}%
\definecolor{currentstroke}{rgb}{0.000000,0.000000,0.000000}%
\pgfsetstrokecolor{currentstroke}%
\pgfsetdash{}{0pt}%
\pgfpathmoveto{\pgfqpoint{10.560231in}{6.251485in}}%
\pgfpathcurveto{\pgfqpoint{10.571282in}{6.251485in}}{\pgfqpoint{10.581881in}{6.255875in}}{\pgfqpoint{10.589694in}{6.263689in}}%
\pgfpathcurveto{\pgfqpoint{10.597508in}{6.271502in}}{\pgfqpoint{10.601898in}{6.282101in}}{\pgfqpoint{10.601898in}{6.293151in}}%
\pgfpathcurveto{\pgfqpoint{10.601898in}{6.304202in}}{\pgfqpoint{10.597508in}{6.314801in}}{\pgfqpoint{10.589694in}{6.322614in}}%
\pgfpathcurveto{\pgfqpoint{10.581881in}{6.330428in}}{\pgfqpoint{10.571282in}{6.334818in}}{\pgfqpoint{10.560231in}{6.334818in}}%
\pgfpathcurveto{\pgfqpoint{10.549181in}{6.334818in}}{\pgfqpoint{10.538582in}{6.330428in}}{\pgfqpoint{10.530769in}{6.322614in}}%
\pgfpathcurveto{\pgfqpoint{10.522955in}{6.314801in}}{\pgfqpoint{10.518565in}{6.304202in}}{\pgfqpoint{10.518565in}{6.293151in}}%
\pgfpathcurveto{\pgfqpoint{10.518565in}{6.282101in}}{\pgfqpoint{10.522955in}{6.271502in}}{\pgfqpoint{10.530769in}{6.263689in}}%
\pgfpathcurveto{\pgfqpoint{10.538582in}{6.255875in}}{\pgfqpoint{10.549181in}{6.251485in}}{\pgfqpoint{10.560231in}{6.251485in}}%
\pgfpathclose%
\pgfusepath{stroke,fill}%
\end{pgfscope}%
\begin{pgfscope}%
\pgfpathrectangle{\pgfqpoint{7.640588in}{4.121437in}}{\pgfqpoint{5.699255in}{2.685432in}}%
\pgfusepath{clip}%
\pgfsetbuttcap%
\pgfsetroundjoin%
\definecolor{currentfill}{rgb}{0.000000,0.000000,0.000000}%
\pgfsetfillcolor{currentfill}%
\pgfsetlinewidth{1.003750pt}%
\definecolor{currentstroke}{rgb}{0.000000,0.000000,0.000000}%
\pgfsetstrokecolor{currentstroke}%
\pgfsetdash{}{0pt}%
\pgfpathmoveto{\pgfqpoint{10.084127in}{5.898997in}}%
\pgfpathcurveto{\pgfqpoint{10.095177in}{5.898997in}}{\pgfqpoint{10.105776in}{5.903388in}}{\pgfqpoint{10.113589in}{5.911201in}}%
\pgfpathcurveto{\pgfqpoint{10.121403in}{5.919015in}}{\pgfqpoint{10.125793in}{5.929614in}}{\pgfqpoint{10.125793in}{5.940664in}}%
\pgfpathcurveto{\pgfqpoint{10.125793in}{5.951714in}}{\pgfqpoint{10.121403in}{5.962313in}}{\pgfqpoint{10.113589in}{5.970127in}}%
\pgfpathcurveto{\pgfqpoint{10.105776in}{5.977940in}}{\pgfqpoint{10.095177in}{5.982331in}}{\pgfqpoint{10.084127in}{5.982331in}}%
\pgfpathcurveto{\pgfqpoint{10.073076in}{5.982331in}}{\pgfqpoint{10.062477in}{5.977940in}}{\pgfqpoint{10.054664in}{5.970127in}}%
\pgfpathcurveto{\pgfqpoint{10.046850in}{5.962313in}}{\pgfqpoint{10.042460in}{5.951714in}}{\pgfqpoint{10.042460in}{5.940664in}}%
\pgfpathcurveto{\pgfqpoint{10.042460in}{5.929614in}}{\pgfqpoint{10.046850in}{5.919015in}}{\pgfqpoint{10.054664in}{5.911201in}}%
\pgfpathcurveto{\pgfqpoint{10.062477in}{5.903388in}}{\pgfqpoint{10.073076in}{5.898997in}}{\pgfqpoint{10.084127in}{5.898997in}}%
\pgfpathclose%
\pgfusepath{stroke,fill}%
\end{pgfscope}%
\begin{pgfscope}%
\pgfpathrectangle{\pgfqpoint{7.640588in}{4.121437in}}{\pgfqpoint{5.699255in}{2.685432in}}%
\pgfusepath{clip}%
\pgfsetbuttcap%
\pgfsetroundjoin%
\definecolor{currentfill}{rgb}{0.000000,0.000000,0.000000}%
\pgfsetfillcolor{currentfill}%
\pgfsetlinewidth{1.003750pt}%
\definecolor{currentstroke}{rgb}{0.000000,0.000000,0.000000}%
\pgfsetstrokecolor{currentstroke}%
\pgfsetdash{}{0pt}%
\pgfpathmoveto{\pgfqpoint{10.112133in}{5.768446in}}%
\pgfpathcurveto{\pgfqpoint{10.123183in}{5.768446in}}{\pgfqpoint{10.133782in}{5.772837in}}{\pgfqpoint{10.141596in}{5.780650in}}%
\pgfpathcurveto{\pgfqpoint{10.149409in}{5.788464in}}{\pgfqpoint{10.153799in}{5.799063in}}{\pgfqpoint{10.153799in}{5.810113in}}%
\pgfpathcurveto{\pgfqpoint{10.153799in}{5.821163in}}{\pgfqpoint{10.149409in}{5.831762in}}{\pgfqpoint{10.141596in}{5.839576in}}%
\pgfpathcurveto{\pgfqpoint{10.133782in}{5.847389in}}{\pgfqpoint{10.123183in}{5.851780in}}{\pgfqpoint{10.112133in}{5.851780in}}%
\pgfpathcurveto{\pgfqpoint{10.101083in}{5.851780in}}{\pgfqpoint{10.090484in}{5.847389in}}{\pgfqpoint{10.082670in}{5.839576in}}%
\pgfpathcurveto{\pgfqpoint{10.074856in}{5.831762in}}{\pgfqpoint{10.070466in}{5.821163in}}{\pgfqpoint{10.070466in}{5.810113in}}%
\pgfpathcurveto{\pgfqpoint{10.070466in}{5.799063in}}{\pgfqpoint{10.074856in}{5.788464in}}{\pgfqpoint{10.082670in}{5.780650in}}%
\pgfpathcurveto{\pgfqpoint{10.090484in}{5.772837in}}{\pgfqpoint{10.101083in}{5.768446in}}{\pgfqpoint{10.112133in}{5.768446in}}%
\pgfpathclose%
\pgfusepath{stroke,fill}%
\end{pgfscope}%
\begin{pgfscope}%
\pgfpathrectangle{\pgfqpoint{7.640588in}{4.121437in}}{\pgfqpoint{5.699255in}{2.685432in}}%
\pgfusepath{clip}%
\pgfsetbuttcap%
\pgfsetroundjoin%
\definecolor{currentfill}{rgb}{0.000000,0.000000,0.000000}%
\pgfsetfillcolor{currentfill}%
\pgfsetlinewidth{1.003750pt}%
\definecolor{currentstroke}{rgb}{0.000000,0.000000,0.000000}%
\pgfsetstrokecolor{currentstroke}%
\pgfsetdash{}{0pt}%
\pgfpathmoveto{\pgfqpoint{9.580016in}{5.768446in}}%
\pgfpathcurveto{\pgfqpoint{9.591066in}{5.768446in}}{\pgfqpoint{9.601665in}{5.772837in}}{\pgfqpoint{9.609478in}{5.780650in}}%
\pgfpathcurveto{\pgfqpoint{9.617292in}{5.788464in}}{\pgfqpoint{9.621682in}{5.799063in}}{\pgfqpoint{9.621682in}{5.810113in}}%
\pgfpathcurveto{\pgfqpoint{9.621682in}{5.821163in}}{\pgfqpoint{9.617292in}{5.831762in}}{\pgfqpoint{9.609478in}{5.839576in}}%
\pgfpathcurveto{\pgfqpoint{9.601665in}{5.847389in}}{\pgfqpoint{9.591066in}{5.851780in}}{\pgfqpoint{9.580016in}{5.851780in}}%
\pgfpathcurveto{\pgfqpoint{9.568965in}{5.851780in}}{\pgfqpoint{9.558366in}{5.847389in}}{\pgfqpoint{9.550553in}{5.839576in}}%
\pgfpathcurveto{\pgfqpoint{9.542739in}{5.831762in}}{\pgfqpoint{9.538349in}{5.821163in}}{\pgfqpoint{9.538349in}{5.810113in}}%
\pgfpathcurveto{\pgfqpoint{9.538349in}{5.799063in}}{\pgfqpoint{9.542739in}{5.788464in}}{\pgfqpoint{9.550553in}{5.780650in}}%
\pgfpathcurveto{\pgfqpoint{9.558366in}{5.772837in}}{\pgfqpoint{9.568965in}{5.768446in}}{\pgfqpoint{9.580016in}{5.768446in}}%
\pgfpathclose%
\pgfusepath{stroke,fill}%
\end{pgfscope}%
\begin{pgfscope}%
\pgfpathrectangle{\pgfqpoint{7.640588in}{4.121437in}}{\pgfqpoint{5.699255in}{2.685432in}}%
\pgfusepath{clip}%
\pgfsetbuttcap%
\pgfsetroundjoin%
\definecolor{currentfill}{rgb}{0.000000,0.000000,0.000000}%
\pgfsetfillcolor{currentfill}%
\pgfsetlinewidth{1.003750pt}%
\definecolor{currentstroke}{rgb}{0.000000,0.000000,0.000000}%
\pgfsetstrokecolor{currentstroke}%
\pgfsetdash{}{0pt}%
\pgfpathmoveto{\pgfqpoint{9.580016in}{5.898997in}}%
\pgfpathcurveto{\pgfqpoint{9.591066in}{5.898997in}}{\pgfqpoint{9.601665in}{5.903388in}}{\pgfqpoint{9.609478in}{5.911201in}}%
\pgfpathcurveto{\pgfqpoint{9.617292in}{5.919015in}}{\pgfqpoint{9.621682in}{5.929614in}}{\pgfqpoint{9.621682in}{5.940664in}}%
\pgfpathcurveto{\pgfqpoint{9.621682in}{5.951714in}}{\pgfqpoint{9.617292in}{5.962313in}}{\pgfqpoint{9.609478in}{5.970127in}}%
\pgfpathcurveto{\pgfqpoint{9.601665in}{5.977940in}}{\pgfqpoint{9.591066in}{5.982331in}}{\pgfqpoint{9.580016in}{5.982331in}}%
\pgfpathcurveto{\pgfqpoint{9.568965in}{5.982331in}}{\pgfqpoint{9.558366in}{5.977940in}}{\pgfqpoint{9.550553in}{5.970127in}}%
\pgfpathcurveto{\pgfqpoint{9.542739in}{5.962313in}}{\pgfqpoint{9.538349in}{5.951714in}}{\pgfqpoint{9.538349in}{5.940664in}}%
\pgfpathcurveto{\pgfqpoint{9.538349in}{5.929614in}}{\pgfqpoint{9.542739in}{5.919015in}}{\pgfqpoint{9.550553in}{5.911201in}}%
\pgfpathcurveto{\pgfqpoint{9.558366in}{5.903388in}}{\pgfqpoint{9.568965in}{5.898997in}}{\pgfqpoint{9.580016in}{5.898997in}}%
\pgfpathclose%
\pgfusepath{stroke,fill}%
\end{pgfscope}%
\begin{pgfscope}%
\pgfpathrectangle{\pgfqpoint{7.640588in}{4.121437in}}{\pgfqpoint{5.699255in}{2.685432in}}%
\pgfusepath{clip}%
\pgfsetbuttcap%
\pgfsetroundjoin%
\definecolor{currentfill}{rgb}{0.000000,0.000000,0.000000}%
\pgfsetfillcolor{currentfill}%
\pgfsetlinewidth{1.003750pt}%
\definecolor{currentstroke}{rgb}{0.000000,0.000000,0.000000}%
\pgfsetstrokecolor{currentstroke}%
\pgfsetdash{}{0pt}%
\pgfpathmoveto{\pgfqpoint{8.823849in}{5.990383in}}%
\pgfpathcurveto{\pgfqpoint{8.834899in}{5.990383in}}{\pgfqpoint{8.845498in}{5.994773in}}{\pgfqpoint{8.853312in}{6.002587in}}%
\pgfpathcurveto{\pgfqpoint{8.861125in}{6.010400in}}{\pgfqpoint{8.865516in}{6.020999in}}{\pgfqpoint{8.865516in}{6.032050in}}%
\pgfpathcurveto{\pgfqpoint{8.865516in}{6.043100in}}{\pgfqpoint{8.861125in}{6.053699in}}{\pgfqpoint{8.853312in}{6.061512in}}%
\pgfpathcurveto{\pgfqpoint{8.845498in}{6.069326in}}{\pgfqpoint{8.834899in}{6.073716in}}{\pgfqpoint{8.823849in}{6.073716in}}%
\pgfpathcurveto{\pgfqpoint{8.812799in}{6.073716in}}{\pgfqpoint{8.802200in}{6.069326in}}{\pgfqpoint{8.794386in}{6.061512in}}%
\pgfpathcurveto{\pgfqpoint{8.786573in}{6.053699in}}{\pgfqpoint{8.782182in}{6.043100in}}{\pgfqpoint{8.782182in}{6.032050in}}%
\pgfpathcurveto{\pgfqpoint{8.782182in}{6.020999in}}{\pgfqpoint{8.786573in}{6.010400in}}{\pgfqpoint{8.794386in}{6.002587in}}%
\pgfpathcurveto{\pgfqpoint{8.802200in}{5.994773in}}{\pgfqpoint{8.812799in}{5.990383in}}{\pgfqpoint{8.823849in}{5.990383in}}%
\pgfpathclose%
\pgfusepath{stroke,fill}%
\end{pgfscope}%
\begin{pgfscope}%
\pgfpathrectangle{\pgfqpoint{7.640588in}{4.121437in}}{\pgfqpoint{5.699255in}{2.685432in}}%
\pgfusepath{clip}%
\pgfsetbuttcap%
\pgfsetroundjoin%
\definecolor{currentfill}{rgb}{0.000000,0.000000,0.000000}%
\pgfsetfillcolor{currentfill}%
\pgfsetlinewidth{1.003750pt}%
\definecolor{currentstroke}{rgb}{0.000000,0.000000,0.000000}%
\pgfsetstrokecolor{currentstroke}%
\pgfsetdash{}{0pt}%
\pgfpathmoveto{\pgfqpoint{8.655812in}{5.820667in}}%
\pgfpathcurveto{\pgfqpoint{8.666862in}{5.820667in}}{\pgfqpoint{8.677461in}{5.825057in}}{\pgfqpoint{8.685275in}{5.832871in}}%
\pgfpathcurveto{\pgfqpoint{8.693088in}{5.840684in}}{\pgfqpoint{8.697479in}{5.851283in}}{\pgfqpoint{8.697479in}{5.862333in}}%
\pgfpathcurveto{\pgfqpoint{8.697479in}{5.873384in}}{\pgfqpoint{8.693088in}{5.883983in}}{\pgfqpoint{8.685275in}{5.891796in}}%
\pgfpathcurveto{\pgfqpoint{8.677461in}{5.899610in}}{\pgfqpoint{8.666862in}{5.904000in}}{\pgfqpoint{8.655812in}{5.904000in}}%
\pgfpathcurveto{\pgfqpoint{8.644762in}{5.904000in}}{\pgfqpoint{8.634163in}{5.899610in}}{\pgfqpoint{8.626349in}{5.891796in}}%
\pgfpathcurveto{\pgfqpoint{8.618536in}{5.883983in}}{\pgfqpoint{8.614145in}{5.873384in}}{\pgfqpoint{8.614145in}{5.862333in}}%
\pgfpathcurveto{\pgfqpoint{8.614145in}{5.851283in}}{\pgfqpoint{8.618536in}{5.840684in}}{\pgfqpoint{8.626349in}{5.832871in}}%
\pgfpathcurveto{\pgfqpoint{8.634163in}{5.825057in}}{\pgfqpoint{8.644762in}{5.820667in}}{\pgfqpoint{8.655812in}{5.820667in}}%
\pgfpathclose%
\pgfusepath{stroke,fill}%
\end{pgfscope}%
\begin{pgfscope}%
\pgfpathrectangle{\pgfqpoint{7.640588in}{4.121437in}}{\pgfqpoint{5.699255in}{2.685432in}}%
\pgfusepath{clip}%
\pgfsetbuttcap%
\pgfsetroundjoin%
\definecolor{currentfill}{rgb}{0.000000,0.000000,0.000000}%
\pgfsetfillcolor{currentfill}%
\pgfsetlinewidth{1.003750pt}%
\definecolor{currentstroke}{rgb}{0.000000,0.000000,0.000000}%
\pgfsetstrokecolor{currentstroke}%
\pgfsetdash{}{0pt}%
\pgfpathmoveto{\pgfqpoint{9.075905in}{5.546510in}}%
\pgfpathcurveto{\pgfqpoint{9.086955in}{5.546510in}}{\pgfqpoint{9.097554in}{5.550900in}}{\pgfqpoint{9.105367in}{5.558714in}}%
\pgfpathcurveto{\pgfqpoint{9.113181in}{5.566527in}}{\pgfqpoint{9.117571in}{5.577126in}}{\pgfqpoint{9.117571in}{5.588177in}}%
\pgfpathcurveto{\pgfqpoint{9.117571in}{5.599227in}}{\pgfqpoint{9.113181in}{5.609826in}}{\pgfqpoint{9.105367in}{5.617639in}}%
\pgfpathcurveto{\pgfqpoint{9.097554in}{5.625453in}}{\pgfqpoint{9.086955in}{5.629843in}}{\pgfqpoint{9.075905in}{5.629843in}}%
\pgfpathcurveto{\pgfqpoint{9.064854in}{5.629843in}}{\pgfqpoint{9.054255in}{5.625453in}}{\pgfqpoint{9.046442in}{5.617639in}}%
\pgfpathcurveto{\pgfqpoint{9.038628in}{5.609826in}}{\pgfqpoint{9.034238in}{5.599227in}}{\pgfqpoint{9.034238in}{5.588177in}}%
\pgfpathcurveto{\pgfqpoint{9.034238in}{5.577126in}}{\pgfqpoint{9.038628in}{5.566527in}}{\pgfqpoint{9.046442in}{5.558714in}}%
\pgfpathcurveto{\pgfqpoint{9.054255in}{5.550900in}}{\pgfqpoint{9.064854in}{5.546510in}}{\pgfqpoint{9.075905in}{5.546510in}}%
\pgfpathclose%
\pgfusepath{stroke,fill}%
\end{pgfscope}%
\begin{pgfscope}%
\pgfpathrectangle{\pgfqpoint{7.640588in}{4.121437in}}{\pgfqpoint{5.699255in}{2.685432in}}%
\pgfusepath{clip}%
\pgfsetbuttcap%
\pgfsetroundjoin%
\definecolor{currentfill}{rgb}{0.000000,0.000000,0.000000}%
\pgfsetfillcolor{currentfill}%
\pgfsetlinewidth{1.003750pt}%
\definecolor{currentstroke}{rgb}{0.000000,0.000000,0.000000}%
\pgfsetstrokecolor{currentstroke}%
\pgfsetdash{}{0pt}%
\pgfpathmoveto{\pgfqpoint{9.271948in}{5.520400in}}%
\pgfpathcurveto{\pgfqpoint{9.282998in}{5.520400in}}{\pgfqpoint{9.293597in}{5.524790in}}{\pgfqpoint{9.301410in}{5.532604in}}%
\pgfpathcurveto{\pgfqpoint{9.309224in}{5.540417in}}{\pgfqpoint{9.313614in}{5.551016in}}{\pgfqpoint{9.313614in}{5.562066in}}%
\pgfpathcurveto{\pgfqpoint{9.313614in}{5.573116in}}{\pgfqpoint{9.309224in}{5.583716in}}{\pgfqpoint{9.301410in}{5.591529in}}%
\pgfpathcurveto{\pgfqpoint{9.293597in}{5.599343in}}{\pgfqpoint{9.282998in}{5.603733in}}{\pgfqpoint{9.271948in}{5.603733in}}%
\pgfpathcurveto{\pgfqpoint{9.260898in}{5.603733in}}{\pgfqpoint{9.250299in}{5.599343in}}{\pgfqpoint{9.242485in}{5.591529in}}%
\pgfpathcurveto{\pgfqpoint{9.234671in}{5.583716in}}{\pgfqpoint{9.230281in}{5.573116in}}{\pgfqpoint{9.230281in}{5.562066in}}%
\pgfpathcurveto{\pgfqpoint{9.230281in}{5.551016in}}{\pgfqpoint{9.234671in}{5.540417in}}{\pgfqpoint{9.242485in}{5.532604in}}%
\pgfpathcurveto{\pgfqpoint{9.250299in}{5.524790in}}{\pgfqpoint{9.260898in}{5.520400in}}{\pgfqpoint{9.271948in}{5.520400in}}%
\pgfpathclose%
\pgfusepath{stroke,fill}%
\end{pgfscope}%
\begin{pgfscope}%
\pgfpathrectangle{\pgfqpoint{7.640588in}{4.121437in}}{\pgfqpoint{5.699255in}{2.685432in}}%
\pgfusepath{clip}%
\pgfsetbuttcap%
\pgfsetroundjoin%
\definecolor{currentfill}{rgb}{0.000000,0.000000,0.000000}%
\pgfsetfillcolor{currentfill}%
\pgfsetlinewidth{1.003750pt}%
\definecolor{currentstroke}{rgb}{0.000000,0.000000,0.000000}%
\pgfsetstrokecolor{currentstroke}%
\pgfsetdash{}{0pt}%
\pgfpathmoveto{\pgfqpoint{9.355966in}{5.598730in}}%
\pgfpathcurveto{\pgfqpoint{9.367016in}{5.598730in}}{\pgfqpoint{9.377615in}{5.603120in}}{\pgfqpoint{9.385429in}{5.610934in}}%
\pgfpathcurveto{\pgfqpoint{9.393243in}{5.618748in}}{\pgfqpoint{9.397633in}{5.629347in}}{\pgfqpoint{9.397633in}{5.640397in}}%
\pgfpathcurveto{\pgfqpoint{9.397633in}{5.651447in}}{\pgfqpoint{9.393243in}{5.662046in}}{\pgfqpoint{9.385429in}{5.669860in}}%
\pgfpathcurveto{\pgfqpoint{9.377615in}{5.677673in}}{\pgfqpoint{9.367016in}{5.682064in}}{\pgfqpoint{9.355966in}{5.682064in}}%
\pgfpathcurveto{\pgfqpoint{9.344916in}{5.682064in}}{\pgfqpoint{9.334317in}{5.677673in}}{\pgfqpoint{9.326503in}{5.669860in}}%
\pgfpathcurveto{\pgfqpoint{9.318690in}{5.662046in}}{\pgfqpoint{9.314300in}{5.651447in}}{\pgfqpoint{9.314300in}{5.640397in}}%
\pgfpathcurveto{\pgfqpoint{9.314300in}{5.629347in}}{\pgfqpoint{9.318690in}{5.618748in}}{\pgfqpoint{9.326503in}{5.610934in}}%
\pgfpathcurveto{\pgfqpoint{9.334317in}{5.603120in}}{\pgfqpoint{9.344916in}{5.598730in}}{\pgfqpoint{9.355966in}{5.598730in}}%
\pgfpathclose%
\pgfusepath{stroke,fill}%
\end{pgfscope}%
\begin{pgfscope}%
\pgfpathrectangle{\pgfqpoint{7.640588in}{4.121437in}}{\pgfqpoint{5.699255in}{2.685432in}}%
\pgfusepath{clip}%
\pgfsetbuttcap%
\pgfsetroundjoin%
\definecolor{currentfill}{rgb}{0.000000,0.000000,0.000000}%
\pgfsetfillcolor{currentfill}%
\pgfsetlinewidth{1.003750pt}%
\definecolor{currentstroke}{rgb}{0.000000,0.000000,0.000000}%
\pgfsetstrokecolor{currentstroke}%
\pgfsetdash{}{0pt}%
\pgfpathmoveto{\pgfqpoint{9.383972in}{5.624840in}}%
\pgfpathcurveto{\pgfqpoint{9.395023in}{5.624840in}}{\pgfqpoint{9.405622in}{5.629231in}}{\pgfqpoint{9.413435in}{5.637044in}}%
\pgfpathcurveto{\pgfqpoint{9.421249in}{5.644858in}}{\pgfqpoint{9.425639in}{5.655457in}}{\pgfqpoint{9.425639in}{5.666507in}}%
\pgfpathcurveto{\pgfqpoint{9.425639in}{5.677557in}}{\pgfqpoint{9.421249in}{5.688156in}}{\pgfqpoint{9.413435in}{5.695970in}}%
\pgfpathcurveto{\pgfqpoint{9.405622in}{5.703783in}}{\pgfqpoint{9.395023in}{5.708174in}}{\pgfqpoint{9.383972in}{5.708174in}}%
\pgfpathcurveto{\pgfqpoint{9.372922in}{5.708174in}}{\pgfqpoint{9.362323in}{5.703783in}}{\pgfqpoint{9.354510in}{5.695970in}}%
\pgfpathcurveto{\pgfqpoint{9.346696in}{5.688156in}}{\pgfqpoint{9.342306in}{5.677557in}}{\pgfqpoint{9.342306in}{5.666507in}}%
\pgfpathcurveto{\pgfqpoint{9.342306in}{5.655457in}}{\pgfqpoint{9.346696in}{5.644858in}}{\pgfqpoint{9.354510in}{5.637044in}}%
\pgfpathcurveto{\pgfqpoint{9.362323in}{5.629231in}}{\pgfqpoint{9.372922in}{5.624840in}}{\pgfqpoint{9.383972in}{5.624840in}}%
\pgfpathclose%
\pgfusepath{stroke,fill}%
\end{pgfscope}%
\begin{pgfscope}%
\pgfpathrectangle{\pgfqpoint{7.640588in}{4.121437in}}{\pgfqpoint{5.699255in}{2.685432in}}%
\pgfusepath{clip}%
\pgfsetbuttcap%
\pgfsetroundjoin%
\definecolor{currentfill}{rgb}{0.000000,0.000000,0.000000}%
\pgfsetfillcolor{currentfill}%
\pgfsetlinewidth{1.003750pt}%
\definecolor{currentstroke}{rgb}{0.000000,0.000000,0.000000}%
\pgfsetstrokecolor{currentstroke}%
\pgfsetdash{}{0pt}%
\pgfpathmoveto{\pgfqpoint{9.860077in}{5.429014in}}%
\pgfpathcurveto{\pgfqpoint{9.871127in}{5.429014in}}{\pgfqpoint{9.881726in}{5.433404in}}{\pgfqpoint{9.889540in}{5.441218in}}%
\pgfpathcurveto{\pgfqpoint{9.897354in}{5.449032in}}{\pgfqpoint{9.901744in}{5.459631in}}{\pgfqpoint{9.901744in}{5.470681in}}%
\pgfpathcurveto{\pgfqpoint{9.901744in}{5.481731in}}{\pgfqpoint{9.897354in}{5.492330in}}{\pgfqpoint{9.889540in}{5.500143in}}%
\pgfpathcurveto{\pgfqpoint{9.881726in}{5.507957in}}{\pgfqpoint{9.871127in}{5.512347in}}{\pgfqpoint{9.860077in}{5.512347in}}%
\pgfpathcurveto{\pgfqpoint{9.849027in}{5.512347in}}{\pgfqpoint{9.838428in}{5.507957in}}{\pgfqpoint{9.830614in}{5.500143in}}%
\pgfpathcurveto{\pgfqpoint{9.822801in}{5.492330in}}{\pgfqpoint{9.818411in}{5.481731in}}{\pgfqpoint{9.818411in}{5.470681in}}%
\pgfpathcurveto{\pgfqpoint{9.818411in}{5.459631in}}{\pgfqpoint{9.822801in}{5.449032in}}{\pgfqpoint{9.830614in}{5.441218in}}%
\pgfpathcurveto{\pgfqpoint{9.838428in}{5.433404in}}{\pgfqpoint{9.849027in}{5.429014in}}{\pgfqpoint{9.860077in}{5.429014in}}%
\pgfpathclose%
\pgfusepath{stroke,fill}%
\end{pgfscope}%
\begin{pgfscope}%
\pgfpathrectangle{\pgfqpoint{7.640588in}{4.121437in}}{\pgfqpoint{5.699255in}{2.685432in}}%
\pgfusepath{clip}%
\pgfsetbuttcap%
\pgfsetroundjoin%
\definecolor{currentfill}{rgb}{0.000000,0.000000,0.000000}%
\pgfsetfillcolor{currentfill}%
\pgfsetlinewidth{1.003750pt}%
\definecolor{currentstroke}{rgb}{0.000000,0.000000,0.000000}%
\pgfsetstrokecolor{currentstroke}%
\pgfsetdash{}{0pt}%
\pgfpathmoveto{\pgfqpoint{9.608022in}{5.363739in}}%
\pgfpathcurveto{\pgfqpoint{9.619072in}{5.363739in}}{\pgfqpoint{9.629671in}{5.368129in}}{\pgfqpoint{9.637485in}{5.375942in}}%
\pgfpathcurveto{\pgfqpoint{9.645298in}{5.383756in}}{\pgfqpoint{9.649688in}{5.394355in}}{\pgfqpoint{9.649688in}{5.405405in}}%
\pgfpathcurveto{\pgfqpoint{9.649688in}{5.416455in}}{\pgfqpoint{9.645298in}{5.427054in}}{\pgfqpoint{9.637485in}{5.434868in}}%
\pgfpathcurveto{\pgfqpoint{9.629671in}{5.442682in}}{\pgfqpoint{9.619072in}{5.447072in}}{\pgfqpoint{9.608022in}{5.447072in}}%
\pgfpathcurveto{\pgfqpoint{9.596972in}{5.447072in}}{\pgfqpoint{9.586373in}{5.442682in}}{\pgfqpoint{9.578559in}{5.434868in}}%
\pgfpathcurveto{\pgfqpoint{9.570745in}{5.427054in}}{\pgfqpoint{9.566355in}{5.416455in}}{\pgfqpoint{9.566355in}{5.405405in}}%
\pgfpathcurveto{\pgfqpoint{9.566355in}{5.394355in}}{\pgfqpoint{9.570745in}{5.383756in}}{\pgfqpoint{9.578559in}{5.375942in}}%
\pgfpathcurveto{\pgfqpoint{9.586373in}{5.368129in}}{\pgfqpoint{9.596972in}{5.363739in}}{\pgfqpoint{9.608022in}{5.363739in}}%
\pgfpathclose%
\pgfusepath{stroke,fill}%
\end{pgfscope}%
\begin{pgfscope}%
\pgfpathrectangle{\pgfqpoint{7.640588in}{4.121437in}}{\pgfqpoint{5.699255in}{2.685432in}}%
\pgfusepath{clip}%
\pgfsetbuttcap%
\pgfsetroundjoin%
\definecolor{currentfill}{rgb}{0.000000,0.000000,0.000000}%
\pgfsetfillcolor{currentfill}%
\pgfsetlinewidth{1.003750pt}%
\definecolor{currentstroke}{rgb}{0.000000,0.000000,0.000000}%
\pgfsetstrokecolor{currentstroke}%
\pgfsetdash{}{0pt}%
\pgfpathmoveto{\pgfqpoint{9.580016in}{5.298463in}}%
\pgfpathcurveto{\pgfqpoint{9.591066in}{5.298463in}}{\pgfqpoint{9.601665in}{5.302853in}}{\pgfqpoint{9.609478in}{5.310667in}}%
\pgfpathcurveto{\pgfqpoint{9.617292in}{5.318481in}}{\pgfqpoint{9.621682in}{5.329080in}}{\pgfqpoint{9.621682in}{5.340130in}}%
\pgfpathcurveto{\pgfqpoint{9.621682in}{5.351180in}}{\pgfqpoint{9.617292in}{5.361779in}}{\pgfqpoint{9.609478in}{5.369593in}}%
\pgfpathcurveto{\pgfqpoint{9.601665in}{5.377406in}}{\pgfqpoint{9.591066in}{5.381796in}}{\pgfqpoint{9.580016in}{5.381796in}}%
\pgfpathcurveto{\pgfqpoint{9.568965in}{5.381796in}}{\pgfqpoint{9.558366in}{5.377406in}}{\pgfqpoint{9.550553in}{5.369593in}}%
\pgfpathcurveto{\pgfqpoint{9.542739in}{5.361779in}}{\pgfqpoint{9.538349in}{5.351180in}}{\pgfqpoint{9.538349in}{5.340130in}}%
\pgfpathcurveto{\pgfqpoint{9.538349in}{5.329080in}}{\pgfqpoint{9.542739in}{5.318481in}}{\pgfqpoint{9.550553in}{5.310667in}}%
\pgfpathcurveto{\pgfqpoint{9.558366in}{5.302853in}}{\pgfqpoint{9.568965in}{5.298463in}}{\pgfqpoint{9.580016in}{5.298463in}}%
\pgfpathclose%
\pgfusepath{stroke,fill}%
\end{pgfscope}%
\begin{pgfscope}%
\pgfpathrectangle{\pgfqpoint{7.640588in}{4.121437in}}{\pgfqpoint{5.699255in}{2.685432in}}%
\pgfusepath{clip}%
\pgfsetbuttcap%
\pgfsetroundjoin%
\definecolor{currentfill}{rgb}{0.000000,0.000000,0.000000}%
\pgfsetfillcolor{currentfill}%
\pgfsetlinewidth{1.003750pt}%
\definecolor{currentstroke}{rgb}{0.000000,0.000000,0.000000}%
\pgfsetstrokecolor{currentstroke}%
\pgfsetdash{}{0pt}%
\pgfpathmoveto{\pgfqpoint{8.487775in}{5.363739in}}%
\pgfpathcurveto{\pgfqpoint{8.498825in}{5.363739in}}{\pgfqpoint{8.509424in}{5.368129in}}{\pgfqpoint{8.517238in}{5.375942in}}%
\pgfpathcurveto{\pgfqpoint{8.525051in}{5.383756in}}{\pgfqpoint{8.529442in}{5.394355in}}{\pgfqpoint{8.529442in}{5.405405in}}%
\pgfpathcurveto{\pgfqpoint{8.529442in}{5.416455in}}{\pgfqpoint{8.525051in}{5.427054in}}{\pgfqpoint{8.517238in}{5.434868in}}%
\pgfpathcurveto{\pgfqpoint{8.509424in}{5.442682in}}{\pgfqpoint{8.498825in}{5.447072in}}{\pgfqpoint{8.487775in}{5.447072in}}%
\pgfpathcurveto{\pgfqpoint{8.476725in}{5.447072in}}{\pgfqpoint{8.466126in}{5.442682in}}{\pgfqpoint{8.458312in}{5.434868in}}%
\pgfpathcurveto{\pgfqpoint{8.450499in}{5.427054in}}{\pgfqpoint{8.446108in}{5.416455in}}{\pgfqpoint{8.446108in}{5.405405in}}%
\pgfpathcurveto{\pgfqpoint{8.446108in}{5.394355in}}{\pgfqpoint{8.450499in}{5.383756in}}{\pgfqpoint{8.458312in}{5.375942in}}%
\pgfpathcurveto{\pgfqpoint{8.466126in}{5.368129in}}{\pgfqpoint{8.476725in}{5.363739in}}{\pgfqpoint{8.487775in}{5.363739in}}%
\pgfpathclose%
\pgfusepath{stroke,fill}%
\end{pgfscope}%
\begin{pgfscope}%
\pgfpathrectangle{\pgfqpoint{7.640588in}{4.121437in}}{\pgfqpoint{5.699255in}{2.685432in}}%
\pgfusepath{clip}%
\pgfsetbuttcap%
\pgfsetroundjoin%
\definecolor{currentfill}{rgb}{0.000000,0.000000,0.000000}%
\pgfsetfillcolor{currentfill}%
\pgfsetlinewidth{1.003750pt}%
\definecolor{currentstroke}{rgb}{0.000000,0.000000,0.000000}%
\pgfsetstrokecolor{currentstroke}%
\pgfsetdash{}{0pt}%
\pgfpathmoveto{\pgfqpoint{8.543787in}{5.167912in}}%
\pgfpathcurveto{\pgfqpoint{8.554837in}{5.167912in}}{\pgfqpoint{8.565436in}{5.172302in}}{\pgfqpoint{8.573250in}{5.180116in}}%
\pgfpathcurveto{\pgfqpoint{8.581064in}{5.187930in}}{\pgfqpoint{8.585454in}{5.198529in}}{\pgfqpoint{8.585454in}{5.209579in}}%
\pgfpathcurveto{\pgfqpoint{8.585454in}{5.220629in}}{\pgfqpoint{8.581064in}{5.231228in}}{\pgfqpoint{8.573250in}{5.239042in}}%
\pgfpathcurveto{\pgfqpoint{8.565436in}{5.246855in}}{\pgfqpoint{8.554837in}{5.251246in}}{\pgfqpoint{8.543787in}{5.251246in}}%
\pgfpathcurveto{\pgfqpoint{8.532737in}{5.251246in}}{\pgfqpoint{8.522138in}{5.246855in}}{\pgfqpoint{8.514325in}{5.239042in}}%
\pgfpathcurveto{\pgfqpoint{8.506511in}{5.231228in}}{\pgfqpoint{8.502121in}{5.220629in}}{\pgfqpoint{8.502121in}{5.209579in}}%
\pgfpathcurveto{\pgfqpoint{8.502121in}{5.198529in}}{\pgfqpoint{8.506511in}{5.187930in}}{\pgfqpoint{8.514325in}{5.180116in}}%
\pgfpathcurveto{\pgfqpoint{8.522138in}{5.172302in}}{\pgfqpoint{8.532737in}{5.167912in}}{\pgfqpoint{8.543787in}{5.167912in}}%
\pgfpathclose%
\pgfusepath{stroke,fill}%
\end{pgfscope}%
\begin{pgfscope}%
\pgfpathrectangle{\pgfqpoint{7.640588in}{4.121437in}}{\pgfqpoint{5.699255in}{2.685432in}}%
\pgfusepath{clip}%
\pgfsetbuttcap%
\pgfsetroundjoin%
\definecolor{currentfill}{rgb}{0.000000,0.000000,0.000000}%
\pgfsetfillcolor{currentfill}%
\pgfsetlinewidth{1.003750pt}%
\definecolor{currentstroke}{rgb}{0.000000,0.000000,0.000000}%
\pgfsetstrokecolor{currentstroke}%
\pgfsetdash{}{0pt}%
\pgfpathmoveto{\pgfqpoint{8.571793in}{5.011251in}}%
\pgfpathcurveto{\pgfqpoint{8.582844in}{5.011251in}}{\pgfqpoint{8.593443in}{5.015641in}}{\pgfqpoint{8.601256in}{5.023455in}}%
\pgfpathcurveto{\pgfqpoint{8.609070in}{5.031269in}}{\pgfqpoint{8.613460in}{5.041868in}}{\pgfqpoint{8.613460in}{5.052918in}}%
\pgfpathcurveto{\pgfqpoint{8.613460in}{5.063968in}}{\pgfqpoint{8.609070in}{5.074567in}}{\pgfqpoint{8.601256in}{5.082381in}}%
\pgfpathcurveto{\pgfqpoint{8.593443in}{5.090194in}}{\pgfqpoint{8.582844in}{5.094584in}}{\pgfqpoint{8.571793in}{5.094584in}}%
\pgfpathcurveto{\pgfqpoint{8.560743in}{5.094584in}}{\pgfqpoint{8.550144in}{5.090194in}}{\pgfqpoint{8.542331in}{5.082381in}}%
\pgfpathcurveto{\pgfqpoint{8.534517in}{5.074567in}}{\pgfqpoint{8.530127in}{5.063968in}}{\pgfqpoint{8.530127in}{5.052918in}}%
\pgfpathcurveto{\pgfqpoint{8.530127in}{5.041868in}}{\pgfqpoint{8.534517in}{5.031269in}}{\pgfqpoint{8.542331in}{5.023455in}}%
\pgfpathcurveto{\pgfqpoint{8.550144in}{5.015641in}}{\pgfqpoint{8.560743in}{5.011251in}}{\pgfqpoint{8.571793in}{5.011251in}}%
\pgfpathclose%
\pgfusepath{stroke,fill}%
\end{pgfscope}%
\begin{pgfscope}%
\pgfpathrectangle{\pgfqpoint{7.640588in}{4.121437in}}{\pgfqpoint{5.699255in}{2.685432in}}%
\pgfusepath{clip}%
\pgfsetbuttcap%
\pgfsetroundjoin%
\definecolor{currentfill}{rgb}{0.000000,0.000000,0.000000}%
\pgfsetfillcolor{currentfill}%
\pgfsetlinewidth{1.003750pt}%
\definecolor{currentstroke}{rgb}{0.000000,0.000000,0.000000}%
\pgfsetstrokecolor{currentstroke}%
\pgfsetdash{}{0pt}%
\pgfpathmoveto{\pgfqpoint{8.543787in}{4.906810in}}%
\pgfpathcurveto{\pgfqpoint{8.554837in}{4.906810in}}{\pgfqpoint{8.565436in}{4.911201in}}{\pgfqpoint{8.573250in}{4.919014in}}%
\pgfpathcurveto{\pgfqpoint{8.581064in}{4.926828in}}{\pgfqpoint{8.585454in}{4.937427in}}{\pgfqpoint{8.585454in}{4.948477in}}%
\pgfpathcurveto{\pgfqpoint{8.585454in}{4.959527in}}{\pgfqpoint{8.581064in}{4.970126in}}{\pgfqpoint{8.573250in}{4.977940in}}%
\pgfpathcurveto{\pgfqpoint{8.565436in}{4.985753in}}{\pgfqpoint{8.554837in}{4.990144in}}{\pgfqpoint{8.543787in}{4.990144in}}%
\pgfpathcurveto{\pgfqpoint{8.532737in}{4.990144in}}{\pgfqpoint{8.522138in}{4.985753in}}{\pgfqpoint{8.514325in}{4.977940in}}%
\pgfpathcurveto{\pgfqpoint{8.506511in}{4.970126in}}{\pgfqpoint{8.502121in}{4.959527in}}{\pgfqpoint{8.502121in}{4.948477in}}%
\pgfpathcurveto{\pgfqpoint{8.502121in}{4.937427in}}{\pgfqpoint{8.506511in}{4.926828in}}{\pgfqpoint{8.514325in}{4.919014in}}%
\pgfpathcurveto{\pgfqpoint{8.522138in}{4.911201in}}{\pgfqpoint{8.532737in}{4.906810in}}{\pgfqpoint{8.543787in}{4.906810in}}%
\pgfpathclose%
\pgfusepath{stroke,fill}%
\end{pgfscope}%
\begin{pgfscope}%
\pgfpathrectangle{\pgfqpoint{7.640588in}{4.121437in}}{\pgfqpoint{5.699255in}{2.685432in}}%
\pgfusepath{clip}%
\pgfsetbuttcap%
\pgfsetroundjoin%
\definecolor{currentfill}{rgb}{0.000000,0.000000,0.000000}%
\pgfsetfillcolor{currentfill}%
\pgfsetlinewidth{1.003750pt}%
\definecolor{currentstroke}{rgb}{0.000000,0.000000,0.000000}%
\pgfsetstrokecolor{currentstroke}%
\pgfsetdash{}{0pt}%
\pgfpathmoveto{\pgfqpoint{8.039676in}{4.906810in}}%
\pgfpathcurveto{\pgfqpoint{8.050726in}{4.906810in}}{\pgfqpoint{8.061325in}{4.911201in}}{\pgfqpoint{8.069139in}{4.919014in}}%
\pgfpathcurveto{\pgfqpoint{8.076953in}{4.926828in}}{\pgfqpoint{8.081343in}{4.937427in}}{\pgfqpoint{8.081343in}{4.948477in}}%
\pgfpathcurveto{\pgfqpoint{8.081343in}{4.959527in}}{\pgfqpoint{8.076953in}{4.970126in}}{\pgfqpoint{8.069139in}{4.977940in}}%
\pgfpathcurveto{\pgfqpoint{8.061325in}{4.985753in}}{\pgfqpoint{8.050726in}{4.990144in}}{\pgfqpoint{8.039676in}{4.990144in}}%
\pgfpathcurveto{\pgfqpoint{8.028626in}{4.990144in}}{\pgfqpoint{8.018027in}{4.985753in}}{\pgfqpoint{8.010213in}{4.977940in}}%
\pgfpathcurveto{\pgfqpoint{8.002400in}{4.970126in}}{\pgfqpoint{7.998010in}{4.959527in}}{\pgfqpoint{7.998010in}{4.948477in}}%
\pgfpathcurveto{\pgfqpoint{7.998010in}{4.937427in}}{\pgfqpoint{8.002400in}{4.926828in}}{\pgfqpoint{8.010213in}{4.919014in}}%
\pgfpathcurveto{\pgfqpoint{8.018027in}{4.911201in}}{\pgfqpoint{8.028626in}{4.906810in}}{\pgfqpoint{8.039676in}{4.906810in}}%
\pgfpathclose%
\pgfusepath{stroke,fill}%
\end{pgfscope}%
\begin{pgfscope}%
\pgfpathrectangle{\pgfqpoint{7.640588in}{4.121437in}}{\pgfqpoint{5.699255in}{2.685432in}}%
\pgfusepath{clip}%
\pgfsetbuttcap%
\pgfsetroundjoin%
\definecolor{currentfill}{rgb}{0.000000,0.000000,0.000000}%
\pgfsetfillcolor{currentfill}%
\pgfsetlinewidth{1.003750pt}%
\definecolor{currentstroke}{rgb}{0.000000,0.000000,0.000000}%
\pgfsetstrokecolor{currentstroke}%
\pgfsetdash{}{0pt}%
\pgfpathmoveto{\pgfqpoint{8.179707in}{4.867645in}}%
\pgfpathcurveto{\pgfqpoint{8.190757in}{4.867645in}}{\pgfqpoint{8.201356in}{4.872035in}}{\pgfqpoint{8.209170in}{4.879849in}}%
\pgfpathcurveto{\pgfqpoint{8.216984in}{4.887663in}}{\pgfqpoint{8.221374in}{4.898262in}}{\pgfqpoint{8.221374in}{4.909312in}}%
\pgfpathcurveto{\pgfqpoint{8.221374in}{4.920362in}}{\pgfqpoint{8.216984in}{4.930961in}}{\pgfqpoint{8.209170in}{4.938775in}}%
\pgfpathcurveto{\pgfqpoint{8.201356in}{4.946588in}}{\pgfqpoint{8.190757in}{4.950978in}}{\pgfqpoint{8.179707in}{4.950978in}}%
\pgfpathcurveto{\pgfqpoint{8.168657in}{4.950978in}}{\pgfqpoint{8.158058in}{4.946588in}}{\pgfqpoint{8.150244in}{4.938775in}}%
\pgfpathcurveto{\pgfqpoint{8.142431in}{4.930961in}}{\pgfqpoint{8.138040in}{4.920362in}}{\pgfqpoint{8.138040in}{4.909312in}}%
\pgfpathcurveto{\pgfqpoint{8.138040in}{4.898262in}}{\pgfqpoint{8.142431in}{4.887663in}}{\pgfqpoint{8.150244in}{4.879849in}}%
\pgfpathcurveto{\pgfqpoint{8.158058in}{4.872035in}}{\pgfqpoint{8.168657in}{4.867645in}}{\pgfqpoint{8.179707in}{4.867645in}}%
\pgfpathclose%
\pgfusepath{stroke,fill}%
\end{pgfscope}%
\begin{pgfscope}%
\pgfpathrectangle{\pgfqpoint{7.640588in}{4.121437in}}{\pgfqpoint{5.699255in}{2.685432in}}%
\pgfusepath{clip}%
\pgfsetbuttcap%
\pgfsetroundjoin%
\definecolor{currentfill}{rgb}{0.000000,0.000000,0.000000}%
\pgfsetfillcolor{currentfill}%
\pgfsetlinewidth{1.003750pt}%
\definecolor{currentstroke}{rgb}{0.000000,0.000000,0.000000}%
\pgfsetstrokecolor{currentstroke}%
\pgfsetdash{}{0pt}%
\pgfpathmoveto{\pgfqpoint{8.571793in}{4.789315in}}%
\pgfpathcurveto{\pgfqpoint{8.582844in}{4.789315in}}{\pgfqpoint{8.593443in}{4.793705in}}{\pgfqpoint{8.601256in}{4.801518in}}%
\pgfpathcurveto{\pgfqpoint{8.609070in}{4.809332in}}{\pgfqpoint{8.613460in}{4.819931in}}{\pgfqpoint{8.613460in}{4.830981in}}%
\pgfpathcurveto{\pgfqpoint{8.613460in}{4.842031in}}{\pgfqpoint{8.609070in}{4.852630in}}{\pgfqpoint{8.601256in}{4.860444in}}%
\pgfpathcurveto{\pgfqpoint{8.593443in}{4.868258in}}{\pgfqpoint{8.582844in}{4.872648in}}{\pgfqpoint{8.571793in}{4.872648in}}%
\pgfpathcurveto{\pgfqpoint{8.560743in}{4.872648in}}{\pgfqpoint{8.550144in}{4.868258in}}{\pgfqpoint{8.542331in}{4.860444in}}%
\pgfpathcurveto{\pgfqpoint{8.534517in}{4.852630in}}{\pgfqpoint{8.530127in}{4.842031in}}{\pgfqpoint{8.530127in}{4.830981in}}%
\pgfpathcurveto{\pgfqpoint{8.530127in}{4.819931in}}{\pgfqpoint{8.534517in}{4.809332in}}{\pgfqpoint{8.542331in}{4.801518in}}%
\pgfpathcurveto{\pgfqpoint{8.550144in}{4.793705in}}{\pgfqpoint{8.560743in}{4.789315in}}{\pgfqpoint{8.571793in}{4.789315in}}%
\pgfpathclose%
\pgfusepath{stroke,fill}%
\end{pgfscope}%
\begin{pgfscope}%
\pgfpathrectangle{\pgfqpoint{7.640588in}{4.121437in}}{\pgfqpoint{5.699255in}{2.685432in}}%
\pgfusepath{clip}%
\pgfsetbuttcap%
\pgfsetroundjoin%
\definecolor{currentfill}{rgb}{0.000000,0.000000,0.000000}%
\pgfsetfillcolor{currentfill}%
\pgfsetlinewidth{1.003750pt}%
\definecolor{currentstroke}{rgb}{0.000000,0.000000,0.000000}%
\pgfsetstrokecolor{currentstroke}%
\pgfsetdash{}{0pt}%
\pgfpathmoveto{\pgfqpoint{9.327960in}{4.815425in}}%
\pgfpathcurveto{\pgfqpoint{9.339010in}{4.815425in}}{\pgfqpoint{9.349609in}{4.819815in}}{\pgfqpoint{9.357423in}{4.827629in}}%
\pgfpathcurveto{\pgfqpoint{9.365236in}{4.835442in}}{\pgfqpoint{9.369627in}{4.846041in}}{\pgfqpoint{9.369627in}{4.857091in}}%
\pgfpathcurveto{\pgfqpoint{9.369627in}{4.868142in}}{\pgfqpoint{9.365236in}{4.878741in}}{\pgfqpoint{9.357423in}{4.886554in}}%
\pgfpathcurveto{\pgfqpoint{9.349609in}{4.894368in}}{\pgfqpoint{9.339010in}{4.898758in}}{\pgfqpoint{9.327960in}{4.898758in}}%
\pgfpathcurveto{\pgfqpoint{9.316910in}{4.898758in}}{\pgfqpoint{9.306311in}{4.894368in}}{\pgfqpoint{9.298497in}{4.886554in}}%
\pgfpathcurveto{\pgfqpoint{9.290684in}{4.878741in}}{\pgfqpoint{9.286293in}{4.868142in}}{\pgfqpoint{9.286293in}{4.857091in}}%
\pgfpathcurveto{\pgfqpoint{9.286293in}{4.846041in}}{\pgfqpoint{9.290684in}{4.835442in}}{\pgfqpoint{9.298497in}{4.827629in}}%
\pgfpathcurveto{\pgfqpoint{9.306311in}{4.819815in}}{\pgfqpoint{9.316910in}{4.815425in}}{\pgfqpoint{9.327960in}{4.815425in}}%
\pgfpathclose%
\pgfusepath{stroke,fill}%
\end{pgfscope}%
\begin{pgfscope}%
\pgfpathrectangle{\pgfqpoint{7.640588in}{4.121437in}}{\pgfqpoint{5.699255in}{2.685432in}}%
\pgfusepath{clip}%
\pgfsetbuttcap%
\pgfsetroundjoin%
\definecolor{currentfill}{rgb}{0.000000,0.000000,0.000000}%
\pgfsetfillcolor{currentfill}%
\pgfsetlinewidth{1.003750pt}%
\definecolor{currentstroke}{rgb}{0.000000,0.000000,0.000000}%
\pgfsetstrokecolor{currentstroke}%
\pgfsetdash{}{0pt}%
\pgfpathmoveto{\pgfqpoint{9.299954in}{4.789315in}}%
\pgfpathcurveto{\pgfqpoint{9.311004in}{4.789315in}}{\pgfqpoint{9.321603in}{4.793705in}}{\pgfqpoint{9.329417in}{4.801518in}}%
\pgfpathcurveto{\pgfqpoint{9.337230in}{4.809332in}}{\pgfqpoint{9.341621in}{4.819931in}}{\pgfqpoint{9.341621in}{4.830981in}}%
\pgfpathcurveto{\pgfqpoint{9.341621in}{4.842031in}}{\pgfqpoint{9.337230in}{4.852630in}}{\pgfqpoint{9.329417in}{4.860444in}}%
\pgfpathcurveto{\pgfqpoint{9.321603in}{4.868258in}}{\pgfqpoint{9.311004in}{4.872648in}}{\pgfqpoint{9.299954in}{4.872648in}}%
\pgfpathcurveto{\pgfqpoint{9.288904in}{4.872648in}}{\pgfqpoint{9.278305in}{4.868258in}}{\pgfqpoint{9.270491in}{4.860444in}}%
\pgfpathcurveto{\pgfqpoint{9.262677in}{4.852630in}}{\pgfqpoint{9.258287in}{4.842031in}}{\pgfqpoint{9.258287in}{4.830981in}}%
\pgfpathcurveto{\pgfqpoint{9.258287in}{4.819931in}}{\pgfqpoint{9.262677in}{4.809332in}}{\pgfqpoint{9.270491in}{4.801518in}}%
\pgfpathcurveto{\pgfqpoint{9.278305in}{4.793705in}}{\pgfqpoint{9.288904in}{4.789315in}}{\pgfqpoint{9.299954in}{4.789315in}}%
\pgfpathclose%
\pgfusepath{stroke,fill}%
\end{pgfscope}%
\begin{pgfscope}%
\pgfpathrectangle{\pgfqpoint{7.640588in}{4.121437in}}{\pgfqpoint{5.699255in}{2.685432in}}%
\pgfusepath{clip}%
\pgfsetbuttcap%
\pgfsetroundjoin%
\definecolor{currentfill}{rgb}{0.000000,0.000000,0.000000}%
\pgfsetfillcolor{currentfill}%
\pgfsetlinewidth{1.003750pt}%
\definecolor{currentstroke}{rgb}{0.000000,0.000000,0.000000}%
\pgfsetstrokecolor{currentstroke}%
\pgfsetdash{}{0pt}%
\pgfpathmoveto{\pgfqpoint{9.047898in}{4.697929in}}%
\pgfpathcurveto{\pgfqpoint{9.058948in}{4.697929in}}{\pgfqpoint{9.069548in}{4.702319in}}{\pgfqpoint{9.077361in}{4.710133in}}%
\pgfpathcurveto{\pgfqpoint{9.085175in}{4.717946in}}{\pgfqpoint{9.089565in}{4.728546in}}{\pgfqpoint{9.089565in}{4.739596in}}%
\pgfpathcurveto{\pgfqpoint{9.089565in}{4.750646in}}{\pgfqpoint{9.085175in}{4.761245in}}{\pgfqpoint{9.077361in}{4.769058in}}%
\pgfpathcurveto{\pgfqpoint{9.069548in}{4.776872in}}{\pgfqpoint{9.058948in}{4.781262in}}{\pgfqpoint{9.047898in}{4.781262in}}%
\pgfpathcurveto{\pgfqpoint{9.036848in}{4.781262in}}{\pgfqpoint{9.026249in}{4.776872in}}{\pgfqpoint{9.018436in}{4.769058in}}%
\pgfpathcurveto{\pgfqpoint{9.010622in}{4.761245in}}{\pgfqpoint{9.006232in}{4.750646in}}{\pgfqpoint{9.006232in}{4.739596in}}%
\pgfpathcurveto{\pgfqpoint{9.006232in}{4.728546in}}{\pgfqpoint{9.010622in}{4.717946in}}{\pgfqpoint{9.018436in}{4.710133in}}%
\pgfpathcurveto{\pgfqpoint{9.026249in}{4.702319in}}{\pgfqpoint{9.036848in}{4.697929in}}{\pgfqpoint{9.047898in}{4.697929in}}%
\pgfpathclose%
\pgfusepath{stroke,fill}%
\end{pgfscope}%
\begin{pgfscope}%
\pgfpathrectangle{\pgfqpoint{7.640588in}{4.121437in}}{\pgfqpoint{5.699255in}{2.685432in}}%
\pgfusepath{clip}%
\pgfsetbuttcap%
\pgfsetroundjoin%
\definecolor{currentfill}{rgb}{0.000000,0.000000,0.000000}%
\pgfsetfillcolor{currentfill}%
\pgfsetlinewidth{1.003750pt}%
\definecolor{currentstroke}{rgb}{0.000000,0.000000,0.000000}%
\pgfsetstrokecolor{currentstroke}%
\pgfsetdash{}{0pt}%
\pgfpathmoveto{\pgfqpoint{9.075905in}{4.606543in}}%
\pgfpathcurveto{\pgfqpoint{9.086955in}{4.606543in}}{\pgfqpoint{9.097554in}{4.610934in}}{\pgfqpoint{9.105367in}{4.618747in}}%
\pgfpathcurveto{\pgfqpoint{9.113181in}{4.626561in}}{\pgfqpoint{9.117571in}{4.637160in}}{\pgfqpoint{9.117571in}{4.648210in}}%
\pgfpathcurveto{\pgfqpoint{9.117571in}{4.659260in}}{\pgfqpoint{9.113181in}{4.669859in}}{\pgfqpoint{9.105367in}{4.677673in}}%
\pgfpathcurveto{\pgfqpoint{9.097554in}{4.685486in}}{\pgfqpoint{9.086955in}{4.689877in}}{\pgfqpoint{9.075905in}{4.689877in}}%
\pgfpathcurveto{\pgfqpoint{9.064854in}{4.689877in}}{\pgfqpoint{9.054255in}{4.685486in}}{\pgfqpoint{9.046442in}{4.677673in}}%
\pgfpathcurveto{\pgfqpoint{9.038628in}{4.669859in}}{\pgfqpoint{9.034238in}{4.659260in}}{\pgfqpoint{9.034238in}{4.648210in}}%
\pgfpathcurveto{\pgfqpoint{9.034238in}{4.637160in}}{\pgfqpoint{9.038628in}{4.626561in}}{\pgfqpoint{9.046442in}{4.618747in}}%
\pgfpathcurveto{\pgfqpoint{9.054255in}{4.610934in}}{\pgfqpoint{9.064854in}{4.606543in}}{\pgfqpoint{9.075905in}{4.606543in}}%
\pgfpathclose%
\pgfusepath{stroke,fill}%
\end{pgfscope}%
\begin{pgfscope}%
\pgfpathrectangle{\pgfqpoint{7.640588in}{4.121437in}}{\pgfqpoint{5.699255in}{2.685432in}}%
\pgfusepath{clip}%
\pgfsetbuttcap%
\pgfsetroundjoin%
\definecolor{currentfill}{rgb}{0.000000,0.000000,0.000000}%
\pgfsetfillcolor{currentfill}%
\pgfsetlinewidth{1.003750pt}%
\definecolor{currentstroke}{rgb}{0.000000,0.000000,0.000000}%
\pgfsetstrokecolor{currentstroke}%
\pgfsetdash{}{0pt}%
\pgfpathmoveto{\pgfqpoint{8.599800in}{4.502103in}}%
\pgfpathcurveto{\pgfqpoint{8.610850in}{4.502103in}}{\pgfqpoint{8.621449in}{4.506493in}}{\pgfqpoint{8.629262in}{4.514307in}}%
\pgfpathcurveto{\pgfqpoint{8.637076in}{4.522120in}}{\pgfqpoint{8.641466in}{4.532719in}}{\pgfqpoint{8.641466in}{4.543769in}}%
\pgfpathcurveto{\pgfqpoint{8.641466in}{4.554819in}}{\pgfqpoint{8.637076in}{4.565418in}}{\pgfqpoint{8.629262in}{4.573232in}}%
\pgfpathcurveto{\pgfqpoint{8.621449in}{4.581046in}}{\pgfqpoint{8.610850in}{4.585436in}}{\pgfqpoint{8.599800in}{4.585436in}}%
\pgfpathcurveto{\pgfqpoint{8.588750in}{4.585436in}}{\pgfqpoint{8.578150in}{4.581046in}}{\pgfqpoint{8.570337in}{4.573232in}}%
\pgfpathcurveto{\pgfqpoint{8.562523in}{4.565418in}}{\pgfqpoint{8.558133in}{4.554819in}}{\pgfqpoint{8.558133in}{4.543769in}}%
\pgfpathcurveto{\pgfqpoint{8.558133in}{4.532719in}}{\pgfqpoint{8.562523in}{4.522120in}}{\pgfqpoint{8.570337in}{4.514307in}}%
\pgfpathcurveto{\pgfqpoint{8.578150in}{4.506493in}}{\pgfqpoint{8.588750in}{4.502103in}}{\pgfqpoint{8.599800in}{4.502103in}}%
\pgfpathclose%
\pgfusepath{stroke,fill}%
\end{pgfscope}%
\begin{pgfscope}%
\pgfpathrectangle{\pgfqpoint{7.640588in}{4.121437in}}{\pgfqpoint{5.699255in}{2.685432in}}%
\pgfusepath{clip}%
\pgfsetbuttcap%
\pgfsetroundjoin%
\definecolor{currentfill}{rgb}{0.000000,0.000000,0.000000}%
\pgfsetfillcolor{currentfill}%
\pgfsetlinewidth{1.003750pt}%
\definecolor{currentstroke}{rgb}{0.000000,0.000000,0.000000}%
\pgfsetstrokecolor{currentstroke}%
\pgfsetdash{}{0pt}%
\pgfpathmoveto{\pgfqpoint{8.291732in}{4.697929in}}%
\pgfpathcurveto{\pgfqpoint{8.302782in}{4.697929in}}{\pgfqpoint{8.313381in}{4.702319in}}{\pgfqpoint{8.321195in}{4.710133in}}%
\pgfpathcurveto{\pgfqpoint{8.329008in}{4.717946in}}{\pgfqpoint{8.333398in}{4.728546in}}{\pgfqpoint{8.333398in}{4.739596in}}%
\pgfpathcurveto{\pgfqpoint{8.333398in}{4.750646in}}{\pgfqpoint{8.329008in}{4.761245in}}{\pgfqpoint{8.321195in}{4.769058in}}%
\pgfpathcurveto{\pgfqpoint{8.313381in}{4.776872in}}{\pgfqpoint{8.302782in}{4.781262in}}{\pgfqpoint{8.291732in}{4.781262in}}%
\pgfpathcurveto{\pgfqpoint{8.280682in}{4.781262in}}{\pgfqpoint{8.270083in}{4.776872in}}{\pgfqpoint{8.262269in}{4.769058in}}%
\pgfpathcurveto{\pgfqpoint{8.254455in}{4.761245in}}{\pgfqpoint{8.250065in}{4.750646in}}{\pgfqpoint{8.250065in}{4.739596in}}%
\pgfpathcurveto{\pgfqpoint{8.250065in}{4.728546in}}{\pgfqpoint{8.254455in}{4.717946in}}{\pgfqpoint{8.262269in}{4.710133in}}%
\pgfpathcurveto{\pgfqpoint{8.270083in}{4.702319in}}{\pgfqpoint{8.280682in}{4.697929in}}{\pgfqpoint{8.291732in}{4.697929in}}%
\pgfpathclose%
\pgfusepath{stroke,fill}%
\end{pgfscope}%
\begin{pgfscope}%
\pgfpathrectangle{\pgfqpoint{7.640588in}{4.121437in}}{\pgfqpoint{5.699255in}{2.685432in}}%
\pgfusepath{clip}%
\pgfsetbuttcap%
\pgfsetroundjoin%
\definecolor{currentfill}{rgb}{0.000000,0.000000,0.000000}%
\pgfsetfillcolor{currentfill}%
\pgfsetlinewidth{1.003750pt}%
\definecolor{currentstroke}{rgb}{0.000000,0.000000,0.000000}%
\pgfsetstrokecolor{currentstroke}%
\pgfsetdash{}{0pt}%
\pgfpathmoveto{\pgfqpoint{8.263726in}{4.345442in}}%
\pgfpathcurveto{\pgfqpoint{8.274776in}{4.345442in}}{\pgfqpoint{8.285375in}{4.349832in}}{\pgfqpoint{8.293188in}{4.357645in}}%
\pgfpathcurveto{\pgfqpoint{8.301002in}{4.365459in}}{\pgfqpoint{8.305392in}{4.376058in}}{\pgfqpoint{8.305392in}{4.387108in}}%
\pgfpathcurveto{\pgfqpoint{8.305392in}{4.398158in}}{\pgfqpoint{8.301002in}{4.408757in}}{\pgfqpoint{8.293188in}{4.416571in}}%
\pgfpathcurveto{\pgfqpoint{8.285375in}{4.424385in}}{\pgfqpoint{8.274776in}{4.428775in}}{\pgfqpoint{8.263726in}{4.428775in}}%
\pgfpathcurveto{\pgfqpoint{8.252675in}{4.428775in}}{\pgfqpoint{8.242076in}{4.424385in}}{\pgfqpoint{8.234263in}{4.416571in}}%
\pgfpathcurveto{\pgfqpoint{8.226449in}{4.408757in}}{\pgfqpoint{8.222059in}{4.398158in}}{\pgfqpoint{8.222059in}{4.387108in}}%
\pgfpathcurveto{\pgfqpoint{8.222059in}{4.376058in}}{\pgfqpoint{8.226449in}{4.365459in}}{\pgfqpoint{8.234263in}{4.357645in}}%
\pgfpathcurveto{\pgfqpoint{8.242076in}{4.349832in}}{\pgfqpoint{8.252675in}{4.345442in}}{\pgfqpoint{8.263726in}{4.345442in}}%
\pgfpathclose%
\pgfusepath{stroke,fill}%
\end{pgfscope}%
\begin{pgfscope}%
\pgfpathrectangle{\pgfqpoint{7.640588in}{4.121437in}}{\pgfqpoint{5.699255in}{2.685432in}}%
\pgfusepath{clip}%
\pgfsetbuttcap%
\pgfsetroundjoin%
\definecolor{currentfill}{rgb}{0.000000,0.000000,0.000000}%
\pgfsetfillcolor{currentfill}%
\pgfsetlinewidth{1.003750pt}%
\definecolor{currentstroke}{rgb}{0.000000,0.000000,0.000000}%
\pgfsetstrokecolor{currentstroke}%
\pgfsetdash{}{0pt}%
\pgfpathmoveto{\pgfqpoint{8.403756in}{4.201836in}}%
\pgfpathcurveto{\pgfqpoint{8.414807in}{4.201836in}}{\pgfqpoint{8.425406in}{4.206226in}}{\pgfqpoint{8.433219in}{4.214039in}}%
\pgfpathcurveto{\pgfqpoint{8.441033in}{4.221853in}}{\pgfqpoint{8.445423in}{4.232452in}}{\pgfqpoint{8.445423in}{4.243502in}}%
\pgfpathcurveto{\pgfqpoint{8.445423in}{4.254552in}}{\pgfqpoint{8.441033in}{4.265151in}}{\pgfqpoint{8.433219in}{4.272965in}}%
\pgfpathcurveto{\pgfqpoint{8.425406in}{4.280779in}}{\pgfqpoint{8.414807in}{4.285169in}}{\pgfqpoint{8.403756in}{4.285169in}}%
\pgfpathcurveto{\pgfqpoint{8.392706in}{4.285169in}}{\pgfqpoint{8.382107in}{4.280779in}}{\pgfqpoint{8.374294in}{4.272965in}}%
\pgfpathcurveto{\pgfqpoint{8.366480in}{4.265151in}}{\pgfqpoint{8.362090in}{4.254552in}}{\pgfqpoint{8.362090in}{4.243502in}}%
\pgfpathcurveto{\pgfqpoint{8.362090in}{4.232452in}}{\pgfqpoint{8.366480in}{4.221853in}}{\pgfqpoint{8.374294in}{4.214039in}}%
\pgfpathcurveto{\pgfqpoint{8.382107in}{4.206226in}}{\pgfqpoint{8.392706in}{4.201836in}}{\pgfqpoint{8.403756in}{4.201836in}}%
\pgfpathclose%
\pgfusepath{stroke,fill}%
\end{pgfscope}%
\begin{pgfscope}%
\pgfpathrectangle{\pgfqpoint{7.640588in}{4.121437in}}{\pgfqpoint{5.699255in}{2.685432in}}%
\pgfusepath{clip}%
\pgfsetbuttcap%
\pgfsetroundjoin%
\definecolor{currentfill}{rgb}{0.000000,0.000000,0.000000}%
\pgfsetfillcolor{currentfill}%
\pgfsetlinewidth{1.003750pt}%
\definecolor{currentstroke}{rgb}{0.000000,0.000000,0.000000}%
\pgfsetstrokecolor{currentstroke}%
\pgfsetdash{}{0pt}%
\pgfpathmoveto{\pgfqpoint{8.515781in}{4.227946in}}%
\pgfpathcurveto{\pgfqpoint{8.526831in}{4.227946in}}{\pgfqpoint{8.537430in}{4.232336in}}{\pgfqpoint{8.545244in}{4.240150in}}%
\pgfpathcurveto{\pgfqpoint{8.553058in}{4.247963in}}{\pgfqpoint{8.557448in}{4.258562in}}{\pgfqpoint{8.557448in}{4.269612in}}%
\pgfpathcurveto{\pgfqpoint{8.557448in}{4.280663in}}{\pgfqpoint{8.553058in}{4.291262in}}{\pgfqpoint{8.545244in}{4.299075in}}%
\pgfpathcurveto{\pgfqpoint{8.537430in}{4.306889in}}{\pgfqpoint{8.526831in}{4.311279in}}{\pgfqpoint{8.515781in}{4.311279in}}%
\pgfpathcurveto{\pgfqpoint{8.504731in}{4.311279in}}{\pgfqpoint{8.494132in}{4.306889in}}{\pgfqpoint{8.486318in}{4.299075in}}%
\pgfpathcurveto{\pgfqpoint{8.478505in}{4.291262in}}{\pgfqpoint{8.474114in}{4.280663in}}{\pgfqpoint{8.474114in}{4.269612in}}%
\pgfpathcurveto{\pgfqpoint{8.474114in}{4.258562in}}{\pgfqpoint{8.478505in}{4.247963in}}{\pgfqpoint{8.486318in}{4.240150in}}%
\pgfpathcurveto{\pgfqpoint{8.494132in}{4.232336in}}{\pgfqpoint{8.504731in}{4.227946in}}{\pgfqpoint{8.515781in}{4.227946in}}%
\pgfpathclose%
\pgfusepath{stroke,fill}%
\end{pgfscope}%
\begin{pgfscope}%
\pgfpathrectangle{\pgfqpoint{7.640588in}{4.121437in}}{\pgfqpoint{5.699255in}{2.685432in}}%
\pgfusepath{clip}%
\pgfsetbuttcap%
\pgfsetroundjoin%
\definecolor{currentfill}{rgb}{0.000000,0.000000,0.000000}%
\pgfsetfillcolor{currentfill}%
\pgfsetlinewidth{1.003750pt}%
\definecolor{currentstroke}{rgb}{0.000000,0.000000,0.000000}%
\pgfsetstrokecolor{currentstroke}%
\pgfsetdash{}{0pt}%
\pgfpathmoveto{\pgfqpoint{9.075905in}{4.449882in}}%
\pgfpathcurveto{\pgfqpoint{9.086955in}{4.449882in}}{\pgfqpoint{9.097554in}{4.454273in}}{\pgfqpoint{9.105367in}{4.462086in}}%
\pgfpathcurveto{\pgfqpoint{9.113181in}{4.469900in}}{\pgfqpoint{9.117571in}{4.480499in}}{\pgfqpoint{9.117571in}{4.491549in}}%
\pgfpathcurveto{\pgfqpoint{9.117571in}{4.502599in}}{\pgfqpoint{9.113181in}{4.513198in}}{\pgfqpoint{9.105367in}{4.521012in}}%
\pgfpathcurveto{\pgfqpoint{9.097554in}{4.528825in}}{\pgfqpoint{9.086955in}{4.533216in}}{\pgfqpoint{9.075905in}{4.533216in}}%
\pgfpathcurveto{\pgfqpoint{9.064854in}{4.533216in}}{\pgfqpoint{9.054255in}{4.528825in}}{\pgfqpoint{9.046442in}{4.521012in}}%
\pgfpathcurveto{\pgfqpoint{9.038628in}{4.513198in}}{\pgfqpoint{9.034238in}{4.502599in}}{\pgfqpoint{9.034238in}{4.491549in}}%
\pgfpathcurveto{\pgfqpoint{9.034238in}{4.480499in}}{\pgfqpoint{9.038628in}{4.469900in}}{\pgfqpoint{9.046442in}{4.462086in}}%
\pgfpathcurveto{\pgfqpoint{9.054255in}{4.454273in}}{\pgfqpoint{9.064854in}{4.449882in}}{\pgfqpoint{9.075905in}{4.449882in}}%
\pgfpathclose%
\pgfusepath{stroke,fill}%
\end{pgfscope}%
\begin{pgfscope}%
\pgfpathrectangle{\pgfqpoint{7.640588in}{4.121437in}}{\pgfqpoint{5.699255in}{2.685432in}}%
\pgfusepath{clip}%
\pgfsetbuttcap%
\pgfsetroundjoin%
\definecolor{currentfill}{rgb}{0.000000,0.000000,0.000000}%
\pgfsetfillcolor{currentfill}%
\pgfsetlinewidth{1.003750pt}%
\definecolor{currentstroke}{rgb}{0.000000,0.000000,0.000000}%
\pgfsetstrokecolor{currentstroke}%
\pgfsetdash{}{0pt}%
\pgfpathmoveto{\pgfqpoint{9.748053in}{4.358497in}}%
\pgfpathcurveto{\pgfqpoint{9.759103in}{4.358497in}}{\pgfqpoint{9.769702in}{4.362887in}}{\pgfqpoint{9.777515in}{4.370701in}}%
\pgfpathcurveto{\pgfqpoint{9.785329in}{4.378514in}}{\pgfqpoint{9.789719in}{4.389113in}}{\pgfqpoint{9.789719in}{4.400163in}}%
\pgfpathcurveto{\pgfqpoint{9.789719in}{4.411213in}}{\pgfqpoint{9.785329in}{4.421812in}}{\pgfqpoint{9.777515in}{4.429626in}}%
\pgfpathcurveto{\pgfqpoint{9.769702in}{4.437440in}}{\pgfqpoint{9.759103in}{4.441830in}}{\pgfqpoint{9.748053in}{4.441830in}}%
\pgfpathcurveto{\pgfqpoint{9.737002in}{4.441830in}}{\pgfqpoint{9.726403in}{4.437440in}}{\pgfqpoint{9.718590in}{4.429626in}}%
\pgfpathcurveto{\pgfqpoint{9.710776in}{4.421812in}}{\pgfqpoint{9.706386in}{4.411213in}}{\pgfqpoint{9.706386in}{4.400163in}}%
\pgfpathcurveto{\pgfqpoint{9.706386in}{4.389113in}}{\pgfqpoint{9.710776in}{4.378514in}}{\pgfqpoint{9.718590in}{4.370701in}}%
\pgfpathcurveto{\pgfqpoint{9.726403in}{4.362887in}}{\pgfqpoint{9.737002in}{4.358497in}}{\pgfqpoint{9.748053in}{4.358497in}}%
\pgfpathclose%
\pgfusepath{stroke,fill}%
\end{pgfscope}%
\begin{pgfscope}%
\pgfpathrectangle{\pgfqpoint{7.640588in}{4.121437in}}{\pgfqpoint{5.699255in}{2.685432in}}%
\pgfusepath{clip}%
\pgfsetbuttcap%
\pgfsetroundjoin%
\definecolor{currentfill}{rgb}{0.000000,0.000000,0.000000}%
\pgfsetfillcolor{currentfill}%
\pgfsetlinewidth{1.003750pt}%
\definecolor{currentstroke}{rgb}{0.000000,0.000000,0.000000}%
\pgfsetstrokecolor{currentstroke}%
\pgfsetdash{}{0pt}%
\pgfpathmoveto{\pgfqpoint{10.140139in}{4.945976in}}%
\pgfpathcurveto{\pgfqpoint{10.151189in}{4.945976in}}{\pgfqpoint{10.161788in}{4.950366in}}{\pgfqpoint{10.169602in}{4.958180in}}%
\pgfpathcurveto{\pgfqpoint{10.177415in}{4.965993in}}{\pgfqpoint{10.181806in}{4.976592in}}{\pgfqpoint{10.181806in}{4.987642in}}%
\pgfpathcurveto{\pgfqpoint{10.181806in}{4.998692in}}{\pgfqpoint{10.177415in}{5.009292in}}{\pgfqpoint{10.169602in}{5.017105in}}%
\pgfpathcurveto{\pgfqpoint{10.161788in}{5.024919in}}{\pgfqpoint{10.151189in}{5.029309in}}{\pgfqpoint{10.140139in}{5.029309in}}%
\pgfpathcurveto{\pgfqpoint{10.129089in}{5.029309in}}{\pgfqpoint{10.118490in}{5.024919in}}{\pgfqpoint{10.110676in}{5.017105in}}%
\pgfpathcurveto{\pgfqpoint{10.102863in}{5.009292in}}{\pgfqpoint{10.098472in}{4.998692in}}{\pgfqpoint{10.098472in}{4.987642in}}%
\pgfpathcurveto{\pgfqpoint{10.098472in}{4.976592in}}{\pgfqpoint{10.102863in}{4.965993in}}{\pgfqpoint{10.110676in}{4.958180in}}%
\pgfpathcurveto{\pgfqpoint{10.118490in}{4.950366in}}{\pgfqpoint{10.129089in}{4.945976in}}{\pgfqpoint{10.140139in}{4.945976in}}%
\pgfpathclose%
\pgfusepath{stroke,fill}%
\end{pgfscope}%
\begin{pgfscope}%
\pgfpathrectangle{\pgfqpoint{7.640588in}{4.121437in}}{\pgfqpoint{5.699255in}{2.685432in}}%
\pgfusepath{clip}%
\pgfsetbuttcap%
\pgfsetroundjoin%
\definecolor{currentfill}{rgb}{0.000000,0.000000,0.000000}%
\pgfsetfillcolor{currentfill}%
\pgfsetlinewidth{1.003750pt}%
\definecolor{currentstroke}{rgb}{0.000000,0.000000,0.000000}%
\pgfsetstrokecolor{currentstroke}%
\pgfsetdash{}{0pt}%
\pgfpathmoveto{\pgfqpoint{10.364188in}{5.050416in}}%
\pgfpathcurveto{\pgfqpoint{10.375238in}{5.050416in}}{\pgfqpoint{10.385837in}{5.054807in}}{\pgfqpoint{10.393651in}{5.062620in}}%
\pgfpathcurveto{\pgfqpoint{10.401465in}{5.070434in}}{\pgfqpoint{10.405855in}{5.081033in}}{\pgfqpoint{10.405855in}{5.092083in}}%
\pgfpathcurveto{\pgfqpoint{10.405855in}{5.103133in}}{\pgfqpoint{10.401465in}{5.113732in}}{\pgfqpoint{10.393651in}{5.121546in}}%
\pgfpathcurveto{\pgfqpoint{10.385837in}{5.129359in}}{\pgfqpoint{10.375238in}{5.133750in}}{\pgfqpoint{10.364188in}{5.133750in}}%
\pgfpathcurveto{\pgfqpoint{10.353138in}{5.133750in}}{\pgfqpoint{10.342539in}{5.129359in}}{\pgfqpoint{10.334726in}{5.121546in}}%
\pgfpathcurveto{\pgfqpoint{10.326912in}{5.113732in}}{\pgfqpoint{10.322522in}{5.103133in}}{\pgfqpoint{10.322522in}{5.092083in}}%
\pgfpathcurveto{\pgfqpoint{10.322522in}{5.081033in}}{\pgfqpoint{10.326912in}{5.070434in}}{\pgfqpoint{10.334726in}{5.062620in}}%
\pgfpathcurveto{\pgfqpoint{10.342539in}{5.054807in}}{\pgfqpoint{10.353138in}{5.050416in}}{\pgfqpoint{10.364188in}{5.050416in}}%
\pgfpathclose%
\pgfusepath{stroke,fill}%
\end{pgfscope}%
\begin{pgfscope}%
\pgfpathrectangle{\pgfqpoint{7.640588in}{4.121437in}}{\pgfqpoint{5.699255in}{2.685432in}}%
\pgfusepath{clip}%
\pgfsetbuttcap%
\pgfsetroundjoin%
\definecolor{currentfill}{rgb}{0.000000,0.000000,0.000000}%
\pgfsetfillcolor{currentfill}%
\pgfsetlinewidth{1.003750pt}%
\definecolor{currentstroke}{rgb}{0.000000,0.000000,0.000000}%
\pgfsetstrokecolor{currentstroke}%
\pgfsetdash{}{0pt}%
\pgfpathmoveto{\pgfqpoint{10.476213in}{5.115692in}}%
\pgfpathcurveto{\pgfqpoint{10.487263in}{5.115692in}}{\pgfqpoint{10.497862in}{5.120082in}}{\pgfqpoint{10.505676in}{5.127896in}}%
\pgfpathcurveto{\pgfqpoint{10.513489in}{5.135709in}}{\pgfqpoint{10.517880in}{5.146308in}}{\pgfqpoint{10.517880in}{5.157359in}}%
\pgfpathcurveto{\pgfqpoint{10.517880in}{5.168409in}}{\pgfqpoint{10.513489in}{5.179008in}}{\pgfqpoint{10.505676in}{5.186821in}}%
\pgfpathcurveto{\pgfqpoint{10.497862in}{5.194635in}}{\pgfqpoint{10.487263in}{5.199025in}}{\pgfqpoint{10.476213in}{5.199025in}}%
\pgfpathcurveto{\pgfqpoint{10.465163in}{5.199025in}}{\pgfqpoint{10.454564in}{5.194635in}}{\pgfqpoint{10.446750in}{5.186821in}}%
\pgfpathcurveto{\pgfqpoint{10.438937in}{5.179008in}}{\pgfqpoint{10.434546in}{5.168409in}}{\pgfqpoint{10.434546in}{5.157359in}}%
\pgfpathcurveto{\pgfqpoint{10.434546in}{5.146308in}}{\pgfqpoint{10.438937in}{5.135709in}}{\pgfqpoint{10.446750in}{5.127896in}}%
\pgfpathcurveto{\pgfqpoint{10.454564in}{5.120082in}}{\pgfqpoint{10.465163in}{5.115692in}}{\pgfqpoint{10.476213in}{5.115692in}}%
\pgfpathclose%
\pgfusepath{stroke,fill}%
\end{pgfscope}%
\begin{pgfscope}%
\pgfpathrectangle{\pgfqpoint{7.640588in}{4.121437in}}{\pgfqpoint{5.699255in}{2.685432in}}%
\pgfusepath{clip}%
\pgfsetbuttcap%
\pgfsetroundjoin%
\definecolor{currentfill}{rgb}{0.000000,0.000000,0.000000}%
\pgfsetfillcolor{currentfill}%
\pgfsetlinewidth{1.003750pt}%
\definecolor{currentstroke}{rgb}{0.000000,0.000000,0.000000}%
\pgfsetstrokecolor{currentstroke}%
\pgfsetdash{}{0pt}%
\pgfpathmoveto{\pgfqpoint{10.560231in}{4.410717in}}%
\pgfpathcurveto{\pgfqpoint{10.571282in}{4.410717in}}{\pgfqpoint{10.581881in}{4.415107in}}{\pgfqpoint{10.589694in}{4.422921in}}%
\pgfpathcurveto{\pgfqpoint{10.597508in}{4.430734in}}{\pgfqpoint{10.601898in}{4.441334in}}{\pgfqpoint{10.601898in}{4.452384in}}%
\pgfpathcurveto{\pgfqpoint{10.601898in}{4.463434in}}{\pgfqpoint{10.597508in}{4.474033in}}{\pgfqpoint{10.589694in}{4.481846in}}%
\pgfpathcurveto{\pgfqpoint{10.581881in}{4.489660in}}{\pgfqpoint{10.571282in}{4.494050in}}{\pgfqpoint{10.560231in}{4.494050in}}%
\pgfpathcurveto{\pgfqpoint{10.549181in}{4.494050in}}{\pgfqpoint{10.538582in}{4.489660in}}{\pgfqpoint{10.530769in}{4.481846in}}%
\pgfpathcurveto{\pgfqpoint{10.522955in}{4.474033in}}{\pgfqpoint{10.518565in}{4.463434in}}{\pgfqpoint{10.518565in}{4.452384in}}%
\pgfpathcurveto{\pgfqpoint{10.518565in}{4.441334in}}{\pgfqpoint{10.522955in}{4.430734in}}{\pgfqpoint{10.530769in}{4.422921in}}%
\pgfpathcurveto{\pgfqpoint{10.538582in}{4.415107in}}{\pgfqpoint{10.549181in}{4.410717in}}{\pgfqpoint{10.560231in}{4.410717in}}%
\pgfpathclose%
\pgfusepath{stroke,fill}%
\end{pgfscope}%
\begin{pgfscope}%
\pgfpathrectangle{\pgfqpoint{7.640588in}{4.121437in}}{\pgfqpoint{5.699255in}{2.685432in}}%
\pgfusepath{clip}%
\pgfsetbuttcap%
\pgfsetroundjoin%
\definecolor{currentfill}{rgb}{0.000000,0.000000,0.000000}%
\pgfsetfillcolor{currentfill}%
\pgfsetlinewidth{1.003750pt}%
\definecolor{currentstroke}{rgb}{0.000000,0.000000,0.000000}%
\pgfsetstrokecolor{currentstroke}%
\pgfsetdash{}{0pt}%
\pgfpathmoveto{\pgfqpoint{10.700262in}{4.345442in}}%
\pgfpathcurveto{\pgfqpoint{10.711312in}{4.345442in}}{\pgfqpoint{10.721911in}{4.349832in}}{\pgfqpoint{10.729725in}{4.357645in}}%
\pgfpathcurveto{\pgfqpoint{10.737539in}{4.365459in}}{\pgfqpoint{10.741929in}{4.376058in}}{\pgfqpoint{10.741929in}{4.387108in}}%
\pgfpathcurveto{\pgfqpoint{10.741929in}{4.398158in}}{\pgfqpoint{10.737539in}{4.408757in}}{\pgfqpoint{10.729725in}{4.416571in}}%
\pgfpathcurveto{\pgfqpoint{10.721911in}{4.424385in}}{\pgfqpoint{10.711312in}{4.428775in}}{\pgfqpoint{10.700262in}{4.428775in}}%
\pgfpathcurveto{\pgfqpoint{10.689212in}{4.428775in}}{\pgfqpoint{10.678613in}{4.424385in}}{\pgfqpoint{10.670800in}{4.416571in}}%
\pgfpathcurveto{\pgfqpoint{10.662986in}{4.408757in}}{\pgfqpoint{10.658596in}{4.398158in}}{\pgfqpoint{10.658596in}{4.387108in}}%
\pgfpathcurveto{\pgfqpoint{10.658596in}{4.376058in}}{\pgfqpoint{10.662986in}{4.365459in}}{\pgfqpoint{10.670800in}{4.357645in}}%
\pgfpathcurveto{\pgfqpoint{10.678613in}{4.349832in}}{\pgfqpoint{10.689212in}{4.345442in}}{\pgfqpoint{10.700262in}{4.345442in}}%
\pgfpathclose%
\pgfusepath{stroke,fill}%
\end{pgfscope}%
\begin{pgfscope}%
\pgfpathrectangle{\pgfqpoint{7.640588in}{4.121437in}}{\pgfqpoint{5.699255in}{2.685432in}}%
\pgfusepath{clip}%
\pgfsetbuttcap%
\pgfsetroundjoin%
\definecolor{currentfill}{rgb}{0.000000,0.000000,0.000000}%
\pgfsetfillcolor{currentfill}%
\pgfsetlinewidth{1.003750pt}%
\definecolor{currentstroke}{rgb}{0.000000,0.000000,0.000000}%
\pgfsetstrokecolor{currentstroke}%
\pgfsetdash{}{0pt}%
\pgfpathmoveto{\pgfqpoint{10.756275in}{4.397662in}}%
\pgfpathcurveto{\pgfqpoint{10.767325in}{4.397662in}}{\pgfqpoint{10.777924in}{4.402052in}}{\pgfqpoint{10.785737in}{4.409866in}}%
\pgfpathcurveto{\pgfqpoint{10.793551in}{4.417679in}}{\pgfqpoint{10.797941in}{4.428278in}}{\pgfqpoint{10.797941in}{4.439329in}}%
\pgfpathcurveto{\pgfqpoint{10.797941in}{4.450379in}}{\pgfqpoint{10.793551in}{4.460978in}}{\pgfqpoint{10.785737in}{4.468791in}}%
\pgfpathcurveto{\pgfqpoint{10.777924in}{4.476605in}}{\pgfqpoint{10.767325in}{4.480995in}}{\pgfqpoint{10.756275in}{4.480995in}}%
\pgfpathcurveto{\pgfqpoint{10.745225in}{4.480995in}}{\pgfqpoint{10.734626in}{4.476605in}}{\pgfqpoint{10.726812in}{4.468791in}}%
\pgfpathcurveto{\pgfqpoint{10.718998in}{4.460978in}}{\pgfqpoint{10.714608in}{4.450379in}}{\pgfqpoint{10.714608in}{4.439329in}}%
\pgfpathcurveto{\pgfqpoint{10.714608in}{4.428278in}}{\pgfqpoint{10.718998in}{4.417679in}}{\pgfqpoint{10.726812in}{4.409866in}}%
\pgfpathcurveto{\pgfqpoint{10.734626in}{4.402052in}}{\pgfqpoint{10.745225in}{4.397662in}}{\pgfqpoint{10.756275in}{4.397662in}}%
\pgfpathclose%
\pgfusepath{stroke,fill}%
\end{pgfscope}%
\begin{pgfscope}%
\pgfpathrectangle{\pgfqpoint{7.640588in}{4.121437in}}{\pgfqpoint{5.699255in}{2.685432in}}%
\pgfusepath{clip}%
\pgfsetbuttcap%
\pgfsetroundjoin%
\definecolor{currentfill}{rgb}{0.000000,0.000000,0.000000}%
\pgfsetfillcolor{currentfill}%
\pgfsetlinewidth{1.003750pt}%
\definecolor{currentstroke}{rgb}{0.000000,0.000000,0.000000}%
\pgfsetstrokecolor{currentstroke}%
\pgfsetdash{}{0pt}%
\pgfpathmoveto{\pgfqpoint{11.260386in}{4.436827in}}%
\pgfpathcurveto{\pgfqpoint{11.271436in}{4.436827in}}{\pgfqpoint{11.282035in}{4.441217in}}{\pgfqpoint{11.289848in}{4.449031in}}%
\pgfpathcurveto{\pgfqpoint{11.297662in}{4.456845in}}{\pgfqpoint{11.302052in}{4.467444in}}{\pgfqpoint{11.302052in}{4.478494in}}%
\pgfpathcurveto{\pgfqpoint{11.302052in}{4.489544in}}{\pgfqpoint{11.297662in}{4.500143in}}{\pgfqpoint{11.289848in}{4.507957in}}%
\pgfpathcurveto{\pgfqpoint{11.282035in}{4.515770in}}{\pgfqpoint{11.271436in}{4.520161in}}{\pgfqpoint{11.260386in}{4.520161in}}%
\pgfpathcurveto{\pgfqpoint{11.249336in}{4.520161in}}{\pgfqpoint{11.238737in}{4.515770in}}{\pgfqpoint{11.230923in}{4.507957in}}%
\pgfpathcurveto{\pgfqpoint{11.223109in}{4.500143in}}{\pgfqpoint{11.218719in}{4.489544in}}{\pgfqpoint{11.218719in}{4.478494in}}%
\pgfpathcurveto{\pgfqpoint{11.218719in}{4.467444in}}{\pgfqpoint{11.223109in}{4.456845in}}{\pgfqpoint{11.230923in}{4.449031in}}%
\pgfpathcurveto{\pgfqpoint{11.238737in}{4.441217in}}{\pgfqpoint{11.249336in}{4.436827in}}{\pgfqpoint{11.260386in}{4.436827in}}%
\pgfpathclose%
\pgfusepath{stroke,fill}%
\end{pgfscope}%
\begin{pgfscope}%
\pgfpathrectangle{\pgfqpoint{7.640588in}{4.121437in}}{\pgfqpoint{5.699255in}{2.685432in}}%
\pgfusepath{clip}%
\pgfsetbuttcap%
\pgfsetroundjoin%
\definecolor{currentfill}{rgb}{0.000000,0.000000,0.000000}%
\pgfsetfillcolor{currentfill}%
\pgfsetlinewidth{1.003750pt}%
\definecolor{currentstroke}{rgb}{0.000000,0.000000,0.000000}%
\pgfsetstrokecolor{currentstroke}%
\pgfsetdash{}{0pt}%
\pgfpathmoveto{\pgfqpoint{11.568454in}{4.436827in}}%
\pgfpathcurveto{\pgfqpoint{11.579504in}{4.436827in}}{\pgfqpoint{11.590103in}{4.441217in}}{\pgfqpoint{11.597916in}{4.449031in}}%
\pgfpathcurveto{\pgfqpoint{11.605730in}{4.456845in}}{\pgfqpoint{11.610120in}{4.467444in}}{\pgfqpoint{11.610120in}{4.478494in}}%
\pgfpathcurveto{\pgfqpoint{11.610120in}{4.489544in}}{\pgfqpoint{11.605730in}{4.500143in}}{\pgfqpoint{11.597916in}{4.507957in}}%
\pgfpathcurveto{\pgfqpoint{11.590103in}{4.515770in}}{\pgfqpoint{11.579504in}{4.520161in}}{\pgfqpoint{11.568454in}{4.520161in}}%
\pgfpathcurveto{\pgfqpoint{11.557403in}{4.520161in}}{\pgfqpoint{11.546804in}{4.515770in}}{\pgfqpoint{11.538991in}{4.507957in}}%
\pgfpathcurveto{\pgfqpoint{11.531177in}{4.500143in}}{\pgfqpoint{11.526787in}{4.489544in}}{\pgfqpoint{11.526787in}{4.478494in}}%
\pgfpathcurveto{\pgfqpoint{11.526787in}{4.467444in}}{\pgfqpoint{11.531177in}{4.456845in}}{\pgfqpoint{11.538991in}{4.449031in}}%
\pgfpathcurveto{\pgfqpoint{11.546804in}{4.441217in}}{\pgfqpoint{11.557403in}{4.436827in}}{\pgfqpoint{11.568454in}{4.436827in}}%
\pgfpathclose%
\pgfusepath{stroke,fill}%
\end{pgfscope}%
\begin{pgfscope}%
\pgfpathrectangle{\pgfqpoint{7.640588in}{4.121437in}}{\pgfqpoint{5.699255in}{2.685432in}}%
\pgfusepath{clip}%
\pgfsetbuttcap%
\pgfsetroundjoin%
\definecolor{currentfill}{rgb}{0.000000,0.000000,0.000000}%
\pgfsetfillcolor{currentfill}%
\pgfsetlinewidth{1.003750pt}%
\definecolor{currentstroke}{rgb}{0.000000,0.000000,0.000000}%
\pgfsetstrokecolor{currentstroke}%
\pgfsetdash{}{0pt}%
\pgfpathmoveto{\pgfqpoint{11.568454in}{4.449882in}}%
\pgfpathcurveto{\pgfqpoint{11.579504in}{4.449882in}}{\pgfqpoint{11.590103in}{4.454273in}}{\pgfqpoint{11.597916in}{4.462086in}}%
\pgfpathcurveto{\pgfqpoint{11.605730in}{4.469900in}}{\pgfqpoint{11.610120in}{4.480499in}}{\pgfqpoint{11.610120in}{4.491549in}}%
\pgfpathcurveto{\pgfqpoint{11.610120in}{4.502599in}}{\pgfqpoint{11.605730in}{4.513198in}}{\pgfqpoint{11.597916in}{4.521012in}}%
\pgfpathcurveto{\pgfqpoint{11.590103in}{4.528825in}}{\pgfqpoint{11.579504in}{4.533216in}}{\pgfqpoint{11.568454in}{4.533216in}}%
\pgfpathcurveto{\pgfqpoint{11.557403in}{4.533216in}}{\pgfqpoint{11.546804in}{4.528825in}}{\pgfqpoint{11.538991in}{4.521012in}}%
\pgfpathcurveto{\pgfqpoint{11.531177in}{4.513198in}}{\pgfqpoint{11.526787in}{4.502599in}}{\pgfqpoint{11.526787in}{4.491549in}}%
\pgfpathcurveto{\pgfqpoint{11.526787in}{4.480499in}}{\pgfqpoint{11.531177in}{4.469900in}}{\pgfqpoint{11.538991in}{4.462086in}}%
\pgfpathcurveto{\pgfqpoint{11.546804in}{4.454273in}}{\pgfqpoint{11.557403in}{4.449882in}}{\pgfqpoint{11.568454in}{4.449882in}}%
\pgfpathclose%
\pgfusepath{stroke,fill}%
\end{pgfscope}%
\begin{pgfscope}%
\pgfpathrectangle{\pgfqpoint{7.640588in}{4.121437in}}{\pgfqpoint{5.699255in}{2.685432in}}%
\pgfusepath{clip}%
\pgfsetbuttcap%
\pgfsetroundjoin%
\definecolor{currentfill}{rgb}{0.000000,0.000000,0.000000}%
\pgfsetfillcolor{currentfill}%
\pgfsetlinewidth{1.003750pt}%
\definecolor{currentstroke}{rgb}{0.000000,0.000000,0.000000}%
\pgfsetstrokecolor{currentstroke}%
\pgfsetdash{}{0pt}%
\pgfpathmoveto{\pgfqpoint{11.904528in}{4.489048in}}%
\pgfpathcurveto{\pgfqpoint{11.915578in}{4.489048in}}{\pgfqpoint{11.926177in}{4.493438in}}{\pgfqpoint{11.933990in}{4.501251in}}%
\pgfpathcurveto{\pgfqpoint{11.941804in}{4.509065in}}{\pgfqpoint{11.946194in}{4.519664in}}{\pgfqpoint{11.946194in}{4.530714in}}%
\pgfpathcurveto{\pgfqpoint{11.946194in}{4.541764in}}{\pgfqpoint{11.941804in}{4.552363in}}{\pgfqpoint{11.933990in}{4.560177in}}%
\pgfpathcurveto{\pgfqpoint{11.926177in}{4.567991in}}{\pgfqpoint{11.915578in}{4.572381in}}{\pgfqpoint{11.904528in}{4.572381in}}%
\pgfpathcurveto{\pgfqpoint{11.893477in}{4.572381in}}{\pgfqpoint{11.882878in}{4.567991in}}{\pgfqpoint{11.875065in}{4.560177in}}%
\pgfpathcurveto{\pgfqpoint{11.867251in}{4.552363in}}{\pgfqpoint{11.862861in}{4.541764in}}{\pgfqpoint{11.862861in}{4.530714in}}%
\pgfpathcurveto{\pgfqpoint{11.862861in}{4.519664in}}{\pgfqpoint{11.867251in}{4.509065in}}{\pgfqpoint{11.875065in}{4.501251in}}%
\pgfpathcurveto{\pgfqpoint{11.882878in}{4.493438in}}{\pgfqpoint{11.893477in}{4.489048in}}{\pgfqpoint{11.904528in}{4.489048in}}%
\pgfpathclose%
\pgfusepath{stroke,fill}%
\end{pgfscope}%
\begin{pgfscope}%
\pgfpathrectangle{\pgfqpoint{7.640588in}{4.121437in}}{\pgfqpoint{5.699255in}{2.685432in}}%
\pgfusepath{clip}%
\pgfsetbuttcap%
\pgfsetroundjoin%
\definecolor{currentfill}{rgb}{0.000000,0.000000,1.000000}%
\pgfsetfillcolor{currentfill}%
\pgfsetlinewidth{1.003750pt}%
\definecolor{currentstroke}{rgb}{0.000000,0.000000,1.000000}%
\pgfsetstrokecolor{currentstroke}%
\pgfsetdash{}{0pt}%
\pgfpathmoveto{\pgfqpoint{11.932534in}{4.502103in}}%
\pgfpathcurveto{\pgfqpoint{11.943584in}{4.502103in}}{\pgfqpoint{11.954183in}{4.506493in}}{\pgfqpoint{11.961997in}{4.514307in}}%
\pgfpathcurveto{\pgfqpoint{11.969810in}{4.522120in}}{\pgfqpoint{11.974200in}{4.532719in}}{\pgfqpoint{11.974200in}{4.543769in}}%
\pgfpathcurveto{\pgfqpoint{11.974200in}{4.554819in}}{\pgfqpoint{11.969810in}{4.565418in}}{\pgfqpoint{11.961997in}{4.573232in}}%
\pgfpathcurveto{\pgfqpoint{11.954183in}{4.581046in}}{\pgfqpoint{11.943584in}{4.585436in}}{\pgfqpoint{11.932534in}{4.585436in}}%
\pgfpathcurveto{\pgfqpoint{11.921484in}{4.585436in}}{\pgfqpoint{11.910885in}{4.581046in}}{\pgfqpoint{11.903071in}{4.573232in}}%
\pgfpathcurveto{\pgfqpoint{11.895257in}{4.565418in}}{\pgfqpoint{11.890867in}{4.554819in}}{\pgfqpoint{11.890867in}{4.543769in}}%
\pgfpathcurveto{\pgfqpoint{11.890867in}{4.532719in}}{\pgfqpoint{11.895257in}{4.522120in}}{\pgfqpoint{11.903071in}{4.514307in}}%
\pgfpathcurveto{\pgfqpoint{11.910885in}{4.506493in}}{\pgfqpoint{11.921484in}{4.502103in}}{\pgfqpoint{11.932534in}{4.502103in}}%
\pgfpathclose%
\pgfusepath{stroke,fill}%
\end{pgfscope}%
\begin{pgfscope}%
\pgfsetbuttcap%
\pgfsetroundjoin%
\definecolor{currentfill}{rgb}{0.000000,0.000000,0.000000}%
\pgfsetfillcolor{currentfill}%
\pgfsetlinewidth{0.803000pt}%
\definecolor{currentstroke}{rgb}{0.000000,0.000000,0.000000}%
\pgfsetstrokecolor{currentstroke}%
\pgfsetdash{}{0pt}%
\pgfsys@defobject{currentmarker}{\pgfqpoint{0.000000in}{-0.048611in}}{\pgfqpoint{0.000000in}{0.000000in}}{%
\pgfpathmoveto{\pgfqpoint{0.000000in}{0.000000in}}%
\pgfpathlineto{\pgfqpoint{0.000000in}{-0.048611in}}%
\pgfusepath{stroke,fill}%
}%
\begin{pgfscope}%
\pgfsys@transformshift{7.815627in}{4.121437in}%
\pgfsys@useobject{currentmarker}{}%
\end{pgfscope}%
\end{pgfscope}%
\begin{pgfscope}%
\definecolor{textcolor}{rgb}{0.000000,0.000000,0.000000}%
\pgfsetstrokecolor{textcolor}%
\pgfsetfillcolor{textcolor}%
\pgftext[x=7.815627in,y=4.024215in,,top]{\color{textcolor}\rmfamily\fontsize{10.000000}{12.000000}\selectfont \(\displaystyle 0\)}%
\end{pgfscope}%
\begin{pgfscope}%
\pgfsetbuttcap%
\pgfsetroundjoin%
\definecolor{currentfill}{rgb}{0.000000,0.000000,0.000000}%
\pgfsetfillcolor{currentfill}%
\pgfsetlinewidth{0.803000pt}%
\definecolor{currentstroke}{rgb}{0.000000,0.000000,0.000000}%
\pgfsetstrokecolor{currentstroke}%
\pgfsetdash{}{0pt}%
\pgfsys@defobject{currentmarker}{\pgfqpoint{0.000000in}{-0.048611in}}{\pgfqpoint{0.000000in}{0.000000in}}{%
\pgfpathmoveto{\pgfqpoint{0.000000in}{0.000000in}}%
\pgfpathlineto{\pgfqpoint{0.000000in}{-0.048611in}}%
\pgfusepath{stroke,fill}%
}%
\begin{pgfscope}%
\pgfsys@transformshift{8.515781in}{4.121437in}%
\pgfsys@useobject{currentmarker}{}%
\end{pgfscope}%
\end{pgfscope}%
\begin{pgfscope}%
\definecolor{textcolor}{rgb}{0.000000,0.000000,0.000000}%
\pgfsetstrokecolor{textcolor}%
\pgfsetfillcolor{textcolor}%
\pgftext[x=8.515781in,y=4.024215in,,top]{\color{textcolor}\rmfamily\fontsize{10.000000}{12.000000}\selectfont \(\displaystyle 25\)}%
\end{pgfscope}%
\begin{pgfscope}%
\pgfsetbuttcap%
\pgfsetroundjoin%
\definecolor{currentfill}{rgb}{0.000000,0.000000,0.000000}%
\pgfsetfillcolor{currentfill}%
\pgfsetlinewidth{0.803000pt}%
\definecolor{currentstroke}{rgb}{0.000000,0.000000,0.000000}%
\pgfsetstrokecolor{currentstroke}%
\pgfsetdash{}{0pt}%
\pgfsys@defobject{currentmarker}{\pgfqpoint{0.000000in}{-0.048611in}}{\pgfqpoint{0.000000in}{0.000000in}}{%
\pgfpathmoveto{\pgfqpoint{0.000000in}{0.000000in}}%
\pgfpathlineto{\pgfqpoint{0.000000in}{-0.048611in}}%
\pgfusepath{stroke,fill}%
}%
\begin{pgfscope}%
\pgfsys@transformshift{9.215935in}{4.121437in}%
\pgfsys@useobject{currentmarker}{}%
\end{pgfscope}%
\end{pgfscope}%
\begin{pgfscope}%
\definecolor{textcolor}{rgb}{0.000000,0.000000,0.000000}%
\pgfsetstrokecolor{textcolor}%
\pgfsetfillcolor{textcolor}%
\pgftext[x=9.215935in,y=4.024215in,,top]{\color{textcolor}\rmfamily\fontsize{10.000000}{12.000000}\selectfont \(\displaystyle 50\)}%
\end{pgfscope}%
\begin{pgfscope}%
\pgfsetbuttcap%
\pgfsetroundjoin%
\definecolor{currentfill}{rgb}{0.000000,0.000000,0.000000}%
\pgfsetfillcolor{currentfill}%
\pgfsetlinewidth{0.803000pt}%
\definecolor{currentstroke}{rgb}{0.000000,0.000000,0.000000}%
\pgfsetstrokecolor{currentstroke}%
\pgfsetdash{}{0pt}%
\pgfsys@defobject{currentmarker}{\pgfqpoint{0.000000in}{-0.048611in}}{\pgfqpoint{0.000000in}{0.000000in}}{%
\pgfpathmoveto{\pgfqpoint{0.000000in}{0.000000in}}%
\pgfpathlineto{\pgfqpoint{0.000000in}{-0.048611in}}%
\pgfusepath{stroke,fill}%
}%
\begin{pgfscope}%
\pgfsys@transformshift{9.916090in}{4.121437in}%
\pgfsys@useobject{currentmarker}{}%
\end{pgfscope}%
\end{pgfscope}%
\begin{pgfscope}%
\definecolor{textcolor}{rgb}{0.000000,0.000000,0.000000}%
\pgfsetstrokecolor{textcolor}%
\pgfsetfillcolor{textcolor}%
\pgftext[x=9.916090in,y=4.024215in,,top]{\color{textcolor}\rmfamily\fontsize{10.000000}{12.000000}\selectfont \(\displaystyle 75\)}%
\end{pgfscope}%
\begin{pgfscope}%
\pgfsetbuttcap%
\pgfsetroundjoin%
\definecolor{currentfill}{rgb}{0.000000,0.000000,0.000000}%
\pgfsetfillcolor{currentfill}%
\pgfsetlinewidth{0.803000pt}%
\definecolor{currentstroke}{rgb}{0.000000,0.000000,0.000000}%
\pgfsetstrokecolor{currentstroke}%
\pgfsetdash{}{0pt}%
\pgfsys@defobject{currentmarker}{\pgfqpoint{0.000000in}{-0.048611in}}{\pgfqpoint{0.000000in}{0.000000in}}{%
\pgfpathmoveto{\pgfqpoint{0.000000in}{0.000000in}}%
\pgfpathlineto{\pgfqpoint{0.000000in}{-0.048611in}}%
\pgfusepath{stroke,fill}%
}%
\begin{pgfscope}%
\pgfsys@transformshift{10.616244in}{4.121437in}%
\pgfsys@useobject{currentmarker}{}%
\end{pgfscope}%
\end{pgfscope}%
\begin{pgfscope}%
\definecolor{textcolor}{rgb}{0.000000,0.000000,0.000000}%
\pgfsetstrokecolor{textcolor}%
\pgfsetfillcolor{textcolor}%
\pgftext[x=10.616244in,y=4.024215in,,top]{\color{textcolor}\rmfamily\fontsize{10.000000}{12.000000}\selectfont \(\displaystyle 100\)}%
\end{pgfscope}%
\begin{pgfscope}%
\pgfsetbuttcap%
\pgfsetroundjoin%
\definecolor{currentfill}{rgb}{0.000000,0.000000,0.000000}%
\pgfsetfillcolor{currentfill}%
\pgfsetlinewidth{0.803000pt}%
\definecolor{currentstroke}{rgb}{0.000000,0.000000,0.000000}%
\pgfsetstrokecolor{currentstroke}%
\pgfsetdash{}{0pt}%
\pgfsys@defobject{currentmarker}{\pgfqpoint{0.000000in}{-0.048611in}}{\pgfqpoint{0.000000in}{0.000000in}}{%
\pgfpathmoveto{\pgfqpoint{0.000000in}{0.000000in}}%
\pgfpathlineto{\pgfqpoint{0.000000in}{-0.048611in}}%
\pgfusepath{stroke,fill}%
}%
\begin{pgfscope}%
\pgfsys@transformshift{11.316398in}{4.121437in}%
\pgfsys@useobject{currentmarker}{}%
\end{pgfscope}%
\end{pgfscope}%
\begin{pgfscope}%
\definecolor{textcolor}{rgb}{0.000000,0.000000,0.000000}%
\pgfsetstrokecolor{textcolor}%
\pgfsetfillcolor{textcolor}%
\pgftext[x=11.316398in,y=4.024215in,,top]{\color{textcolor}\rmfamily\fontsize{10.000000}{12.000000}\selectfont \(\displaystyle 125\)}%
\end{pgfscope}%
\begin{pgfscope}%
\pgfsetbuttcap%
\pgfsetroundjoin%
\definecolor{currentfill}{rgb}{0.000000,0.000000,0.000000}%
\pgfsetfillcolor{currentfill}%
\pgfsetlinewidth{0.803000pt}%
\definecolor{currentstroke}{rgb}{0.000000,0.000000,0.000000}%
\pgfsetstrokecolor{currentstroke}%
\pgfsetdash{}{0pt}%
\pgfsys@defobject{currentmarker}{\pgfqpoint{0.000000in}{-0.048611in}}{\pgfqpoint{0.000000in}{0.000000in}}{%
\pgfpathmoveto{\pgfqpoint{0.000000in}{0.000000in}}%
\pgfpathlineto{\pgfqpoint{0.000000in}{-0.048611in}}%
\pgfusepath{stroke,fill}%
}%
\begin{pgfscope}%
\pgfsys@transformshift{12.016552in}{4.121437in}%
\pgfsys@useobject{currentmarker}{}%
\end{pgfscope}%
\end{pgfscope}%
\begin{pgfscope}%
\definecolor{textcolor}{rgb}{0.000000,0.000000,0.000000}%
\pgfsetstrokecolor{textcolor}%
\pgfsetfillcolor{textcolor}%
\pgftext[x=12.016552in,y=4.024215in,,top]{\color{textcolor}\rmfamily\fontsize{10.000000}{12.000000}\selectfont \(\displaystyle 150\)}%
\end{pgfscope}%
\begin{pgfscope}%
\pgfsetbuttcap%
\pgfsetroundjoin%
\definecolor{currentfill}{rgb}{0.000000,0.000000,0.000000}%
\pgfsetfillcolor{currentfill}%
\pgfsetlinewidth{0.803000pt}%
\definecolor{currentstroke}{rgb}{0.000000,0.000000,0.000000}%
\pgfsetstrokecolor{currentstroke}%
\pgfsetdash{}{0pt}%
\pgfsys@defobject{currentmarker}{\pgfqpoint{0.000000in}{-0.048611in}}{\pgfqpoint{0.000000in}{0.000000in}}{%
\pgfpathmoveto{\pgfqpoint{0.000000in}{0.000000in}}%
\pgfpathlineto{\pgfqpoint{0.000000in}{-0.048611in}}%
\pgfusepath{stroke,fill}%
}%
\begin{pgfscope}%
\pgfsys@transformshift{12.716706in}{4.121437in}%
\pgfsys@useobject{currentmarker}{}%
\end{pgfscope}%
\end{pgfscope}%
\begin{pgfscope}%
\definecolor{textcolor}{rgb}{0.000000,0.000000,0.000000}%
\pgfsetstrokecolor{textcolor}%
\pgfsetfillcolor{textcolor}%
\pgftext[x=12.716706in,y=4.024215in,,top]{\color{textcolor}\rmfamily\fontsize{10.000000}{12.000000}\selectfont \(\displaystyle 175\)}%
\end{pgfscope}%
\begin{pgfscope}%
\pgfsetbuttcap%
\pgfsetroundjoin%
\definecolor{currentfill}{rgb}{0.000000,0.000000,0.000000}%
\pgfsetfillcolor{currentfill}%
\pgfsetlinewidth{0.803000pt}%
\definecolor{currentstroke}{rgb}{0.000000,0.000000,0.000000}%
\pgfsetstrokecolor{currentstroke}%
\pgfsetdash{}{0pt}%
\pgfsys@defobject{currentmarker}{\pgfqpoint{-0.048611in}{0.000000in}}{\pgfqpoint{0.000000in}{0.000000in}}{%
\pgfpathmoveto{\pgfqpoint{0.000000in}{0.000000in}}%
\pgfpathlineto{\pgfqpoint{-0.048611in}{0.000000in}}%
\pgfusepath{stroke,fill}%
}%
\begin{pgfscope}%
\pgfsys@transformshift{7.640588in}{4.243502in}%
\pgfsys@useobject{currentmarker}{}%
\end{pgfscope}%
\end{pgfscope}%
\begin{pgfscope}%
\definecolor{textcolor}{rgb}{0.000000,0.000000,0.000000}%
\pgfsetstrokecolor{textcolor}%
\pgfsetfillcolor{textcolor}%
\pgftext[x=7.473921in, y=4.190741in, left, base]{\color{textcolor}\rmfamily\fontsize{10.000000}{12.000000}\selectfont \(\displaystyle 0\)}%
\end{pgfscope}%
\begin{pgfscope}%
\pgfsetbuttcap%
\pgfsetroundjoin%
\definecolor{currentfill}{rgb}{0.000000,0.000000,0.000000}%
\pgfsetfillcolor{currentfill}%
\pgfsetlinewidth{0.803000pt}%
\definecolor{currentstroke}{rgb}{0.000000,0.000000,0.000000}%
\pgfsetstrokecolor{currentstroke}%
\pgfsetdash{}{0pt}%
\pgfsys@defobject{currentmarker}{\pgfqpoint{-0.048611in}{0.000000in}}{\pgfqpoint{0.000000in}{0.000000in}}{%
\pgfpathmoveto{\pgfqpoint{0.000000in}{0.000000in}}%
\pgfpathlineto{\pgfqpoint{-0.048611in}{0.000000in}}%
\pgfusepath{stroke,fill}%
}%
\begin{pgfscope}%
\pgfsys@transformshift{7.640588in}{4.569879in}%
\pgfsys@useobject{currentmarker}{}%
\end{pgfscope}%
\end{pgfscope}%
\begin{pgfscope}%
\definecolor{textcolor}{rgb}{0.000000,0.000000,0.000000}%
\pgfsetstrokecolor{textcolor}%
\pgfsetfillcolor{textcolor}%
\pgftext[x=7.404477in, y=4.517118in, left, base]{\color{textcolor}\rmfamily\fontsize{10.000000}{12.000000}\selectfont \(\displaystyle 25\)}%
\end{pgfscope}%
\begin{pgfscope}%
\pgfsetbuttcap%
\pgfsetroundjoin%
\definecolor{currentfill}{rgb}{0.000000,0.000000,0.000000}%
\pgfsetfillcolor{currentfill}%
\pgfsetlinewidth{0.803000pt}%
\definecolor{currentstroke}{rgb}{0.000000,0.000000,0.000000}%
\pgfsetstrokecolor{currentstroke}%
\pgfsetdash{}{0pt}%
\pgfsys@defobject{currentmarker}{\pgfqpoint{-0.048611in}{0.000000in}}{\pgfqpoint{0.000000in}{0.000000in}}{%
\pgfpathmoveto{\pgfqpoint{0.000000in}{0.000000in}}%
\pgfpathlineto{\pgfqpoint{-0.048611in}{0.000000in}}%
\pgfusepath{stroke,fill}%
}%
\begin{pgfscope}%
\pgfsys@transformshift{7.640588in}{4.896257in}%
\pgfsys@useobject{currentmarker}{}%
\end{pgfscope}%
\end{pgfscope}%
\begin{pgfscope}%
\definecolor{textcolor}{rgb}{0.000000,0.000000,0.000000}%
\pgfsetstrokecolor{textcolor}%
\pgfsetfillcolor{textcolor}%
\pgftext[x=7.404477in, y=4.843495in, left, base]{\color{textcolor}\rmfamily\fontsize{10.000000}{12.000000}\selectfont \(\displaystyle 50\)}%
\end{pgfscope}%
\begin{pgfscope}%
\pgfsetbuttcap%
\pgfsetroundjoin%
\definecolor{currentfill}{rgb}{0.000000,0.000000,0.000000}%
\pgfsetfillcolor{currentfill}%
\pgfsetlinewidth{0.803000pt}%
\definecolor{currentstroke}{rgb}{0.000000,0.000000,0.000000}%
\pgfsetstrokecolor{currentstroke}%
\pgfsetdash{}{0pt}%
\pgfsys@defobject{currentmarker}{\pgfqpoint{-0.048611in}{0.000000in}}{\pgfqpoint{0.000000in}{0.000000in}}{%
\pgfpathmoveto{\pgfqpoint{0.000000in}{0.000000in}}%
\pgfpathlineto{\pgfqpoint{-0.048611in}{0.000000in}}%
\pgfusepath{stroke,fill}%
}%
\begin{pgfscope}%
\pgfsys@transformshift{7.640588in}{5.222634in}%
\pgfsys@useobject{currentmarker}{}%
\end{pgfscope}%
\end{pgfscope}%
\begin{pgfscope}%
\definecolor{textcolor}{rgb}{0.000000,0.000000,0.000000}%
\pgfsetstrokecolor{textcolor}%
\pgfsetfillcolor{textcolor}%
\pgftext[x=7.404477in, y=5.169872in, left, base]{\color{textcolor}\rmfamily\fontsize{10.000000}{12.000000}\selectfont \(\displaystyle 75\)}%
\end{pgfscope}%
\begin{pgfscope}%
\pgfsetbuttcap%
\pgfsetroundjoin%
\definecolor{currentfill}{rgb}{0.000000,0.000000,0.000000}%
\pgfsetfillcolor{currentfill}%
\pgfsetlinewidth{0.803000pt}%
\definecolor{currentstroke}{rgb}{0.000000,0.000000,0.000000}%
\pgfsetstrokecolor{currentstroke}%
\pgfsetdash{}{0pt}%
\pgfsys@defobject{currentmarker}{\pgfqpoint{-0.048611in}{0.000000in}}{\pgfqpoint{0.000000in}{0.000000in}}{%
\pgfpathmoveto{\pgfqpoint{0.000000in}{0.000000in}}%
\pgfpathlineto{\pgfqpoint{-0.048611in}{0.000000in}}%
\pgfusepath{stroke,fill}%
}%
\begin{pgfscope}%
\pgfsys@transformshift{7.640588in}{5.549011in}%
\pgfsys@useobject{currentmarker}{}%
\end{pgfscope}%
\end{pgfscope}%
\begin{pgfscope}%
\definecolor{textcolor}{rgb}{0.000000,0.000000,0.000000}%
\pgfsetstrokecolor{textcolor}%
\pgfsetfillcolor{textcolor}%
\pgftext[x=7.335032in, y=5.496250in, left, base]{\color{textcolor}\rmfamily\fontsize{10.000000}{12.000000}\selectfont \(\displaystyle 100\)}%
\end{pgfscope}%
\begin{pgfscope}%
\pgfsetbuttcap%
\pgfsetroundjoin%
\definecolor{currentfill}{rgb}{0.000000,0.000000,0.000000}%
\pgfsetfillcolor{currentfill}%
\pgfsetlinewidth{0.803000pt}%
\definecolor{currentstroke}{rgb}{0.000000,0.000000,0.000000}%
\pgfsetstrokecolor{currentstroke}%
\pgfsetdash{}{0pt}%
\pgfsys@defobject{currentmarker}{\pgfqpoint{-0.048611in}{0.000000in}}{\pgfqpoint{0.000000in}{0.000000in}}{%
\pgfpathmoveto{\pgfqpoint{0.000000in}{0.000000in}}%
\pgfpathlineto{\pgfqpoint{-0.048611in}{0.000000in}}%
\pgfusepath{stroke,fill}%
}%
\begin{pgfscope}%
\pgfsys@transformshift{7.640588in}{5.875389in}%
\pgfsys@useobject{currentmarker}{}%
\end{pgfscope}%
\end{pgfscope}%
\begin{pgfscope}%
\definecolor{textcolor}{rgb}{0.000000,0.000000,0.000000}%
\pgfsetstrokecolor{textcolor}%
\pgfsetfillcolor{textcolor}%
\pgftext[x=7.335032in, y=5.822627in, left, base]{\color{textcolor}\rmfamily\fontsize{10.000000}{12.000000}\selectfont \(\displaystyle 125\)}%
\end{pgfscope}%
\begin{pgfscope}%
\pgfsetbuttcap%
\pgfsetroundjoin%
\definecolor{currentfill}{rgb}{0.000000,0.000000,0.000000}%
\pgfsetfillcolor{currentfill}%
\pgfsetlinewidth{0.803000pt}%
\definecolor{currentstroke}{rgb}{0.000000,0.000000,0.000000}%
\pgfsetstrokecolor{currentstroke}%
\pgfsetdash{}{0pt}%
\pgfsys@defobject{currentmarker}{\pgfqpoint{-0.048611in}{0.000000in}}{\pgfqpoint{0.000000in}{0.000000in}}{%
\pgfpathmoveto{\pgfqpoint{0.000000in}{0.000000in}}%
\pgfpathlineto{\pgfqpoint{-0.048611in}{0.000000in}}%
\pgfusepath{stroke,fill}%
}%
\begin{pgfscope}%
\pgfsys@transformshift{7.640588in}{6.201766in}%
\pgfsys@useobject{currentmarker}{}%
\end{pgfscope}%
\end{pgfscope}%
\begin{pgfscope}%
\definecolor{textcolor}{rgb}{0.000000,0.000000,0.000000}%
\pgfsetstrokecolor{textcolor}%
\pgfsetfillcolor{textcolor}%
\pgftext[x=7.335032in, y=6.149004in, left, base]{\color{textcolor}\rmfamily\fontsize{10.000000}{12.000000}\selectfont \(\displaystyle 150\)}%
\end{pgfscope}%
\begin{pgfscope}%
\pgfsetbuttcap%
\pgfsetroundjoin%
\definecolor{currentfill}{rgb}{0.000000,0.000000,0.000000}%
\pgfsetfillcolor{currentfill}%
\pgfsetlinewidth{0.803000pt}%
\definecolor{currentstroke}{rgb}{0.000000,0.000000,0.000000}%
\pgfsetstrokecolor{currentstroke}%
\pgfsetdash{}{0pt}%
\pgfsys@defobject{currentmarker}{\pgfqpoint{-0.048611in}{0.000000in}}{\pgfqpoint{0.000000in}{0.000000in}}{%
\pgfpathmoveto{\pgfqpoint{0.000000in}{0.000000in}}%
\pgfpathlineto{\pgfqpoint{-0.048611in}{0.000000in}}%
\pgfusepath{stroke,fill}%
}%
\begin{pgfscope}%
\pgfsys@transformshift{7.640588in}{6.528143in}%
\pgfsys@useobject{currentmarker}{}%
\end{pgfscope}%
\end{pgfscope}%
\begin{pgfscope}%
\definecolor{textcolor}{rgb}{0.000000,0.000000,0.000000}%
\pgfsetstrokecolor{textcolor}%
\pgfsetfillcolor{textcolor}%
\pgftext[x=7.335032in, y=6.475382in, left, base]{\color{textcolor}\rmfamily\fontsize{10.000000}{12.000000}\selectfont \(\displaystyle 175\)}%
\end{pgfscope}%
\begin{pgfscope}%
\pgfsetrectcap%
\pgfsetmiterjoin%
\pgfsetlinewidth{0.803000pt}%
\definecolor{currentstroke}{rgb}{0.000000,0.000000,0.000000}%
\pgfsetstrokecolor{currentstroke}%
\pgfsetdash{}{0pt}%
\pgfpathmoveto{\pgfqpoint{7.640588in}{4.121437in}}%
\pgfpathlineto{\pgfqpoint{7.640588in}{6.806869in}}%
\pgfusepath{stroke}%
\end{pgfscope}%
\begin{pgfscope}%
\pgfsetrectcap%
\pgfsetmiterjoin%
\pgfsetlinewidth{0.803000pt}%
\definecolor{currentstroke}{rgb}{0.000000,0.000000,0.000000}%
\pgfsetstrokecolor{currentstroke}%
\pgfsetdash{}{0pt}%
\pgfpathmoveto{\pgfqpoint{13.339844in}{4.121437in}}%
\pgfpathlineto{\pgfqpoint{13.339844in}{6.806869in}}%
\pgfusepath{stroke}%
\end{pgfscope}%
\begin{pgfscope}%
\pgfsetrectcap%
\pgfsetmiterjoin%
\pgfsetlinewidth{0.803000pt}%
\definecolor{currentstroke}{rgb}{0.000000,0.000000,0.000000}%
\pgfsetstrokecolor{currentstroke}%
\pgfsetdash{}{0pt}%
\pgfpathmoveto{\pgfqpoint{7.640588in}{4.121437in}}%
\pgfpathlineto{\pgfqpoint{13.339844in}{4.121437in}}%
\pgfusepath{stroke}%
\end{pgfscope}%
\begin{pgfscope}%
\pgfsetrectcap%
\pgfsetmiterjoin%
\pgfsetlinewidth{0.803000pt}%
\definecolor{currentstroke}{rgb}{0.000000,0.000000,0.000000}%
\pgfsetstrokecolor{currentstroke}%
\pgfsetdash{}{0pt}%
\pgfpathmoveto{\pgfqpoint{7.640588in}{6.806869in}}%
\pgfpathlineto{\pgfqpoint{13.339844in}{6.806869in}}%
\pgfusepath{stroke}%
\end{pgfscope}%
\begin{pgfscope}%
\definecolor{textcolor}{rgb}{0.000000,0.000000,0.000000}%
\pgfsetstrokecolor{textcolor}%
\pgfsetfillcolor{textcolor}%
\pgftext[x=10.490216in,y=6.890203in,,base]{\color{textcolor}\rmfamily\fontsize{12.000000}{14.400000}\selectfont twoopt 1600.7}%
\end{pgfscope}%
\begin{pgfscope}%
\pgfsetbuttcap%
\pgfsetmiterjoin%
\definecolor{currentfill}{rgb}{1.000000,1.000000,1.000000}%
\pgfsetfillcolor{currentfill}%
\pgfsetlinewidth{0.000000pt}%
\definecolor{currentstroke}{rgb}{0.000000,0.000000,0.000000}%
\pgfsetstrokecolor{currentstroke}%
\pgfsetstrokeopacity{0.000000}%
\pgfsetdash{}{0pt}%
\pgfpathmoveto{\pgfqpoint{0.970666in}{0.566125in}}%
\pgfpathlineto{\pgfqpoint{6.669922in}{0.566125in}}%
\pgfpathlineto{\pgfqpoint{6.669922in}{3.251557in}}%
\pgfpathlineto{\pgfqpoint{0.970666in}{3.251557in}}%
\pgfpathclose%
\pgfusepath{fill}%
\end{pgfscope}%
\begin{pgfscope}%
\pgfpathrectangle{\pgfqpoint{0.970666in}{0.566125in}}{\pgfqpoint{5.699255in}{2.685432in}}%
\pgfusepath{clip}%
\pgfsetrectcap%
\pgfsetroundjoin%
\pgfsetlinewidth{1.505625pt}%
\definecolor{currentstroke}{rgb}{0.000000,0.000000,0.000000}%
\pgfsetstrokecolor{currentstroke}%
\pgfsetdash{}{0pt}%
\pgfpathmoveto{\pgfqpoint{5.402643in}{1.458440in}}%
\pgfpathlineto{\pgfqpoint{4.982550in}{1.510660in}}%
\pgfpathlineto{\pgfqpoint{4.730495in}{1.432330in}}%
\pgfpathlineto{\pgfqpoint{4.506445in}{1.510660in}}%
\pgfpathlineto{\pgfqpoint{4.142365in}{1.562881in}}%
\pgfpathlineto{\pgfqpoint{4.058347in}{1.549826in}}%
\pgfpathlineto{\pgfqpoint{3.806291in}{1.602046in}}%
\pgfpathlineto{\pgfqpoint{3.694266in}{1.536771in}}%
\pgfpathlineto{\pgfqpoint{3.470217in}{1.432330in}}%
\pgfpathlineto{\pgfqpoint{2.658038in}{1.301779in}}%
\pgfpathlineto{\pgfqpoint{2.630032in}{1.275669in}}%
\pgfpathlineto{\pgfqpoint{2.377976in}{1.184283in}}%
\pgfpathlineto{\pgfqpoint{2.405983in}{1.092898in}}%
\pgfpathlineto{\pgfqpoint{2.405983in}{0.936236in}}%
\pgfpathlineto{\pgfqpoint{1.929878in}{0.988457in}}%
\pgfpathlineto{\pgfqpoint{1.593804in}{0.831796in}}%
\pgfpathlineto{\pgfqpoint{1.733835in}{0.688190in}}%
\pgfpathlineto{\pgfqpoint{1.845859in}{0.714300in}}%
\pgfpathlineto{\pgfqpoint{1.621810in}{1.184283in}}%
\pgfpathlineto{\pgfqpoint{1.901872in}{1.275669in}}%
\pgfpathlineto{\pgfqpoint{1.873865in}{1.393165in}}%
\pgfpathlineto{\pgfqpoint{1.901872in}{1.497605in}}%
\pgfpathlineto{\pgfqpoint{1.873865in}{1.654266in}}%
\pgfpathlineto{\pgfqpoint{1.817853in}{1.850093in}}%
\pgfpathlineto{\pgfqpoint{2.405983in}{2.032864in}}%
\pgfpathlineto{\pgfqpoint{2.602026in}{2.006754in}}%
\pgfpathlineto{\pgfqpoint{2.686044in}{2.085084in}}%
\pgfpathlineto{\pgfqpoint{2.714051in}{2.111195in}}%
\pgfpathlineto{\pgfqpoint{2.910094in}{2.254801in}}%
\pgfpathlineto{\pgfqpoint{2.910094in}{2.385351in}}%
\pgfpathlineto{\pgfqpoint{3.414205in}{2.385351in}}%
\pgfpathlineto{\pgfqpoint{3.442211in}{2.254801in}}%
\pgfpathlineto{\pgfqpoint{3.946322in}{2.450627in}}%
\pgfpathlineto{\pgfqpoint{4.226384in}{2.359241in}}%
\pgfpathlineto{\pgfqpoint{4.310402in}{2.293966in}}%
\pgfpathlineto{\pgfqpoint{5.318624in}{2.333131in}}%
\pgfpathlineto{\pgfqpoint{5.430649in}{2.241745in}}%
\pgfpathlineto{\pgfqpoint{5.486661in}{2.463682in}}%
\pgfpathlineto{\pgfqpoint{5.570680in}{2.502847in}}%
\pgfpathlineto{\pgfqpoint{5.906754in}{2.372296in}}%
\pgfpathlineto{\pgfqpoint{6.326846in}{2.267856in}}%
\pgfpathlineto{\pgfqpoint{6.410865in}{2.424517in}}%
\pgfpathlineto{\pgfqpoint{5.990772in}{2.711729in}}%
\pgfpathlineto{\pgfqpoint{5.906754in}{2.803114in}}%
\pgfpathlineto{\pgfqpoint{6.102797in}{2.855335in}}%
\pgfpathlineto{\pgfqpoint{6.242828in}{3.116437in}}%
\pgfpathlineto{\pgfqpoint{5.710711in}{3.129492in}}%
\pgfpathlineto{\pgfqpoint{5.318624in}{2.842280in}}%
\pgfpathlineto{\pgfqpoint{5.234606in}{2.868390in}}%
\pgfpathlineto{\pgfqpoint{4.982550in}{2.946720in}}%
\pgfpathlineto{\pgfqpoint{4.842519in}{2.998941in}}%
\pgfpathlineto{\pgfqpoint{4.702489in}{2.933665in}}%
\pgfpathlineto{\pgfqpoint{3.946322in}{3.077271in}}%
\pgfpathlineto{\pgfqpoint{3.554236in}{2.894500in}}%
\pgfpathlineto{\pgfqpoint{3.526229in}{2.829225in}}%
\pgfpathlineto{\pgfqpoint{3.162149in}{2.920610in}}%
\pgfpathlineto{\pgfqpoint{2.602026in}{2.750894in}}%
\pgfpathlineto{\pgfqpoint{2.377976in}{2.698674in}}%
\pgfpathlineto{\pgfqpoint{2.209939in}{2.620343in}}%
\pgfpathlineto{\pgfqpoint{2.153927in}{2.476737in}}%
\pgfpathlineto{\pgfqpoint{1.985890in}{2.307021in}}%
\pgfpathlineto{\pgfqpoint{1.677822in}{2.424517in}}%
\pgfpathlineto{\pgfqpoint{1.397761in}{2.424517in}}%
\pgfpathlineto{\pgfqpoint{1.285736in}{2.711729in}}%
\pgfpathlineto{\pgfqpoint{1.229724in}{2.777004in}}%
\pgfpathlineto{\pgfqpoint{1.369754in}{2.855335in}}%
\pgfpathlineto{\pgfqpoint{1.733835in}{2.842280in}}%
\pgfpathlineto{\pgfqpoint{1.845859in}{2.724784in}}%
\pgfpathlineto{\pgfqpoint{2.377976in}{2.868390in}}%
\pgfpathlineto{\pgfqpoint{2.461995in}{3.116437in}}%
\pgfpathlineto{\pgfqpoint{3.078131in}{2.607288in}}%
\pgfpathlineto{\pgfqpoint{3.890310in}{2.737839in}}%
\pgfpathlineto{\pgfqpoint{3.806291in}{1.863148in}}%
\pgfpathlineto{\pgfqpoint{3.190155in}{1.915368in}}%
\pgfpathlineto{\pgfqpoint{2.938100in}{1.850093in}}%
\pgfpathlineto{\pgfqpoint{2.910094in}{1.784817in}}%
\pgfpathlineto{\pgfqpoint{4.562458in}{1.719542in}}%
\pgfpathlineto{\pgfqpoint{5.794729in}{2.006754in}}%
\pgfpathlineto{\pgfqpoint{6.214822in}{1.667321in}}%
\pgfpathlineto{\pgfqpoint{5.514667in}{1.223448in}}%
\pgfpathlineto{\pgfqpoint{5.626692in}{1.079842in}}%
\pgfpathlineto{\pgfqpoint{5.318624in}{1.027622in}}%
\pgfpathlineto{\pgfqpoint{5.262612in}{0.988457in}}%
\pgfpathlineto{\pgfqpoint{5.234606in}{0.975402in}}%
\pgfpathlineto{\pgfqpoint{5.262612in}{1.079842in}}%
\pgfpathlineto{\pgfqpoint{4.898532in}{0.936236in}}%
\pgfpathlineto{\pgfqpoint{4.898532in}{0.923181in}}%
\pgfpathlineto{\pgfqpoint{4.590464in}{0.923181in}}%
\pgfpathlineto{\pgfqpoint{4.086353in}{0.884016in}}%
\pgfpathlineto{\pgfqpoint{4.030340in}{0.831796in}}%
\pgfpathlineto{\pgfqpoint{3.890310in}{0.897071in}}%
\pgfpathlineto{\pgfqpoint{3.078131in}{0.844851in}}%
\pgfpathlineto{\pgfqpoint{1.509785in}{1.353999in}}%
\pgfpathlineto{\pgfqpoint{1.369754in}{1.393165in}}%
\pgfpathlineto{\pgfqpoint{5.990772in}{0.884016in}}%
\pgfpathlineto{\pgfqpoint{5.402643in}{1.458440in}}%
\pgfusepath{stroke}%
\end{pgfscope}%
\begin{pgfscope}%
\pgfpathrectangle{\pgfqpoint{0.970666in}{0.566125in}}{\pgfqpoint{5.699255in}{2.685432in}}%
\pgfusepath{clip}%
\pgfsetbuttcap%
\pgfsetroundjoin%
\definecolor{currentfill}{rgb}{1.000000,0.000000,0.000000}%
\pgfsetfillcolor{currentfill}%
\pgfsetlinewidth{1.003750pt}%
\definecolor{currentstroke}{rgb}{1.000000,0.000000,0.000000}%
\pgfsetstrokecolor{currentstroke}%
\pgfsetdash{}{0pt}%
\pgfpathmoveto{\pgfqpoint{5.402643in}{1.416773in}}%
\pgfpathcurveto{\pgfqpoint{5.413693in}{1.416773in}}{\pgfqpoint{5.424292in}{1.421164in}}{\pgfqpoint{5.432106in}{1.428977in}}%
\pgfpathcurveto{\pgfqpoint{5.439919in}{1.436791in}}{\pgfqpoint{5.444309in}{1.447390in}}{\pgfqpoint{5.444309in}{1.458440in}}%
\pgfpathcurveto{\pgfqpoint{5.444309in}{1.469490in}}{\pgfqpoint{5.439919in}{1.480089in}}{\pgfqpoint{5.432106in}{1.487903in}}%
\pgfpathcurveto{\pgfqpoint{5.424292in}{1.495716in}}{\pgfqpoint{5.413693in}{1.500107in}}{\pgfqpoint{5.402643in}{1.500107in}}%
\pgfpathcurveto{\pgfqpoint{5.391593in}{1.500107in}}{\pgfqpoint{5.380994in}{1.495716in}}{\pgfqpoint{5.373180in}{1.487903in}}%
\pgfpathcurveto{\pgfqpoint{5.365366in}{1.480089in}}{\pgfqpoint{5.360976in}{1.469490in}}{\pgfqpoint{5.360976in}{1.458440in}}%
\pgfpathcurveto{\pgfqpoint{5.360976in}{1.447390in}}{\pgfqpoint{5.365366in}{1.436791in}}{\pgfqpoint{5.373180in}{1.428977in}}%
\pgfpathcurveto{\pgfqpoint{5.380994in}{1.421164in}}{\pgfqpoint{5.391593in}{1.416773in}}{\pgfqpoint{5.402643in}{1.416773in}}%
\pgfpathclose%
\pgfusepath{stroke,fill}%
\end{pgfscope}%
\begin{pgfscope}%
\pgfpathrectangle{\pgfqpoint{0.970666in}{0.566125in}}{\pgfqpoint{5.699255in}{2.685432in}}%
\pgfusepath{clip}%
\pgfsetbuttcap%
\pgfsetroundjoin%
\definecolor{currentfill}{rgb}{0.750000,0.750000,0.000000}%
\pgfsetfillcolor{currentfill}%
\pgfsetlinewidth{1.003750pt}%
\definecolor{currentstroke}{rgb}{0.750000,0.750000,0.000000}%
\pgfsetstrokecolor{currentstroke}%
\pgfsetdash{}{0pt}%
\pgfpathmoveto{\pgfqpoint{4.982550in}{1.468994in}}%
\pgfpathcurveto{\pgfqpoint{4.993600in}{1.468994in}}{\pgfqpoint{5.004199in}{1.473384in}}{\pgfqpoint{5.012013in}{1.481198in}}%
\pgfpathcurveto{\pgfqpoint{5.019827in}{1.489011in}}{\pgfqpoint{5.024217in}{1.499610in}}{\pgfqpoint{5.024217in}{1.510660in}}%
\pgfpathcurveto{\pgfqpoint{5.024217in}{1.521711in}}{\pgfqpoint{5.019827in}{1.532310in}}{\pgfqpoint{5.012013in}{1.540123in}}%
\pgfpathcurveto{\pgfqpoint{5.004199in}{1.547937in}}{\pgfqpoint{4.993600in}{1.552327in}}{\pgfqpoint{4.982550in}{1.552327in}}%
\pgfpathcurveto{\pgfqpoint{4.971500in}{1.552327in}}{\pgfqpoint{4.960901in}{1.547937in}}{\pgfqpoint{4.953087in}{1.540123in}}%
\pgfpathcurveto{\pgfqpoint{4.945274in}{1.532310in}}{\pgfqpoint{4.940884in}{1.521711in}}{\pgfqpoint{4.940884in}{1.510660in}}%
\pgfpathcurveto{\pgfqpoint{4.940884in}{1.499610in}}{\pgfqpoint{4.945274in}{1.489011in}}{\pgfqpoint{4.953087in}{1.481198in}}%
\pgfpathcurveto{\pgfqpoint{4.960901in}{1.473384in}}{\pgfqpoint{4.971500in}{1.468994in}}{\pgfqpoint{4.982550in}{1.468994in}}%
\pgfpathclose%
\pgfusepath{stroke,fill}%
\end{pgfscope}%
\begin{pgfscope}%
\pgfpathrectangle{\pgfqpoint{0.970666in}{0.566125in}}{\pgfqpoint{5.699255in}{2.685432in}}%
\pgfusepath{clip}%
\pgfsetbuttcap%
\pgfsetroundjoin%
\definecolor{currentfill}{rgb}{0.000000,0.000000,0.000000}%
\pgfsetfillcolor{currentfill}%
\pgfsetlinewidth{1.003750pt}%
\definecolor{currentstroke}{rgb}{0.000000,0.000000,0.000000}%
\pgfsetstrokecolor{currentstroke}%
\pgfsetdash{}{0pt}%
\pgfpathmoveto{\pgfqpoint{4.730495in}{1.390663in}}%
\pgfpathcurveto{\pgfqpoint{4.741545in}{1.390663in}}{\pgfqpoint{4.752144in}{1.395053in}}{\pgfqpoint{4.759957in}{1.402867in}}%
\pgfpathcurveto{\pgfqpoint{4.767771in}{1.410681in}}{\pgfqpoint{4.772161in}{1.421280in}}{\pgfqpoint{4.772161in}{1.432330in}}%
\pgfpathcurveto{\pgfqpoint{4.772161in}{1.443380in}}{\pgfqpoint{4.767771in}{1.453979in}}{\pgfqpoint{4.759957in}{1.461793in}}%
\pgfpathcurveto{\pgfqpoint{4.752144in}{1.469606in}}{\pgfqpoint{4.741545in}{1.473997in}}{\pgfqpoint{4.730495in}{1.473997in}}%
\pgfpathcurveto{\pgfqpoint{4.719445in}{1.473997in}}{\pgfqpoint{4.708846in}{1.469606in}}{\pgfqpoint{4.701032in}{1.461793in}}%
\pgfpathcurveto{\pgfqpoint{4.693218in}{1.453979in}}{\pgfqpoint{4.688828in}{1.443380in}}{\pgfqpoint{4.688828in}{1.432330in}}%
\pgfpathcurveto{\pgfqpoint{4.688828in}{1.421280in}}{\pgfqpoint{4.693218in}{1.410681in}}{\pgfqpoint{4.701032in}{1.402867in}}%
\pgfpathcurveto{\pgfqpoint{4.708846in}{1.395053in}}{\pgfqpoint{4.719445in}{1.390663in}}{\pgfqpoint{4.730495in}{1.390663in}}%
\pgfpathclose%
\pgfusepath{stroke,fill}%
\end{pgfscope}%
\begin{pgfscope}%
\pgfpathrectangle{\pgfqpoint{0.970666in}{0.566125in}}{\pgfqpoint{5.699255in}{2.685432in}}%
\pgfusepath{clip}%
\pgfsetbuttcap%
\pgfsetroundjoin%
\definecolor{currentfill}{rgb}{0.000000,0.000000,0.000000}%
\pgfsetfillcolor{currentfill}%
\pgfsetlinewidth{1.003750pt}%
\definecolor{currentstroke}{rgb}{0.000000,0.000000,0.000000}%
\pgfsetstrokecolor{currentstroke}%
\pgfsetdash{}{0pt}%
\pgfpathmoveto{\pgfqpoint{4.506445in}{1.468994in}}%
\pgfpathcurveto{\pgfqpoint{4.517495in}{1.468994in}}{\pgfqpoint{4.528094in}{1.473384in}}{\pgfqpoint{4.535908in}{1.481198in}}%
\pgfpathcurveto{\pgfqpoint{4.543722in}{1.489011in}}{\pgfqpoint{4.548112in}{1.499610in}}{\pgfqpoint{4.548112in}{1.510660in}}%
\pgfpathcurveto{\pgfqpoint{4.548112in}{1.521711in}}{\pgfqpoint{4.543722in}{1.532310in}}{\pgfqpoint{4.535908in}{1.540123in}}%
\pgfpathcurveto{\pgfqpoint{4.528094in}{1.547937in}}{\pgfqpoint{4.517495in}{1.552327in}}{\pgfqpoint{4.506445in}{1.552327in}}%
\pgfpathcurveto{\pgfqpoint{4.495395in}{1.552327in}}{\pgfqpoint{4.484796in}{1.547937in}}{\pgfqpoint{4.476983in}{1.540123in}}%
\pgfpathcurveto{\pgfqpoint{4.469169in}{1.532310in}}{\pgfqpoint{4.464779in}{1.521711in}}{\pgfqpoint{4.464779in}{1.510660in}}%
\pgfpathcurveto{\pgfqpoint{4.464779in}{1.499610in}}{\pgfqpoint{4.469169in}{1.489011in}}{\pgfqpoint{4.476983in}{1.481198in}}%
\pgfpathcurveto{\pgfqpoint{4.484796in}{1.473384in}}{\pgfqpoint{4.495395in}{1.468994in}}{\pgfqpoint{4.506445in}{1.468994in}}%
\pgfpathclose%
\pgfusepath{stroke,fill}%
\end{pgfscope}%
\begin{pgfscope}%
\pgfpathrectangle{\pgfqpoint{0.970666in}{0.566125in}}{\pgfqpoint{5.699255in}{2.685432in}}%
\pgfusepath{clip}%
\pgfsetbuttcap%
\pgfsetroundjoin%
\definecolor{currentfill}{rgb}{0.000000,0.000000,0.000000}%
\pgfsetfillcolor{currentfill}%
\pgfsetlinewidth{1.003750pt}%
\definecolor{currentstroke}{rgb}{0.000000,0.000000,0.000000}%
\pgfsetstrokecolor{currentstroke}%
\pgfsetdash{}{0pt}%
\pgfpathmoveto{\pgfqpoint{4.142365in}{1.521214in}}%
\pgfpathcurveto{\pgfqpoint{4.153415in}{1.521214in}}{\pgfqpoint{4.164014in}{1.525604in}}{\pgfqpoint{4.171828in}{1.533418in}}%
\pgfpathcurveto{\pgfqpoint{4.179642in}{1.541232in}}{\pgfqpoint{4.184032in}{1.551831in}}{\pgfqpoint{4.184032in}{1.562881in}}%
\pgfpathcurveto{\pgfqpoint{4.184032in}{1.573931in}}{\pgfqpoint{4.179642in}{1.584530in}}{\pgfqpoint{4.171828in}{1.592344in}}%
\pgfpathcurveto{\pgfqpoint{4.164014in}{1.600157in}}{\pgfqpoint{4.153415in}{1.604547in}}{\pgfqpoint{4.142365in}{1.604547in}}%
\pgfpathcurveto{\pgfqpoint{4.131315in}{1.604547in}}{\pgfqpoint{4.120716in}{1.600157in}}{\pgfqpoint{4.112902in}{1.592344in}}%
\pgfpathcurveto{\pgfqpoint{4.105089in}{1.584530in}}{\pgfqpoint{4.100698in}{1.573931in}}{\pgfqpoint{4.100698in}{1.562881in}}%
\pgfpathcurveto{\pgfqpoint{4.100698in}{1.551831in}}{\pgfqpoint{4.105089in}{1.541232in}}{\pgfqpoint{4.112902in}{1.533418in}}%
\pgfpathcurveto{\pgfqpoint{4.120716in}{1.525604in}}{\pgfqpoint{4.131315in}{1.521214in}}{\pgfqpoint{4.142365in}{1.521214in}}%
\pgfpathclose%
\pgfusepath{stroke,fill}%
\end{pgfscope}%
\begin{pgfscope}%
\pgfpathrectangle{\pgfqpoint{0.970666in}{0.566125in}}{\pgfqpoint{5.699255in}{2.685432in}}%
\pgfusepath{clip}%
\pgfsetbuttcap%
\pgfsetroundjoin%
\definecolor{currentfill}{rgb}{0.000000,0.000000,0.000000}%
\pgfsetfillcolor{currentfill}%
\pgfsetlinewidth{1.003750pt}%
\definecolor{currentstroke}{rgb}{0.000000,0.000000,0.000000}%
\pgfsetstrokecolor{currentstroke}%
\pgfsetdash{}{0pt}%
\pgfpathmoveto{\pgfqpoint{4.058347in}{1.508159in}}%
\pgfpathcurveto{\pgfqpoint{4.069397in}{1.508159in}}{\pgfqpoint{4.079996in}{1.512549in}}{\pgfqpoint{4.087809in}{1.520363in}}%
\pgfpathcurveto{\pgfqpoint{4.095623in}{1.528177in}}{\pgfqpoint{4.100013in}{1.538776in}}{\pgfqpoint{4.100013in}{1.549826in}}%
\pgfpathcurveto{\pgfqpoint{4.100013in}{1.560876in}}{\pgfqpoint{4.095623in}{1.571475in}}{\pgfqpoint{4.087809in}{1.579288in}}%
\pgfpathcurveto{\pgfqpoint{4.079996in}{1.587102in}}{\pgfqpoint{4.069397in}{1.591492in}}{\pgfqpoint{4.058347in}{1.591492in}}%
\pgfpathcurveto{\pgfqpoint{4.047296in}{1.591492in}}{\pgfqpoint{4.036697in}{1.587102in}}{\pgfqpoint{4.028884in}{1.579288in}}%
\pgfpathcurveto{\pgfqpoint{4.021070in}{1.571475in}}{\pgfqpoint{4.016680in}{1.560876in}}{\pgfqpoint{4.016680in}{1.549826in}}%
\pgfpathcurveto{\pgfqpoint{4.016680in}{1.538776in}}{\pgfqpoint{4.021070in}{1.528177in}}{\pgfqpoint{4.028884in}{1.520363in}}%
\pgfpathcurveto{\pgfqpoint{4.036697in}{1.512549in}}{\pgfqpoint{4.047296in}{1.508159in}}{\pgfqpoint{4.058347in}{1.508159in}}%
\pgfpathclose%
\pgfusepath{stroke,fill}%
\end{pgfscope}%
\begin{pgfscope}%
\pgfpathrectangle{\pgfqpoint{0.970666in}{0.566125in}}{\pgfqpoint{5.699255in}{2.685432in}}%
\pgfusepath{clip}%
\pgfsetbuttcap%
\pgfsetroundjoin%
\definecolor{currentfill}{rgb}{0.000000,0.000000,0.000000}%
\pgfsetfillcolor{currentfill}%
\pgfsetlinewidth{1.003750pt}%
\definecolor{currentstroke}{rgb}{0.000000,0.000000,0.000000}%
\pgfsetstrokecolor{currentstroke}%
\pgfsetdash{}{0pt}%
\pgfpathmoveto{\pgfqpoint{3.806291in}{1.560379in}}%
\pgfpathcurveto{\pgfqpoint{3.817341in}{1.560379in}}{\pgfqpoint{3.827940in}{1.564770in}}{\pgfqpoint{3.835754in}{1.572583in}}%
\pgfpathcurveto{\pgfqpoint{3.843568in}{1.580397in}}{\pgfqpoint{3.847958in}{1.590996in}}{\pgfqpoint{3.847958in}{1.602046in}}%
\pgfpathcurveto{\pgfqpoint{3.847958in}{1.613096in}}{\pgfqpoint{3.843568in}{1.623695in}}{\pgfqpoint{3.835754in}{1.631509in}}%
\pgfpathcurveto{\pgfqpoint{3.827940in}{1.639322in}}{\pgfqpoint{3.817341in}{1.643713in}}{\pgfqpoint{3.806291in}{1.643713in}}%
\pgfpathcurveto{\pgfqpoint{3.795241in}{1.643713in}}{\pgfqpoint{3.784642in}{1.639322in}}{\pgfqpoint{3.776828in}{1.631509in}}%
\pgfpathcurveto{\pgfqpoint{3.769015in}{1.623695in}}{\pgfqpoint{3.764624in}{1.613096in}}{\pgfqpoint{3.764624in}{1.602046in}}%
\pgfpathcurveto{\pgfqpoint{3.764624in}{1.590996in}}{\pgfqpoint{3.769015in}{1.580397in}}{\pgfqpoint{3.776828in}{1.572583in}}%
\pgfpathcurveto{\pgfqpoint{3.784642in}{1.564770in}}{\pgfqpoint{3.795241in}{1.560379in}}{\pgfqpoint{3.806291in}{1.560379in}}%
\pgfpathclose%
\pgfusepath{stroke,fill}%
\end{pgfscope}%
\begin{pgfscope}%
\pgfpathrectangle{\pgfqpoint{0.970666in}{0.566125in}}{\pgfqpoint{5.699255in}{2.685432in}}%
\pgfusepath{clip}%
\pgfsetbuttcap%
\pgfsetroundjoin%
\definecolor{currentfill}{rgb}{0.000000,0.000000,0.000000}%
\pgfsetfillcolor{currentfill}%
\pgfsetlinewidth{1.003750pt}%
\definecolor{currentstroke}{rgb}{0.000000,0.000000,0.000000}%
\pgfsetstrokecolor{currentstroke}%
\pgfsetdash{}{0pt}%
\pgfpathmoveto{\pgfqpoint{3.694266in}{1.495104in}}%
\pgfpathcurveto{\pgfqpoint{3.705317in}{1.495104in}}{\pgfqpoint{3.715916in}{1.499494in}}{\pgfqpoint{3.723729in}{1.507308in}}%
\pgfpathcurveto{\pgfqpoint{3.731543in}{1.515121in}}{\pgfqpoint{3.735933in}{1.525720in}}{\pgfqpoint{3.735933in}{1.536771in}}%
\pgfpathcurveto{\pgfqpoint{3.735933in}{1.547821in}}{\pgfqpoint{3.731543in}{1.558420in}}{\pgfqpoint{3.723729in}{1.566233in}}%
\pgfpathcurveto{\pgfqpoint{3.715916in}{1.574047in}}{\pgfqpoint{3.705317in}{1.578437in}}{\pgfqpoint{3.694266in}{1.578437in}}%
\pgfpathcurveto{\pgfqpoint{3.683216in}{1.578437in}}{\pgfqpoint{3.672617in}{1.574047in}}{\pgfqpoint{3.664804in}{1.566233in}}%
\pgfpathcurveto{\pgfqpoint{3.656990in}{1.558420in}}{\pgfqpoint{3.652600in}{1.547821in}}{\pgfqpoint{3.652600in}{1.536771in}}%
\pgfpathcurveto{\pgfqpoint{3.652600in}{1.525720in}}{\pgfqpoint{3.656990in}{1.515121in}}{\pgfqpoint{3.664804in}{1.507308in}}%
\pgfpathcurveto{\pgfqpoint{3.672617in}{1.499494in}}{\pgfqpoint{3.683216in}{1.495104in}}{\pgfqpoint{3.694266in}{1.495104in}}%
\pgfpathclose%
\pgfusepath{stroke,fill}%
\end{pgfscope}%
\begin{pgfscope}%
\pgfpathrectangle{\pgfqpoint{0.970666in}{0.566125in}}{\pgfqpoint{5.699255in}{2.685432in}}%
\pgfusepath{clip}%
\pgfsetbuttcap%
\pgfsetroundjoin%
\definecolor{currentfill}{rgb}{0.000000,0.000000,0.000000}%
\pgfsetfillcolor{currentfill}%
\pgfsetlinewidth{1.003750pt}%
\definecolor{currentstroke}{rgb}{0.000000,0.000000,0.000000}%
\pgfsetstrokecolor{currentstroke}%
\pgfsetdash{}{0pt}%
\pgfpathmoveto{\pgfqpoint{3.470217in}{1.390663in}}%
\pgfpathcurveto{\pgfqpoint{3.481267in}{1.390663in}}{\pgfqpoint{3.491866in}{1.395053in}}{\pgfqpoint{3.499680in}{1.402867in}}%
\pgfpathcurveto{\pgfqpoint{3.507493in}{1.410681in}}{\pgfqpoint{3.511884in}{1.421280in}}{\pgfqpoint{3.511884in}{1.432330in}}%
\pgfpathcurveto{\pgfqpoint{3.511884in}{1.443380in}}{\pgfqpoint{3.507493in}{1.453979in}}{\pgfqpoint{3.499680in}{1.461793in}}%
\pgfpathcurveto{\pgfqpoint{3.491866in}{1.469606in}}{\pgfqpoint{3.481267in}{1.473997in}}{\pgfqpoint{3.470217in}{1.473997in}}%
\pgfpathcurveto{\pgfqpoint{3.459167in}{1.473997in}}{\pgfqpoint{3.448568in}{1.469606in}}{\pgfqpoint{3.440754in}{1.461793in}}%
\pgfpathcurveto{\pgfqpoint{3.432941in}{1.453979in}}{\pgfqpoint{3.428550in}{1.443380in}}{\pgfqpoint{3.428550in}{1.432330in}}%
\pgfpathcurveto{\pgfqpoint{3.428550in}{1.421280in}}{\pgfqpoint{3.432941in}{1.410681in}}{\pgfqpoint{3.440754in}{1.402867in}}%
\pgfpathcurveto{\pgfqpoint{3.448568in}{1.395053in}}{\pgfqpoint{3.459167in}{1.390663in}}{\pgfqpoint{3.470217in}{1.390663in}}%
\pgfpathclose%
\pgfusepath{stroke,fill}%
\end{pgfscope}%
\begin{pgfscope}%
\pgfpathrectangle{\pgfqpoint{0.970666in}{0.566125in}}{\pgfqpoint{5.699255in}{2.685432in}}%
\pgfusepath{clip}%
\pgfsetbuttcap%
\pgfsetroundjoin%
\definecolor{currentfill}{rgb}{0.000000,0.000000,0.000000}%
\pgfsetfillcolor{currentfill}%
\pgfsetlinewidth{1.003750pt}%
\definecolor{currentstroke}{rgb}{0.000000,0.000000,0.000000}%
\pgfsetstrokecolor{currentstroke}%
\pgfsetdash{}{0pt}%
\pgfpathmoveto{\pgfqpoint{2.658038in}{1.260112in}}%
\pgfpathcurveto{\pgfqpoint{2.669088in}{1.260112in}}{\pgfqpoint{2.679687in}{1.264503in}}{\pgfqpoint{2.687501in}{1.272316in}}%
\pgfpathcurveto{\pgfqpoint{2.695315in}{1.280130in}}{\pgfqpoint{2.699705in}{1.290729in}}{\pgfqpoint{2.699705in}{1.301779in}}%
\pgfpathcurveto{\pgfqpoint{2.699705in}{1.312829in}}{\pgfqpoint{2.695315in}{1.323428in}}{\pgfqpoint{2.687501in}{1.331242in}}%
\pgfpathcurveto{\pgfqpoint{2.679687in}{1.339055in}}{\pgfqpoint{2.669088in}{1.343446in}}{\pgfqpoint{2.658038in}{1.343446in}}%
\pgfpathcurveto{\pgfqpoint{2.646988in}{1.343446in}}{\pgfqpoint{2.636389in}{1.339055in}}{\pgfqpoint{2.628575in}{1.331242in}}%
\pgfpathcurveto{\pgfqpoint{2.620762in}{1.323428in}}{\pgfqpoint{2.616372in}{1.312829in}}{\pgfqpoint{2.616372in}{1.301779in}}%
\pgfpathcurveto{\pgfqpoint{2.616372in}{1.290729in}}{\pgfqpoint{2.620762in}{1.280130in}}{\pgfqpoint{2.628575in}{1.272316in}}%
\pgfpathcurveto{\pgfqpoint{2.636389in}{1.264503in}}{\pgfqpoint{2.646988in}{1.260112in}}{\pgfqpoint{2.658038in}{1.260112in}}%
\pgfpathclose%
\pgfusepath{stroke,fill}%
\end{pgfscope}%
\begin{pgfscope}%
\pgfpathrectangle{\pgfqpoint{0.970666in}{0.566125in}}{\pgfqpoint{5.699255in}{2.685432in}}%
\pgfusepath{clip}%
\pgfsetbuttcap%
\pgfsetroundjoin%
\definecolor{currentfill}{rgb}{0.000000,0.000000,0.000000}%
\pgfsetfillcolor{currentfill}%
\pgfsetlinewidth{1.003750pt}%
\definecolor{currentstroke}{rgb}{0.000000,0.000000,0.000000}%
\pgfsetstrokecolor{currentstroke}%
\pgfsetdash{}{0pt}%
\pgfpathmoveto{\pgfqpoint{2.630032in}{1.234002in}}%
\pgfpathcurveto{\pgfqpoint{2.641082in}{1.234002in}}{\pgfqpoint{2.651681in}{1.238392in}}{\pgfqpoint{2.659495in}{1.246206in}}%
\pgfpathcurveto{\pgfqpoint{2.667308in}{1.254020in}}{\pgfqpoint{2.671699in}{1.264619in}}{\pgfqpoint{2.671699in}{1.275669in}}%
\pgfpathcurveto{\pgfqpoint{2.671699in}{1.286719in}}{\pgfqpoint{2.667308in}{1.297318in}}{\pgfqpoint{2.659495in}{1.305132in}}%
\pgfpathcurveto{\pgfqpoint{2.651681in}{1.312945in}}{\pgfqpoint{2.641082in}{1.317335in}}{\pgfqpoint{2.630032in}{1.317335in}}%
\pgfpathcurveto{\pgfqpoint{2.618982in}{1.317335in}}{\pgfqpoint{2.608383in}{1.312945in}}{\pgfqpoint{2.600569in}{1.305132in}}%
\pgfpathcurveto{\pgfqpoint{2.592756in}{1.297318in}}{\pgfqpoint{2.588365in}{1.286719in}}{\pgfqpoint{2.588365in}{1.275669in}}%
\pgfpathcurveto{\pgfqpoint{2.588365in}{1.264619in}}{\pgfqpoint{2.592756in}{1.254020in}}{\pgfqpoint{2.600569in}{1.246206in}}%
\pgfpathcurveto{\pgfqpoint{2.608383in}{1.238392in}}{\pgfqpoint{2.618982in}{1.234002in}}{\pgfqpoint{2.630032in}{1.234002in}}%
\pgfpathclose%
\pgfusepath{stroke,fill}%
\end{pgfscope}%
\begin{pgfscope}%
\pgfpathrectangle{\pgfqpoint{0.970666in}{0.566125in}}{\pgfqpoint{5.699255in}{2.685432in}}%
\pgfusepath{clip}%
\pgfsetbuttcap%
\pgfsetroundjoin%
\definecolor{currentfill}{rgb}{0.000000,0.000000,0.000000}%
\pgfsetfillcolor{currentfill}%
\pgfsetlinewidth{1.003750pt}%
\definecolor{currentstroke}{rgb}{0.000000,0.000000,0.000000}%
\pgfsetstrokecolor{currentstroke}%
\pgfsetdash{}{0pt}%
\pgfpathmoveto{\pgfqpoint{2.377976in}{1.142616in}}%
\pgfpathcurveto{\pgfqpoint{2.389027in}{1.142616in}}{\pgfqpoint{2.399626in}{1.147007in}}{\pgfqpoint{2.407439in}{1.154820in}}%
\pgfpathcurveto{\pgfqpoint{2.415253in}{1.162634in}}{\pgfqpoint{2.419643in}{1.173233in}}{\pgfqpoint{2.419643in}{1.184283in}}%
\pgfpathcurveto{\pgfqpoint{2.419643in}{1.195333in}}{\pgfqpoint{2.415253in}{1.205932in}}{\pgfqpoint{2.407439in}{1.213746in}}%
\pgfpathcurveto{\pgfqpoint{2.399626in}{1.221560in}}{\pgfqpoint{2.389027in}{1.225950in}}{\pgfqpoint{2.377976in}{1.225950in}}%
\pgfpathcurveto{\pgfqpoint{2.366926in}{1.225950in}}{\pgfqpoint{2.356327in}{1.221560in}}{\pgfqpoint{2.348514in}{1.213746in}}%
\pgfpathcurveto{\pgfqpoint{2.340700in}{1.205932in}}{\pgfqpoint{2.336310in}{1.195333in}}{\pgfqpoint{2.336310in}{1.184283in}}%
\pgfpathcurveto{\pgfqpoint{2.336310in}{1.173233in}}{\pgfqpoint{2.340700in}{1.162634in}}{\pgfqpoint{2.348514in}{1.154820in}}%
\pgfpathcurveto{\pgfqpoint{2.356327in}{1.147007in}}{\pgfqpoint{2.366926in}{1.142616in}}{\pgfqpoint{2.377976in}{1.142616in}}%
\pgfpathclose%
\pgfusepath{stroke,fill}%
\end{pgfscope}%
\begin{pgfscope}%
\pgfpathrectangle{\pgfqpoint{0.970666in}{0.566125in}}{\pgfqpoint{5.699255in}{2.685432in}}%
\pgfusepath{clip}%
\pgfsetbuttcap%
\pgfsetroundjoin%
\definecolor{currentfill}{rgb}{0.000000,0.000000,0.000000}%
\pgfsetfillcolor{currentfill}%
\pgfsetlinewidth{1.003750pt}%
\definecolor{currentstroke}{rgb}{0.000000,0.000000,0.000000}%
\pgfsetstrokecolor{currentstroke}%
\pgfsetdash{}{0pt}%
\pgfpathmoveto{\pgfqpoint{2.405983in}{1.051231in}}%
\pgfpathcurveto{\pgfqpoint{2.417033in}{1.051231in}}{\pgfqpoint{2.427632in}{1.055621in}}{\pgfqpoint{2.435445in}{1.063435in}}%
\pgfpathcurveto{\pgfqpoint{2.443259in}{1.071248in}}{\pgfqpoint{2.447649in}{1.081847in}}{\pgfqpoint{2.447649in}{1.092898in}}%
\pgfpathcurveto{\pgfqpoint{2.447649in}{1.103948in}}{\pgfqpoint{2.443259in}{1.114547in}}{\pgfqpoint{2.435445in}{1.122360in}}%
\pgfpathcurveto{\pgfqpoint{2.427632in}{1.130174in}}{\pgfqpoint{2.417033in}{1.134564in}}{\pgfqpoint{2.405983in}{1.134564in}}%
\pgfpathcurveto{\pgfqpoint{2.394933in}{1.134564in}}{\pgfqpoint{2.384333in}{1.130174in}}{\pgfqpoint{2.376520in}{1.122360in}}%
\pgfpathcurveto{\pgfqpoint{2.368706in}{1.114547in}}{\pgfqpoint{2.364316in}{1.103948in}}{\pgfqpoint{2.364316in}{1.092898in}}%
\pgfpathcurveto{\pgfqpoint{2.364316in}{1.081847in}}{\pgfqpoint{2.368706in}{1.071248in}}{\pgfqpoint{2.376520in}{1.063435in}}%
\pgfpathcurveto{\pgfqpoint{2.384333in}{1.055621in}}{\pgfqpoint{2.394933in}{1.051231in}}{\pgfqpoint{2.405983in}{1.051231in}}%
\pgfpathclose%
\pgfusepath{stroke,fill}%
\end{pgfscope}%
\begin{pgfscope}%
\pgfpathrectangle{\pgfqpoint{0.970666in}{0.566125in}}{\pgfqpoint{5.699255in}{2.685432in}}%
\pgfusepath{clip}%
\pgfsetbuttcap%
\pgfsetroundjoin%
\definecolor{currentfill}{rgb}{0.000000,0.000000,0.000000}%
\pgfsetfillcolor{currentfill}%
\pgfsetlinewidth{1.003750pt}%
\definecolor{currentstroke}{rgb}{0.000000,0.000000,0.000000}%
\pgfsetstrokecolor{currentstroke}%
\pgfsetdash{}{0pt}%
\pgfpathmoveto{\pgfqpoint{2.405983in}{0.894570in}}%
\pgfpathcurveto{\pgfqpoint{2.417033in}{0.894570in}}{\pgfqpoint{2.427632in}{0.898960in}}{\pgfqpoint{2.435445in}{0.906774in}}%
\pgfpathcurveto{\pgfqpoint{2.443259in}{0.914587in}}{\pgfqpoint{2.447649in}{0.925186in}}{\pgfqpoint{2.447649in}{0.936236in}}%
\pgfpathcurveto{\pgfqpoint{2.447649in}{0.947287in}}{\pgfqpoint{2.443259in}{0.957886in}}{\pgfqpoint{2.435445in}{0.965699in}}%
\pgfpathcurveto{\pgfqpoint{2.427632in}{0.973513in}}{\pgfqpoint{2.417033in}{0.977903in}}{\pgfqpoint{2.405983in}{0.977903in}}%
\pgfpathcurveto{\pgfqpoint{2.394933in}{0.977903in}}{\pgfqpoint{2.384333in}{0.973513in}}{\pgfqpoint{2.376520in}{0.965699in}}%
\pgfpathcurveto{\pgfqpoint{2.368706in}{0.957886in}}{\pgfqpoint{2.364316in}{0.947287in}}{\pgfqpoint{2.364316in}{0.936236in}}%
\pgfpathcurveto{\pgfqpoint{2.364316in}{0.925186in}}{\pgfqpoint{2.368706in}{0.914587in}}{\pgfqpoint{2.376520in}{0.906774in}}%
\pgfpathcurveto{\pgfqpoint{2.384333in}{0.898960in}}{\pgfqpoint{2.394933in}{0.894570in}}{\pgfqpoint{2.405983in}{0.894570in}}%
\pgfpathclose%
\pgfusepath{stroke,fill}%
\end{pgfscope}%
\begin{pgfscope}%
\pgfpathrectangle{\pgfqpoint{0.970666in}{0.566125in}}{\pgfqpoint{5.699255in}{2.685432in}}%
\pgfusepath{clip}%
\pgfsetbuttcap%
\pgfsetroundjoin%
\definecolor{currentfill}{rgb}{0.000000,0.000000,0.000000}%
\pgfsetfillcolor{currentfill}%
\pgfsetlinewidth{1.003750pt}%
\definecolor{currentstroke}{rgb}{0.000000,0.000000,0.000000}%
\pgfsetstrokecolor{currentstroke}%
\pgfsetdash{}{0pt}%
\pgfpathmoveto{\pgfqpoint{1.929878in}{0.946790in}}%
\pgfpathcurveto{\pgfqpoint{1.940928in}{0.946790in}}{\pgfqpoint{1.951527in}{0.951180in}}{\pgfqpoint{1.959341in}{0.958994in}}%
\pgfpathcurveto{\pgfqpoint{1.967154in}{0.966808in}}{\pgfqpoint{1.971544in}{0.977407in}}{\pgfqpoint{1.971544in}{0.988457in}}%
\pgfpathcurveto{\pgfqpoint{1.971544in}{0.999507in}}{\pgfqpoint{1.967154in}{1.010106in}}{\pgfqpoint{1.959341in}{1.017920in}}%
\pgfpathcurveto{\pgfqpoint{1.951527in}{1.025733in}}{\pgfqpoint{1.940928in}{1.030123in}}{\pgfqpoint{1.929878in}{1.030123in}}%
\pgfpathcurveto{\pgfqpoint{1.918828in}{1.030123in}}{\pgfqpoint{1.908229in}{1.025733in}}{\pgfqpoint{1.900415in}{1.017920in}}%
\pgfpathcurveto{\pgfqpoint{1.892601in}{1.010106in}}{\pgfqpoint{1.888211in}{0.999507in}}{\pgfqpoint{1.888211in}{0.988457in}}%
\pgfpathcurveto{\pgfqpoint{1.888211in}{0.977407in}}{\pgfqpoint{1.892601in}{0.966808in}}{\pgfqpoint{1.900415in}{0.958994in}}%
\pgfpathcurveto{\pgfqpoint{1.908229in}{0.951180in}}{\pgfqpoint{1.918828in}{0.946790in}}{\pgfqpoint{1.929878in}{0.946790in}}%
\pgfpathclose%
\pgfusepath{stroke,fill}%
\end{pgfscope}%
\begin{pgfscope}%
\pgfpathrectangle{\pgfqpoint{0.970666in}{0.566125in}}{\pgfqpoint{5.699255in}{2.685432in}}%
\pgfusepath{clip}%
\pgfsetbuttcap%
\pgfsetroundjoin%
\definecolor{currentfill}{rgb}{0.000000,0.000000,0.000000}%
\pgfsetfillcolor{currentfill}%
\pgfsetlinewidth{1.003750pt}%
\definecolor{currentstroke}{rgb}{0.000000,0.000000,0.000000}%
\pgfsetstrokecolor{currentstroke}%
\pgfsetdash{}{0pt}%
\pgfpathmoveto{\pgfqpoint{1.593804in}{0.790129in}}%
\pgfpathcurveto{\pgfqpoint{1.604854in}{0.790129in}}{\pgfqpoint{1.615453in}{0.794519in}}{\pgfqpoint{1.623267in}{0.802333in}}%
\pgfpathcurveto{\pgfqpoint{1.631080in}{0.810147in}}{\pgfqpoint{1.635470in}{0.820746in}}{\pgfqpoint{1.635470in}{0.831796in}}%
\pgfpathcurveto{\pgfqpoint{1.635470in}{0.842846in}}{\pgfqpoint{1.631080in}{0.853445in}}{\pgfqpoint{1.623267in}{0.861258in}}%
\pgfpathcurveto{\pgfqpoint{1.615453in}{0.869072in}}{\pgfqpoint{1.604854in}{0.873462in}}{\pgfqpoint{1.593804in}{0.873462in}}%
\pgfpathcurveto{\pgfqpoint{1.582754in}{0.873462in}}{\pgfqpoint{1.572155in}{0.869072in}}{\pgfqpoint{1.564341in}{0.861258in}}%
\pgfpathcurveto{\pgfqpoint{1.556527in}{0.853445in}}{\pgfqpoint{1.552137in}{0.842846in}}{\pgfqpoint{1.552137in}{0.831796in}}%
\pgfpathcurveto{\pgfqpoint{1.552137in}{0.820746in}}{\pgfqpoint{1.556527in}{0.810147in}}{\pgfqpoint{1.564341in}{0.802333in}}%
\pgfpathcurveto{\pgfqpoint{1.572155in}{0.794519in}}{\pgfqpoint{1.582754in}{0.790129in}}{\pgfqpoint{1.593804in}{0.790129in}}%
\pgfpathclose%
\pgfusepath{stroke,fill}%
\end{pgfscope}%
\begin{pgfscope}%
\pgfpathrectangle{\pgfqpoint{0.970666in}{0.566125in}}{\pgfqpoint{5.699255in}{2.685432in}}%
\pgfusepath{clip}%
\pgfsetbuttcap%
\pgfsetroundjoin%
\definecolor{currentfill}{rgb}{0.000000,0.000000,0.000000}%
\pgfsetfillcolor{currentfill}%
\pgfsetlinewidth{1.003750pt}%
\definecolor{currentstroke}{rgb}{0.000000,0.000000,0.000000}%
\pgfsetstrokecolor{currentstroke}%
\pgfsetdash{}{0pt}%
\pgfpathmoveto{\pgfqpoint{1.733835in}{0.646523in}}%
\pgfpathcurveto{\pgfqpoint{1.744885in}{0.646523in}}{\pgfqpoint{1.755484in}{0.650913in}}{\pgfqpoint{1.763297in}{0.658727in}}%
\pgfpathcurveto{\pgfqpoint{1.771111in}{0.666541in}}{\pgfqpoint{1.775501in}{0.677140in}}{\pgfqpoint{1.775501in}{0.688190in}}%
\pgfpathcurveto{\pgfqpoint{1.775501in}{0.699240in}}{\pgfqpoint{1.771111in}{0.709839in}}{\pgfqpoint{1.763297in}{0.717652in}}%
\pgfpathcurveto{\pgfqpoint{1.755484in}{0.725466in}}{\pgfqpoint{1.744885in}{0.729856in}}{\pgfqpoint{1.733835in}{0.729856in}}%
\pgfpathcurveto{\pgfqpoint{1.722784in}{0.729856in}}{\pgfqpoint{1.712185in}{0.725466in}}{\pgfqpoint{1.704372in}{0.717652in}}%
\pgfpathcurveto{\pgfqpoint{1.696558in}{0.709839in}}{\pgfqpoint{1.692168in}{0.699240in}}{\pgfqpoint{1.692168in}{0.688190in}}%
\pgfpathcurveto{\pgfqpoint{1.692168in}{0.677140in}}{\pgfqpoint{1.696558in}{0.666541in}}{\pgfqpoint{1.704372in}{0.658727in}}%
\pgfpathcurveto{\pgfqpoint{1.712185in}{0.650913in}}{\pgfqpoint{1.722784in}{0.646523in}}{\pgfqpoint{1.733835in}{0.646523in}}%
\pgfpathclose%
\pgfusepath{stroke,fill}%
\end{pgfscope}%
\begin{pgfscope}%
\pgfpathrectangle{\pgfqpoint{0.970666in}{0.566125in}}{\pgfqpoint{5.699255in}{2.685432in}}%
\pgfusepath{clip}%
\pgfsetbuttcap%
\pgfsetroundjoin%
\definecolor{currentfill}{rgb}{0.000000,0.000000,0.000000}%
\pgfsetfillcolor{currentfill}%
\pgfsetlinewidth{1.003750pt}%
\definecolor{currentstroke}{rgb}{0.000000,0.000000,0.000000}%
\pgfsetstrokecolor{currentstroke}%
\pgfsetdash{}{0pt}%
\pgfpathmoveto{\pgfqpoint{1.845859in}{0.672633in}}%
\pgfpathcurveto{\pgfqpoint{1.856909in}{0.672633in}}{\pgfqpoint{1.867508in}{0.677023in}}{\pgfqpoint{1.875322in}{0.684837in}}%
\pgfpathcurveto{\pgfqpoint{1.883136in}{0.692651in}}{\pgfqpoint{1.887526in}{0.703250in}}{\pgfqpoint{1.887526in}{0.714300in}}%
\pgfpathcurveto{\pgfqpoint{1.887526in}{0.725350in}}{\pgfqpoint{1.883136in}{0.735949in}}{\pgfqpoint{1.875322in}{0.743763in}}%
\pgfpathcurveto{\pgfqpoint{1.867508in}{0.751576in}}{\pgfqpoint{1.856909in}{0.755967in}}{\pgfqpoint{1.845859in}{0.755967in}}%
\pgfpathcurveto{\pgfqpoint{1.834809in}{0.755967in}}{\pgfqpoint{1.824210in}{0.751576in}}{\pgfqpoint{1.816396in}{0.743763in}}%
\pgfpathcurveto{\pgfqpoint{1.808583in}{0.735949in}}{\pgfqpoint{1.804193in}{0.725350in}}{\pgfqpoint{1.804193in}{0.714300in}}%
\pgfpathcurveto{\pgfqpoint{1.804193in}{0.703250in}}{\pgfqpoint{1.808583in}{0.692651in}}{\pgfqpoint{1.816396in}{0.684837in}}%
\pgfpathcurveto{\pgfqpoint{1.824210in}{0.677023in}}{\pgfqpoint{1.834809in}{0.672633in}}{\pgfqpoint{1.845859in}{0.672633in}}%
\pgfpathclose%
\pgfusepath{stroke,fill}%
\end{pgfscope}%
\begin{pgfscope}%
\pgfpathrectangle{\pgfqpoint{0.970666in}{0.566125in}}{\pgfqpoint{5.699255in}{2.685432in}}%
\pgfusepath{clip}%
\pgfsetbuttcap%
\pgfsetroundjoin%
\definecolor{currentfill}{rgb}{0.000000,0.000000,0.000000}%
\pgfsetfillcolor{currentfill}%
\pgfsetlinewidth{1.003750pt}%
\definecolor{currentstroke}{rgb}{0.000000,0.000000,0.000000}%
\pgfsetstrokecolor{currentstroke}%
\pgfsetdash{}{0pt}%
\pgfpathmoveto{\pgfqpoint{1.621810in}{1.142616in}}%
\pgfpathcurveto{\pgfqpoint{1.632860in}{1.142616in}}{\pgfqpoint{1.643459in}{1.147007in}}{\pgfqpoint{1.651273in}{1.154820in}}%
\pgfpathcurveto{\pgfqpoint{1.659086in}{1.162634in}}{\pgfqpoint{1.663477in}{1.173233in}}{\pgfqpoint{1.663477in}{1.184283in}}%
\pgfpathcurveto{\pgfqpoint{1.663477in}{1.195333in}}{\pgfqpoint{1.659086in}{1.205932in}}{\pgfqpoint{1.651273in}{1.213746in}}%
\pgfpathcurveto{\pgfqpoint{1.643459in}{1.221560in}}{\pgfqpoint{1.632860in}{1.225950in}}{\pgfqpoint{1.621810in}{1.225950in}}%
\pgfpathcurveto{\pgfqpoint{1.610760in}{1.225950in}}{\pgfqpoint{1.600161in}{1.221560in}}{\pgfqpoint{1.592347in}{1.213746in}}%
\pgfpathcurveto{\pgfqpoint{1.584534in}{1.205932in}}{\pgfqpoint{1.580143in}{1.195333in}}{\pgfqpoint{1.580143in}{1.184283in}}%
\pgfpathcurveto{\pgfqpoint{1.580143in}{1.173233in}}{\pgfqpoint{1.584534in}{1.162634in}}{\pgfqpoint{1.592347in}{1.154820in}}%
\pgfpathcurveto{\pgfqpoint{1.600161in}{1.147007in}}{\pgfqpoint{1.610760in}{1.142616in}}{\pgfqpoint{1.621810in}{1.142616in}}%
\pgfpathclose%
\pgfusepath{stroke,fill}%
\end{pgfscope}%
\begin{pgfscope}%
\pgfpathrectangle{\pgfqpoint{0.970666in}{0.566125in}}{\pgfqpoint{5.699255in}{2.685432in}}%
\pgfusepath{clip}%
\pgfsetbuttcap%
\pgfsetroundjoin%
\definecolor{currentfill}{rgb}{0.000000,0.000000,0.000000}%
\pgfsetfillcolor{currentfill}%
\pgfsetlinewidth{1.003750pt}%
\definecolor{currentstroke}{rgb}{0.000000,0.000000,0.000000}%
\pgfsetstrokecolor{currentstroke}%
\pgfsetdash{}{0pt}%
\pgfpathmoveto{\pgfqpoint{1.901872in}{1.234002in}}%
\pgfpathcurveto{\pgfqpoint{1.912922in}{1.234002in}}{\pgfqpoint{1.923521in}{1.238392in}}{\pgfqpoint{1.931334in}{1.246206in}}%
\pgfpathcurveto{\pgfqpoint{1.939148in}{1.254020in}}{\pgfqpoint{1.943538in}{1.264619in}}{\pgfqpoint{1.943538in}{1.275669in}}%
\pgfpathcurveto{\pgfqpoint{1.943538in}{1.286719in}}{\pgfqpoint{1.939148in}{1.297318in}}{\pgfqpoint{1.931334in}{1.305132in}}%
\pgfpathcurveto{\pgfqpoint{1.923521in}{1.312945in}}{\pgfqpoint{1.912922in}{1.317335in}}{\pgfqpoint{1.901872in}{1.317335in}}%
\pgfpathcurveto{\pgfqpoint{1.890821in}{1.317335in}}{\pgfqpoint{1.880222in}{1.312945in}}{\pgfqpoint{1.872409in}{1.305132in}}%
\pgfpathcurveto{\pgfqpoint{1.864595in}{1.297318in}}{\pgfqpoint{1.860205in}{1.286719in}}{\pgfqpoint{1.860205in}{1.275669in}}%
\pgfpathcurveto{\pgfqpoint{1.860205in}{1.264619in}}{\pgfqpoint{1.864595in}{1.254020in}}{\pgfqpoint{1.872409in}{1.246206in}}%
\pgfpathcurveto{\pgfqpoint{1.880222in}{1.238392in}}{\pgfqpoint{1.890821in}{1.234002in}}{\pgfqpoint{1.901872in}{1.234002in}}%
\pgfpathclose%
\pgfusepath{stroke,fill}%
\end{pgfscope}%
\begin{pgfscope}%
\pgfpathrectangle{\pgfqpoint{0.970666in}{0.566125in}}{\pgfqpoint{5.699255in}{2.685432in}}%
\pgfusepath{clip}%
\pgfsetbuttcap%
\pgfsetroundjoin%
\definecolor{currentfill}{rgb}{0.000000,0.000000,0.000000}%
\pgfsetfillcolor{currentfill}%
\pgfsetlinewidth{1.003750pt}%
\definecolor{currentstroke}{rgb}{0.000000,0.000000,0.000000}%
\pgfsetstrokecolor{currentstroke}%
\pgfsetdash{}{0pt}%
\pgfpathmoveto{\pgfqpoint{1.873865in}{1.351498in}}%
\pgfpathcurveto{\pgfqpoint{1.884916in}{1.351498in}}{\pgfqpoint{1.895515in}{1.355888in}}{\pgfqpoint{1.903328in}{1.363702in}}%
\pgfpathcurveto{\pgfqpoint{1.911142in}{1.371515in}}{\pgfqpoint{1.915532in}{1.382114in}}{\pgfqpoint{1.915532in}{1.393165in}}%
\pgfpathcurveto{\pgfqpoint{1.915532in}{1.404215in}}{\pgfqpoint{1.911142in}{1.414814in}}{\pgfqpoint{1.903328in}{1.422627in}}%
\pgfpathcurveto{\pgfqpoint{1.895515in}{1.430441in}}{\pgfqpoint{1.884916in}{1.434831in}}{\pgfqpoint{1.873865in}{1.434831in}}%
\pgfpathcurveto{\pgfqpoint{1.862815in}{1.434831in}}{\pgfqpoint{1.852216in}{1.430441in}}{\pgfqpoint{1.844403in}{1.422627in}}%
\pgfpathcurveto{\pgfqpoint{1.836589in}{1.414814in}}{\pgfqpoint{1.832199in}{1.404215in}}{\pgfqpoint{1.832199in}{1.393165in}}%
\pgfpathcurveto{\pgfqpoint{1.832199in}{1.382114in}}{\pgfqpoint{1.836589in}{1.371515in}}{\pgfqpoint{1.844403in}{1.363702in}}%
\pgfpathcurveto{\pgfqpoint{1.852216in}{1.355888in}}{\pgfqpoint{1.862815in}{1.351498in}}{\pgfqpoint{1.873865in}{1.351498in}}%
\pgfpathclose%
\pgfusepath{stroke,fill}%
\end{pgfscope}%
\begin{pgfscope}%
\pgfpathrectangle{\pgfqpoint{0.970666in}{0.566125in}}{\pgfqpoint{5.699255in}{2.685432in}}%
\pgfusepath{clip}%
\pgfsetbuttcap%
\pgfsetroundjoin%
\definecolor{currentfill}{rgb}{0.000000,0.000000,0.000000}%
\pgfsetfillcolor{currentfill}%
\pgfsetlinewidth{1.003750pt}%
\definecolor{currentstroke}{rgb}{0.000000,0.000000,0.000000}%
\pgfsetstrokecolor{currentstroke}%
\pgfsetdash{}{0pt}%
\pgfpathmoveto{\pgfqpoint{1.901872in}{1.455939in}}%
\pgfpathcurveto{\pgfqpoint{1.912922in}{1.455939in}}{\pgfqpoint{1.923521in}{1.460329in}}{\pgfqpoint{1.931334in}{1.468143in}}%
\pgfpathcurveto{\pgfqpoint{1.939148in}{1.475956in}}{\pgfqpoint{1.943538in}{1.486555in}}{\pgfqpoint{1.943538in}{1.497605in}}%
\pgfpathcurveto{\pgfqpoint{1.943538in}{1.508655in}}{\pgfqpoint{1.939148in}{1.519254in}}{\pgfqpoint{1.931334in}{1.527068in}}%
\pgfpathcurveto{\pgfqpoint{1.923521in}{1.534882in}}{\pgfqpoint{1.912922in}{1.539272in}}{\pgfqpoint{1.901872in}{1.539272in}}%
\pgfpathcurveto{\pgfqpoint{1.890821in}{1.539272in}}{\pgfqpoint{1.880222in}{1.534882in}}{\pgfqpoint{1.872409in}{1.527068in}}%
\pgfpathcurveto{\pgfqpoint{1.864595in}{1.519254in}}{\pgfqpoint{1.860205in}{1.508655in}}{\pgfqpoint{1.860205in}{1.497605in}}%
\pgfpathcurveto{\pgfqpoint{1.860205in}{1.486555in}}{\pgfqpoint{1.864595in}{1.475956in}}{\pgfqpoint{1.872409in}{1.468143in}}%
\pgfpathcurveto{\pgfqpoint{1.880222in}{1.460329in}}{\pgfqpoint{1.890821in}{1.455939in}}{\pgfqpoint{1.901872in}{1.455939in}}%
\pgfpathclose%
\pgfusepath{stroke,fill}%
\end{pgfscope}%
\begin{pgfscope}%
\pgfpathrectangle{\pgfqpoint{0.970666in}{0.566125in}}{\pgfqpoint{5.699255in}{2.685432in}}%
\pgfusepath{clip}%
\pgfsetbuttcap%
\pgfsetroundjoin%
\definecolor{currentfill}{rgb}{0.000000,0.000000,0.000000}%
\pgfsetfillcolor{currentfill}%
\pgfsetlinewidth{1.003750pt}%
\definecolor{currentstroke}{rgb}{0.000000,0.000000,0.000000}%
\pgfsetstrokecolor{currentstroke}%
\pgfsetdash{}{0pt}%
\pgfpathmoveto{\pgfqpoint{1.873865in}{1.612600in}}%
\pgfpathcurveto{\pgfqpoint{1.884916in}{1.612600in}}{\pgfqpoint{1.895515in}{1.616990in}}{\pgfqpoint{1.903328in}{1.624804in}}%
\pgfpathcurveto{\pgfqpoint{1.911142in}{1.632617in}}{\pgfqpoint{1.915532in}{1.643216in}}{\pgfqpoint{1.915532in}{1.654266in}}%
\pgfpathcurveto{\pgfqpoint{1.915532in}{1.665317in}}{\pgfqpoint{1.911142in}{1.675916in}}{\pgfqpoint{1.903328in}{1.683729in}}%
\pgfpathcurveto{\pgfqpoint{1.895515in}{1.691543in}}{\pgfqpoint{1.884916in}{1.695933in}}{\pgfqpoint{1.873865in}{1.695933in}}%
\pgfpathcurveto{\pgfqpoint{1.862815in}{1.695933in}}{\pgfqpoint{1.852216in}{1.691543in}}{\pgfqpoint{1.844403in}{1.683729in}}%
\pgfpathcurveto{\pgfqpoint{1.836589in}{1.675916in}}{\pgfqpoint{1.832199in}{1.665317in}}{\pgfqpoint{1.832199in}{1.654266in}}%
\pgfpathcurveto{\pgfqpoint{1.832199in}{1.643216in}}{\pgfqpoint{1.836589in}{1.632617in}}{\pgfqpoint{1.844403in}{1.624804in}}%
\pgfpathcurveto{\pgfqpoint{1.852216in}{1.616990in}}{\pgfqpoint{1.862815in}{1.612600in}}{\pgfqpoint{1.873865in}{1.612600in}}%
\pgfpathclose%
\pgfusepath{stroke,fill}%
\end{pgfscope}%
\begin{pgfscope}%
\pgfpathrectangle{\pgfqpoint{0.970666in}{0.566125in}}{\pgfqpoint{5.699255in}{2.685432in}}%
\pgfusepath{clip}%
\pgfsetbuttcap%
\pgfsetroundjoin%
\definecolor{currentfill}{rgb}{0.000000,0.000000,0.000000}%
\pgfsetfillcolor{currentfill}%
\pgfsetlinewidth{1.003750pt}%
\definecolor{currentstroke}{rgb}{0.000000,0.000000,0.000000}%
\pgfsetstrokecolor{currentstroke}%
\pgfsetdash{}{0pt}%
\pgfpathmoveto{\pgfqpoint{1.817853in}{1.808426in}}%
\pgfpathcurveto{\pgfqpoint{1.828903in}{1.808426in}}{\pgfqpoint{1.839502in}{1.812816in}}{\pgfqpoint{1.847316in}{1.820630in}}%
\pgfpathcurveto{\pgfqpoint{1.855130in}{1.828444in}}{\pgfqpoint{1.859520in}{1.839043in}}{\pgfqpoint{1.859520in}{1.850093in}}%
\pgfpathcurveto{\pgfqpoint{1.859520in}{1.861143in}}{\pgfqpoint{1.855130in}{1.871742in}}{\pgfqpoint{1.847316in}{1.879556in}}%
\pgfpathcurveto{\pgfqpoint{1.839502in}{1.887369in}}{\pgfqpoint{1.828903in}{1.891759in}}{\pgfqpoint{1.817853in}{1.891759in}}%
\pgfpathcurveto{\pgfqpoint{1.806803in}{1.891759in}}{\pgfqpoint{1.796204in}{1.887369in}}{\pgfqpoint{1.788390in}{1.879556in}}%
\pgfpathcurveto{\pgfqpoint{1.780577in}{1.871742in}}{\pgfqpoint{1.776186in}{1.861143in}}{\pgfqpoint{1.776186in}{1.850093in}}%
\pgfpathcurveto{\pgfqpoint{1.776186in}{1.839043in}}{\pgfqpoint{1.780577in}{1.828444in}}{\pgfqpoint{1.788390in}{1.820630in}}%
\pgfpathcurveto{\pgfqpoint{1.796204in}{1.812816in}}{\pgfqpoint{1.806803in}{1.808426in}}{\pgfqpoint{1.817853in}{1.808426in}}%
\pgfpathclose%
\pgfusepath{stroke,fill}%
\end{pgfscope}%
\begin{pgfscope}%
\pgfpathrectangle{\pgfqpoint{0.970666in}{0.566125in}}{\pgfqpoint{5.699255in}{2.685432in}}%
\pgfusepath{clip}%
\pgfsetbuttcap%
\pgfsetroundjoin%
\definecolor{currentfill}{rgb}{0.000000,0.000000,0.000000}%
\pgfsetfillcolor{currentfill}%
\pgfsetlinewidth{1.003750pt}%
\definecolor{currentstroke}{rgb}{0.000000,0.000000,0.000000}%
\pgfsetstrokecolor{currentstroke}%
\pgfsetdash{}{0pt}%
\pgfpathmoveto{\pgfqpoint{2.405983in}{1.991197in}}%
\pgfpathcurveto{\pgfqpoint{2.417033in}{1.991197in}}{\pgfqpoint{2.427632in}{1.995588in}}{\pgfqpoint{2.435445in}{2.003401in}}%
\pgfpathcurveto{\pgfqpoint{2.443259in}{2.011215in}}{\pgfqpoint{2.447649in}{2.021814in}}{\pgfqpoint{2.447649in}{2.032864in}}%
\pgfpathcurveto{\pgfqpoint{2.447649in}{2.043914in}}{\pgfqpoint{2.443259in}{2.054513in}}{\pgfqpoint{2.435445in}{2.062327in}}%
\pgfpathcurveto{\pgfqpoint{2.427632in}{2.070140in}}{\pgfqpoint{2.417033in}{2.074531in}}{\pgfqpoint{2.405983in}{2.074531in}}%
\pgfpathcurveto{\pgfqpoint{2.394933in}{2.074531in}}{\pgfqpoint{2.384333in}{2.070140in}}{\pgfqpoint{2.376520in}{2.062327in}}%
\pgfpathcurveto{\pgfqpoint{2.368706in}{2.054513in}}{\pgfqpoint{2.364316in}{2.043914in}}{\pgfqpoint{2.364316in}{2.032864in}}%
\pgfpathcurveto{\pgfqpoint{2.364316in}{2.021814in}}{\pgfqpoint{2.368706in}{2.011215in}}{\pgfqpoint{2.376520in}{2.003401in}}%
\pgfpathcurveto{\pgfqpoint{2.384333in}{1.995588in}}{\pgfqpoint{2.394933in}{1.991197in}}{\pgfqpoint{2.405983in}{1.991197in}}%
\pgfpathclose%
\pgfusepath{stroke,fill}%
\end{pgfscope}%
\begin{pgfscope}%
\pgfpathrectangle{\pgfqpoint{0.970666in}{0.566125in}}{\pgfqpoint{5.699255in}{2.685432in}}%
\pgfusepath{clip}%
\pgfsetbuttcap%
\pgfsetroundjoin%
\definecolor{currentfill}{rgb}{0.000000,0.000000,0.000000}%
\pgfsetfillcolor{currentfill}%
\pgfsetlinewidth{1.003750pt}%
\definecolor{currentstroke}{rgb}{0.000000,0.000000,0.000000}%
\pgfsetstrokecolor{currentstroke}%
\pgfsetdash{}{0pt}%
\pgfpathmoveto{\pgfqpoint{2.602026in}{1.965087in}}%
\pgfpathcurveto{\pgfqpoint{2.613076in}{1.965087in}}{\pgfqpoint{2.623675in}{1.969477in}}{\pgfqpoint{2.631489in}{1.977291in}}%
\pgfpathcurveto{\pgfqpoint{2.639302in}{1.985105in}}{\pgfqpoint{2.643692in}{1.995704in}}{\pgfqpoint{2.643692in}{2.006754in}}%
\pgfpathcurveto{\pgfqpoint{2.643692in}{2.017804in}}{\pgfqpoint{2.639302in}{2.028403in}}{\pgfqpoint{2.631489in}{2.036217in}}%
\pgfpathcurveto{\pgfqpoint{2.623675in}{2.044030in}}{\pgfqpoint{2.613076in}{2.048421in}}{\pgfqpoint{2.602026in}{2.048421in}}%
\pgfpathcurveto{\pgfqpoint{2.590976in}{2.048421in}}{\pgfqpoint{2.580377in}{2.044030in}}{\pgfqpoint{2.572563in}{2.036217in}}%
\pgfpathcurveto{\pgfqpoint{2.564749in}{2.028403in}}{\pgfqpoint{2.560359in}{2.017804in}}{\pgfqpoint{2.560359in}{2.006754in}}%
\pgfpathcurveto{\pgfqpoint{2.560359in}{1.995704in}}{\pgfqpoint{2.564749in}{1.985105in}}{\pgfqpoint{2.572563in}{1.977291in}}%
\pgfpathcurveto{\pgfqpoint{2.580377in}{1.969477in}}{\pgfqpoint{2.590976in}{1.965087in}}{\pgfqpoint{2.602026in}{1.965087in}}%
\pgfpathclose%
\pgfusepath{stroke,fill}%
\end{pgfscope}%
\begin{pgfscope}%
\pgfpathrectangle{\pgfqpoint{0.970666in}{0.566125in}}{\pgfqpoint{5.699255in}{2.685432in}}%
\pgfusepath{clip}%
\pgfsetbuttcap%
\pgfsetroundjoin%
\definecolor{currentfill}{rgb}{0.000000,0.000000,0.000000}%
\pgfsetfillcolor{currentfill}%
\pgfsetlinewidth{1.003750pt}%
\definecolor{currentstroke}{rgb}{0.000000,0.000000,0.000000}%
\pgfsetstrokecolor{currentstroke}%
\pgfsetdash{}{0pt}%
\pgfpathmoveto{\pgfqpoint{2.686044in}{2.043418in}}%
\pgfpathcurveto{\pgfqpoint{2.697094in}{2.043418in}}{\pgfqpoint{2.707694in}{2.047808in}}{\pgfqpoint{2.715507in}{2.055622in}}%
\pgfpathcurveto{\pgfqpoint{2.723321in}{2.063435in}}{\pgfqpoint{2.727711in}{2.074034in}}{\pgfqpoint{2.727711in}{2.085084in}}%
\pgfpathcurveto{\pgfqpoint{2.727711in}{2.096135in}}{\pgfqpoint{2.723321in}{2.106734in}}{\pgfqpoint{2.715507in}{2.114547in}}%
\pgfpathcurveto{\pgfqpoint{2.707694in}{2.122361in}}{\pgfqpoint{2.697094in}{2.126751in}}{\pgfqpoint{2.686044in}{2.126751in}}%
\pgfpathcurveto{\pgfqpoint{2.674994in}{2.126751in}}{\pgfqpoint{2.664395in}{2.122361in}}{\pgfqpoint{2.656582in}{2.114547in}}%
\pgfpathcurveto{\pgfqpoint{2.648768in}{2.106734in}}{\pgfqpoint{2.644378in}{2.096135in}}{\pgfqpoint{2.644378in}{2.085084in}}%
\pgfpathcurveto{\pgfqpoint{2.644378in}{2.074034in}}{\pgfqpoint{2.648768in}{2.063435in}}{\pgfqpoint{2.656582in}{2.055622in}}%
\pgfpathcurveto{\pgfqpoint{2.664395in}{2.047808in}}{\pgfqpoint{2.674994in}{2.043418in}}{\pgfqpoint{2.686044in}{2.043418in}}%
\pgfpathclose%
\pgfusepath{stroke,fill}%
\end{pgfscope}%
\begin{pgfscope}%
\pgfpathrectangle{\pgfqpoint{0.970666in}{0.566125in}}{\pgfqpoint{5.699255in}{2.685432in}}%
\pgfusepath{clip}%
\pgfsetbuttcap%
\pgfsetroundjoin%
\definecolor{currentfill}{rgb}{0.000000,0.000000,0.000000}%
\pgfsetfillcolor{currentfill}%
\pgfsetlinewidth{1.003750pt}%
\definecolor{currentstroke}{rgb}{0.000000,0.000000,0.000000}%
\pgfsetstrokecolor{currentstroke}%
\pgfsetdash{}{0pt}%
\pgfpathmoveto{\pgfqpoint{2.714051in}{2.069528in}}%
\pgfpathcurveto{\pgfqpoint{2.725101in}{2.069528in}}{\pgfqpoint{2.735700in}{2.073918in}}{\pgfqpoint{2.743513in}{2.081732in}}%
\pgfpathcurveto{\pgfqpoint{2.751327in}{2.089545in}}{\pgfqpoint{2.755717in}{2.100144in}}{\pgfqpoint{2.755717in}{2.111195in}}%
\pgfpathcurveto{\pgfqpoint{2.755717in}{2.122245in}}{\pgfqpoint{2.751327in}{2.132844in}}{\pgfqpoint{2.743513in}{2.140657in}}%
\pgfpathcurveto{\pgfqpoint{2.735700in}{2.148471in}}{\pgfqpoint{2.725101in}{2.152861in}}{\pgfqpoint{2.714051in}{2.152861in}}%
\pgfpathcurveto{\pgfqpoint{2.703000in}{2.152861in}}{\pgfqpoint{2.692401in}{2.148471in}}{\pgfqpoint{2.684588in}{2.140657in}}%
\pgfpathcurveto{\pgfqpoint{2.676774in}{2.132844in}}{\pgfqpoint{2.672384in}{2.122245in}}{\pgfqpoint{2.672384in}{2.111195in}}%
\pgfpathcurveto{\pgfqpoint{2.672384in}{2.100144in}}{\pgfqpoint{2.676774in}{2.089545in}}{\pgfqpoint{2.684588in}{2.081732in}}%
\pgfpathcurveto{\pgfqpoint{2.692401in}{2.073918in}}{\pgfqpoint{2.703000in}{2.069528in}}{\pgfqpoint{2.714051in}{2.069528in}}%
\pgfpathclose%
\pgfusepath{stroke,fill}%
\end{pgfscope}%
\begin{pgfscope}%
\pgfpathrectangle{\pgfqpoint{0.970666in}{0.566125in}}{\pgfqpoint{5.699255in}{2.685432in}}%
\pgfusepath{clip}%
\pgfsetbuttcap%
\pgfsetroundjoin%
\definecolor{currentfill}{rgb}{0.000000,0.000000,0.000000}%
\pgfsetfillcolor{currentfill}%
\pgfsetlinewidth{1.003750pt}%
\definecolor{currentstroke}{rgb}{0.000000,0.000000,0.000000}%
\pgfsetstrokecolor{currentstroke}%
\pgfsetdash{}{0pt}%
\pgfpathmoveto{\pgfqpoint{2.910094in}{2.213134in}}%
\pgfpathcurveto{\pgfqpoint{2.921144in}{2.213134in}}{\pgfqpoint{2.931743in}{2.217524in}}{\pgfqpoint{2.939556in}{2.225338in}}%
\pgfpathcurveto{\pgfqpoint{2.947370in}{2.233151in}}{\pgfqpoint{2.951760in}{2.243750in}}{\pgfqpoint{2.951760in}{2.254801in}}%
\pgfpathcurveto{\pgfqpoint{2.951760in}{2.265851in}}{\pgfqpoint{2.947370in}{2.276450in}}{\pgfqpoint{2.939556in}{2.284263in}}%
\pgfpathcurveto{\pgfqpoint{2.931743in}{2.292077in}}{\pgfqpoint{2.921144in}{2.296467in}}{\pgfqpoint{2.910094in}{2.296467in}}%
\pgfpathcurveto{\pgfqpoint{2.899044in}{2.296467in}}{\pgfqpoint{2.888445in}{2.292077in}}{\pgfqpoint{2.880631in}{2.284263in}}%
\pgfpathcurveto{\pgfqpoint{2.872817in}{2.276450in}}{\pgfqpoint{2.868427in}{2.265851in}}{\pgfqpoint{2.868427in}{2.254801in}}%
\pgfpathcurveto{\pgfqpoint{2.868427in}{2.243750in}}{\pgfqpoint{2.872817in}{2.233151in}}{\pgfqpoint{2.880631in}{2.225338in}}%
\pgfpathcurveto{\pgfqpoint{2.888445in}{2.217524in}}{\pgfqpoint{2.899044in}{2.213134in}}{\pgfqpoint{2.910094in}{2.213134in}}%
\pgfpathclose%
\pgfusepath{stroke,fill}%
\end{pgfscope}%
\begin{pgfscope}%
\pgfpathrectangle{\pgfqpoint{0.970666in}{0.566125in}}{\pgfqpoint{5.699255in}{2.685432in}}%
\pgfusepath{clip}%
\pgfsetbuttcap%
\pgfsetroundjoin%
\definecolor{currentfill}{rgb}{0.000000,0.000000,0.000000}%
\pgfsetfillcolor{currentfill}%
\pgfsetlinewidth{1.003750pt}%
\definecolor{currentstroke}{rgb}{0.000000,0.000000,0.000000}%
\pgfsetstrokecolor{currentstroke}%
\pgfsetdash{}{0pt}%
\pgfpathmoveto{\pgfqpoint{2.910094in}{2.343685in}}%
\pgfpathcurveto{\pgfqpoint{2.921144in}{2.343685in}}{\pgfqpoint{2.931743in}{2.348075in}}{\pgfqpoint{2.939556in}{2.355889in}}%
\pgfpathcurveto{\pgfqpoint{2.947370in}{2.363702in}}{\pgfqpoint{2.951760in}{2.374301in}}{\pgfqpoint{2.951760in}{2.385351in}}%
\pgfpathcurveto{\pgfqpoint{2.951760in}{2.396402in}}{\pgfqpoint{2.947370in}{2.407001in}}{\pgfqpoint{2.939556in}{2.414814in}}%
\pgfpathcurveto{\pgfqpoint{2.931743in}{2.422628in}}{\pgfqpoint{2.921144in}{2.427018in}}{\pgfqpoint{2.910094in}{2.427018in}}%
\pgfpathcurveto{\pgfqpoint{2.899044in}{2.427018in}}{\pgfqpoint{2.888445in}{2.422628in}}{\pgfqpoint{2.880631in}{2.414814in}}%
\pgfpathcurveto{\pgfqpoint{2.872817in}{2.407001in}}{\pgfqpoint{2.868427in}{2.396402in}}{\pgfqpoint{2.868427in}{2.385351in}}%
\pgfpathcurveto{\pgfqpoint{2.868427in}{2.374301in}}{\pgfqpoint{2.872817in}{2.363702in}}{\pgfqpoint{2.880631in}{2.355889in}}%
\pgfpathcurveto{\pgfqpoint{2.888445in}{2.348075in}}{\pgfqpoint{2.899044in}{2.343685in}}{\pgfqpoint{2.910094in}{2.343685in}}%
\pgfpathclose%
\pgfusepath{stroke,fill}%
\end{pgfscope}%
\begin{pgfscope}%
\pgfpathrectangle{\pgfqpoint{0.970666in}{0.566125in}}{\pgfqpoint{5.699255in}{2.685432in}}%
\pgfusepath{clip}%
\pgfsetbuttcap%
\pgfsetroundjoin%
\definecolor{currentfill}{rgb}{0.000000,0.000000,0.000000}%
\pgfsetfillcolor{currentfill}%
\pgfsetlinewidth{1.003750pt}%
\definecolor{currentstroke}{rgb}{0.000000,0.000000,0.000000}%
\pgfsetstrokecolor{currentstroke}%
\pgfsetdash{}{0pt}%
\pgfpathmoveto{\pgfqpoint{3.414205in}{2.343685in}}%
\pgfpathcurveto{\pgfqpoint{3.425255in}{2.343685in}}{\pgfqpoint{3.435854in}{2.348075in}}{\pgfqpoint{3.443668in}{2.355889in}}%
\pgfpathcurveto{\pgfqpoint{3.451481in}{2.363702in}}{\pgfqpoint{3.455871in}{2.374301in}}{\pgfqpoint{3.455871in}{2.385351in}}%
\pgfpathcurveto{\pgfqpoint{3.455871in}{2.396402in}}{\pgfqpoint{3.451481in}{2.407001in}}{\pgfqpoint{3.443668in}{2.414814in}}%
\pgfpathcurveto{\pgfqpoint{3.435854in}{2.422628in}}{\pgfqpoint{3.425255in}{2.427018in}}{\pgfqpoint{3.414205in}{2.427018in}}%
\pgfpathcurveto{\pgfqpoint{3.403155in}{2.427018in}}{\pgfqpoint{3.392556in}{2.422628in}}{\pgfqpoint{3.384742in}{2.414814in}}%
\pgfpathcurveto{\pgfqpoint{3.376928in}{2.407001in}}{\pgfqpoint{3.372538in}{2.396402in}}{\pgfqpoint{3.372538in}{2.385351in}}%
\pgfpathcurveto{\pgfqpoint{3.372538in}{2.374301in}}{\pgfqpoint{3.376928in}{2.363702in}}{\pgfqpoint{3.384742in}{2.355889in}}%
\pgfpathcurveto{\pgfqpoint{3.392556in}{2.348075in}}{\pgfqpoint{3.403155in}{2.343685in}}{\pgfqpoint{3.414205in}{2.343685in}}%
\pgfpathclose%
\pgfusepath{stroke,fill}%
\end{pgfscope}%
\begin{pgfscope}%
\pgfpathrectangle{\pgfqpoint{0.970666in}{0.566125in}}{\pgfqpoint{5.699255in}{2.685432in}}%
\pgfusepath{clip}%
\pgfsetbuttcap%
\pgfsetroundjoin%
\definecolor{currentfill}{rgb}{0.000000,0.000000,0.000000}%
\pgfsetfillcolor{currentfill}%
\pgfsetlinewidth{1.003750pt}%
\definecolor{currentstroke}{rgb}{0.000000,0.000000,0.000000}%
\pgfsetstrokecolor{currentstroke}%
\pgfsetdash{}{0pt}%
\pgfpathmoveto{\pgfqpoint{3.442211in}{2.213134in}}%
\pgfpathcurveto{\pgfqpoint{3.453261in}{2.213134in}}{\pgfqpoint{3.463860in}{2.217524in}}{\pgfqpoint{3.471674in}{2.225338in}}%
\pgfpathcurveto{\pgfqpoint{3.479487in}{2.233151in}}{\pgfqpoint{3.483878in}{2.243750in}}{\pgfqpoint{3.483878in}{2.254801in}}%
\pgfpathcurveto{\pgfqpoint{3.483878in}{2.265851in}}{\pgfqpoint{3.479487in}{2.276450in}}{\pgfqpoint{3.471674in}{2.284263in}}%
\pgfpathcurveto{\pgfqpoint{3.463860in}{2.292077in}}{\pgfqpoint{3.453261in}{2.296467in}}{\pgfqpoint{3.442211in}{2.296467in}}%
\pgfpathcurveto{\pgfqpoint{3.431161in}{2.296467in}}{\pgfqpoint{3.420562in}{2.292077in}}{\pgfqpoint{3.412748in}{2.284263in}}%
\pgfpathcurveto{\pgfqpoint{3.404934in}{2.276450in}}{\pgfqpoint{3.400544in}{2.265851in}}{\pgfqpoint{3.400544in}{2.254801in}}%
\pgfpathcurveto{\pgfqpoint{3.400544in}{2.243750in}}{\pgfqpoint{3.404934in}{2.233151in}}{\pgfqpoint{3.412748in}{2.225338in}}%
\pgfpathcurveto{\pgfqpoint{3.420562in}{2.217524in}}{\pgfqpoint{3.431161in}{2.213134in}}{\pgfqpoint{3.442211in}{2.213134in}}%
\pgfpathclose%
\pgfusepath{stroke,fill}%
\end{pgfscope}%
\begin{pgfscope}%
\pgfpathrectangle{\pgfqpoint{0.970666in}{0.566125in}}{\pgfqpoint{5.699255in}{2.685432in}}%
\pgfusepath{clip}%
\pgfsetbuttcap%
\pgfsetroundjoin%
\definecolor{currentfill}{rgb}{0.000000,0.000000,0.000000}%
\pgfsetfillcolor{currentfill}%
\pgfsetlinewidth{1.003750pt}%
\definecolor{currentstroke}{rgb}{0.000000,0.000000,0.000000}%
\pgfsetstrokecolor{currentstroke}%
\pgfsetdash{}{0pt}%
\pgfpathmoveto{\pgfqpoint{3.946322in}{2.408960in}}%
\pgfpathcurveto{\pgfqpoint{3.957372in}{2.408960in}}{\pgfqpoint{3.967971in}{2.413351in}}{\pgfqpoint{3.975785in}{2.421164in}}%
\pgfpathcurveto{\pgfqpoint{3.983598in}{2.428978in}}{\pgfqpoint{3.987989in}{2.439577in}}{\pgfqpoint{3.987989in}{2.450627in}}%
\pgfpathcurveto{\pgfqpoint{3.987989in}{2.461677in}}{\pgfqpoint{3.983598in}{2.472276in}}{\pgfqpoint{3.975785in}{2.480090in}}%
\pgfpathcurveto{\pgfqpoint{3.967971in}{2.487903in}}{\pgfqpoint{3.957372in}{2.492294in}}{\pgfqpoint{3.946322in}{2.492294in}}%
\pgfpathcurveto{\pgfqpoint{3.935272in}{2.492294in}}{\pgfqpoint{3.924673in}{2.487903in}}{\pgfqpoint{3.916859in}{2.480090in}}%
\pgfpathcurveto{\pgfqpoint{3.909046in}{2.472276in}}{\pgfqpoint{3.904655in}{2.461677in}}{\pgfqpoint{3.904655in}{2.450627in}}%
\pgfpathcurveto{\pgfqpoint{3.904655in}{2.439577in}}{\pgfqpoint{3.909046in}{2.428978in}}{\pgfqpoint{3.916859in}{2.421164in}}%
\pgfpathcurveto{\pgfqpoint{3.924673in}{2.413351in}}{\pgfqpoint{3.935272in}{2.408960in}}{\pgfqpoint{3.946322in}{2.408960in}}%
\pgfpathclose%
\pgfusepath{stroke,fill}%
\end{pgfscope}%
\begin{pgfscope}%
\pgfpathrectangle{\pgfqpoint{0.970666in}{0.566125in}}{\pgfqpoint{5.699255in}{2.685432in}}%
\pgfusepath{clip}%
\pgfsetbuttcap%
\pgfsetroundjoin%
\definecolor{currentfill}{rgb}{0.000000,0.000000,0.000000}%
\pgfsetfillcolor{currentfill}%
\pgfsetlinewidth{1.003750pt}%
\definecolor{currentstroke}{rgb}{0.000000,0.000000,0.000000}%
\pgfsetstrokecolor{currentstroke}%
\pgfsetdash{}{0pt}%
\pgfpathmoveto{\pgfqpoint{4.226384in}{2.317575in}}%
\pgfpathcurveto{\pgfqpoint{4.237434in}{2.317575in}}{\pgfqpoint{4.248033in}{2.321965in}}{\pgfqpoint{4.255846in}{2.329779in}}%
\pgfpathcurveto{\pgfqpoint{4.263660in}{2.337592in}}{\pgfqpoint{4.268050in}{2.348191in}}{\pgfqpoint{4.268050in}{2.359241in}}%
\pgfpathcurveto{\pgfqpoint{4.268050in}{2.370291in}}{\pgfqpoint{4.263660in}{2.380890in}}{\pgfqpoint{4.255846in}{2.388704in}}%
\pgfpathcurveto{\pgfqpoint{4.248033in}{2.396518in}}{\pgfqpoint{4.237434in}{2.400908in}}{\pgfqpoint{4.226384in}{2.400908in}}%
\pgfpathcurveto{\pgfqpoint{4.215334in}{2.400908in}}{\pgfqpoint{4.204734in}{2.396518in}}{\pgfqpoint{4.196921in}{2.388704in}}%
\pgfpathcurveto{\pgfqpoint{4.189107in}{2.380890in}}{\pgfqpoint{4.184717in}{2.370291in}}{\pgfqpoint{4.184717in}{2.359241in}}%
\pgfpathcurveto{\pgfqpoint{4.184717in}{2.348191in}}{\pgfqpoint{4.189107in}{2.337592in}}{\pgfqpoint{4.196921in}{2.329779in}}%
\pgfpathcurveto{\pgfqpoint{4.204734in}{2.321965in}}{\pgfqpoint{4.215334in}{2.317575in}}{\pgfqpoint{4.226384in}{2.317575in}}%
\pgfpathclose%
\pgfusepath{stroke,fill}%
\end{pgfscope}%
\begin{pgfscope}%
\pgfpathrectangle{\pgfqpoint{0.970666in}{0.566125in}}{\pgfqpoint{5.699255in}{2.685432in}}%
\pgfusepath{clip}%
\pgfsetbuttcap%
\pgfsetroundjoin%
\definecolor{currentfill}{rgb}{0.000000,0.000000,0.000000}%
\pgfsetfillcolor{currentfill}%
\pgfsetlinewidth{1.003750pt}%
\definecolor{currentstroke}{rgb}{0.000000,0.000000,0.000000}%
\pgfsetstrokecolor{currentstroke}%
\pgfsetdash{}{0pt}%
\pgfpathmoveto{\pgfqpoint{4.310402in}{2.252299in}}%
\pgfpathcurveto{\pgfqpoint{4.321452in}{2.252299in}}{\pgfqpoint{4.332051in}{2.256689in}}{\pgfqpoint{4.339865in}{2.264503in}}%
\pgfpathcurveto{\pgfqpoint{4.347679in}{2.272317in}}{\pgfqpoint{4.352069in}{2.282916in}}{\pgfqpoint{4.352069in}{2.293966in}}%
\pgfpathcurveto{\pgfqpoint{4.352069in}{2.305016in}}{\pgfqpoint{4.347679in}{2.315615in}}{\pgfqpoint{4.339865in}{2.323429in}}%
\pgfpathcurveto{\pgfqpoint{4.332051in}{2.331242in}}{\pgfqpoint{4.321452in}{2.335632in}}{\pgfqpoint{4.310402in}{2.335632in}}%
\pgfpathcurveto{\pgfqpoint{4.299352in}{2.335632in}}{\pgfqpoint{4.288753in}{2.331242in}}{\pgfqpoint{4.280939in}{2.323429in}}%
\pgfpathcurveto{\pgfqpoint{4.273126in}{2.315615in}}{\pgfqpoint{4.268735in}{2.305016in}}{\pgfqpoint{4.268735in}{2.293966in}}%
\pgfpathcurveto{\pgfqpoint{4.268735in}{2.282916in}}{\pgfqpoint{4.273126in}{2.272317in}}{\pgfqpoint{4.280939in}{2.264503in}}%
\pgfpathcurveto{\pgfqpoint{4.288753in}{2.256689in}}{\pgfqpoint{4.299352in}{2.252299in}}{\pgfqpoint{4.310402in}{2.252299in}}%
\pgfpathclose%
\pgfusepath{stroke,fill}%
\end{pgfscope}%
\begin{pgfscope}%
\pgfpathrectangle{\pgfqpoint{0.970666in}{0.566125in}}{\pgfqpoint{5.699255in}{2.685432in}}%
\pgfusepath{clip}%
\pgfsetbuttcap%
\pgfsetroundjoin%
\definecolor{currentfill}{rgb}{0.000000,0.000000,0.000000}%
\pgfsetfillcolor{currentfill}%
\pgfsetlinewidth{1.003750pt}%
\definecolor{currentstroke}{rgb}{0.000000,0.000000,0.000000}%
\pgfsetstrokecolor{currentstroke}%
\pgfsetdash{}{0pt}%
\pgfpathmoveto{\pgfqpoint{5.318624in}{2.291464in}}%
\pgfpathcurveto{\pgfqpoint{5.329674in}{2.291464in}}{\pgfqpoint{5.340273in}{2.295855in}}{\pgfqpoint{5.348087in}{2.303668in}}%
\pgfpathcurveto{\pgfqpoint{5.355901in}{2.311482in}}{\pgfqpoint{5.360291in}{2.322081in}}{\pgfqpoint{5.360291in}{2.333131in}}%
\pgfpathcurveto{\pgfqpoint{5.360291in}{2.344181in}}{\pgfqpoint{5.355901in}{2.354780in}}{\pgfqpoint{5.348087in}{2.362594in}}%
\pgfpathcurveto{\pgfqpoint{5.340273in}{2.370408in}}{\pgfqpoint{5.329674in}{2.374798in}}{\pgfqpoint{5.318624in}{2.374798in}}%
\pgfpathcurveto{\pgfqpoint{5.307574in}{2.374798in}}{\pgfqpoint{5.296975in}{2.370408in}}{\pgfqpoint{5.289161in}{2.362594in}}%
\pgfpathcurveto{\pgfqpoint{5.281348in}{2.354780in}}{\pgfqpoint{5.276958in}{2.344181in}}{\pgfqpoint{5.276958in}{2.333131in}}%
\pgfpathcurveto{\pgfqpoint{5.276958in}{2.322081in}}{\pgfqpoint{5.281348in}{2.311482in}}{\pgfqpoint{5.289161in}{2.303668in}}%
\pgfpathcurveto{\pgfqpoint{5.296975in}{2.295855in}}{\pgfqpoint{5.307574in}{2.291464in}}{\pgfqpoint{5.318624in}{2.291464in}}%
\pgfpathclose%
\pgfusepath{stroke,fill}%
\end{pgfscope}%
\begin{pgfscope}%
\pgfpathrectangle{\pgfqpoint{0.970666in}{0.566125in}}{\pgfqpoint{5.699255in}{2.685432in}}%
\pgfusepath{clip}%
\pgfsetbuttcap%
\pgfsetroundjoin%
\definecolor{currentfill}{rgb}{0.000000,0.000000,0.000000}%
\pgfsetfillcolor{currentfill}%
\pgfsetlinewidth{1.003750pt}%
\definecolor{currentstroke}{rgb}{0.000000,0.000000,0.000000}%
\pgfsetstrokecolor{currentstroke}%
\pgfsetdash{}{0pt}%
\pgfpathmoveto{\pgfqpoint{5.430649in}{2.200079in}}%
\pgfpathcurveto{\pgfqpoint{5.441699in}{2.200079in}}{\pgfqpoint{5.452298in}{2.204469in}}{\pgfqpoint{5.460112in}{2.212283in}}%
\pgfpathcurveto{\pgfqpoint{5.467925in}{2.220096in}}{\pgfqpoint{5.472316in}{2.230695in}}{\pgfqpoint{5.472316in}{2.241745in}}%
\pgfpathcurveto{\pgfqpoint{5.472316in}{2.252796in}}{\pgfqpoint{5.467925in}{2.263395in}}{\pgfqpoint{5.460112in}{2.271208in}}%
\pgfpathcurveto{\pgfqpoint{5.452298in}{2.279022in}}{\pgfqpoint{5.441699in}{2.283412in}}{\pgfqpoint{5.430649in}{2.283412in}}%
\pgfpathcurveto{\pgfqpoint{5.419599in}{2.283412in}}{\pgfqpoint{5.409000in}{2.279022in}}{\pgfqpoint{5.401186in}{2.271208in}}%
\pgfpathcurveto{\pgfqpoint{5.393372in}{2.263395in}}{\pgfqpoint{5.388982in}{2.252796in}}{\pgfqpoint{5.388982in}{2.241745in}}%
\pgfpathcurveto{\pgfqpoint{5.388982in}{2.230695in}}{\pgfqpoint{5.393372in}{2.220096in}}{\pgfqpoint{5.401186in}{2.212283in}}%
\pgfpathcurveto{\pgfqpoint{5.409000in}{2.204469in}}{\pgfqpoint{5.419599in}{2.200079in}}{\pgfqpoint{5.430649in}{2.200079in}}%
\pgfpathclose%
\pgfusepath{stroke,fill}%
\end{pgfscope}%
\begin{pgfscope}%
\pgfpathrectangle{\pgfqpoint{0.970666in}{0.566125in}}{\pgfqpoint{5.699255in}{2.685432in}}%
\pgfusepath{clip}%
\pgfsetbuttcap%
\pgfsetroundjoin%
\definecolor{currentfill}{rgb}{0.000000,0.000000,0.000000}%
\pgfsetfillcolor{currentfill}%
\pgfsetlinewidth{1.003750pt}%
\definecolor{currentstroke}{rgb}{0.000000,0.000000,0.000000}%
\pgfsetstrokecolor{currentstroke}%
\pgfsetdash{}{0pt}%
\pgfpathmoveto{\pgfqpoint{5.486661in}{2.422015in}}%
\pgfpathcurveto{\pgfqpoint{5.497711in}{2.422015in}}{\pgfqpoint{5.508310in}{2.426406in}}{\pgfqpoint{5.516124in}{2.434219in}}%
\pgfpathcurveto{\pgfqpoint{5.523938in}{2.442033in}}{\pgfqpoint{5.528328in}{2.452632in}}{\pgfqpoint{5.528328in}{2.463682in}}%
\pgfpathcurveto{\pgfqpoint{5.528328in}{2.474732in}}{\pgfqpoint{5.523938in}{2.485331in}}{\pgfqpoint{5.516124in}{2.493145in}}%
\pgfpathcurveto{\pgfqpoint{5.508310in}{2.500958in}}{\pgfqpoint{5.497711in}{2.505349in}}{\pgfqpoint{5.486661in}{2.505349in}}%
\pgfpathcurveto{\pgfqpoint{5.475611in}{2.505349in}}{\pgfqpoint{5.465012in}{2.500958in}}{\pgfqpoint{5.457198in}{2.493145in}}%
\pgfpathcurveto{\pgfqpoint{5.449385in}{2.485331in}}{\pgfqpoint{5.444995in}{2.474732in}}{\pgfqpoint{5.444995in}{2.463682in}}%
\pgfpathcurveto{\pgfqpoint{5.444995in}{2.452632in}}{\pgfqpoint{5.449385in}{2.442033in}}{\pgfqpoint{5.457198in}{2.434219in}}%
\pgfpathcurveto{\pgfqpoint{5.465012in}{2.426406in}}{\pgfqpoint{5.475611in}{2.422015in}}{\pgfqpoint{5.486661in}{2.422015in}}%
\pgfpathclose%
\pgfusepath{stroke,fill}%
\end{pgfscope}%
\begin{pgfscope}%
\pgfpathrectangle{\pgfqpoint{0.970666in}{0.566125in}}{\pgfqpoint{5.699255in}{2.685432in}}%
\pgfusepath{clip}%
\pgfsetbuttcap%
\pgfsetroundjoin%
\definecolor{currentfill}{rgb}{0.000000,0.000000,0.000000}%
\pgfsetfillcolor{currentfill}%
\pgfsetlinewidth{1.003750pt}%
\definecolor{currentstroke}{rgb}{0.000000,0.000000,0.000000}%
\pgfsetstrokecolor{currentstroke}%
\pgfsetdash{}{0pt}%
\pgfpathmoveto{\pgfqpoint{5.570680in}{2.461181in}}%
\pgfpathcurveto{\pgfqpoint{5.581730in}{2.461181in}}{\pgfqpoint{5.592329in}{2.465571in}}{\pgfqpoint{5.600143in}{2.473384in}}%
\pgfpathcurveto{\pgfqpoint{5.607956in}{2.481198in}}{\pgfqpoint{5.612346in}{2.491797in}}{\pgfqpoint{5.612346in}{2.502847in}}%
\pgfpathcurveto{\pgfqpoint{5.612346in}{2.513897in}}{\pgfqpoint{5.607956in}{2.524496in}}{\pgfqpoint{5.600143in}{2.532310in}}%
\pgfpathcurveto{\pgfqpoint{5.592329in}{2.540124in}}{\pgfqpoint{5.581730in}{2.544514in}}{\pgfqpoint{5.570680in}{2.544514in}}%
\pgfpathcurveto{\pgfqpoint{5.559630in}{2.544514in}}{\pgfqpoint{5.549031in}{2.540124in}}{\pgfqpoint{5.541217in}{2.532310in}}%
\pgfpathcurveto{\pgfqpoint{5.533403in}{2.524496in}}{\pgfqpoint{5.529013in}{2.513897in}}{\pgfqpoint{5.529013in}{2.502847in}}%
\pgfpathcurveto{\pgfqpoint{5.529013in}{2.491797in}}{\pgfqpoint{5.533403in}{2.481198in}}{\pgfqpoint{5.541217in}{2.473384in}}%
\pgfpathcurveto{\pgfqpoint{5.549031in}{2.465571in}}{\pgfqpoint{5.559630in}{2.461181in}}{\pgfqpoint{5.570680in}{2.461181in}}%
\pgfpathclose%
\pgfusepath{stroke,fill}%
\end{pgfscope}%
\begin{pgfscope}%
\pgfpathrectangle{\pgfqpoint{0.970666in}{0.566125in}}{\pgfqpoint{5.699255in}{2.685432in}}%
\pgfusepath{clip}%
\pgfsetbuttcap%
\pgfsetroundjoin%
\definecolor{currentfill}{rgb}{0.000000,0.000000,0.000000}%
\pgfsetfillcolor{currentfill}%
\pgfsetlinewidth{1.003750pt}%
\definecolor{currentstroke}{rgb}{0.000000,0.000000,0.000000}%
\pgfsetstrokecolor{currentstroke}%
\pgfsetdash{}{0pt}%
\pgfpathmoveto{\pgfqpoint{5.906754in}{2.330630in}}%
\pgfpathcurveto{\pgfqpoint{5.917804in}{2.330630in}}{\pgfqpoint{5.928403in}{2.335020in}}{\pgfqpoint{5.936217in}{2.342834in}}%
\pgfpathcurveto{\pgfqpoint{5.944030in}{2.350647in}}{\pgfqpoint{5.948420in}{2.361246in}}{\pgfqpoint{5.948420in}{2.372296in}}%
\pgfpathcurveto{\pgfqpoint{5.948420in}{2.383347in}}{\pgfqpoint{5.944030in}{2.393946in}}{\pgfqpoint{5.936217in}{2.401759in}}%
\pgfpathcurveto{\pgfqpoint{5.928403in}{2.409573in}}{\pgfqpoint{5.917804in}{2.413963in}}{\pgfqpoint{5.906754in}{2.413963in}}%
\pgfpathcurveto{\pgfqpoint{5.895704in}{2.413963in}}{\pgfqpoint{5.885105in}{2.409573in}}{\pgfqpoint{5.877291in}{2.401759in}}%
\pgfpathcurveto{\pgfqpoint{5.869477in}{2.393946in}}{\pgfqpoint{5.865087in}{2.383347in}}{\pgfqpoint{5.865087in}{2.372296in}}%
\pgfpathcurveto{\pgfqpoint{5.865087in}{2.361246in}}{\pgfqpoint{5.869477in}{2.350647in}}{\pgfqpoint{5.877291in}{2.342834in}}%
\pgfpathcurveto{\pgfqpoint{5.885105in}{2.335020in}}{\pgfqpoint{5.895704in}{2.330630in}}{\pgfqpoint{5.906754in}{2.330630in}}%
\pgfpathclose%
\pgfusepath{stroke,fill}%
\end{pgfscope}%
\begin{pgfscope}%
\pgfpathrectangle{\pgfqpoint{0.970666in}{0.566125in}}{\pgfqpoint{5.699255in}{2.685432in}}%
\pgfusepath{clip}%
\pgfsetbuttcap%
\pgfsetroundjoin%
\definecolor{currentfill}{rgb}{0.000000,0.000000,0.000000}%
\pgfsetfillcolor{currentfill}%
\pgfsetlinewidth{1.003750pt}%
\definecolor{currentstroke}{rgb}{0.000000,0.000000,0.000000}%
\pgfsetstrokecolor{currentstroke}%
\pgfsetdash{}{0pt}%
\pgfpathmoveto{\pgfqpoint{6.326846in}{2.226189in}}%
\pgfpathcurveto{\pgfqpoint{6.337896in}{2.226189in}}{\pgfqpoint{6.348495in}{2.230579in}}{\pgfqpoint{6.356309in}{2.238393in}}%
\pgfpathcurveto{\pgfqpoint{6.364123in}{2.246206in}}{\pgfqpoint{6.368513in}{2.256806in}}{\pgfqpoint{6.368513in}{2.267856in}}%
\pgfpathcurveto{\pgfqpoint{6.368513in}{2.278906in}}{\pgfqpoint{6.364123in}{2.289505in}}{\pgfqpoint{6.356309in}{2.297318in}}%
\pgfpathcurveto{\pgfqpoint{6.348495in}{2.305132in}}{\pgfqpoint{6.337896in}{2.309522in}}{\pgfqpoint{6.326846in}{2.309522in}}%
\pgfpathcurveto{\pgfqpoint{6.315796in}{2.309522in}}{\pgfqpoint{6.305197in}{2.305132in}}{\pgfqpoint{6.297384in}{2.297318in}}%
\pgfpathcurveto{\pgfqpoint{6.289570in}{2.289505in}}{\pgfqpoint{6.285180in}{2.278906in}}{\pgfqpoint{6.285180in}{2.267856in}}%
\pgfpathcurveto{\pgfqpoint{6.285180in}{2.256806in}}{\pgfqpoint{6.289570in}{2.246206in}}{\pgfqpoint{6.297384in}{2.238393in}}%
\pgfpathcurveto{\pgfqpoint{6.305197in}{2.230579in}}{\pgfqpoint{6.315796in}{2.226189in}}{\pgfqpoint{6.326846in}{2.226189in}}%
\pgfpathclose%
\pgfusepath{stroke,fill}%
\end{pgfscope}%
\begin{pgfscope}%
\pgfpathrectangle{\pgfqpoint{0.970666in}{0.566125in}}{\pgfqpoint{5.699255in}{2.685432in}}%
\pgfusepath{clip}%
\pgfsetbuttcap%
\pgfsetroundjoin%
\definecolor{currentfill}{rgb}{0.000000,0.000000,0.000000}%
\pgfsetfillcolor{currentfill}%
\pgfsetlinewidth{1.003750pt}%
\definecolor{currentstroke}{rgb}{0.000000,0.000000,0.000000}%
\pgfsetstrokecolor{currentstroke}%
\pgfsetdash{}{0pt}%
\pgfpathmoveto{\pgfqpoint{6.410865in}{2.382850in}}%
\pgfpathcurveto{\pgfqpoint{6.421915in}{2.382850in}}{\pgfqpoint{6.432514in}{2.387240in}}{\pgfqpoint{6.440328in}{2.395054in}}%
\pgfpathcurveto{\pgfqpoint{6.448141in}{2.402868in}}{\pgfqpoint{6.452531in}{2.413467in}}{\pgfqpoint{6.452531in}{2.424517in}}%
\pgfpathcurveto{\pgfqpoint{6.452531in}{2.435567in}}{\pgfqpoint{6.448141in}{2.446166in}}{\pgfqpoint{6.440328in}{2.453980in}}%
\pgfpathcurveto{\pgfqpoint{6.432514in}{2.461793in}}{\pgfqpoint{6.421915in}{2.466183in}}{\pgfqpoint{6.410865in}{2.466183in}}%
\pgfpathcurveto{\pgfqpoint{6.399815in}{2.466183in}}{\pgfqpoint{6.389216in}{2.461793in}}{\pgfqpoint{6.381402in}{2.453980in}}%
\pgfpathcurveto{\pgfqpoint{6.373588in}{2.446166in}}{\pgfqpoint{6.369198in}{2.435567in}}{\pgfqpoint{6.369198in}{2.424517in}}%
\pgfpathcurveto{\pgfqpoint{6.369198in}{2.413467in}}{\pgfqpoint{6.373588in}{2.402868in}}{\pgfqpoint{6.381402in}{2.395054in}}%
\pgfpathcurveto{\pgfqpoint{6.389216in}{2.387240in}}{\pgfqpoint{6.399815in}{2.382850in}}{\pgfqpoint{6.410865in}{2.382850in}}%
\pgfpathclose%
\pgfusepath{stroke,fill}%
\end{pgfscope}%
\begin{pgfscope}%
\pgfpathrectangle{\pgfqpoint{0.970666in}{0.566125in}}{\pgfqpoint{5.699255in}{2.685432in}}%
\pgfusepath{clip}%
\pgfsetbuttcap%
\pgfsetroundjoin%
\definecolor{currentfill}{rgb}{0.000000,0.000000,0.000000}%
\pgfsetfillcolor{currentfill}%
\pgfsetlinewidth{1.003750pt}%
\definecolor{currentstroke}{rgb}{0.000000,0.000000,0.000000}%
\pgfsetstrokecolor{currentstroke}%
\pgfsetdash{}{0pt}%
\pgfpathmoveto{\pgfqpoint{5.990772in}{2.670062in}}%
\pgfpathcurveto{\pgfqpoint{6.001822in}{2.670062in}}{\pgfqpoint{6.012421in}{2.674452in}}{\pgfqpoint{6.020235in}{2.682266in}}%
\pgfpathcurveto{\pgfqpoint{6.028049in}{2.690080in}}{\pgfqpoint{6.032439in}{2.700679in}}{\pgfqpoint{6.032439in}{2.711729in}}%
\pgfpathcurveto{\pgfqpoint{6.032439in}{2.722779in}}{\pgfqpoint{6.028049in}{2.733378in}}{\pgfqpoint{6.020235in}{2.741192in}}%
\pgfpathcurveto{\pgfqpoint{6.012421in}{2.749005in}}{\pgfqpoint{6.001822in}{2.753395in}}{\pgfqpoint{5.990772in}{2.753395in}}%
\pgfpathcurveto{\pgfqpoint{5.979722in}{2.753395in}}{\pgfqpoint{5.969123in}{2.749005in}}{\pgfqpoint{5.961309in}{2.741192in}}%
\pgfpathcurveto{\pgfqpoint{5.953496in}{2.733378in}}{\pgfqpoint{5.949106in}{2.722779in}}{\pgfqpoint{5.949106in}{2.711729in}}%
\pgfpathcurveto{\pgfqpoint{5.949106in}{2.700679in}}{\pgfqpoint{5.953496in}{2.690080in}}{\pgfqpoint{5.961309in}{2.682266in}}%
\pgfpathcurveto{\pgfqpoint{5.969123in}{2.674452in}}{\pgfqpoint{5.979722in}{2.670062in}}{\pgfqpoint{5.990772in}{2.670062in}}%
\pgfpathclose%
\pgfusepath{stroke,fill}%
\end{pgfscope}%
\begin{pgfscope}%
\pgfpathrectangle{\pgfqpoint{0.970666in}{0.566125in}}{\pgfqpoint{5.699255in}{2.685432in}}%
\pgfusepath{clip}%
\pgfsetbuttcap%
\pgfsetroundjoin%
\definecolor{currentfill}{rgb}{0.000000,0.000000,0.000000}%
\pgfsetfillcolor{currentfill}%
\pgfsetlinewidth{1.003750pt}%
\definecolor{currentstroke}{rgb}{0.000000,0.000000,0.000000}%
\pgfsetstrokecolor{currentstroke}%
\pgfsetdash{}{0pt}%
\pgfpathmoveto{\pgfqpoint{5.906754in}{2.761448in}}%
\pgfpathcurveto{\pgfqpoint{5.917804in}{2.761448in}}{\pgfqpoint{5.928403in}{2.765838in}}{\pgfqpoint{5.936217in}{2.773652in}}%
\pgfpathcurveto{\pgfqpoint{5.944030in}{2.781465in}}{\pgfqpoint{5.948420in}{2.792064in}}{\pgfqpoint{5.948420in}{2.803114in}}%
\pgfpathcurveto{\pgfqpoint{5.948420in}{2.814164in}}{\pgfqpoint{5.944030in}{2.824764in}}{\pgfqpoint{5.936217in}{2.832577in}}%
\pgfpathcurveto{\pgfqpoint{5.928403in}{2.840391in}}{\pgfqpoint{5.917804in}{2.844781in}}{\pgfqpoint{5.906754in}{2.844781in}}%
\pgfpathcurveto{\pgfqpoint{5.895704in}{2.844781in}}{\pgfqpoint{5.885105in}{2.840391in}}{\pgfqpoint{5.877291in}{2.832577in}}%
\pgfpathcurveto{\pgfqpoint{5.869477in}{2.824764in}}{\pgfqpoint{5.865087in}{2.814164in}}{\pgfqpoint{5.865087in}{2.803114in}}%
\pgfpathcurveto{\pgfqpoint{5.865087in}{2.792064in}}{\pgfqpoint{5.869477in}{2.781465in}}{\pgfqpoint{5.877291in}{2.773652in}}%
\pgfpathcurveto{\pgfqpoint{5.885105in}{2.765838in}}{\pgfqpoint{5.895704in}{2.761448in}}{\pgfqpoint{5.906754in}{2.761448in}}%
\pgfpathclose%
\pgfusepath{stroke,fill}%
\end{pgfscope}%
\begin{pgfscope}%
\pgfpathrectangle{\pgfqpoint{0.970666in}{0.566125in}}{\pgfqpoint{5.699255in}{2.685432in}}%
\pgfusepath{clip}%
\pgfsetbuttcap%
\pgfsetroundjoin%
\definecolor{currentfill}{rgb}{0.000000,0.000000,0.000000}%
\pgfsetfillcolor{currentfill}%
\pgfsetlinewidth{1.003750pt}%
\definecolor{currentstroke}{rgb}{0.000000,0.000000,0.000000}%
\pgfsetstrokecolor{currentstroke}%
\pgfsetdash{}{0pt}%
\pgfpathmoveto{\pgfqpoint{6.102797in}{2.813668in}}%
\pgfpathcurveto{\pgfqpoint{6.113847in}{2.813668in}}{\pgfqpoint{6.124446in}{2.818058in}}{\pgfqpoint{6.132260in}{2.825872in}}%
\pgfpathcurveto{\pgfqpoint{6.140073in}{2.833686in}}{\pgfqpoint{6.144464in}{2.844285in}}{\pgfqpoint{6.144464in}{2.855335in}}%
\pgfpathcurveto{\pgfqpoint{6.144464in}{2.866385in}}{\pgfqpoint{6.140073in}{2.876984in}}{\pgfqpoint{6.132260in}{2.884798in}}%
\pgfpathcurveto{\pgfqpoint{6.124446in}{2.892611in}}{\pgfqpoint{6.113847in}{2.897001in}}{\pgfqpoint{6.102797in}{2.897001in}}%
\pgfpathcurveto{\pgfqpoint{6.091747in}{2.897001in}}{\pgfqpoint{6.081148in}{2.892611in}}{\pgfqpoint{6.073334in}{2.884798in}}%
\pgfpathcurveto{\pgfqpoint{6.065521in}{2.876984in}}{\pgfqpoint{6.061130in}{2.866385in}}{\pgfqpoint{6.061130in}{2.855335in}}%
\pgfpathcurveto{\pgfqpoint{6.061130in}{2.844285in}}{\pgfqpoint{6.065521in}{2.833686in}}{\pgfqpoint{6.073334in}{2.825872in}}%
\pgfpathcurveto{\pgfqpoint{6.081148in}{2.818058in}}{\pgfqpoint{6.091747in}{2.813668in}}{\pgfqpoint{6.102797in}{2.813668in}}%
\pgfpathclose%
\pgfusepath{stroke,fill}%
\end{pgfscope}%
\begin{pgfscope}%
\pgfpathrectangle{\pgfqpoint{0.970666in}{0.566125in}}{\pgfqpoint{5.699255in}{2.685432in}}%
\pgfusepath{clip}%
\pgfsetbuttcap%
\pgfsetroundjoin%
\definecolor{currentfill}{rgb}{0.000000,0.000000,0.000000}%
\pgfsetfillcolor{currentfill}%
\pgfsetlinewidth{1.003750pt}%
\definecolor{currentstroke}{rgb}{0.000000,0.000000,0.000000}%
\pgfsetstrokecolor{currentstroke}%
\pgfsetdash{}{0pt}%
\pgfpathmoveto{\pgfqpoint{6.242828in}{3.074770in}}%
\pgfpathcurveto{\pgfqpoint{6.253878in}{3.074770in}}{\pgfqpoint{6.264477in}{3.079160in}}{\pgfqpoint{6.272291in}{3.086974in}}%
\pgfpathcurveto{\pgfqpoint{6.280104in}{3.094787in}}{\pgfqpoint{6.284494in}{3.105386in}}{\pgfqpoint{6.284494in}{3.116437in}}%
\pgfpathcurveto{\pgfqpoint{6.284494in}{3.127487in}}{\pgfqpoint{6.280104in}{3.138086in}}{\pgfqpoint{6.272291in}{3.145899in}}%
\pgfpathcurveto{\pgfqpoint{6.264477in}{3.153713in}}{\pgfqpoint{6.253878in}{3.158103in}}{\pgfqpoint{6.242828in}{3.158103in}}%
\pgfpathcurveto{\pgfqpoint{6.231778in}{3.158103in}}{\pgfqpoint{6.221179in}{3.153713in}}{\pgfqpoint{6.213365in}{3.145899in}}%
\pgfpathcurveto{\pgfqpoint{6.205551in}{3.138086in}}{\pgfqpoint{6.201161in}{3.127487in}}{\pgfqpoint{6.201161in}{3.116437in}}%
\pgfpathcurveto{\pgfqpoint{6.201161in}{3.105386in}}{\pgfqpoint{6.205551in}{3.094787in}}{\pgfqpoint{6.213365in}{3.086974in}}%
\pgfpathcurveto{\pgfqpoint{6.221179in}{3.079160in}}{\pgfqpoint{6.231778in}{3.074770in}}{\pgfqpoint{6.242828in}{3.074770in}}%
\pgfpathclose%
\pgfusepath{stroke,fill}%
\end{pgfscope}%
\begin{pgfscope}%
\pgfpathrectangle{\pgfqpoint{0.970666in}{0.566125in}}{\pgfqpoint{5.699255in}{2.685432in}}%
\pgfusepath{clip}%
\pgfsetbuttcap%
\pgfsetroundjoin%
\definecolor{currentfill}{rgb}{0.000000,0.000000,0.000000}%
\pgfsetfillcolor{currentfill}%
\pgfsetlinewidth{1.003750pt}%
\definecolor{currentstroke}{rgb}{0.000000,0.000000,0.000000}%
\pgfsetstrokecolor{currentstroke}%
\pgfsetdash{}{0pt}%
\pgfpathmoveto{\pgfqpoint{5.710711in}{3.087825in}}%
\pgfpathcurveto{\pgfqpoint{5.721761in}{3.087825in}}{\pgfqpoint{5.732360in}{3.092215in}}{\pgfqpoint{5.740173in}{3.100029in}}%
\pgfpathcurveto{\pgfqpoint{5.747987in}{3.107842in}}{\pgfqpoint{5.752377in}{3.118441in}}{\pgfqpoint{5.752377in}{3.129492in}}%
\pgfpathcurveto{\pgfqpoint{5.752377in}{3.140542in}}{\pgfqpoint{5.747987in}{3.151141in}}{\pgfqpoint{5.740173in}{3.158954in}}%
\pgfpathcurveto{\pgfqpoint{5.732360in}{3.166768in}}{\pgfqpoint{5.721761in}{3.171158in}}{\pgfqpoint{5.710711in}{3.171158in}}%
\pgfpathcurveto{\pgfqpoint{5.699660in}{3.171158in}}{\pgfqpoint{5.689061in}{3.166768in}}{\pgfqpoint{5.681248in}{3.158954in}}%
\pgfpathcurveto{\pgfqpoint{5.673434in}{3.151141in}}{\pgfqpoint{5.669044in}{3.140542in}}{\pgfqpoint{5.669044in}{3.129492in}}%
\pgfpathcurveto{\pgfqpoint{5.669044in}{3.118441in}}{\pgfqpoint{5.673434in}{3.107842in}}{\pgfqpoint{5.681248in}{3.100029in}}%
\pgfpathcurveto{\pgfqpoint{5.689061in}{3.092215in}}{\pgfqpoint{5.699660in}{3.087825in}}{\pgfqpoint{5.710711in}{3.087825in}}%
\pgfpathclose%
\pgfusepath{stroke,fill}%
\end{pgfscope}%
\begin{pgfscope}%
\pgfpathrectangle{\pgfqpoint{0.970666in}{0.566125in}}{\pgfqpoint{5.699255in}{2.685432in}}%
\pgfusepath{clip}%
\pgfsetbuttcap%
\pgfsetroundjoin%
\definecolor{currentfill}{rgb}{0.000000,0.000000,0.000000}%
\pgfsetfillcolor{currentfill}%
\pgfsetlinewidth{1.003750pt}%
\definecolor{currentstroke}{rgb}{0.000000,0.000000,0.000000}%
\pgfsetstrokecolor{currentstroke}%
\pgfsetdash{}{0pt}%
\pgfpathmoveto{\pgfqpoint{5.318624in}{2.800613in}}%
\pgfpathcurveto{\pgfqpoint{5.329674in}{2.800613in}}{\pgfqpoint{5.340273in}{2.805003in}}{\pgfqpoint{5.348087in}{2.812817in}}%
\pgfpathcurveto{\pgfqpoint{5.355901in}{2.820630in}}{\pgfqpoint{5.360291in}{2.831229in}}{\pgfqpoint{5.360291in}{2.842280in}}%
\pgfpathcurveto{\pgfqpoint{5.360291in}{2.853330in}}{\pgfqpoint{5.355901in}{2.863929in}}{\pgfqpoint{5.348087in}{2.871742in}}%
\pgfpathcurveto{\pgfqpoint{5.340273in}{2.879556in}}{\pgfqpoint{5.329674in}{2.883946in}}{\pgfqpoint{5.318624in}{2.883946in}}%
\pgfpathcurveto{\pgfqpoint{5.307574in}{2.883946in}}{\pgfqpoint{5.296975in}{2.879556in}}{\pgfqpoint{5.289161in}{2.871742in}}%
\pgfpathcurveto{\pgfqpoint{5.281348in}{2.863929in}}{\pgfqpoint{5.276958in}{2.853330in}}{\pgfqpoint{5.276958in}{2.842280in}}%
\pgfpathcurveto{\pgfqpoint{5.276958in}{2.831229in}}{\pgfqpoint{5.281348in}{2.820630in}}{\pgfqpoint{5.289161in}{2.812817in}}%
\pgfpathcurveto{\pgfqpoint{5.296975in}{2.805003in}}{\pgfqpoint{5.307574in}{2.800613in}}{\pgfqpoint{5.318624in}{2.800613in}}%
\pgfpathclose%
\pgfusepath{stroke,fill}%
\end{pgfscope}%
\begin{pgfscope}%
\pgfpathrectangle{\pgfqpoint{0.970666in}{0.566125in}}{\pgfqpoint{5.699255in}{2.685432in}}%
\pgfusepath{clip}%
\pgfsetbuttcap%
\pgfsetroundjoin%
\definecolor{currentfill}{rgb}{0.000000,0.000000,0.000000}%
\pgfsetfillcolor{currentfill}%
\pgfsetlinewidth{1.003750pt}%
\definecolor{currentstroke}{rgb}{0.000000,0.000000,0.000000}%
\pgfsetstrokecolor{currentstroke}%
\pgfsetdash{}{0pt}%
\pgfpathmoveto{\pgfqpoint{5.234606in}{2.826723in}}%
\pgfpathcurveto{\pgfqpoint{5.245656in}{2.826723in}}{\pgfqpoint{5.256255in}{2.831113in}}{\pgfqpoint{5.264068in}{2.838927in}}%
\pgfpathcurveto{\pgfqpoint{5.271882in}{2.846741in}}{\pgfqpoint{5.276272in}{2.857340in}}{\pgfqpoint{5.276272in}{2.868390in}}%
\pgfpathcurveto{\pgfqpoint{5.276272in}{2.879440in}}{\pgfqpoint{5.271882in}{2.890039in}}{\pgfqpoint{5.264068in}{2.897853in}}%
\pgfpathcurveto{\pgfqpoint{5.256255in}{2.905666in}}{\pgfqpoint{5.245656in}{2.910056in}}{\pgfqpoint{5.234606in}{2.910056in}}%
\pgfpathcurveto{\pgfqpoint{5.223556in}{2.910056in}}{\pgfqpoint{5.212957in}{2.905666in}}{\pgfqpoint{5.205143in}{2.897853in}}%
\pgfpathcurveto{\pgfqpoint{5.197329in}{2.890039in}}{\pgfqpoint{5.192939in}{2.879440in}}{\pgfqpoint{5.192939in}{2.868390in}}%
\pgfpathcurveto{\pgfqpoint{5.192939in}{2.857340in}}{\pgfqpoint{5.197329in}{2.846741in}}{\pgfqpoint{5.205143in}{2.838927in}}%
\pgfpathcurveto{\pgfqpoint{5.212957in}{2.831113in}}{\pgfqpoint{5.223556in}{2.826723in}}{\pgfqpoint{5.234606in}{2.826723in}}%
\pgfpathclose%
\pgfusepath{stroke,fill}%
\end{pgfscope}%
\begin{pgfscope}%
\pgfpathrectangle{\pgfqpoint{0.970666in}{0.566125in}}{\pgfqpoint{5.699255in}{2.685432in}}%
\pgfusepath{clip}%
\pgfsetbuttcap%
\pgfsetroundjoin%
\definecolor{currentfill}{rgb}{0.000000,0.000000,0.000000}%
\pgfsetfillcolor{currentfill}%
\pgfsetlinewidth{1.003750pt}%
\definecolor{currentstroke}{rgb}{0.000000,0.000000,0.000000}%
\pgfsetstrokecolor{currentstroke}%
\pgfsetdash{}{0pt}%
\pgfpathmoveto{\pgfqpoint{4.982550in}{2.905054in}}%
\pgfpathcurveto{\pgfqpoint{4.993600in}{2.905054in}}{\pgfqpoint{5.004199in}{2.909444in}}{\pgfqpoint{5.012013in}{2.917258in}}%
\pgfpathcurveto{\pgfqpoint{5.019827in}{2.925071in}}{\pgfqpoint{5.024217in}{2.935670in}}{\pgfqpoint{5.024217in}{2.946720in}}%
\pgfpathcurveto{\pgfqpoint{5.024217in}{2.957770in}}{\pgfqpoint{5.019827in}{2.968370in}}{\pgfqpoint{5.012013in}{2.976183in}}%
\pgfpathcurveto{\pgfqpoint{5.004199in}{2.983997in}}{\pgfqpoint{4.993600in}{2.988387in}}{\pgfqpoint{4.982550in}{2.988387in}}%
\pgfpathcurveto{\pgfqpoint{4.971500in}{2.988387in}}{\pgfqpoint{4.960901in}{2.983997in}}{\pgfqpoint{4.953087in}{2.976183in}}%
\pgfpathcurveto{\pgfqpoint{4.945274in}{2.968370in}}{\pgfqpoint{4.940884in}{2.957770in}}{\pgfqpoint{4.940884in}{2.946720in}}%
\pgfpathcurveto{\pgfqpoint{4.940884in}{2.935670in}}{\pgfqpoint{4.945274in}{2.925071in}}{\pgfqpoint{4.953087in}{2.917258in}}%
\pgfpathcurveto{\pgfqpoint{4.960901in}{2.909444in}}{\pgfqpoint{4.971500in}{2.905054in}}{\pgfqpoint{4.982550in}{2.905054in}}%
\pgfpathclose%
\pgfusepath{stroke,fill}%
\end{pgfscope}%
\begin{pgfscope}%
\pgfpathrectangle{\pgfqpoint{0.970666in}{0.566125in}}{\pgfqpoint{5.699255in}{2.685432in}}%
\pgfusepath{clip}%
\pgfsetbuttcap%
\pgfsetroundjoin%
\definecolor{currentfill}{rgb}{0.000000,0.000000,0.000000}%
\pgfsetfillcolor{currentfill}%
\pgfsetlinewidth{1.003750pt}%
\definecolor{currentstroke}{rgb}{0.000000,0.000000,0.000000}%
\pgfsetstrokecolor{currentstroke}%
\pgfsetdash{}{0pt}%
\pgfpathmoveto{\pgfqpoint{4.842519in}{2.957274in}}%
\pgfpathcurveto{\pgfqpoint{4.853569in}{2.957274in}}{\pgfqpoint{4.864169in}{2.961664in}}{\pgfqpoint{4.871982in}{2.969478in}}%
\pgfpathcurveto{\pgfqpoint{4.879796in}{2.977292in}}{\pgfqpoint{4.884186in}{2.987891in}}{\pgfqpoint{4.884186in}{2.998941in}}%
\pgfpathcurveto{\pgfqpoint{4.884186in}{3.009991in}}{\pgfqpoint{4.879796in}{3.020590in}}{\pgfqpoint{4.871982in}{3.028403in}}%
\pgfpathcurveto{\pgfqpoint{4.864169in}{3.036217in}}{\pgfqpoint{4.853569in}{3.040607in}}{\pgfqpoint{4.842519in}{3.040607in}}%
\pgfpathcurveto{\pgfqpoint{4.831469in}{3.040607in}}{\pgfqpoint{4.820870in}{3.036217in}}{\pgfqpoint{4.813057in}{3.028403in}}%
\pgfpathcurveto{\pgfqpoint{4.805243in}{3.020590in}}{\pgfqpoint{4.800853in}{3.009991in}}{\pgfqpoint{4.800853in}{2.998941in}}%
\pgfpathcurveto{\pgfqpoint{4.800853in}{2.987891in}}{\pgfqpoint{4.805243in}{2.977292in}}{\pgfqpoint{4.813057in}{2.969478in}}%
\pgfpathcurveto{\pgfqpoint{4.820870in}{2.961664in}}{\pgfqpoint{4.831469in}{2.957274in}}{\pgfqpoint{4.842519in}{2.957274in}}%
\pgfpathclose%
\pgfusepath{stroke,fill}%
\end{pgfscope}%
\begin{pgfscope}%
\pgfpathrectangle{\pgfqpoint{0.970666in}{0.566125in}}{\pgfqpoint{5.699255in}{2.685432in}}%
\pgfusepath{clip}%
\pgfsetbuttcap%
\pgfsetroundjoin%
\definecolor{currentfill}{rgb}{0.000000,0.000000,0.000000}%
\pgfsetfillcolor{currentfill}%
\pgfsetlinewidth{1.003750pt}%
\definecolor{currentstroke}{rgb}{0.000000,0.000000,0.000000}%
\pgfsetstrokecolor{currentstroke}%
\pgfsetdash{}{0pt}%
\pgfpathmoveto{\pgfqpoint{4.702489in}{2.891999in}}%
\pgfpathcurveto{\pgfqpoint{4.713539in}{2.891999in}}{\pgfqpoint{4.724138in}{2.896389in}}{\pgfqpoint{4.731951in}{2.904202in}}%
\pgfpathcurveto{\pgfqpoint{4.739765in}{2.912016in}}{\pgfqpoint{4.744155in}{2.922615in}}{\pgfqpoint{4.744155in}{2.933665in}}%
\pgfpathcurveto{\pgfqpoint{4.744155in}{2.944715in}}{\pgfqpoint{4.739765in}{2.955314in}}{\pgfqpoint{4.731951in}{2.963128in}}%
\pgfpathcurveto{\pgfqpoint{4.724138in}{2.970942in}}{\pgfqpoint{4.713539in}{2.975332in}}{\pgfqpoint{4.702489in}{2.975332in}}%
\pgfpathcurveto{\pgfqpoint{4.691438in}{2.975332in}}{\pgfqpoint{4.680839in}{2.970942in}}{\pgfqpoint{4.673026in}{2.963128in}}%
\pgfpathcurveto{\pgfqpoint{4.665212in}{2.955314in}}{\pgfqpoint{4.660822in}{2.944715in}}{\pgfqpoint{4.660822in}{2.933665in}}%
\pgfpathcurveto{\pgfqpoint{4.660822in}{2.922615in}}{\pgfqpoint{4.665212in}{2.912016in}}{\pgfqpoint{4.673026in}{2.904202in}}%
\pgfpathcurveto{\pgfqpoint{4.680839in}{2.896389in}}{\pgfqpoint{4.691438in}{2.891999in}}{\pgfqpoint{4.702489in}{2.891999in}}%
\pgfpathclose%
\pgfusepath{stroke,fill}%
\end{pgfscope}%
\begin{pgfscope}%
\pgfpathrectangle{\pgfqpoint{0.970666in}{0.566125in}}{\pgfqpoint{5.699255in}{2.685432in}}%
\pgfusepath{clip}%
\pgfsetbuttcap%
\pgfsetroundjoin%
\definecolor{currentfill}{rgb}{0.000000,0.000000,0.000000}%
\pgfsetfillcolor{currentfill}%
\pgfsetlinewidth{1.003750pt}%
\definecolor{currentstroke}{rgb}{0.000000,0.000000,0.000000}%
\pgfsetstrokecolor{currentstroke}%
\pgfsetdash{}{0pt}%
\pgfpathmoveto{\pgfqpoint{3.946322in}{3.035605in}}%
\pgfpathcurveto{\pgfqpoint{3.957372in}{3.035605in}}{\pgfqpoint{3.967971in}{3.039995in}}{\pgfqpoint{3.975785in}{3.047808in}}%
\pgfpathcurveto{\pgfqpoint{3.983598in}{3.055622in}}{\pgfqpoint{3.987989in}{3.066221in}}{\pgfqpoint{3.987989in}{3.077271in}}%
\pgfpathcurveto{\pgfqpoint{3.987989in}{3.088321in}}{\pgfqpoint{3.983598in}{3.098920in}}{\pgfqpoint{3.975785in}{3.106734in}}%
\pgfpathcurveto{\pgfqpoint{3.967971in}{3.114548in}}{\pgfqpoint{3.957372in}{3.118938in}}{\pgfqpoint{3.946322in}{3.118938in}}%
\pgfpathcurveto{\pgfqpoint{3.935272in}{3.118938in}}{\pgfqpoint{3.924673in}{3.114548in}}{\pgfqpoint{3.916859in}{3.106734in}}%
\pgfpathcurveto{\pgfqpoint{3.909046in}{3.098920in}}{\pgfqpoint{3.904655in}{3.088321in}}{\pgfqpoint{3.904655in}{3.077271in}}%
\pgfpathcurveto{\pgfqpoint{3.904655in}{3.066221in}}{\pgfqpoint{3.909046in}{3.055622in}}{\pgfqpoint{3.916859in}{3.047808in}}%
\pgfpathcurveto{\pgfqpoint{3.924673in}{3.039995in}}{\pgfqpoint{3.935272in}{3.035605in}}{\pgfqpoint{3.946322in}{3.035605in}}%
\pgfpathclose%
\pgfusepath{stroke,fill}%
\end{pgfscope}%
\begin{pgfscope}%
\pgfpathrectangle{\pgfqpoint{0.970666in}{0.566125in}}{\pgfqpoint{5.699255in}{2.685432in}}%
\pgfusepath{clip}%
\pgfsetbuttcap%
\pgfsetroundjoin%
\definecolor{currentfill}{rgb}{0.000000,0.000000,0.000000}%
\pgfsetfillcolor{currentfill}%
\pgfsetlinewidth{1.003750pt}%
\definecolor{currentstroke}{rgb}{0.000000,0.000000,0.000000}%
\pgfsetstrokecolor{currentstroke}%
\pgfsetdash{}{0pt}%
\pgfpathmoveto{\pgfqpoint{3.554236in}{2.852833in}}%
\pgfpathcurveto{\pgfqpoint{3.565286in}{2.852833in}}{\pgfqpoint{3.575885in}{2.857224in}}{\pgfqpoint{3.583698in}{2.865037in}}%
\pgfpathcurveto{\pgfqpoint{3.591512in}{2.872851in}}{\pgfqpoint{3.595902in}{2.883450in}}{\pgfqpoint{3.595902in}{2.894500in}}%
\pgfpathcurveto{\pgfqpoint{3.595902in}{2.905550in}}{\pgfqpoint{3.591512in}{2.916149in}}{\pgfqpoint{3.583698in}{2.923963in}}%
\pgfpathcurveto{\pgfqpoint{3.575885in}{2.931776in}}{\pgfqpoint{3.565286in}{2.936167in}}{\pgfqpoint{3.554236in}{2.936167in}}%
\pgfpathcurveto{\pgfqpoint{3.543185in}{2.936167in}}{\pgfqpoint{3.532586in}{2.931776in}}{\pgfqpoint{3.524773in}{2.923963in}}%
\pgfpathcurveto{\pgfqpoint{3.516959in}{2.916149in}}{\pgfqpoint{3.512569in}{2.905550in}}{\pgfqpoint{3.512569in}{2.894500in}}%
\pgfpathcurveto{\pgfqpoint{3.512569in}{2.883450in}}{\pgfqpoint{3.516959in}{2.872851in}}{\pgfqpoint{3.524773in}{2.865037in}}%
\pgfpathcurveto{\pgfqpoint{3.532586in}{2.857224in}}{\pgfqpoint{3.543185in}{2.852833in}}{\pgfqpoint{3.554236in}{2.852833in}}%
\pgfpathclose%
\pgfusepath{stroke,fill}%
\end{pgfscope}%
\begin{pgfscope}%
\pgfpathrectangle{\pgfqpoint{0.970666in}{0.566125in}}{\pgfqpoint{5.699255in}{2.685432in}}%
\pgfusepath{clip}%
\pgfsetbuttcap%
\pgfsetroundjoin%
\definecolor{currentfill}{rgb}{0.000000,0.000000,0.000000}%
\pgfsetfillcolor{currentfill}%
\pgfsetlinewidth{1.003750pt}%
\definecolor{currentstroke}{rgb}{0.000000,0.000000,0.000000}%
\pgfsetstrokecolor{currentstroke}%
\pgfsetdash{}{0pt}%
\pgfpathmoveto{\pgfqpoint{3.526229in}{2.787558in}}%
\pgfpathcurveto{\pgfqpoint{3.537280in}{2.787558in}}{\pgfqpoint{3.547879in}{2.791948in}}{\pgfqpoint{3.555692in}{2.799762in}}%
\pgfpathcurveto{\pgfqpoint{3.563506in}{2.807575in}}{\pgfqpoint{3.567896in}{2.818174in}}{\pgfqpoint{3.567896in}{2.829225in}}%
\pgfpathcurveto{\pgfqpoint{3.567896in}{2.840275in}}{\pgfqpoint{3.563506in}{2.850874in}}{\pgfqpoint{3.555692in}{2.858687in}}%
\pgfpathcurveto{\pgfqpoint{3.547879in}{2.866501in}}{\pgfqpoint{3.537280in}{2.870891in}}{\pgfqpoint{3.526229in}{2.870891in}}%
\pgfpathcurveto{\pgfqpoint{3.515179in}{2.870891in}}{\pgfqpoint{3.504580in}{2.866501in}}{\pgfqpoint{3.496767in}{2.858687in}}%
\pgfpathcurveto{\pgfqpoint{3.488953in}{2.850874in}}{\pgfqpoint{3.484563in}{2.840275in}}{\pgfqpoint{3.484563in}{2.829225in}}%
\pgfpathcurveto{\pgfqpoint{3.484563in}{2.818174in}}{\pgfqpoint{3.488953in}{2.807575in}}{\pgfqpoint{3.496767in}{2.799762in}}%
\pgfpathcurveto{\pgfqpoint{3.504580in}{2.791948in}}{\pgfqpoint{3.515179in}{2.787558in}}{\pgfqpoint{3.526229in}{2.787558in}}%
\pgfpathclose%
\pgfusepath{stroke,fill}%
\end{pgfscope}%
\begin{pgfscope}%
\pgfpathrectangle{\pgfqpoint{0.970666in}{0.566125in}}{\pgfqpoint{5.699255in}{2.685432in}}%
\pgfusepath{clip}%
\pgfsetbuttcap%
\pgfsetroundjoin%
\definecolor{currentfill}{rgb}{0.000000,0.000000,0.000000}%
\pgfsetfillcolor{currentfill}%
\pgfsetlinewidth{1.003750pt}%
\definecolor{currentstroke}{rgb}{0.000000,0.000000,0.000000}%
\pgfsetstrokecolor{currentstroke}%
\pgfsetdash{}{0pt}%
\pgfpathmoveto{\pgfqpoint{3.162149in}{2.878944in}}%
\pgfpathcurveto{\pgfqpoint{3.173199in}{2.878944in}}{\pgfqpoint{3.183798in}{2.883334in}}{\pgfqpoint{3.191612in}{2.891147in}}%
\pgfpathcurveto{\pgfqpoint{3.199426in}{2.898961in}}{\pgfqpoint{3.203816in}{2.909560in}}{\pgfqpoint{3.203816in}{2.920610in}}%
\pgfpathcurveto{\pgfqpoint{3.203816in}{2.931660in}}{\pgfqpoint{3.199426in}{2.942259in}}{\pgfqpoint{3.191612in}{2.950073in}}%
\pgfpathcurveto{\pgfqpoint{3.183798in}{2.957887in}}{\pgfqpoint{3.173199in}{2.962277in}}{\pgfqpoint{3.162149in}{2.962277in}}%
\pgfpathcurveto{\pgfqpoint{3.151099in}{2.962277in}}{\pgfqpoint{3.140500in}{2.957887in}}{\pgfqpoint{3.132686in}{2.950073in}}%
\pgfpathcurveto{\pgfqpoint{3.124873in}{2.942259in}}{\pgfqpoint{3.120483in}{2.931660in}}{\pgfqpoint{3.120483in}{2.920610in}}%
\pgfpathcurveto{\pgfqpoint{3.120483in}{2.909560in}}{\pgfqpoint{3.124873in}{2.898961in}}{\pgfqpoint{3.132686in}{2.891147in}}%
\pgfpathcurveto{\pgfqpoint{3.140500in}{2.883334in}}{\pgfqpoint{3.151099in}{2.878944in}}{\pgfqpoint{3.162149in}{2.878944in}}%
\pgfpathclose%
\pgfusepath{stroke,fill}%
\end{pgfscope}%
\begin{pgfscope}%
\pgfpathrectangle{\pgfqpoint{0.970666in}{0.566125in}}{\pgfqpoint{5.699255in}{2.685432in}}%
\pgfusepath{clip}%
\pgfsetbuttcap%
\pgfsetroundjoin%
\definecolor{currentfill}{rgb}{0.000000,0.000000,0.000000}%
\pgfsetfillcolor{currentfill}%
\pgfsetlinewidth{1.003750pt}%
\definecolor{currentstroke}{rgb}{0.000000,0.000000,0.000000}%
\pgfsetstrokecolor{currentstroke}%
\pgfsetdash{}{0pt}%
\pgfpathmoveto{\pgfqpoint{2.602026in}{2.709227in}}%
\pgfpathcurveto{\pgfqpoint{2.613076in}{2.709227in}}{\pgfqpoint{2.623675in}{2.713618in}}{\pgfqpoint{2.631489in}{2.721431in}}%
\pgfpathcurveto{\pgfqpoint{2.639302in}{2.729245in}}{\pgfqpoint{2.643692in}{2.739844in}}{\pgfqpoint{2.643692in}{2.750894in}}%
\pgfpathcurveto{\pgfqpoint{2.643692in}{2.761944in}}{\pgfqpoint{2.639302in}{2.772543in}}{\pgfqpoint{2.631489in}{2.780357in}}%
\pgfpathcurveto{\pgfqpoint{2.623675in}{2.788170in}}{\pgfqpoint{2.613076in}{2.792561in}}{\pgfqpoint{2.602026in}{2.792561in}}%
\pgfpathcurveto{\pgfqpoint{2.590976in}{2.792561in}}{\pgfqpoint{2.580377in}{2.788170in}}{\pgfqpoint{2.572563in}{2.780357in}}%
\pgfpathcurveto{\pgfqpoint{2.564749in}{2.772543in}}{\pgfqpoint{2.560359in}{2.761944in}}{\pgfqpoint{2.560359in}{2.750894in}}%
\pgfpathcurveto{\pgfqpoint{2.560359in}{2.739844in}}{\pgfqpoint{2.564749in}{2.729245in}}{\pgfqpoint{2.572563in}{2.721431in}}%
\pgfpathcurveto{\pgfqpoint{2.580377in}{2.713618in}}{\pgfqpoint{2.590976in}{2.709227in}}{\pgfqpoint{2.602026in}{2.709227in}}%
\pgfpathclose%
\pgfusepath{stroke,fill}%
\end{pgfscope}%
\begin{pgfscope}%
\pgfpathrectangle{\pgfqpoint{0.970666in}{0.566125in}}{\pgfqpoint{5.699255in}{2.685432in}}%
\pgfusepath{clip}%
\pgfsetbuttcap%
\pgfsetroundjoin%
\definecolor{currentfill}{rgb}{0.000000,0.000000,0.000000}%
\pgfsetfillcolor{currentfill}%
\pgfsetlinewidth{1.003750pt}%
\definecolor{currentstroke}{rgb}{0.000000,0.000000,0.000000}%
\pgfsetstrokecolor{currentstroke}%
\pgfsetdash{}{0pt}%
\pgfpathmoveto{\pgfqpoint{2.377976in}{2.657007in}}%
\pgfpathcurveto{\pgfqpoint{2.389027in}{2.657007in}}{\pgfqpoint{2.399626in}{2.661397in}}{\pgfqpoint{2.407439in}{2.669211in}}%
\pgfpathcurveto{\pgfqpoint{2.415253in}{2.677024in}}{\pgfqpoint{2.419643in}{2.687624in}}{\pgfqpoint{2.419643in}{2.698674in}}%
\pgfpathcurveto{\pgfqpoint{2.419643in}{2.709724in}}{\pgfqpoint{2.415253in}{2.720323in}}{\pgfqpoint{2.407439in}{2.728136in}}%
\pgfpathcurveto{\pgfqpoint{2.399626in}{2.735950in}}{\pgfqpoint{2.389027in}{2.740340in}}{\pgfqpoint{2.377976in}{2.740340in}}%
\pgfpathcurveto{\pgfqpoint{2.366926in}{2.740340in}}{\pgfqpoint{2.356327in}{2.735950in}}{\pgfqpoint{2.348514in}{2.728136in}}%
\pgfpathcurveto{\pgfqpoint{2.340700in}{2.720323in}}{\pgfqpoint{2.336310in}{2.709724in}}{\pgfqpoint{2.336310in}{2.698674in}}%
\pgfpathcurveto{\pgfqpoint{2.336310in}{2.687624in}}{\pgfqpoint{2.340700in}{2.677024in}}{\pgfqpoint{2.348514in}{2.669211in}}%
\pgfpathcurveto{\pgfqpoint{2.356327in}{2.661397in}}{\pgfqpoint{2.366926in}{2.657007in}}{\pgfqpoint{2.377976in}{2.657007in}}%
\pgfpathclose%
\pgfusepath{stroke,fill}%
\end{pgfscope}%
\begin{pgfscope}%
\pgfpathrectangle{\pgfqpoint{0.970666in}{0.566125in}}{\pgfqpoint{5.699255in}{2.685432in}}%
\pgfusepath{clip}%
\pgfsetbuttcap%
\pgfsetroundjoin%
\definecolor{currentfill}{rgb}{0.000000,0.000000,0.000000}%
\pgfsetfillcolor{currentfill}%
\pgfsetlinewidth{1.003750pt}%
\definecolor{currentstroke}{rgb}{0.000000,0.000000,0.000000}%
\pgfsetstrokecolor{currentstroke}%
\pgfsetdash{}{0pt}%
\pgfpathmoveto{\pgfqpoint{2.209939in}{2.578676in}}%
\pgfpathcurveto{\pgfqpoint{2.220990in}{2.578676in}}{\pgfqpoint{2.231589in}{2.583067in}}{\pgfqpoint{2.239402in}{2.590880in}}%
\pgfpathcurveto{\pgfqpoint{2.247216in}{2.598694in}}{\pgfqpoint{2.251606in}{2.609293in}}{\pgfqpoint{2.251606in}{2.620343in}}%
\pgfpathcurveto{\pgfqpoint{2.251606in}{2.631393in}}{\pgfqpoint{2.247216in}{2.641992in}}{\pgfqpoint{2.239402in}{2.649806in}}%
\pgfpathcurveto{\pgfqpoint{2.231589in}{2.657619in}}{\pgfqpoint{2.220990in}{2.662010in}}{\pgfqpoint{2.209939in}{2.662010in}}%
\pgfpathcurveto{\pgfqpoint{2.198889in}{2.662010in}}{\pgfqpoint{2.188290in}{2.657619in}}{\pgfqpoint{2.180477in}{2.649806in}}%
\pgfpathcurveto{\pgfqpoint{2.172663in}{2.641992in}}{\pgfqpoint{2.168273in}{2.631393in}}{\pgfqpoint{2.168273in}{2.620343in}}%
\pgfpathcurveto{\pgfqpoint{2.168273in}{2.609293in}}{\pgfqpoint{2.172663in}{2.598694in}}{\pgfqpoint{2.180477in}{2.590880in}}%
\pgfpathcurveto{\pgfqpoint{2.188290in}{2.583067in}}{\pgfqpoint{2.198889in}{2.578676in}}{\pgfqpoint{2.209939in}{2.578676in}}%
\pgfpathclose%
\pgfusepath{stroke,fill}%
\end{pgfscope}%
\begin{pgfscope}%
\pgfpathrectangle{\pgfqpoint{0.970666in}{0.566125in}}{\pgfqpoint{5.699255in}{2.685432in}}%
\pgfusepath{clip}%
\pgfsetbuttcap%
\pgfsetroundjoin%
\definecolor{currentfill}{rgb}{0.000000,0.000000,0.000000}%
\pgfsetfillcolor{currentfill}%
\pgfsetlinewidth{1.003750pt}%
\definecolor{currentstroke}{rgb}{0.000000,0.000000,0.000000}%
\pgfsetstrokecolor{currentstroke}%
\pgfsetdash{}{0pt}%
\pgfpathmoveto{\pgfqpoint{2.153927in}{2.435070in}}%
\pgfpathcurveto{\pgfqpoint{2.164977in}{2.435070in}}{\pgfqpoint{2.175576in}{2.439461in}}{\pgfqpoint{2.183390in}{2.447274in}}%
\pgfpathcurveto{\pgfqpoint{2.191204in}{2.455088in}}{\pgfqpoint{2.195594in}{2.465687in}}{\pgfqpoint{2.195594in}{2.476737in}}%
\pgfpathcurveto{\pgfqpoint{2.195594in}{2.487787in}}{\pgfqpoint{2.191204in}{2.498386in}}{\pgfqpoint{2.183390in}{2.506200in}}%
\pgfpathcurveto{\pgfqpoint{2.175576in}{2.514014in}}{\pgfqpoint{2.164977in}{2.518404in}}{\pgfqpoint{2.153927in}{2.518404in}}%
\pgfpathcurveto{\pgfqpoint{2.142877in}{2.518404in}}{\pgfqpoint{2.132278in}{2.514014in}}{\pgfqpoint{2.124464in}{2.506200in}}%
\pgfpathcurveto{\pgfqpoint{2.116651in}{2.498386in}}{\pgfqpoint{2.112260in}{2.487787in}}{\pgfqpoint{2.112260in}{2.476737in}}%
\pgfpathcurveto{\pgfqpoint{2.112260in}{2.465687in}}{\pgfqpoint{2.116651in}{2.455088in}}{\pgfqpoint{2.124464in}{2.447274in}}%
\pgfpathcurveto{\pgfqpoint{2.132278in}{2.439461in}}{\pgfqpoint{2.142877in}{2.435070in}}{\pgfqpoint{2.153927in}{2.435070in}}%
\pgfpathclose%
\pgfusepath{stroke,fill}%
\end{pgfscope}%
\begin{pgfscope}%
\pgfpathrectangle{\pgfqpoint{0.970666in}{0.566125in}}{\pgfqpoint{5.699255in}{2.685432in}}%
\pgfusepath{clip}%
\pgfsetbuttcap%
\pgfsetroundjoin%
\definecolor{currentfill}{rgb}{0.000000,0.000000,0.000000}%
\pgfsetfillcolor{currentfill}%
\pgfsetlinewidth{1.003750pt}%
\definecolor{currentstroke}{rgb}{0.000000,0.000000,0.000000}%
\pgfsetstrokecolor{currentstroke}%
\pgfsetdash{}{0pt}%
\pgfpathmoveto{\pgfqpoint{1.985890in}{2.265354in}}%
\pgfpathcurveto{\pgfqpoint{1.996940in}{2.265354in}}{\pgfqpoint{2.007539in}{2.269745in}}{\pgfqpoint{2.015353in}{2.277558in}}%
\pgfpathcurveto{\pgfqpoint{2.023167in}{2.285372in}}{\pgfqpoint{2.027557in}{2.295971in}}{\pgfqpoint{2.027557in}{2.307021in}}%
\pgfpathcurveto{\pgfqpoint{2.027557in}{2.318071in}}{\pgfqpoint{2.023167in}{2.328670in}}{\pgfqpoint{2.015353in}{2.336484in}}%
\pgfpathcurveto{\pgfqpoint{2.007539in}{2.344297in}}{\pgfqpoint{1.996940in}{2.348688in}}{\pgfqpoint{1.985890in}{2.348688in}}%
\pgfpathcurveto{\pgfqpoint{1.974840in}{2.348688in}}{\pgfqpoint{1.964241in}{2.344297in}}{\pgfqpoint{1.956427in}{2.336484in}}%
\pgfpathcurveto{\pgfqpoint{1.948614in}{2.328670in}}{\pgfqpoint{1.944223in}{2.318071in}}{\pgfqpoint{1.944223in}{2.307021in}}%
\pgfpathcurveto{\pgfqpoint{1.944223in}{2.295971in}}{\pgfqpoint{1.948614in}{2.285372in}}{\pgfqpoint{1.956427in}{2.277558in}}%
\pgfpathcurveto{\pgfqpoint{1.964241in}{2.269745in}}{\pgfqpoint{1.974840in}{2.265354in}}{\pgfqpoint{1.985890in}{2.265354in}}%
\pgfpathclose%
\pgfusepath{stroke,fill}%
\end{pgfscope}%
\begin{pgfscope}%
\pgfpathrectangle{\pgfqpoint{0.970666in}{0.566125in}}{\pgfqpoint{5.699255in}{2.685432in}}%
\pgfusepath{clip}%
\pgfsetbuttcap%
\pgfsetroundjoin%
\definecolor{currentfill}{rgb}{0.000000,0.000000,0.000000}%
\pgfsetfillcolor{currentfill}%
\pgfsetlinewidth{1.003750pt}%
\definecolor{currentstroke}{rgb}{0.000000,0.000000,0.000000}%
\pgfsetstrokecolor{currentstroke}%
\pgfsetdash{}{0pt}%
\pgfpathmoveto{\pgfqpoint{1.677822in}{2.382850in}}%
\pgfpathcurveto{\pgfqpoint{1.688872in}{2.382850in}}{\pgfqpoint{1.699471in}{2.387240in}}{\pgfqpoint{1.707285in}{2.395054in}}%
\pgfpathcurveto{\pgfqpoint{1.715099in}{2.402868in}}{\pgfqpoint{1.719489in}{2.413467in}}{\pgfqpoint{1.719489in}{2.424517in}}%
\pgfpathcurveto{\pgfqpoint{1.719489in}{2.435567in}}{\pgfqpoint{1.715099in}{2.446166in}}{\pgfqpoint{1.707285in}{2.453980in}}%
\pgfpathcurveto{\pgfqpoint{1.699471in}{2.461793in}}{\pgfqpoint{1.688872in}{2.466183in}}{\pgfqpoint{1.677822in}{2.466183in}}%
\pgfpathcurveto{\pgfqpoint{1.666772in}{2.466183in}}{\pgfqpoint{1.656173in}{2.461793in}}{\pgfqpoint{1.648359in}{2.453980in}}%
\pgfpathcurveto{\pgfqpoint{1.640546in}{2.446166in}}{\pgfqpoint{1.636156in}{2.435567in}}{\pgfqpoint{1.636156in}{2.424517in}}%
\pgfpathcurveto{\pgfqpoint{1.636156in}{2.413467in}}{\pgfqpoint{1.640546in}{2.402868in}}{\pgfqpoint{1.648359in}{2.395054in}}%
\pgfpathcurveto{\pgfqpoint{1.656173in}{2.387240in}}{\pgfqpoint{1.666772in}{2.382850in}}{\pgfqpoint{1.677822in}{2.382850in}}%
\pgfpathclose%
\pgfusepath{stroke,fill}%
\end{pgfscope}%
\begin{pgfscope}%
\pgfpathrectangle{\pgfqpoint{0.970666in}{0.566125in}}{\pgfqpoint{5.699255in}{2.685432in}}%
\pgfusepath{clip}%
\pgfsetbuttcap%
\pgfsetroundjoin%
\definecolor{currentfill}{rgb}{0.000000,0.000000,0.000000}%
\pgfsetfillcolor{currentfill}%
\pgfsetlinewidth{1.003750pt}%
\definecolor{currentstroke}{rgb}{0.000000,0.000000,0.000000}%
\pgfsetstrokecolor{currentstroke}%
\pgfsetdash{}{0pt}%
\pgfpathmoveto{\pgfqpoint{1.397761in}{2.382850in}}%
\pgfpathcurveto{\pgfqpoint{1.408811in}{2.382850in}}{\pgfqpoint{1.419410in}{2.387240in}}{\pgfqpoint{1.427223in}{2.395054in}}%
\pgfpathcurveto{\pgfqpoint{1.435037in}{2.402868in}}{\pgfqpoint{1.439427in}{2.413467in}}{\pgfqpoint{1.439427in}{2.424517in}}%
\pgfpathcurveto{\pgfqpoint{1.439427in}{2.435567in}}{\pgfqpoint{1.435037in}{2.446166in}}{\pgfqpoint{1.427223in}{2.453980in}}%
\pgfpathcurveto{\pgfqpoint{1.419410in}{2.461793in}}{\pgfqpoint{1.408811in}{2.466183in}}{\pgfqpoint{1.397761in}{2.466183in}}%
\pgfpathcurveto{\pgfqpoint{1.386710in}{2.466183in}}{\pgfqpoint{1.376111in}{2.461793in}}{\pgfqpoint{1.368298in}{2.453980in}}%
\pgfpathcurveto{\pgfqpoint{1.360484in}{2.446166in}}{\pgfqpoint{1.356094in}{2.435567in}}{\pgfqpoint{1.356094in}{2.424517in}}%
\pgfpathcurveto{\pgfqpoint{1.356094in}{2.413467in}}{\pgfqpoint{1.360484in}{2.402868in}}{\pgfqpoint{1.368298in}{2.395054in}}%
\pgfpathcurveto{\pgfqpoint{1.376111in}{2.387240in}}{\pgfqpoint{1.386710in}{2.382850in}}{\pgfqpoint{1.397761in}{2.382850in}}%
\pgfpathclose%
\pgfusepath{stroke,fill}%
\end{pgfscope}%
\begin{pgfscope}%
\pgfpathrectangle{\pgfqpoint{0.970666in}{0.566125in}}{\pgfqpoint{5.699255in}{2.685432in}}%
\pgfusepath{clip}%
\pgfsetbuttcap%
\pgfsetroundjoin%
\definecolor{currentfill}{rgb}{0.000000,0.000000,0.000000}%
\pgfsetfillcolor{currentfill}%
\pgfsetlinewidth{1.003750pt}%
\definecolor{currentstroke}{rgb}{0.000000,0.000000,0.000000}%
\pgfsetstrokecolor{currentstroke}%
\pgfsetdash{}{0pt}%
\pgfpathmoveto{\pgfqpoint{1.285736in}{2.670062in}}%
\pgfpathcurveto{\pgfqpoint{1.296786in}{2.670062in}}{\pgfqpoint{1.307385in}{2.674452in}}{\pgfqpoint{1.315199in}{2.682266in}}%
\pgfpathcurveto{\pgfqpoint{1.323012in}{2.690080in}}{\pgfqpoint{1.327403in}{2.700679in}}{\pgfqpoint{1.327403in}{2.711729in}}%
\pgfpathcurveto{\pgfqpoint{1.327403in}{2.722779in}}{\pgfqpoint{1.323012in}{2.733378in}}{\pgfqpoint{1.315199in}{2.741192in}}%
\pgfpathcurveto{\pgfqpoint{1.307385in}{2.749005in}}{\pgfqpoint{1.296786in}{2.753395in}}{\pgfqpoint{1.285736in}{2.753395in}}%
\pgfpathcurveto{\pgfqpoint{1.274686in}{2.753395in}}{\pgfqpoint{1.264087in}{2.749005in}}{\pgfqpoint{1.256273in}{2.741192in}}%
\pgfpathcurveto{\pgfqpoint{1.248459in}{2.733378in}}{\pgfqpoint{1.244069in}{2.722779in}}{\pgfqpoint{1.244069in}{2.711729in}}%
\pgfpathcurveto{\pgfqpoint{1.244069in}{2.700679in}}{\pgfqpoint{1.248459in}{2.690080in}}{\pgfqpoint{1.256273in}{2.682266in}}%
\pgfpathcurveto{\pgfqpoint{1.264087in}{2.674452in}}{\pgfqpoint{1.274686in}{2.670062in}}{\pgfqpoint{1.285736in}{2.670062in}}%
\pgfpathclose%
\pgfusepath{stroke,fill}%
\end{pgfscope}%
\begin{pgfscope}%
\pgfpathrectangle{\pgfqpoint{0.970666in}{0.566125in}}{\pgfqpoint{5.699255in}{2.685432in}}%
\pgfusepath{clip}%
\pgfsetbuttcap%
\pgfsetroundjoin%
\definecolor{currentfill}{rgb}{0.000000,0.000000,0.000000}%
\pgfsetfillcolor{currentfill}%
\pgfsetlinewidth{1.003750pt}%
\definecolor{currentstroke}{rgb}{0.000000,0.000000,0.000000}%
\pgfsetstrokecolor{currentstroke}%
\pgfsetdash{}{0pt}%
\pgfpathmoveto{\pgfqpoint{1.229724in}{2.735338in}}%
\pgfpathcurveto{\pgfqpoint{1.240774in}{2.735338in}}{\pgfqpoint{1.251373in}{2.739728in}}{\pgfqpoint{1.259186in}{2.747541in}}%
\pgfpathcurveto{\pgfqpoint{1.267000in}{2.755355in}}{\pgfqpoint{1.271390in}{2.765954in}}{\pgfqpoint{1.271390in}{2.777004in}}%
\pgfpathcurveto{\pgfqpoint{1.271390in}{2.788054in}}{\pgfqpoint{1.267000in}{2.798653in}}{\pgfqpoint{1.259186in}{2.806467in}}%
\pgfpathcurveto{\pgfqpoint{1.251373in}{2.814281in}}{\pgfqpoint{1.240774in}{2.818671in}}{\pgfqpoint{1.229724in}{2.818671in}}%
\pgfpathcurveto{\pgfqpoint{1.218673in}{2.818671in}}{\pgfqpoint{1.208074in}{2.814281in}}{\pgfqpoint{1.200261in}{2.806467in}}%
\pgfpathcurveto{\pgfqpoint{1.192447in}{2.798653in}}{\pgfqpoint{1.188057in}{2.788054in}}{\pgfqpoint{1.188057in}{2.777004in}}%
\pgfpathcurveto{\pgfqpoint{1.188057in}{2.765954in}}{\pgfqpoint{1.192447in}{2.755355in}}{\pgfqpoint{1.200261in}{2.747541in}}%
\pgfpathcurveto{\pgfqpoint{1.208074in}{2.739728in}}{\pgfqpoint{1.218673in}{2.735338in}}{\pgfqpoint{1.229724in}{2.735338in}}%
\pgfpathclose%
\pgfusepath{stroke,fill}%
\end{pgfscope}%
\begin{pgfscope}%
\pgfpathrectangle{\pgfqpoint{0.970666in}{0.566125in}}{\pgfqpoint{5.699255in}{2.685432in}}%
\pgfusepath{clip}%
\pgfsetbuttcap%
\pgfsetroundjoin%
\definecolor{currentfill}{rgb}{0.000000,0.000000,0.000000}%
\pgfsetfillcolor{currentfill}%
\pgfsetlinewidth{1.003750pt}%
\definecolor{currentstroke}{rgb}{0.000000,0.000000,0.000000}%
\pgfsetstrokecolor{currentstroke}%
\pgfsetdash{}{0pt}%
\pgfpathmoveto{\pgfqpoint{1.369754in}{2.813668in}}%
\pgfpathcurveto{\pgfqpoint{1.380805in}{2.813668in}}{\pgfqpoint{1.391404in}{2.818058in}}{\pgfqpoint{1.399217in}{2.825872in}}%
\pgfpathcurveto{\pgfqpoint{1.407031in}{2.833686in}}{\pgfqpoint{1.411421in}{2.844285in}}{\pgfqpoint{1.411421in}{2.855335in}}%
\pgfpathcurveto{\pgfqpoint{1.411421in}{2.866385in}}{\pgfqpoint{1.407031in}{2.876984in}}{\pgfqpoint{1.399217in}{2.884798in}}%
\pgfpathcurveto{\pgfqpoint{1.391404in}{2.892611in}}{\pgfqpoint{1.380805in}{2.897001in}}{\pgfqpoint{1.369754in}{2.897001in}}%
\pgfpathcurveto{\pgfqpoint{1.358704in}{2.897001in}}{\pgfqpoint{1.348105in}{2.892611in}}{\pgfqpoint{1.340292in}{2.884798in}}%
\pgfpathcurveto{\pgfqpoint{1.332478in}{2.876984in}}{\pgfqpoint{1.328088in}{2.866385in}}{\pgfqpoint{1.328088in}{2.855335in}}%
\pgfpathcurveto{\pgfqpoint{1.328088in}{2.844285in}}{\pgfqpoint{1.332478in}{2.833686in}}{\pgfqpoint{1.340292in}{2.825872in}}%
\pgfpathcurveto{\pgfqpoint{1.348105in}{2.818058in}}{\pgfqpoint{1.358704in}{2.813668in}}{\pgfqpoint{1.369754in}{2.813668in}}%
\pgfpathclose%
\pgfusepath{stroke,fill}%
\end{pgfscope}%
\begin{pgfscope}%
\pgfpathrectangle{\pgfqpoint{0.970666in}{0.566125in}}{\pgfqpoint{5.699255in}{2.685432in}}%
\pgfusepath{clip}%
\pgfsetbuttcap%
\pgfsetroundjoin%
\definecolor{currentfill}{rgb}{0.000000,0.000000,0.000000}%
\pgfsetfillcolor{currentfill}%
\pgfsetlinewidth{1.003750pt}%
\definecolor{currentstroke}{rgb}{0.000000,0.000000,0.000000}%
\pgfsetstrokecolor{currentstroke}%
\pgfsetdash{}{0pt}%
\pgfpathmoveto{\pgfqpoint{1.733835in}{2.800613in}}%
\pgfpathcurveto{\pgfqpoint{1.744885in}{2.800613in}}{\pgfqpoint{1.755484in}{2.805003in}}{\pgfqpoint{1.763297in}{2.812817in}}%
\pgfpathcurveto{\pgfqpoint{1.771111in}{2.820630in}}{\pgfqpoint{1.775501in}{2.831229in}}{\pgfqpoint{1.775501in}{2.842280in}}%
\pgfpathcurveto{\pgfqpoint{1.775501in}{2.853330in}}{\pgfqpoint{1.771111in}{2.863929in}}{\pgfqpoint{1.763297in}{2.871742in}}%
\pgfpathcurveto{\pgfqpoint{1.755484in}{2.879556in}}{\pgfqpoint{1.744885in}{2.883946in}}{\pgfqpoint{1.733835in}{2.883946in}}%
\pgfpathcurveto{\pgfqpoint{1.722784in}{2.883946in}}{\pgfqpoint{1.712185in}{2.879556in}}{\pgfqpoint{1.704372in}{2.871742in}}%
\pgfpathcurveto{\pgfqpoint{1.696558in}{2.863929in}}{\pgfqpoint{1.692168in}{2.853330in}}{\pgfqpoint{1.692168in}{2.842280in}}%
\pgfpathcurveto{\pgfqpoint{1.692168in}{2.831229in}}{\pgfqpoint{1.696558in}{2.820630in}}{\pgfqpoint{1.704372in}{2.812817in}}%
\pgfpathcurveto{\pgfqpoint{1.712185in}{2.805003in}}{\pgfqpoint{1.722784in}{2.800613in}}{\pgfqpoint{1.733835in}{2.800613in}}%
\pgfpathclose%
\pgfusepath{stroke,fill}%
\end{pgfscope}%
\begin{pgfscope}%
\pgfpathrectangle{\pgfqpoint{0.970666in}{0.566125in}}{\pgfqpoint{5.699255in}{2.685432in}}%
\pgfusepath{clip}%
\pgfsetbuttcap%
\pgfsetroundjoin%
\definecolor{currentfill}{rgb}{0.000000,0.000000,0.000000}%
\pgfsetfillcolor{currentfill}%
\pgfsetlinewidth{1.003750pt}%
\definecolor{currentstroke}{rgb}{0.000000,0.000000,0.000000}%
\pgfsetstrokecolor{currentstroke}%
\pgfsetdash{}{0pt}%
\pgfpathmoveto{\pgfqpoint{1.845859in}{2.683117in}}%
\pgfpathcurveto{\pgfqpoint{1.856909in}{2.683117in}}{\pgfqpoint{1.867508in}{2.687507in}}{\pgfqpoint{1.875322in}{2.695321in}}%
\pgfpathcurveto{\pgfqpoint{1.883136in}{2.703135in}}{\pgfqpoint{1.887526in}{2.713734in}}{\pgfqpoint{1.887526in}{2.724784in}}%
\pgfpathcurveto{\pgfqpoint{1.887526in}{2.735834in}}{\pgfqpoint{1.883136in}{2.746433in}}{\pgfqpoint{1.875322in}{2.754247in}}%
\pgfpathcurveto{\pgfqpoint{1.867508in}{2.762060in}}{\pgfqpoint{1.856909in}{2.766450in}}{\pgfqpoint{1.845859in}{2.766450in}}%
\pgfpathcurveto{\pgfqpoint{1.834809in}{2.766450in}}{\pgfqpoint{1.824210in}{2.762060in}}{\pgfqpoint{1.816396in}{2.754247in}}%
\pgfpathcurveto{\pgfqpoint{1.808583in}{2.746433in}}{\pgfqpoint{1.804193in}{2.735834in}}{\pgfqpoint{1.804193in}{2.724784in}}%
\pgfpathcurveto{\pgfqpoint{1.804193in}{2.713734in}}{\pgfqpoint{1.808583in}{2.703135in}}{\pgfqpoint{1.816396in}{2.695321in}}%
\pgfpathcurveto{\pgfqpoint{1.824210in}{2.687507in}}{\pgfqpoint{1.834809in}{2.683117in}}{\pgfqpoint{1.845859in}{2.683117in}}%
\pgfpathclose%
\pgfusepath{stroke,fill}%
\end{pgfscope}%
\begin{pgfscope}%
\pgfpathrectangle{\pgfqpoint{0.970666in}{0.566125in}}{\pgfqpoint{5.699255in}{2.685432in}}%
\pgfusepath{clip}%
\pgfsetbuttcap%
\pgfsetroundjoin%
\definecolor{currentfill}{rgb}{0.000000,0.000000,0.000000}%
\pgfsetfillcolor{currentfill}%
\pgfsetlinewidth{1.003750pt}%
\definecolor{currentstroke}{rgb}{0.000000,0.000000,0.000000}%
\pgfsetstrokecolor{currentstroke}%
\pgfsetdash{}{0pt}%
\pgfpathmoveto{\pgfqpoint{2.377976in}{2.826723in}}%
\pgfpathcurveto{\pgfqpoint{2.389027in}{2.826723in}}{\pgfqpoint{2.399626in}{2.831113in}}{\pgfqpoint{2.407439in}{2.838927in}}%
\pgfpathcurveto{\pgfqpoint{2.415253in}{2.846741in}}{\pgfqpoint{2.419643in}{2.857340in}}{\pgfqpoint{2.419643in}{2.868390in}}%
\pgfpathcurveto{\pgfqpoint{2.419643in}{2.879440in}}{\pgfqpoint{2.415253in}{2.890039in}}{\pgfqpoint{2.407439in}{2.897853in}}%
\pgfpathcurveto{\pgfqpoint{2.399626in}{2.905666in}}{\pgfqpoint{2.389027in}{2.910056in}}{\pgfqpoint{2.377976in}{2.910056in}}%
\pgfpathcurveto{\pgfqpoint{2.366926in}{2.910056in}}{\pgfqpoint{2.356327in}{2.905666in}}{\pgfqpoint{2.348514in}{2.897853in}}%
\pgfpathcurveto{\pgfqpoint{2.340700in}{2.890039in}}{\pgfqpoint{2.336310in}{2.879440in}}{\pgfqpoint{2.336310in}{2.868390in}}%
\pgfpathcurveto{\pgfqpoint{2.336310in}{2.857340in}}{\pgfqpoint{2.340700in}{2.846741in}}{\pgfqpoint{2.348514in}{2.838927in}}%
\pgfpathcurveto{\pgfqpoint{2.356327in}{2.831113in}}{\pgfqpoint{2.366926in}{2.826723in}}{\pgfqpoint{2.377976in}{2.826723in}}%
\pgfpathclose%
\pgfusepath{stroke,fill}%
\end{pgfscope}%
\begin{pgfscope}%
\pgfpathrectangle{\pgfqpoint{0.970666in}{0.566125in}}{\pgfqpoint{5.699255in}{2.685432in}}%
\pgfusepath{clip}%
\pgfsetbuttcap%
\pgfsetroundjoin%
\definecolor{currentfill}{rgb}{0.000000,0.000000,0.000000}%
\pgfsetfillcolor{currentfill}%
\pgfsetlinewidth{1.003750pt}%
\definecolor{currentstroke}{rgb}{0.000000,0.000000,0.000000}%
\pgfsetstrokecolor{currentstroke}%
\pgfsetdash{}{0pt}%
\pgfpathmoveto{\pgfqpoint{2.461995in}{3.074770in}}%
\pgfpathcurveto{\pgfqpoint{2.473045in}{3.074770in}}{\pgfqpoint{2.483644in}{3.079160in}}{\pgfqpoint{2.491458in}{3.086974in}}%
\pgfpathcurveto{\pgfqpoint{2.499271in}{3.094787in}}{\pgfqpoint{2.503662in}{3.105386in}}{\pgfqpoint{2.503662in}{3.116437in}}%
\pgfpathcurveto{\pgfqpoint{2.503662in}{3.127487in}}{\pgfqpoint{2.499271in}{3.138086in}}{\pgfqpoint{2.491458in}{3.145899in}}%
\pgfpathcurveto{\pgfqpoint{2.483644in}{3.153713in}}{\pgfqpoint{2.473045in}{3.158103in}}{\pgfqpoint{2.461995in}{3.158103in}}%
\pgfpathcurveto{\pgfqpoint{2.450945in}{3.158103in}}{\pgfqpoint{2.440346in}{3.153713in}}{\pgfqpoint{2.432532in}{3.145899in}}%
\pgfpathcurveto{\pgfqpoint{2.424719in}{3.138086in}}{\pgfqpoint{2.420328in}{3.127487in}}{\pgfqpoint{2.420328in}{3.116437in}}%
\pgfpathcurveto{\pgfqpoint{2.420328in}{3.105386in}}{\pgfqpoint{2.424719in}{3.094787in}}{\pgfqpoint{2.432532in}{3.086974in}}%
\pgfpathcurveto{\pgfqpoint{2.440346in}{3.079160in}}{\pgfqpoint{2.450945in}{3.074770in}}{\pgfqpoint{2.461995in}{3.074770in}}%
\pgfpathclose%
\pgfusepath{stroke,fill}%
\end{pgfscope}%
\begin{pgfscope}%
\pgfpathrectangle{\pgfqpoint{0.970666in}{0.566125in}}{\pgfqpoint{5.699255in}{2.685432in}}%
\pgfusepath{clip}%
\pgfsetbuttcap%
\pgfsetroundjoin%
\definecolor{currentfill}{rgb}{0.000000,0.000000,0.000000}%
\pgfsetfillcolor{currentfill}%
\pgfsetlinewidth{1.003750pt}%
\definecolor{currentstroke}{rgb}{0.000000,0.000000,0.000000}%
\pgfsetstrokecolor{currentstroke}%
\pgfsetdash{}{0pt}%
\pgfpathmoveto{\pgfqpoint{3.078131in}{2.565621in}}%
\pgfpathcurveto{\pgfqpoint{3.089181in}{2.565621in}}{\pgfqpoint{3.099780in}{2.570012in}}{\pgfqpoint{3.107593in}{2.577825in}}%
\pgfpathcurveto{\pgfqpoint{3.115407in}{2.585639in}}{\pgfqpoint{3.119797in}{2.596238in}}{\pgfqpoint{3.119797in}{2.607288in}}%
\pgfpathcurveto{\pgfqpoint{3.119797in}{2.618338in}}{\pgfqpoint{3.115407in}{2.628937in}}{\pgfqpoint{3.107593in}{2.636751in}}%
\pgfpathcurveto{\pgfqpoint{3.099780in}{2.644564in}}{\pgfqpoint{3.089181in}{2.648955in}}{\pgfqpoint{3.078131in}{2.648955in}}%
\pgfpathcurveto{\pgfqpoint{3.067081in}{2.648955in}}{\pgfqpoint{3.056482in}{2.644564in}}{\pgfqpoint{3.048668in}{2.636751in}}%
\pgfpathcurveto{\pgfqpoint{3.040854in}{2.628937in}}{\pgfqpoint{3.036464in}{2.618338in}}{\pgfqpoint{3.036464in}{2.607288in}}%
\pgfpathcurveto{\pgfqpoint{3.036464in}{2.596238in}}{\pgfqpoint{3.040854in}{2.585639in}}{\pgfqpoint{3.048668in}{2.577825in}}%
\pgfpathcurveto{\pgfqpoint{3.056482in}{2.570012in}}{\pgfqpoint{3.067081in}{2.565621in}}{\pgfqpoint{3.078131in}{2.565621in}}%
\pgfpathclose%
\pgfusepath{stroke,fill}%
\end{pgfscope}%
\begin{pgfscope}%
\pgfpathrectangle{\pgfqpoint{0.970666in}{0.566125in}}{\pgfqpoint{5.699255in}{2.685432in}}%
\pgfusepath{clip}%
\pgfsetbuttcap%
\pgfsetroundjoin%
\definecolor{currentfill}{rgb}{0.000000,0.000000,0.000000}%
\pgfsetfillcolor{currentfill}%
\pgfsetlinewidth{1.003750pt}%
\definecolor{currentstroke}{rgb}{0.000000,0.000000,0.000000}%
\pgfsetstrokecolor{currentstroke}%
\pgfsetdash{}{0pt}%
\pgfpathmoveto{\pgfqpoint{3.890310in}{2.696172in}}%
\pgfpathcurveto{\pgfqpoint{3.901360in}{2.696172in}}{\pgfqpoint{3.911959in}{2.700563in}}{\pgfqpoint{3.919772in}{2.708376in}}%
\pgfpathcurveto{\pgfqpoint{3.927586in}{2.716190in}}{\pgfqpoint{3.931976in}{2.726789in}}{\pgfqpoint{3.931976in}{2.737839in}}%
\pgfpathcurveto{\pgfqpoint{3.931976in}{2.748889in}}{\pgfqpoint{3.927586in}{2.759488in}}{\pgfqpoint{3.919772in}{2.767302in}}%
\pgfpathcurveto{\pgfqpoint{3.911959in}{2.775115in}}{\pgfqpoint{3.901360in}{2.779506in}}{\pgfqpoint{3.890310in}{2.779506in}}%
\pgfpathcurveto{\pgfqpoint{3.879259in}{2.779506in}}{\pgfqpoint{3.868660in}{2.775115in}}{\pgfqpoint{3.860847in}{2.767302in}}%
\pgfpathcurveto{\pgfqpoint{3.853033in}{2.759488in}}{\pgfqpoint{3.848643in}{2.748889in}}{\pgfqpoint{3.848643in}{2.737839in}}%
\pgfpathcurveto{\pgfqpoint{3.848643in}{2.726789in}}{\pgfqpoint{3.853033in}{2.716190in}}{\pgfqpoint{3.860847in}{2.708376in}}%
\pgfpathcurveto{\pgfqpoint{3.868660in}{2.700563in}}{\pgfqpoint{3.879259in}{2.696172in}}{\pgfqpoint{3.890310in}{2.696172in}}%
\pgfpathclose%
\pgfusepath{stroke,fill}%
\end{pgfscope}%
\begin{pgfscope}%
\pgfpathrectangle{\pgfqpoint{0.970666in}{0.566125in}}{\pgfqpoint{5.699255in}{2.685432in}}%
\pgfusepath{clip}%
\pgfsetbuttcap%
\pgfsetroundjoin%
\definecolor{currentfill}{rgb}{0.000000,0.000000,0.000000}%
\pgfsetfillcolor{currentfill}%
\pgfsetlinewidth{1.003750pt}%
\definecolor{currentstroke}{rgb}{0.000000,0.000000,0.000000}%
\pgfsetstrokecolor{currentstroke}%
\pgfsetdash{}{0pt}%
\pgfpathmoveto{\pgfqpoint{3.806291in}{1.821481in}}%
\pgfpathcurveto{\pgfqpoint{3.817341in}{1.821481in}}{\pgfqpoint{3.827940in}{1.825871in}}{\pgfqpoint{3.835754in}{1.833685in}}%
\pgfpathcurveto{\pgfqpoint{3.843568in}{1.841499in}}{\pgfqpoint{3.847958in}{1.852098in}}{\pgfqpoint{3.847958in}{1.863148in}}%
\pgfpathcurveto{\pgfqpoint{3.847958in}{1.874198in}}{\pgfqpoint{3.843568in}{1.884797in}}{\pgfqpoint{3.835754in}{1.892611in}}%
\pgfpathcurveto{\pgfqpoint{3.827940in}{1.900424in}}{\pgfqpoint{3.817341in}{1.904815in}}{\pgfqpoint{3.806291in}{1.904815in}}%
\pgfpathcurveto{\pgfqpoint{3.795241in}{1.904815in}}{\pgfqpoint{3.784642in}{1.900424in}}{\pgfqpoint{3.776828in}{1.892611in}}%
\pgfpathcurveto{\pgfqpoint{3.769015in}{1.884797in}}{\pgfqpoint{3.764624in}{1.874198in}}{\pgfqpoint{3.764624in}{1.863148in}}%
\pgfpathcurveto{\pgfqpoint{3.764624in}{1.852098in}}{\pgfqpoint{3.769015in}{1.841499in}}{\pgfqpoint{3.776828in}{1.833685in}}%
\pgfpathcurveto{\pgfqpoint{3.784642in}{1.825871in}}{\pgfqpoint{3.795241in}{1.821481in}}{\pgfqpoint{3.806291in}{1.821481in}}%
\pgfpathclose%
\pgfusepath{stroke,fill}%
\end{pgfscope}%
\begin{pgfscope}%
\pgfpathrectangle{\pgfqpoint{0.970666in}{0.566125in}}{\pgfqpoint{5.699255in}{2.685432in}}%
\pgfusepath{clip}%
\pgfsetbuttcap%
\pgfsetroundjoin%
\definecolor{currentfill}{rgb}{0.000000,0.000000,0.000000}%
\pgfsetfillcolor{currentfill}%
\pgfsetlinewidth{1.003750pt}%
\definecolor{currentstroke}{rgb}{0.000000,0.000000,0.000000}%
\pgfsetstrokecolor{currentstroke}%
\pgfsetdash{}{0pt}%
\pgfpathmoveto{\pgfqpoint{3.190155in}{1.873702in}}%
\pgfpathcurveto{\pgfqpoint{3.201206in}{1.873702in}}{\pgfqpoint{3.211805in}{1.878092in}}{\pgfqpoint{3.219618in}{1.885905in}}%
\pgfpathcurveto{\pgfqpoint{3.227432in}{1.893719in}}{\pgfqpoint{3.231822in}{1.904318in}}{\pgfqpoint{3.231822in}{1.915368in}}%
\pgfpathcurveto{\pgfqpoint{3.231822in}{1.926418in}}{\pgfqpoint{3.227432in}{1.937017in}}{\pgfqpoint{3.219618in}{1.944831in}}%
\pgfpathcurveto{\pgfqpoint{3.211805in}{1.952645in}}{\pgfqpoint{3.201206in}{1.957035in}}{\pgfqpoint{3.190155in}{1.957035in}}%
\pgfpathcurveto{\pgfqpoint{3.179105in}{1.957035in}}{\pgfqpoint{3.168506in}{1.952645in}}{\pgfqpoint{3.160693in}{1.944831in}}%
\pgfpathcurveto{\pgfqpoint{3.152879in}{1.937017in}}{\pgfqpoint{3.148489in}{1.926418in}}{\pgfqpoint{3.148489in}{1.915368in}}%
\pgfpathcurveto{\pgfqpoint{3.148489in}{1.904318in}}{\pgfqpoint{3.152879in}{1.893719in}}{\pgfqpoint{3.160693in}{1.885905in}}%
\pgfpathcurveto{\pgfqpoint{3.168506in}{1.878092in}}{\pgfqpoint{3.179105in}{1.873702in}}{\pgfqpoint{3.190155in}{1.873702in}}%
\pgfpathclose%
\pgfusepath{stroke,fill}%
\end{pgfscope}%
\begin{pgfscope}%
\pgfpathrectangle{\pgfqpoint{0.970666in}{0.566125in}}{\pgfqpoint{5.699255in}{2.685432in}}%
\pgfusepath{clip}%
\pgfsetbuttcap%
\pgfsetroundjoin%
\definecolor{currentfill}{rgb}{0.000000,0.000000,0.000000}%
\pgfsetfillcolor{currentfill}%
\pgfsetlinewidth{1.003750pt}%
\definecolor{currentstroke}{rgb}{0.000000,0.000000,0.000000}%
\pgfsetstrokecolor{currentstroke}%
\pgfsetdash{}{0pt}%
\pgfpathmoveto{\pgfqpoint{2.938100in}{1.808426in}}%
\pgfpathcurveto{\pgfqpoint{2.949150in}{1.808426in}}{\pgfqpoint{2.959749in}{1.812816in}}{\pgfqpoint{2.967563in}{1.820630in}}%
\pgfpathcurveto{\pgfqpoint{2.975376in}{1.828444in}}{\pgfqpoint{2.979767in}{1.839043in}}{\pgfqpoint{2.979767in}{1.850093in}}%
\pgfpathcurveto{\pgfqpoint{2.979767in}{1.861143in}}{\pgfqpoint{2.975376in}{1.871742in}}{\pgfqpoint{2.967563in}{1.879556in}}%
\pgfpathcurveto{\pgfqpoint{2.959749in}{1.887369in}}{\pgfqpoint{2.949150in}{1.891759in}}{\pgfqpoint{2.938100in}{1.891759in}}%
\pgfpathcurveto{\pgfqpoint{2.927050in}{1.891759in}}{\pgfqpoint{2.916451in}{1.887369in}}{\pgfqpoint{2.908637in}{1.879556in}}%
\pgfpathcurveto{\pgfqpoint{2.900823in}{1.871742in}}{\pgfqpoint{2.896433in}{1.861143in}}{\pgfqpoint{2.896433in}{1.850093in}}%
\pgfpathcurveto{\pgfqpoint{2.896433in}{1.839043in}}{\pgfqpoint{2.900823in}{1.828444in}}{\pgfqpoint{2.908637in}{1.820630in}}%
\pgfpathcurveto{\pgfqpoint{2.916451in}{1.812816in}}{\pgfqpoint{2.927050in}{1.808426in}}{\pgfqpoint{2.938100in}{1.808426in}}%
\pgfpathclose%
\pgfusepath{stroke,fill}%
\end{pgfscope}%
\begin{pgfscope}%
\pgfpathrectangle{\pgfqpoint{0.970666in}{0.566125in}}{\pgfqpoint{5.699255in}{2.685432in}}%
\pgfusepath{clip}%
\pgfsetbuttcap%
\pgfsetroundjoin%
\definecolor{currentfill}{rgb}{0.000000,0.000000,0.000000}%
\pgfsetfillcolor{currentfill}%
\pgfsetlinewidth{1.003750pt}%
\definecolor{currentstroke}{rgb}{0.000000,0.000000,0.000000}%
\pgfsetstrokecolor{currentstroke}%
\pgfsetdash{}{0pt}%
\pgfpathmoveto{\pgfqpoint{2.910094in}{1.743151in}}%
\pgfpathcurveto{\pgfqpoint{2.921144in}{1.743151in}}{\pgfqpoint{2.931743in}{1.747541in}}{\pgfqpoint{2.939556in}{1.755355in}}%
\pgfpathcurveto{\pgfqpoint{2.947370in}{1.763168in}}{\pgfqpoint{2.951760in}{1.773767in}}{\pgfqpoint{2.951760in}{1.784817in}}%
\pgfpathcurveto{\pgfqpoint{2.951760in}{1.795867in}}{\pgfqpoint{2.947370in}{1.806466in}}{\pgfqpoint{2.939556in}{1.814280in}}%
\pgfpathcurveto{\pgfqpoint{2.931743in}{1.822094in}}{\pgfqpoint{2.921144in}{1.826484in}}{\pgfqpoint{2.910094in}{1.826484in}}%
\pgfpathcurveto{\pgfqpoint{2.899044in}{1.826484in}}{\pgfqpoint{2.888445in}{1.822094in}}{\pgfqpoint{2.880631in}{1.814280in}}%
\pgfpathcurveto{\pgfqpoint{2.872817in}{1.806466in}}{\pgfqpoint{2.868427in}{1.795867in}}{\pgfqpoint{2.868427in}{1.784817in}}%
\pgfpathcurveto{\pgfqpoint{2.868427in}{1.773767in}}{\pgfqpoint{2.872817in}{1.763168in}}{\pgfqpoint{2.880631in}{1.755355in}}%
\pgfpathcurveto{\pgfqpoint{2.888445in}{1.747541in}}{\pgfqpoint{2.899044in}{1.743151in}}{\pgfqpoint{2.910094in}{1.743151in}}%
\pgfpathclose%
\pgfusepath{stroke,fill}%
\end{pgfscope}%
\begin{pgfscope}%
\pgfpathrectangle{\pgfqpoint{0.970666in}{0.566125in}}{\pgfqpoint{5.699255in}{2.685432in}}%
\pgfusepath{clip}%
\pgfsetbuttcap%
\pgfsetroundjoin%
\definecolor{currentfill}{rgb}{0.000000,0.000000,0.000000}%
\pgfsetfillcolor{currentfill}%
\pgfsetlinewidth{1.003750pt}%
\definecolor{currentstroke}{rgb}{0.000000,0.000000,0.000000}%
\pgfsetstrokecolor{currentstroke}%
\pgfsetdash{}{0pt}%
\pgfpathmoveto{\pgfqpoint{4.562458in}{1.677875in}}%
\pgfpathcurveto{\pgfqpoint{4.573508in}{1.677875in}}{\pgfqpoint{4.584107in}{1.682265in}}{\pgfqpoint{4.591920in}{1.690079in}}%
\pgfpathcurveto{\pgfqpoint{4.599734in}{1.697893in}}{\pgfqpoint{4.604124in}{1.708492in}}{\pgfqpoint{4.604124in}{1.719542in}}%
\pgfpathcurveto{\pgfqpoint{4.604124in}{1.730592in}}{\pgfqpoint{4.599734in}{1.741191in}}{\pgfqpoint{4.591920in}{1.749005in}}%
\pgfpathcurveto{\pgfqpoint{4.584107in}{1.756818in}}{\pgfqpoint{4.573508in}{1.761209in}}{\pgfqpoint{4.562458in}{1.761209in}}%
\pgfpathcurveto{\pgfqpoint{4.551408in}{1.761209in}}{\pgfqpoint{4.540808in}{1.756818in}}{\pgfqpoint{4.532995in}{1.749005in}}%
\pgfpathcurveto{\pgfqpoint{4.525181in}{1.741191in}}{\pgfqpoint{4.520791in}{1.730592in}}{\pgfqpoint{4.520791in}{1.719542in}}%
\pgfpathcurveto{\pgfqpoint{4.520791in}{1.708492in}}{\pgfqpoint{4.525181in}{1.697893in}}{\pgfqpoint{4.532995in}{1.690079in}}%
\pgfpathcurveto{\pgfqpoint{4.540808in}{1.682265in}}{\pgfqpoint{4.551408in}{1.677875in}}{\pgfqpoint{4.562458in}{1.677875in}}%
\pgfpathclose%
\pgfusepath{stroke,fill}%
\end{pgfscope}%
\begin{pgfscope}%
\pgfpathrectangle{\pgfqpoint{0.970666in}{0.566125in}}{\pgfqpoint{5.699255in}{2.685432in}}%
\pgfusepath{clip}%
\pgfsetbuttcap%
\pgfsetroundjoin%
\definecolor{currentfill}{rgb}{0.000000,0.000000,0.000000}%
\pgfsetfillcolor{currentfill}%
\pgfsetlinewidth{1.003750pt}%
\definecolor{currentstroke}{rgb}{0.000000,0.000000,0.000000}%
\pgfsetstrokecolor{currentstroke}%
\pgfsetdash{}{0pt}%
\pgfpathmoveto{\pgfqpoint{5.794729in}{1.965087in}}%
\pgfpathcurveto{\pgfqpoint{5.805779in}{1.965087in}}{\pgfqpoint{5.816378in}{1.969477in}}{\pgfqpoint{5.824192in}{1.977291in}}%
\pgfpathcurveto{\pgfqpoint{5.832005in}{1.985105in}}{\pgfqpoint{5.836396in}{1.995704in}}{\pgfqpoint{5.836396in}{2.006754in}}%
\pgfpathcurveto{\pgfqpoint{5.836396in}{2.017804in}}{\pgfqpoint{5.832005in}{2.028403in}}{\pgfqpoint{5.824192in}{2.036217in}}%
\pgfpathcurveto{\pgfqpoint{5.816378in}{2.044030in}}{\pgfqpoint{5.805779in}{2.048421in}}{\pgfqpoint{5.794729in}{2.048421in}}%
\pgfpathcurveto{\pgfqpoint{5.783679in}{2.048421in}}{\pgfqpoint{5.773080in}{2.044030in}}{\pgfqpoint{5.765266in}{2.036217in}}%
\pgfpathcurveto{\pgfqpoint{5.757453in}{2.028403in}}{\pgfqpoint{5.753062in}{2.017804in}}{\pgfqpoint{5.753062in}{2.006754in}}%
\pgfpathcurveto{\pgfqpoint{5.753062in}{1.995704in}}{\pgfqpoint{5.757453in}{1.985105in}}{\pgfqpoint{5.765266in}{1.977291in}}%
\pgfpathcurveto{\pgfqpoint{5.773080in}{1.969477in}}{\pgfqpoint{5.783679in}{1.965087in}}{\pgfqpoint{5.794729in}{1.965087in}}%
\pgfpathclose%
\pgfusepath{stroke,fill}%
\end{pgfscope}%
\begin{pgfscope}%
\pgfpathrectangle{\pgfqpoint{0.970666in}{0.566125in}}{\pgfqpoint{5.699255in}{2.685432in}}%
\pgfusepath{clip}%
\pgfsetbuttcap%
\pgfsetroundjoin%
\definecolor{currentfill}{rgb}{0.000000,0.000000,0.000000}%
\pgfsetfillcolor{currentfill}%
\pgfsetlinewidth{1.003750pt}%
\definecolor{currentstroke}{rgb}{0.000000,0.000000,0.000000}%
\pgfsetstrokecolor{currentstroke}%
\pgfsetdash{}{0pt}%
\pgfpathmoveto{\pgfqpoint{6.214822in}{1.625655in}}%
\pgfpathcurveto{\pgfqpoint{6.225872in}{1.625655in}}{\pgfqpoint{6.236471in}{1.630045in}}{\pgfqpoint{6.244284in}{1.637859in}}%
\pgfpathcurveto{\pgfqpoint{6.252098in}{1.645672in}}{\pgfqpoint{6.256488in}{1.656271in}}{\pgfqpoint{6.256488in}{1.667321in}}%
\pgfpathcurveto{\pgfqpoint{6.256488in}{1.678372in}}{\pgfqpoint{6.252098in}{1.688971in}}{\pgfqpoint{6.244284in}{1.696784in}}%
\pgfpathcurveto{\pgfqpoint{6.236471in}{1.704598in}}{\pgfqpoint{6.225872in}{1.708988in}}{\pgfqpoint{6.214822in}{1.708988in}}%
\pgfpathcurveto{\pgfqpoint{6.203772in}{1.708988in}}{\pgfqpoint{6.193172in}{1.704598in}}{\pgfqpoint{6.185359in}{1.696784in}}%
\pgfpathcurveto{\pgfqpoint{6.177545in}{1.688971in}}{\pgfqpoint{6.173155in}{1.678372in}}{\pgfqpoint{6.173155in}{1.667321in}}%
\pgfpathcurveto{\pgfqpoint{6.173155in}{1.656271in}}{\pgfqpoint{6.177545in}{1.645672in}}{\pgfqpoint{6.185359in}{1.637859in}}%
\pgfpathcurveto{\pgfqpoint{6.193172in}{1.630045in}}{\pgfqpoint{6.203772in}{1.625655in}}{\pgfqpoint{6.214822in}{1.625655in}}%
\pgfpathclose%
\pgfusepath{stroke,fill}%
\end{pgfscope}%
\begin{pgfscope}%
\pgfpathrectangle{\pgfqpoint{0.970666in}{0.566125in}}{\pgfqpoint{5.699255in}{2.685432in}}%
\pgfusepath{clip}%
\pgfsetbuttcap%
\pgfsetroundjoin%
\definecolor{currentfill}{rgb}{0.000000,0.000000,0.000000}%
\pgfsetfillcolor{currentfill}%
\pgfsetlinewidth{1.003750pt}%
\definecolor{currentstroke}{rgb}{0.000000,0.000000,0.000000}%
\pgfsetstrokecolor{currentstroke}%
\pgfsetdash{}{0pt}%
\pgfpathmoveto{\pgfqpoint{5.514667in}{1.181782in}}%
\pgfpathcurveto{\pgfqpoint{5.525718in}{1.181782in}}{\pgfqpoint{5.536317in}{1.186172in}}{\pgfqpoint{5.544130in}{1.193986in}}%
\pgfpathcurveto{\pgfqpoint{5.551944in}{1.201799in}}{\pgfqpoint{5.556334in}{1.212398in}}{\pgfqpoint{5.556334in}{1.223448in}}%
\pgfpathcurveto{\pgfqpoint{5.556334in}{1.234499in}}{\pgfqpoint{5.551944in}{1.245098in}}{\pgfqpoint{5.544130in}{1.252911in}}%
\pgfpathcurveto{\pgfqpoint{5.536317in}{1.260725in}}{\pgfqpoint{5.525718in}{1.265115in}}{\pgfqpoint{5.514667in}{1.265115in}}%
\pgfpathcurveto{\pgfqpoint{5.503617in}{1.265115in}}{\pgfqpoint{5.493018in}{1.260725in}}{\pgfqpoint{5.485205in}{1.252911in}}%
\pgfpathcurveto{\pgfqpoint{5.477391in}{1.245098in}}{\pgfqpoint{5.473001in}{1.234499in}}{\pgfqpoint{5.473001in}{1.223448in}}%
\pgfpathcurveto{\pgfqpoint{5.473001in}{1.212398in}}{\pgfqpoint{5.477391in}{1.201799in}}{\pgfqpoint{5.485205in}{1.193986in}}%
\pgfpathcurveto{\pgfqpoint{5.493018in}{1.186172in}}{\pgfqpoint{5.503617in}{1.181782in}}{\pgfqpoint{5.514667in}{1.181782in}}%
\pgfpathclose%
\pgfusepath{stroke,fill}%
\end{pgfscope}%
\begin{pgfscope}%
\pgfpathrectangle{\pgfqpoint{0.970666in}{0.566125in}}{\pgfqpoint{5.699255in}{2.685432in}}%
\pgfusepath{clip}%
\pgfsetbuttcap%
\pgfsetroundjoin%
\definecolor{currentfill}{rgb}{0.000000,0.000000,0.000000}%
\pgfsetfillcolor{currentfill}%
\pgfsetlinewidth{1.003750pt}%
\definecolor{currentstroke}{rgb}{0.000000,0.000000,0.000000}%
\pgfsetstrokecolor{currentstroke}%
\pgfsetdash{}{0pt}%
\pgfpathmoveto{\pgfqpoint{5.626692in}{1.038176in}}%
\pgfpathcurveto{\pgfqpoint{5.637742in}{1.038176in}}{\pgfqpoint{5.648341in}{1.042566in}}{\pgfqpoint{5.656155in}{1.050380in}}%
\pgfpathcurveto{\pgfqpoint{5.663968in}{1.058193in}}{\pgfqpoint{5.668359in}{1.068792in}}{\pgfqpoint{5.668359in}{1.079842in}}%
\pgfpathcurveto{\pgfqpoint{5.668359in}{1.090893in}}{\pgfqpoint{5.663968in}{1.101492in}}{\pgfqpoint{5.656155in}{1.109305in}}%
\pgfpathcurveto{\pgfqpoint{5.648341in}{1.117119in}}{\pgfqpoint{5.637742in}{1.121509in}}{\pgfqpoint{5.626692in}{1.121509in}}%
\pgfpathcurveto{\pgfqpoint{5.615642in}{1.121509in}}{\pgfqpoint{5.605043in}{1.117119in}}{\pgfqpoint{5.597229in}{1.109305in}}%
\pgfpathcurveto{\pgfqpoint{5.589416in}{1.101492in}}{\pgfqpoint{5.585025in}{1.090893in}}{\pgfqpoint{5.585025in}{1.079842in}}%
\pgfpathcurveto{\pgfqpoint{5.585025in}{1.068792in}}{\pgfqpoint{5.589416in}{1.058193in}}{\pgfqpoint{5.597229in}{1.050380in}}%
\pgfpathcurveto{\pgfqpoint{5.605043in}{1.042566in}}{\pgfqpoint{5.615642in}{1.038176in}}{\pgfqpoint{5.626692in}{1.038176in}}%
\pgfpathclose%
\pgfusepath{stroke,fill}%
\end{pgfscope}%
\begin{pgfscope}%
\pgfpathrectangle{\pgfqpoint{0.970666in}{0.566125in}}{\pgfqpoint{5.699255in}{2.685432in}}%
\pgfusepath{clip}%
\pgfsetbuttcap%
\pgfsetroundjoin%
\definecolor{currentfill}{rgb}{0.000000,0.000000,0.000000}%
\pgfsetfillcolor{currentfill}%
\pgfsetlinewidth{1.003750pt}%
\definecolor{currentstroke}{rgb}{0.000000,0.000000,0.000000}%
\pgfsetstrokecolor{currentstroke}%
\pgfsetdash{}{0pt}%
\pgfpathmoveto{\pgfqpoint{5.318624in}{0.985955in}}%
\pgfpathcurveto{\pgfqpoint{5.329674in}{0.985955in}}{\pgfqpoint{5.340273in}{0.990346in}}{\pgfqpoint{5.348087in}{0.998159in}}%
\pgfpathcurveto{\pgfqpoint{5.355901in}{1.005973in}}{\pgfqpoint{5.360291in}{1.016572in}}{\pgfqpoint{5.360291in}{1.027622in}}%
\pgfpathcurveto{\pgfqpoint{5.360291in}{1.038672in}}{\pgfqpoint{5.355901in}{1.049271in}}{\pgfqpoint{5.348087in}{1.057085in}}%
\pgfpathcurveto{\pgfqpoint{5.340273in}{1.064898in}}{\pgfqpoint{5.329674in}{1.069289in}}{\pgfqpoint{5.318624in}{1.069289in}}%
\pgfpathcurveto{\pgfqpoint{5.307574in}{1.069289in}}{\pgfqpoint{5.296975in}{1.064898in}}{\pgfqpoint{5.289161in}{1.057085in}}%
\pgfpathcurveto{\pgfqpoint{5.281348in}{1.049271in}}{\pgfqpoint{5.276958in}{1.038672in}}{\pgfqpoint{5.276958in}{1.027622in}}%
\pgfpathcurveto{\pgfqpoint{5.276958in}{1.016572in}}{\pgfqpoint{5.281348in}{1.005973in}}{\pgfqpoint{5.289161in}{0.998159in}}%
\pgfpathcurveto{\pgfqpoint{5.296975in}{0.990346in}}{\pgfqpoint{5.307574in}{0.985955in}}{\pgfqpoint{5.318624in}{0.985955in}}%
\pgfpathclose%
\pgfusepath{stroke,fill}%
\end{pgfscope}%
\begin{pgfscope}%
\pgfpathrectangle{\pgfqpoint{0.970666in}{0.566125in}}{\pgfqpoint{5.699255in}{2.685432in}}%
\pgfusepath{clip}%
\pgfsetbuttcap%
\pgfsetroundjoin%
\definecolor{currentfill}{rgb}{0.000000,0.000000,0.000000}%
\pgfsetfillcolor{currentfill}%
\pgfsetlinewidth{1.003750pt}%
\definecolor{currentstroke}{rgb}{0.000000,0.000000,0.000000}%
\pgfsetstrokecolor{currentstroke}%
\pgfsetdash{}{0pt}%
\pgfpathmoveto{\pgfqpoint{5.262612in}{0.946790in}}%
\pgfpathcurveto{\pgfqpoint{5.273662in}{0.946790in}}{\pgfqpoint{5.284261in}{0.951180in}}{\pgfqpoint{5.292075in}{0.958994in}}%
\pgfpathcurveto{\pgfqpoint{5.299888in}{0.966808in}}{\pgfqpoint{5.304279in}{0.977407in}}{\pgfqpoint{5.304279in}{0.988457in}}%
\pgfpathcurveto{\pgfqpoint{5.304279in}{0.999507in}}{\pgfqpoint{5.299888in}{1.010106in}}{\pgfqpoint{5.292075in}{1.017920in}}%
\pgfpathcurveto{\pgfqpoint{5.284261in}{1.025733in}}{\pgfqpoint{5.273662in}{1.030123in}}{\pgfqpoint{5.262612in}{1.030123in}}%
\pgfpathcurveto{\pgfqpoint{5.251562in}{1.030123in}}{\pgfqpoint{5.240963in}{1.025733in}}{\pgfqpoint{5.233149in}{1.017920in}}%
\pgfpathcurveto{\pgfqpoint{5.225335in}{1.010106in}}{\pgfqpoint{5.220945in}{0.999507in}}{\pgfqpoint{5.220945in}{0.988457in}}%
\pgfpathcurveto{\pgfqpoint{5.220945in}{0.977407in}}{\pgfqpoint{5.225335in}{0.966808in}}{\pgfqpoint{5.233149in}{0.958994in}}%
\pgfpathcurveto{\pgfqpoint{5.240963in}{0.951180in}}{\pgfqpoint{5.251562in}{0.946790in}}{\pgfqpoint{5.262612in}{0.946790in}}%
\pgfpathclose%
\pgfusepath{stroke,fill}%
\end{pgfscope}%
\begin{pgfscope}%
\pgfpathrectangle{\pgfqpoint{0.970666in}{0.566125in}}{\pgfqpoint{5.699255in}{2.685432in}}%
\pgfusepath{clip}%
\pgfsetbuttcap%
\pgfsetroundjoin%
\definecolor{currentfill}{rgb}{0.000000,0.000000,0.000000}%
\pgfsetfillcolor{currentfill}%
\pgfsetlinewidth{1.003750pt}%
\definecolor{currentstroke}{rgb}{0.000000,0.000000,0.000000}%
\pgfsetstrokecolor{currentstroke}%
\pgfsetdash{}{0pt}%
\pgfpathmoveto{\pgfqpoint{5.234606in}{0.933735in}}%
\pgfpathcurveto{\pgfqpoint{5.245656in}{0.933735in}}{\pgfqpoint{5.256255in}{0.938125in}}{\pgfqpoint{5.264068in}{0.945939in}}%
\pgfpathcurveto{\pgfqpoint{5.271882in}{0.953753in}}{\pgfqpoint{5.276272in}{0.964352in}}{\pgfqpoint{5.276272in}{0.975402in}}%
\pgfpathcurveto{\pgfqpoint{5.276272in}{0.986452in}}{\pgfqpoint{5.271882in}{0.997051in}}{\pgfqpoint{5.264068in}{1.004864in}}%
\pgfpathcurveto{\pgfqpoint{5.256255in}{1.012678in}}{\pgfqpoint{5.245656in}{1.017068in}}{\pgfqpoint{5.234606in}{1.017068in}}%
\pgfpathcurveto{\pgfqpoint{5.223556in}{1.017068in}}{\pgfqpoint{5.212957in}{1.012678in}}{\pgfqpoint{5.205143in}{1.004864in}}%
\pgfpathcurveto{\pgfqpoint{5.197329in}{0.997051in}}{\pgfqpoint{5.192939in}{0.986452in}}{\pgfqpoint{5.192939in}{0.975402in}}%
\pgfpathcurveto{\pgfqpoint{5.192939in}{0.964352in}}{\pgfqpoint{5.197329in}{0.953753in}}{\pgfqpoint{5.205143in}{0.945939in}}%
\pgfpathcurveto{\pgfqpoint{5.212957in}{0.938125in}}{\pgfqpoint{5.223556in}{0.933735in}}{\pgfqpoint{5.234606in}{0.933735in}}%
\pgfpathclose%
\pgfusepath{stroke,fill}%
\end{pgfscope}%
\begin{pgfscope}%
\pgfpathrectangle{\pgfqpoint{0.970666in}{0.566125in}}{\pgfqpoint{5.699255in}{2.685432in}}%
\pgfusepath{clip}%
\pgfsetbuttcap%
\pgfsetroundjoin%
\definecolor{currentfill}{rgb}{0.000000,0.000000,0.000000}%
\pgfsetfillcolor{currentfill}%
\pgfsetlinewidth{1.003750pt}%
\definecolor{currentstroke}{rgb}{0.000000,0.000000,0.000000}%
\pgfsetstrokecolor{currentstroke}%
\pgfsetdash{}{0pt}%
\pgfpathmoveto{\pgfqpoint{5.262612in}{1.038176in}}%
\pgfpathcurveto{\pgfqpoint{5.273662in}{1.038176in}}{\pgfqpoint{5.284261in}{1.042566in}}{\pgfqpoint{5.292075in}{1.050380in}}%
\pgfpathcurveto{\pgfqpoint{5.299888in}{1.058193in}}{\pgfqpoint{5.304279in}{1.068792in}}{\pgfqpoint{5.304279in}{1.079842in}}%
\pgfpathcurveto{\pgfqpoint{5.304279in}{1.090893in}}{\pgfqpoint{5.299888in}{1.101492in}}{\pgfqpoint{5.292075in}{1.109305in}}%
\pgfpathcurveto{\pgfqpoint{5.284261in}{1.117119in}}{\pgfqpoint{5.273662in}{1.121509in}}{\pgfqpoint{5.262612in}{1.121509in}}%
\pgfpathcurveto{\pgfqpoint{5.251562in}{1.121509in}}{\pgfqpoint{5.240963in}{1.117119in}}{\pgfqpoint{5.233149in}{1.109305in}}%
\pgfpathcurveto{\pgfqpoint{5.225335in}{1.101492in}}{\pgfqpoint{5.220945in}{1.090893in}}{\pgfqpoint{5.220945in}{1.079842in}}%
\pgfpathcurveto{\pgfqpoint{5.220945in}{1.068792in}}{\pgfqpoint{5.225335in}{1.058193in}}{\pgfqpoint{5.233149in}{1.050380in}}%
\pgfpathcurveto{\pgfqpoint{5.240963in}{1.042566in}}{\pgfqpoint{5.251562in}{1.038176in}}{\pgfqpoint{5.262612in}{1.038176in}}%
\pgfpathclose%
\pgfusepath{stroke,fill}%
\end{pgfscope}%
\begin{pgfscope}%
\pgfpathrectangle{\pgfqpoint{0.970666in}{0.566125in}}{\pgfqpoint{5.699255in}{2.685432in}}%
\pgfusepath{clip}%
\pgfsetbuttcap%
\pgfsetroundjoin%
\definecolor{currentfill}{rgb}{0.000000,0.000000,0.000000}%
\pgfsetfillcolor{currentfill}%
\pgfsetlinewidth{1.003750pt}%
\definecolor{currentstroke}{rgb}{0.000000,0.000000,0.000000}%
\pgfsetstrokecolor{currentstroke}%
\pgfsetdash{}{0pt}%
\pgfpathmoveto{\pgfqpoint{4.898532in}{0.894570in}}%
\pgfpathcurveto{\pgfqpoint{4.909582in}{0.894570in}}{\pgfqpoint{4.920181in}{0.898960in}}{\pgfqpoint{4.927994in}{0.906774in}}%
\pgfpathcurveto{\pgfqpoint{4.935808in}{0.914587in}}{\pgfqpoint{4.940198in}{0.925186in}}{\pgfqpoint{4.940198in}{0.936236in}}%
\pgfpathcurveto{\pgfqpoint{4.940198in}{0.947287in}}{\pgfqpoint{4.935808in}{0.957886in}}{\pgfqpoint{4.927994in}{0.965699in}}%
\pgfpathcurveto{\pgfqpoint{4.920181in}{0.973513in}}{\pgfqpoint{4.909582in}{0.977903in}}{\pgfqpoint{4.898532in}{0.977903in}}%
\pgfpathcurveto{\pgfqpoint{4.887482in}{0.977903in}}{\pgfqpoint{4.876883in}{0.973513in}}{\pgfqpoint{4.869069in}{0.965699in}}%
\pgfpathcurveto{\pgfqpoint{4.861255in}{0.957886in}}{\pgfqpoint{4.856865in}{0.947287in}}{\pgfqpoint{4.856865in}{0.936236in}}%
\pgfpathcurveto{\pgfqpoint{4.856865in}{0.925186in}}{\pgfqpoint{4.861255in}{0.914587in}}{\pgfqpoint{4.869069in}{0.906774in}}%
\pgfpathcurveto{\pgfqpoint{4.876883in}{0.898960in}}{\pgfqpoint{4.887482in}{0.894570in}}{\pgfqpoint{4.898532in}{0.894570in}}%
\pgfpathclose%
\pgfusepath{stroke,fill}%
\end{pgfscope}%
\begin{pgfscope}%
\pgfpathrectangle{\pgfqpoint{0.970666in}{0.566125in}}{\pgfqpoint{5.699255in}{2.685432in}}%
\pgfusepath{clip}%
\pgfsetbuttcap%
\pgfsetroundjoin%
\definecolor{currentfill}{rgb}{0.000000,0.000000,0.000000}%
\pgfsetfillcolor{currentfill}%
\pgfsetlinewidth{1.003750pt}%
\definecolor{currentstroke}{rgb}{0.000000,0.000000,0.000000}%
\pgfsetstrokecolor{currentstroke}%
\pgfsetdash{}{0pt}%
\pgfpathmoveto{\pgfqpoint{4.898532in}{0.881515in}}%
\pgfpathcurveto{\pgfqpoint{4.909582in}{0.881515in}}{\pgfqpoint{4.920181in}{0.885905in}}{\pgfqpoint{4.927994in}{0.893719in}}%
\pgfpathcurveto{\pgfqpoint{4.935808in}{0.901532in}}{\pgfqpoint{4.940198in}{0.912131in}}{\pgfqpoint{4.940198in}{0.923181in}}%
\pgfpathcurveto{\pgfqpoint{4.940198in}{0.934231in}}{\pgfqpoint{4.935808in}{0.944830in}}{\pgfqpoint{4.927994in}{0.952644in}}%
\pgfpathcurveto{\pgfqpoint{4.920181in}{0.960458in}}{\pgfqpoint{4.909582in}{0.964848in}}{\pgfqpoint{4.898532in}{0.964848in}}%
\pgfpathcurveto{\pgfqpoint{4.887482in}{0.964848in}}{\pgfqpoint{4.876883in}{0.960458in}}{\pgfqpoint{4.869069in}{0.952644in}}%
\pgfpathcurveto{\pgfqpoint{4.861255in}{0.944830in}}{\pgfqpoint{4.856865in}{0.934231in}}{\pgfqpoint{4.856865in}{0.923181in}}%
\pgfpathcurveto{\pgfqpoint{4.856865in}{0.912131in}}{\pgfqpoint{4.861255in}{0.901532in}}{\pgfqpoint{4.869069in}{0.893719in}}%
\pgfpathcurveto{\pgfqpoint{4.876883in}{0.885905in}}{\pgfqpoint{4.887482in}{0.881515in}}{\pgfqpoint{4.898532in}{0.881515in}}%
\pgfpathclose%
\pgfusepath{stroke,fill}%
\end{pgfscope}%
\begin{pgfscope}%
\pgfpathrectangle{\pgfqpoint{0.970666in}{0.566125in}}{\pgfqpoint{5.699255in}{2.685432in}}%
\pgfusepath{clip}%
\pgfsetbuttcap%
\pgfsetroundjoin%
\definecolor{currentfill}{rgb}{0.000000,0.000000,0.000000}%
\pgfsetfillcolor{currentfill}%
\pgfsetlinewidth{1.003750pt}%
\definecolor{currentstroke}{rgb}{0.000000,0.000000,0.000000}%
\pgfsetstrokecolor{currentstroke}%
\pgfsetdash{}{0pt}%
\pgfpathmoveto{\pgfqpoint{4.590464in}{0.881515in}}%
\pgfpathcurveto{\pgfqpoint{4.601514in}{0.881515in}}{\pgfqpoint{4.612113in}{0.885905in}}{\pgfqpoint{4.619927in}{0.893719in}}%
\pgfpathcurveto{\pgfqpoint{4.627740in}{0.901532in}}{\pgfqpoint{4.632130in}{0.912131in}}{\pgfqpoint{4.632130in}{0.923181in}}%
\pgfpathcurveto{\pgfqpoint{4.632130in}{0.934231in}}{\pgfqpoint{4.627740in}{0.944830in}}{\pgfqpoint{4.619927in}{0.952644in}}%
\pgfpathcurveto{\pgfqpoint{4.612113in}{0.960458in}}{\pgfqpoint{4.601514in}{0.964848in}}{\pgfqpoint{4.590464in}{0.964848in}}%
\pgfpathcurveto{\pgfqpoint{4.579414in}{0.964848in}}{\pgfqpoint{4.568815in}{0.960458in}}{\pgfqpoint{4.561001in}{0.952644in}}%
\pgfpathcurveto{\pgfqpoint{4.553187in}{0.944830in}}{\pgfqpoint{4.548797in}{0.934231in}}{\pgfqpoint{4.548797in}{0.923181in}}%
\pgfpathcurveto{\pgfqpoint{4.548797in}{0.912131in}}{\pgfqpoint{4.553187in}{0.901532in}}{\pgfqpoint{4.561001in}{0.893719in}}%
\pgfpathcurveto{\pgfqpoint{4.568815in}{0.885905in}}{\pgfqpoint{4.579414in}{0.881515in}}{\pgfqpoint{4.590464in}{0.881515in}}%
\pgfpathclose%
\pgfusepath{stroke,fill}%
\end{pgfscope}%
\begin{pgfscope}%
\pgfpathrectangle{\pgfqpoint{0.970666in}{0.566125in}}{\pgfqpoint{5.699255in}{2.685432in}}%
\pgfusepath{clip}%
\pgfsetbuttcap%
\pgfsetroundjoin%
\definecolor{currentfill}{rgb}{0.000000,0.000000,0.000000}%
\pgfsetfillcolor{currentfill}%
\pgfsetlinewidth{1.003750pt}%
\definecolor{currentstroke}{rgb}{0.000000,0.000000,0.000000}%
\pgfsetstrokecolor{currentstroke}%
\pgfsetdash{}{0pt}%
\pgfpathmoveto{\pgfqpoint{4.086353in}{0.842349in}}%
\pgfpathcurveto{\pgfqpoint{4.097403in}{0.842349in}}{\pgfqpoint{4.108002in}{0.846740in}}{\pgfqpoint{4.115816in}{0.854553in}}%
\pgfpathcurveto{\pgfqpoint{4.123629in}{0.862367in}}{\pgfqpoint{4.128019in}{0.872966in}}{\pgfqpoint{4.128019in}{0.884016in}}%
\pgfpathcurveto{\pgfqpoint{4.128019in}{0.895066in}}{\pgfqpoint{4.123629in}{0.905665in}}{\pgfqpoint{4.115816in}{0.913479in}}%
\pgfpathcurveto{\pgfqpoint{4.108002in}{0.921292in}}{\pgfqpoint{4.097403in}{0.925683in}}{\pgfqpoint{4.086353in}{0.925683in}}%
\pgfpathcurveto{\pgfqpoint{4.075303in}{0.925683in}}{\pgfqpoint{4.064704in}{0.921292in}}{\pgfqpoint{4.056890in}{0.913479in}}%
\pgfpathcurveto{\pgfqpoint{4.049076in}{0.905665in}}{\pgfqpoint{4.044686in}{0.895066in}}{\pgfqpoint{4.044686in}{0.884016in}}%
\pgfpathcurveto{\pgfqpoint{4.044686in}{0.872966in}}{\pgfqpoint{4.049076in}{0.862367in}}{\pgfqpoint{4.056890in}{0.854553in}}%
\pgfpathcurveto{\pgfqpoint{4.064704in}{0.846740in}}{\pgfqpoint{4.075303in}{0.842349in}}{\pgfqpoint{4.086353in}{0.842349in}}%
\pgfpathclose%
\pgfusepath{stroke,fill}%
\end{pgfscope}%
\begin{pgfscope}%
\pgfpathrectangle{\pgfqpoint{0.970666in}{0.566125in}}{\pgfqpoint{5.699255in}{2.685432in}}%
\pgfusepath{clip}%
\pgfsetbuttcap%
\pgfsetroundjoin%
\definecolor{currentfill}{rgb}{0.000000,0.000000,0.000000}%
\pgfsetfillcolor{currentfill}%
\pgfsetlinewidth{1.003750pt}%
\definecolor{currentstroke}{rgb}{0.000000,0.000000,0.000000}%
\pgfsetstrokecolor{currentstroke}%
\pgfsetdash{}{0pt}%
\pgfpathmoveto{\pgfqpoint{4.030340in}{0.790129in}}%
\pgfpathcurveto{\pgfqpoint{4.041391in}{0.790129in}}{\pgfqpoint{4.051990in}{0.794519in}}{\pgfqpoint{4.059803in}{0.802333in}}%
\pgfpathcurveto{\pgfqpoint{4.067617in}{0.810147in}}{\pgfqpoint{4.072007in}{0.820746in}}{\pgfqpoint{4.072007in}{0.831796in}}%
\pgfpathcurveto{\pgfqpoint{4.072007in}{0.842846in}}{\pgfqpoint{4.067617in}{0.853445in}}{\pgfqpoint{4.059803in}{0.861258in}}%
\pgfpathcurveto{\pgfqpoint{4.051990in}{0.869072in}}{\pgfqpoint{4.041391in}{0.873462in}}{\pgfqpoint{4.030340in}{0.873462in}}%
\pgfpathcurveto{\pgfqpoint{4.019290in}{0.873462in}}{\pgfqpoint{4.008691in}{0.869072in}}{\pgfqpoint{4.000878in}{0.861258in}}%
\pgfpathcurveto{\pgfqpoint{3.993064in}{0.853445in}}{\pgfqpoint{3.988674in}{0.842846in}}{\pgfqpoint{3.988674in}{0.831796in}}%
\pgfpathcurveto{\pgfqpoint{3.988674in}{0.820746in}}{\pgfqpoint{3.993064in}{0.810147in}}{\pgfqpoint{4.000878in}{0.802333in}}%
\pgfpathcurveto{\pgfqpoint{4.008691in}{0.794519in}}{\pgfqpoint{4.019290in}{0.790129in}}{\pgfqpoint{4.030340in}{0.790129in}}%
\pgfpathclose%
\pgfusepath{stroke,fill}%
\end{pgfscope}%
\begin{pgfscope}%
\pgfpathrectangle{\pgfqpoint{0.970666in}{0.566125in}}{\pgfqpoint{5.699255in}{2.685432in}}%
\pgfusepath{clip}%
\pgfsetbuttcap%
\pgfsetroundjoin%
\definecolor{currentfill}{rgb}{0.000000,0.000000,0.000000}%
\pgfsetfillcolor{currentfill}%
\pgfsetlinewidth{1.003750pt}%
\definecolor{currentstroke}{rgb}{0.000000,0.000000,0.000000}%
\pgfsetstrokecolor{currentstroke}%
\pgfsetdash{}{0pt}%
\pgfpathmoveto{\pgfqpoint{3.890310in}{0.855404in}}%
\pgfpathcurveto{\pgfqpoint{3.901360in}{0.855404in}}{\pgfqpoint{3.911959in}{0.859795in}}{\pgfqpoint{3.919772in}{0.867608in}}%
\pgfpathcurveto{\pgfqpoint{3.927586in}{0.875422in}}{\pgfqpoint{3.931976in}{0.886021in}}{\pgfqpoint{3.931976in}{0.897071in}}%
\pgfpathcurveto{\pgfqpoint{3.931976in}{0.908121in}}{\pgfqpoint{3.927586in}{0.918720in}}{\pgfqpoint{3.919772in}{0.926534in}}%
\pgfpathcurveto{\pgfqpoint{3.911959in}{0.934348in}}{\pgfqpoint{3.901360in}{0.938738in}}{\pgfqpoint{3.890310in}{0.938738in}}%
\pgfpathcurveto{\pgfqpoint{3.879259in}{0.938738in}}{\pgfqpoint{3.868660in}{0.934348in}}{\pgfqpoint{3.860847in}{0.926534in}}%
\pgfpathcurveto{\pgfqpoint{3.853033in}{0.918720in}}{\pgfqpoint{3.848643in}{0.908121in}}{\pgfqpoint{3.848643in}{0.897071in}}%
\pgfpathcurveto{\pgfqpoint{3.848643in}{0.886021in}}{\pgfqpoint{3.853033in}{0.875422in}}{\pgfqpoint{3.860847in}{0.867608in}}%
\pgfpathcurveto{\pgfqpoint{3.868660in}{0.859795in}}{\pgfqpoint{3.879259in}{0.855404in}}{\pgfqpoint{3.890310in}{0.855404in}}%
\pgfpathclose%
\pgfusepath{stroke,fill}%
\end{pgfscope}%
\begin{pgfscope}%
\pgfpathrectangle{\pgfqpoint{0.970666in}{0.566125in}}{\pgfqpoint{5.699255in}{2.685432in}}%
\pgfusepath{clip}%
\pgfsetbuttcap%
\pgfsetroundjoin%
\definecolor{currentfill}{rgb}{0.000000,0.000000,0.000000}%
\pgfsetfillcolor{currentfill}%
\pgfsetlinewidth{1.003750pt}%
\definecolor{currentstroke}{rgb}{0.000000,0.000000,0.000000}%
\pgfsetstrokecolor{currentstroke}%
\pgfsetdash{}{0pt}%
\pgfpathmoveto{\pgfqpoint{3.078131in}{0.803184in}}%
\pgfpathcurveto{\pgfqpoint{3.089181in}{0.803184in}}{\pgfqpoint{3.099780in}{0.807574in}}{\pgfqpoint{3.107593in}{0.815388in}}%
\pgfpathcurveto{\pgfqpoint{3.115407in}{0.823202in}}{\pgfqpoint{3.119797in}{0.833801in}}{\pgfqpoint{3.119797in}{0.844851in}}%
\pgfpathcurveto{\pgfqpoint{3.119797in}{0.855901in}}{\pgfqpoint{3.115407in}{0.866500in}}{\pgfqpoint{3.107593in}{0.874314in}}%
\pgfpathcurveto{\pgfqpoint{3.099780in}{0.882127in}}{\pgfqpoint{3.089181in}{0.886517in}}{\pgfqpoint{3.078131in}{0.886517in}}%
\pgfpathcurveto{\pgfqpoint{3.067081in}{0.886517in}}{\pgfqpoint{3.056482in}{0.882127in}}{\pgfqpoint{3.048668in}{0.874314in}}%
\pgfpathcurveto{\pgfqpoint{3.040854in}{0.866500in}}{\pgfqpoint{3.036464in}{0.855901in}}{\pgfqpoint{3.036464in}{0.844851in}}%
\pgfpathcurveto{\pgfqpoint{3.036464in}{0.833801in}}{\pgfqpoint{3.040854in}{0.823202in}}{\pgfqpoint{3.048668in}{0.815388in}}%
\pgfpathcurveto{\pgfqpoint{3.056482in}{0.807574in}}{\pgfqpoint{3.067081in}{0.803184in}}{\pgfqpoint{3.078131in}{0.803184in}}%
\pgfpathclose%
\pgfusepath{stroke,fill}%
\end{pgfscope}%
\begin{pgfscope}%
\pgfpathrectangle{\pgfqpoint{0.970666in}{0.566125in}}{\pgfqpoint{5.699255in}{2.685432in}}%
\pgfusepath{clip}%
\pgfsetbuttcap%
\pgfsetroundjoin%
\definecolor{currentfill}{rgb}{0.000000,0.000000,0.000000}%
\pgfsetfillcolor{currentfill}%
\pgfsetlinewidth{1.003750pt}%
\definecolor{currentstroke}{rgb}{0.000000,0.000000,0.000000}%
\pgfsetstrokecolor{currentstroke}%
\pgfsetdash{}{0pt}%
\pgfpathmoveto{\pgfqpoint{1.509785in}{1.312333in}}%
\pgfpathcurveto{\pgfqpoint{1.520835in}{1.312333in}}{\pgfqpoint{1.531434in}{1.316723in}}{\pgfqpoint{1.539248in}{1.324537in}}%
\pgfpathcurveto{\pgfqpoint{1.547062in}{1.332350in}}{\pgfqpoint{1.551452in}{1.342949in}}{\pgfqpoint{1.551452in}{1.353999in}}%
\pgfpathcurveto{\pgfqpoint{1.551452in}{1.365049in}}{\pgfqpoint{1.547062in}{1.375648in}}{\pgfqpoint{1.539248in}{1.383462in}}%
\pgfpathcurveto{\pgfqpoint{1.531434in}{1.391276in}}{\pgfqpoint{1.520835in}{1.395666in}}{\pgfqpoint{1.509785in}{1.395666in}}%
\pgfpathcurveto{\pgfqpoint{1.498735in}{1.395666in}}{\pgfqpoint{1.488136in}{1.391276in}}{\pgfqpoint{1.480322in}{1.383462in}}%
\pgfpathcurveto{\pgfqpoint{1.472509in}{1.375648in}}{\pgfqpoint{1.468119in}{1.365049in}}{\pgfqpoint{1.468119in}{1.353999in}}%
\pgfpathcurveto{\pgfqpoint{1.468119in}{1.342949in}}{\pgfqpoint{1.472509in}{1.332350in}}{\pgfqpoint{1.480322in}{1.324537in}}%
\pgfpathcurveto{\pgfqpoint{1.488136in}{1.316723in}}{\pgfqpoint{1.498735in}{1.312333in}}{\pgfqpoint{1.509785in}{1.312333in}}%
\pgfpathclose%
\pgfusepath{stroke,fill}%
\end{pgfscope}%
\begin{pgfscope}%
\pgfpathrectangle{\pgfqpoint{0.970666in}{0.566125in}}{\pgfqpoint{5.699255in}{2.685432in}}%
\pgfusepath{clip}%
\pgfsetbuttcap%
\pgfsetroundjoin%
\definecolor{currentfill}{rgb}{0.000000,0.000000,0.000000}%
\pgfsetfillcolor{currentfill}%
\pgfsetlinewidth{1.003750pt}%
\definecolor{currentstroke}{rgb}{0.000000,0.000000,0.000000}%
\pgfsetstrokecolor{currentstroke}%
\pgfsetdash{}{0pt}%
\pgfpathmoveto{\pgfqpoint{1.369754in}{1.351498in}}%
\pgfpathcurveto{\pgfqpoint{1.380805in}{1.351498in}}{\pgfqpoint{1.391404in}{1.355888in}}{\pgfqpoint{1.399217in}{1.363702in}}%
\pgfpathcurveto{\pgfqpoint{1.407031in}{1.371515in}}{\pgfqpoint{1.411421in}{1.382114in}}{\pgfqpoint{1.411421in}{1.393165in}}%
\pgfpathcurveto{\pgfqpoint{1.411421in}{1.404215in}}{\pgfqpoint{1.407031in}{1.414814in}}{\pgfqpoint{1.399217in}{1.422627in}}%
\pgfpathcurveto{\pgfqpoint{1.391404in}{1.430441in}}{\pgfqpoint{1.380805in}{1.434831in}}{\pgfqpoint{1.369754in}{1.434831in}}%
\pgfpathcurveto{\pgfqpoint{1.358704in}{1.434831in}}{\pgfqpoint{1.348105in}{1.430441in}}{\pgfqpoint{1.340292in}{1.422627in}}%
\pgfpathcurveto{\pgfqpoint{1.332478in}{1.414814in}}{\pgfqpoint{1.328088in}{1.404215in}}{\pgfqpoint{1.328088in}{1.393165in}}%
\pgfpathcurveto{\pgfqpoint{1.328088in}{1.382114in}}{\pgfqpoint{1.332478in}{1.371515in}}{\pgfqpoint{1.340292in}{1.363702in}}%
\pgfpathcurveto{\pgfqpoint{1.348105in}{1.355888in}}{\pgfqpoint{1.358704in}{1.351498in}}{\pgfqpoint{1.369754in}{1.351498in}}%
\pgfpathclose%
\pgfusepath{stroke,fill}%
\end{pgfscope}%
\begin{pgfscope}%
\pgfpathrectangle{\pgfqpoint{0.970666in}{0.566125in}}{\pgfqpoint{5.699255in}{2.685432in}}%
\pgfusepath{clip}%
\pgfsetbuttcap%
\pgfsetroundjoin%
\definecolor{currentfill}{rgb}{0.000000,0.000000,1.000000}%
\pgfsetfillcolor{currentfill}%
\pgfsetlinewidth{1.003750pt}%
\definecolor{currentstroke}{rgb}{0.000000,0.000000,1.000000}%
\pgfsetstrokecolor{currentstroke}%
\pgfsetdash{}{0pt}%
\pgfpathmoveto{\pgfqpoint{5.990772in}{0.842349in}}%
\pgfpathcurveto{\pgfqpoint{6.001822in}{0.842349in}}{\pgfqpoint{6.012421in}{0.846740in}}{\pgfqpoint{6.020235in}{0.854553in}}%
\pgfpathcurveto{\pgfqpoint{6.028049in}{0.862367in}}{\pgfqpoint{6.032439in}{0.872966in}}{\pgfqpoint{6.032439in}{0.884016in}}%
\pgfpathcurveto{\pgfqpoint{6.032439in}{0.895066in}}{\pgfqpoint{6.028049in}{0.905665in}}{\pgfqpoint{6.020235in}{0.913479in}}%
\pgfpathcurveto{\pgfqpoint{6.012421in}{0.921292in}}{\pgfqpoint{6.001822in}{0.925683in}}{\pgfqpoint{5.990772in}{0.925683in}}%
\pgfpathcurveto{\pgfqpoint{5.979722in}{0.925683in}}{\pgfqpoint{5.969123in}{0.921292in}}{\pgfqpoint{5.961309in}{0.913479in}}%
\pgfpathcurveto{\pgfqpoint{5.953496in}{0.905665in}}{\pgfqpoint{5.949106in}{0.895066in}}{\pgfqpoint{5.949106in}{0.884016in}}%
\pgfpathcurveto{\pgfqpoint{5.949106in}{0.872966in}}{\pgfqpoint{5.953496in}{0.862367in}}{\pgfqpoint{5.961309in}{0.854553in}}%
\pgfpathcurveto{\pgfqpoint{5.969123in}{0.846740in}}{\pgfqpoint{5.979722in}{0.842349in}}{\pgfqpoint{5.990772in}{0.842349in}}%
\pgfpathclose%
\pgfusepath{stroke,fill}%
\end{pgfscope}%
\begin{pgfscope}%
\pgfsetbuttcap%
\pgfsetroundjoin%
\definecolor{currentfill}{rgb}{0.000000,0.000000,0.000000}%
\pgfsetfillcolor{currentfill}%
\pgfsetlinewidth{0.803000pt}%
\definecolor{currentstroke}{rgb}{0.000000,0.000000,0.000000}%
\pgfsetstrokecolor{currentstroke}%
\pgfsetdash{}{0pt}%
\pgfsys@defobject{currentmarker}{\pgfqpoint{0.000000in}{-0.048611in}}{\pgfqpoint{0.000000in}{0.000000in}}{%
\pgfpathmoveto{\pgfqpoint{0.000000in}{0.000000in}}%
\pgfpathlineto{\pgfqpoint{0.000000in}{-0.048611in}}%
\pgfusepath{stroke,fill}%
}%
\begin{pgfscope}%
\pgfsys@transformshift{1.145705in}{0.566125in}%
\pgfsys@useobject{currentmarker}{}%
\end{pgfscope}%
\end{pgfscope}%
\begin{pgfscope}%
\definecolor{textcolor}{rgb}{0.000000,0.000000,0.000000}%
\pgfsetstrokecolor{textcolor}%
\pgfsetfillcolor{textcolor}%
\pgftext[x=1.145705in,y=0.468902in,,top]{\color{textcolor}\rmfamily\fontsize{10.000000}{12.000000}\selectfont \(\displaystyle 0\)}%
\end{pgfscope}%
\begin{pgfscope}%
\pgfsetbuttcap%
\pgfsetroundjoin%
\definecolor{currentfill}{rgb}{0.000000,0.000000,0.000000}%
\pgfsetfillcolor{currentfill}%
\pgfsetlinewidth{0.803000pt}%
\definecolor{currentstroke}{rgb}{0.000000,0.000000,0.000000}%
\pgfsetstrokecolor{currentstroke}%
\pgfsetdash{}{0pt}%
\pgfsys@defobject{currentmarker}{\pgfqpoint{0.000000in}{-0.048611in}}{\pgfqpoint{0.000000in}{0.000000in}}{%
\pgfpathmoveto{\pgfqpoint{0.000000in}{0.000000in}}%
\pgfpathlineto{\pgfqpoint{0.000000in}{-0.048611in}}%
\pgfusepath{stroke,fill}%
}%
\begin{pgfscope}%
\pgfsys@transformshift{1.845859in}{0.566125in}%
\pgfsys@useobject{currentmarker}{}%
\end{pgfscope}%
\end{pgfscope}%
\begin{pgfscope}%
\definecolor{textcolor}{rgb}{0.000000,0.000000,0.000000}%
\pgfsetstrokecolor{textcolor}%
\pgfsetfillcolor{textcolor}%
\pgftext[x=1.845859in,y=0.468902in,,top]{\color{textcolor}\rmfamily\fontsize{10.000000}{12.000000}\selectfont \(\displaystyle 25\)}%
\end{pgfscope}%
\begin{pgfscope}%
\pgfsetbuttcap%
\pgfsetroundjoin%
\definecolor{currentfill}{rgb}{0.000000,0.000000,0.000000}%
\pgfsetfillcolor{currentfill}%
\pgfsetlinewidth{0.803000pt}%
\definecolor{currentstroke}{rgb}{0.000000,0.000000,0.000000}%
\pgfsetstrokecolor{currentstroke}%
\pgfsetdash{}{0pt}%
\pgfsys@defobject{currentmarker}{\pgfqpoint{0.000000in}{-0.048611in}}{\pgfqpoint{0.000000in}{0.000000in}}{%
\pgfpathmoveto{\pgfqpoint{0.000000in}{0.000000in}}%
\pgfpathlineto{\pgfqpoint{0.000000in}{-0.048611in}}%
\pgfusepath{stroke,fill}%
}%
\begin{pgfscope}%
\pgfsys@transformshift{2.546013in}{0.566125in}%
\pgfsys@useobject{currentmarker}{}%
\end{pgfscope}%
\end{pgfscope}%
\begin{pgfscope}%
\definecolor{textcolor}{rgb}{0.000000,0.000000,0.000000}%
\pgfsetstrokecolor{textcolor}%
\pgfsetfillcolor{textcolor}%
\pgftext[x=2.546013in,y=0.468902in,,top]{\color{textcolor}\rmfamily\fontsize{10.000000}{12.000000}\selectfont \(\displaystyle 50\)}%
\end{pgfscope}%
\begin{pgfscope}%
\pgfsetbuttcap%
\pgfsetroundjoin%
\definecolor{currentfill}{rgb}{0.000000,0.000000,0.000000}%
\pgfsetfillcolor{currentfill}%
\pgfsetlinewidth{0.803000pt}%
\definecolor{currentstroke}{rgb}{0.000000,0.000000,0.000000}%
\pgfsetstrokecolor{currentstroke}%
\pgfsetdash{}{0pt}%
\pgfsys@defobject{currentmarker}{\pgfqpoint{0.000000in}{-0.048611in}}{\pgfqpoint{0.000000in}{0.000000in}}{%
\pgfpathmoveto{\pgfqpoint{0.000000in}{0.000000in}}%
\pgfpathlineto{\pgfqpoint{0.000000in}{-0.048611in}}%
\pgfusepath{stroke,fill}%
}%
\begin{pgfscope}%
\pgfsys@transformshift{3.246168in}{0.566125in}%
\pgfsys@useobject{currentmarker}{}%
\end{pgfscope}%
\end{pgfscope}%
\begin{pgfscope}%
\definecolor{textcolor}{rgb}{0.000000,0.000000,0.000000}%
\pgfsetstrokecolor{textcolor}%
\pgfsetfillcolor{textcolor}%
\pgftext[x=3.246168in,y=0.468902in,,top]{\color{textcolor}\rmfamily\fontsize{10.000000}{12.000000}\selectfont \(\displaystyle 75\)}%
\end{pgfscope}%
\begin{pgfscope}%
\pgfsetbuttcap%
\pgfsetroundjoin%
\definecolor{currentfill}{rgb}{0.000000,0.000000,0.000000}%
\pgfsetfillcolor{currentfill}%
\pgfsetlinewidth{0.803000pt}%
\definecolor{currentstroke}{rgb}{0.000000,0.000000,0.000000}%
\pgfsetstrokecolor{currentstroke}%
\pgfsetdash{}{0pt}%
\pgfsys@defobject{currentmarker}{\pgfqpoint{0.000000in}{-0.048611in}}{\pgfqpoint{0.000000in}{0.000000in}}{%
\pgfpathmoveto{\pgfqpoint{0.000000in}{0.000000in}}%
\pgfpathlineto{\pgfqpoint{0.000000in}{-0.048611in}}%
\pgfusepath{stroke,fill}%
}%
\begin{pgfscope}%
\pgfsys@transformshift{3.946322in}{0.566125in}%
\pgfsys@useobject{currentmarker}{}%
\end{pgfscope}%
\end{pgfscope}%
\begin{pgfscope}%
\definecolor{textcolor}{rgb}{0.000000,0.000000,0.000000}%
\pgfsetstrokecolor{textcolor}%
\pgfsetfillcolor{textcolor}%
\pgftext[x=3.946322in,y=0.468902in,,top]{\color{textcolor}\rmfamily\fontsize{10.000000}{12.000000}\selectfont \(\displaystyle 100\)}%
\end{pgfscope}%
\begin{pgfscope}%
\pgfsetbuttcap%
\pgfsetroundjoin%
\definecolor{currentfill}{rgb}{0.000000,0.000000,0.000000}%
\pgfsetfillcolor{currentfill}%
\pgfsetlinewidth{0.803000pt}%
\definecolor{currentstroke}{rgb}{0.000000,0.000000,0.000000}%
\pgfsetstrokecolor{currentstroke}%
\pgfsetdash{}{0pt}%
\pgfsys@defobject{currentmarker}{\pgfqpoint{0.000000in}{-0.048611in}}{\pgfqpoint{0.000000in}{0.000000in}}{%
\pgfpathmoveto{\pgfqpoint{0.000000in}{0.000000in}}%
\pgfpathlineto{\pgfqpoint{0.000000in}{-0.048611in}}%
\pgfusepath{stroke,fill}%
}%
\begin{pgfscope}%
\pgfsys@transformshift{4.646476in}{0.566125in}%
\pgfsys@useobject{currentmarker}{}%
\end{pgfscope}%
\end{pgfscope}%
\begin{pgfscope}%
\definecolor{textcolor}{rgb}{0.000000,0.000000,0.000000}%
\pgfsetstrokecolor{textcolor}%
\pgfsetfillcolor{textcolor}%
\pgftext[x=4.646476in,y=0.468902in,,top]{\color{textcolor}\rmfamily\fontsize{10.000000}{12.000000}\selectfont \(\displaystyle 125\)}%
\end{pgfscope}%
\begin{pgfscope}%
\pgfsetbuttcap%
\pgfsetroundjoin%
\definecolor{currentfill}{rgb}{0.000000,0.000000,0.000000}%
\pgfsetfillcolor{currentfill}%
\pgfsetlinewidth{0.803000pt}%
\definecolor{currentstroke}{rgb}{0.000000,0.000000,0.000000}%
\pgfsetstrokecolor{currentstroke}%
\pgfsetdash{}{0pt}%
\pgfsys@defobject{currentmarker}{\pgfqpoint{0.000000in}{-0.048611in}}{\pgfqpoint{0.000000in}{0.000000in}}{%
\pgfpathmoveto{\pgfqpoint{0.000000in}{0.000000in}}%
\pgfpathlineto{\pgfqpoint{0.000000in}{-0.048611in}}%
\pgfusepath{stroke,fill}%
}%
\begin{pgfscope}%
\pgfsys@transformshift{5.346630in}{0.566125in}%
\pgfsys@useobject{currentmarker}{}%
\end{pgfscope}%
\end{pgfscope}%
\begin{pgfscope}%
\definecolor{textcolor}{rgb}{0.000000,0.000000,0.000000}%
\pgfsetstrokecolor{textcolor}%
\pgfsetfillcolor{textcolor}%
\pgftext[x=5.346630in,y=0.468902in,,top]{\color{textcolor}\rmfamily\fontsize{10.000000}{12.000000}\selectfont \(\displaystyle 150\)}%
\end{pgfscope}%
\begin{pgfscope}%
\pgfsetbuttcap%
\pgfsetroundjoin%
\definecolor{currentfill}{rgb}{0.000000,0.000000,0.000000}%
\pgfsetfillcolor{currentfill}%
\pgfsetlinewidth{0.803000pt}%
\definecolor{currentstroke}{rgb}{0.000000,0.000000,0.000000}%
\pgfsetstrokecolor{currentstroke}%
\pgfsetdash{}{0pt}%
\pgfsys@defobject{currentmarker}{\pgfqpoint{0.000000in}{-0.048611in}}{\pgfqpoint{0.000000in}{0.000000in}}{%
\pgfpathmoveto{\pgfqpoint{0.000000in}{0.000000in}}%
\pgfpathlineto{\pgfqpoint{0.000000in}{-0.048611in}}%
\pgfusepath{stroke,fill}%
}%
\begin{pgfscope}%
\pgfsys@transformshift{6.046785in}{0.566125in}%
\pgfsys@useobject{currentmarker}{}%
\end{pgfscope}%
\end{pgfscope}%
\begin{pgfscope}%
\definecolor{textcolor}{rgb}{0.000000,0.000000,0.000000}%
\pgfsetstrokecolor{textcolor}%
\pgfsetfillcolor{textcolor}%
\pgftext[x=6.046785in,y=0.468902in,,top]{\color{textcolor}\rmfamily\fontsize{10.000000}{12.000000}\selectfont \(\displaystyle 175\)}%
\end{pgfscope}%
\begin{pgfscope}%
\pgfsetbuttcap%
\pgfsetroundjoin%
\definecolor{currentfill}{rgb}{0.000000,0.000000,0.000000}%
\pgfsetfillcolor{currentfill}%
\pgfsetlinewidth{0.803000pt}%
\definecolor{currentstroke}{rgb}{0.000000,0.000000,0.000000}%
\pgfsetstrokecolor{currentstroke}%
\pgfsetdash{}{0pt}%
\pgfsys@defobject{currentmarker}{\pgfqpoint{-0.048611in}{0.000000in}}{\pgfqpoint{0.000000in}{0.000000in}}{%
\pgfpathmoveto{\pgfqpoint{0.000000in}{0.000000in}}%
\pgfpathlineto{\pgfqpoint{-0.048611in}{0.000000in}}%
\pgfusepath{stroke,fill}%
}%
\begin{pgfscope}%
\pgfsys@transformshift{0.970666in}{0.688190in}%
\pgfsys@useobject{currentmarker}{}%
\end{pgfscope}%
\end{pgfscope}%
\begin{pgfscope}%
\definecolor{textcolor}{rgb}{0.000000,0.000000,0.000000}%
\pgfsetstrokecolor{textcolor}%
\pgfsetfillcolor{textcolor}%
\pgftext[x=0.804000in, y=0.635428in, left, base]{\color{textcolor}\rmfamily\fontsize{10.000000}{12.000000}\selectfont \(\displaystyle 0\)}%
\end{pgfscope}%
\begin{pgfscope}%
\pgfsetbuttcap%
\pgfsetroundjoin%
\definecolor{currentfill}{rgb}{0.000000,0.000000,0.000000}%
\pgfsetfillcolor{currentfill}%
\pgfsetlinewidth{0.803000pt}%
\definecolor{currentstroke}{rgb}{0.000000,0.000000,0.000000}%
\pgfsetstrokecolor{currentstroke}%
\pgfsetdash{}{0pt}%
\pgfsys@defobject{currentmarker}{\pgfqpoint{-0.048611in}{0.000000in}}{\pgfqpoint{0.000000in}{0.000000in}}{%
\pgfpathmoveto{\pgfqpoint{0.000000in}{0.000000in}}%
\pgfpathlineto{\pgfqpoint{-0.048611in}{0.000000in}}%
\pgfusepath{stroke,fill}%
}%
\begin{pgfscope}%
\pgfsys@transformshift{0.970666in}{1.014567in}%
\pgfsys@useobject{currentmarker}{}%
\end{pgfscope}%
\end{pgfscope}%
\begin{pgfscope}%
\definecolor{textcolor}{rgb}{0.000000,0.000000,0.000000}%
\pgfsetstrokecolor{textcolor}%
\pgfsetfillcolor{textcolor}%
\pgftext[x=0.734555in, y=0.961805in, left, base]{\color{textcolor}\rmfamily\fontsize{10.000000}{12.000000}\selectfont \(\displaystyle 25\)}%
\end{pgfscope}%
\begin{pgfscope}%
\pgfsetbuttcap%
\pgfsetroundjoin%
\definecolor{currentfill}{rgb}{0.000000,0.000000,0.000000}%
\pgfsetfillcolor{currentfill}%
\pgfsetlinewidth{0.803000pt}%
\definecolor{currentstroke}{rgb}{0.000000,0.000000,0.000000}%
\pgfsetstrokecolor{currentstroke}%
\pgfsetdash{}{0pt}%
\pgfsys@defobject{currentmarker}{\pgfqpoint{-0.048611in}{0.000000in}}{\pgfqpoint{0.000000in}{0.000000in}}{%
\pgfpathmoveto{\pgfqpoint{0.000000in}{0.000000in}}%
\pgfpathlineto{\pgfqpoint{-0.048611in}{0.000000in}}%
\pgfusepath{stroke,fill}%
}%
\begin{pgfscope}%
\pgfsys@transformshift{0.970666in}{1.340944in}%
\pgfsys@useobject{currentmarker}{}%
\end{pgfscope}%
\end{pgfscope}%
\begin{pgfscope}%
\definecolor{textcolor}{rgb}{0.000000,0.000000,0.000000}%
\pgfsetstrokecolor{textcolor}%
\pgfsetfillcolor{textcolor}%
\pgftext[x=0.734555in, y=1.288183in, left, base]{\color{textcolor}\rmfamily\fontsize{10.000000}{12.000000}\selectfont \(\displaystyle 50\)}%
\end{pgfscope}%
\begin{pgfscope}%
\pgfsetbuttcap%
\pgfsetroundjoin%
\definecolor{currentfill}{rgb}{0.000000,0.000000,0.000000}%
\pgfsetfillcolor{currentfill}%
\pgfsetlinewidth{0.803000pt}%
\definecolor{currentstroke}{rgb}{0.000000,0.000000,0.000000}%
\pgfsetstrokecolor{currentstroke}%
\pgfsetdash{}{0pt}%
\pgfsys@defobject{currentmarker}{\pgfqpoint{-0.048611in}{0.000000in}}{\pgfqpoint{0.000000in}{0.000000in}}{%
\pgfpathmoveto{\pgfqpoint{0.000000in}{0.000000in}}%
\pgfpathlineto{\pgfqpoint{-0.048611in}{0.000000in}}%
\pgfusepath{stroke,fill}%
}%
\begin{pgfscope}%
\pgfsys@transformshift{0.970666in}{1.667321in}%
\pgfsys@useobject{currentmarker}{}%
\end{pgfscope}%
\end{pgfscope}%
\begin{pgfscope}%
\definecolor{textcolor}{rgb}{0.000000,0.000000,0.000000}%
\pgfsetstrokecolor{textcolor}%
\pgfsetfillcolor{textcolor}%
\pgftext[x=0.734555in, y=1.614560in, left, base]{\color{textcolor}\rmfamily\fontsize{10.000000}{12.000000}\selectfont \(\displaystyle 75\)}%
\end{pgfscope}%
\begin{pgfscope}%
\pgfsetbuttcap%
\pgfsetroundjoin%
\definecolor{currentfill}{rgb}{0.000000,0.000000,0.000000}%
\pgfsetfillcolor{currentfill}%
\pgfsetlinewidth{0.803000pt}%
\definecolor{currentstroke}{rgb}{0.000000,0.000000,0.000000}%
\pgfsetstrokecolor{currentstroke}%
\pgfsetdash{}{0pt}%
\pgfsys@defobject{currentmarker}{\pgfqpoint{-0.048611in}{0.000000in}}{\pgfqpoint{0.000000in}{0.000000in}}{%
\pgfpathmoveto{\pgfqpoint{0.000000in}{0.000000in}}%
\pgfpathlineto{\pgfqpoint{-0.048611in}{0.000000in}}%
\pgfusepath{stroke,fill}%
}%
\begin{pgfscope}%
\pgfsys@transformshift{0.970666in}{1.993699in}%
\pgfsys@useobject{currentmarker}{}%
\end{pgfscope}%
\end{pgfscope}%
\begin{pgfscope}%
\definecolor{textcolor}{rgb}{0.000000,0.000000,0.000000}%
\pgfsetstrokecolor{textcolor}%
\pgfsetfillcolor{textcolor}%
\pgftext[x=0.665110in, y=1.940937in, left, base]{\color{textcolor}\rmfamily\fontsize{10.000000}{12.000000}\selectfont \(\displaystyle 100\)}%
\end{pgfscope}%
\begin{pgfscope}%
\pgfsetbuttcap%
\pgfsetroundjoin%
\definecolor{currentfill}{rgb}{0.000000,0.000000,0.000000}%
\pgfsetfillcolor{currentfill}%
\pgfsetlinewidth{0.803000pt}%
\definecolor{currentstroke}{rgb}{0.000000,0.000000,0.000000}%
\pgfsetstrokecolor{currentstroke}%
\pgfsetdash{}{0pt}%
\pgfsys@defobject{currentmarker}{\pgfqpoint{-0.048611in}{0.000000in}}{\pgfqpoint{0.000000in}{0.000000in}}{%
\pgfpathmoveto{\pgfqpoint{0.000000in}{0.000000in}}%
\pgfpathlineto{\pgfqpoint{-0.048611in}{0.000000in}}%
\pgfusepath{stroke,fill}%
}%
\begin{pgfscope}%
\pgfsys@transformshift{0.970666in}{2.320076in}%
\pgfsys@useobject{currentmarker}{}%
\end{pgfscope}%
\end{pgfscope}%
\begin{pgfscope}%
\definecolor{textcolor}{rgb}{0.000000,0.000000,0.000000}%
\pgfsetstrokecolor{textcolor}%
\pgfsetfillcolor{textcolor}%
\pgftext[x=0.665110in, y=2.267314in, left, base]{\color{textcolor}\rmfamily\fontsize{10.000000}{12.000000}\selectfont \(\displaystyle 125\)}%
\end{pgfscope}%
\begin{pgfscope}%
\pgfsetbuttcap%
\pgfsetroundjoin%
\definecolor{currentfill}{rgb}{0.000000,0.000000,0.000000}%
\pgfsetfillcolor{currentfill}%
\pgfsetlinewidth{0.803000pt}%
\definecolor{currentstroke}{rgb}{0.000000,0.000000,0.000000}%
\pgfsetstrokecolor{currentstroke}%
\pgfsetdash{}{0pt}%
\pgfsys@defobject{currentmarker}{\pgfqpoint{-0.048611in}{0.000000in}}{\pgfqpoint{0.000000in}{0.000000in}}{%
\pgfpathmoveto{\pgfqpoint{0.000000in}{0.000000in}}%
\pgfpathlineto{\pgfqpoint{-0.048611in}{0.000000in}}%
\pgfusepath{stroke,fill}%
}%
\begin{pgfscope}%
\pgfsys@transformshift{0.970666in}{2.646453in}%
\pgfsys@useobject{currentmarker}{}%
\end{pgfscope}%
\end{pgfscope}%
\begin{pgfscope}%
\definecolor{textcolor}{rgb}{0.000000,0.000000,0.000000}%
\pgfsetstrokecolor{textcolor}%
\pgfsetfillcolor{textcolor}%
\pgftext[x=0.665110in, y=2.593692in, left, base]{\color{textcolor}\rmfamily\fontsize{10.000000}{12.000000}\selectfont \(\displaystyle 150\)}%
\end{pgfscope}%
\begin{pgfscope}%
\pgfsetbuttcap%
\pgfsetroundjoin%
\definecolor{currentfill}{rgb}{0.000000,0.000000,0.000000}%
\pgfsetfillcolor{currentfill}%
\pgfsetlinewidth{0.803000pt}%
\definecolor{currentstroke}{rgb}{0.000000,0.000000,0.000000}%
\pgfsetstrokecolor{currentstroke}%
\pgfsetdash{}{0pt}%
\pgfsys@defobject{currentmarker}{\pgfqpoint{-0.048611in}{0.000000in}}{\pgfqpoint{0.000000in}{0.000000in}}{%
\pgfpathmoveto{\pgfqpoint{0.000000in}{0.000000in}}%
\pgfpathlineto{\pgfqpoint{-0.048611in}{0.000000in}}%
\pgfusepath{stroke,fill}%
}%
\begin{pgfscope}%
\pgfsys@transformshift{0.970666in}{2.972831in}%
\pgfsys@useobject{currentmarker}{}%
\end{pgfscope}%
\end{pgfscope}%
\begin{pgfscope}%
\definecolor{textcolor}{rgb}{0.000000,0.000000,0.000000}%
\pgfsetstrokecolor{textcolor}%
\pgfsetfillcolor{textcolor}%
\pgftext[x=0.665110in, y=2.920069in, left, base]{\color{textcolor}\rmfamily\fontsize{10.000000}{12.000000}\selectfont \(\displaystyle 175\)}%
\end{pgfscope}%
\begin{pgfscope}%
\pgfsetrectcap%
\pgfsetmiterjoin%
\pgfsetlinewidth{0.803000pt}%
\definecolor{currentstroke}{rgb}{0.000000,0.000000,0.000000}%
\pgfsetstrokecolor{currentstroke}%
\pgfsetdash{}{0pt}%
\pgfpathmoveto{\pgfqpoint{0.970666in}{0.566125in}}%
\pgfpathlineto{\pgfqpoint{0.970666in}{3.251557in}}%
\pgfusepath{stroke}%
\end{pgfscope}%
\begin{pgfscope}%
\pgfsetrectcap%
\pgfsetmiterjoin%
\pgfsetlinewidth{0.803000pt}%
\definecolor{currentstroke}{rgb}{0.000000,0.000000,0.000000}%
\pgfsetstrokecolor{currentstroke}%
\pgfsetdash{}{0pt}%
\pgfpathmoveto{\pgfqpoint{6.669922in}{0.566125in}}%
\pgfpathlineto{\pgfqpoint{6.669922in}{3.251557in}}%
\pgfusepath{stroke}%
\end{pgfscope}%
\begin{pgfscope}%
\pgfsetrectcap%
\pgfsetmiterjoin%
\pgfsetlinewidth{0.803000pt}%
\definecolor{currentstroke}{rgb}{0.000000,0.000000,0.000000}%
\pgfsetstrokecolor{currentstroke}%
\pgfsetdash{}{0pt}%
\pgfpathmoveto{\pgfqpoint{0.970666in}{0.566125in}}%
\pgfpathlineto{\pgfqpoint{6.669922in}{0.566125in}}%
\pgfusepath{stroke}%
\end{pgfscope}%
\begin{pgfscope}%
\pgfsetrectcap%
\pgfsetmiterjoin%
\pgfsetlinewidth{0.803000pt}%
\definecolor{currentstroke}{rgb}{0.000000,0.000000,0.000000}%
\pgfsetstrokecolor{currentstroke}%
\pgfsetdash{}{0pt}%
\pgfpathmoveto{\pgfqpoint{0.970666in}{3.251557in}}%
\pgfpathlineto{\pgfqpoint{6.669922in}{3.251557in}}%
\pgfusepath{stroke}%
\end{pgfscope}%
\begin{pgfscope}%
\definecolor{textcolor}{rgb}{0.000000,0.000000,0.000000}%
\pgfsetstrokecolor{textcolor}%
\pgfsetfillcolor{textcolor}%
\pgftext[x=3.820294in,y=3.334890in,,base]{\color{textcolor}\rmfamily\fontsize{12.000000}{14.400000}\selectfont greedy 1706.2}%
\end{pgfscope}%
\begin{pgfscope}%
\pgfsetbuttcap%
\pgfsetmiterjoin%
\definecolor{currentfill}{rgb}{1.000000,1.000000,1.000000}%
\pgfsetfillcolor{currentfill}%
\pgfsetlinewidth{0.000000pt}%
\definecolor{currentstroke}{rgb}{0.000000,0.000000,0.000000}%
\pgfsetstrokecolor{currentstroke}%
\pgfsetstrokeopacity{0.000000}%
\pgfsetdash{}{0pt}%
\pgfpathmoveto{\pgfqpoint{7.640588in}{0.566125in}}%
\pgfpathlineto{\pgfqpoint{13.339844in}{0.566125in}}%
\pgfpathlineto{\pgfqpoint{13.339844in}{3.251557in}}%
\pgfpathlineto{\pgfqpoint{7.640588in}{3.251557in}}%
\pgfpathclose%
\pgfusepath{fill}%
\end{pgfscope}%
\begin{pgfscope}%
\pgfpathrectangle{\pgfqpoint{7.640588in}{0.566125in}}{\pgfqpoint{5.699255in}{2.685432in}}%
\pgfusepath{clip}%
\pgfsetrectcap%
\pgfsetroundjoin%
\pgfsetlinewidth{1.505625pt}%
\definecolor{currentstroke}{rgb}{0.000000,0.000000,0.000000}%
\pgfsetstrokecolor{currentstroke}%
\pgfsetdash{}{0pt}%
\pgfpathmoveto{\pgfqpoint{12.996768in}{2.267856in}}%
\pgfpathlineto{\pgfqpoint{13.080787in}{2.424517in}}%
\pgfpathlineto{\pgfqpoint{12.660694in}{2.711729in}}%
\pgfpathlineto{\pgfqpoint{12.576676in}{2.803114in}}%
\pgfpathlineto{\pgfqpoint{12.772719in}{2.855335in}}%
\pgfpathlineto{\pgfqpoint{12.912750in}{3.116437in}}%
\pgfpathlineto{\pgfqpoint{12.380632in}{3.129492in}}%
\pgfpathlineto{\pgfqpoint{11.988546in}{2.842280in}}%
\pgfpathlineto{\pgfqpoint{11.904528in}{2.868390in}}%
\pgfpathlineto{\pgfqpoint{11.652472in}{2.946720in}}%
\pgfpathlineto{\pgfqpoint{11.512441in}{2.998941in}}%
\pgfpathlineto{\pgfqpoint{11.372410in}{2.933665in}}%
\pgfpathlineto{\pgfqpoint{10.616244in}{3.077271in}}%
\pgfpathlineto{\pgfqpoint{10.560231in}{2.737839in}}%
\pgfpathlineto{\pgfqpoint{10.224157in}{2.894500in}}%
\pgfpathlineto{\pgfqpoint{10.196151in}{2.829225in}}%
\pgfpathlineto{\pgfqpoint{9.748053in}{2.607288in}}%
\pgfpathlineto{\pgfqpoint{9.832071in}{2.920610in}}%
\pgfpathlineto{\pgfqpoint{9.131917in}{3.116437in}}%
\pgfpathlineto{\pgfqpoint{8.403756in}{2.842280in}}%
\pgfpathlineto{\pgfqpoint{8.039676in}{2.855335in}}%
\pgfpathlineto{\pgfqpoint{7.899645in}{2.777004in}}%
\pgfpathlineto{\pgfqpoint{7.955658in}{2.711729in}}%
\pgfpathlineto{\pgfqpoint{8.515781in}{2.724784in}}%
\pgfpathlineto{\pgfqpoint{9.047898in}{2.868390in}}%
\pgfpathlineto{\pgfqpoint{9.271948in}{2.750894in}}%
\pgfpathlineto{\pgfqpoint{9.047898in}{2.698674in}}%
\pgfpathlineto{\pgfqpoint{8.879861in}{2.620343in}}%
\pgfpathlineto{\pgfqpoint{8.067682in}{2.424517in}}%
\pgfpathlineto{\pgfqpoint{8.347744in}{2.424517in}}%
\pgfpathlineto{\pgfqpoint{8.655812in}{2.307021in}}%
\pgfpathlineto{\pgfqpoint{8.823849in}{2.476737in}}%
\pgfpathlineto{\pgfqpoint{9.580016in}{2.385351in}}%
\pgfpathlineto{\pgfqpoint{9.580016in}{2.254801in}}%
\pgfpathlineto{\pgfqpoint{9.383972in}{2.111195in}}%
\pgfpathlineto{\pgfqpoint{9.355966in}{2.085084in}}%
\pgfpathlineto{\pgfqpoint{9.271948in}{2.006754in}}%
\pgfpathlineto{\pgfqpoint{9.075905in}{2.032864in}}%
\pgfpathlineto{\pgfqpoint{8.487775in}{1.850093in}}%
\pgfpathlineto{\pgfqpoint{8.543787in}{1.654266in}}%
\pgfpathlineto{\pgfqpoint{8.571793in}{1.497605in}}%
\pgfpathlineto{\pgfqpoint{8.543787in}{1.393165in}}%
\pgfpathlineto{\pgfqpoint{8.571793in}{1.275669in}}%
\pgfpathlineto{\pgfqpoint{8.179707in}{1.353999in}}%
\pgfpathlineto{\pgfqpoint{8.039676in}{1.393165in}}%
\pgfpathlineto{\pgfqpoint{8.291732in}{1.184283in}}%
\pgfpathlineto{\pgfqpoint{8.599800in}{0.988457in}}%
\pgfpathlineto{\pgfqpoint{8.263726in}{0.831796in}}%
\pgfpathlineto{\pgfqpoint{8.403756in}{0.688190in}}%
\pgfpathlineto{\pgfqpoint{8.515781in}{0.714300in}}%
\pgfpathlineto{\pgfqpoint{9.075905in}{0.936236in}}%
\pgfpathlineto{\pgfqpoint{9.748053in}{0.844851in}}%
\pgfpathlineto{\pgfqpoint{9.075905in}{1.092898in}}%
\pgfpathlineto{\pgfqpoint{9.047898in}{1.184283in}}%
\pgfpathlineto{\pgfqpoint{9.299954in}{1.275669in}}%
\pgfpathlineto{\pgfqpoint{9.327960in}{1.301779in}}%
\pgfpathlineto{\pgfqpoint{10.140139in}{1.432330in}}%
\pgfpathlineto{\pgfqpoint{10.364188in}{1.536771in}}%
\pgfpathlineto{\pgfqpoint{10.560231in}{0.897071in}}%
\pgfpathlineto{\pgfqpoint{10.700262in}{0.831796in}}%
\pgfpathlineto{\pgfqpoint{10.756275in}{0.884016in}}%
\pgfpathlineto{\pgfqpoint{11.260386in}{0.923181in}}%
\pgfpathlineto{\pgfqpoint{11.568454in}{0.923181in}}%
\pgfpathlineto{\pgfqpoint{11.568454in}{0.936236in}}%
\pgfpathlineto{\pgfqpoint{11.904528in}{0.975402in}}%
\pgfpathlineto{\pgfqpoint{11.932534in}{0.988457in}}%
\pgfpathlineto{\pgfqpoint{11.932534in}{1.079842in}}%
\pgfpathlineto{\pgfqpoint{11.988546in}{1.027622in}}%
\pgfpathlineto{\pgfqpoint{12.660694in}{0.884016in}}%
\pgfpathlineto{\pgfqpoint{12.296614in}{1.079842in}}%
\pgfpathlineto{\pgfqpoint{12.184589in}{1.223448in}}%
\pgfpathlineto{\pgfqpoint{12.072565in}{1.458440in}}%
\pgfpathlineto{\pgfqpoint{11.652472in}{1.510660in}}%
\pgfpathlineto{\pgfqpoint{11.400417in}{1.432330in}}%
\pgfpathlineto{\pgfqpoint{11.176367in}{1.510660in}}%
\pgfpathlineto{\pgfqpoint{11.232380in}{1.719542in}}%
\pgfpathlineto{\pgfqpoint{10.812287in}{1.562881in}}%
\pgfpathlineto{\pgfqpoint{10.728268in}{1.549826in}}%
\pgfpathlineto{\pgfqpoint{10.476213in}{1.602046in}}%
\pgfpathlineto{\pgfqpoint{10.476213in}{1.863148in}}%
\pgfpathlineto{\pgfqpoint{9.860077in}{1.915368in}}%
\pgfpathlineto{\pgfqpoint{9.580016in}{1.784817in}}%
\pgfpathlineto{\pgfqpoint{9.608022in}{1.850093in}}%
\pgfpathlineto{\pgfqpoint{10.112133in}{2.254801in}}%
\pgfpathlineto{\pgfqpoint{10.084127in}{2.385351in}}%
\pgfpathlineto{\pgfqpoint{10.616244in}{2.450627in}}%
\pgfpathlineto{\pgfqpoint{10.896306in}{2.359241in}}%
\pgfpathlineto{\pgfqpoint{10.980324in}{2.293966in}}%
\pgfpathlineto{\pgfqpoint{12.100571in}{2.241745in}}%
\pgfpathlineto{\pgfqpoint{11.988546in}{2.333131in}}%
\pgfpathlineto{\pgfqpoint{12.156583in}{2.463682in}}%
\pgfpathlineto{\pgfqpoint{12.240602in}{2.502847in}}%
\pgfpathlineto{\pgfqpoint{12.576676in}{2.372296in}}%
\pgfpathlineto{\pgfqpoint{12.464651in}{2.006754in}}%
\pgfpathlineto{\pgfqpoint{12.884744in}{1.667321in}}%
\pgfpathlineto{\pgfqpoint{12.996768in}{2.267856in}}%
\pgfusepath{stroke}%
\end{pgfscope}%
\begin{pgfscope}%
\pgfpathrectangle{\pgfqpoint{7.640588in}{0.566125in}}{\pgfqpoint{5.699255in}{2.685432in}}%
\pgfusepath{clip}%
\pgfsetbuttcap%
\pgfsetroundjoin%
\definecolor{currentfill}{rgb}{1.000000,0.000000,0.000000}%
\pgfsetfillcolor{currentfill}%
\pgfsetlinewidth{1.003750pt}%
\definecolor{currentstroke}{rgb}{1.000000,0.000000,0.000000}%
\pgfsetstrokecolor{currentstroke}%
\pgfsetdash{}{0pt}%
\pgfpathmoveto{\pgfqpoint{12.996768in}{2.226189in}}%
\pgfpathcurveto{\pgfqpoint{13.007818in}{2.226189in}}{\pgfqpoint{13.018417in}{2.230579in}}{\pgfqpoint{13.026231in}{2.238393in}}%
\pgfpathcurveto{\pgfqpoint{13.034045in}{2.246206in}}{\pgfqpoint{13.038435in}{2.256806in}}{\pgfqpoint{13.038435in}{2.267856in}}%
\pgfpathcurveto{\pgfqpoint{13.038435in}{2.278906in}}{\pgfqpoint{13.034045in}{2.289505in}}{\pgfqpoint{13.026231in}{2.297318in}}%
\pgfpathcurveto{\pgfqpoint{13.018417in}{2.305132in}}{\pgfqpoint{13.007818in}{2.309522in}}{\pgfqpoint{12.996768in}{2.309522in}}%
\pgfpathcurveto{\pgfqpoint{12.985718in}{2.309522in}}{\pgfqpoint{12.975119in}{2.305132in}}{\pgfqpoint{12.967305in}{2.297318in}}%
\pgfpathcurveto{\pgfqpoint{12.959492in}{2.289505in}}{\pgfqpoint{12.955102in}{2.278906in}}{\pgfqpoint{12.955102in}{2.267856in}}%
\pgfpathcurveto{\pgfqpoint{12.955102in}{2.256806in}}{\pgfqpoint{12.959492in}{2.246206in}}{\pgfqpoint{12.967305in}{2.238393in}}%
\pgfpathcurveto{\pgfqpoint{12.975119in}{2.230579in}}{\pgfqpoint{12.985718in}{2.226189in}}{\pgfqpoint{12.996768in}{2.226189in}}%
\pgfpathclose%
\pgfusepath{stroke,fill}%
\end{pgfscope}%
\begin{pgfscope}%
\pgfpathrectangle{\pgfqpoint{7.640588in}{0.566125in}}{\pgfqpoint{5.699255in}{2.685432in}}%
\pgfusepath{clip}%
\pgfsetbuttcap%
\pgfsetroundjoin%
\definecolor{currentfill}{rgb}{0.750000,0.750000,0.000000}%
\pgfsetfillcolor{currentfill}%
\pgfsetlinewidth{1.003750pt}%
\definecolor{currentstroke}{rgb}{0.750000,0.750000,0.000000}%
\pgfsetstrokecolor{currentstroke}%
\pgfsetdash{}{0pt}%
\pgfpathmoveto{\pgfqpoint{13.080787in}{2.382850in}}%
\pgfpathcurveto{\pgfqpoint{13.091837in}{2.382850in}}{\pgfqpoint{13.102436in}{2.387240in}}{\pgfqpoint{13.110249in}{2.395054in}}%
\pgfpathcurveto{\pgfqpoint{13.118063in}{2.402868in}}{\pgfqpoint{13.122453in}{2.413467in}}{\pgfqpoint{13.122453in}{2.424517in}}%
\pgfpathcurveto{\pgfqpoint{13.122453in}{2.435567in}}{\pgfqpoint{13.118063in}{2.446166in}}{\pgfqpoint{13.110249in}{2.453980in}}%
\pgfpathcurveto{\pgfqpoint{13.102436in}{2.461793in}}{\pgfqpoint{13.091837in}{2.466183in}}{\pgfqpoint{13.080787in}{2.466183in}}%
\pgfpathcurveto{\pgfqpoint{13.069737in}{2.466183in}}{\pgfqpoint{13.059138in}{2.461793in}}{\pgfqpoint{13.051324in}{2.453980in}}%
\pgfpathcurveto{\pgfqpoint{13.043510in}{2.446166in}}{\pgfqpoint{13.039120in}{2.435567in}}{\pgfqpoint{13.039120in}{2.424517in}}%
\pgfpathcurveto{\pgfqpoint{13.039120in}{2.413467in}}{\pgfqpoint{13.043510in}{2.402868in}}{\pgfqpoint{13.051324in}{2.395054in}}%
\pgfpathcurveto{\pgfqpoint{13.059138in}{2.387240in}}{\pgfqpoint{13.069737in}{2.382850in}}{\pgfqpoint{13.080787in}{2.382850in}}%
\pgfpathclose%
\pgfusepath{stroke,fill}%
\end{pgfscope}%
\begin{pgfscope}%
\pgfpathrectangle{\pgfqpoint{7.640588in}{0.566125in}}{\pgfqpoint{5.699255in}{2.685432in}}%
\pgfusepath{clip}%
\pgfsetbuttcap%
\pgfsetroundjoin%
\definecolor{currentfill}{rgb}{0.000000,0.000000,0.000000}%
\pgfsetfillcolor{currentfill}%
\pgfsetlinewidth{1.003750pt}%
\definecolor{currentstroke}{rgb}{0.000000,0.000000,0.000000}%
\pgfsetstrokecolor{currentstroke}%
\pgfsetdash{}{0pt}%
\pgfpathmoveto{\pgfqpoint{12.660694in}{2.670062in}}%
\pgfpathcurveto{\pgfqpoint{12.671744in}{2.670062in}}{\pgfqpoint{12.682343in}{2.674452in}}{\pgfqpoint{12.690157in}{2.682266in}}%
\pgfpathcurveto{\pgfqpoint{12.697971in}{2.690080in}}{\pgfqpoint{12.702361in}{2.700679in}}{\pgfqpoint{12.702361in}{2.711729in}}%
\pgfpathcurveto{\pgfqpoint{12.702361in}{2.722779in}}{\pgfqpoint{12.697971in}{2.733378in}}{\pgfqpoint{12.690157in}{2.741192in}}%
\pgfpathcurveto{\pgfqpoint{12.682343in}{2.749005in}}{\pgfqpoint{12.671744in}{2.753395in}}{\pgfqpoint{12.660694in}{2.753395in}}%
\pgfpathcurveto{\pgfqpoint{12.649644in}{2.753395in}}{\pgfqpoint{12.639045in}{2.749005in}}{\pgfqpoint{12.631231in}{2.741192in}}%
\pgfpathcurveto{\pgfqpoint{12.623418in}{2.733378in}}{\pgfqpoint{12.619027in}{2.722779in}}{\pgfqpoint{12.619027in}{2.711729in}}%
\pgfpathcurveto{\pgfqpoint{12.619027in}{2.700679in}}{\pgfqpoint{12.623418in}{2.690080in}}{\pgfqpoint{12.631231in}{2.682266in}}%
\pgfpathcurveto{\pgfqpoint{12.639045in}{2.674452in}}{\pgfqpoint{12.649644in}{2.670062in}}{\pgfqpoint{12.660694in}{2.670062in}}%
\pgfpathclose%
\pgfusepath{stroke,fill}%
\end{pgfscope}%
\begin{pgfscope}%
\pgfpathrectangle{\pgfqpoint{7.640588in}{0.566125in}}{\pgfqpoint{5.699255in}{2.685432in}}%
\pgfusepath{clip}%
\pgfsetbuttcap%
\pgfsetroundjoin%
\definecolor{currentfill}{rgb}{0.000000,0.000000,0.000000}%
\pgfsetfillcolor{currentfill}%
\pgfsetlinewidth{1.003750pt}%
\definecolor{currentstroke}{rgb}{0.000000,0.000000,0.000000}%
\pgfsetstrokecolor{currentstroke}%
\pgfsetdash{}{0pt}%
\pgfpathmoveto{\pgfqpoint{12.576676in}{2.761448in}}%
\pgfpathcurveto{\pgfqpoint{12.587726in}{2.761448in}}{\pgfqpoint{12.598325in}{2.765838in}}{\pgfqpoint{12.606138in}{2.773652in}}%
\pgfpathcurveto{\pgfqpoint{12.613952in}{2.781465in}}{\pgfqpoint{12.618342in}{2.792064in}}{\pgfqpoint{12.618342in}{2.803114in}}%
\pgfpathcurveto{\pgfqpoint{12.618342in}{2.814164in}}{\pgfqpoint{12.613952in}{2.824764in}}{\pgfqpoint{12.606138in}{2.832577in}}%
\pgfpathcurveto{\pgfqpoint{12.598325in}{2.840391in}}{\pgfqpoint{12.587726in}{2.844781in}}{\pgfqpoint{12.576676in}{2.844781in}}%
\pgfpathcurveto{\pgfqpoint{12.565626in}{2.844781in}}{\pgfqpoint{12.555026in}{2.840391in}}{\pgfqpoint{12.547213in}{2.832577in}}%
\pgfpathcurveto{\pgfqpoint{12.539399in}{2.824764in}}{\pgfqpoint{12.535009in}{2.814164in}}{\pgfqpoint{12.535009in}{2.803114in}}%
\pgfpathcurveto{\pgfqpoint{12.535009in}{2.792064in}}{\pgfqpoint{12.539399in}{2.781465in}}{\pgfqpoint{12.547213in}{2.773652in}}%
\pgfpathcurveto{\pgfqpoint{12.555026in}{2.765838in}}{\pgfqpoint{12.565626in}{2.761448in}}{\pgfqpoint{12.576676in}{2.761448in}}%
\pgfpathclose%
\pgfusepath{stroke,fill}%
\end{pgfscope}%
\begin{pgfscope}%
\pgfpathrectangle{\pgfqpoint{7.640588in}{0.566125in}}{\pgfqpoint{5.699255in}{2.685432in}}%
\pgfusepath{clip}%
\pgfsetbuttcap%
\pgfsetroundjoin%
\definecolor{currentfill}{rgb}{0.000000,0.000000,0.000000}%
\pgfsetfillcolor{currentfill}%
\pgfsetlinewidth{1.003750pt}%
\definecolor{currentstroke}{rgb}{0.000000,0.000000,0.000000}%
\pgfsetstrokecolor{currentstroke}%
\pgfsetdash{}{0pt}%
\pgfpathmoveto{\pgfqpoint{12.772719in}{2.813668in}}%
\pgfpathcurveto{\pgfqpoint{12.783769in}{2.813668in}}{\pgfqpoint{12.794368in}{2.818058in}}{\pgfqpoint{12.802182in}{2.825872in}}%
\pgfpathcurveto{\pgfqpoint{12.809995in}{2.833686in}}{\pgfqpoint{12.814385in}{2.844285in}}{\pgfqpoint{12.814385in}{2.855335in}}%
\pgfpathcurveto{\pgfqpoint{12.814385in}{2.866385in}}{\pgfqpoint{12.809995in}{2.876984in}}{\pgfqpoint{12.802182in}{2.884798in}}%
\pgfpathcurveto{\pgfqpoint{12.794368in}{2.892611in}}{\pgfqpoint{12.783769in}{2.897001in}}{\pgfqpoint{12.772719in}{2.897001in}}%
\pgfpathcurveto{\pgfqpoint{12.761669in}{2.897001in}}{\pgfqpoint{12.751070in}{2.892611in}}{\pgfqpoint{12.743256in}{2.884798in}}%
\pgfpathcurveto{\pgfqpoint{12.735442in}{2.876984in}}{\pgfqpoint{12.731052in}{2.866385in}}{\pgfqpoint{12.731052in}{2.855335in}}%
\pgfpathcurveto{\pgfqpoint{12.731052in}{2.844285in}}{\pgfqpoint{12.735442in}{2.833686in}}{\pgfqpoint{12.743256in}{2.825872in}}%
\pgfpathcurveto{\pgfqpoint{12.751070in}{2.818058in}}{\pgfqpoint{12.761669in}{2.813668in}}{\pgfqpoint{12.772719in}{2.813668in}}%
\pgfpathclose%
\pgfusepath{stroke,fill}%
\end{pgfscope}%
\begin{pgfscope}%
\pgfpathrectangle{\pgfqpoint{7.640588in}{0.566125in}}{\pgfqpoint{5.699255in}{2.685432in}}%
\pgfusepath{clip}%
\pgfsetbuttcap%
\pgfsetroundjoin%
\definecolor{currentfill}{rgb}{0.000000,0.000000,0.000000}%
\pgfsetfillcolor{currentfill}%
\pgfsetlinewidth{1.003750pt}%
\definecolor{currentstroke}{rgb}{0.000000,0.000000,0.000000}%
\pgfsetstrokecolor{currentstroke}%
\pgfsetdash{}{0pt}%
\pgfpathmoveto{\pgfqpoint{12.912750in}{3.074770in}}%
\pgfpathcurveto{\pgfqpoint{12.923800in}{3.074770in}}{\pgfqpoint{12.934399in}{3.079160in}}{\pgfqpoint{12.942212in}{3.086974in}}%
\pgfpathcurveto{\pgfqpoint{12.950026in}{3.094787in}}{\pgfqpoint{12.954416in}{3.105386in}}{\pgfqpoint{12.954416in}{3.116437in}}%
\pgfpathcurveto{\pgfqpoint{12.954416in}{3.127487in}}{\pgfqpoint{12.950026in}{3.138086in}}{\pgfqpoint{12.942212in}{3.145899in}}%
\pgfpathcurveto{\pgfqpoint{12.934399in}{3.153713in}}{\pgfqpoint{12.923800in}{3.158103in}}{\pgfqpoint{12.912750in}{3.158103in}}%
\pgfpathcurveto{\pgfqpoint{12.901700in}{3.158103in}}{\pgfqpoint{12.891101in}{3.153713in}}{\pgfqpoint{12.883287in}{3.145899in}}%
\pgfpathcurveto{\pgfqpoint{12.875473in}{3.138086in}}{\pgfqpoint{12.871083in}{3.127487in}}{\pgfqpoint{12.871083in}{3.116437in}}%
\pgfpathcurveto{\pgfqpoint{12.871083in}{3.105386in}}{\pgfqpoint{12.875473in}{3.094787in}}{\pgfqpoint{12.883287in}{3.086974in}}%
\pgfpathcurveto{\pgfqpoint{12.891101in}{3.079160in}}{\pgfqpoint{12.901700in}{3.074770in}}{\pgfqpoint{12.912750in}{3.074770in}}%
\pgfpathclose%
\pgfusepath{stroke,fill}%
\end{pgfscope}%
\begin{pgfscope}%
\pgfpathrectangle{\pgfqpoint{7.640588in}{0.566125in}}{\pgfqpoint{5.699255in}{2.685432in}}%
\pgfusepath{clip}%
\pgfsetbuttcap%
\pgfsetroundjoin%
\definecolor{currentfill}{rgb}{0.000000,0.000000,0.000000}%
\pgfsetfillcolor{currentfill}%
\pgfsetlinewidth{1.003750pt}%
\definecolor{currentstroke}{rgb}{0.000000,0.000000,0.000000}%
\pgfsetstrokecolor{currentstroke}%
\pgfsetdash{}{0pt}%
\pgfpathmoveto{\pgfqpoint{12.380632in}{3.087825in}}%
\pgfpathcurveto{\pgfqpoint{12.391683in}{3.087825in}}{\pgfqpoint{12.402282in}{3.092215in}}{\pgfqpoint{12.410095in}{3.100029in}}%
\pgfpathcurveto{\pgfqpoint{12.417909in}{3.107842in}}{\pgfqpoint{12.422299in}{3.118441in}}{\pgfqpoint{12.422299in}{3.129492in}}%
\pgfpathcurveto{\pgfqpoint{12.422299in}{3.140542in}}{\pgfqpoint{12.417909in}{3.151141in}}{\pgfqpoint{12.410095in}{3.158954in}}%
\pgfpathcurveto{\pgfqpoint{12.402282in}{3.166768in}}{\pgfqpoint{12.391683in}{3.171158in}}{\pgfqpoint{12.380632in}{3.171158in}}%
\pgfpathcurveto{\pgfqpoint{12.369582in}{3.171158in}}{\pgfqpoint{12.358983in}{3.166768in}}{\pgfqpoint{12.351170in}{3.158954in}}%
\pgfpathcurveto{\pgfqpoint{12.343356in}{3.151141in}}{\pgfqpoint{12.338966in}{3.140542in}}{\pgfqpoint{12.338966in}{3.129492in}}%
\pgfpathcurveto{\pgfqpoint{12.338966in}{3.118441in}}{\pgfqpoint{12.343356in}{3.107842in}}{\pgfqpoint{12.351170in}{3.100029in}}%
\pgfpathcurveto{\pgfqpoint{12.358983in}{3.092215in}}{\pgfqpoint{12.369582in}{3.087825in}}{\pgfqpoint{12.380632in}{3.087825in}}%
\pgfpathclose%
\pgfusepath{stroke,fill}%
\end{pgfscope}%
\begin{pgfscope}%
\pgfpathrectangle{\pgfqpoint{7.640588in}{0.566125in}}{\pgfqpoint{5.699255in}{2.685432in}}%
\pgfusepath{clip}%
\pgfsetbuttcap%
\pgfsetroundjoin%
\definecolor{currentfill}{rgb}{0.000000,0.000000,0.000000}%
\pgfsetfillcolor{currentfill}%
\pgfsetlinewidth{1.003750pt}%
\definecolor{currentstroke}{rgb}{0.000000,0.000000,0.000000}%
\pgfsetstrokecolor{currentstroke}%
\pgfsetdash{}{0pt}%
\pgfpathmoveto{\pgfqpoint{11.988546in}{2.800613in}}%
\pgfpathcurveto{\pgfqpoint{11.999596in}{2.800613in}}{\pgfqpoint{12.010195in}{2.805003in}}{\pgfqpoint{12.018009in}{2.812817in}}%
\pgfpathcurveto{\pgfqpoint{12.025823in}{2.820630in}}{\pgfqpoint{12.030213in}{2.831229in}}{\pgfqpoint{12.030213in}{2.842280in}}%
\pgfpathcurveto{\pgfqpoint{12.030213in}{2.853330in}}{\pgfqpoint{12.025823in}{2.863929in}}{\pgfqpoint{12.018009in}{2.871742in}}%
\pgfpathcurveto{\pgfqpoint{12.010195in}{2.879556in}}{\pgfqpoint{11.999596in}{2.883946in}}{\pgfqpoint{11.988546in}{2.883946in}}%
\pgfpathcurveto{\pgfqpoint{11.977496in}{2.883946in}}{\pgfqpoint{11.966897in}{2.879556in}}{\pgfqpoint{11.959083in}{2.871742in}}%
\pgfpathcurveto{\pgfqpoint{11.951270in}{2.863929in}}{\pgfqpoint{11.946879in}{2.853330in}}{\pgfqpoint{11.946879in}{2.842280in}}%
\pgfpathcurveto{\pgfqpoint{11.946879in}{2.831229in}}{\pgfqpoint{11.951270in}{2.820630in}}{\pgfqpoint{11.959083in}{2.812817in}}%
\pgfpathcurveto{\pgfqpoint{11.966897in}{2.805003in}}{\pgfqpoint{11.977496in}{2.800613in}}{\pgfqpoint{11.988546in}{2.800613in}}%
\pgfpathclose%
\pgfusepath{stroke,fill}%
\end{pgfscope}%
\begin{pgfscope}%
\pgfpathrectangle{\pgfqpoint{7.640588in}{0.566125in}}{\pgfqpoint{5.699255in}{2.685432in}}%
\pgfusepath{clip}%
\pgfsetbuttcap%
\pgfsetroundjoin%
\definecolor{currentfill}{rgb}{0.000000,0.000000,0.000000}%
\pgfsetfillcolor{currentfill}%
\pgfsetlinewidth{1.003750pt}%
\definecolor{currentstroke}{rgb}{0.000000,0.000000,0.000000}%
\pgfsetstrokecolor{currentstroke}%
\pgfsetdash{}{0pt}%
\pgfpathmoveto{\pgfqpoint{11.904528in}{2.826723in}}%
\pgfpathcurveto{\pgfqpoint{11.915578in}{2.826723in}}{\pgfqpoint{11.926177in}{2.831113in}}{\pgfqpoint{11.933990in}{2.838927in}}%
\pgfpathcurveto{\pgfqpoint{11.941804in}{2.846741in}}{\pgfqpoint{11.946194in}{2.857340in}}{\pgfqpoint{11.946194in}{2.868390in}}%
\pgfpathcurveto{\pgfqpoint{11.946194in}{2.879440in}}{\pgfqpoint{11.941804in}{2.890039in}}{\pgfqpoint{11.933990in}{2.897853in}}%
\pgfpathcurveto{\pgfqpoint{11.926177in}{2.905666in}}{\pgfqpoint{11.915578in}{2.910056in}}{\pgfqpoint{11.904528in}{2.910056in}}%
\pgfpathcurveto{\pgfqpoint{11.893477in}{2.910056in}}{\pgfqpoint{11.882878in}{2.905666in}}{\pgfqpoint{11.875065in}{2.897853in}}%
\pgfpathcurveto{\pgfqpoint{11.867251in}{2.890039in}}{\pgfqpoint{11.862861in}{2.879440in}}{\pgfqpoint{11.862861in}{2.868390in}}%
\pgfpathcurveto{\pgfqpoint{11.862861in}{2.857340in}}{\pgfqpoint{11.867251in}{2.846741in}}{\pgfqpoint{11.875065in}{2.838927in}}%
\pgfpathcurveto{\pgfqpoint{11.882878in}{2.831113in}}{\pgfqpoint{11.893477in}{2.826723in}}{\pgfqpoint{11.904528in}{2.826723in}}%
\pgfpathclose%
\pgfusepath{stroke,fill}%
\end{pgfscope}%
\begin{pgfscope}%
\pgfpathrectangle{\pgfqpoint{7.640588in}{0.566125in}}{\pgfqpoint{5.699255in}{2.685432in}}%
\pgfusepath{clip}%
\pgfsetbuttcap%
\pgfsetroundjoin%
\definecolor{currentfill}{rgb}{0.000000,0.000000,0.000000}%
\pgfsetfillcolor{currentfill}%
\pgfsetlinewidth{1.003750pt}%
\definecolor{currentstroke}{rgb}{0.000000,0.000000,0.000000}%
\pgfsetstrokecolor{currentstroke}%
\pgfsetdash{}{0pt}%
\pgfpathmoveto{\pgfqpoint{11.652472in}{2.905054in}}%
\pgfpathcurveto{\pgfqpoint{11.663522in}{2.905054in}}{\pgfqpoint{11.674121in}{2.909444in}}{\pgfqpoint{11.681935in}{2.917258in}}%
\pgfpathcurveto{\pgfqpoint{11.689748in}{2.925071in}}{\pgfqpoint{11.694139in}{2.935670in}}{\pgfqpoint{11.694139in}{2.946720in}}%
\pgfpathcurveto{\pgfqpoint{11.694139in}{2.957770in}}{\pgfqpoint{11.689748in}{2.968370in}}{\pgfqpoint{11.681935in}{2.976183in}}%
\pgfpathcurveto{\pgfqpoint{11.674121in}{2.983997in}}{\pgfqpoint{11.663522in}{2.988387in}}{\pgfqpoint{11.652472in}{2.988387in}}%
\pgfpathcurveto{\pgfqpoint{11.641422in}{2.988387in}}{\pgfqpoint{11.630823in}{2.983997in}}{\pgfqpoint{11.623009in}{2.976183in}}%
\pgfpathcurveto{\pgfqpoint{11.615196in}{2.968370in}}{\pgfqpoint{11.610805in}{2.957770in}}{\pgfqpoint{11.610805in}{2.946720in}}%
\pgfpathcurveto{\pgfqpoint{11.610805in}{2.935670in}}{\pgfqpoint{11.615196in}{2.925071in}}{\pgfqpoint{11.623009in}{2.917258in}}%
\pgfpathcurveto{\pgfqpoint{11.630823in}{2.909444in}}{\pgfqpoint{11.641422in}{2.905054in}}{\pgfqpoint{11.652472in}{2.905054in}}%
\pgfpathclose%
\pgfusepath{stroke,fill}%
\end{pgfscope}%
\begin{pgfscope}%
\pgfpathrectangle{\pgfqpoint{7.640588in}{0.566125in}}{\pgfqpoint{5.699255in}{2.685432in}}%
\pgfusepath{clip}%
\pgfsetbuttcap%
\pgfsetroundjoin%
\definecolor{currentfill}{rgb}{0.000000,0.000000,0.000000}%
\pgfsetfillcolor{currentfill}%
\pgfsetlinewidth{1.003750pt}%
\definecolor{currentstroke}{rgb}{0.000000,0.000000,0.000000}%
\pgfsetstrokecolor{currentstroke}%
\pgfsetdash{}{0pt}%
\pgfpathmoveto{\pgfqpoint{11.512441in}{2.957274in}}%
\pgfpathcurveto{\pgfqpoint{11.523491in}{2.957274in}}{\pgfqpoint{11.534090in}{2.961664in}}{\pgfqpoint{11.541904in}{2.969478in}}%
\pgfpathcurveto{\pgfqpoint{11.549718in}{2.977292in}}{\pgfqpoint{11.554108in}{2.987891in}}{\pgfqpoint{11.554108in}{2.998941in}}%
\pgfpathcurveto{\pgfqpoint{11.554108in}{3.009991in}}{\pgfqpoint{11.549718in}{3.020590in}}{\pgfqpoint{11.541904in}{3.028403in}}%
\pgfpathcurveto{\pgfqpoint{11.534090in}{3.036217in}}{\pgfqpoint{11.523491in}{3.040607in}}{\pgfqpoint{11.512441in}{3.040607in}}%
\pgfpathcurveto{\pgfqpoint{11.501391in}{3.040607in}}{\pgfqpoint{11.490792in}{3.036217in}}{\pgfqpoint{11.482978in}{3.028403in}}%
\pgfpathcurveto{\pgfqpoint{11.475165in}{3.020590in}}{\pgfqpoint{11.470775in}{3.009991in}}{\pgfqpoint{11.470775in}{2.998941in}}%
\pgfpathcurveto{\pgfqpoint{11.470775in}{2.987891in}}{\pgfqpoint{11.475165in}{2.977292in}}{\pgfqpoint{11.482978in}{2.969478in}}%
\pgfpathcurveto{\pgfqpoint{11.490792in}{2.961664in}}{\pgfqpoint{11.501391in}{2.957274in}}{\pgfqpoint{11.512441in}{2.957274in}}%
\pgfpathclose%
\pgfusepath{stroke,fill}%
\end{pgfscope}%
\begin{pgfscope}%
\pgfpathrectangle{\pgfqpoint{7.640588in}{0.566125in}}{\pgfqpoint{5.699255in}{2.685432in}}%
\pgfusepath{clip}%
\pgfsetbuttcap%
\pgfsetroundjoin%
\definecolor{currentfill}{rgb}{0.000000,0.000000,0.000000}%
\pgfsetfillcolor{currentfill}%
\pgfsetlinewidth{1.003750pt}%
\definecolor{currentstroke}{rgb}{0.000000,0.000000,0.000000}%
\pgfsetstrokecolor{currentstroke}%
\pgfsetdash{}{0pt}%
\pgfpathmoveto{\pgfqpoint{11.372410in}{2.891999in}}%
\pgfpathcurveto{\pgfqpoint{11.383461in}{2.891999in}}{\pgfqpoint{11.394060in}{2.896389in}}{\pgfqpoint{11.401873in}{2.904202in}}%
\pgfpathcurveto{\pgfqpoint{11.409687in}{2.912016in}}{\pgfqpoint{11.414077in}{2.922615in}}{\pgfqpoint{11.414077in}{2.933665in}}%
\pgfpathcurveto{\pgfqpoint{11.414077in}{2.944715in}}{\pgfqpoint{11.409687in}{2.955314in}}{\pgfqpoint{11.401873in}{2.963128in}}%
\pgfpathcurveto{\pgfqpoint{11.394060in}{2.970942in}}{\pgfqpoint{11.383461in}{2.975332in}}{\pgfqpoint{11.372410in}{2.975332in}}%
\pgfpathcurveto{\pgfqpoint{11.361360in}{2.975332in}}{\pgfqpoint{11.350761in}{2.970942in}}{\pgfqpoint{11.342948in}{2.963128in}}%
\pgfpathcurveto{\pgfqpoint{11.335134in}{2.955314in}}{\pgfqpoint{11.330744in}{2.944715in}}{\pgfqpoint{11.330744in}{2.933665in}}%
\pgfpathcurveto{\pgfqpoint{11.330744in}{2.922615in}}{\pgfqpoint{11.335134in}{2.912016in}}{\pgfqpoint{11.342948in}{2.904202in}}%
\pgfpathcurveto{\pgfqpoint{11.350761in}{2.896389in}}{\pgfqpoint{11.361360in}{2.891999in}}{\pgfqpoint{11.372410in}{2.891999in}}%
\pgfpathclose%
\pgfusepath{stroke,fill}%
\end{pgfscope}%
\begin{pgfscope}%
\pgfpathrectangle{\pgfqpoint{7.640588in}{0.566125in}}{\pgfqpoint{5.699255in}{2.685432in}}%
\pgfusepath{clip}%
\pgfsetbuttcap%
\pgfsetroundjoin%
\definecolor{currentfill}{rgb}{0.000000,0.000000,0.000000}%
\pgfsetfillcolor{currentfill}%
\pgfsetlinewidth{1.003750pt}%
\definecolor{currentstroke}{rgb}{0.000000,0.000000,0.000000}%
\pgfsetstrokecolor{currentstroke}%
\pgfsetdash{}{0pt}%
\pgfpathmoveto{\pgfqpoint{10.616244in}{3.035605in}}%
\pgfpathcurveto{\pgfqpoint{10.627294in}{3.035605in}}{\pgfqpoint{10.637893in}{3.039995in}}{\pgfqpoint{10.645707in}{3.047808in}}%
\pgfpathcurveto{\pgfqpoint{10.653520in}{3.055622in}}{\pgfqpoint{10.657910in}{3.066221in}}{\pgfqpoint{10.657910in}{3.077271in}}%
\pgfpathcurveto{\pgfqpoint{10.657910in}{3.088321in}}{\pgfqpoint{10.653520in}{3.098920in}}{\pgfqpoint{10.645707in}{3.106734in}}%
\pgfpathcurveto{\pgfqpoint{10.637893in}{3.114548in}}{\pgfqpoint{10.627294in}{3.118938in}}{\pgfqpoint{10.616244in}{3.118938in}}%
\pgfpathcurveto{\pgfqpoint{10.605194in}{3.118938in}}{\pgfqpoint{10.594595in}{3.114548in}}{\pgfqpoint{10.586781in}{3.106734in}}%
\pgfpathcurveto{\pgfqpoint{10.578967in}{3.098920in}}{\pgfqpoint{10.574577in}{3.088321in}}{\pgfqpoint{10.574577in}{3.077271in}}%
\pgfpathcurveto{\pgfqpoint{10.574577in}{3.066221in}}{\pgfqpoint{10.578967in}{3.055622in}}{\pgfqpoint{10.586781in}{3.047808in}}%
\pgfpathcurveto{\pgfqpoint{10.594595in}{3.039995in}}{\pgfqpoint{10.605194in}{3.035605in}}{\pgfqpoint{10.616244in}{3.035605in}}%
\pgfpathclose%
\pgfusepath{stroke,fill}%
\end{pgfscope}%
\begin{pgfscope}%
\pgfpathrectangle{\pgfqpoint{7.640588in}{0.566125in}}{\pgfqpoint{5.699255in}{2.685432in}}%
\pgfusepath{clip}%
\pgfsetbuttcap%
\pgfsetroundjoin%
\definecolor{currentfill}{rgb}{0.000000,0.000000,0.000000}%
\pgfsetfillcolor{currentfill}%
\pgfsetlinewidth{1.003750pt}%
\definecolor{currentstroke}{rgb}{0.000000,0.000000,0.000000}%
\pgfsetstrokecolor{currentstroke}%
\pgfsetdash{}{0pt}%
\pgfpathmoveto{\pgfqpoint{10.560231in}{2.696172in}}%
\pgfpathcurveto{\pgfqpoint{10.571282in}{2.696172in}}{\pgfqpoint{10.581881in}{2.700563in}}{\pgfqpoint{10.589694in}{2.708376in}}%
\pgfpathcurveto{\pgfqpoint{10.597508in}{2.716190in}}{\pgfqpoint{10.601898in}{2.726789in}}{\pgfqpoint{10.601898in}{2.737839in}}%
\pgfpathcurveto{\pgfqpoint{10.601898in}{2.748889in}}{\pgfqpoint{10.597508in}{2.759488in}}{\pgfqpoint{10.589694in}{2.767302in}}%
\pgfpathcurveto{\pgfqpoint{10.581881in}{2.775115in}}{\pgfqpoint{10.571282in}{2.779506in}}{\pgfqpoint{10.560231in}{2.779506in}}%
\pgfpathcurveto{\pgfqpoint{10.549181in}{2.779506in}}{\pgfqpoint{10.538582in}{2.775115in}}{\pgfqpoint{10.530769in}{2.767302in}}%
\pgfpathcurveto{\pgfqpoint{10.522955in}{2.759488in}}{\pgfqpoint{10.518565in}{2.748889in}}{\pgfqpoint{10.518565in}{2.737839in}}%
\pgfpathcurveto{\pgfqpoint{10.518565in}{2.726789in}}{\pgfqpoint{10.522955in}{2.716190in}}{\pgfqpoint{10.530769in}{2.708376in}}%
\pgfpathcurveto{\pgfqpoint{10.538582in}{2.700563in}}{\pgfqpoint{10.549181in}{2.696172in}}{\pgfqpoint{10.560231in}{2.696172in}}%
\pgfpathclose%
\pgfusepath{stroke,fill}%
\end{pgfscope}%
\begin{pgfscope}%
\pgfpathrectangle{\pgfqpoint{7.640588in}{0.566125in}}{\pgfqpoint{5.699255in}{2.685432in}}%
\pgfusepath{clip}%
\pgfsetbuttcap%
\pgfsetroundjoin%
\definecolor{currentfill}{rgb}{0.000000,0.000000,0.000000}%
\pgfsetfillcolor{currentfill}%
\pgfsetlinewidth{1.003750pt}%
\definecolor{currentstroke}{rgb}{0.000000,0.000000,0.000000}%
\pgfsetstrokecolor{currentstroke}%
\pgfsetdash{}{0pt}%
\pgfpathmoveto{\pgfqpoint{10.224157in}{2.852833in}}%
\pgfpathcurveto{\pgfqpoint{10.235208in}{2.852833in}}{\pgfqpoint{10.245807in}{2.857224in}}{\pgfqpoint{10.253620in}{2.865037in}}%
\pgfpathcurveto{\pgfqpoint{10.261434in}{2.872851in}}{\pgfqpoint{10.265824in}{2.883450in}}{\pgfqpoint{10.265824in}{2.894500in}}%
\pgfpathcurveto{\pgfqpoint{10.265824in}{2.905550in}}{\pgfqpoint{10.261434in}{2.916149in}}{\pgfqpoint{10.253620in}{2.923963in}}%
\pgfpathcurveto{\pgfqpoint{10.245807in}{2.931776in}}{\pgfqpoint{10.235208in}{2.936167in}}{\pgfqpoint{10.224157in}{2.936167in}}%
\pgfpathcurveto{\pgfqpoint{10.213107in}{2.936167in}}{\pgfqpoint{10.202508in}{2.931776in}}{\pgfqpoint{10.194695in}{2.923963in}}%
\pgfpathcurveto{\pgfqpoint{10.186881in}{2.916149in}}{\pgfqpoint{10.182491in}{2.905550in}}{\pgfqpoint{10.182491in}{2.894500in}}%
\pgfpathcurveto{\pgfqpoint{10.182491in}{2.883450in}}{\pgfqpoint{10.186881in}{2.872851in}}{\pgfqpoint{10.194695in}{2.865037in}}%
\pgfpathcurveto{\pgfqpoint{10.202508in}{2.857224in}}{\pgfqpoint{10.213107in}{2.852833in}}{\pgfqpoint{10.224157in}{2.852833in}}%
\pgfpathclose%
\pgfusepath{stroke,fill}%
\end{pgfscope}%
\begin{pgfscope}%
\pgfpathrectangle{\pgfqpoint{7.640588in}{0.566125in}}{\pgfqpoint{5.699255in}{2.685432in}}%
\pgfusepath{clip}%
\pgfsetbuttcap%
\pgfsetroundjoin%
\definecolor{currentfill}{rgb}{0.000000,0.000000,0.000000}%
\pgfsetfillcolor{currentfill}%
\pgfsetlinewidth{1.003750pt}%
\definecolor{currentstroke}{rgb}{0.000000,0.000000,0.000000}%
\pgfsetstrokecolor{currentstroke}%
\pgfsetdash{}{0pt}%
\pgfpathmoveto{\pgfqpoint{10.196151in}{2.787558in}}%
\pgfpathcurveto{\pgfqpoint{10.207201in}{2.787558in}}{\pgfqpoint{10.217800in}{2.791948in}}{\pgfqpoint{10.225614in}{2.799762in}}%
\pgfpathcurveto{\pgfqpoint{10.233428in}{2.807575in}}{\pgfqpoint{10.237818in}{2.818174in}}{\pgfqpoint{10.237818in}{2.829225in}}%
\pgfpathcurveto{\pgfqpoint{10.237818in}{2.840275in}}{\pgfqpoint{10.233428in}{2.850874in}}{\pgfqpoint{10.225614in}{2.858687in}}%
\pgfpathcurveto{\pgfqpoint{10.217800in}{2.866501in}}{\pgfqpoint{10.207201in}{2.870891in}}{\pgfqpoint{10.196151in}{2.870891in}}%
\pgfpathcurveto{\pgfqpoint{10.185101in}{2.870891in}}{\pgfqpoint{10.174502in}{2.866501in}}{\pgfqpoint{10.166689in}{2.858687in}}%
\pgfpathcurveto{\pgfqpoint{10.158875in}{2.850874in}}{\pgfqpoint{10.154485in}{2.840275in}}{\pgfqpoint{10.154485in}{2.829225in}}%
\pgfpathcurveto{\pgfqpoint{10.154485in}{2.818174in}}{\pgfqpoint{10.158875in}{2.807575in}}{\pgfqpoint{10.166689in}{2.799762in}}%
\pgfpathcurveto{\pgfqpoint{10.174502in}{2.791948in}}{\pgfqpoint{10.185101in}{2.787558in}}{\pgfqpoint{10.196151in}{2.787558in}}%
\pgfpathclose%
\pgfusepath{stroke,fill}%
\end{pgfscope}%
\begin{pgfscope}%
\pgfpathrectangle{\pgfqpoint{7.640588in}{0.566125in}}{\pgfqpoint{5.699255in}{2.685432in}}%
\pgfusepath{clip}%
\pgfsetbuttcap%
\pgfsetroundjoin%
\definecolor{currentfill}{rgb}{0.000000,0.000000,0.000000}%
\pgfsetfillcolor{currentfill}%
\pgfsetlinewidth{1.003750pt}%
\definecolor{currentstroke}{rgb}{0.000000,0.000000,0.000000}%
\pgfsetstrokecolor{currentstroke}%
\pgfsetdash{}{0pt}%
\pgfpathmoveto{\pgfqpoint{9.748053in}{2.565621in}}%
\pgfpathcurveto{\pgfqpoint{9.759103in}{2.565621in}}{\pgfqpoint{9.769702in}{2.570012in}}{\pgfqpoint{9.777515in}{2.577825in}}%
\pgfpathcurveto{\pgfqpoint{9.785329in}{2.585639in}}{\pgfqpoint{9.789719in}{2.596238in}}{\pgfqpoint{9.789719in}{2.607288in}}%
\pgfpathcurveto{\pgfqpoint{9.789719in}{2.618338in}}{\pgfqpoint{9.785329in}{2.628937in}}{\pgfqpoint{9.777515in}{2.636751in}}%
\pgfpathcurveto{\pgfqpoint{9.769702in}{2.644564in}}{\pgfqpoint{9.759103in}{2.648955in}}{\pgfqpoint{9.748053in}{2.648955in}}%
\pgfpathcurveto{\pgfqpoint{9.737002in}{2.648955in}}{\pgfqpoint{9.726403in}{2.644564in}}{\pgfqpoint{9.718590in}{2.636751in}}%
\pgfpathcurveto{\pgfqpoint{9.710776in}{2.628937in}}{\pgfqpoint{9.706386in}{2.618338in}}{\pgfqpoint{9.706386in}{2.607288in}}%
\pgfpathcurveto{\pgfqpoint{9.706386in}{2.596238in}}{\pgfqpoint{9.710776in}{2.585639in}}{\pgfqpoint{9.718590in}{2.577825in}}%
\pgfpathcurveto{\pgfqpoint{9.726403in}{2.570012in}}{\pgfqpoint{9.737002in}{2.565621in}}{\pgfqpoint{9.748053in}{2.565621in}}%
\pgfpathclose%
\pgfusepath{stroke,fill}%
\end{pgfscope}%
\begin{pgfscope}%
\pgfpathrectangle{\pgfqpoint{7.640588in}{0.566125in}}{\pgfqpoint{5.699255in}{2.685432in}}%
\pgfusepath{clip}%
\pgfsetbuttcap%
\pgfsetroundjoin%
\definecolor{currentfill}{rgb}{0.000000,0.000000,0.000000}%
\pgfsetfillcolor{currentfill}%
\pgfsetlinewidth{1.003750pt}%
\definecolor{currentstroke}{rgb}{0.000000,0.000000,0.000000}%
\pgfsetstrokecolor{currentstroke}%
\pgfsetdash{}{0pt}%
\pgfpathmoveto{\pgfqpoint{9.832071in}{2.878944in}}%
\pgfpathcurveto{\pgfqpoint{9.843121in}{2.878944in}}{\pgfqpoint{9.853720in}{2.883334in}}{\pgfqpoint{9.861534in}{2.891147in}}%
\pgfpathcurveto{\pgfqpoint{9.869347in}{2.898961in}}{\pgfqpoint{9.873738in}{2.909560in}}{\pgfqpoint{9.873738in}{2.920610in}}%
\pgfpathcurveto{\pgfqpoint{9.873738in}{2.931660in}}{\pgfqpoint{9.869347in}{2.942259in}}{\pgfqpoint{9.861534in}{2.950073in}}%
\pgfpathcurveto{\pgfqpoint{9.853720in}{2.957887in}}{\pgfqpoint{9.843121in}{2.962277in}}{\pgfqpoint{9.832071in}{2.962277in}}%
\pgfpathcurveto{\pgfqpoint{9.821021in}{2.962277in}}{\pgfqpoint{9.810422in}{2.957887in}}{\pgfqpoint{9.802608in}{2.950073in}}%
\pgfpathcurveto{\pgfqpoint{9.794795in}{2.942259in}}{\pgfqpoint{9.790404in}{2.931660in}}{\pgfqpoint{9.790404in}{2.920610in}}%
\pgfpathcurveto{\pgfqpoint{9.790404in}{2.909560in}}{\pgfqpoint{9.794795in}{2.898961in}}{\pgfqpoint{9.802608in}{2.891147in}}%
\pgfpathcurveto{\pgfqpoint{9.810422in}{2.883334in}}{\pgfqpoint{9.821021in}{2.878944in}}{\pgfqpoint{9.832071in}{2.878944in}}%
\pgfpathclose%
\pgfusepath{stroke,fill}%
\end{pgfscope}%
\begin{pgfscope}%
\pgfpathrectangle{\pgfqpoint{7.640588in}{0.566125in}}{\pgfqpoint{5.699255in}{2.685432in}}%
\pgfusepath{clip}%
\pgfsetbuttcap%
\pgfsetroundjoin%
\definecolor{currentfill}{rgb}{0.000000,0.000000,0.000000}%
\pgfsetfillcolor{currentfill}%
\pgfsetlinewidth{1.003750pt}%
\definecolor{currentstroke}{rgb}{0.000000,0.000000,0.000000}%
\pgfsetstrokecolor{currentstroke}%
\pgfsetdash{}{0pt}%
\pgfpathmoveto{\pgfqpoint{9.131917in}{3.074770in}}%
\pgfpathcurveto{\pgfqpoint{9.142967in}{3.074770in}}{\pgfqpoint{9.153566in}{3.079160in}}{\pgfqpoint{9.161380in}{3.086974in}}%
\pgfpathcurveto{\pgfqpoint{9.169193in}{3.094787in}}{\pgfqpoint{9.173584in}{3.105386in}}{\pgfqpoint{9.173584in}{3.116437in}}%
\pgfpathcurveto{\pgfqpoint{9.173584in}{3.127487in}}{\pgfqpoint{9.169193in}{3.138086in}}{\pgfqpoint{9.161380in}{3.145899in}}%
\pgfpathcurveto{\pgfqpoint{9.153566in}{3.153713in}}{\pgfqpoint{9.142967in}{3.158103in}}{\pgfqpoint{9.131917in}{3.158103in}}%
\pgfpathcurveto{\pgfqpoint{9.120867in}{3.158103in}}{\pgfqpoint{9.110268in}{3.153713in}}{\pgfqpoint{9.102454in}{3.145899in}}%
\pgfpathcurveto{\pgfqpoint{9.094640in}{3.138086in}}{\pgfqpoint{9.090250in}{3.127487in}}{\pgfqpoint{9.090250in}{3.116437in}}%
\pgfpathcurveto{\pgfqpoint{9.090250in}{3.105386in}}{\pgfqpoint{9.094640in}{3.094787in}}{\pgfqpoint{9.102454in}{3.086974in}}%
\pgfpathcurveto{\pgfqpoint{9.110268in}{3.079160in}}{\pgfqpoint{9.120867in}{3.074770in}}{\pgfqpoint{9.131917in}{3.074770in}}%
\pgfpathclose%
\pgfusepath{stroke,fill}%
\end{pgfscope}%
\begin{pgfscope}%
\pgfpathrectangle{\pgfqpoint{7.640588in}{0.566125in}}{\pgfqpoint{5.699255in}{2.685432in}}%
\pgfusepath{clip}%
\pgfsetbuttcap%
\pgfsetroundjoin%
\definecolor{currentfill}{rgb}{0.000000,0.000000,0.000000}%
\pgfsetfillcolor{currentfill}%
\pgfsetlinewidth{1.003750pt}%
\definecolor{currentstroke}{rgb}{0.000000,0.000000,0.000000}%
\pgfsetstrokecolor{currentstroke}%
\pgfsetdash{}{0pt}%
\pgfpathmoveto{\pgfqpoint{8.403756in}{2.800613in}}%
\pgfpathcurveto{\pgfqpoint{8.414807in}{2.800613in}}{\pgfqpoint{8.425406in}{2.805003in}}{\pgfqpoint{8.433219in}{2.812817in}}%
\pgfpathcurveto{\pgfqpoint{8.441033in}{2.820630in}}{\pgfqpoint{8.445423in}{2.831229in}}{\pgfqpoint{8.445423in}{2.842280in}}%
\pgfpathcurveto{\pgfqpoint{8.445423in}{2.853330in}}{\pgfqpoint{8.441033in}{2.863929in}}{\pgfqpoint{8.433219in}{2.871742in}}%
\pgfpathcurveto{\pgfqpoint{8.425406in}{2.879556in}}{\pgfqpoint{8.414807in}{2.883946in}}{\pgfqpoint{8.403756in}{2.883946in}}%
\pgfpathcurveto{\pgfqpoint{8.392706in}{2.883946in}}{\pgfqpoint{8.382107in}{2.879556in}}{\pgfqpoint{8.374294in}{2.871742in}}%
\pgfpathcurveto{\pgfqpoint{8.366480in}{2.863929in}}{\pgfqpoint{8.362090in}{2.853330in}}{\pgfqpoint{8.362090in}{2.842280in}}%
\pgfpathcurveto{\pgfqpoint{8.362090in}{2.831229in}}{\pgfqpoint{8.366480in}{2.820630in}}{\pgfqpoint{8.374294in}{2.812817in}}%
\pgfpathcurveto{\pgfqpoint{8.382107in}{2.805003in}}{\pgfqpoint{8.392706in}{2.800613in}}{\pgfqpoint{8.403756in}{2.800613in}}%
\pgfpathclose%
\pgfusepath{stroke,fill}%
\end{pgfscope}%
\begin{pgfscope}%
\pgfpathrectangle{\pgfqpoint{7.640588in}{0.566125in}}{\pgfqpoint{5.699255in}{2.685432in}}%
\pgfusepath{clip}%
\pgfsetbuttcap%
\pgfsetroundjoin%
\definecolor{currentfill}{rgb}{0.000000,0.000000,0.000000}%
\pgfsetfillcolor{currentfill}%
\pgfsetlinewidth{1.003750pt}%
\definecolor{currentstroke}{rgb}{0.000000,0.000000,0.000000}%
\pgfsetstrokecolor{currentstroke}%
\pgfsetdash{}{0pt}%
\pgfpathmoveto{\pgfqpoint{8.039676in}{2.813668in}}%
\pgfpathcurveto{\pgfqpoint{8.050726in}{2.813668in}}{\pgfqpoint{8.061325in}{2.818058in}}{\pgfqpoint{8.069139in}{2.825872in}}%
\pgfpathcurveto{\pgfqpoint{8.076953in}{2.833686in}}{\pgfqpoint{8.081343in}{2.844285in}}{\pgfqpoint{8.081343in}{2.855335in}}%
\pgfpathcurveto{\pgfqpoint{8.081343in}{2.866385in}}{\pgfqpoint{8.076953in}{2.876984in}}{\pgfqpoint{8.069139in}{2.884798in}}%
\pgfpathcurveto{\pgfqpoint{8.061325in}{2.892611in}}{\pgfqpoint{8.050726in}{2.897001in}}{\pgfqpoint{8.039676in}{2.897001in}}%
\pgfpathcurveto{\pgfqpoint{8.028626in}{2.897001in}}{\pgfqpoint{8.018027in}{2.892611in}}{\pgfqpoint{8.010213in}{2.884798in}}%
\pgfpathcurveto{\pgfqpoint{8.002400in}{2.876984in}}{\pgfqpoint{7.998010in}{2.866385in}}{\pgfqpoint{7.998010in}{2.855335in}}%
\pgfpathcurveto{\pgfqpoint{7.998010in}{2.844285in}}{\pgfqpoint{8.002400in}{2.833686in}}{\pgfqpoint{8.010213in}{2.825872in}}%
\pgfpathcurveto{\pgfqpoint{8.018027in}{2.818058in}}{\pgfqpoint{8.028626in}{2.813668in}}{\pgfqpoint{8.039676in}{2.813668in}}%
\pgfpathclose%
\pgfusepath{stroke,fill}%
\end{pgfscope}%
\begin{pgfscope}%
\pgfpathrectangle{\pgfqpoint{7.640588in}{0.566125in}}{\pgfqpoint{5.699255in}{2.685432in}}%
\pgfusepath{clip}%
\pgfsetbuttcap%
\pgfsetroundjoin%
\definecolor{currentfill}{rgb}{0.000000,0.000000,0.000000}%
\pgfsetfillcolor{currentfill}%
\pgfsetlinewidth{1.003750pt}%
\definecolor{currentstroke}{rgb}{0.000000,0.000000,0.000000}%
\pgfsetstrokecolor{currentstroke}%
\pgfsetdash{}{0pt}%
\pgfpathmoveto{\pgfqpoint{7.899645in}{2.735338in}}%
\pgfpathcurveto{\pgfqpoint{7.910696in}{2.735338in}}{\pgfqpoint{7.921295in}{2.739728in}}{\pgfqpoint{7.929108in}{2.747541in}}%
\pgfpathcurveto{\pgfqpoint{7.936922in}{2.755355in}}{\pgfqpoint{7.941312in}{2.765954in}}{\pgfqpoint{7.941312in}{2.777004in}}%
\pgfpathcurveto{\pgfqpoint{7.941312in}{2.788054in}}{\pgfqpoint{7.936922in}{2.798653in}}{\pgfqpoint{7.929108in}{2.806467in}}%
\pgfpathcurveto{\pgfqpoint{7.921295in}{2.814281in}}{\pgfqpoint{7.910696in}{2.818671in}}{\pgfqpoint{7.899645in}{2.818671in}}%
\pgfpathcurveto{\pgfqpoint{7.888595in}{2.818671in}}{\pgfqpoint{7.877996in}{2.814281in}}{\pgfqpoint{7.870183in}{2.806467in}}%
\pgfpathcurveto{\pgfqpoint{7.862369in}{2.798653in}}{\pgfqpoint{7.857979in}{2.788054in}}{\pgfqpoint{7.857979in}{2.777004in}}%
\pgfpathcurveto{\pgfqpoint{7.857979in}{2.765954in}}{\pgfqpoint{7.862369in}{2.755355in}}{\pgfqpoint{7.870183in}{2.747541in}}%
\pgfpathcurveto{\pgfqpoint{7.877996in}{2.739728in}}{\pgfqpoint{7.888595in}{2.735338in}}{\pgfqpoint{7.899645in}{2.735338in}}%
\pgfpathclose%
\pgfusepath{stroke,fill}%
\end{pgfscope}%
\begin{pgfscope}%
\pgfpathrectangle{\pgfqpoint{7.640588in}{0.566125in}}{\pgfqpoint{5.699255in}{2.685432in}}%
\pgfusepath{clip}%
\pgfsetbuttcap%
\pgfsetroundjoin%
\definecolor{currentfill}{rgb}{0.000000,0.000000,0.000000}%
\pgfsetfillcolor{currentfill}%
\pgfsetlinewidth{1.003750pt}%
\definecolor{currentstroke}{rgb}{0.000000,0.000000,0.000000}%
\pgfsetstrokecolor{currentstroke}%
\pgfsetdash{}{0pt}%
\pgfpathmoveto{\pgfqpoint{7.955658in}{2.670062in}}%
\pgfpathcurveto{\pgfqpoint{7.966708in}{2.670062in}}{\pgfqpoint{7.977307in}{2.674452in}}{\pgfqpoint{7.985121in}{2.682266in}}%
\pgfpathcurveto{\pgfqpoint{7.992934in}{2.690080in}}{\pgfqpoint{7.997324in}{2.700679in}}{\pgfqpoint{7.997324in}{2.711729in}}%
\pgfpathcurveto{\pgfqpoint{7.997324in}{2.722779in}}{\pgfqpoint{7.992934in}{2.733378in}}{\pgfqpoint{7.985121in}{2.741192in}}%
\pgfpathcurveto{\pgfqpoint{7.977307in}{2.749005in}}{\pgfqpoint{7.966708in}{2.753395in}}{\pgfqpoint{7.955658in}{2.753395in}}%
\pgfpathcurveto{\pgfqpoint{7.944608in}{2.753395in}}{\pgfqpoint{7.934009in}{2.749005in}}{\pgfqpoint{7.926195in}{2.741192in}}%
\pgfpathcurveto{\pgfqpoint{7.918381in}{2.733378in}}{\pgfqpoint{7.913991in}{2.722779in}}{\pgfqpoint{7.913991in}{2.711729in}}%
\pgfpathcurveto{\pgfqpoint{7.913991in}{2.700679in}}{\pgfqpoint{7.918381in}{2.690080in}}{\pgfqpoint{7.926195in}{2.682266in}}%
\pgfpathcurveto{\pgfqpoint{7.934009in}{2.674452in}}{\pgfqpoint{7.944608in}{2.670062in}}{\pgfqpoint{7.955658in}{2.670062in}}%
\pgfpathclose%
\pgfusepath{stroke,fill}%
\end{pgfscope}%
\begin{pgfscope}%
\pgfpathrectangle{\pgfqpoint{7.640588in}{0.566125in}}{\pgfqpoint{5.699255in}{2.685432in}}%
\pgfusepath{clip}%
\pgfsetbuttcap%
\pgfsetroundjoin%
\definecolor{currentfill}{rgb}{0.000000,0.000000,0.000000}%
\pgfsetfillcolor{currentfill}%
\pgfsetlinewidth{1.003750pt}%
\definecolor{currentstroke}{rgb}{0.000000,0.000000,0.000000}%
\pgfsetstrokecolor{currentstroke}%
\pgfsetdash{}{0pt}%
\pgfpathmoveto{\pgfqpoint{8.515781in}{2.683117in}}%
\pgfpathcurveto{\pgfqpoint{8.526831in}{2.683117in}}{\pgfqpoint{8.537430in}{2.687507in}}{\pgfqpoint{8.545244in}{2.695321in}}%
\pgfpathcurveto{\pgfqpoint{8.553058in}{2.703135in}}{\pgfqpoint{8.557448in}{2.713734in}}{\pgfqpoint{8.557448in}{2.724784in}}%
\pgfpathcurveto{\pgfqpoint{8.557448in}{2.735834in}}{\pgfqpoint{8.553058in}{2.746433in}}{\pgfqpoint{8.545244in}{2.754247in}}%
\pgfpathcurveto{\pgfqpoint{8.537430in}{2.762060in}}{\pgfqpoint{8.526831in}{2.766450in}}{\pgfqpoint{8.515781in}{2.766450in}}%
\pgfpathcurveto{\pgfqpoint{8.504731in}{2.766450in}}{\pgfqpoint{8.494132in}{2.762060in}}{\pgfqpoint{8.486318in}{2.754247in}}%
\pgfpathcurveto{\pgfqpoint{8.478505in}{2.746433in}}{\pgfqpoint{8.474114in}{2.735834in}}{\pgfqpoint{8.474114in}{2.724784in}}%
\pgfpathcurveto{\pgfqpoint{8.474114in}{2.713734in}}{\pgfqpoint{8.478505in}{2.703135in}}{\pgfqpoint{8.486318in}{2.695321in}}%
\pgfpathcurveto{\pgfqpoint{8.494132in}{2.687507in}}{\pgfqpoint{8.504731in}{2.683117in}}{\pgfqpoint{8.515781in}{2.683117in}}%
\pgfpathclose%
\pgfusepath{stroke,fill}%
\end{pgfscope}%
\begin{pgfscope}%
\pgfpathrectangle{\pgfqpoint{7.640588in}{0.566125in}}{\pgfqpoint{5.699255in}{2.685432in}}%
\pgfusepath{clip}%
\pgfsetbuttcap%
\pgfsetroundjoin%
\definecolor{currentfill}{rgb}{0.000000,0.000000,0.000000}%
\pgfsetfillcolor{currentfill}%
\pgfsetlinewidth{1.003750pt}%
\definecolor{currentstroke}{rgb}{0.000000,0.000000,0.000000}%
\pgfsetstrokecolor{currentstroke}%
\pgfsetdash{}{0pt}%
\pgfpathmoveto{\pgfqpoint{9.047898in}{2.826723in}}%
\pgfpathcurveto{\pgfqpoint{9.058948in}{2.826723in}}{\pgfqpoint{9.069548in}{2.831113in}}{\pgfqpoint{9.077361in}{2.838927in}}%
\pgfpathcurveto{\pgfqpoint{9.085175in}{2.846741in}}{\pgfqpoint{9.089565in}{2.857340in}}{\pgfqpoint{9.089565in}{2.868390in}}%
\pgfpathcurveto{\pgfqpoint{9.089565in}{2.879440in}}{\pgfqpoint{9.085175in}{2.890039in}}{\pgfqpoint{9.077361in}{2.897853in}}%
\pgfpathcurveto{\pgfqpoint{9.069548in}{2.905666in}}{\pgfqpoint{9.058948in}{2.910056in}}{\pgfqpoint{9.047898in}{2.910056in}}%
\pgfpathcurveto{\pgfqpoint{9.036848in}{2.910056in}}{\pgfqpoint{9.026249in}{2.905666in}}{\pgfqpoint{9.018436in}{2.897853in}}%
\pgfpathcurveto{\pgfqpoint{9.010622in}{2.890039in}}{\pgfqpoint{9.006232in}{2.879440in}}{\pgfqpoint{9.006232in}{2.868390in}}%
\pgfpathcurveto{\pgfqpoint{9.006232in}{2.857340in}}{\pgfqpoint{9.010622in}{2.846741in}}{\pgfqpoint{9.018436in}{2.838927in}}%
\pgfpathcurveto{\pgfqpoint{9.026249in}{2.831113in}}{\pgfqpoint{9.036848in}{2.826723in}}{\pgfqpoint{9.047898in}{2.826723in}}%
\pgfpathclose%
\pgfusepath{stroke,fill}%
\end{pgfscope}%
\begin{pgfscope}%
\pgfpathrectangle{\pgfqpoint{7.640588in}{0.566125in}}{\pgfqpoint{5.699255in}{2.685432in}}%
\pgfusepath{clip}%
\pgfsetbuttcap%
\pgfsetroundjoin%
\definecolor{currentfill}{rgb}{0.000000,0.000000,0.000000}%
\pgfsetfillcolor{currentfill}%
\pgfsetlinewidth{1.003750pt}%
\definecolor{currentstroke}{rgb}{0.000000,0.000000,0.000000}%
\pgfsetstrokecolor{currentstroke}%
\pgfsetdash{}{0pt}%
\pgfpathmoveto{\pgfqpoint{9.271948in}{2.709227in}}%
\pgfpathcurveto{\pgfqpoint{9.282998in}{2.709227in}}{\pgfqpoint{9.293597in}{2.713618in}}{\pgfqpoint{9.301410in}{2.721431in}}%
\pgfpathcurveto{\pgfqpoint{9.309224in}{2.729245in}}{\pgfqpoint{9.313614in}{2.739844in}}{\pgfqpoint{9.313614in}{2.750894in}}%
\pgfpathcurveto{\pgfqpoint{9.313614in}{2.761944in}}{\pgfqpoint{9.309224in}{2.772543in}}{\pgfqpoint{9.301410in}{2.780357in}}%
\pgfpathcurveto{\pgfqpoint{9.293597in}{2.788170in}}{\pgfqpoint{9.282998in}{2.792561in}}{\pgfqpoint{9.271948in}{2.792561in}}%
\pgfpathcurveto{\pgfqpoint{9.260898in}{2.792561in}}{\pgfqpoint{9.250299in}{2.788170in}}{\pgfqpoint{9.242485in}{2.780357in}}%
\pgfpathcurveto{\pgfqpoint{9.234671in}{2.772543in}}{\pgfqpoint{9.230281in}{2.761944in}}{\pgfqpoint{9.230281in}{2.750894in}}%
\pgfpathcurveto{\pgfqpoint{9.230281in}{2.739844in}}{\pgfqpoint{9.234671in}{2.729245in}}{\pgfqpoint{9.242485in}{2.721431in}}%
\pgfpathcurveto{\pgfqpoint{9.250299in}{2.713618in}}{\pgfqpoint{9.260898in}{2.709227in}}{\pgfqpoint{9.271948in}{2.709227in}}%
\pgfpathclose%
\pgfusepath{stroke,fill}%
\end{pgfscope}%
\begin{pgfscope}%
\pgfpathrectangle{\pgfqpoint{7.640588in}{0.566125in}}{\pgfqpoint{5.699255in}{2.685432in}}%
\pgfusepath{clip}%
\pgfsetbuttcap%
\pgfsetroundjoin%
\definecolor{currentfill}{rgb}{0.000000,0.000000,0.000000}%
\pgfsetfillcolor{currentfill}%
\pgfsetlinewidth{1.003750pt}%
\definecolor{currentstroke}{rgb}{0.000000,0.000000,0.000000}%
\pgfsetstrokecolor{currentstroke}%
\pgfsetdash{}{0pt}%
\pgfpathmoveto{\pgfqpoint{9.047898in}{2.657007in}}%
\pgfpathcurveto{\pgfqpoint{9.058948in}{2.657007in}}{\pgfqpoint{9.069548in}{2.661397in}}{\pgfqpoint{9.077361in}{2.669211in}}%
\pgfpathcurveto{\pgfqpoint{9.085175in}{2.677024in}}{\pgfqpoint{9.089565in}{2.687624in}}{\pgfqpoint{9.089565in}{2.698674in}}%
\pgfpathcurveto{\pgfqpoint{9.089565in}{2.709724in}}{\pgfqpoint{9.085175in}{2.720323in}}{\pgfqpoint{9.077361in}{2.728136in}}%
\pgfpathcurveto{\pgfqpoint{9.069548in}{2.735950in}}{\pgfqpoint{9.058948in}{2.740340in}}{\pgfqpoint{9.047898in}{2.740340in}}%
\pgfpathcurveto{\pgfqpoint{9.036848in}{2.740340in}}{\pgfqpoint{9.026249in}{2.735950in}}{\pgfqpoint{9.018436in}{2.728136in}}%
\pgfpathcurveto{\pgfqpoint{9.010622in}{2.720323in}}{\pgfqpoint{9.006232in}{2.709724in}}{\pgfqpoint{9.006232in}{2.698674in}}%
\pgfpathcurveto{\pgfqpoint{9.006232in}{2.687624in}}{\pgfqpoint{9.010622in}{2.677024in}}{\pgfqpoint{9.018436in}{2.669211in}}%
\pgfpathcurveto{\pgfqpoint{9.026249in}{2.661397in}}{\pgfqpoint{9.036848in}{2.657007in}}{\pgfqpoint{9.047898in}{2.657007in}}%
\pgfpathclose%
\pgfusepath{stroke,fill}%
\end{pgfscope}%
\begin{pgfscope}%
\pgfpathrectangle{\pgfqpoint{7.640588in}{0.566125in}}{\pgfqpoint{5.699255in}{2.685432in}}%
\pgfusepath{clip}%
\pgfsetbuttcap%
\pgfsetroundjoin%
\definecolor{currentfill}{rgb}{0.000000,0.000000,0.000000}%
\pgfsetfillcolor{currentfill}%
\pgfsetlinewidth{1.003750pt}%
\definecolor{currentstroke}{rgb}{0.000000,0.000000,0.000000}%
\pgfsetstrokecolor{currentstroke}%
\pgfsetdash{}{0pt}%
\pgfpathmoveto{\pgfqpoint{8.879861in}{2.578676in}}%
\pgfpathcurveto{\pgfqpoint{8.890911in}{2.578676in}}{\pgfqpoint{8.901511in}{2.583067in}}{\pgfqpoint{8.909324in}{2.590880in}}%
\pgfpathcurveto{\pgfqpoint{8.917138in}{2.598694in}}{\pgfqpoint{8.921528in}{2.609293in}}{\pgfqpoint{8.921528in}{2.620343in}}%
\pgfpathcurveto{\pgfqpoint{8.921528in}{2.631393in}}{\pgfqpoint{8.917138in}{2.641992in}}{\pgfqpoint{8.909324in}{2.649806in}}%
\pgfpathcurveto{\pgfqpoint{8.901511in}{2.657619in}}{\pgfqpoint{8.890911in}{2.662010in}}{\pgfqpoint{8.879861in}{2.662010in}}%
\pgfpathcurveto{\pgfqpoint{8.868811in}{2.662010in}}{\pgfqpoint{8.858212in}{2.657619in}}{\pgfqpoint{8.850399in}{2.649806in}}%
\pgfpathcurveto{\pgfqpoint{8.842585in}{2.641992in}}{\pgfqpoint{8.838195in}{2.631393in}}{\pgfqpoint{8.838195in}{2.620343in}}%
\pgfpathcurveto{\pgfqpoint{8.838195in}{2.609293in}}{\pgfqpoint{8.842585in}{2.598694in}}{\pgfqpoint{8.850399in}{2.590880in}}%
\pgfpathcurveto{\pgfqpoint{8.858212in}{2.583067in}}{\pgfqpoint{8.868811in}{2.578676in}}{\pgfqpoint{8.879861in}{2.578676in}}%
\pgfpathclose%
\pgfusepath{stroke,fill}%
\end{pgfscope}%
\begin{pgfscope}%
\pgfpathrectangle{\pgfqpoint{7.640588in}{0.566125in}}{\pgfqpoint{5.699255in}{2.685432in}}%
\pgfusepath{clip}%
\pgfsetbuttcap%
\pgfsetroundjoin%
\definecolor{currentfill}{rgb}{0.000000,0.000000,0.000000}%
\pgfsetfillcolor{currentfill}%
\pgfsetlinewidth{1.003750pt}%
\definecolor{currentstroke}{rgb}{0.000000,0.000000,0.000000}%
\pgfsetstrokecolor{currentstroke}%
\pgfsetdash{}{0pt}%
\pgfpathmoveto{\pgfqpoint{8.067682in}{2.382850in}}%
\pgfpathcurveto{\pgfqpoint{8.078733in}{2.382850in}}{\pgfqpoint{8.089332in}{2.387240in}}{\pgfqpoint{8.097145in}{2.395054in}}%
\pgfpathcurveto{\pgfqpoint{8.104959in}{2.402868in}}{\pgfqpoint{8.109349in}{2.413467in}}{\pgfqpoint{8.109349in}{2.424517in}}%
\pgfpathcurveto{\pgfqpoint{8.109349in}{2.435567in}}{\pgfqpoint{8.104959in}{2.446166in}}{\pgfqpoint{8.097145in}{2.453980in}}%
\pgfpathcurveto{\pgfqpoint{8.089332in}{2.461793in}}{\pgfqpoint{8.078733in}{2.466183in}}{\pgfqpoint{8.067682in}{2.466183in}}%
\pgfpathcurveto{\pgfqpoint{8.056632in}{2.466183in}}{\pgfqpoint{8.046033in}{2.461793in}}{\pgfqpoint{8.038220in}{2.453980in}}%
\pgfpathcurveto{\pgfqpoint{8.030406in}{2.446166in}}{\pgfqpoint{8.026016in}{2.435567in}}{\pgfqpoint{8.026016in}{2.424517in}}%
\pgfpathcurveto{\pgfqpoint{8.026016in}{2.413467in}}{\pgfqpoint{8.030406in}{2.402868in}}{\pgfqpoint{8.038220in}{2.395054in}}%
\pgfpathcurveto{\pgfqpoint{8.046033in}{2.387240in}}{\pgfqpoint{8.056632in}{2.382850in}}{\pgfqpoint{8.067682in}{2.382850in}}%
\pgfpathclose%
\pgfusepath{stroke,fill}%
\end{pgfscope}%
\begin{pgfscope}%
\pgfpathrectangle{\pgfqpoint{7.640588in}{0.566125in}}{\pgfqpoint{5.699255in}{2.685432in}}%
\pgfusepath{clip}%
\pgfsetbuttcap%
\pgfsetroundjoin%
\definecolor{currentfill}{rgb}{0.000000,0.000000,0.000000}%
\pgfsetfillcolor{currentfill}%
\pgfsetlinewidth{1.003750pt}%
\definecolor{currentstroke}{rgb}{0.000000,0.000000,0.000000}%
\pgfsetstrokecolor{currentstroke}%
\pgfsetdash{}{0pt}%
\pgfpathmoveto{\pgfqpoint{8.347744in}{2.382850in}}%
\pgfpathcurveto{\pgfqpoint{8.358794in}{2.382850in}}{\pgfqpoint{8.369393in}{2.387240in}}{\pgfqpoint{8.377207in}{2.395054in}}%
\pgfpathcurveto{\pgfqpoint{8.385021in}{2.402868in}}{\pgfqpoint{8.389411in}{2.413467in}}{\pgfqpoint{8.389411in}{2.424517in}}%
\pgfpathcurveto{\pgfqpoint{8.389411in}{2.435567in}}{\pgfqpoint{8.385021in}{2.446166in}}{\pgfqpoint{8.377207in}{2.453980in}}%
\pgfpathcurveto{\pgfqpoint{8.369393in}{2.461793in}}{\pgfqpoint{8.358794in}{2.466183in}}{\pgfqpoint{8.347744in}{2.466183in}}%
\pgfpathcurveto{\pgfqpoint{8.336694in}{2.466183in}}{\pgfqpoint{8.326095in}{2.461793in}}{\pgfqpoint{8.318281in}{2.453980in}}%
\pgfpathcurveto{\pgfqpoint{8.310468in}{2.446166in}}{\pgfqpoint{8.306077in}{2.435567in}}{\pgfqpoint{8.306077in}{2.424517in}}%
\pgfpathcurveto{\pgfqpoint{8.306077in}{2.413467in}}{\pgfqpoint{8.310468in}{2.402868in}}{\pgfqpoint{8.318281in}{2.395054in}}%
\pgfpathcurveto{\pgfqpoint{8.326095in}{2.387240in}}{\pgfqpoint{8.336694in}{2.382850in}}{\pgfqpoint{8.347744in}{2.382850in}}%
\pgfpathclose%
\pgfusepath{stroke,fill}%
\end{pgfscope}%
\begin{pgfscope}%
\pgfpathrectangle{\pgfqpoint{7.640588in}{0.566125in}}{\pgfqpoint{5.699255in}{2.685432in}}%
\pgfusepath{clip}%
\pgfsetbuttcap%
\pgfsetroundjoin%
\definecolor{currentfill}{rgb}{0.000000,0.000000,0.000000}%
\pgfsetfillcolor{currentfill}%
\pgfsetlinewidth{1.003750pt}%
\definecolor{currentstroke}{rgb}{0.000000,0.000000,0.000000}%
\pgfsetstrokecolor{currentstroke}%
\pgfsetdash{}{0pt}%
\pgfpathmoveto{\pgfqpoint{8.655812in}{2.265354in}}%
\pgfpathcurveto{\pgfqpoint{8.666862in}{2.265354in}}{\pgfqpoint{8.677461in}{2.269745in}}{\pgfqpoint{8.685275in}{2.277558in}}%
\pgfpathcurveto{\pgfqpoint{8.693088in}{2.285372in}}{\pgfqpoint{8.697479in}{2.295971in}}{\pgfqpoint{8.697479in}{2.307021in}}%
\pgfpathcurveto{\pgfqpoint{8.697479in}{2.318071in}}{\pgfqpoint{8.693088in}{2.328670in}}{\pgfqpoint{8.685275in}{2.336484in}}%
\pgfpathcurveto{\pgfqpoint{8.677461in}{2.344297in}}{\pgfqpoint{8.666862in}{2.348688in}}{\pgfqpoint{8.655812in}{2.348688in}}%
\pgfpathcurveto{\pgfqpoint{8.644762in}{2.348688in}}{\pgfqpoint{8.634163in}{2.344297in}}{\pgfqpoint{8.626349in}{2.336484in}}%
\pgfpathcurveto{\pgfqpoint{8.618536in}{2.328670in}}{\pgfqpoint{8.614145in}{2.318071in}}{\pgfqpoint{8.614145in}{2.307021in}}%
\pgfpathcurveto{\pgfqpoint{8.614145in}{2.295971in}}{\pgfqpoint{8.618536in}{2.285372in}}{\pgfqpoint{8.626349in}{2.277558in}}%
\pgfpathcurveto{\pgfqpoint{8.634163in}{2.269745in}}{\pgfqpoint{8.644762in}{2.265354in}}{\pgfqpoint{8.655812in}{2.265354in}}%
\pgfpathclose%
\pgfusepath{stroke,fill}%
\end{pgfscope}%
\begin{pgfscope}%
\pgfpathrectangle{\pgfqpoint{7.640588in}{0.566125in}}{\pgfqpoint{5.699255in}{2.685432in}}%
\pgfusepath{clip}%
\pgfsetbuttcap%
\pgfsetroundjoin%
\definecolor{currentfill}{rgb}{0.000000,0.000000,0.000000}%
\pgfsetfillcolor{currentfill}%
\pgfsetlinewidth{1.003750pt}%
\definecolor{currentstroke}{rgb}{0.000000,0.000000,0.000000}%
\pgfsetstrokecolor{currentstroke}%
\pgfsetdash{}{0pt}%
\pgfpathmoveto{\pgfqpoint{8.823849in}{2.435070in}}%
\pgfpathcurveto{\pgfqpoint{8.834899in}{2.435070in}}{\pgfqpoint{8.845498in}{2.439461in}}{\pgfqpoint{8.853312in}{2.447274in}}%
\pgfpathcurveto{\pgfqpoint{8.861125in}{2.455088in}}{\pgfqpoint{8.865516in}{2.465687in}}{\pgfqpoint{8.865516in}{2.476737in}}%
\pgfpathcurveto{\pgfqpoint{8.865516in}{2.487787in}}{\pgfqpoint{8.861125in}{2.498386in}}{\pgfqpoint{8.853312in}{2.506200in}}%
\pgfpathcurveto{\pgfqpoint{8.845498in}{2.514014in}}{\pgfqpoint{8.834899in}{2.518404in}}{\pgfqpoint{8.823849in}{2.518404in}}%
\pgfpathcurveto{\pgfqpoint{8.812799in}{2.518404in}}{\pgfqpoint{8.802200in}{2.514014in}}{\pgfqpoint{8.794386in}{2.506200in}}%
\pgfpathcurveto{\pgfqpoint{8.786573in}{2.498386in}}{\pgfqpoint{8.782182in}{2.487787in}}{\pgfqpoint{8.782182in}{2.476737in}}%
\pgfpathcurveto{\pgfqpoint{8.782182in}{2.465687in}}{\pgfqpoint{8.786573in}{2.455088in}}{\pgfqpoint{8.794386in}{2.447274in}}%
\pgfpathcurveto{\pgfqpoint{8.802200in}{2.439461in}}{\pgfqpoint{8.812799in}{2.435070in}}{\pgfqpoint{8.823849in}{2.435070in}}%
\pgfpathclose%
\pgfusepath{stroke,fill}%
\end{pgfscope}%
\begin{pgfscope}%
\pgfpathrectangle{\pgfqpoint{7.640588in}{0.566125in}}{\pgfqpoint{5.699255in}{2.685432in}}%
\pgfusepath{clip}%
\pgfsetbuttcap%
\pgfsetroundjoin%
\definecolor{currentfill}{rgb}{0.000000,0.000000,0.000000}%
\pgfsetfillcolor{currentfill}%
\pgfsetlinewidth{1.003750pt}%
\definecolor{currentstroke}{rgb}{0.000000,0.000000,0.000000}%
\pgfsetstrokecolor{currentstroke}%
\pgfsetdash{}{0pt}%
\pgfpathmoveto{\pgfqpoint{9.580016in}{2.343685in}}%
\pgfpathcurveto{\pgfqpoint{9.591066in}{2.343685in}}{\pgfqpoint{9.601665in}{2.348075in}}{\pgfqpoint{9.609478in}{2.355889in}}%
\pgfpathcurveto{\pgfqpoint{9.617292in}{2.363702in}}{\pgfqpoint{9.621682in}{2.374301in}}{\pgfqpoint{9.621682in}{2.385351in}}%
\pgfpathcurveto{\pgfqpoint{9.621682in}{2.396402in}}{\pgfqpoint{9.617292in}{2.407001in}}{\pgfqpoint{9.609478in}{2.414814in}}%
\pgfpathcurveto{\pgfqpoint{9.601665in}{2.422628in}}{\pgfqpoint{9.591066in}{2.427018in}}{\pgfqpoint{9.580016in}{2.427018in}}%
\pgfpathcurveto{\pgfqpoint{9.568965in}{2.427018in}}{\pgfqpoint{9.558366in}{2.422628in}}{\pgfqpoint{9.550553in}{2.414814in}}%
\pgfpathcurveto{\pgfqpoint{9.542739in}{2.407001in}}{\pgfqpoint{9.538349in}{2.396402in}}{\pgfqpoint{9.538349in}{2.385351in}}%
\pgfpathcurveto{\pgfqpoint{9.538349in}{2.374301in}}{\pgfqpoint{9.542739in}{2.363702in}}{\pgfqpoint{9.550553in}{2.355889in}}%
\pgfpathcurveto{\pgfqpoint{9.558366in}{2.348075in}}{\pgfqpoint{9.568965in}{2.343685in}}{\pgfqpoint{9.580016in}{2.343685in}}%
\pgfpathclose%
\pgfusepath{stroke,fill}%
\end{pgfscope}%
\begin{pgfscope}%
\pgfpathrectangle{\pgfqpoint{7.640588in}{0.566125in}}{\pgfqpoint{5.699255in}{2.685432in}}%
\pgfusepath{clip}%
\pgfsetbuttcap%
\pgfsetroundjoin%
\definecolor{currentfill}{rgb}{0.000000,0.000000,0.000000}%
\pgfsetfillcolor{currentfill}%
\pgfsetlinewidth{1.003750pt}%
\definecolor{currentstroke}{rgb}{0.000000,0.000000,0.000000}%
\pgfsetstrokecolor{currentstroke}%
\pgfsetdash{}{0pt}%
\pgfpathmoveto{\pgfqpoint{9.580016in}{2.213134in}}%
\pgfpathcurveto{\pgfqpoint{9.591066in}{2.213134in}}{\pgfqpoint{9.601665in}{2.217524in}}{\pgfqpoint{9.609478in}{2.225338in}}%
\pgfpathcurveto{\pgfqpoint{9.617292in}{2.233151in}}{\pgfqpoint{9.621682in}{2.243750in}}{\pgfqpoint{9.621682in}{2.254801in}}%
\pgfpathcurveto{\pgfqpoint{9.621682in}{2.265851in}}{\pgfqpoint{9.617292in}{2.276450in}}{\pgfqpoint{9.609478in}{2.284263in}}%
\pgfpathcurveto{\pgfqpoint{9.601665in}{2.292077in}}{\pgfqpoint{9.591066in}{2.296467in}}{\pgfqpoint{9.580016in}{2.296467in}}%
\pgfpathcurveto{\pgfqpoint{9.568965in}{2.296467in}}{\pgfqpoint{9.558366in}{2.292077in}}{\pgfqpoint{9.550553in}{2.284263in}}%
\pgfpathcurveto{\pgfqpoint{9.542739in}{2.276450in}}{\pgfqpoint{9.538349in}{2.265851in}}{\pgfqpoint{9.538349in}{2.254801in}}%
\pgfpathcurveto{\pgfqpoint{9.538349in}{2.243750in}}{\pgfqpoint{9.542739in}{2.233151in}}{\pgfqpoint{9.550553in}{2.225338in}}%
\pgfpathcurveto{\pgfqpoint{9.558366in}{2.217524in}}{\pgfqpoint{9.568965in}{2.213134in}}{\pgfqpoint{9.580016in}{2.213134in}}%
\pgfpathclose%
\pgfusepath{stroke,fill}%
\end{pgfscope}%
\begin{pgfscope}%
\pgfpathrectangle{\pgfqpoint{7.640588in}{0.566125in}}{\pgfqpoint{5.699255in}{2.685432in}}%
\pgfusepath{clip}%
\pgfsetbuttcap%
\pgfsetroundjoin%
\definecolor{currentfill}{rgb}{0.000000,0.000000,0.000000}%
\pgfsetfillcolor{currentfill}%
\pgfsetlinewidth{1.003750pt}%
\definecolor{currentstroke}{rgb}{0.000000,0.000000,0.000000}%
\pgfsetstrokecolor{currentstroke}%
\pgfsetdash{}{0pt}%
\pgfpathmoveto{\pgfqpoint{9.383972in}{2.069528in}}%
\pgfpathcurveto{\pgfqpoint{9.395023in}{2.069528in}}{\pgfqpoint{9.405622in}{2.073918in}}{\pgfqpoint{9.413435in}{2.081732in}}%
\pgfpathcurveto{\pgfqpoint{9.421249in}{2.089545in}}{\pgfqpoint{9.425639in}{2.100144in}}{\pgfqpoint{9.425639in}{2.111195in}}%
\pgfpathcurveto{\pgfqpoint{9.425639in}{2.122245in}}{\pgfqpoint{9.421249in}{2.132844in}}{\pgfqpoint{9.413435in}{2.140657in}}%
\pgfpathcurveto{\pgfqpoint{9.405622in}{2.148471in}}{\pgfqpoint{9.395023in}{2.152861in}}{\pgfqpoint{9.383972in}{2.152861in}}%
\pgfpathcurveto{\pgfqpoint{9.372922in}{2.152861in}}{\pgfqpoint{9.362323in}{2.148471in}}{\pgfqpoint{9.354510in}{2.140657in}}%
\pgfpathcurveto{\pgfqpoint{9.346696in}{2.132844in}}{\pgfqpoint{9.342306in}{2.122245in}}{\pgfqpoint{9.342306in}{2.111195in}}%
\pgfpathcurveto{\pgfqpoint{9.342306in}{2.100144in}}{\pgfqpoint{9.346696in}{2.089545in}}{\pgfqpoint{9.354510in}{2.081732in}}%
\pgfpathcurveto{\pgfqpoint{9.362323in}{2.073918in}}{\pgfqpoint{9.372922in}{2.069528in}}{\pgfqpoint{9.383972in}{2.069528in}}%
\pgfpathclose%
\pgfusepath{stroke,fill}%
\end{pgfscope}%
\begin{pgfscope}%
\pgfpathrectangle{\pgfqpoint{7.640588in}{0.566125in}}{\pgfqpoint{5.699255in}{2.685432in}}%
\pgfusepath{clip}%
\pgfsetbuttcap%
\pgfsetroundjoin%
\definecolor{currentfill}{rgb}{0.000000,0.000000,0.000000}%
\pgfsetfillcolor{currentfill}%
\pgfsetlinewidth{1.003750pt}%
\definecolor{currentstroke}{rgb}{0.000000,0.000000,0.000000}%
\pgfsetstrokecolor{currentstroke}%
\pgfsetdash{}{0pt}%
\pgfpathmoveto{\pgfqpoint{9.355966in}{2.043418in}}%
\pgfpathcurveto{\pgfqpoint{9.367016in}{2.043418in}}{\pgfqpoint{9.377615in}{2.047808in}}{\pgfqpoint{9.385429in}{2.055622in}}%
\pgfpathcurveto{\pgfqpoint{9.393243in}{2.063435in}}{\pgfqpoint{9.397633in}{2.074034in}}{\pgfqpoint{9.397633in}{2.085084in}}%
\pgfpathcurveto{\pgfqpoint{9.397633in}{2.096135in}}{\pgfqpoint{9.393243in}{2.106734in}}{\pgfqpoint{9.385429in}{2.114547in}}%
\pgfpathcurveto{\pgfqpoint{9.377615in}{2.122361in}}{\pgfqpoint{9.367016in}{2.126751in}}{\pgfqpoint{9.355966in}{2.126751in}}%
\pgfpathcurveto{\pgfqpoint{9.344916in}{2.126751in}}{\pgfqpoint{9.334317in}{2.122361in}}{\pgfqpoint{9.326503in}{2.114547in}}%
\pgfpathcurveto{\pgfqpoint{9.318690in}{2.106734in}}{\pgfqpoint{9.314300in}{2.096135in}}{\pgfqpoint{9.314300in}{2.085084in}}%
\pgfpathcurveto{\pgfqpoint{9.314300in}{2.074034in}}{\pgfqpoint{9.318690in}{2.063435in}}{\pgfqpoint{9.326503in}{2.055622in}}%
\pgfpathcurveto{\pgfqpoint{9.334317in}{2.047808in}}{\pgfqpoint{9.344916in}{2.043418in}}{\pgfqpoint{9.355966in}{2.043418in}}%
\pgfpathclose%
\pgfusepath{stroke,fill}%
\end{pgfscope}%
\begin{pgfscope}%
\pgfpathrectangle{\pgfqpoint{7.640588in}{0.566125in}}{\pgfqpoint{5.699255in}{2.685432in}}%
\pgfusepath{clip}%
\pgfsetbuttcap%
\pgfsetroundjoin%
\definecolor{currentfill}{rgb}{0.000000,0.000000,0.000000}%
\pgfsetfillcolor{currentfill}%
\pgfsetlinewidth{1.003750pt}%
\definecolor{currentstroke}{rgb}{0.000000,0.000000,0.000000}%
\pgfsetstrokecolor{currentstroke}%
\pgfsetdash{}{0pt}%
\pgfpathmoveto{\pgfqpoint{9.271948in}{1.965087in}}%
\pgfpathcurveto{\pgfqpoint{9.282998in}{1.965087in}}{\pgfqpoint{9.293597in}{1.969477in}}{\pgfqpoint{9.301410in}{1.977291in}}%
\pgfpathcurveto{\pgfqpoint{9.309224in}{1.985105in}}{\pgfqpoint{9.313614in}{1.995704in}}{\pgfqpoint{9.313614in}{2.006754in}}%
\pgfpathcurveto{\pgfqpoint{9.313614in}{2.017804in}}{\pgfqpoint{9.309224in}{2.028403in}}{\pgfqpoint{9.301410in}{2.036217in}}%
\pgfpathcurveto{\pgfqpoint{9.293597in}{2.044030in}}{\pgfqpoint{9.282998in}{2.048421in}}{\pgfqpoint{9.271948in}{2.048421in}}%
\pgfpathcurveto{\pgfqpoint{9.260898in}{2.048421in}}{\pgfqpoint{9.250299in}{2.044030in}}{\pgfqpoint{9.242485in}{2.036217in}}%
\pgfpathcurveto{\pgfqpoint{9.234671in}{2.028403in}}{\pgfqpoint{9.230281in}{2.017804in}}{\pgfqpoint{9.230281in}{2.006754in}}%
\pgfpathcurveto{\pgfqpoint{9.230281in}{1.995704in}}{\pgfqpoint{9.234671in}{1.985105in}}{\pgfqpoint{9.242485in}{1.977291in}}%
\pgfpathcurveto{\pgfqpoint{9.250299in}{1.969477in}}{\pgfqpoint{9.260898in}{1.965087in}}{\pgfqpoint{9.271948in}{1.965087in}}%
\pgfpathclose%
\pgfusepath{stroke,fill}%
\end{pgfscope}%
\begin{pgfscope}%
\pgfpathrectangle{\pgfqpoint{7.640588in}{0.566125in}}{\pgfqpoint{5.699255in}{2.685432in}}%
\pgfusepath{clip}%
\pgfsetbuttcap%
\pgfsetroundjoin%
\definecolor{currentfill}{rgb}{0.000000,0.000000,0.000000}%
\pgfsetfillcolor{currentfill}%
\pgfsetlinewidth{1.003750pt}%
\definecolor{currentstroke}{rgb}{0.000000,0.000000,0.000000}%
\pgfsetstrokecolor{currentstroke}%
\pgfsetdash{}{0pt}%
\pgfpathmoveto{\pgfqpoint{9.075905in}{1.991197in}}%
\pgfpathcurveto{\pgfqpoint{9.086955in}{1.991197in}}{\pgfqpoint{9.097554in}{1.995588in}}{\pgfqpoint{9.105367in}{2.003401in}}%
\pgfpathcurveto{\pgfqpoint{9.113181in}{2.011215in}}{\pgfqpoint{9.117571in}{2.021814in}}{\pgfqpoint{9.117571in}{2.032864in}}%
\pgfpathcurveto{\pgfqpoint{9.117571in}{2.043914in}}{\pgfqpoint{9.113181in}{2.054513in}}{\pgfqpoint{9.105367in}{2.062327in}}%
\pgfpathcurveto{\pgfqpoint{9.097554in}{2.070140in}}{\pgfqpoint{9.086955in}{2.074531in}}{\pgfqpoint{9.075905in}{2.074531in}}%
\pgfpathcurveto{\pgfqpoint{9.064854in}{2.074531in}}{\pgfqpoint{9.054255in}{2.070140in}}{\pgfqpoint{9.046442in}{2.062327in}}%
\pgfpathcurveto{\pgfqpoint{9.038628in}{2.054513in}}{\pgfqpoint{9.034238in}{2.043914in}}{\pgfqpoint{9.034238in}{2.032864in}}%
\pgfpathcurveto{\pgfqpoint{9.034238in}{2.021814in}}{\pgfqpoint{9.038628in}{2.011215in}}{\pgfqpoint{9.046442in}{2.003401in}}%
\pgfpathcurveto{\pgfqpoint{9.054255in}{1.995588in}}{\pgfqpoint{9.064854in}{1.991197in}}{\pgfqpoint{9.075905in}{1.991197in}}%
\pgfpathclose%
\pgfusepath{stroke,fill}%
\end{pgfscope}%
\begin{pgfscope}%
\pgfpathrectangle{\pgfqpoint{7.640588in}{0.566125in}}{\pgfqpoint{5.699255in}{2.685432in}}%
\pgfusepath{clip}%
\pgfsetbuttcap%
\pgfsetroundjoin%
\definecolor{currentfill}{rgb}{0.000000,0.000000,0.000000}%
\pgfsetfillcolor{currentfill}%
\pgfsetlinewidth{1.003750pt}%
\definecolor{currentstroke}{rgb}{0.000000,0.000000,0.000000}%
\pgfsetstrokecolor{currentstroke}%
\pgfsetdash{}{0pt}%
\pgfpathmoveto{\pgfqpoint{8.487775in}{1.808426in}}%
\pgfpathcurveto{\pgfqpoint{8.498825in}{1.808426in}}{\pgfqpoint{8.509424in}{1.812816in}}{\pgfqpoint{8.517238in}{1.820630in}}%
\pgfpathcurveto{\pgfqpoint{8.525051in}{1.828444in}}{\pgfqpoint{8.529442in}{1.839043in}}{\pgfqpoint{8.529442in}{1.850093in}}%
\pgfpathcurveto{\pgfqpoint{8.529442in}{1.861143in}}{\pgfqpoint{8.525051in}{1.871742in}}{\pgfqpoint{8.517238in}{1.879556in}}%
\pgfpathcurveto{\pgfqpoint{8.509424in}{1.887369in}}{\pgfqpoint{8.498825in}{1.891759in}}{\pgfqpoint{8.487775in}{1.891759in}}%
\pgfpathcurveto{\pgfqpoint{8.476725in}{1.891759in}}{\pgfqpoint{8.466126in}{1.887369in}}{\pgfqpoint{8.458312in}{1.879556in}}%
\pgfpathcurveto{\pgfqpoint{8.450499in}{1.871742in}}{\pgfqpoint{8.446108in}{1.861143in}}{\pgfqpoint{8.446108in}{1.850093in}}%
\pgfpathcurveto{\pgfqpoint{8.446108in}{1.839043in}}{\pgfqpoint{8.450499in}{1.828444in}}{\pgfqpoint{8.458312in}{1.820630in}}%
\pgfpathcurveto{\pgfqpoint{8.466126in}{1.812816in}}{\pgfqpoint{8.476725in}{1.808426in}}{\pgfqpoint{8.487775in}{1.808426in}}%
\pgfpathclose%
\pgfusepath{stroke,fill}%
\end{pgfscope}%
\begin{pgfscope}%
\pgfpathrectangle{\pgfqpoint{7.640588in}{0.566125in}}{\pgfqpoint{5.699255in}{2.685432in}}%
\pgfusepath{clip}%
\pgfsetbuttcap%
\pgfsetroundjoin%
\definecolor{currentfill}{rgb}{0.000000,0.000000,0.000000}%
\pgfsetfillcolor{currentfill}%
\pgfsetlinewidth{1.003750pt}%
\definecolor{currentstroke}{rgb}{0.000000,0.000000,0.000000}%
\pgfsetstrokecolor{currentstroke}%
\pgfsetdash{}{0pt}%
\pgfpathmoveto{\pgfqpoint{8.543787in}{1.612600in}}%
\pgfpathcurveto{\pgfqpoint{8.554837in}{1.612600in}}{\pgfqpoint{8.565436in}{1.616990in}}{\pgfqpoint{8.573250in}{1.624804in}}%
\pgfpathcurveto{\pgfqpoint{8.581064in}{1.632617in}}{\pgfqpoint{8.585454in}{1.643216in}}{\pgfqpoint{8.585454in}{1.654266in}}%
\pgfpathcurveto{\pgfqpoint{8.585454in}{1.665317in}}{\pgfqpoint{8.581064in}{1.675916in}}{\pgfqpoint{8.573250in}{1.683729in}}%
\pgfpathcurveto{\pgfqpoint{8.565436in}{1.691543in}}{\pgfqpoint{8.554837in}{1.695933in}}{\pgfqpoint{8.543787in}{1.695933in}}%
\pgfpathcurveto{\pgfqpoint{8.532737in}{1.695933in}}{\pgfqpoint{8.522138in}{1.691543in}}{\pgfqpoint{8.514325in}{1.683729in}}%
\pgfpathcurveto{\pgfqpoint{8.506511in}{1.675916in}}{\pgfqpoint{8.502121in}{1.665317in}}{\pgfqpoint{8.502121in}{1.654266in}}%
\pgfpathcurveto{\pgfqpoint{8.502121in}{1.643216in}}{\pgfqpoint{8.506511in}{1.632617in}}{\pgfqpoint{8.514325in}{1.624804in}}%
\pgfpathcurveto{\pgfqpoint{8.522138in}{1.616990in}}{\pgfqpoint{8.532737in}{1.612600in}}{\pgfqpoint{8.543787in}{1.612600in}}%
\pgfpathclose%
\pgfusepath{stroke,fill}%
\end{pgfscope}%
\begin{pgfscope}%
\pgfpathrectangle{\pgfqpoint{7.640588in}{0.566125in}}{\pgfqpoint{5.699255in}{2.685432in}}%
\pgfusepath{clip}%
\pgfsetbuttcap%
\pgfsetroundjoin%
\definecolor{currentfill}{rgb}{0.000000,0.000000,0.000000}%
\pgfsetfillcolor{currentfill}%
\pgfsetlinewidth{1.003750pt}%
\definecolor{currentstroke}{rgb}{0.000000,0.000000,0.000000}%
\pgfsetstrokecolor{currentstroke}%
\pgfsetdash{}{0pt}%
\pgfpathmoveto{\pgfqpoint{8.571793in}{1.455939in}}%
\pgfpathcurveto{\pgfqpoint{8.582844in}{1.455939in}}{\pgfqpoint{8.593443in}{1.460329in}}{\pgfqpoint{8.601256in}{1.468143in}}%
\pgfpathcurveto{\pgfqpoint{8.609070in}{1.475956in}}{\pgfqpoint{8.613460in}{1.486555in}}{\pgfqpoint{8.613460in}{1.497605in}}%
\pgfpathcurveto{\pgfqpoint{8.613460in}{1.508655in}}{\pgfqpoint{8.609070in}{1.519254in}}{\pgfqpoint{8.601256in}{1.527068in}}%
\pgfpathcurveto{\pgfqpoint{8.593443in}{1.534882in}}{\pgfqpoint{8.582844in}{1.539272in}}{\pgfqpoint{8.571793in}{1.539272in}}%
\pgfpathcurveto{\pgfqpoint{8.560743in}{1.539272in}}{\pgfqpoint{8.550144in}{1.534882in}}{\pgfqpoint{8.542331in}{1.527068in}}%
\pgfpathcurveto{\pgfqpoint{8.534517in}{1.519254in}}{\pgfqpoint{8.530127in}{1.508655in}}{\pgfqpoint{8.530127in}{1.497605in}}%
\pgfpathcurveto{\pgfqpoint{8.530127in}{1.486555in}}{\pgfqpoint{8.534517in}{1.475956in}}{\pgfqpoint{8.542331in}{1.468143in}}%
\pgfpathcurveto{\pgfqpoint{8.550144in}{1.460329in}}{\pgfqpoint{8.560743in}{1.455939in}}{\pgfqpoint{8.571793in}{1.455939in}}%
\pgfpathclose%
\pgfusepath{stroke,fill}%
\end{pgfscope}%
\begin{pgfscope}%
\pgfpathrectangle{\pgfqpoint{7.640588in}{0.566125in}}{\pgfqpoint{5.699255in}{2.685432in}}%
\pgfusepath{clip}%
\pgfsetbuttcap%
\pgfsetroundjoin%
\definecolor{currentfill}{rgb}{0.000000,0.000000,0.000000}%
\pgfsetfillcolor{currentfill}%
\pgfsetlinewidth{1.003750pt}%
\definecolor{currentstroke}{rgb}{0.000000,0.000000,0.000000}%
\pgfsetstrokecolor{currentstroke}%
\pgfsetdash{}{0pt}%
\pgfpathmoveto{\pgfqpoint{8.543787in}{1.351498in}}%
\pgfpathcurveto{\pgfqpoint{8.554837in}{1.351498in}}{\pgfqpoint{8.565436in}{1.355888in}}{\pgfqpoint{8.573250in}{1.363702in}}%
\pgfpathcurveto{\pgfqpoint{8.581064in}{1.371515in}}{\pgfqpoint{8.585454in}{1.382114in}}{\pgfqpoint{8.585454in}{1.393165in}}%
\pgfpathcurveto{\pgfqpoint{8.585454in}{1.404215in}}{\pgfqpoint{8.581064in}{1.414814in}}{\pgfqpoint{8.573250in}{1.422627in}}%
\pgfpathcurveto{\pgfqpoint{8.565436in}{1.430441in}}{\pgfqpoint{8.554837in}{1.434831in}}{\pgfqpoint{8.543787in}{1.434831in}}%
\pgfpathcurveto{\pgfqpoint{8.532737in}{1.434831in}}{\pgfqpoint{8.522138in}{1.430441in}}{\pgfqpoint{8.514325in}{1.422627in}}%
\pgfpathcurveto{\pgfqpoint{8.506511in}{1.414814in}}{\pgfqpoint{8.502121in}{1.404215in}}{\pgfqpoint{8.502121in}{1.393165in}}%
\pgfpathcurveto{\pgfqpoint{8.502121in}{1.382114in}}{\pgfqpoint{8.506511in}{1.371515in}}{\pgfqpoint{8.514325in}{1.363702in}}%
\pgfpathcurveto{\pgfqpoint{8.522138in}{1.355888in}}{\pgfqpoint{8.532737in}{1.351498in}}{\pgfqpoint{8.543787in}{1.351498in}}%
\pgfpathclose%
\pgfusepath{stroke,fill}%
\end{pgfscope}%
\begin{pgfscope}%
\pgfpathrectangle{\pgfqpoint{7.640588in}{0.566125in}}{\pgfqpoint{5.699255in}{2.685432in}}%
\pgfusepath{clip}%
\pgfsetbuttcap%
\pgfsetroundjoin%
\definecolor{currentfill}{rgb}{0.000000,0.000000,0.000000}%
\pgfsetfillcolor{currentfill}%
\pgfsetlinewidth{1.003750pt}%
\definecolor{currentstroke}{rgb}{0.000000,0.000000,0.000000}%
\pgfsetstrokecolor{currentstroke}%
\pgfsetdash{}{0pt}%
\pgfpathmoveto{\pgfqpoint{8.571793in}{1.234002in}}%
\pgfpathcurveto{\pgfqpoint{8.582844in}{1.234002in}}{\pgfqpoint{8.593443in}{1.238392in}}{\pgfqpoint{8.601256in}{1.246206in}}%
\pgfpathcurveto{\pgfqpoint{8.609070in}{1.254020in}}{\pgfqpoint{8.613460in}{1.264619in}}{\pgfqpoint{8.613460in}{1.275669in}}%
\pgfpathcurveto{\pgfqpoint{8.613460in}{1.286719in}}{\pgfqpoint{8.609070in}{1.297318in}}{\pgfqpoint{8.601256in}{1.305132in}}%
\pgfpathcurveto{\pgfqpoint{8.593443in}{1.312945in}}{\pgfqpoint{8.582844in}{1.317335in}}{\pgfqpoint{8.571793in}{1.317335in}}%
\pgfpathcurveto{\pgfqpoint{8.560743in}{1.317335in}}{\pgfqpoint{8.550144in}{1.312945in}}{\pgfqpoint{8.542331in}{1.305132in}}%
\pgfpathcurveto{\pgfqpoint{8.534517in}{1.297318in}}{\pgfqpoint{8.530127in}{1.286719in}}{\pgfqpoint{8.530127in}{1.275669in}}%
\pgfpathcurveto{\pgfqpoint{8.530127in}{1.264619in}}{\pgfqpoint{8.534517in}{1.254020in}}{\pgfqpoint{8.542331in}{1.246206in}}%
\pgfpathcurveto{\pgfqpoint{8.550144in}{1.238392in}}{\pgfqpoint{8.560743in}{1.234002in}}{\pgfqpoint{8.571793in}{1.234002in}}%
\pgfpathclose%
\pgfusepath{stroke,fill}%
\end{pgfscope}%
\begin{pgfscope}%
\pgfpathrectangle{\pgfqpoint{7.640588in}{0.566125in}}{\pgfqpoint{5.699255in}{2.685432in}}%
\pgfusepath{clip}%
\pgfsetbuttcap%
\pgfsetroundjoin%
\definecolor{currentfill}{rgb}{0.000000,0.000000,0.000000}%
\pgfsetfillcolor{currentfill}%
\pgfsetlinewidth{1.003750pt}%
\definecolor{currentstroke}{rgb}{0.000000,0.000000,0.000000}%
\pgfsetstrokecolor{currentstroke}%
\pgfsetdash{}{0pt}%
\pgfpathmoveto{\pgfqpoint{8.179707in}{1.312333in}}%
\pgfpathcurveto{\pgfqpoint{8.190757in}{1.312333in}}{\pgfqpoint{8.201356in}{1.316723in}}{\pgfqpoint{8.209170in}{1.324537in}}%
\pgfpathcurveto{\pgfqpoint{8.216984in}{1.332350in}}{\pgfqpoint{8.221374in}{1.342949in}}{\pgfqpoint{8.221374in}{1.353999in}}%
\pgfpathcurveto{\pgfqpoint{8.221374in}{1.365049in}}{\pgfqpoint{8.216984in}{1.375648in}}{\pgfqpoint{8.209170in}{1.383462in}}%
\pgfpathcurveto{\pgfqpoint{8.201356in}{1.391276in}}{\pgfqpoint{8.190757in}{1.395666in}}{\pgfqpoint{8.179707in}{1.395666in}}%
\pgfpathcurveto{\pgfqpoint{8.168657in}{1.395666in}}{\pgfqpoint{8.158058in}{1.391276in}}{\pgfqpoint{8.150244in}{1.383462in}}%
\pgfpathcurveto{\pgfqpoint{8.142431in}{1.375648in}}{\pgfqpoint{8.138040in}{1.365049in}}{\pgfqpoint{8.138040in}{1.353999in}}%
\pgfpathcurveto{\pgfqpoint{8.138040in}{1.342949in}}{\pgfqpoint{8.142431in}{1.332350in}}{\pgfqpoint{8.150244in}{1.324537in}}%
\pgfpathcurveto{\pgfqpoint{8.158058in}{1.316723in}}{\pgfqpoint{8.168657in}{1.312333in}}{\pgfqpoint{8.179707in}{1.312333in}}%
\pgfpathclose%
\pgfusepath{stroke,fill}%
\end{pgfscope}%
\begin{pgfscope}%
\pgfpathrectangle{\pgfqpoint{7.640588in}{0.566125in}}{\pgfqpoint{5.699255in}{2.685432in}}%
\pgfusepath{clip}%
\pgfsetbuttcap%
\pgfsetroundjoin%
\definecolor{currentfill}{rgb}{0.000000,0.000000,0.000000}%
\pgfsetfillcolor{currentfill}%
\pgfsetlinewidth{1.003750pt}%
\definecolor{currentstroke}{rgb}{0.000000,0.000000,0.000000}%
\pgfsetstrokecolor{currentstroke}%
\pgfsetdash{}{0pt}%
\pgfpathmoveto{\pgfqpoint{8.039676in}{1.351498in}}%
\pgfpathcurveto{\pgfqpoint{8.050726in}{1.351498in}}{\pgfqpoint{8.061325in}{1.355888in}}{\pgfqpoint{8.069139in}{1.363702in}}%
\pgfpathcurveto{\pgfqpoint{8.076953in}{1.371515in}}{\pgfqpoint{8.081343in}{1.382114in}}{\pgfqpoint{8.081343in}{1.393165in}}%
\pgfpathcurveto{\pgfqpoint{8.081343in}{1.404215in}}{\pgfqpoint{8.076953in}{1.414814in}}{\pgfqpoint{8.069139in}{1.422627in}}%
\pgfpathcurveto{\pgfqpoint{8.061325in}{1.430441in}}{\pgfqpoint{8.050726in}{1.434831in}}{\pgfqpoint{8.039676in}{1.434831in}}%
\pgfpathcurveto{\pgfqpoint{8.028626in}{1.434831in}}{\pgfqpoint{8.018027in}{1.430441in}}{\pgfqpoint{8.010213in}{1.422627in}}%
\pgfpathcurveto{\pgfqpoint{8.002400in}{1.414814in}}{\pgfqpoint{7.998010in}{1.404215in}}{\pgfqpoint{7.998010in}{1.393165in}}%
\pgfpathcurveto{\pgfqpoint{7.998010in}{1.382114in}}{\pgfqpoint{8.002400in}{1.371515in}}{\pgfqpoint{8.010213in}{1.363702in}}%
\pgfpathcurveto{\pgfqpoint{8.018027in}{1.355888in}}{\pgfqpoint{8.028626in}{1.351498in}}{\pgfqpoint{8.039676in}{1.351498in}}%
\pgfpathclose%
\pgfusepath{stroke,fill}%
\end{pgfscope}%
\begin{pgfscope}%
\pgfpathrectangle{\pgfqpoint{7.640588in}{0.566125in}}{\pgfqpoint{5.699255in}{2.685432in}}%
\pgfusepath{clip}%
\pgfsetbuttcap%
\pgfsetroundjoin%
\definecolor{currentfill}{rgb}{0.000000,0.000000,0.000000}%
\pgfsetfillcolor{currentfill}%
\pgfsetlinewidth{1.003750pt}%
\definecolor{currentstroke}{rgb}{0.000000,0.000000,0.000000}%
\pgfsetstrokecolor{currentstroke}%
\pgfsetdash{}{0pt}%
\pgfpathmoveto{\pgfqpoint{8.291732in}{1.142616in}}%
\pgfpathcurveto{\pgfqpoint{8.302782in}{1.142616in}}{\pgfqpoint{8.313381in}{1.147007in}}{\pgfqpoint{8.321195in}{1.154820in}}%
\pgfpathcurveto{\pgfqpoint{8.329008in}{1.162634in}}{\pgfqpoint{8.333398in}{1.173233in}}{\pgfqpoint{8.333398in}{1.184283in}}%
\pgfpathcurveto{\pgfqpoint{8.333398in}{1.195333in}}{\pgfqpoint{8.329008in}{1.205932in}}{\pgfqpoint{8.321195in}{1.213746in}}%
\pgfpathcurveto{\pgfqpoint{8.313381in}{1.221560in}}{\pgfqpoint{8.302782in}{1.225950in}}{\pgfqpoint{8.291732in}{1.225950in}}%
\pgfpathcurveto{\pgfqpoint{8.280682in}{1.225950in}}{\pgfqpoint{8.270083in}{1.221560in}}{\pgfqpoint{8.262269in}{1.213746in}}%
\pgfpathcurveto{\pgfqpoint{8.254455in}{1.205932in}}{\pgfqpoint{8.250065in}{1.195333in}}{\pgfqpoint{8.250065in}{1.184283in}}%
\pgfpathcurveto{\pgfqpoint{8.250065in}{1.173233in}}{\pgfqpoint{8.254455in}{1.162634in}}{\pgfqpoint{8.262269in}{1.154820in}}%
\pgfpathcurveto{\pgfqpoint{8.270083in}{1.147007in}}{\pgfqpoint{8.280682in}{1.142616in}}{\pgfqpoint{8.291732in}{1.142616in}}%
\pgfpathclose%
\pgfusepath{stroke,fill}%
\end{pgfscope}%
\begin{pgfscope}%
\pgfpathrectangle{\pgfqpoint{7.640588in}{0.566125in}}{\pgfqpoint{5.699255in}{2.685432in}}%
\pgfusepath{clip}%
\pgfsetbuttcap%
\pgfsetroundjoin%
\definecolor{currentfill}{rgb}{0.000000,0.000000,0.000000}%
\pgfsetfillcolor{currentfill}%
\pgfsetlinewidth{1.003750pt}%
\definecolor{currentstroke}{rgb}{0.000000,0.000000,0.000000}%
\pgfsetstrokecolor{currentstroke}%
\pgfsetdash{}{0pt}%
\pgfpathmoveto{\pgfqpoint{8.599800in}{0.946790in}}%
\pgfpathcurveto{\pgfqpoint{8.610850in}{0.946790in}}{\pgfqpoint{8.621449in}{0.951180in}}{\pgfqpoint{8.629262in}{0.958994in}}%
\pgfpathcurveto{\pgfqpoint{8.637076in}{0.966808in}}{\pgfqpoint{8.641466in}{0.977407in}}{\pgfqpoint{8.641466in}{0.988457in}}%
\pgfpathcurveto{\pgfqpoint{8.641466in}{0.999507in}}{\pgfqpoint{8.637076in}{1.010106in}}{\pgfqpoint{8.629262in}{1.017920in}}%
\pgfpathcurveto{\pgfqpoint{8.621449in}{1.025733in}}{\pgfqpoint{8.610850in}{1.030123in}}{\pgfqpoint{8.599800in}{1.030123in}}%
\pgfpathcurveto{\pgfqpoint{8.588750in}{1.030123in}}{\pgfqpoint{8.578150in}{1.025733in}}{\pgfqpoint{8.570337in}{1.017920in}}%
\pgfpathcurveto{\pgfqpoint{8.562523in}{1.010106in}}{\pgfqpoint{8.558133in}{0.999507in}}{\pgfqpoint{8.558133in}{0.988457in}}%
\pgfpathcurveto{\pgfqpoint{8.558133in}{0.977407in}}{\pgfqpoint{8.562523in}{0.966808in}}{\pgfqpoint{8.570337in}{0.958994in}}%
\pgfpathcurveto{\pgfqpoint{8.578150in}{0.951180in}}{\pgfqpoint{8.588750in}{0.946790in}}{\pgfqpoint{8.599800in}{0.946790in}}%
\pgfpathclose%
\pgfusepath{stroke,fill}%
\end{pgfscope}%
\begin{pgfscope}%
\pgfpathrectangle{\pgfqpoint{7.640588in}{0.566125in}}{\pgfqpoint{5.699255in}{2.685432in}}%
\pgfusepath{clip}%
\pgfsetbuttcap%
\pgfsetroundjoin%
\definecolor{currentfill}{rgb}{0.000000,0.000000,0.000000}%
\pgfsetfillcolor{currentfill}%
\pgfsetlinewidth{1.003750pt}%
\definecolor{currentstroke}{rgb}{0.000000,0.000000,0.000000}%
\pgfsetstrokecolor{currentstroke}%
\pgfsetdash{}{0pt}%
\pgfpathmoveto{\pgfqpoint{8.263726in}{0.790129in}}%
\pgfpathcurveto{\pgfqpoint{8.274776in}{0.790129in}}{\pgfqpoint{8.285375in}{0.794519in}}{\pgfqpoint{8.293188in}{0.802333in}}%
\pgfpathcurveto{\pgfqpoint{8.301002in}{0.810147in}}{\pgfqpoint{8.305392in}{0.820746in}}{\pgfqpoint{8.305392in}{0.831796in}}%
\pgfpathcurveto{\pgfqpoint{8.305392in}{0.842846in}}{\pgfqpoint{8.301002in}{0.853445in}}{\pgfqpoint{8.293188in}{0.861258in}}%
\pgfpathcurveto{\pgfqpoint{8.285375in}{0.869072in}}{\pgfqpoint{8.274776in}{0.873462in}}{\pgfqpoint{8.263726in}{0.873462in}}%
\pgfpathcurveto{\pgfqpoint{8.252675in}{0.873462in}}{\pgfqpoint{8.242076in}{0.869072in}}{\pgfqpoint{8.234263in}{0.861258in}}%
\pgfpathcurveto{\pgfqpoint{8.226449in}{0.853445in}}{\pgfqpoint{8.222059in}{0.842846in}}{\pgfqpoint{8.222059in}{0.831796in}}%
\pgfpathcurveto{\pgfqpoint{8.222059in}{0.820746in}}{\pgfqpoint{8.226449in}{0.810147in}}{\pgfqpoint{8.234263in}{0.802333in}}%
\pgfpathcurveto{\pgfqpoint{8.242076in}{0.794519in}}{\pgfqpoint{8.252675in}{0.790129in}}{\pgfqpoint{8.263726in}{0.790129in}}%
\pgfpathclose%
\pgfusepath{stroke,fill}%
\end{pgfscope}%
\begin{pgfscope}%
\pgfpathrectangle{\pgfqpoint{7.640588in}{0.566125in}}{\pgfqpoint{5.699255in}{2.685432in}}%
\pgfusepath{clip}%
\pgfsetbuttcap%
\pgfsetroundjoin%
\definecolor{currentfill}{rgb}{0.000000,0.000000,0.000000}%
\pgfsetfillcolor{currentfill}%
\pgfsetlinewidth{1.003750pt}%
\definecolor{currentstroke}{rgb}{0.000000,0.000000,0.000000}%
\pgfsetstrokecolor{currentstroke}%
\pgfsetdash{}{0pt}%
\pgfpathmoveto{\pgfqpoint{8.403756in}{0.646523in}}%
\pgfpathcurveto{\pgfqpoint{8.414807in}{0.646523in}}{\pgfqpoint{8.425406in}{0.650913in}}{\pgfqpoint{8.433219in}{0.658727in}}%
\pgfpathcurveto{\pgfqpoint{8.441033in}{0.666541in}}{\pgfqpoint{8.445423in}{0.677140in}}{\pgfqpoint{8.445423in}{0.688190in}}%
\pgfpathcurveto{\pgfqpoint{8.445423in}{0.699240in}}{\pgfqpoint{8.441033in}{0.709839in}}{\pgfqpoint{8.433219in}{0.717652in}}%
\pgfpathcurveto{\pgfqpoint{8.425406in}{0.725466in}}{\pgfqpoint{8.414807in}{0.729856in}}{\pgfqpoint{8.403756in}{0.729856in}}%
\pgfpathcurveto{\pgfqpoint{8.392706in}{0.729856in}}{\pgfqpoint{8.382107in}{0.725466in}}{\pgfqpoint{8.374294in}{0.717652in}}%
\pgfpathcurveto{\pgfqpoint{8.366480in}{0.709839in}}{\pgfqpoint{8.362090in}{0.699240in}}{\pgfqpoint{8.362090in}{0.688190in}}%
\pgfpathcurveto{\pgfqpoint{8.362090in}{0.677140in}}{\pgfqpoint{8.366480in}{0.666541in}}{\pgfqpoint{8.374294in}{0.658727in}}%
\pgfpathcurveto{\pgfqpoint{8.382107in}{0.650913in}}{\pgfqpoint{8.392706in}{0.646523in}}{\pgfqpoint{8.403756in}{0.646523in}}%
\pgfpathclose%
\pgfusepath{stroke,fill}%
\end{pgfscope}%
\begin{pgfscope}%
\pgfpathrectangle{\pgfqpoint{7.640588in}{0.566125in}}{\pgfqpoint{5.699255in}{2.685432in}}%
\pgfusepath{clip}%
\pgfsetbuttcap%
\pgfsetroundjoin%
\definecolor{currentfill}{rgb}{0.000000,0.000000,0.000000}%
\pgfsetfillcolor{currentfill}%
\pgfsetlinewidth{1.003750pt}%
\definecolor{currentstroke}{rgb}{0.000000,0.000000,0.000000}%
\pgfsetstrokecolor{currentstroke}%
\pgfsetdash{}{0pt}%
\pgfpathmoveto{\pgfqpoint{8.515781in}{0.672633in}}%
\pgfpathcurveto{\pgfqpoint{8.526831in}{0.672633in}}{\pgfqpoint{8.537430in}{0.677023in}}{\pgfqpoint{8.545244in}{0.684837in}}%
\pgfpathcurveto{\pgfqpoint{8.553058in}{0.692651in}}{\pgfqpoint{8.557448in}{0.703250in}}{\pgfqpoint{8.557448in}{0.714300in}}%
\pgfpathcurveto{\pgfqpoint{8.557448in}{0.725350in}}{\pgfqpoint{8.553058in}{0.735949in}}{\pgfqpoint{8.545244in}{0.743763in}}%
\pgfpathcurveto{\pgfqpoint{8.537430in}{0.751576in}}{\pgfqpoint{8.526831in}{0.755967in}}{\pgfqpoint{8.515781in}{0.755967in}}%
\pgfpathcurveto{\pgfqpoint{8.504731in}{0.755967in}}{\pgfqpoint{8.494132in}{0.751576in}}{\pgfqpoint{8.486318in}{0.743763in}}%
\pgfpathcurveto{\pgfqpoint{8.478505in}{0.735949in}}{\pgfqpoint{8.474114in}{0.725350in}}{\pgfqpoint{8.474114in}{0.714300in}}%
\pgfpathcurveto{\pgfqpoint{8.474114in}{0.703250in}}{\pgfqpoint{8.478505in}{0.692651in}}{\pgfqpoint{8.486318in}{0.684837in}}%
\pgfpathcurveto{\pgfqpoint{8.494132in}{0.677023in}}{\pgfqpoint{8.504731in}{0.672633in}}{\pgfqpoint{8.515781in}{0.672633in}}%
\pgfpathclose%
\pgfusepath{stroke,fill}%
\end{pgfscope}%
\begin{pgfscope}%
\pgfpathrectangle{\pgfqpoint{7.640588in}{0.566125in}}{\pgfqpoint{5.699255in}{2.685432in}}%
\pgfusepath{clip}%
\pgfsetbuttcap%
\pgfsetroundjoin%
\definecolor{currentfill}{rgb}{0.000000,0.000000,0.000000}%
\pgfsetfillcolor{currentfill}%
\pgfsetlinewidth{1.003750pt}%
\definecolor{currentstroke}{rgb}{0.000000,0.000000,0.000000}%
\pgfsetstrokecolor{currentstroke}%
\pgfsetdash{}{0pt}%
\pgfpathmoveto{\pgfqpoint{9.075905in}{0.894570in}}%
\pgfpathcurveto{\pgfqpoint{9.086955in}{0.894570in}}{\pgfqpoint{9.097554in}{0.898960in}}{\pgfqpoint{9.105367in}{0.906774in}}%
\pgfpathcurveto{\pgfqpoint{9.113181in}{0.914587in}}{\pgfqpoint{9.117571in}{0.925186in}}{\pgfqpoint{9.117571in}{0.936236in}}%
\pgfpathcurveto{\pgfqpoint{9.117571in}{0.947287in}}{\pgfqpoint{9.113181in}{0.957886in}}{\pgfqpoint{9.105367in}{0.965699in}}%
\pgfpathcurveto{\pgfqpoint{9.097554in}{0.973513in}}{\pgfqpoint{9.086955in}{0.977903in}}{\pgfqpoint{9.075905in}{0.977903in}}%
\pgfpathcurveto{\pgfqpoint{9.064854in}{0.977903in}}{\pgfqpoint{9.054255in}{0.973513in}}{\pgfqpoint{9.046442in}{0.965699in}}%
\pgfpathcurveto{\pgfqpoint{9.038628in}{0.957886in}}{\pgfqpoint{9.034238in}{0.947287in}}{\pgfqpoint{9.034238in}{0.936236in}}%
\pgfpathcurveto{\pgfqpoint{9.034238in}{0.925186in}}{\pgfqpoint{9.038628in}{0.914587in}}{\pgfqpoint{9.046442in}{0.906774in}}%
\pgfpathcurveto{\pgfqpoint{9.054255in}{0.898960in}}{\pgfqpoint{9.064854in}{0.894570in}}{\pgfqpoint{9.075905in}{0.894570in}}%
\pgfpathclose%
\pgfusepath{stroke,fill}%
\end{pgfscope}%
\begin{pgfscope}%
\pgfpathrectangle{\pgfqpoint{7.640588in}{0.566125in}}{\pgfqpoint{5.699255in}{2.685432in}}%
\pgfusepath{clip}%
\pgfsetbuttcap%
\pgfsetroundjoin%
\definecolor{currentfill}{rgb}{0.000000,0.000000,0.000000}%
\pgfsetfillcolor{currentfill}%
\pgfsetlinewidth{1.003750pt}%
\definecolor{currentstroke}{rgb}{0.000000,0.000000,0.000000}%
\pgfsetstrokecolor{currentstroke}%
\pgfsetdash{}{0pt}%
\pgfpathmoveto{\pgfqpoint{9.748053in}{0.803184in}}%
\pgfpathcurveto{\pgfqpoint{9.759103in}{0.803184in}}{\pgfqpoint{9.769702in}{0.807574in}}{\pgfqpoint{9.777515in}{0.815388in}}%
\pgfpathcurveto{\pgfqpoint{9.785329in}{0.823202in}}{\pgfqpoint{9.789719in}{0.833801in}}{\pgfqpoint{9.789719in}{0.844851in}}%
\pgfpathcurveto{\pgfqpoint{9.789719in}{0.855901in}}{\pgfqpoint{9.785329in}{0.866500in}}{\pgfqpoint{9.777515in}{0.874314in}}%
\pgfpathcurveto{\pgfqpoint{9.769702in}{0.882127in}}{\pgfqpoint{9.759103in}{0.886517in}}{\pgfqpoint{9.748053in}{0.886517in}}%
\pgfpathcurveto{\pgfqpoint{9.737002in}{0.886517in}}{\pgfqpoint{9.726403in}{0.882127in}}{\pgfqpoint{9.718590in}{0.874314in}}%
\pgfpathcurveto{\pgfqpoint{9.710776in}{0.866500in}}{\pgfqpoint{9.706386in}{0.855901in}}{\pgfqpoint{9.706386in}{0.844851in}}%
\pgfpathcurveto{\pgfqpoint{9.706386in}{0.833801in}}{\pgfqpoint{9.710776in}{0.823202in}}{\pgfqpoint{9.718590in}{0.815388in}}%
\pgfpathcurveto{\pgfqpoint{9.726403in}{0.807574in}}{\pgfqpoint{9.737002in}{0.803184in}}{\pgfqpoint{9.748053in}{0.803184in}}%
\pgfpathclose%
\pgfusepath{stroke,fill}%
\end{pgfscope}%
\begin{pgfscope}%
\pgfpathrectangle{\pgfqpoint{7.640588in}{0.566125in}}{\pgfqpoint{5.699255in}{2.685432in}}%
\pgfusepath{clip}%
\pgfsetbuttcap%
\pgfsetroundjoin%
\definecolor{currentfill}{rgb}{0.000000,0.000000,0.000000}%
\pgfsetfillcolor{currentfill}%
\pgfsetlinewidth{1.003750pt}%
\definecolor{currentstroke}{rgb}{0.000000,0.000000,0.000000}%
\pgfsetstrokecolor{currentstroke}%
\pgfsetdash{}{0pt}%
\pgfpathmoveto{\pgfqpoint{9.075905in}{1.051231in}}%
\pgfpathcurveto{\pgfqpoint{9.086955in}{1.051231in}}{\pgfqpoint{9.097554in}{1.055621in}}{\pgfqpoint{9.105367in}{1.063435in}}%
\pgfpathcurveto{\pgfqpoint{9.113181in}{1.071248in}}{\pgfqpoint{9.117571in}{1.081847in}}{\pgfqpoint{9.117571in}{1.092898in}}%
\pgfpathcurveto{\pgfqpoint{9.117571in}{1.103948in}}{\pgfqpoint{9.113181in}{1.114547in}}{\pgfqpoint{9.105367in}{1.122360in}}%
\pgfpathcurveto{\pgfqpoint{9.097554in}{1.130174in}}{\pgfqpoint{9.086955in}{1.134564in}}{\pgfqpoint{9.075905in}{1.134564in}}%
\pgfpathcurveto{\pgfqpoint{9.064854in}{1.134564in}}{\pgfqpoint{9.054255in}{1.130174in}}{\pgfqpoint{9.046442in}{1.122360in}}%
\pgfpathcurveto{\pgfqpoint{9.038628in}{1.114547in}}{\pgfqpoint{9.034238in}{1.103948in}}{\pgfqpoint{9.034238in}{1.092898in}}%
\pgfpathcurveto{\pgfqpoint{9.034238in}{1.081847in}}{\pgfqpoint{9.038628in}{1.071248in}}{\pgfqpoint{9.046442in}{1.063435in}}%
\pgfpathcurveto{\pgfqpoint{9.054255in}{1.055621in}}{\pgfqpoint{9.064854in}{1.051231in}}{\pgfqpoint{9.075905in}{1.051231in}}%
\pgfpathclose%
\pgfusepath{stroke,fill}%
\end{pgfscope}%
\begin{pgfscope}%
\pgfpathrectangle{\pgfqpoint{7.640588in}{0.566125in}}{\pgfqpoint{5.699255in}{2.685432in}}%
\pgfusepath{clip}%
\pgfsetbuttcap%
\pgfsetroundjoin%
\definecolor{currentfill}{rgb}{0.000000,0.000000,0.000000}%
\pgfsetfillcolor{currentfill}%
\pgfsetlinewidth{1.003750pt}%
\definecolor{currentstroke}{rgb}{0.000000,0.000000,0.000000}%
\pgfsetstrokecolor{currentstroke}%
\pgfsetdash{}{0pt}%
\pgfpathmoveto{\pgfqpoint{9.047898in}{1.142616in}}%
\pgfpathcurveto{\pgfqpoint{9.058948in}{1.142616in}}{\pgfqpoint{9.069548in}{1.147007in}}{\pgfqpoint{9.077361in}{1.154820in}}%
\pgfpathcurveto{\pgfqpoint{9.085175in}{1.162634in}}{\pgfqpoint{9.089565in}{1.173233in}}{\pgfqpoint{9.089565in}{1.184283in}}%
\pgfpathcurveto{\pgfqpoint{9.089565in}{1.195333in}}{\pgfqpoint{9.085175in}{1.205932in}}{\pgfqpoint{9.077361in}{1.213746in}}%
\pgfpathcurveto{\pgfqpoint{9.069548in}{1.221560in}}{\pgfqpoint{9.058948in}{1.225950in}}{\pgfqpoint{9.047898in}{1.225950in}}%
\pgfpathcurveto{\pgfqpoint{9.036848in}{1.225950in}}{\pgfqpoint{9.026249in}{1.221560in}}{\pgfqpoint{9.018436in}{1.213746in}}%
\pgfpathcurveto{\pgfqpoint{9.010622in}{1.205932in}}{\pgfqpoint{9.006232in}{1.195333in}}{\pgfqpoint{9.006232in}{1.184283in}}%
\pgfpathcurveto{\pgfqpoint{9.006232in}{1.173233in}}{\pgfqpoint{9.010622in}{1.162634in}}{\pgfqpoint{9.018436in}{1.154820in}}%
\pgfpathcurveto{\pgfqpoint{9.026249in}{1.147007in}}{\pgfqpoint{9.036848in}{1.142616in}}{\pgfqpoint{9.047898in}{1.142616in}}%
\pgfpathclose%
\pgfusepath{stroke,fill}%
\end{pgfscope}%
\begin{pgfscope}%
\pgfpathrectangle{\pgfqpoint{7.640588in}{0.566125in}}{\pgfqpoint{5.699255in}{2.685432in}}%
\pgfusepath{clip}%
\pgfsetbuttcap%
\pgfsetroundjoin%
\definecolor{currentfill}{rgb}{0.000000,0.000000,0.000000}%
\pgfsetfillcolor{currentfill}%
\pgfsetlinewidth{1.003750pt}%
\definecolor{currentstroke}{rgb}{0.000000,0.000000,0.000000}%
\pgfsetstrokecolor{currentstroke}%
\pgfsetdash{}{0pt}%
\pgfpathmoveto{\pgfqpoint{9.299954in}{1.234002in}}%
\pgfpathcurveto{\pgfqpoint{9.311004in}{1.234002in}}{\pgfqpoint{9.321603in}{1.238392in}}{\pgfqpoint{9.329417in}{1.246206in}}%
\pgfpathcurveto{\pgfqpoint{9.337230in}{1.254020in}}{\pgfqpoint{9.341621in}{1.264619in}}{\pgfqpoint{9.341621in}{1.275669in}}%
\pgfpathcurveto{\pgfqpoint{9.341621in}{1.286719in}}{\pgfqpoint{9.337230in}{1.297318in}}{\pgfqpoint{9.329417in}{1.305132in}}%
\pgfpathcurveto{\pgfqpoint{9.321603in}{1.312945in}}{\pgfqpoint{9.311004in}{1.317335in}}{\pgfqpoint{9.299954in}{1.317335in}}%
\pgfpathcurveto{\pgfqpoint{9.288904in}{1.317335in}}{\pgfqpoint{9.278305in}{1.312945in}}{\pgfqpoint{9.270491in}{1.305132in}}%
\pgfpathcurveto{\pgfqpoint{9.262677in}{1.297318in}}{\pgfqpoint{9.258287in}{1.286719in}}{\pgfqpoint{9.258287in}{1.275669in}}%
\pgfpathcurveto{\pgfqpoint{9.258287in}{1.264619in}}{\pgfqpoint{9.262677in}{1.254020in}}{\pgfqpoint{9.270491in}{1.246206in}}%
\pgfpathcurveto{\pgfqpoint{9.278305in}{1.238392in}}{\pgfqpoint{9.288904in}{1.234002in}}{\pgfqpoint{9.299954in}{1.234002in}}%
\pgfpathclose%
\pgfusepath{stroke,fill}%
\end{pgfscope}%
\begin{pgfscope}%
\pgfpathrectangle{\pgfqpoint{7.640588in}{0.566125in}}{\pgfqpoint{5.699255in}{2.685432in}}%
\pgfusepath{clip}%
\pgfsetbuttcap%
\pgfsetroundjoin%
\definecolor{currentfill}{rgb}{0.000000,0.000000,0.000000}%
\pgfsetfillcolor{currentfill}%
\pgfsetlinewidth{1.003750pt}%
\definecolor{currentstroke}{rgb}{0.000000,0.000000,0.000000}%
\pgfsetstrokecolor{currentstroke}%
\pgfsetdash{}{0pt}%
\pgfpathmoveto{\pgfqpoint{9.327960in}{1.260112in}}%
\pgfpathcurveto{\pgfqpoint{9.339010in}{1.260112in}}{\pgfqpoint{9.349609in}{1.264503in}}{\pgfqpoint{9.357423in}{1.272316in}}%
\pgfpathcurveto{\pgfqpoint{9.365236in}{1.280130in}}{\pgfqpoint{9.369627in}{1.290729in}}{\pgfqpoint{9.369627in}{1.301779in}}%
\pgfpathcurveto{\pgfqpoint{9.369627in}{1.312829in}}{\pgfqpoint{9.365236in}{1.323428in}}{\pgfqpoint{9.357423in}{1.331242in}}%
\pgfpathcurveto{\pgfqpoint{9.349609in}{1.339055in}}{\pgfqpoint{9.339010in}{1.343446in}}{\pgfqpoint{9.327960in}{1.343446in}}%
\pgfpathcurveto{\pgfqpoint{9.316910in}{1.343446in}}{\pgfqpoint{9.306311in}{1.339055in}}{\pgfqpoint{9.298497in}{1.331242in}}%
\pgfpathcurveto{\pgfqpoint{9.290684in}{1.323428in}}{\pgfqpoint{9.286293in}{1.312829in}}{\pgfqpoint{9.286293in}{1.301779in}}%
\pgfpathcurveto{\pgfqpoint{9.286293in}{1.290729in}}{\pgfqpoint{9.290684in}{1.280130in}}{\pgfqpoint{9.298497in}{1.272316in}}%
\pgfpathcurveto{\pgfqpoint{9.306311in}{1.264503in}}{\pgfqpoint{9.316910in}{1.260112in}}{\pgfqpoint{9.327960in}{1.260112in}}%
\pgfpathclose%
\pgfusepath{stroke,fill}%
\end{pgfscope}%
\begin{pgfscope}%
\pgfpathrectangle{\pgfqpoint{7.640588in}{0.566125in}}{\pgfqpoint{5.699255in}{2.685432in}}%
\pgfusepath{clip}%
\pgfsetbuttcap%
\pgfsetroundjoin%
\definecolor{currentfill}{rgb}{0.000000,0.000000,0.000000}%
\pgfsetfillcolor{currentfill}%
\pgfsetlinewidth{1.003750pt}%
\definecolor{currentstroke}{rgb}{0.000000,0.000000,0.000000}%
\pgfsetstrokecolor{currentstroke}%
\pgfsetdash{}{0pt}%
\pgfpathmoveto{\pgfqpoint{10.140139in}{1.390663in}}%
\pgfpathcurveto{\pgfqpoint{10.151189in}{1.390663in}}{\pgfqpoint{10.161788in}{1.395053in}}{\pgfqpoint{10.169602in}{1.402867in}}%
\pgfpathcurveto{\pgfqpoint{10.177415in}{1.410681in}}{\pgfqpoint{10.181806in}{1.421280in}}{\pgfqpoint{10.181806in}{1.432330in}}%
\pgfpathcurveto{\pgfqpoint{10.181806in}{1.443380in}}{\pgfqpoint{10.177415in}{1.453979in}}{\pgfqpoint{10.169602in}{1.461793in}}%
\pgfpathcurveto{\pgfqpoint{10.161788in}{1.469606in}}{\pgfqpoint{10.151189in}{1.473997in}}{\pgfqpoint{10.140139in}{1.473997in}}%
\pgfpathcurveto{\pgfqpoint{10.129089in}{1.473997in}}{\pgfqpoint{10.118490in}{1.469606in}}{\pgfqpoint{10.110676in}{1.461793in}}%
\pgfpathcurveto{\pgfqpoint{10.102863in}{1.453979in}}{\pgfqpoint{10.098472in}{1.443380in}}{\pgfqpoint{10.098472in}{1.432330in}}%
\pgfpathcurveto{\pgfqpoint{10.098472in}{1.421280in}}{\pgfqpoint{10.102863in}{1.410681in}}{\pgfqpoint{10.110676in}{1.402867in}}%
\pgfpathcurveto{\pgfqpoint{10.118490in}{1.395053in}}{\pgfqpoint{10.129089in}{1.390663in}}{\pgfqpoint{10.140139in}{1.390663in}}%
\pgfpathclose%
\pgfusepath{stroke,fill}%
\end{pgfscope}%
\begin{pgfscope}%
\pgfpathrectangle{\pgfqpoint{7.640588in}{0.566125in}}{\pgfqpoint{5.699255in}{2.685432in}}%
\pgfusepath{clip}%
\pgfsetbuttcap%
\pgfsetroundjoin%
\definecolor{currentfill}{rgb}{0.000000,0.000000,0.000000}%
\pgfsetfillcolor{currentfill}%
\pgfsetlinewidth{1.003750pt}%
\definecolor{currentstroke}{rgb}{0.000000,0.000000,0.000000}%
\pgfsetstrokecolor{currentstroke}%
\pgfsetdash{}{0pt}%
\pgfpathmoveto{\pgfqpoint{10.364188in}{1.495104in}}%
\pgfpathcurveto{\pgfqpoint{10.375238in}{1.495104in}}{\pgfqpoint{10.385837in}{1.499494in}}{\pgfqpoint{10.393651in}{1.507308in}}%
\pgfpathcurveto{\pgfqpoint{10.401465in}{1.515121in}}{\pgfqpoint{10.405855in}{1.525720in}}{\pgfqpoint{10.405855in}{1.536771in}}%
\pgfpathcurveto{\pgfqpoint{10.405855in}{1.547821in}}{\pgfqpoint{10.401465in}{1.558420in}}{\pgfqpoint{10.393651in}{1.566233in}}%
\pgfpathcurveto{\pgfqpoint{10.385837in}{1.574047in}}{\pgfqpoint{10.375238in}{1.578437in}}{\pgfqpoint{10.364188in}{1.578437in}}%
\pgfpathcurveto{\pgfqpoint{10.353138in}{1.578437in}}{\pgfqpoint{10.342539in}{1.574047in}}{\pgfqpoint{10.334726in}{1.566233in}}%
\pgfpathcurveto{\pgfqpoint{10.326912in}{1.558420in}}{\pgfqpoint{10.322522in}{1.547821in}}{\pgfqpoint{10.322522in}{1.536771in}}%
\pgfpathcurveto{\pgfqpoint{10.322522in}{1.525720in}}{\pgfqpoint{10.326912in}{1.515121in}}{\pgfqpoint{10.334726in}{1.507308in}}%
\pgfpathcurveto{\pgfqpoint{10.342539in}{1.499494in}}{\pgfqpoint{10.353138in}{1.495104in}}{\pgfqpoint{10.364188in}{1.495104in}}%
\pgfpathclose%
\pgfusepath{stroke,fill}%
\end{pgfscope}%
\begin{pgfscope}%
\pgfpathrectangle{\pgfqpoint{7.640588in}{0.566125in}}{\pgfqpoint{5.699255in}{2.685432in}}%
\pgfusepath{clip}%
\pgfsetbuttcap%
\pgfsetroundjoin%
\definecolor{currentfill}{rgb}{0.000000,0.000000,0.000000}%
\pgfsetfillcolor{currentfill}%
\pgfsetlinewidth{1.003750pt}%
\definecolor{currentstroke}{rgb}{0.000000,0.000000,0.000000}%
\pgfsetstrokecolor{currentstroke}%
\pgfsetdash{}{0pt}%
\pgfpathmoveto{\pgfqpoint{10.560231in}{0.855404in}}%
\pgfpathcurveto{\pgfqpoint{10.571282in}{0.855404in}}{\pgfqpoint{10.581881in}{0.859795in}}{\pgfqpoint{10.589694in}{0.867608in}}%
\pgfpathcurveto{\pgfqpoint{10.597508in}{0.875422in}}{\pgfqpoint{10.601898in}{0.886021in}}{\pgfqpoint{10.601898in}{0.897071in}}%
\pgfpathcurveto{\pgfqpoint{10.601898in}{0.908121in}}{\pgfqpoint{10.597508in}{0.918720in}}{\pgfqpoint{10.589694in}{0.926534in}}%
\pgfpathcurveto{\pgfqpoint{10.581881in}{0.934348in}}{\pgfqpoint{10.571282in}{0.938738in}}{\pgfqpoint{10.560231in}{0.938738in}}%
\pgfpathcurveto{\pgfqpoint{10.549181in}{0.938738in}}{\pgfqpoint{10.538582in}{0.934348in}}{\pgfqpoint{10.530769in}{0.926534in}}%
\pgfpathcurveto{\pgfqpoint{10.522955in}{0.918720in}}{\pgfqpoint{10.518565in}{0.908121in}}{\pgfqpoint{10.518565in}{0.897071in}}%
\pgfpathcurveto{\pgfqpoint{10.518565in}{0.886021in}}{\pgfqpoint{10.522955in}{0.875422in}}{\pgfqpoint{10.530769in}{0.867608in}}%
\pgfpathcurveto{\pgfqpoint{10.538582in}{0.859795in}}{\pgfqpoint{10.549181in}{0.855404in}}{\pgfqpoint{10.560231in}{0.855404in}}%
\pgfpathclose%
\pgfusepath{stroke,fill}%
\end{pgfscope}%
\begin{pgfscope}%
\pgfpathrectangle{\pgfqpoint{7.640588in}{0.566125in}}{\pgfqpoint{5.699255in}{2.685432in}}%
\pgfusepath{clip}%
\pgfsetbuttcap%
\pgfsetroundjoin%
\definecolor{currentfill}{rgb}{0.000000,0.000000,0.000000}%
\pgfsetfillcolor{currentfill}%
\pgfsetlinewidth{1.003750pt}%
\definecolor{currentstroke}{rgb}{0.000000,0.000000,0.000000}%
\pgfsetstrokecolor{currentstroke}%
\pgfsetdash{}{0pt}%
\pgfpathmoveto{\pgfqpoint{10.700262in}{0.790129in}}%
\pgfpathcurveto{\pgfqpoint{10.711312in}{0.790129in}}{\pgfqpoint{10.721911in}{0.794519in}}{\pgfqpoint{10.729725in}{0.802333in}}%
\pgfpathcurveto{\pgfqpoint{10.737539in}{0.810147in}}{\pgfqpoint{10.741929in}{0.820746in}}{\pgfqpoint{10.741929in}{0.831796in}}%
\pgfpathcurveto{\pgfqpoint{10.741929in}{0.842846in}}{\pgfqpoint{10.737539in}{0.853445in}}{\pgfqpoint{10.729725in}{0.861258in}}%
\pgfpathcurveto{\pgfqpoint{10.721911in}{0.869072in}}{\pgfqpoint{10.711312in}{0.873462in}}{\pgfqpoint{10.700262in}{0.873462in}}%
\pgfpathcurveto{\pgfqpoint{10.689212in}{0.873462in}}{\pgfqpoint{10.678613in}{0.869072in}}{\pgfqpoint{10.670800in}{0.861258in}}%
\pgfpathcurveto{\pgfqpoint{10.662986in}{0.853445in}}{\pgfqpoint{10.658596in}{0.842846in}}{\pgfqpoint{10.658596in}{0.831796in}}%
\pgfpathcurveto{\pgfqpoint{10.658596in}{0.820746in}}{\pgfqpoint{10.662986in}{0.810147in}}{\pgfqpoint{10.670800in}{0.802333in}}%
\pgfpathcurveto{\pgfqpoint{10.678613in}{0.794519in}}{\pgfqpoint{10.689212in}{0.790129in}}{\pgfqpoint{10.700262in}{0.790129in}}%
\pgfpathclose%
\pgfusepath{stroke,fill}%
\end{pgfscope}%
\begin{pgfscope}%
\pgfpathrectangle{\pgfqpoint{7.640588in}{0.566125in}}{\pgfqpoint{5.699255in}{2.685432in}}%
\pgfusepath{clip}%
\pgfsetbuttcap%
\pgfsetroundjoin%
\definecolor{currentfill}{rgb}{0.000000,0.000000,0.000000}%
\pgfsetfillcolor{currentfill}%
\pgfsetlinewidth{1.003750pt}%
\definecolor{currentstroke}{rgb}{0.000000,0.000000,0.000000}%
\pgfsetstrokecolor{currentstroke}%
\pgfsetdash{}{0pt}%
\pgfpathmoveto{\pgfqpoint{10.756275in}{0.842349in}}%
\pgfpathcurveto{\pgfqpoint{10.767325in}{0.842349in}}{\pgfqpoint{10.777924in}{0.846740in}}{\pgfqpoint{10.785737in}{0.854553in}}%
\pgfpathcurveto{\pgfqpoint{10.793551in}{0.862367in}}{\pgfqpoint{10.797941in}{0.872966in}}{\pgfqpoint{10.797941in}{0.884016in}}%
\pgfpathcurveto{\pgfqpoint{10.797941in}{0.895066in}}{\pgfqpoint{10.793551in}{0.905665in}}{\pgfqpoint{10.785737in}{0.913479in}}%
\pgfpathcurveto{\pgfqpoint{10.777924in}{0.921292in}}{\pgfqpoint{10.767325in}{0.925683in}}{\pgfqpoint{10.756275in}{0.925683in}}%
\pgfpathcurveto{\pgfqpoint{10.745225in}{0.925683in}}{\pgfqpoint{10.734626in}{0.921292in}}{\pgfqpoint{10.726812in}{0.913479in}}%
\pgfpathcurveto{\pgfqpoint{10.718998in}{0.905665in}}{\pgfqpoint{10.714608in}{0.895066in}}{\pgfqpoint{10.714608in}{0.884016in}}%
\pgfpathcurveto{\pgfqpoint{10.714608in}{0.872966in}}{\pgfqpoint{10.718998in}{0.862367in}}{\pgfqpoint{10.726812in}{0.854553in}}%
\pgfpathcurveto{\pgfqpoint{10.734626in}{0.846740in}}{\pgfqpoint{10.745225in}{0.842349in}}{\pgfqpoint{10.756275in}{0.842349in}}%
\pgfpathclose%
\pgfusepath{stroke,fill}%
\end{pgfscope}%
\begin{pgfscope}%
\pgfpathrectangle{\pgfqpoint{7.640588in}{0.566125in}}{\pgfqpoint{5.699255in}{2.685432in}}%
\pgfusepath{clip}%
\pgfsetbuttcap%
\pgfsetroundjoin%
\definecolor{currentfill}{rgb}{0.000000,0.000000,0.000000}%
\pgfsetfillcolor{currentfill}%
\pgfsetlinewidth{1.003750pt}%
\definecolor{currentstroke}{rgb}{0.000000,0.000000,0.000000}%
\pgfsetstrokecolor{currentstroke}%
\pgfsetdash{}{0pt}%
\pgfpathmoveto{\pgfqpoint{11.260386in}{0.881515in}}%
\pgfpathcurveto{\pgfqpoint{11.271436in}{0.881515in}}{\pgfqpoint{11.282035in}{0.885905in}}{\pgfqpoint{11.289848in}{0.893719in}}%
\pgfpathcurveto{\pgfqpoint{11.297662in}{0.901532in}}{\pgfqpoint{11.302052in}{0.912131in}}{\pgfqpoint{11.302052in}{0.923181in}}%
\pgfpathcurveto{\pgfqpoint{11.302052in}{0.934231in}}{\pgfqpoint{11.297662in}{0.944830in}}{\pgfqpoint{11.289848in}{0.952644in}}%
\pgfpathcurveto{\pgfqpoint{11.282035in}{0.960458in}}{\pgfqpoint{11.271436in}{0.964848in}}{\pgfqpoint{11.260386in}{0.964848in}}%
\pgfpathcurveto{\pgfqpoint{11.249336in}{0.964848in}}{\pgfqpoint{11.238737in}{0.960458in}}{\pgfqpoint{11.230923in}{0.952644in}}%
\pgfpathcurveto{\pgfqpoint{11.223109in}{0.944830in}}{\pgfqpoint{11.218719in}{0.934231in}}{\pgfqpoint{11.218719in}{0.923181in}}%
\pgfpathcurveto{\pgfqpoint{11.218719in}{0.912131in}}{\pgfqpoint{11.223109in}{0.901532in}}{\pgfqpoint{11.230923in}{0.893719in}}%
\pgfpathcurveto{\pgfqpoint{11.238737in}{0.885905in}}{\pgfqpoint{11.249336in}{0.881515in}}{\pgfqpoint{11.260386in}{0.881515in}}%
\pgfpathclose%
\pgfusepath{stroke,fill}%
\end{pgfscope}%
\begin{pgfscope}%
\pgfpathrectangle{\pgfqpoint{7.640588in}{0.566125in}}{\pgfqpoint{5.699255in}{2.685432in}}%
\pgfusepath{clip}%
\pgfsetbuttcap%
\pgfsetroundjoin%
\definecolor{currentfill}{rgb}{0.000000,0.000000,0.000000}%
\pgfsetfillcolor{currentfill}%
\pgfsetlinewidth{1.003750pt}%
\definecolor{currentstroke}{rgb}{0.000000,0.000000,0.000000}%
\pgfsetstrokecolor{currentstroke}%
\pgfsetdash{}{0pt}%
\pgfpathmoveto{\pgfqpoint{11.568454in}{0.881515in}}%
\pgfpathcurveto{\pgfqpoint{11.579504in}{0.881515in}}{\pgfqpoint{11.590103in}{0.885905in}}{\pgfqpoint{11.597916in}{0.893719in}}%
\pgfpathcurveto{\pgfqpoint{11.605730in}{0.901532in}}{\pgfqpoint{11.610120in}{0.912131in}}{\pgfqpoint{11.610120in}{0.923181in}}%
\pgfpathcurveto{\pgfqpoint{11.610120in}{0.934231in}}{\pgfqpoint{11.605730in}{0.944830in}}{\pgfqpoint{11.597916in}{0.952644in}}%
\pgfpathcurveto{\pgfqpoint{11.590103in}{0.960458in}}{\pgfqpoint{11.579504in}{0.964848in}}{\pgfqpoint{11.568454in}{0.964848in}}%
\pgfpathcurveto{\pgfqpoint{11.557403in}{0.964848in}}{\pgfqpoint{11.546804in}{0.960458in}}{\pgfqpoint{11.538991in}{0.952644in}}%
\pgfpathcurveto{\pgfqpoint{11.531177in}{0.944830in}}{\pgfqpoint{11.526787in}{0.934231in}}{\pgfqpoint{11.526787in}{0.923181in}}%
\pgfpathcurveto{\pgfqpoint{11.526787in}{0.912131in}}{\pgfqpoint{11.531177in}{0.901532in}}{\pgfqpoint{11.538991in}{0.893719in}}%
\pgfpathcurveto{\pgfqpoint{11.546804in}{0.885905in}}{\pgfqpoint{11.557403in}{0.881515in}}{\pgfqpoint{11.568454in}{0.881515in}}%
\pgfpathclose%
\pgfusepath{stroke,fill}%
\end{pgfscope}%
\begin{pgfscope}%
\pgfpathrectangle{\pgfqpoint{7.640588in}{0.566125in}}{\pgfqpoint{5.699255in}{2.685432in}}%
\pgfusepath{clip}%
\pgfsetbuttcap%
\pgfsetroundjoin%
\definecolor{currentfill}{rgb}{0.000000,0.000000,0.000000}%
\pgfsetfillcolor{currentfill}%
\pgfsetlinewidth{1.003750pt}%
\definecolor{currentstroke}{rgb}{0.000000,0.000000,0.000000}%
\pgfsetstrokecolor{currentstroke}%
\pgfsetdash{}{0pt}%
\pgfpathmoveto{\pgfqpoint{11.568454in}{0.894570in}}%
\pgfpathcurveto{\pgfqpoint{11.579504in}{0.894570in}}{\pgfqpoint{11.590103in}{0.898960in}}{\pgfqpoint{11.597916in}{0.906774in}}%
\pgfpathcurveto{\pgfqpoint{11.605730in}{0.914587in}}{\pgfqpoint{11.610120in}{0.925186in}}{\pgfqpoint{11.610120in}{0.936236in}}%
\pgfpathcurveto{\pgfqpoint{11.610120in}{0.947287in}}{\pgfqpoint{11.605730in}{0.957886in}}{\pgfqpoint{11.597916in}{0.965699in}}%
\pgfpathcurveto{\pgfqpoint{11.590103in}{0.973513in}}{\pgfqpoint{11.579504in}{0.977903in}}{\pgfqpoint{11.568454in}{0.977903in}}%
\pgfpathcurveto{\pgfqpoint{11.557403in}{0.977903in}}{\pgfqpoint{11.546804in}{0.973513in}}{\pgfqpoint{11.538991in}{0.965699in}}%
\pgfpathcurveto{\pgfqpoint{11.531177in}{0.957886in}}{\pgfqpoint{11.526787in}{0.947287in}}{\pgfqpoint{11.526787in}{0.936236in}}%
\pgfpathcurveto{\pgfqpoint{11.526787in}{0.925186in}}{\pgfqpoint{11.531177in}{0.914587in}}{\pgfqpoint{11.538991in}{0.906774in}}%
\pgfpathcurveto{\pgfqpoint{11.546804in}{0.898960in}}{\pgfqpoint{11.557403in}{0.894570in}}{\pgfqpoint{11.568454in}{0.894570in}}%
\pgfpathclose%
\pgfusepath{stroke,fill}%
\end{pgfscope}%
\begin{pgfscope}%
\pgfpathrectangle{\pgfqpoint{7.640588in}{0.566125in}}{\pgfqpoint{5.699255in}{2.685432in}}%
\pgfusepath{clip}%
\pgfsetbuttcap%
\pgfsetroundjoin%
\definecolor{currentfill}{rgb}{0.000000,0.000000,0.000000}%
\pgfsetfillcolor{currentfill}%
\pgfsetlinewidth{1.003750pt}%
\definecolor{currentstroke}{rgb}{0.000000,0.000000,0.000000}%
\pgfsetstrokecolor{currentstroke}%
\pgfsetdash{}{0pt}%
\pgfpathmoveto{\pgfqpoint{11.904528in}{0.933735in}}%
\pgfpathcurveto{\pgfqpoint{11.915578in}{0.933735in}}{\pgfqpoint{11.926177in}{0.938125in}}{\pgfqpoint{11.933990in}{0.945939in}}%
\pgfpathcurveto{\pgfqpoint{11.941804in}{0.953753in}}{\pgfqpoint{11.946194in}{0.964352in}}{\pgfqpoint{11.946194in}{0.975402in}}%
\pgfpathcurveto{\pgfqpoint{11.946194in}{0.986452in}}{\pgfqpoint{11.941804in}{0.997051in}}{\pgfqpoint{11.933990in}{1.004864in}}%
\pgfpathcurveto{\pgfqpoint{11.926177in}{1.012678in}}{\pgfqpoint{11.915578in}{1.017068in}}{\pgfqpoint{11.904528in}{1.017068in}}%
\pgfpathcurveto{\pgfqpoint{11.893477in}{1.017068in}}{\pgfqpoint{11.882878in}{1.012678in}}{\pgfqpoint{11.875065in}{1.004864in}}%
\pgfpathcurveto{\pgfqpoint{11.867251in}{0.997051in}}{\pgfqpoint{11.862861in}{0.986452in}}{\pgfqpoint{11.862861in}{0.975402in}}%
\pgfpathcurveto{\pgfqpoint{11.862861in}{0.964352in}}{\pgfqpoint{11.867251in}{0.953753in}}{\pgfqpoint{11.875065in}{0.945939in}}%
\pgfpathcurveto{\pgfqpoint{11.882878in}{0.938125in}}{\pgfqpoint{11.893477in}{0.933735in}}{\pgfqpoint{11.904528in}{0.933735in}}%
\pgfpathclose%
\pgfusepath{stroke,fill}%
\end{pgfscope}%
\begin{pgfscope}%
\pgfpathrectangle{\pgfqpoint{7.640588in}{0.566125in}}{\pgfqpoint{5.699255in}{2.685432in}}%
\pgfusepath{clip}%
\pgfsetbuttcap%
\pgfsetroundjoin%
\definecolor{currentfill}{rgb}{0.000000,0.000000,0.000000}%
\pgfsetfillcolor{currentfill}%
\pgfsetlinewidth{1.003750pt}%
\definecolor{currentstroke}{rgb}{0.000000,0.000000,0.000000}%
\pgfsetstrokecolor{currentstroke}%
\pgfsetdash{}{0pt}%
\pgfpathmoveto{\pgfqpoint{11.932534in}{0.946790in}}%
\pgfpathcurveto{\pgfqpoint{11.943584in}{0.946790in}}{\pgfqpoint{11.954183in}{0.951180in}}{\pgfqpoint{11.961997in}{0.958994in}}%
\pgfpathcurveto{\pgfqpoint{11.969810in}{0.966808in}}{\pgfqpoint{11.974200in}{0.977407in}}{\pgfqpoint{11.974200in}{0.988457in}}%
\pgfpathcurveto{\pgfqpoint{11.974200in}{0.999507in}}{\pgfqpoint{11.969810in}{1.010106in}}{\pgfqpoint{11.961997in}{1.017920in}}%
\pgfpathcurveto{\pgfqpoint{11.954183in}{1.025733in}}{\pgfqpoint{11.943584in}{1.030123in}}{\pgfqpoint{11.932534in}{1.030123in}}%
\pgfpathcurveto{\pgfqpoint{11.921484in}{1.030123in}}{\pgfqpoint{11.910885in}{1.025733in}}{\pgfqpoint{11.903071in}{1.017920in}}%
\pgfpathcurveto{\pgfqpoint{11.895257in}{1.010106in}}{\pgfqpoint{11.890867in}{0.999507in}}{\pgfqpoint{11.890867in}{0.988457in}}%
\pgfpathcurveto{\pgfqpoint{11.890867in}{0.977407in}}{\pgfqpoint{11.895257in}{0.966808in}}{\pgfqpoint{11.903071in}{0.958994in}}%
\pgfpathcurveto{\pgfqpoint{11.910885in}{0.951180in}}{\pgfqpoint{11.921484in}{0.946790in}}{\pgfqpoint{11.932534in}{0.946790in}}%
\pgfpathclose%
\pgfusepath{stroke,fill}%
\end{pgfscope}%
\begin{pgfscope}%
\pgfpathrectangle{\pgfqpoint{7.640588in}{0.566125in}}{\pgfqpoint{5.699255in}{2.685432in}}%
\pgfusepath{clip}%
\pgfsetbuttcap%
\pgfsetroundjoin%
\definecolor{currentfill}{rgb}{0.000000,0.000000,0.000000}%
\pgfsetfillcolor{currentfill}%
\pgfsetlinewidth{1.003750pt}%
\definecolor{currentstroke}{rgb}{0.000000,0.000000,0.000000}%
\pgfsetstrokecolor{currentstroke}%
\pgfsetdash{}{0pt}%
\pgfpathmoveto{\pgfqpoint{11.932534in}{1.038176in}}%
\pgfpathcurveto{\pgfqpoint{11.943584in}{1.038176in}}{\pgfqpoint{11.954183in}{1.042566in}}{\pgfqpoint{11.961997in}{1.050380in}}%
\pgfpathcurveto{\pgfqpoint{11.969810in}{1.058193in}}{\pgfqpoint{11.974200in}{1.068792in}}{\pgfqpoint{11.974200in}{1.079842in}}%
\pgfpathcurveto{\pgfqpoint{11.974200in}{1.090893in}}{\pgfqpoint{11.969810in}{1.101492in}}{\pgfqpoint{11.961997in}{1.109305in}}%
\pgfpathcurveto{\pgfqpoint{11.954183in}{1.117119in}}{\pgfqpoint{11.943584in}{1.121509in}}{\pgfqpoint{11.932534in}{1.121509in}}%
\pgfpathcurveto{\pgfqpoint{11.921484in}{1.121509in}}{\pgfqpoint{11.910885in}{1.117119in}}{\pgfqpoint{11.903071in}{1.109305in}}%
\pgfpathcurveto{\pgfqpoint{11.895257in}{1.101492in}}{\pgfqpoint{11.890867in}{1.090893in}}{\pgfqpoint{11.890867in}{1.079842in}}%
\pgfpathcurveto{\pgfqpoint{11.890867in}{1.068792in}}{\pgfqpoint{11.895257in}{1.058193in}}{\pgfqpoint{11.903071in}{1.050380in}}%
\pgfpathcurveto{\pgfqpoint{11.910885in}{1.042566in}}{\pgfqpoint{11.921484in}{1.038176in}}{\pgfqpoint{11.932534in}{1.038176in}}%
\pgfpathclose%
\pgfusepath{stroke,fill}%
\end{pgfscope}%
\begin{pgfscope}%
\pgfpathrectangle{\pgfqpoint{7.640588in}{0.566125in}}{\pgfqpoint{5.699255in}{2.685432in}}%
\pgfusepath{clip}%
\pgfsetbuttcap%
\pgfsetroundjoin%
\definecolor{currentfill}{rgb}{0.000000,0.000000,0.000000}%
\pgfsetfillcolor{currentfill}%
\pgfsetlinewidth{1.003750pt}%
\definecolor{currentstroke}{rgb}{0.000000,0.000000,0.000000}%
\pgfsetstrokecolor{currentstroke}%
\pgfsetdash{}{0pt}%
\pgfpathmoveto{\pgfqpoint{11.988546in}{0.985955in}}%
\pgfpathcurveto{\pgfqpoint{11.999596in}{0.985955in}}{\pgfqpoint{12.010195in}{0.990346in}}{\pgfqpoint{12.018009in}{0.998159in}}%
\pgfpathcurveto{\pgfqpoint{12.025823in}{1.005973in}}{\pgfqpoint{12.030213in}{1.016572in}}{\pgfqpoint{12.030213in}{1.027622in}}%
\pgfpathcurveto{\pgfqpoint{12.030213in}{1.038672in}}{\pgfqpoint{12.025823in}{1.049271in}}{\pgfqpoint{12.018009in}{1.057085in}}%
\pgfpathcurveto{\pgfqpoint{12.010195in}{1.064898in}}{\pgfqpoint{11.999596in}{1.069289in}}{\pgfqpoint{11.988546in}{1.069289in}}%
\pgfpathcurveto{\pgfqpoint{11.977496in}{1.069289in}}{\pgfqpoint{11.966897in}{1.064898in}}{\pgfqpoint{11.959083in}{1.057085in}}%
\pgfpathcurveto{\pgfqpoint{11.951270in}{1.049271in}}{\pgfqpoint{11.946879in}{1.038672in}}{\pgfqpoint{11.946879in}{1.027622in}}%
\pgfpathcurveto{\pgfqpoint{11.946879in}{1.016572in}}{\pgfqpoint{11.951270in}{1.005973in}}{\pgfqpoint{11.959083in}{0.998159in}}%
\pgfpathcurveto{\pgfqpoint{11.966897in}{0.990346in}}{\pgfqpoint{11.977496in}{0.985955in}}{\pgfqpoint{11.988546in}{0.985955in}}%
\pgfpathclose%
\pgfusepath{stroke,fill}%
\end{pgfscope}%
\begin{pgfscope}%
\pgfpathrectangle{\pgfqpoint{7.640588in}{0.566125in}}{\pgfqpoint{5.699255in}{2.685432in}}%
\pgfusepath{clip}%
\pgfsetbuttcap%
\pgfsetroundjoin%
\definecolor{currentfill}{rgb}{0.000000,0.000000,0.000000}%
\pgfsetfillcolor{currentfill}%
\pgfsetlinewidth{1.003750pt}%
\definecolor{currentstroke}{rgb}{0.000000,0.000000,0.000000}%
\pgfsetstrokecolor{currentstroke}%
\pgfsetdash{}{0pt}%
\pgfpathmoveto{\pgfqpoint{12.660694in}{0.842349in}}%
\pgfpathcurveto{\pgfqpoint{12.671744in}{0.842349in}}{\pgfqpoint{12.682343in}{0.846740in}}{\pgfqpoint{12.690157in}{0.854553in}}%
\pgfpathcurveto{\pgfqpoint{12.697971in}{0.862367in}}{\pgfqpoint{12.702361in}{0.872966in}}{\pgfqpoint{12.702361in}{0.884016in}}%
\pgfpathcurveto{\pgfqpoint{12.702361in}{0.895066in}}{\pgfqpoint{12.697971in}{0.905665in}}{\pgfqpoint{12.690157in}{0.913479in}}%
\pgfpathcurveto{\pgfqpoint{12.682343in}{0.921292in}}{\pgfqpoint{12.671744in}{0.925683in}}{\pgfqpoint{12.660694in}{0.925683in}}%
\pgfpathcurveto{\pgfqpoint{12.649644in}{0.925683in}}{\pgfqpoint{12.639045in}{0.921292in}}{\pgfqpoint{12.631231in}{0.913479in}}%
\pgfpathcurveto{\pgfqpoint{12.623418in}{0.905665in}}{\pgfqpoint{12.619027in}{0.895066in}}{\pgfqpoint{12.619027in}{0.884016in}}%
\pgfpathcurveto{\pgfqpoint{12.619027in}{0.872966in}}{\pgfqpoint{12.623418in}{0.862367in}}{\pgfqpoint{12.631231in}{0.854553in}}%
\pgfpathcurveto{\pgfqpoint{12.639045in}{0.846740in}}{\pgfqpoint{12.649644in}{0.842349in}}{\pgfqpoint{12.660694in}{0.842349in}}%
\pgfpathclose%
\pgfusepath{stroke,fill}%
\end{pgfscope}%
\begin{pgfscope}%
\pgfpathrectangle{\pgfqpoint{7.640588in}{0.566125in}}{\pgfqpoint{5.699255in}{2.685432in}}%
\pgfusepath{clip}%
\pgfsetbuttcap%
\pgfsetroundjoin%
\definecolor{currentfill}{rgb}{0.000000,0.000000,0.000000}%
\pgfsetfillcolor{currentfill}%
\pgfsetlinewidth{1.003750pt}%
\definecolor{currentstroke}{rgb}{0.000000,0.000000,0.000000}%
\pgfsetstrokecolor{currentstroke}%
\pgfsetdash{}{0pt}%
\pgfpathmoveto{\pgfqpoint{12.296614in}{1.038176in}}%
\pgfpathcurveto{\pgfqpoint{12.307664in}{1.038176in}}{\pgfqpoint{12.318263in}{1.042566in}}{\pgfqpoint{12.326077in}{1.050380in}}%
\pgfpathcurveto{\pgfqpoint{12.333890in}{1.058193in}}{\pgfqpoint{12.338281in}{1.068792in}}{\pgfqpoint{12.338281in}{1.079842in}}%
\pgfpathcurveto{\pgfqpoint{12.338281in}{1.090893in}}{\pgfqpoint{12.333890in}{1.101492in}}{\pgfqpoint{12.326077in}{1.109305in}}%
\pgfpathcurveto{\pgfqpoint{12.318263in}{1.117119in}}{\pgfqpoint{12.307664in}{1.121509in}}{\pgfqpoint{12.296614in}{1.121509in}}%
\pgfpathcurveto{\pgfqpoint{12.285564in}{1.121509in}}{\pgfqpoint{12.274965in}{1.117119in}}{\pgfqpoint{12.267151in}{1.109305in}}%
\pgfpathcurveto{\pgfqpoint{12.259338in}{1.101492in}}{\pgfqpoint{12.254947in}{1.090893in}}{\pgfqpoint{12.254947in}{1.079842in}}%
\pgfpathcurveto{\pgfqpoint{12.254947in}{1.068792in}}{\pgfqpoint{12.259338in}{1.058193in}}{\pgfqpoint{12.267151in}{1.050380in}}%
\pgfpathcurveto{\pgfqpoint{12.274965in}{1.042566in}}{\pgfqpoint{12.285564in}{1.038176in}}{\pgfqpoint{12.296614in}{1.038176in}}%
\pgfpathclose%
\pgfusepath{stroke,fill}%
\end{pgfscope}%
\begin{pgfscope}%
\pgfpathrectangle{\pgfqpoint{7.640588in}{0.566125in}}{\pgfqpoint{5.699255in}{2.685432in}}%
\pgfusepath{clip}%
\pgfsetbuttcap%
\pgfsetroundjoin%
\definecolor{currentfill}{rgb}{0.000000,0.000000,0.000000}%
\pgfsetfillcolor{currentfill}%
\pgfsetlinewidth{1.003750pt}%
\definecolor{currentstroke}{rgb}{0.000000,0.000000,0.000000}%
\pgfsetstrokecolor{currentstroke}%
\pgfsetdash{}{0pt}%
\pgfpathmoveto{\pgfqpoint{12.184589in}{1.181782in}}%
\pgfpathcurveto{\pgfqpoint{12.195639in}{1.181782in}}{\pgfqpoint{12.206238in}{1.186172in}}{\pgfqpoint{12.214052in}{1.193986in}}%
\pgfpathcurveto{\pgfqpoint{12.221866in}{1.201799in}}{\pgfqpoint{12.226256in}{1.212398in}}{\pgfqpoint{12.226256in}{1.223448in}}%
\pgfpathcurveto{\pgfqpoint{12.226256in}{1.234499in}}{\pgfqpoint{12.221866in}{1.245098in}}{\pgfqpoint{12.214052in}{1.252911in}}%
\pgfpathcurveto{\pgfqpoint{12.206238in}{1.260725in}}{\pgfqpoint{12.195639in}{1.265115in}}{\pgfqpoint{12.184589in}{1.265115in}}%
\pgfpathcurveto{\pgfqpoint{12.173539in}{1.265115in}}{\pgfqpoint{12.162940in}{1.260725in}}{\pgfqpoint{12.155126in}{1.252911in}}%
\pgfpathcurveto{\pgfqpoint{12.147313in}{1.245098in}}{\pgfqpoint{12.142923in}{1.234499in}}{\pgfqpoint{12.142923in}{1.223448in}}%
\pgfpathcurveto{\pgfqpoint{12.142923in}{1.212398in}}{\pgfqpoint{12.147313in}{1.201799in}}{\pgfqpoint{12.155126in}{1.193986in}}%
\pgfpathcurveto{\pgfqpoint{12.162940in}{1.186172in}}{\pgfqpoint{12.173539in}{1.181782in}}{\pgfqpoint{12.184589in}{1.181782in}}%
\pgfpathclose%
\pgfusepath{stroke,fill}%
\end{pgfscope}%
\begin{pgfscope}%
\pgfpathrectangle{\pgfqpoint{7.640588in}{0.566125in}}{\pgfqpoint{5.699255in}{2.685432in}}%
\pgfusepath{clip}%
\pgfsetbuttcap%
\pgfsetroundjoin%
\definecolor{currentfill}{rgb}{0.000000,0.000000,0.000000}%
\pgfsetfillcolor{currentfill}%
\pgfsetlinewidth{1.003750pt}%
\definecolor{currentstroke}{rgb}{0.000000,0.000000,0.000000}%
\pgfsetstrokecolor{currentstroke}%
\pgfsetdash{}{0pt}%
\pgfpathmoveto{\pgfqpoint{12.072565in}{1.416773in}}%
\pgfpathcurveto{\pgfqpoint{12.083615in}{1.416773in}}{\pgfqpoint{12.094214in}{1.421164in}}{\pgfqpoint{12.102027in}{1.428977in}}%
\pgfpathcurveto{\pgfqpoint{12.109841in}{1.436791in}}{\pgfqpoint{12.114231in}{1.447390in}}{\pgfqpoint{12.114231in}{1.458440in}}%
\pgfpathcurveto{\pgfqpoint{12.114231in}{1.469490in}}{\pgfqpoint{12.109841in}{1.480089in}}{\pgfqpoint{12.102027in}{1.487903in}}%
\pgfpathcurveto{\pgfqpoint{12.094214in}{1.495716in}}{\pgfqpoint{12.083615in}{1.500107in}}{\pgfqpoint{12.072565in}{1.500107in}}%
\pgfpathcurveto{\pgfqpoint{12.061514in}{1.500107in}}{\pgfqpoint{12.050915in}{1.495716in}}{\pgfqpoint{12.043102in}{1.487903in}}%
\pgfpathcurveto{\pgfqpoint{12.035288in}{1.480089in}}{\pgfqpoint{12.030898in}{1.469490in}}{\pgfqpoint{12.030898in}{1.458440in}}%
\pgfpathcurveto{\pgfqpoint{12.030898in}{1.447390in}}{\pgfqpoint{12.035288in}{1.436791in}}{\pgfqpoint{12.043102in}{1.428977in}}%
\pgfpathcurveto{\pgfqpoint{12.050915in}{1.421164in}}{\pgfqpoint{12.061514in}{1.416773in}}{\pgfqpoint{12.072565in}{1.416773in}}%
\pgfpathclose%
\pgfusepath{stroke,fill}%
\end{pgfscope}%
\begin{pgfscope}%
\pgfpathrectangle{\pgfqpoint{7.640588in}{0.566125in}}{\pgfqpoint{5.699255in}{2.685432in}}%
\pgfusepath{clip}%
\pgfsetbuttcap%
\pgfsetroundjoin%
\definecolor{currentfill}{rgb}{0.000000,0.000000,0.000000}%
\pgfsetfillcolor{currentfill}%
\pgfsetlinewidth{1.003750pt}%
\definecolor{currentstroke}{rgb}{0.000000,0.000000,0.000000}%
\pgfsetstrokecolor{currentstroke}%
\pgfsetdash{}{0pt}%
\pgfpathmoveto{\pgfqpoint{11.652472in}{1.468994in}}%
\pgfpathcurveto{\pgfqpoint{11.663522in}{1.468994in}}{\pgfqpoint{11.674121in}{1.473384in}}{\pgfqpoint{11.681935in}{1.481198in}}%
\pgfpathcurveto{\pgfqpoint{11.689748in}{1.489011in}}{\pgfqpoint{11.694139in}{1.499610in}}{\pgfqpoint{11.694139in}{1.510660in}}%
\pgfpathcurveto{\pgfqpoint{11.694139in}{1.521711in}}{\pgfqpoint{11.689748in}{1.532310in}}{\pgfqpoint{11.681935in}{1.540123in}}%
\pgfpathcurveto{\pgfqpoint{11.674121in}{1.547937in}}{\pgfqpoint{11.663522in}{1.552327in}}{\pgfqpoint{11.652472in}{1.552327in}}%
\pgfpathcurveto{\pgfqpoint{11.641422in}{1.552327in}}{\pgfqpoint{11.630823in}{1.547937in}}{\pgfqpoint{11.623009in}{1.540123in}}%
\pgfpathcurveto{\pgfqpoint{11.615196in}{1.532310in}}{\pgfqpoint{11.610805in}{1.521711in}}{\pgfqpoint{11.610805in}{1.510660in}}%
\pgfpathcurveto{\pgfqpoint{11.610805in}{1.499610in}}{\pgfqpoint{11.615196in}{1.489011in}}{\pgfqpoint{11.623009in}{1.481198in}}%
\pgfpathcurveto{\pgfqpoint{11.630823in}{1.473384in}}{\pgfqpoint{11.641422in}{1.468994in}}{\pgfqpoint{11.652472in}{1.468994in}}%
\pgfpathclose%
\pgfusepath{stroke,fill}%
\end{pgfscope}%
\begin{pgfscope}%
\pgfpathrectangle{\pgfqpoint{7.640588in}{0.566125in}}{\pgfqpoint{5.699255in}{2.685432in}}%
\pgfusepath{clip}%
\pgfsetbuttcap%
\pgfsetroundjoin%
\definecolor{currentfill}{rgb}{0.000000,0.000000,0.000000}%
\pgfsetfillcolor{currentfill}%
\pgfsetlinewidth{1.003750pt}%
\definecolor{currentstroke}{rgb}{0.000000,0.000000,0.000000}%
\pgfsetstrokecolor{currentstroke}%
\pgfsetdash{}{0pt}%
\pgfpathmoveto{\pgfqpoint{11.400417in}{1.390663in}}%
\pgfpathcurveto{\pgfqpoint{11.411467in}{1.390663in}}{\pgfqpoint{11.422066in}{1.395053in}}{\pgfqpoint{11.429879in}{1.402867in}}%
\pgfpathcurveto{\pgfqpoint{11.437693in}{1.410681in}}{\pgfqpoint{11.442083in}{1.421280in}}{\pgfqpoint{11.442083in}{1.432330in}}%
\pgfpathcurveto{\pgfqpoint{11.442083in}{1.443380in}}{\pgfqpoint{11.437693in}{1.453979in}}{\pgfqpoint{11.429879in}{1.461793in}}%
\pgfpathcurveto{\pgfqpoint{11.422066in}{1.469606in}}{\pgfqpoint{11.411467in}{1.473997in}}{\pgfqpoint{11.400417in}{1.473997in}}%
\pgfpathcurveto{\pgfqpoint{11.389366in}{1.473997in}}{\pgfqpoint{11.378767in}{1.469606in}}{\pgfqpoint{11.370954in}{1.461793in}}%
\pgfpathcurveto{\pgfqpoint{11.363140in}{1.453979in}}{\pgfqpoint{11.358750in}{1.443380in}}{\pgfqpoint{11.358750in}{1.432330in}}%
\pgfpathcurveto{\pgfqpoint{11.358750in}{1.421280in}}{\pgfqpoint{11.363140in}{1.410681in}}{\pgfqpoint{11.370954in}{1.402867in}}%
\pgfpathcurveto{\pgfqpoint{11.378767in}{1.395053in}}{\pgfqpoint{11.389366in}{1.390663in}}{\pgfqpoint{11.400417in}{1.390663in}}%
\pgfpathclose%
\pgfusepath{stroke,fill}%
\end{pgfscope}%
\begin{pgfscope}%
\pgfpathrectangle{\pgfqpoint{7.640588in}{0.566125in}}{\pgfqpoint{5.699255in}{2.685432in}}%
\pgfusepath{clip}%
\pgfsetbuttcap%
\pgfsetroundjoin%
\definecolor{currentfill}{rgb}{0.000000,0.000000,0.000000}%
\pgfsetfillcolor{currentfill}%
\pgfsetlinewidth{1.003750pt}%
\definecolor{currentstroke}{rgb}{0.000000,0.000000,0.000000}%
\pgfsetstrokecolor{currentstroke}%
\pgfsetdash{}{0pt}%
\pgfpathmoveto{\pgfqpoint{11.176367in}{1.468994in}}%
\pgfpathcurveto{\pgfqpoint{11.187417in}{1.468994in}}{\pgfqpoint{11.198016in}{1.473384in}}{\pgfqpoint{11.205830in}{1.481198in}}%
\pgfpathcurveto{\pgfqpoint{11.213644in}{1.489011in}}{\pgfqpoint{11.218034in}{1.499610in}}{\pgfqpoint{11.218034in}{1.510660in}}%
\pgfpathcurveto{\pgfqpoint{11.218034in}{1.521711in}}{\pgfqpoint{11.213644in}{1.532310in}}{\pgfqpoint{11.205830in}{1.540123in}}%
\pgfpathcurveto{\pgfqpoint{11.198016in}{1.547937in}}{\pgfqpoint{11.187417in}{1.552327in}}{\pgfqpoint{11.176367in}{1.552327in}}%
\pgfpathcurveto{\pgfqpoint{11.165317in}{1.552327in}}{\pgfqpoint{11.154718in}{1.547937in}}{\pgfqpoint{11.146904in}{1.540123in}}%
\pgfpathcurveto{\pgfqpoint{11.139091in}{1.532310in}}{\pgfqpoint{11.134701in}{1.521711in}}{\pgfqpoint{11.134701in}{1.510660in}}%
\pgfpathcurveto{\pgfqpoint{11.134701in}{1.499610in}}{\pgfqpoint{11.139091in}{1.489011in}}{\pgfqpoint{11.146904in}{1.481198in}}%
\pgfpathcurveto{\pgfqpoint{11.154718in}{1.473384in}}{\pgfqpoint{11.165317in}{1.468994in}}{\pgfqpoint{11.176367in}{1.468994in}}%
\pgfpathclose%
\pgfusepath{stroke,fill}%
\end{pgfscope}%
\begin{pgfscope}%
\pgfpathrectangle{\pgfqpoint{7.640588in}{0.566125in}}{\pgfqpoint{5.699255in}{2.685432in}}%
\pgfusepath{clip}%
\pgfsetbuttcap%
\pgfsetroundjoin%
\definecolor{currentfill}{rgb}{0.000000,0.000000,0.000000}%
\pgfsetfillcolor{currentfill}%
\pgfsetlinewidth{1.003750pt}%
\definecolor{currentstroke}{rgb}{0.000000,0.000000,0.000000}%
\pgfsetstrokecolor{currentstroke}%
\pgfsetdash{}{0pt}%
\pgfpathmoveto{\pgfqpoint{11.232380in}{1.677875in}}%
\pgfpathcurveto{\pgfqpoint{11.243430in}{1.677875in}}{\pgfqpoint{11.254029in}{1.682265in}}{\pgfqpoint{11.261842in}{1.690079in}}%
\pgfpathcurveto{\pgfqpoint{11.269656in}{1.697893in}}{\pgfqpoint{11.274046in}{1.708492in}}{\pgfqpoint{11.274046in}{1.719542in}}%
\pgfpathcurveto{\pgfqpoint{11.274046in}{1.730592in}}{\pgfqpoint{11.269656in}{1.741191in}}{\pgfqpoint{11.261842in}{1.749005in}}%
\pgfpathcurveto{\pgfqpoint{11.254029in}{1.756818in}}{\pgfqpoint{11.243430in}{1.761209in}}{\pgfqpoint{11.232380in}{1.761209in}}%
\pgfpathcurveto{\pgfqpoint{11.221329in}{1.761209in}}{\pgfqpoint{11.210730in}{1.756818in}}{\pgfqpoint{11.202917in}{1.749005in}}%
\pgfpathcurveto{\pgfqpoint{11.195103in}{1.741191in}}{\pgfqpoint{11.190713in}{1.730592in}}{\pgfqpoint{11.190713in}{1.719542in}}%
\pgfpathcurveto{\pgfqpoint{11.190713in}{1.708492in}}{\pgfqpoint{11.195103in}{1.697893in}}{\pgfqpoint{11.202917in}{1.690079in}}%
\pgfpathcurveto{\pgfqpoint{11.210730in}{1.682265in}}{\pgfqpoint{11.221329in}{1.677875in}}{\pgfqpoint{11.232380in}{1.677875in}}%
\pgfpathclose%
\pgfusepath{stroke,fill}%
\end{pgfscope}%
\begin{pgfscope}%
\pgfpathrectangle{\pgfqpoint{7.640588in}{0.566125in}}{\pgfqpoint{5.699255in}{2.685432in}}%
\pgfusepath{clip}%
\pgfsetbuttcap%
\pgfsetroundjoin%
\definecolor{currentfill}{rgb}{0.000000,0.000000,0.000000}%
\pgfsetfillcolor{currentfill}%
\pgfsetlinewidth{1.003750pt}%
\definecolor{currentstroke}{rgb}{0.000000,0.000000,0.000000}%
\pgfsetstrokecolor{currentstroke}%
\pgfsetdash{}{0pt}%
\pgfpathmoveto{\pgfqpoint{10.812287in}{1.521214in}}%
\pgfpathcurveto{\pgfqpoint{10.823337in}{1.521214in}}{\pgfqpoint{10.833936in}{1.525604in}}{\pgfqpoint{10.841750in}{1.533418in}}%
\pgfpathcurveto{\pgfqpoint{10.849563in}{1.541232in}}{\pgfqpoint{10.853954in}{1.551831in}}{\pgfqpoint{10.853954in}{1.562881in}}%
\pgfpathcurveto{\pgfqpoint{10.853954in}{1.573931in}}{\pgfqpoint{10.849563in}{1.584530in}}{\pgfqpoint{10.841750in}{1.592344in}}%
\pgfpathcurveto{\pgfqpoint{10.833936in}{1.600157in}}{\pgfqpoint{10.823337in}{1.604547in}}{\pgfqpoint{10.812287in}{1.604547in}}%
\pgfpathcurveto{\pgfqpoint{10.801237in}{1.604547in}}{\pgfqpoint{10.790638in}{1.600157in}}{\pgfqpoint{10.782824in}{1.592344in}}%
\pgfpathcurveto{\pgfqpoint{10.775011in}{1.584530in}}{\pgfqpoint{10.770620in}{1.573931in}}{\pgfqpoint{10.770620in}{1.562881in}}%
\pgfpathcurveto{\pgfqpoint{10.770620in}{1.551831in}}{\pgfqpoint{10.775011in}{1.541232in}}{\pgfqpoint{10.782824in}{1.533418in}}%
\pgfpathcurveto{\pgfqpoint{10.790638in}{1.525604in}}{\pgfqpoint{10.801237in}{1.521214in}}{\pgfqpoint{10.812287in}{1.521214in}}%
\pgfpathclose%
\pgfusepath{stroke,fill}%
\end{pgfscope}%
\begin{pgfscope}%
\pgfpathrectangle{\pgfqpoint{7.640588in}{0.566125in}}{\pgfqpoint{5.699255in}{2.685432in}}%
\pgfusepath{clip}%
\pgfsetbuttcap%
\pgfsetroundjoin%
\definecolor{currentfill}{rgb}{0.000000,0.000000,0.000000}%
\pgfsetfillcolor{currentfill}%
\pgfsetlinewidth{1.003750pt}%
\definecolor{currentstroke}{rgb}{0.000000,0.000000,0.000000}%
\pgfsetstrokecolor{currentstroke}%
\pgfsetdash{}{0pt}%
\pgfpathmoveto{\pgfqpoint{10.728268in}{1.508159in}}%
\pgfpathcurveto{\pgfqpoint{10.739319in}{1.508159in}}{\pgfqpoint{10.749918in}{1.512549in}}{\pgfqpoint{10.757731in}{1.520363in}}%
\pgfpathcurveto{\pgfqpoint{10.765545in}{1.528177in}}{\pgfqpoint{10.769935in}{1.538776in}}{\pgfqpoint{10.769935in}{1.549826in}}%
\pgfpathcurveto{\pgfqpoint{10.769935in}{1.560876in}}{\pgfqpoint{10.765545in}{1.571475in}}{\pgfqpoint{10.757731in}{1.579288in}}%
\pgfpathcurveto{\pgfqpoint{10.749918in}{1.587102in}}{\pgfqpoint{10.739319in}{1.591492in}}{\pgfqpoint{10.728268in}{1.591492in}}%
\pgfpathcurveto{\pgfqpoint{10.717218in}{1.591492in}}{\pgfqpoint{10.706619in}{1.587102in}}{\pgfqpoint{10.698806in}{1.579288in}}%
\pgfpathcurveto{\pgfqpoint{10.690992in}{1.571475in}}{\pgfqpoint{10.686602in}{1.560876in}}{\pgfqpoint{10.686602in}{1.549826in}}%
\pgfpathcurveto{\pgfqpoint{10.686602in}{1.538776in}}{\pgfqpoint{10.690992in}{1.528177in}}{\pgfqpoint{10.698806in}{1.520363in}}%
\pgfpathcurveto{\pgfqpoint{10.706619in}{1.512549in}}{\pgfqpoint{10.717218in}{1.508159in}}{\pgfqpoint{10.728268in}{1.508159in}}%
\pgfpathclose%
\pgfusepath{stroke,fill}%
\end{pgfscope}%
\begin{pgfscope}%
\pgfpathrectangle{\pgfqpoint{7.640588in}{0.566125in}}{\pgfqpoint{5.699255in}{2.685432in}}%
\pgfusepath{clip}%
\pgfsetbuttcap%
\pgfsetroundjoin%
\definecolor{currentfill}{rgb}{0.000000,0.000000,0.000000}%
\pgfsetfillcolor{currentfill}%
\pgfsetlinewidth{1.003750pt}%
\definecolor{currentstroke}{rgb}{0.000000,0.000000,0.000000}%
\pgfsetstrokecolor{currentstroke}%
\pgfsetdash{}{0pt}%
\pgfpathmoveto{\pgfqpoint{10.476213in}{1.560379in}}%
\pgfpathcurveto{\pgfqpoint{10.487263in}{1.560379in}}{\pgfqpoint{10.497862in}{1.564770in}}{\pgfqpoint{10.505676in}{1.572583in}}%
\pgfpathcurveto{\pgfqpoint{10.513489in}{1.580397in}}{\pgfqpoint{10.517880in}{1.590996in}}{\pgfqpoint{10.517880in}{1.602046in}}%
\pgfpathcurveto{\pgfqpoint{10.517880in}{1.613096in}}{\pgfqpoint{10.513489in}{1.623695in}}{\pgfqpoint{10.505676in}{1.631509in}}%
\pgfpathcurveto{\pgfqpoint{10.497862in}{1.639322in}}{\pgfqpoint{10.487263in}{1.643713in}}{\pgfqpoint{10.476213in}{1.643713in}}%
\pgfpathcurveto{\pgfqpoint{10.465163in}{1.643713in}}{\pgfqpoint{10.454564in}{1.639322in}}{\pgfqpoint{10.446750in}{1.631509in}}%
\pgfpathcurveto{\pgfqpoint{10.438937in}{1.623695in}}{\pgfqpoint{10.434546in}{1.613096in}}{\pgfqpoint{10.434546in}{1.602046in}}%
\pgfpathcurveto{\pgfqpoint{10.434546in}{1.590996in}}{\pgfqpoint{10.438937in}{1.580397in}}{\pgfqpoint{10.446750in}{1.572583in}}%
\pgfpathcurveto{\pgfqpoint{10.454564in}{1.564770in}}{\pgfqpoint{10.465163in}{1.560379in}}{\pgfqpoint{10.476213in}{1.560379in}}%
\pgfpathclose%
\pgfusepath{stroke,fill}%
\end{pgfscope}%
\begin{pgfscope}%
\pgfpathrectangle{\pgfqpoint{7.640588in}{0.566125in}}{\pgfqpoint{5.699255in}{2.685432in}}%
\pgfusepath{clip}%
\pgfsetbuttcap%
\pgfsetroundjoin%
\definecolor{currentfill}{rgb}{0.000000,0.000000,0.000000}%
\pgfsetfillcolor{currentfill}%
\pgfsetlinewidth{1.003750pt}%
\definecolor{currentstroke}{rgb}{0.000000,0.000000,0.000000}%
\pgfsetstrokecolor{currentstroke}%
\pgfsetdash{}{0pt}%
\pgfpathmoveto{\pgfqpoint{10.476213in}{1.821481in}}%
\pgfpathcurveto{\pgfqpoint{10.487263in}{1.821481in}}{\pgfqpoint{10.497862in}{1.825871in}}{\pgfqpoint{10.505676in}{1.833685in}}%
\pgfpathcurveto{\pgfqpoint{10.513489in}{1.841499in}}{\pgfqpoint{10.517880in}{1.852098in}}{\pgfqpoint{10.517880in}{1.863148in}}%
\pgfpathcurveto{\pgfqpoint{10.517880in}{1.874198in}}{\pgfqpoint{10.513489in}{1.884797in}}{\pgfqpoint{10.505676in}{1.892611in}}%
\pgfpathcurveto{\pgfqpoint{10.497862in}{1.900424in}}{\pgfqpoint{10.487263in}{1.904815in}}{\pgfqpoint{10.476213in}{1.904815in}}%
\pgfpathcurveto{\pgfqpoint{10.465163in}{1.904815in}}{\pgfqpoint{10.454564in}{1.900424in}}{\pgfqpoint{10.446750in}{1.892611in}}%
\pgfpathcurveto{\pgfqpoint{10.438937in}{1.884797in}}{\pgfqpoint{10.434546in}{1.874198in}}{\pgfqpoint{10.434546in}{1.863148in}}%
\pgfpathcurveto{\pgfqpoint{10.434546in}{1.852098in}}{\pgfqpoint{10.438937in}{1.841499in}}{\pgfqpoint{10.446750in}{1.833685in}}%
\pgfpathcurveto{\pgfqpoint{10.454564in}{1.825871in}}{\pgfqpoint{10.465163in}{1.821481in}}{\pgfqpoint{10.476213in}{1.821481in}}%
\pgfpathclose%
\pgfusepath{stroke,fill}%
\end{pgfscope}%
\begin{pgfscope}%
\pgfpathrectangle{\pgfqpoint{7.640588in}{0.566125in}}{\pgfqpoint{5.699255in}{2.685432in}}%
\pgfusepath{clip}%
\pgfsetbuttcap%
\pgfsetroundjoin%
\definecolor{currentfill}{rgb}{0.000000,0.000000,0.000000}%
\pgfsetfillcolor{currentfill}%
\pgfsetlinewidth{1.003750pt}%
\definecolor{currentstroke}{rgb}{0.000000,0.000000,0.000000}%
\pgfsetstrokecolor{currentstroke}%
\pgfsetdash{}{0pt}%
\pgfpathmoveto{\pgfqpoint{9.860077in}{1.873702in}}%
\pgfpathcurveto{\pgfqpoint{9.871127in}{1.873702in}}{\pgfqpoint{9.881726in}{1.878092in}}{\pgfqpoint{9.889540in}{1.885905in}}%
\pgfpathcurveto{\pgfqpoint{9.897354in}{1.893719in}}{\pgfqpoint{9.901744in}{1.904318in}}{\pgfqpoint{9.901744in}{1.915368in}}%
\pgfpathcurveto{\pgfqpoint{9.901744in}{1.926418in}}{\pgfqpoint{9.897354in}{1.937017in}}{\pgfqpoint{9.889540in}{1.944831in}}%
\pgfpathcurveto{\pgfqpoint{9.881726in}{1.952645in}}{\pgfqpoint{9.871127in}{1.957035in}}{\pgfqpoint{9.860077in}{1.957035in}}%
\pgfpathcurveto{\pgfqpoint{9.849027in}{1.957035in}}{\pgfqpoint{9.838428in}{1.952645in}}{\pgfqpoint{9.830614in}{1.944831in}}%
\pgfpathcurveto{\pgfqpoint{9.822801in}{1.937017in}}{\pgfqpoint{9.818411in}{1.926418in}}{\pgfqpoint{9.818411in}{1.915368in}}%
\pgfpathcurveto{\pgfqpoint{9.818411in}{1.904318in}}{\pgfqpoint{9.822801in}{1.893719in}}{\pgfqpoint{9.830614in}{1.885905in}}%
\pgfpathcurveto{\pgfqpoint{9.838428in}{1.878092in}}{\pgfqpoint{9.849027in}{1.873702in}}{\pgfqpoint{9.860077in}{1.873702in}}%
\pgfpathclose%
\pgfusepath{stroke,fill}%
\end{pgfscope}%
\begin{pgfscope}%
\pgfpathrectangle{\pgfqpoint{7.640588in}{0.566125in}}{\pgfqpoint{5.699255in}{2.685432in}}%
\pgfusepath{clip}%
\pgfsetbuttcap%
\pgfsetroundjoin%
\definecolor{currentfill}{rgb}{0.000000,0.000000,0.000000}%
\pgfsetfillcolor{currentfill}%
\pgfsetlinewidth{1.003750pt}%
\definecolor{currentstroke}{rgb}{0.000000,0.000000,0.000000}%
\pgfsetstrokecolor{currentstroke}%
\pgfsetdash{}{0pt}%
\pgfpathmoveto{\pgfqpoint{9.580016in}{1.743151in}}%
\pgfpathcurveto{\pgfqpoint{9.591066in}{1.743151in}}{\pgfqpoint{9.601665in}{1.747541in}}{\pgfqpoint{9.609478in}{1.755355in}}%
\pgfpathcurveto{\pgfqpoint{9.617292in}{1.763168in}}{\pgfqpoint{9.621682in}{1.773767in}}{\pgfqpoint{9.621682in}{1.784817in}}%
\pgfpathcurveto{\pgfqpoint{9.621682in}{1.795867in}}{\pgfqpoint{9.617292in}{1.806466in}}{\pgfqpoint{9.609478in}{1.814280in}}%
\pgfpathcurveto{\pgfqpoint{9.601665in}{1.822094in}}{\pgfqpoint{9.591066in}{1.826484in}}{\pgfqpoint{9.580016in}{1.826484in}}%
\pgfpathcurveto{\pgfqpoint{9.568965in}{1.826484in}}{\pgfqpoint{9.558366in}{1.822094in}}{\pgfqpoint{9.550553in}{1.814280in}}%
\pgfpathcurveto{\pgfqpoint{9.542739in}{1.806466in}}{\pgfqpoint{9.538349in}{1.795867in}}{\pgfqpoint{9.538349in}{1.784817in}}%
\pgfpathcurveto{\pgfqpoint{9.538349in}{1.773767in}}{\pgfqpoint{9.542739in}{1.763168in}}{\pgfqpoint{9.550553in}{1.755355in}}%
\pgfpathcurveto{\pgfqpoint{9.558366in}{1.747541in}}{\pgfqpoint{9.568965in}{1.743151in}}{\pgfqpoint{9.580016in}{1.743151in}}%
\pgfpathclose%
\pgfusepath{stroke,fill}%
\end{pgfscope}%
\begin{pgfscope}%
\pgfpathrectangle{\pgfqpoint{7.640588in}{0.566125in}}{\pgfqpoint{5.699255in}{2.685432in}}%
\pgfusepath{clip}%
\pgfsetbuttcap%
\pgfsetroundjoin%
\definecolor{currentfill}{rgb}{0.000000,0.000000,0.000000}%
\pgfsetfillcolor{currentfill}%
\pgfsetlinewidth{1.003750pt}%
\definecolor{currentstroke}{rgb}{0.000000,0.000000,0.000000}%
\pgfsetstrokecolor{currentstroke}%
\pgfsetdash{}{0pt}%
\pgfpathmoveto{\pgfqpoint{9.608022in}{1.808426in}}%
\pgfpathcurveto{\pgfqpoint{9.619072in}{1.808426in}}{\pgfqpoint{9.629671in}{1.812816in}}{\pgfqpoint{9.637485in}{1.820630in}}%
\pgfpathcurveto{\pgfqpoint{9.645298in}{1.828444in}}{\pgfqpoint{9.649688in}{1.839043in}}{\pgfqpoint{9.649688in}{1.850093in}}%
\pgfpathcurveto{\pgfqpoint{9.649688in}{1.861143in}}{\pgfqpoint{9.645298in}{1.871742in}}{\pgfqpoint{9.637485in}{1.879556in}}%
\pgfpathcurveto{\pgfqpoint{9.629671in}{1.887369in}}{\pgfqpoint{9.619072in}{1.891759in}}{\pgfqpoint{9.608022in}{1.891759in}}%
\pgfpathcurveto{\pgfqpoint{9.596972in}{1.891759in}}{\pgfqpoint{9.586373in}{1.887369in}}{\pgfqpoint{9.578559in}{1.879556in}}%
\pgfpathcurveto{\pgfqpoint{9.570745in}{1.871742in}}{\pgfqpoint{9.566355in}{1.861143in}}{\pgfqpoint{9.566355in}{1.850093in}}%
\pgfpathcurveto{\pgfqpoint{9.566355in}{1.839043in}}{\pgfqpoint{9.570745in}{1.828444in}}{\pgfqpoint{9.578559in}{1.820630in}}%
\pgfpathcurveto{\pgfqpoint{9.586373in}{1.812816in}}{\pgfqpoint{9.596972in}{1.808426in}}{\pgfqpoint{9.608022in}{1.808426in}}%
\pgfpathclose%
\pgfusepath{stroke,fill}%
\end{pgfscope}%
\begin{pgfscope}%
\pgfpathrectangle{\pgfqpoint{7.640588in}{0.566125in}}{\pgfqpoint{5.699255in}{2.685432in}}%
\pgfusepath{clip}%
\pgfsetbuttcap%
\pgfsetroundjoin%
\definecolor{currentfill}{rgb}{0.000000,0.000000,0.000000}%
\pgfsetfillcolor{currentfill}%
\pgfsetlinewidth{1.003750pt}%
\definecolor{currentstroke}{rgb}{0.000000,0.000000,0.000000}%
\pgfsetstrokecolor{currentstroke}%
\pgfsetdash{}{0pt}%
\pgfpathmoveto{\pgfqpoint{10.112133in}{2.213134in}}%
\pgfpathcurveto{\pgfqpoint{10.123183in}{2.213134in}}{\pgfqpoint{10.133782in}{2.217524in}}{\pgfqpoint{10.141596in}{2.225338in}}%
\pgfpathcurveto{\pgfqpoint{10.149409in}{2.233151in}}{\pgfqpoint{10.153799in}{2.243750in}}{\pgfqpoint{10.153799in}{2.254801in}}%
\pgfpathcurveto{\pgfqpoint{10.153799in}{2.265851in}}{\pgfqpoint{10.149409in}{2.276450in}}{\pgfqpoint{10.141596in}{2.284263in}}%
\pgfpathcurveto{\pgfqpoint{10.133782in}{2.292077in}}{\pgfqpoint{10.123183in}{2.296467in}}{\pgfqpoint{10.112133in}{2.296467in}}%
\pgfpathcurveto{\pgfqpoint{10.101083in}{2.296467in}}{\pgfqpoint{10.090484in}{2.292077in}}{\pgfqpoint{10.082670in}{2.284263in}}%
\pgfpathcurveto{\pgfqpoint{10.074856in}{2.276450in}}{\pgfqpoint{10.070466in}{2.265851in}}{\pgfqpoint{10.070466in}{2.254801in}}%
\pgfpathcurveto{\pgfqpoint{10.070466in}{2.243750in}}{\pgfqpoint{10.074856in}{2.233151in}}{\pgfqpoint{10.082670in}{2.225338in}}%
\pgfpathcurveto{\pgfqpoint{10.090484in}{2.217524in}}{\pgfqpoint{10.101083in}{2.213134in}}{\pgfqpoint{10.112133in}{2.213134in}}%
\pgfpathclose%
\pgfusepath{stroke,fill}%
\end{pgfscope}%
\begin{pgfscope}%
\pgfpathrectangle{\pgfqpoint{7.640588in}{0.566125in}}{\pgfqpoint{5.699255in}{2.685432in}}%
\pgfusepath{clip}%
\pgfsetbuttcap%
\pgfsetroundjoin%
\definecolor{currentfill}{rgb}{0.000000,0.000000,0.000000}%
\pgfsetfillcolor{currentfill}%
\pgfsetlinewidth{1.003750pt}%
\definecolor{currentstroke}{rgb}{0.000000,0.000000,0.000000}%
\pgfsetstrokecolor{currentstroke}%
\pgfsetdash{}{0pt}%
\pgfpathmoveto{\pgfqpoint{10.084127in}{2.343685in}}%
\pgfpathcurveto{\pgfqpoint{10.095177in}{2.343685in}}{\pgfqpoint{10.105776in}{2.348075in}}{\pgfqpoint{10.113589in}{2.355889in}}%
\pgfpathcurveto{\pgfqpoint{10.121403in}{2.363702in}}{\pgfqpoint{10.125793in}{2.374301in}}{\pgfqpoint{10.125793in}{2.385351in}}%
\pgfpathcurveto{\pgfqpoint{10.125793in}{2.396402in}}{\pgfqpoint{10.121403in}{2.407001in}}{\pgfqpoint{10.113589in}{2.414814in}}%
\pgfpathcurveto{\pgfqpoint{10.105776in}{2.422628in}}{\pgfqpoint{10.095177in}{2.427018in}}{\pgfqpoint{10.084127in}{2.427018in}}%
\pgfpathcurveto{\pgfqpoint{10.073076in}{2.427018in}}{\pgfqpoint{10.062477in}{2.422628in}}{\pgfqpoint{10.054664in}{2.414814in}}%
\pgfpathcurveto{\pgfqpoint{10.046850in}{2.407001in}}{\pgfqpoint{10.042460in}{2.396402in}}{\pgfqpoint{10.042460in}{2.385351in}}%
\pgfpathcurveto{\pgfqpoint{10.042460in}{2.374301in}}{\pgfqpoint{10.046850in}{2.363702in}}{\pgfqpoint{10.054664in}{2.355889in}}%
\pgfpathcurveto{\pgfqpoint{10.062477in}{2.348075in}}{\pgfqpoint{10.073076in}{2.343685in}}{\pgfqpoint{10.084127in}{2.343685in}}%
\pgfpathclose%
\pgfusepath{stroke,fill}%
\end{pgfscope}%
\begin{pgfscope}%
\pgfpathrectangle{\pgfqpoint{7.640588in}{0.566125in}}{\pgfqpoint{5.699255in}{2.685432in}}%
\pgfusepath{clip}%
\pgfsetbuttcap%
\pgfsetroundjoin%
\definecolor{currentfill}{rgb}{0.000000,0.000000,0.000000}%
\pgfsetfillcolor{currentfill}%
\pgfsetlinewidth{1.003750pt}%
\definecolor{currentstroke}{rgb}{0.000000,0.000000,0.000000}%
\pgfsetstrokecolor{currentstroke}%
\pgfsetdash{}{0pt}%
\pgfpathmoveto{\pgfqpoint{10.616244in}{2.408960in}}%
\pgfpathcurveto{\pgfqpoint{10.627294in}{2.408960in}}{\pgfqpoint{10.637893in}{2.413351in}}{\pgfqpoint{10.645707in}{2.421164in}}%
\pgfpathcurveto{\pgfqpoint{10.653520in}{2.428978in}}{\pgfqpoint{10.657910in}{2.439577in}}{\pgfqpoint{10.657910in}{2.450627in}}%
\pgfpathcurveto{\pgfqpoint{10.657910in}{2.461677in}}{\pgfqpoint{10.653520in}{2.472276in}}{\pgfqpoint{10.645707in}{2.480090in}}%
\pgfpathcurveto{\pgfqpoint{10.637893in}{2.487903in}}{\pgfqpoint{10.627294in}{2.492294in}}{\pgfqpoint{10.616244in}{2.492294in}}%
\pgfpathcurveto{\pgfqpoint{10.605194in}{2.492294in}}{\pgfqpoint{10.594595in}{2.487903in}}{\pgfqpoint{10.586781in}{2.480090in}}%
\pgfpathcurveto{\pgfqpoint{10.578967in}{2.472276in}}{\pgfqpoint{10.574577in}{2.461677in}}{\pgfqpoint{10.574577in}{2.450627in}}%
\pgfpathcurveto{\pgfqpoint{10.574577in}{2.439577in}}{\pgfqpoint{10.578967in}{2.428978in}}{\pgfqpoint{10.586781in}{2.421164in}}%
\pgfpathcurveto{\pgfqpoint{10.594595in}{2.413351in}}{\pgfqpoint{10.605194in}{2.408960in}}{\pgfqpoint{10.616244in}{2.408960in}}%
\pgfpathclose%
\pgfusepath{stroke,fill}%
\end{pgfscope}%
\begin{pgfscope}%
\pgfpathrectangle{\pgfqpoint{7.640588in}{0.566125in}}{\pgfqpoint{5.699255in}{2.685432in}}%
\pgfusepath{clip}%
\pgfsetbuttcap%
\pgfsetroundjoin%
\definecolor{currentfill}{rgb}{0.000000,0.000000,0.000000}%
\pgfsetfillcolor{currentfill}%
\pgfsetlinewidth{1.003750pt}%
\definecolor{currentstroke}{rgb}{0.000000,0.000000,0.000000}%
\pgfsetstrokecolor{currentstroke}%
\pgfsetdash{}{0pt}%
\pgfpathmoveto{\pgfqpoint{10.896306in}{2.317575in}}%
\pgfpathcurveto{\pgfqpoint{10.907356in}{2.317575in}}{\pgfqpoint{10.917955in}{2.321965in}}{\pgfqpoint{10.925768in}{2.329779in}}%
\pgfpathcurveto{\pgfqpoint{10.933582in}{2.337592in}}{\pgfqpoint{10.937972in}{2.348191in}}{\pgfqpoint{10.937972in}{2.359241in}}%
\pgfpathcurveto{\pgfqpoint{10.937972in}{2.370291in}}{\pgfqpoint{10.933582in}{2.380890in}}{\pgfqpoint{10.925768in}{2.388704in}}%
\pgfpathcurveto{\pgfqpoint{10.917955in}{2.396518in}}{\pgfqpoint{10.907356in}{2.400908in}}{\pgfqpoint{10.896306in}{2.400908in}}%
\pgfpathcurveto{\pgfqpoint{10.885255in}{2.400908in}}{\pgfqpoint{10.874656in}{2.396518in}}{\pgfqpoint{10.866843in}{2.388704in}}%
\pgfpathcurveto{\pgfqpoint{10.859029in}{2.380890in}}{\pgfqpoint{10.854639in}{2.370291in}}{\pgfqpoint{10.854639in}{2.359241in}}%
\pgfpathcurveto{\pgfqpoint{10.854639in}{2.348191in}}{\pgfqpoint{10.859029in}{2.337592in}}{\pgfqpoint{10.866843in}{2.329779in}}%
\pgfpathcurveto{\pgfqpoint{10.874656in}{2.321965in}}{\pgfqpoint{10.885255in}{2.317575in}}{\pgfqpoint{10.896306in}{2.317575in}}%
\pgfpathclose%
\pgfusepath{stroke,fill}%
\end{pgfscope}%
\begin{pgfscope}%
\pgfpathrectangle{\pgfqpoint{7.640588in}{0.566125in}}{\pgfqpoint{5.699255in}{2.685432in}}%
\pgfusepath{clip}%
\pgfsetbuttcap%
\pgfsetroundjoin%
\definecolor{currentfill}{rgb}{0.000000,0.000000,0.000000}%
\pgfsetfillcolor{currentfill}%
\pgfsetlinewidth{1.003750pt}%
\definecolor{currentstroke}{rgb}{0.000000,0.000000,0.000000}%
\pgfsetstrokecolor{currentstroke}%
\pgfsetdash{}{0pt}%
\pgfpathmoveto{\pgfqpoint{10.980324in}{2.252299in}}%
\pgfpathcurveto{\pgfqpoint{10.991374in}{2.252299in}}{\pgfqpoint{11.001973in}{2.256689in}}{\pgfqpoint{11.009787in}{2.264503in}}%
\pgfpathcurveto{\pgfqpoint{11.017600in}{2.272317in}}{\pgfqpoint{11.021991in}{2.282916in}}{\pgfqpoint{11.021991in}{2.293966in}}%
\pgfpathcurveto{\pgfqpoint{11.021991in}{2.305016in}}{\pgfqpoint{11.017600in}{2.315615in}}{\pgfqpoint{11.009787in}{2.323429in}}%
\pgfpathcurveto{\pgfqpoint{11.001973in}{2.331242in}}{\pgfqpoint{10.991374in}{2.335632in}}{\pgfqpoint{10.980324in}{2.335632in}}%
\pgfpathcurveto{\pgfqpoint{10.969274in}{2.335632in}}{\pgfqpoint{10.958675in}{2.331242in}}{\pgfqpoint{10.950861in}{2.323429in}}%
\pgfpathcurveto{\pgfqpoint{10.943048in}{2.315615in}}{\pgfqpoint{10.938657in}{2.305016in}}{\pgfqpoint{10.938657in}{2.293966in}}%
\pgfpathcurveto{\pgfqpoint{10.938657in}{2.282916in}}{\pgfqpoint{10.943048in}{2.272317in}}{\pgfqpoint{10.950861in}{2.264503in}}%
\pgfpathcurveto{\pgfqpoint{10.958675in}{2.256689in}}{\pgfqpoint{10.969274in}{2.252299in}}{\pgfqpoint{10.980324in}{2.252299in}}%
\pgfpathclose%
\pgfusepath{stroke,fill}%
\end{pgfscope}%
\begin{pgfscope}%
\pgfpathrectangle{\pgfqpoint{7.640588in}{0.566125in}}{\pgfqpoint{5.699255in}{2.685432in}}%
\pgfusepath{clip}%
\pgfsetbuttcap%
\pgfsetroundjoin%
\definecolor{currentfill}{rgb}{0.000000,0.000000,0.000000}%
\pgfsetfillcolor{currentfill}%
\pgfsetlinewidth{1.003750pt}%
\definecolor{currentstroke}{rgb}{0.000000,0.000000,0.000000}%
\pgfsetstrokecolor{currentstroke}%
\pgfsetdash{}{0pt}%
\pgfpathmoveto{\pgfqpoint{12.100571in}{2.200079in}}%
\pgfpathcurveto{\pgfqpoint{12.111621in}{2.200079in}}{\pgfqpoint{12.122220in}{2.204469in}}{\pgfqpoint{12.130034in}{2.212283in}}%
\pgfpathcurveto{\pgfqpoint{12.137847in}{2.220096in}}{\pgfqpoint{12.142237in}{2.230695in}}{\pgfqpoint{12.142237in}{2.241745in}}%
\pgfpathcurveto{\pgfqpoint{12.142237in}{2.252796in}}{\pgfqpoint{12.137847in}{2.263395in}}{\pgfqpoint{12.130034in}{2.271208in}}%
\pgfpathcurveto{\pgfqpoint{12.122220in}{2.279022in}}{\pgfqpoint{12.111621in}{2.283412in}}{\pgfqpoint{12.100571in}{2.283412in}}%
\pgfpathcurveto{\pgfqpoint{12.089521in}{2.283412in}}{\pgfqpoint{12.078922in}{2.279022in}}{\pgfqpoint{12.071108in}{2.271208in}}%
\pgfpathcurveto{\pgfqpoint{12.063294in}{2.263395in}}{\pgfqpoint{12.058904in}{2.252796in}}{\pgfqpoint{12.058904in}{2.241745in}}%
\pgfpathcurveto{\pgfqpoint{12.058904in}{2.230695in}}{\pgfqpoint{12.063294in}{2.220096in}}{\pgfqpoint{12.071108in}{2.212283in}}%
\pgfpathcurveto{\pgfqpoint{12.078922in}{2.204469in}}{\pgfqpoint{12.089521in}{2.200079in}}{\pgfqpoint{12.100571in}{2.200079in}}%
\pgfpathclose%
\pgfusepath{stroke,fill}%
\end{pgfscope}%
\begin{pgfscope}%
\pgfpathrectangle{\pgfqpoint{7.640588in}{0.566125in}}{\pgfqpoint{5.699255in}{2.685432in}}%
\pgfusepath{clip}%
\pgfsetbuttcap%
\pgfsetroundjoin%
\definecolor{currentfill}{rgb}{0.000000,0.000000,0.000000}%
\pgfsetfillcolor{currentfill}%
\pgfsetlinewidth{1.003750pt}%
\definecolor{currentstroke}{rgb}{0.000000,0.000000,0.000000}%
\pgfsetstrokecolor{currentstroke}%
\pgfsetdash{}{0pt}%
\pgfpathmoveto{\pgfqpoint{11.988546in}{2.291464in}}%
\pgfpathcurveto{\pgfqpoint{11.999596in}{2.291464in}}{\pgfqpoint{12.010195in}{2.295855in}}{\pgfqpoint{12.018009in}{2.303668in}}%
\pgfpathcurveto{\pgfqpoint{12.025823in}{2.311482in}}{\pgfqpoint{12.030213in}{2.322081in}}{\pgfqpoint{12.030213in}{2.333131in}}%
\pgfpathcurveto{\pgfqpoint{12.030213in}{2.344181in}}{\pgfqpoint{12.025823in}{2.354780in}}{\pgfqpoint{12.018009in}{2.362594in}}%
\pgfpathcurveto{\pgfqpoint{12.010195in}{2.370408in}}{\pgfqpoint{11.999596in}{2.374798in}}{\pgfqpoint{11.988546in}{2.374798in}}%
\pgfpathcurveto{\pgfqpoint{11.977496in}{2.374798in}}{\pgfqpoint{11.966897in}{2.370408in}}{\pgfqpoint{11.959083in}{2.362594in}}%
\pgfpathcurveto{\pgfqpoint{11.951270in}{2.354780in}}{\pgfqpoint{11.946879in}{2.344181in}}{\pgfqpoint{11.946879in}{2.333131in}}%
\pgfpathcurveto{\pgfqpoint{11.946879in}{2.322081in}}{\pgfqpoint{11.951270in}{2.311482in}}{\pgfqpoint{11.959083in}{2.303668in}}%
\pgfpathcurveto{\pgfqpoint{11.966897in}{2.295855in}}{\pgfqpoint{11.977496in}{2.291464in}}{\pgfqpoint{11.988546in}{2.291464in}}%
\pgfpathclose%
\pgfusepath{stroke,fill}%
\end{pgfscope}%
\begin{pgfscope}%
\pgfpathrectangle{\pgfqpoint{7.640588in}{0.566125in}}{\pgfqpoint{5.699255in}{2.685432in}}%
\pgfusepath{clip}%
\pgfsetbuttcap%
\pgfsetroundjoin%
\definecolor{currentfill}{rgb}{0.000000,0.000000,0.000000}%
\pgfsetfillcolor{currentfill}%
\pgfsetlinewidth{1.003750pt}%
\definecolor{currentstroke}{rgb}{0.000000,0.000000,0.000000}%
\pgfsetstrokecolor{currentstroke}%
\pgfsetdash{}{0pt}%
\pgfpathmoveto{\pgfqpoint{12.156583in}{2.422015in}}%
\pgfpathcurveto{\pgfqpoint{12.167633in}{2.422015in}}{\pgfqpoint{12.178232in}{2.426406in}}{\pgfqpoint{12.186046in}{2.434219in}}%
\pgfpathcurveto{\pgfqpoint{12.193860in}{2.442033in}}{\pgfqpoint{12.198250in}{2.452632in}}{\pgfqpoint{12.198250in}{2.463682in}}%
\pgfpathcurveto{\pgfqpoint{12.198250in}{2.474732in}}{\pgfqpoint{12.193860in}{2.485331in}}{\pgfqpoint{12.186046in}{2.493145in}}%
\pgfpathcurveto{\pgfqpoint{12.178232in}{2.500958in}}{\pgfqpoint{12.167633in}{2.505349in}}{\pgfqpoint{12.156583in}{2.505349in}}%
\pgfpathcurveto{\pgfqpoint{12.145533in}{2.505349in}}{\pgfqpoint{12.134934in}{2.500958in}}{\pgfqpoint{12.127120in}{2.493145in}}%
\pgfpathcurveto{\pgfqpoint{12.119307in}{2.485331in}}{\pgfqpoint{12.114916in}{2.474732in}}{\pgfqpoint{12.114916in}{2.463682in}}%
\pgfpathcurveto{\pgfqpoint{12.114916in}{2.452632in}}{\pgfqpoint{12.119307in}{2.442033in}}{\pgfqpoint{12.127120in}{2.434219in}}%
\pgfpathcurveto{\pgfqpoint{12.134934in}{2.426406in}}{\pgfqpoint{12.145533in}{2.422015in}}{\pgfqpoint{12.156583in}{2.422015in}}%
\pgfpathclose%
\pgfusepath{stroke,fill}%
\end{pgfscope}%
\begin{pgfscope}%
\pgfpathrectangle{\pgfqpoint{7.640588in}{0.566125in}}{\pgfqpoint{5.699255in}{2.685432in}}%
\pgfusepath{clip}%
\pgfsetbuttcap%
\pgfsetroundjoin%
\definecolor{currentfill}{rgb}{0.000000,0.000000,0.000000}%
\pgfsetfillcolor{currentfill}%
\pgfsetlinewidth{1.003750pt}%
\definecolor{currentstroke}{rgb}{0.000000,0.000000,0.000000}%
\pgfsetstrokecolor{currentstroke}%
\pgfsetdash{}{0pt}%
\pgfpathmoveto{\pgfqpoint{12.240602in}{2.461181in}}%
\pgfpathcurveto{\pgfqpoint{12.251652in}{2.461181in}}{\pgfqpoint{12.262251in}{2.465571in}}{\pgfqpoint{12.270064in}{2.473384in}}%
\pgfpathcurveto{\pgfqpoint{12.277878in}{2.481198in}}{\pgfqpoint{12.282268in}{2.491797in}}{\pgfqpoint{12.282268in}{2.502847in}}%
\pgfpathcurveto{\pgfqpoint{12.282268in}{2.513897in}}{\pgfqpoint{12.277878in}{2.524496in}}{\pgfqpoint{12.270064in}{2.532310in}}%
\pgfpathcurveto{\pgfqpoint{12.262251in}{2.540124in}}{\pgfqpoint{12.251652in}{2.544514in}}{\pgfqpoint{12.240602in}{2.544514in}}%
\pgfpathcurveto{\pgfqpoint{12.229551in}{2.544514in}}{\pgfqpoint{12.218952in}{2.540124in}}{\pgfqpoint{12.211139in}{2.532310in}}%
\pgfpathcurveto{\pgfqpoint{12.203325in}{2.524496in}}{\pgfqpoint{12.198935in}{2.513897in}}{\pgfqpoint{12.198935in}{2.502847in}}%
\pgfpathcurveto{\pgfqpoint{12.198935in}{2.491797in}}{\pgfqpoint{12.203325in}{2.481198in}}{\pgfqpoint{12.211139in}{2.473384in}}%
\pgfpathcurveto{\pgfqpoint{12.218952in}{2.465571in}}{\pgfqpoint{12.229551in}{2.461181in}}{\pgfqpoint{12.240602in}{2.461181in}}%
\pgfpathclose%
\pgfusepath{stroke,fill}%
\end{pgfscope}%
\begin{pgfscope}%
\pgfpathrectangle{\pgfqpoint{7.640588in}{0.566125in}}{\pgfqpoint{5.699255in}{2.685432in}}%
\pgfusepath{clip}%
\pgfsetbuttcap%
\pgfsetroundjoin%
\definecolor{currentfill}{rgb}{0.000000,0.000000,0.000000}%
\pgfsetfillcolor{currentfill}%
\pgfsetlinewidth{1.003750pt}%
\definecolor{currentstroke}{rgb}{0.000000,0.000000,0.000000}%
\pgfsetstrokecolor{currentstroke}%
\pgfsetdash{}{0pt}%
\pgfpathmoveto{\pgfqpoint{12.576676in}{2.330630in}}%
\pgfpathcurveto{\pgfqpoint{12.587726in}{2.330630in}}{\pgfqpoint{12.598325in}{2.335020in}}{\pgfqpoint{12.606138in}{2.342834in}}%
\pgfpathcurveto{\pgfqpoint{12.613952in}{2.350647in}}{\pgfqpoint{12.618342in}{2.361246in}}{\pgfqpoint{12.618342in}{2.372296in}}%
\pgfpathcurveto{\pgfqpoint{12.618342in}{2.383347in}}{\pgfqpoint{12.613952in}{2.393946in}}{\pgfqpoint{12.606138in}{2.401759in}}%
\pgfpathcurveto{\pgfqpoint{12.598325in}{2.409573in}}{\pgfqpoint{12.587726in}{2.413963in}}{\pgfqpoint{12.576676in}{2.413963in}}%
\pgfpathcurveto{\pgfqpoint{12.565626in}{2.413963in}}{\pgfqpoint{12.555026in}{2.409573in}}{\pgfqpoint{12.547213in}{2.401759in}}%
\pgfpathcurveto{\pgfqpoint{12.539399in}{2.393946in}}{\pgfqpoint{12.535009in}{2.383347in}}{\pgfqpoint{12.535009in}{2.372296in}}%
\pgfpathcurveto{\pgfqpoint{12.535009in}{2.361246in}}{\pgfqpoint{12.539399in}{2.350647in}}{\pgfqpoint{12.547213in}{2.342834in}}%
\pgfpathcurveto{\pgfqpoint{12.555026in}{2.335020in}}{\pgfqpoint{12.565626in}{2.330630in}}{\pgfqpoint{12.576676in}{2.330630in}}%
\pgfpathclose%
\pgfusepath{stroke,fill}%
\end{pgfscope}%
\begin{pgfscope}%
\pgfpathrectangle{\pgfqpoint{7.640588in}{0.566125in}}{\pgfqpoint{5.699255in}{2.685432in}}%
\pgfusepath{clip}%
\pgfsetbuttcap%
\pgfsetroundjoin%
\definecolor{currentfill}{rgb}{0.000000,0.000000,0.000000}%
\pgfsetfillcolor{currentfill}%
\pgfsetlinewidth{1.003750pt}%
\definecolor{currentstroke}{rgb}{0.000000,0.000000,0.000000}%
\pgfsetstrokecolor{currentstroke}%
\pgfsetdash{}{0pt}%
\pgfpathmoveto{\pgfqpoint{12.464651in}{1.965087in}}%
\pgfpathcurveto{\pgfqpoint{12.475701in}{1.965087in}}{\pgfqpoint{12.486300in}{1.969477in}}{\pgfqpoint{12.494114in}{1.977291in}}%
\pgfpathcurveto{\pgfqpoint{12.501927in}{1.985105in}}{\pgfqpoint{12.506318in}{1.995704in}}{\pgfqpoint{12.506318in}{2.006754in}}%
\pgfpathcurveto{\pgfqpoint{12.506318in}{2.017804in}}{\pgfqpoint{12.501927in}{2.028403in}}{\pgfqpoint{12.494114in}{2.036217in}}%
\pgfpathcurveto{\pgfqpoint{12.486300in}{2.044030in}}{\pgfqpoint{12.475701in}{2.048421in}}{\pgfqpoint{12.464651in}{2.048421in}}%
\pgfpathcurveto{\pgfqpoint{12.453601in}{2.048421in}}{\pgfqpoint{12.443002in}{2.044030in}}{\pgfqpoint{12.435188in}{2.036217in}}%
\pgfpathcurveto{\pgfqpoint{12.427375in}{2.028403in}}{\pgfqpoint{12.422984in}{2.017804in}}{\pgfqpoint{12.422984in}{2.006754in}}%
\pgfpathcurveto{\pgfqpoint{12.422984in}{1.995704in}}{\pgfqpoint{12.427375in}{1.985105in}}{\pgfqpoint{12.435188in}{1.977291in}}%
\pgfpathcurveto{\pgfqpoint{12.443002in}{1.969477in}}{\pgfqpoint{12.453601in}{1.965087in}}{\pgfqpoint{12.464651in}{1.965087in}}%
\pgfpathclose%
\pgfusepath{stroke,fill}%
\end{pgfscope}%
\begin{pgfscope}%
\pgfpathrectangle{\pgfqpoint{7.640588in}{0.566125in}}{\pgfqpoint{5.699255in}{2.685432in}}%
\pgfusepath{clip}%
\pgfsetbuttcap%
\pgfsetroundjoin%
\definecolor{currentfill}{rgb}{0.000000,0.000000,1.000000}%
\pgfsetfillcolor{currentfill}%
\pgfsetlinewidth{1.003750pt}%
\definecolor{currentstroke}{rgb}{0.000000,0.000000,1.000000}%
\pgfsetstrokecolor{currentstroke}%
\pgfsetdash{}{0pt}%
\pgfpathmoveto{\pgfqpoint{12.884744in}{1.625655in}}%
\pgfpathcurveto{\pgfqpoint{12.895794in}{1.625655in}}{\pgfqpoint{12.906393in}{1.630045in}}{\pgfqpoint{12.914206in}{1.637859in}}%
\pgfpathcurveto{\pgfqpoint{12.922020in}{1.645672in}}{\pgfqpoint{12.926410in}{1.656271in}}{\pgfqpoint{12.926410in}{1.667321in}}%
\pgfpathcurveto{\pgfqpoint{12.926410in}{1.678372in}}{\pgfqpoint{12.922020in}{1.688971in}}{\pgfqpoint{12.914206in}{1.696784in}}%
\pgfpathcurveto{\pgfqpoint{12.906393in}{1.704598in}}{\pgfqpoint{12.895794in}{1.708988in}}{\pgfqpoint{12.884744in}{1.708988in}}%
\pgfpathcurveto{\pgfqpoint{12.873693in}{1.708988in}}{\pgfqpoint{12.863094in}{1.704598in}}{\pgfqpoint{12.855281in}{1.696784in}}%
\pgfpathcurveto{\pgfqpoint{12.847467in}{1.688971in}}{\pgfqpoint{12.843077in}{1.678372in}}{\pgfqpoint{12.843077in}{1.667321in}}%
\pgfpathcurveto{\pgfqpoint{12.843077in}{1.656271in}}{\pgfqpoint{12.847467in}{1.645672in}}{\pgfqpoint{12.855281in}{1.637859in}}%
\pgfpathcurveto{\pgfqpoint{12.863094in}{1.630045in}}{\pgfqpoint{12.873693in}{1.625655in}}{\pgfqpoint{12.884744in}{1.625655in}}%
\pgfpathclose%
\pgfusepath{stroke,fill}%
\end{pgfscope}%
\begin{pgfscope}%
\pgfsetbuttcap%
\pgfsetroundjoin%
\definecolor{currentfill}{rgb}{0.000000,0.000000,0.000000}%
\pgfsetfillcolor{currentfill}%
\pgfsetlinewidth{0.803000pt}%
\definecolor{currentstroke}{rgb}{0.000000,0.000000,0.000000}%
\pgfsetstrokecolor{currentstroke}%
\pgfsetdash{}{0pt}%
\pgfsys@defobject{currentmarker}{\pgfqpoint{0.000000in}{-0.048611in}}{\pgfqpoint{0.000000in}{0.000000in}}{%
\pgfpathmoveto{\pgfqpoint{0.000000in}{0.000000in}}%
\pgfpathlineto{\pgfqpoint{0.000000in}{-0.048611in}}%
\pgfusepath{stroke,fill}%
}%
\begin{pgfscope}%
\pgfsys@transformshift{7.815627in}{0.566125in}%
\pgfsys@useobject{currentmarker}{}%
\end{pgfscope}%
\end{pgfscope}%
\begin{pgfscope}%
\definecolor{textcolor}{rgb}{0.000000,0.000000,0.000000}%
\pgfsetstrokecolor{textcolor}%
\pgfsetfillcolor{textcolor}%
\pgftext[x=7.815627in,y=0.468902in,,top]{\color{textcolor}\rmfamily\fontsize{10.000000}{12.000000}\selectfont \(\displaystyle 0\)}%
\end{pgfscope}%
\begin{pgfscope}%
\pgfsetbuttcap%
\pgfsetroundjoin%
\definecolor{currentfill}{rgb}{0.000000,0.000000,0.000000}%
\pgfsetfillcolor{currentfill}%
\pgfsetlinewidth{0.803000pt}%
\definecolor{currentstroke}{rgb}{0.000000,0.000000,0.000000}%
\pgfsetstrokecolor{currentstroke}%
\pgfsetdash{}{0pt}%
\pgfsys@defobject{currentmarker}{\pgfqpoint{0.000000in}{-0.048611in}}{\pgfqpoint{0.000000in}{0.000000in}}{%
\pgfpathmoveto{\pgfqpoint{0.000000in}{0.000000in}}%
\pgfpathlineto{\pgfqpoint{0.000000in}{-0.048611in}}%
\pgfusepath{stroke,fill}%
}%
\begin{pgfscope}%
\pgfsys@transformshift{8.515781in}{0.566125in}%
\pgfsys@useobject{currentmarker}{}%
\end{pgfscope}%
\end{pgfscope}%
\begin{pgfscope}%
\definecolor{textcolor}{rgb}{0.000000,0.000000,0.000000}%
\pgfsetstrokecolor{textcolor}%
\pgfsetfillcolor{textcolor}%
\pgftext[x=8.515781in,y=0.468902in,,top]{\color{textcolor}\rmfamily\fontsize{10.000000}{12.000000}\selectfont \(\displaystyle 25\)}%
\end{pgfscope}%
\begin{pgfscope}%
\pgfsetbuttcap%
\pgfsetroundjoin%
\definecolor{currentfill}{rgb}{0.000000,0.000000,0.000000}%
\pgfsetfillcolor{currentfill}%
\pgfsetlinewidth{0.803000pt}%
\definecolor{currentstroke}{rgb}{0.000000,0.000000,0.000000}%
\pgfsetstrokecolor{currentstroke}%
\pgfsetdash{}{0pt}%
\pgfsys@defobject{currentmarker}{\pgfqpoint{0.000000in}{-0.048611in}}{\pgfqpoint{0.000000in}{0.000000in}}{%
\pgfpathmoveto{\pgfqpoint{0.000000in}{0.000000in}}%
\pgfpathlineto{\pgfqpoint{0.000000in}{-0.048611in}}%
\pgfusepath{stroke,fill}%
}%
\begin{pgfscope}%
\pgfsys@transformshift{9.215935in}{0.566125in}%
\pgfsys@useobject{currentmarker}{}%
\end{pgfscope}%
\end{pgfscope}%
\begin{pgfscope}%
\definecolor{textcolor}{rgb}{0.000000,0.000000,0.000000}%
\pgfsetstrokecolor{textcolor}%
\pgfsetfillcolor{textcolor}%
\pgftext[x=9.215935in,y=0.468902in,,top]{\color{textcolor}\rmfamily\fontsize{10.000000}{12.000000}\selectfont \(\displaystyle 50\)}%
\end{pgfscope}%
\begin{pgfscope}%
\pgfsetbuttcap%
\pgfsetroundjoin%
\definecolor{currentfill}{rgb}{0.000000,0.000000,0.000000}%
\pgfsetfillcolor{currentfill}%
\pgfsetlinewidth{0.803000pt}%
\definecolor{currentstroke}{rgb}{0.000000,0.000000,0.000000}%
\pgfsetstrokecolor{currentstroke}%
\pgfsetdash{}{0pt}%
\pgfsys@defobject{currentmarker}{\pgfqpoint{0.000000in}{-0.048611in}}{\pgfqpoint{0.000000in}{0.000000in}}{%
\pgfpathmoveto{\pgfqpoint{0.000000in}{0.000000in}}%
\pgfpathlineto{\pgfqpoint{0.000000in}{-0.048611in}}%
\pgfusepath{stroke,fill}%
}%
\begin{pgfscope}%
\pgfsys@transformshift{9.916090in}{0.566125in}%
\pgfsys@useobject{currentmarker}{}%
\end{pgfscope}%
\end{pgfscope}%
\begin{pgfscope}%
\definecolor{textcolor}{rgb}{0.000000,0.000000,0.000000}%
\pgfsetstrokecolor{textcolor}%
\pgfsetfillcolor{textcolor}%
\pgftext[x=9.916090in,y=0.468902in,,top]{\color{textcolor}\rmfamily\fontsize{10.000000}{12.000000}\selectfont \(\displaystyle 75\)}%
\end{pgfscope}%
\begin{pgfscope}%
\pgfsetbuttcap%
\pgfsetroundjoin%
\definecolor{currentfill}{rgb}{0.000000,0.000000,0.000000}%
\pgfsetfillcolor{currentfill}%
\pgfsetlinewidth{0.803000pt}%
\definecolor{currentstroke}{rgb}{0.000000,0.000000,0.000000}%
\pgfsetstrokecolor{currentstroke}%
\pgfsetdash{}{0pt}%
\pgfsys@defobject{currentmarker}{\pgfqpoint{0.000000in}{-0.048611in}}{\pgfqpoint{0.000000in}{0.000000in}}{%
\pgfpathmoveto{\pgfqpoint{0.000000in}{0.000000in}}%
\pgfpathlineto{\pgfqpoint{0.000000in}{-0.048611in}}%
\pgfusepath{stroke,fill}%
}%
\begin{pgfscope}%
\pgfsys@transformshift{10.616244in}{0.566125in}%
\pgfsys@useobject{currentmarker}{}%
\end{pgfscope}%
\end{pgfscope}%
\begin{pgfscope}%
\definecolor{textcolor}{rgb}{0.000000,0.000000,0.000000}%
\pgfsetstrokecolor{textcolor}%
\pgfsetfillcolor{textcolor}%
\pgftext[x=10.616244in,y=0.468902in,,top]{\color{textcolor}\rmfamily\fontsize{10.000000}{12.000000}\selectfont \(\displaystyle 100\)}%
\end{pgfscope}%
\begin{pgfscope}%
\pgfsetbuttcap%
\pgfsetroundjoin%
\definecolor{currentfill}{rgb}{0.000000,0.000000,0.000000}%
\pgfsetfillcolor{currentfill}%
\pgfsetlinewidth{0.803000pt}%
\definecolor{currentstroke}{rgb}{0.000000,0.000000,0.000000}%
\pgfsetstrokecolor{currentstroke}%
\pgfsetdash{}{0pt}%
\pgfsys@defobject{currentmarker}{\pgfqpoint{0.000000in}{-0.048611in}}{\pgfqpoint{0.000000in}{0.000000in}}{%
\pgfpathmoveto{\pgfqpoint{0.000000in}{0.000000in}}%
\pgfpathlineto{\pgfqpoint{0.000000in}{-0.048611in}}%
\pgfusepath{stroke,fill}%
}%
\begin{pgfscope}%
\pgfsys@transformshift{11.316398in}{0.566125in}%
\pgfsys@useobject{currentmarker}{}%
\end{pgfscope}%
\end{pgfscope}%
\begin{pgfscope}%
\definecolor{textcolor}{rgb}{0.000000,0.000000,0.000000}%
\pgfsetstrokecolor{textcolor}%
\pgfsetfillcolor{textcolor}%
\pgftext[x=11.316398in,y=0.468902in,,top]{\color{textcolor}\rmfamily\fontsize{10.000000}{12.000000}\selectfont \(\displaystyle 125\)}%
\end{pgfscope}%
\begin{pgfscope}%
\pgfsetbuttcap%
\pgfsetroundjoin%
\definecolor{currentfill}{rgb}{0.000000,0.000000,0.000000}%
\pgfsetfillcolor{currentfill}%
\pgfsetlinewidth{0.803000pt}%
\definecolor{currentstroke}{rgb}{0.000000,0.000000,0.000000}%
\pgfsetstrokecolor{currentstroke}%
\pgfsetdash{}{0pt}%
\pgfsys@defobject{currentmarker}{\pgfqpoint{0.000000in}{-0.048611in}}{\pgfqpoint{0.000000in}{0.000000in}}{%
\pgfpathmoveto{\pgfqpoint{0.000000in}{0.000000in}}%
\pgfpathlineto{\pgfqpoint{0.000000in}{-0.048611in}}%
\pgfusepath{stroke,fill}%
}%
\begin{pgfscope}%
\pgfsys@transformshift{12.016552in}{0.566125in}%
\pgfsys@useobject{currentmarker}{}%
\end{pgfscope}%
\end{pgfscope}%
\begin{pgfscope}%
\definecolor{textcolor}{rgb}{0.000000,0.000000,0.000000}%
\pgfsetstrokecolor{textcolor}%
\pgfsetfillcolor{textcolor}%
\pgftext[x=12.016552in,y=0.468902in,,top]{\color{textcolor}\rmfamily\fontsize{10.000000}{12.000000}\selectfont \(\displaystyle 150\)}%
\end{pgfscope}%
\begin{pgfscope}%
\pgfsetbuttcap%
\pgfsetroundjoin%
\definecolor{currentfill}{rgb}{0.000000,0.000000,0.000000}%
\pgfsetfillcolor{currentfill}%
\pgfsetlinewidth{0.803000pt}%
\definecolor{currentstroke}{rgb}{0.000000,0.000000,0.000000}%
\pgfsetstrokecolor{currentstroke}%
\pgfsetdash{}{0pt}%
\pgfsys@defobject{currentmarker}{\pgfqpoint{0.000000in}{-0.048611in}}{\pgfqpoint{0.000000in}{0.000000in}}{%
\pgfpathmoveto{\pgfqpoint{0.000000in}{0.000000in}}%
\pgfpathlineto{\pgfqpoint{0.000000in}{-0.048611in}}%
\pgfusepath{stroke,fill}%
}%
\begin{pgfscope}%
\pgfsys@transformshift{12.716706in}{0.566125in}%
\pgfsys@useobject{currentmarker}{}%
\end{pgfscope}%
\end{pgfscope}%
\begin{pgfscope}%
\definecolor{textcolor}{rgb}{0.000000,0.000000,0.000000}%
\pgfsetstrokecolor{textcolor}%
\pgfsetfillcolor{textcolor}%
\pgftext[x=12.716706in,y=0.468902in,,top]{\color{textcolor}\rmfamily\fontsize{10.000000}{12.000000}\selectfont \(\displaystyle 175\)}%
\end{pgfscope}%
\begin{pgfscope}%
\pgfsetbuttcap%
\pgfsetroundjoin%
\definecolor{currentfill}{rgb}{0.000000,0.000000,0.000000}%
\pgfsetfillcolor{currentfill}%
\pgfsetlinewidth{0.803000pt}%
\definecolor{currentstroke}{rgb}{0.000000,0.000000,0.000000}%
\pgfsetstrokecolor{currentstroke}%
\pgfsetdash{}{0pt}%
\pgfsys@defobject{currentmarker}{\pgfqpoint{-0.048611in}{0.000000in}}{\pgfqpoint{0.000000in}{0.000000in}}{%
\pgfpathmoveto{\pgfqpoint{0.000000in}{0.000000in}}%
\pgfpathlineto{\pgfqpoint{-0.048611in}{0.000000in}}%
\pgfusepath{stroke,fill}%
}%
\begin{pgfscope}%
\pgfsys@transformshift{7.640588in}{0.688190in}%
\pgfsys@useobject{currentmarker}{}%
\end{pgfscope}%
\end{pgfscope}%
\begin{pgfscope}%
\definecolor{textcolor}{rgb}{0.000000,0.000000,0.000000}%
\pgfsetstrokecolor{textcolor}%
\pgfsetfillcolor{textcolor}%
\pgftext[x=7.473921in, y=0.635428in, left, base]{\color{textcolor}\rmfamily\fontsize{10.000000}{12.000000}\selectfont \(\displaystyle 0\)}%
\end{pgfscope}%
\begin{pgfscope}%
\pgfsetbuttcap%
\pgfsetroundjoin%
\definecolor{currentfill}{rgb}{0.000000,0.000000,0.000000}%
\pgfsetfillcolor{currentfill}%
\pgfsetlinewidth{0.803000pt}%
\definecolor{currentstroke}{rgb}{0.000000,0.000000,0.000000}%
\pgfsetstrokecolor{currentstroke}%
\pgfsetdash{}{0pt}%
\pgfsys@defobject{currentmarker}{\pgfqpoint{-0.048611in}{0.000000in}}{\pgfqpoint{0.000000in}{0.000000in}}{%
\pgfpathmoveto{\pgfqpoint{0.000000in}{0.000000in}}%
\pgfpathlineto{\pgfqpoint{-0.048611in}{0.000000in}}%
\pgfusepath{stroke,fill}%
}%
\begin{pgfscope}%
\pgfsys@transformshift{7.640588in}{1.014567in}%
\pgfsys@useobject{currentmarker}{}%
\end{pgfscope}%
\end{pgfscope}%
\begin{pgfscope}%
\definecolor{textcolor}{rgb}{0.000000,0.000000,0.000000}%
\pgfsetstrokecolor{textcolor}%
\pgfsetfillcolor{textcolor}%
\pgftext[x=7.404477in, y=0.961805in, left, base]{\color{textcolor}\rmfamily\fontsize{10.000000}{12.000000}\selectfont \(\displaystyle 25\)}%
\end{pgfscope}%
\begin{pgfscope}%
\pgfsetbuttcap%
\pgfsetroundjoin%
\definecolor{currentfill}{rgb}{0.000000,0.000000,0.000000}%
\pgfsetfillcolor{currentfill}%
\pgfsetlinewidth{0.803000pt}%
\definecolor{currentstroke}{rgb}{0.000000,0.000000,0.000000}%
\pgfsetstrokecolor{currentstroke}%
\pgfsetdash{}{0pt}%
\pgfsys@defobject{currentmarker}{\pgfqpoint{-0.048611in}{0.000000in}}{\pgfqpoint{0.000000in}{0.000000in}}{%
\pgfpathmoveto{\pgfqpoint{0.000000in}{0.000000in}}%
\pgfpathlineto{\pgfqpoint{-0.048611in}{0.000000in}}%
\pgfusepath{stroke,fill}%
}%
\begin{pgfscope}%
\pgfsys@transformshift{7.640588in}{1.340944in}%
\pgfsys@useobject{currentmarker}{}%
\end{pgfscope}%
\end{pgfscope}%
\begin{pgfscope}%
\definecolor{textcolor}{rgb}{0.000000,0.000000,0.000000}%
\pgfsetstrokecolor{textcolor}%
\pgfsetfillcolor{textcolor}%
\pgftext[x=7.404477in, y=1.288183in, left, base]{\color{textcolor}\rmfamily\fontsize{10.000000}{12.000000}\selectfont \(\displaystyle 50\)}%
\end{pgfscope}%
\begin{pgfscope}%
\pgfsetbuttcap%
\pgfsetroundjoin%
\definecolor{currentfill}{rgb}{0.000000,0.000000,0.000000}%
\pgfsetfillcolor{currentfill}%
\pgfsetlinewidth{0.803000pt}%
\definecolor{currentstroke}{rgb}{0.000000,0.000000,0.000000}%
\pgfsetstrokecolor{currentstroke}%
\pgfsetdash{}{0pt}%
\pgfsys@defobject{currentmarker}{\pgfqpoint{-0.048611in}{0.000000in}}{\pgfqpoint{0.000000in}{0.000000in}}{%
\pgfpathmoveto{\pgfqpoint{0.000000in}{0.000000in}}%
\pgfpathlineto{\pgfqpoint{-0.048611in}{0.000000in}}%
\pgfusepath{stroke,fill}%
}%
\begin{pgfscope}%
\pgfsys@transformshift{7.640588in}{1.667321in}%
\pgfsys@useobject{currentmarker}{}%
\end{pgfscope}%
\end{pgfscope}%
\begin{pgfscope}%
\definecolor{textcolor}{rgb}{0.000000,0.000000,0.000000}%
\pgfsetstrokecolor{textcolor}%
\pgfsetfillcolor{textcolor}%
\pgftext[x=7.404477in, y=1.614560in, left, base]{\color{textcolor}\rmfamily\fontsize{10.000000}{12.000000}\selectfont \(\displaystyle 75\)}%
\end{pgfscope}%
\begin{pgfscope}%
\pgfsetbuttcap%
\pgfsetroundjoin%
\definecolor{currentfill}{rgb}{0.000000,0.000000,0.000000}%
\pgfsetfillcolor{currentfill}%
\pgfsetlinewidth{0.803000pt}%
\definecolor{currentstroke}{rgb}{0.000000,0.000000,0.000000}%
\pgfsetstrokecolor{currentstroke}%
\pgfsetdash{}{0pt}%
\pgfsys@defobject{currentmarker}{\pgfqpoint{-0.048611in}{0.000000in}}{\pgfqpoint{0.000000in}{0.000000in}}{%
\pgfpathmoveto{\pgfqpoint{0.000000in}{0.000000in}}%
\pgfpathlineto{\pgfqpoint{-0.048611in}{0.000000in}}%
\pgfusepath{stroke,fill}%
}%
\begin{pgfscope}%
\pgfsys@transformshift{7.640588in}{1.993699in}%
\pgfsys@useobject{currentmarker}{}%
\end{pgfscope}%
\end{pgfscope}%
\begin{pgfscope}%
\definecolor{textcolor}{rgb}{0.000000,0.000000,0.000000}%
\pgfsetstrokecolor{textcolor}%
\pgfsetfillcolor{textcolor}%
\pgftext[x=7.335032in, y=1.940937in, left, base]{\color{textcolor}\rmfamily\fontsize{10.000000}{12.000000}\selectfont \(\displaystyle 100\)}%
\end{pgfscope}%
\begin{pgfscope}%
\pgfsetbuttcap%
\pgfsetroundjoin%
\definecolor{currentfill}{rgb}{0.000000,0.000000,0.000000}%
\pgfsetfillcolor{currentfill}%
\pgfsetlinewidth{0.803000pt}%
\definecolor{currentstroke}{rgb}{0.000000,0.000000,0.000000}%
\pgfsetstrokecolor{currentstroke}%
\pgfsetdash{}{0pt}%
\pgfsys@defobject{currentmarker}{\pgfqpoint{-0.048611in}{0.000000in}}{\pgfqpoint{0.000000in}{0.000000in}}{%
\pgfpathmoveto{\pgfqpoint{0.000000in}{0.000000in}}%
\pgfpathlineto{\pgfqpoint{-0.048611in}{0.000000in}}%
\pgfusepath{stroke,fill}%
}%
\begin{pgfscope}%
\pgfsys@transformshift{7.640588in}{2.320076in}%
\pgfsys@useobject{currentmarker}{}%
\end{pgfscope}%
\end{pgfscope}%
\begin{pgfscope}%
\definecolor{textcolor}{rgb}{0.000000,0.000000,0.000000}%
\pgfsetstrokecolor{textcolor}%
\pgfsetfillcolor{textcolor}%
\pgftext[x=7.335032in, y=2.267314in, left, base]{\color{textcolor}\rmfamily\fontsize{10.000000}{12.000000}\selectfont \(\displaystyle 125\)}%
\end{pgfscope}%
\begin{pgfscope}%
\pgfsetbuttcap%
\pgfsetroundjoin%
\definecolor{currentfill}{rgb}{0.000000,0.000000,0.000000}%
\pgfsetfillcolor{currentfill}%
\pgfsetlinewidth{0.803000pt}%
\definecolor{currentstroke}{rgb}{0.000000,0.000000,0.000000}%
\pgfsetstrokecolor{currentstroke}%
\pgfsetdash{}{0pt}%
\pgfsys@defobject{currentmarker}{\pgfqpoint{-0.048611in}{0.000000in}}{\pgfqpoint{0.000000in}{0.000000in}}{%
\pgfpathmoveto{\pgfqpoint{0.000000in}{0.000000in}}%
\pgfpathlineto{\pgfqpoint{-0.048611in}{0.000000in}}%
\pgfusepath{stroke,fill}%
}%
\begin{pgfscope}%
\pgfsys@transformshift{7.640588in}{2.646453in}%
\pgfsys@useobject{currentmarker}{}%
\end{pgfscope}%
\end{pgfscope}%
\begin{pgfscope}%
\definecolor{textcolor}{rgb}{0.000000,0.000000,0.000000}%
\pgfsetstrokecolor{textcolor}%
\pgfsetfillcolor{textcolor}%
\pgftext[x=7.335032in, y=2.593692in, left, base]{\color{textcolor}\rmfamily\fontsize{10.000000}{12.000000}\selectfont \(\displaystyle 150\)}%
\end{pgfscope}%
\begin{pgfscope}%
\pgfsetbuttcap%
\pgfsetroundjoin%
\definecolor{currentfill}{rgb}{0.000000,0.000000,0.000000}%
\pgfsetfillcolor{currentfill}%
\pgfsetlinewidth{0.803000pt}%
\definecolor{currentstroke}{rgb}{0.000000,0.000000,0.000000}%
\pgfsetstrokecolor{currentstroke}%
\pgfsetdash{}{0pt}%
\pgfsys@defobject{currentmarker}{\pgfqpoint{-0.048611in}{0.000000in}}{\pgfqpoint{0.000000in}{0.000000in}}{%
\pgfpathmoveto{\pgfqpoint{0.000000in}{0.000000in}}%
\pgfpathlineto{\pgfqpoint{-0.048611in}{0.000000in}}%
\pgfusepath{stroke,fill}%
}%
\begin{pgfscope}%
\pgfsys@transformshift{7.640588in}{2.972831in}%
\pgfsys@useobject{currentmarker}{}%
\end{pgfscope}%
\end{pgfscope}%
\begin{pgfscope}%
\definecolor{textcolor}{rgb}{0.000000,0.000000,0.000000}%
\pgfsetstrokecolor{textcolor}%
\pgfsetfillcolor{textcolor}%
\pgftext[x=7.335032in, y=2.920069in, left, base]{\color{textcolor}\rmfamily\fontsize{10.000000}{12.000000}\selectfont \(\displaystyle 175\)}%
\end{pgfscope}%
\begin{pgfscope}%
\pgfsetrectcap%
\pgfsetmiterjoin%
\pgfsetlinewidth{0.803000pt}%
\definecolor{currentstroke}{rgb}{0.000000,0.000000,0.000000}%
\pgfsetstrokecolor{currentstroke}%
\pgfsetdash{}{0pt}%
\pgfpathmoveto{\pgfqpoint{7.640588in}{0.566125in}}%
\pgfpathlineto{\pgfqpoint{7.640588in}{3.251557in}}%
\pgfusepath{stroke}%
\end{pgfscope}%
\begin{pgfscope}%
\pgfsetrectcap%
\pgfsetmiterjoin%
\pgfsetlinewidth{0.803000pt}%
\definecolor{currentstroke}{rgb}{0.000000,0.000000,0.000000}%
\pgfsetstrokecolor{currentstroke}%
\pgfsetdash{}{0pt}%
\pgfpathmoveto{\pgfqpoint{13.339844in}{0.566125in}}%
\pgfpathlineto{\pgfqpoint{13.339844in}{3.251557in}}%
\pgfusepath{stroke}%
\end{pgfscope}%
\begin{pgfscope}%
\pgfsetrectcap%
\pgfsetmiterjoin%
\pgfsetlinewidth{0.803000pt}%
\definecolor{currentstroke}{rgb}{0.000000,0.000000,0.000000}%
\pgfsetstrokecolor{currentstroke}%
\pgfsetdash{}{0pt}%
\pgfpathmoveto{\pgfqpoint{7.640588in}{0.566125in}}%
\pgfpathlineto{\pgfqpoint{13.339844in}{0.566125in}}%
\pgfusepath{stroke}%
\end{pgfscope}%
\begin{pgfscope}%
\pgfsetrectcap%
\pgfsetmiterjoin%
\pgfsetlinewidth{0.803000pt}%
\definecolor{currentstroke}{rgb}{0.000000,0.000000,0.000000}%
\pgfsetstrokecolor{currentstroke}%
\pgfsetdash{}{0pt}%
\pgfpathmoveto{\pgfqpoint{7.640588in}{3.251557in}}%
\pgfpathlineto{\pgfqpoint{13.339844in}{3.251557in}}%
\pgfusepath{stroke}%
\end{pgfscope}%
\begin{pgfscope}%
\definecolor{textcolor}{rgb}{0.000000,0.000000,0.000000}%
\pgfsetstrokecolor{textcolor}%
\pgfsetfillcolor{textcolor}%
\pgftext[x=10.490216in,y=3.334890in,,base]{\color{textcolor}\rmfamily\fontsize{12.000000}{14.400000}\selectfont custom 1484.4}%
\end{pgfscope}%
\end{pgfpicture}%
\makeatother%
\endgroup%
}
\end{figure}
\begin{figure}[p]
    \caption{Example Euclidian TSP problem where custom fares quite badly}
    \centering
    \scalebox{0.5}{%% Creator: Matplotlib, PGF backend
%%
%% To include the figure in your LaTeX document, write
%%   \input{<filename>.pgf}
%%
%% Make sure the required packages are loaded in your preamble
%%   \usepackage{pgf}
%%
%% and, on pdftex
%%   \usepackage[utf8]{inputenc}\DeclareUnicodeCharacter{2212}{-}
%%
%% or, on luatex and xetex
%%   \usepackage{unicode-math}
%%
%% Figures using additional raster images can only be included by \input if
%% they are in the same directory as the main LaTeX file. For loading figures
%% from other directories you can use the `import` package
%%   \usepackage{import}
%%
%% and then include the figures with
%%   \import{<path to file>}{<filename>.pgf}
%%
%% Matplotlib used the following preamble
%%   \usepackage{fontspec}
%%   \setmainfont{DejaVuSerif.ttf}[Path=/home/maks/.local/share/virtualenvs/CW3-zMJxnm_q/lib/python3.7/site-packages/matplotlib/mpl-data/fonts/ttf/]
%%   \setsansfont{DejaVuSans.ttf}[Path=/home/maks/.local/share/virtualenvs/CW3-zMJxnm_q/lib/python3.7/site-packages/matplotlib/mpl-data/fonts/ttf/]
%%   \setmonofont{DejaVuSansMono.ttf}[Path=/home/maks/.local/share/virtualenvs/CW3-zMJxnm_q/lib/python3.7/site-packages/matplotlib/mpl-data/fonts/ttf/]
%%
\begingroup%
\makeatletter%
\begin{pgfpicture}%
\pgfpathrectangle{\pgfpointorigin}{\pgfqpoint{13.660000in}{7.340000in}}%
\pgfusepath{use as bounding box, clip}%
\begin{pgfscope}%
\pgfsetbuttcap%
\pgfsetmiterjoin%
\definecolor{currentfill}{rgb}{1.000000,1.000000,1.000000}%
\pgfsetfillcolor{currentfill}%
\pgfsetlinewidth{0.000000pt}%
\definecolor{currentstroke}{rgb}{1.000000,1.000000,1.000000}%
\pgfsetstrokecolor{currentstroke}%
\pgfsetdash{}{0pt}%
\pgfpathmoveto{\pgfqpoint{0.000000in}{0.000000in}}%
\pgfpathlineto{\pgfqpoint{13.660000in}{0.000000in}}%
\pgfpathlineto{\pgfqpoint{13.660000in}{7.340000in}}%
\pgfpathlineto{\pgfqpoint{0.000000in}{7.340000in}}%
\pgfpathclose%
\pgfusepath{fill}%
\end{pgfscope}%
\begin{pgfscope}%
\pgfsetbuttcap%
\pgfsetmiterjoin%
\definecolor{currentfill}{rgb}{1.000000,1.000000,1.000000}%
\pgfsetfillcolor{currentfill}%
\pgfsetlinewidth{0.000000pt}%
\definecolor{currentstroke}{rgb}{0.000000,0.000000,0.000000}%
\pgfsetstrokecolor{currentstroke}%
\pgfsetstrokeopacity{0.000000}%
\pgfsetdash{}{0pt}%
\pgfpathmoveto{\pgfqpoint{0.970666in}{4.121437in}}%
\pgfpathlineto{\pgfqpoint{6.669922in}{4.121437in}}%
\pgfpathlineto{\pgfqpoint{6.669922in}{6.806869in}}%
\pgfpathlineto{\pgfqpoint{0.970666in}{6.806869in}}%
\pgfpathclose%
\pgfusepath{fill}%
\end{pgfscope}%
\begin{pgfscope}%
\pgfpathrectangle{\pgfqpoint{0.970666in}{4.121437in}}{\pgfqpoint{5.699255in}{2.685432in}}%
\pgfusepath{clip}%
\pgfsetrectcap%
\pgfsetroundjoin%
\pgfsetlinewidth{1.505625pt}%
\definecolor{currentstroke}{rgb}{0.000000,0.000000,0.000000}%
\pgfsetstrokecolor{currentstroke}%
\pgfsetdash{}{0pt}%
\pgfpathmoveto{\pgfqpoint{4.435555in}{5.745842in}}%
\pgfpathlineto{\pgfqpoint{6.119426in}{6.684804in}}%
\pgfpathlineto{\pgfqpoint{4.144116in}{6.418765in}}%
\pgfpathlineto{\pgfqpoint{2.265952in}{6.184024in}}%
\pgfpathlineto{\pgfqpoint{1.877366in}{4.650386in}}%
\pgfpathlineto{\pgfqpoint{3.140269in}{4.509542in}}%
\pgfpathlineto{\pgfqpoint{3.528855in}{5.292010in}}%
\pgfpathlineto{\pgfqpoint{1.618309in}{5.949284in}}%
\pgfpathlineto{\pgfqpoint{3.528855in}{6.324869in}}%
\pgfpathlineto{\pgfqpoint{4.144116in}{6.011881in}}%
\pgfpathlineto{\pgfqpoint{5.115579in}{6.309219in}}%
\pgfpathlineto{\pgfqpoint{2.104041in}{6.669155in}}%
\pgfpathlineto{\pgfqpoint{2.201188in}{6.497012in}}%
\pgfpathlineto{\pgfqpoint{1.747838in}{4.619087in}}%
\pgfpathlineto{\pgfqpoint{5.925133in}{5.964933in}}%
\pgfpathlineto{\pgfqpoint{5.471783in}{4.712983in}}%
\pgfpathlineto{\pgfqpoint{5.309872in}{4.243502in}}%
\pgfpathlineto{\pgfqpoint{3.949823in}{4.353048in}}%
\pgfpathlineto{\pgfqpoint{2.201188in}{5.651946in}}%
\pgfpathlineto{\pgfqpoint{6.151808in}{6.434414in}}%
\pgfpathlineto{\pgfqpoint{5.763222in}{6.622207in}}%
\pgfpathlineto{\pgfqpoint{2.201188in}{6.543960in}}%
\pgfpathlineto{\pgfqpoint{1.650691in}{4.493892in}}%
\pgfpathlineto{\pgfqpoint{3.593619in}{4.775581in}}%
\pgfpathlineto{\pgfqpoint{4.759376in}{4.775581in}}%
\pgfpathlineto{\pgfqpoint{2.201188in}{5.495452in}}%
\pgfpathlineto{\pgfqpoint{2.104041in}{6.528310in}}%
\pgfpathlineto{\pgfqpoint{3.626001in}{5.964933in}}%
\pgfpathlineto{\pgfqpoint{3.237416in}{4.728633in}}%
\pgfpathlineto{\pgfqpoint{4.759376in}{5.385906in}}%
\pgfpathlineto{\pgfqpoint{6.184190in}{4.838178in}}%
\pgfpathlineto{\pgfqpoint{5.277490in}{4.822529in}}%
\pgfpathlineto{\pgfqpoint{4.046969in}{4.916425in}}%
\pgfpathlineto{\pgfqpoint{2.363098in}{4.572139in}}%
\pgfpathlineto{\pgfqpoint{2.492627in}{6.653505in}}%
\pgfpathlineto{\pgfqpoint{5.471783in}{6.309219in}}%
\pgfpathlineto{\pgfqpoint{6.022279in}{5.448504in}}%
\pgfpathlineto{\pgfqpoint{4.791758in}{4.243502in}}%
\pgfpathlineto{\pgfqpoint{2.945977in}{5.432854in}}%
\pgfpathlineto{\pgfqpoint{1.262106in}{5.151166in}}%
\pgfpathlineto{\pgfqpoint{2.525009in}{4.525191in}}%
\pgfpathlineto{\pgfqpoint{3.496473in}{5.198114in}}%
\pgfpathlineto{\pgfqpoint{3.431709in}{5.651946in}}%
\pgfpathlineto{\pgfqpoint{4.208880in}{5.354608in}}%
\pgfpathlineto{\pgfqpoint{5.374637in}{6.043180in}}%
\pgfpathlineto{\pgfqpoint{5.730840in}{5.777141in}}%
\pgfpathlineto{\pgfqpoint{5.795604in}{5.010321in}}%
\pgfpathlineto{\pgfqpoint{2.686920in}{4.431295in}}%
\pgfpathlineto{\pgfqpoint{4.500319in}{5.730192in}}%
\pgfpathlineto{\pgfqpoint{2.945977in}{6.152725in}}%
\pgfpathlineto{\pgfqpoint{3.172652in}{6.277920in}}%
\pgfpathlineto{\pgfqpoint{4.888905in}{6.434414in}}%
\pgfpathlineto{\pgfqpoint{4.824140in}{6.387466in}}%
\pgfpathlineto{\pgfqpoint{3.464091in}{5.464153in}}%
\pgfpathlineto{\pgfqpoint{2.006895in}{4.556490in}}%
\pgfpathlineto{\pgfqpoint{2.945977in}{4.306100in}}%
\pgfpathlineto{\pgfqpoint{5.374637in}{4.885126in}}%
\pgfpathlineto{\pgfqpoint{1.488781in}{4.712983in}}%
\pgfpathlineto{\pgfqpoint{2.654537in}{5.338958in}}%
\pgfpathlineto{\pgfqpoint{2.363098in}{5.714543in}}%
\pgfpathlineto{\pgfqpoint{3.431709in}{5.589348in}}%
\pgfpathlineto{\pgfqpoint{3.982205in}{4.384347in}}%
\pgfpathlineto{\pgfqpoint{3.140269in}{5.432854in}}%
\pgfpathlineto{\pgfqpoint{3.334562in}{5.855387in}}%
\pgfpathlineto{\pgfqpoint{5.601311in}{5.917985in}}%
\pgfpathlineto{\pgfqpoint{5.147962in}{5.417205in}}%
\pgfpathlineto{\pgfqpoint{2.233570in}{4.509542in}}%
\pgfpathlineto{\pgfqpoint{1.456398in}{4.337398in}}%
\pgfpathlineto{\pgfqpoint{1.229724in}{4.634736in}}%
\pgfpathlineto{\pgfqpoint{5.277490in}{4.744282in}}%
\pgfpathlineto{\pgfqpoint{4.662230in}{6.418765in}}%
\pgfpathlineto{\pgfqpoint{4.338408in}{6.450064in}}%
\pgfpathlineto{\pgfqpoint{4.565083in}{4.446944in}}%
\pgfpathlineto{\pgfqpoint{4.888905in}{4.712983in}}%
\pgfpathlineto{\pgfqpoint{5.342254in}{5.072919in}}%
\pgfpathlineto{\pgfqpoint{2.557391in}{5.479803in}}%
\pgfpathlineto{\pgfqpoint{5.536547in}{6.637856in}}%
\pgfpathlineto{\pgfqpoint{5.795604in}{6.653505in}}%
\pgfpathlineto{\pgfqpoint{6.410865in}{6.340518in}}%
\pgfpathlineto{\pgfqpoint{2.686920in}{6.043180in}}%
\pgfpathlineto{\pgfqpoint{1.456398in}{5.917985in}}%
\pgfpathlineto{\pgfqpoint{2.136423in}{4.681685in}}%
\pgfpathlineto{\pgfqpoint{4.435555in}{5.745842in}}%
\pgfusepath{stroke}%
\end{pgfscope}%
\begin{pgfscope}%
\pgfpathrectangle{\pgfqpoint{0.970666in}{4.121437in}}{\pgfqpoint{5.699255in}{2.685432in}}%
\pgfusepath{clip}%
\pgfsetbuttcap%
\pgfsetroundjoin%
\definecolor{currentfill}{rgb}{1.000000,0.000000,0.000000}%
\pgfsetfillcolor{currentfill}%
\pgfsetlinewidth{1.003750pt}%
\definecolor{currentstroke}{rgb}{1.000000,0.000000,0.000000}%
\pgfsetstrokecolor{currentstroke}%
\pgfsetdash{}{0pt}%
\pgfpathmoveto{\pgfqpoint{4.435555in}{5.704175in}}%
\pgfpathcurveto{\pgfqpoint{4.446605in}{5.704175in}}{\pgfqpoint{4.457204in}{5.708565in}}{\pgfqpoint{4.465017in}{5.716379in}}%
\pgfpathcurveto{\pgfqpoint{4.472831in}{5.724193in}}{\pgfqpoint{4.477221in}{5.734792in}}{\pgfqpoint{4.477221in}{5.745842in}}%
\pgfpathcurveto{\pgfqpoint{4.477221in}{5.756892in}}{\pgfqpoint{4.472831in}{5.767491in}}{\pgfqpoint{4.465017in}{5.775305in}}%
\pgfpathcurveto{\pgfqpoint{4.457204in}{5.783118in}}{\pgfqpoint{4.446605in}{5.787509in}}{\pgfqpoint{4.435555in}{5.787509in}}%
\pgfpathcurveto{\pgfqpoint{4.424505in}{5.787509in}}{\pgfqpoint{4.413906in}{5.783118in}}{\pgfqpoint{4.406092in}{5.775305in}}%
\pgfpathcurveto{\pgfqpoint{4.398278in}{5.767491in}}{\pgfqpoint{4.393888in}{5.756892in}}{\pgfqpoint{4.393888in}{5.745842in}}%
\pgfpathcurveto{\pgfqpoint{4.393888in}{5.734792in}}{\pgfqpoint{4.398278in}{5.724193in}}{\pgfqpoint{4.406092in}{5.716379in}}%
\pgfpathcurveto{\pgfqpoint{4.413906in}{5.708565in}}{\pgfqpoint{4.424505in}{5.704175in}}{\pgfqpoint{4.435555in}{5.704175in}}%
\pgfpathclose%
\pgfusepath{stroke,fill}%
\end{pgfscope}%
\begin{pgfscope}%
\pgfpathrectangle{\pgfqpoint{0.970666in}{4.121437in}}{\pgfqpoint{5.699255in}{2.685432in}}%
\pgfusepath{clip}%
\pgfsetbuttcap%
\pgfsetroundjoin%
\definecolor{currentfill}{rgb}{0.750000,0.750000,0.000000}%
\pgfsetfillcolor{currentfill}%
\pgfsetlinewidth{1.003750pt}%
\definecolor{currentstroke}{rgb}{0.750000,0.750000,0.000000}%
\pgfsetstrokecolor{currentstroke}%
\pgfsetdash{}{0pt}%
\pgfpathmoveto{\pgfqpoint{6.119426in}{6.643137in}}%
\pgfpathcurveto{\pgfqpoint{6.130476in}{6.643137in}}{\pgfqpoint{6.141075in}{6.647528in}}{\pgfqpoint{6.148888in}{6.655341in}}%
\pgfpathcurveto{\pgfqpoint{6.156702in}{6.663155in}}{\pgfqpoint{6.161092in}{6.673754in}}{\pgfqpoint{6.161092in}{6.684804in}}%
\pgfpathcurveto{\pgfqpoint{6.161092in}{6.695854in}}{\pgfqpoint{6.156702in}{6.706453in}}{\pgfqpoint{6.148888in}{6.714267in}}%
\pgfpathcurveto{\pgfqpoint{6.141075in}{6.722081in}}{\pgfqpoint{6.130476in}{6.726471in}}{\pgfqpoint{6.119426in}{6.726471in}}%
\pgfpathcurveto{\pgfqpoint{6.108375in}{6.726471in}}{\pgfqpoint{6.097776in}{6.722081in}}{\pgfqpoint{6.089963in}{6.714267in}}%
\pgfpathcurveto{\pgfqpoint{6.082149in}{6.706453in}}{\pgfqpoint{6.077759in}{6.695854in}}{\pgfqpoint{6.077759in}{6.684804in}}%
\pgfpathcurveto{\pgfqpoint{6.077759in}{6.673754in}}{\pgfqpoint{6.082149in}{6.663155in}}{\pgfqpoint{6.089963in}{6.655341in}}%
\pgfpathcurveto{\pgfqpoint{6.097776in}{6.647528in}}{\pgfqpoint{6.108375in}{6.643137in}}{\pgfqpoint{6.119426in}{6.643137in}}%
\pgfpathclose%
\pgfusepath{stroke,fill}%
\end{pgfscope}%
\begin{pgfscope}%
\pgfpathrectangle{\pgfqpoint{0.970666in}{4.121437in}}{\pgfqpoint{5.699255in}{2.685432in}}%
\pgfusepath{clip}%
\pgfsetbuttcap%
\pgfsetroundjoin%
\definecolor{currentfill}{rgb}{0.000000,0.000000,0.000000}%
\pgfsetfillcolor{currentfill}%
\pgfsetlinewidth{1.003750pt}%
\definecolor{currentstroke}{rgb}{0.000000,0.000000,0.000000}%
\pgfsetstrokecolor{currentstroke}%
\pgfsetdash{}{0pt}%
\pgfpathmoveto{\pgfqpoint{4.144116in}{6.377098in}}%
\pgfpathcurveto{\pgfqpoint{4.155166in}{6.377098in}}{\pgfqpoint{4.165765in}{6.381488in}}{\pgfqpoint{4.173578in}{6.389302in}}%
\pgfpathcurveto{\pgfqpoint{4.181392in}{6.397116in}}{\pgfqpoint{4.185782in}{6.407715in}}{\pgfqpoint{4.185782in}{6.418765in}}%
\pgfpathcurveto{\pgfqpoint{4.185782in}{6.429815in}}{\pgfqpoint{4.181392in}{6.440414in}}{\pgfqpoint{4.173578in}{6.448228in}}%
\pgfpathcurveto{\pgfqpoint{4.165765in}{6.456041in}}{\pgfqpoint{4.155166in}{6.460431in}}{\pgfqpoint{4.144116in}{6.460431in}}%
\pgfpathcurveto{\pgfqpoint{4.133065in}{6.460431in}}{\pgfqpoint{4.122466in}{6.456041in}}{\pgfqpoint{4.114653in}{6.448228in}}%
\pgfpathcurveto{\pgfqpoint{4.106839in}{6.440414in}}{\pgfqpoint{4.102449in}{6.429815in}}{\pgfqpoint{4.102449in}{6.418765in}}%
\pgfpathcurveto{\pgfqpoint{4.102449in}{6.407715in}}{\pgfqpoint{4.106839in}{6.397116in}}{\pgfqpoint{4.114653in}{6.389302in}}%
\pgfpathcurveto{\pgfqpoint{4.122466in}{6.381488in}}{\pgfqpoint{4.133065in}{6.377098in}}{\pgfqpoint{4.144116in}{6.377098in}}%
\pgfpathclose%
\pgfusepath{stroke,fill}%
\end{pgfscope}%
\begin{pgfscope}%
\pgfpathrectangle{\pgfqpoint{0.970666in}{4.121437in}}{\pgfqpoint{5.699255in}{2.685432in}}%
\pgfusepath{clip}%
\pgfsetbuttcap%
\pgfsetroundjoin%
\definecolor{currentfill}{rgb}{0.000000,0.000000,0.000000}%
\pgfsetfillcolor{currentfill}%
\pgfsetlinewidth{1.003750pt}%
\definecolor{currentstroke}{rgb}{0.000000,0.000000,0.000000}%
\pgfsetstrokecolor{currentstroke}%
\pgfsetdash{}{0pt}%
\pgfpathmoveto{\pgfqpoint{2.265952in}{6.142358in}}%
\pgfpathcurveto{\pgfqpoint{2.277002in}{6.142358in}}{\pgfqpoint{2.287601in}{6.146748in}}{\pgfqpoint{2.295415in}{6.154561in}}%
\pgfpathcurveto{\pgfqpoint{2.303228in}{6.162375in}}{\pgfqpoint{2.307618in}{6.172974in}}{\pgfqpoint{2.307618in}{6.184024in}}%
\pgfpathcurveto{\pgfqpoint{2.307618in}{6.195074in}}{\pgfqpoint{2.303228in}{6.205673in}}{\pgfqpoint{2.295415in}{6.213487in}}%
\pgfpathcurveto{\pgfqpoint{2.287601in}{6.221301in}}{\pgfqpoint{2.277002in}{6.225691in}}{\pgfqpoint{2.265952in}{6.225691in}}%
\pgfpathcurveto{\pgfqpoint{2.254902in}{6.225691in}}{\pgfqpoint{2.244303in}{6.221301in}}{\pgfqpoint{2.236489in}{6.213487in}}%
\pgfpathcurveto{\pgfqpoint{2.228675in}{6.205673in}}{\pgfqpoint{2.224285in}{6.195074in}}{\pgfqpoint{2.224285in}{6.184024in}}%
\pgfpathcurveto{\pgfqpoint{2.224285in}{6.172974in}}{\pgfqpoint{2.228675in}{6.162375in}}{\pgfqpoint{2.236489in}{6.154561in}}%
\pgfpathcurveto{\pgfqpoint{2.244303in}{6.146748in}}{\pgfqpoint{2.254902in}{6.142358in}}{\pgfqpoint{2.265952in}{6.142358in}}%
\pgfpathclose%
\pgfusepath{stroke,fill}%
\end{pgfscope}%
\begin{pgfscope}%
\pgfpathrectangle{\pgfqpoint{0.970666in}{4.121437in}}{\pgfqpoint{5.699255in}{2.685432in}}%
\pgfusepath{clip}%
\pgfsetbuttcap%
\pgfsetroundjoin%
\definecolor{currentfill}{rgb}{0.000000,0.000000,0.000000}%
\pgfsetfillcolor{currentfill}%
\pgfsetlinewidth{1.003750pt}%
\definecolor{currentstroke}{rgb}{0.000000,0.000000,0.000000}%
\pgfsetstrokecolor{currentstroke}%
\pgfsetdash{}{0pt}%
\pgfpathmoveto{\pgfqpoint{1.877366in}{4.608719in}}%
\pgfpathcurveto{\pgfqpoint{1.888416in}{4.608719in}}{\pgfqpoint{1.899015in}{4.613109in}}{\pgfqpoint{1.906829in}{4.620923in}}%
\pgfpathcurveto{\pgfqpoint{1.914643in}{4.628737in}}{\pgfqpoint{1.919033in}{4.639336in}}{\pgfqpoint{1.919033in}{4.650386in}}%
\pgfpathcurveto{\pgfqpoint{1.919033in}{4.661436in}}{\pgfqpoint{1.914643in}{4.672035in}}{\pgfqpoint{1.906829in}{4.679849in}}%
\pgfpathcurveto{\pgfqpoint{1.899015in}{4.687662in}}{\pgfqpoint{1.888416in}{4.692053in}}{\pgfqpoint{1.877366in}{4.692053in}}%
\pgfpathcurveto{\pgfqpoint{1.866316in}{4.692053in}}{\pgfqpoint{1.855717in}{4.687662in}}{\pgfqpoint{1.847903in}{4.679849in}}%
\pgfpathcurveto{\pgfqpoint{1.840090in}{4.672035in}}{\pgfqpoint{1.835700in}{4.661436in}}{\pgfqpoint{1.835700in}{4.650386in}}%
\pgfpathcurveto{\pgfqpoint{1.835700in}{4.639336in}}{\pgfqpoint{1.840090in}{4.628737in}}{\pgfqpoint{1.847903in}{4.620923in}}%
\pgfpathcurveto{\pgfqpoint{1.855717in}{4.613109in}}{\pgfqpoint{1.866316in}{4.608719in}}{\pgfqpoint{1.877366in}{4.608719in}}%
\pgfpathclose%
\pgfusepath{stroke,fill}%
\end{pgfscope}%
\begin{pgfscope}%
\pgfpathrectangle{\pgfqpoint{0.970666in}{4.121437in}}{\pgfqpoint{5.699255in}{2.685432in}}%
\pgfusepath{clip}%
\pgfsetbuttcap%
\pgfsetroundjoin%
\definecolor{currentfill}{rgb}{0.000000,0.000000,0.000000}%
\pgfsetfillcolor{currentfill}%
\pgfsetlinewidth{1.003750pt}%
\definecolor{currentstroke}{rgb}{0.000000,0.000000,0.000000}%
\pgfsetstrokecolor{currentstroke}%
\pgfsetdash{}{0pt}%
\pgfpathmoveto{\pgfqpoint{3.140269in}{4.467875in}}%
\pgfpathcurveto{\pgfqpoint{3.151320in}{4.467875in}}{\pgfqpoint{3.161919in}{4.472265in}}{\pgfqpoint{3.169732in}{4.480079in}}%
\pgfpathcurveto{\pgfqpoint{3.177546in}{4.487892in}}{\pgfqpoint{3.181936in}{4.498491in}}{\pgfqpoint{3.181936in}{4.509542in}}%
\pgfpathcurveto{\pgfqpoint{3.181936in}{4.520592in}}{\pgfqpoint{3.177546in}{4.531191in}}{\pgfqpoint{3.169732in}{4.539004in}}%
\pgfpathcurveto{\pgfqpoint{3.161919in}{4.546818in}}{\pgfqpoint{3.151320in}{4.551208in}}{\pgfqpoint{3.140269in}{4.551208in}}%
\pgfpathcurveto{\pgfqpoint{3.129219in}{4.551208in}}{\pgfqpoint{3.118620in}{4.546818in}}{\pgfqpoint{3.110807in}{4.539004in}}%
\pgfpathcurveto{\pgfqpoint{3.102993in}{4.531191in}}{\pgfqpoint{3.098603in}{4.520592in}}{\pgfqpoint{3.098603in}{4.509542in}}%
\pgfpathcurveto{\pgfqpoint{3.098603in}{4.498491in}}{\pgfqpoint{3.102993in}{4.487892in}}{\pgfqpoint{3.110807in}{4.480079in}}%
\pgfpathcurveto{\pgfqpoint{3.118620in}{4.472265in}}{\pgfqpoint{3.129219in}{4.467875in}}{\pgfqpoint{3.140269in}{4.467875in}}%
\pgfpathclose%
\pgfusepath{stroke,fill}%
\end{pgfscope}%
\begin{pgfscope}%
\pgfpathrectangle{\pgfqpoint{0.970666in}{4.121437in}}{\pgfqpoint{5.699255in}{2.685432in}}%
\pgfusepath{clip}%
\pgfsetbuttcap%
\pgfsetroundjoin%
\definecolor{currentfill}{rgb}{0.000000,0.000000,0.000000}%
\pgfsetfillcolor{currentfill}%
\pgfsetlinewidth{1.003750pt}%
\definecolor{currentstroke}{rgb}{0.000000,0.000000,0.000000}%
\pgfsetstrokecolor{currentstroke}%
\pgfsetdash{}{0pt}%
\pgfpathmoveto{\pgfqpoint{3.528855in}{5.250343in}}%
\pgfpathcurveto{\pgfqpoint{3.539905in}{5.250343in}}{\pgfqpoint{3.550504in}{5.254734in}}{\pgfqpoint{3.558318in}{5.262547in}}%
\pgfpathcurveto{\pgfqpoint{3.566131in}{5.270361in}}{\pgfqpoint{3.570522in}{5.280960in}}{\pgfqpoint{3.570522in}{5.292010in}}%
\pgfpathcurveto{\pgfqpoint{3.570522in}{5.303060in}}{\pgfqpoint{3.566131in}{5.313659in}}{\pgfqpoint{3.558318in}{5.321473in}}%
\pgfpathcurveto{\pgfqpoint{3.550504in}{5.329286in}}{\pgfqpoint{3.539905in}{5.333677in}}{\pgfqpoint{3.528855in}{5.333677in}}%
\pgfpathcurveto{\pgfqpoint{3.517805in}{5.333677in}}{\pgfqpoint{3.507206in}{5.329286in}}{\pgfqpoint{3.499392in}{5.321473in}}%
\pgfpathcurveto{\pgfqpoint{3.491579in}{5.313659in}}{\pgfqpoint{3.487188in}{5.303060in}}{\pgfqpoint{3.487188in}{5.292010in}}%
\pgfpathcurveto{\pgfqpoint{3.487188in}{5.280960in}}{\pgfqpoint{3.491579in}{5.270361in}}{\pgfqpoint{3.499392in}{5.262547in}}%
\pgfpathcurveto{\pgfqpoint{3.507206in}{5.254734in}}{\pgfqpoint{3.517805in}{5.250343in}}{\pgfqpoint{3.528855in}{5.250343in}}%
\pgfpathclose%
\pgfusepath{stroke,fill}%
\end{pgfscope}%
\begin{pgfscope}%
\pgfpathrectangle{\pgfqpoint{0.970666in}{4.121437in}}{\pgfqpoint{5.699255in}{2.685432in}}%
\pgfusepath{clip}%
\pgfsetbuttcap%
\pgfsetroundjoin%
\definecolor{currentfill}{rgb}{0.000000,0.000000,0.000000}%
\pgfsetfillcolor{currentfill}%
\pgfsetlinewidth{1.003750pt}%
\definecolor{currentstroke}{rgb}{0.000000,0.000000,0.000000}%
\pgfsetstrokecolor{currentstroke}%
\pgfsetdash{}{0pt}%
\pgfpathmoveto{\pgfqpoint{1.618309in}{5.907617in}}%
\pgfpathcurveto{\pgfqpoint{1.629359in}{5.907617in}}{\pgfqpoint{1.639958in}{5.912007in}}{\pgfqpoint{1.647772in}{5.919821in}}%
\pgfpathcurveto{\pgfqpoint{1.655586in}{5.927635in}}{\pgfqpoint{1.659976in}{5.938234in}}{\pgfqpoint{1.659976in}{5.949284in}}%
\pgfpathcurveto{\pgfqpoint{1.659976in}{5.960334in}}{\pgfqpoint{1.655586in}{5.970933in}}{\pgfqpoint{1.647772in}{5.978746in}}%
\pgfpathcurveto{\pgfqpoint{1.639958in}{5.986560in}}{\pgfqpoint{1.629359in}{5.990950in}}{\pgfqpoint{1.618309in}{5.990950in}}%
\pgfpathcurveto{\pgfqpoint{1.607259in}{5.990950in}}{\pgfqpoint{1.596660in}{5.986560in}}{\pgfqpoint{1.588846in}{5.978746in}}%
\pgfpathcurveto{\pgfqpoint{1.581033in}{5.970933in}}{\pgfqpoint{1.576642in}{5.960334in}}{\pgfqpoint{1.576642in}{5.949284in}}%
\pgfpathcurveto{\pgfqpoint{1.576642in}{5.938234in}}{\pgfqpoint{1.581033in}{5.927635in}}{\pgfqpoint{1.588846in}{5.919821in}}%
\pgfpathcurveto{\pgfqpoint{1.596660in}{5.912007in}}{\pgfqpoint{1.607259in}{5.907617in}}{\pgfqpoint{1.618309in}{5.907617in}}%
\pgfpathclose%
\pgfusepath{stroke,fill}%
\end{pgfscope}%
\begin{pgfscope}%
\pgfpathrectangle{\pgfqpoint{0.970666in}{4.121437in}}{\pgfqpoint{5.699255in}{2.685432in}}%
\pgfusepath{clip}%
\pgfsetbuttcap%
\pgfsetroundjoin%
\definecolor{currentfill}{rgb}{0.000000,0.000000,0.000000}%
\pgfsetfillcolor{currentfill}%
\pgfsetlinewidth{1.003750pt}%
\definecolor{currentstroke}{rgb}{0.000000,0.000000,0.000000}%
\pgfsetstrokecolor{currentstroke}%
\pgfsetdash{}{0pt}%
\pgfpathmoveto{\pgfqpoint{3.528855in}{6.283202in}}%
\pgfpathcurveto{\pgfqpoint{3.539905in}{6.283202in}}{\pgfqpoint{3.550504in}{6.287592in}}{\pgfqpoint{3.558318in}{6.295406in}}%
\pgfpathcurveto{\pgfqpoint{3.566131in}{6.303219in}}{\pgfqpoint{3.570522in}{6.313818in}}{\pgfqpoint{3.570522in}{6.324869in}}%
\pgfpathcurveto{\pgfqpoint{3.570522in}{6.335919in}}{\pgfqpoint{3.566131in}{6.346518in}}{\pgfqpoint{3.558318in}{6.354331in}}%
\pgfpathcurveto{\pgfqpoint{3.550504in}{6.362145in}}{\pgfqpoint{3.539905in}{6.366535in}}{\pgfqpoint{3.528855in}{6.366535in}}%
\pgfpathcurveto{\pgfqpoint{3.517805in}{6.366535in}}{\pgfqpoint{3.507206in}{6.362145in}}{\pgfqpoint{3.499392in}{6.354331in}}%
\pgfpathcurveto{\pgfqpoint{3.491579in}{6.346518in}}{\pgfqpoint{3.487188in}{6.335919in}}{\pgfqpoint{3.487188in}{6.324869in}}%
\pgfpathcurveto{\pgfqpoint{3.487188in}{6.313818in}}{\pgfqpoint{3.491579in}{6.303219in}}{\pgfqpoint{3.499392in}{6.295406in}}%
\pgfpathcurveto{\pgfqpoint{3.507206in}{6.287592in}}{\pgfqpoint{3.517805in}{6.283202in}}{\pgfqpoint{3.528855in}{6.283202in}}%
\pgfpathclose%
\pgfusepath{stroke,fill}%
\end{pgfscope}%
\begin{pgfscope}%
\pgfpathrectangle{\pgfqpoint{0.970666in}{4.121437in}}{\pgfqpoint{5.699255in}{2.685432in}}%
\pgfusepath{clip}%
\pgfsetbuttcap%
\pgfsetroundjoin%
\definecolor{currentfill}{rgb}{0.000000,0.000000,0.000000}%
\pgfsetfillcolor{currentfill}%
\pgfsetlinewidth{1.003750pt}%
\definecolor{currentstroke}{rgb}{0.000000,0.000000,0.000000}%
\pgfsetstrokecolor{currentstroke}%
\pgfsetdash{}{0pt}%
\pgfpathmoveto{\pgfqpoint{4.144116in}{5.970214in}}%
\pgfpathcurveto{\pgfqpoint{4.155166in}{5.970214in}}{\pgfqpoint{4.165765in}{5.974605in}}{\pgfqpoint{4.173578in}{5.982418in}}%
\pgfpathcurveto{\pgfqpoint{4.181392in}{5.990232in}}{\pgfqpoint{4.185782in}{6.000831in}}{\pgfqpoint{4.185782in}{6.011881in}}%
\pgfpathcurveto{\pgfqpoint{4.185782in}{6.022931in}}{\pgfqpoint{4.181392in}{6.033530in}}{\pgfqpoint{4.173578in}{6.041344in}}%
\pgfpathcurveto{\pgfqpoint{4.165765in}{6.049158in}}{\pgfqpoint{4.155166in}{6.053548in}}{\pgfqpoint{4.144116in}{6.053548in}}%
\pgfpathcurveto{\pgfqpoint{4.133065in}{6.053548in}}{\pgfqpoint{4.122466in}{6.049158in}}{\pgfqpoint{4.114653in}{6.041344in}}%
\pgfpathcurveto{\pgfqpoint{4.106839in}{6.033530in}}{\pgfqpoint{4.102449in}{6.022931in}}{\pgfqpoint{4.102449in}{6.011881in}}%
\pgfpathcurveto{\pgfqpoint{4.102449in}{6.000831in}}{\pgfqpoint{4.106839in}{5.990232in}}{\pgfqpoint{4.114653in}{5.982418in}}%
\pgfpathcurveto{\pgfqpoint{4.122466in}{5.974605in}}{\pgfqpoint{4.133065in}{5.970214in}}{\pgfqpoint{4.144116in}{5.970214in}}%
\pgfpathclose%
\pgfusepath{stroke,fill}%
\end{pgfscope}%
\begin{pgfscope}%
\pgfpathrectangle{\pgfqpoint{0.970666in}{4.121437in}}{\pgfqpoint{5.699255in}{2.685432in}}%
\pgfusepath{clip}%
\pgfsetbuttcap%
\pgfsetroundjoin%
\definecolor{currentfill}{rgb}{0.000000,0.000000,0.000000}%
\pgfsetfillcolor{currentfill}%
\pgfsetlinewidth{1.003750pt}%
\definecolor{currentstroke}{rgb}{0.000000,0.000000,0.000000}%
\pgfsetstrokecolor{currentstroke}%
\pgfsetdash{}{0pt}%
\pgfpathmoveto{\pgfqpoint{5.115579in}{6.267553in}}%
\pgfpathcurveto{\pgfqpoint{5.126630in}{6.267553in}}{\pgfqpoint{5.137229in}{6.271943in}}{\pgfqpoint{5.145042in}{6.279756in}}%
\pgfpathcurveto{\pgfqpoint{5.152856in}{6.287570in}}{\pgfqpoint{5.157246in}{6.298169in}}{\pgfqpoint{5.157246in}{6.309219in}}%
\pgfpathcurveto{\pgfqpoint{5.157246in}{6.320269in}}{\pgfqpoint{5.152856in}{6.330868in}}{\pgfqpoint{5.145042in}{6.338682in}}%
\pgfpathcurveto{\pgfqpoint{5.137229in}{6.346496in}}{\pgfqpoint{5.126630in}{6.350886in}}{\pgfqpoint{5.115579in}{6.350886in}}%
\pgfpathcurveto{\pgfqpoint{5.104529in}{6.350886in}}{\pgfqpoint{5.093930in}{6.346496in}}{\pgfqpoint{5.086117in}{6.338682in}}%
\pgfpathcurveto{\pgfqpoint{5.078303in}{6.330868in}}{\pgfqpoint{5.073913in}{6.320269in}}{\pgfqpoint{5.073913in}{6.309219in}}%
\pgfpathcurveto{\pgfqpoint{5.073913in}{6.298169in}}{\pgfqpoint{5.078303in}{6.287570in}}{\pgfqpoint{5.086117in}{6.279756in}}%
\pgfpathcurveto{\pgfqpoint{5.093930in}{6.271943in}}{\pgfqpoint{5.104529in}{6.267553in}}{\pgfqpoint{5.115579in}{6.267553in}}%
\pgfpathclose%
\pgfusepath{stroke,fill}%
\end{pgfscope}%
\begin{pgfscope}%
\pgfpathrectangle{\pgfqpoint{0.970666in}{4.121437in}}{\pgfqpoint{5.699255in}{2.685432in}}%
\pgfusepath{clip}%
\pgfsetbuttcap%
\pgfsetroundjoin%
\definecolor{currentfill}{rgb}{0.000000,0.000000,0.000000}%
\pgfsetfillcolor{currentfill}%
\pgfsetlinewidth{1.003750pt}%
\definecolor{currentstroke}{rgb}{0.000000,0.000000,0.000000}%
\pgfsetstrokecolor{currentstroke}%
\pgfsetdash{}{0pt}%
\pgfpathmoveto{\pgfqpoint{2.104041in}{6.627488in}}%
\pgfpathcurveto{\pgfqpoint{2.115091in}{6.627488in}}{\pgfqpoint{2.125690in}{6.631878in}}{\pgfqpoint{2.133504in}{6.639692in}}%
\pgfpathcurveto{\pgfqpoint{2.141318in}{6.647506in}}{\pgfqpoint{2.145708in}{6.658105in}}{\pgfqpoint{2.145708in}{6.669155in}}%
\pgfpathcurveto{\pgfqpoint{2.145708in}{6.680205in}}{\pgfqpoint{2.141318in}{6.690804in}}{\pgfqpoint{2.133504in}{6.698618in}}%
\pgfpathcurveto{\pgfqpoint{2.125690in}{6.706431in}}{\pgfqpoint{2.115091in}{6.710821in}}{\pgfqpoint{2.104041in}{6.710821in}}%
\pgfpathcurveto{\pgfqpoint{2.092991in}{6.710821in}}{\pgfqpoint{2.082392in}{6.706431in}}{\pgfqpoint{2.074578in}{6.698618in}}%
\pgfpathcurveto{\pgfqpoint{2.066765in}{6.690804in}}{\pgfqpoint{2.062374in}{6.680205in}}{\pgfqpoint{2.062374in}{6.669155in}}%
\pgfpathcurveto{\pgfqpoint{2.062374in}{6.658105in}}{\pgfqpoint{2.066765in}{6.647506in}}{\pgfqpoint{2.074578in}{6.639692in}}%
\pgfpathcurveto{\pgfqpoint{2.082392in}{6.631878in}}{\pgfqpoint{2.092991in}{6.627488in}}{\pgfqpoint{2.104041in}{6.627488in}}%
\pgfpathclose%
\pgfusepath{stroke,fill}%
\end{pgfscope}%
\begin{pgfscope}%
\pgfpathrectangle{\pgfqpoint{0.970666in}{4.121437in}}{\pgfqpoint{5.699255in}{2.685432in}}%
\pgfusepath{clip}%
\pgfsetbuttcap%
\pgfsetroundjoin%
\definecolor{currentfill}{rgb}{0.000000,0.000000,0.000000}%
\pgfsetfillcolor{currentfill}%
\pgfsetlinewidth{1.003750pt}%
\definecolor{currentstroke}{rgb}{0.000000,0.000000,0.000000}%
\pgfsetstrokecolor{currentstroke}%
\pgfsetdash{}{0pt}%
\pgfpathmoveto{\pgfqpoint{2.201188in}{6.455345in}}%
\pgfpathcurveto{\pgfqpoint{2.212238in}{6.455345in}}{\pgfqpoint{2.222837in}{6.459735in}}{\pgfqpoint{2.230650in}{6.467549in}}%
\pgfpathcurveto{\pgfqpoint{2.238464in}{6.475363in}}{\pgfqpoint{2.242854in}{6.485962in}}{\pgfqpoint{2.242854in}{6.497012in}}%
\pgfpathcurveto{\pgfqpoint{2.242854in}{6.508062in}}{\pgfqpoint{2.238464in}{6.518661in}}{\pgfqpoint{2.230650in}{6.526474in}}%
\pgfpathcurveto{\pgfqpoint{2.222837in}{6.534288in}}{\pgfqpoint{2.212238in}{6.538678in}}{\pgfqpoint{2.201188in}{6.538678in}}%
\pgfpathcurveto{\pgfqpoint{2.190137in}{6.538678in}}{\pgfqpoint{2.179538in}{6.534288in}}{\pgfqpoint{2.171725in}{6.526474in}}%
\pgfpathcurveto{\pgfqpoint{2.163911in}{6.518661in}}{\pgfqpoint{2.159521in}{6.508062in}}{\pgfqpoint{2.159521in}{6.497012in}}%
\pgfpathcurveto{\pgfqpoint{2.159521in}{6.485962in}}{\pgfqpoint{2.163911in}{6.475363in}}{\pgfqpoint{2.171725in}{6.467549in}}%
\pgfpathcurveto{\pgfqpoint{2.179538in}{6.459735in}}{\pgfqpoint{2.190137in}{6.455345in}}{\pgfqpoint{2.201188in}{6.455345in}}%
\pgfpathclose%
\pgfusepath{stroke,fill}%
\end{pgfscope}%
\begin{pgfscope}%
\pgfpathrectangle{\pgfqpoint{0.970666in}{4.121437in}}{\pgfqpoint{5.699255in}{2.685432in}}%
\pgfusepath{clip}%
\pgfsetbuttcap%
\pgfsetroundjoin%
\definecolor{currentfill}{rgb}{0.000000,0.000000,0.000000}%
\pgfsetfillcolor{currentfill}%
\pgfsetlinewidth{1.003750pt}%
\definecolor{currentstroke}{rgb}{0.000000,0.000000,0.000000}%
\pgfsetstrokecolor{currentstroke}%
\pgfsetdash{}{0pt}%
\pgfpathmoveto{\pgfqpoint{1.747838in}{4.577420in}}%
\pgfpathcurveto{\pgfqpoint{1.758888in}{4.577420in}}{\pgfqpoint{1.769487in}{4.581811in}}{\pgfqpoint{1.777300in}{4.589624in}}%
\pgfpathcurveto{\pgfqpoint{1.785114in}{4.597438in}}{\pgfqpoint{1.789504in}{4.608037in}}{\pgfqpoint{1.789504in}{4.619087in}}%
\pgfpathcurveto{\pgfqpoint{1.789504in}{4.630137in}}{\pgfqpoint{1.785114in}{4.640736in}}{\pgfqpoint{1.777300in}{4.648550in}}%
\pgfpathcurveto{\pgfqpoint{1.769487in}{4.656364in}}{\pgfqpoint{1.758888in}{4.660754in}}{\pgfqpoint{1.747838in}{4.660754in}}%
\pgfpathcurveto{\pgfqpoint{1.736788in}{4.660754in}}{\pgfqpoint{1.726189in}{4.656364in}}{\pgfqpoint{1.718375in}{4.648550in}}%
\pgfpathcurveto{\pgfqpoint{1.710561in}{4.640736in}}{\pgfqpoint{1.706171in}{4.630137in}}{\pgfqpoint{1.706171in}{4.619087in}}%
\pgfpathcurveto{\pgfqpoint{1.706171in}{4.608037in}}{\pgfqpoint{1.710561in}{4.597438in}}{\pgfqpoint{1.718375in}{4.589624in}}%
\pgfpathcurveto{\pgfqpoint{1.726189in}{4.581811in}}{\pgfqpoint{1.736788in}{4.577420in}}{\pgfqpoint{1.747838in}{4.577420in}}%
\pgfpathclose%
\pgfusepath{stroke,fill}%
\end{pgfscope}%
\begin{pgfscope}%
\pgfpathrectangle{\pgfqpoint{0.970666in}{4.121437in}}{\pgfqpoint{5.699255in}{2.685432in}}%
\pgfusepath{clip}%
\pgfsetbuttcap%
\pgfsetroundjoin%
\definecolor{currentfill}{rgb}{0.000000,0.000000,0.000000}%
\pgfsetfillcolor{currentfill}%
\pgfsetlinewidth{1.003750pt}%
\definecolor{currentstroke}{rgb}{0.000000,0.000000,0.000000}%
\pgfsetstrokecolor{currentstroke}%
\pgfsetdash{}{0pt}%
\pgfpathmoveto{\pgfqpoint{5.925133in}{5.923266in}}%
\pgfpathcurveto{\pgfqpoint{5.936183in}{5.923266in}}{\pgfqpoint{5.946782in}{5.927657in}}{\pgfqpoint{5.954596in}{5.935470in}}%
\pgfpathcurveto{\pgfqpoint{5.962409in}{5.943284in}}{\pgfqpoint{5.966799in}{5.953883in}}{\pgfqpoint{5.966799in}{5.964933in}}%
\pgfpathcurveto{\pgfqpoint{5.966799in}{5.975983in}}{\pgfqpoint{5.962409in}{5.986582in}}{\pgfqpoint{5.954596in}{5.994396in}}%
\pgfpathcurveto{\pgfqpoint{5.946782in}{6.002209in}}{\pgfqpoint{5.936183in}{6.006600in}}{\pgfqpoint{5.925133in}{6.006600in}}%
\pgfpathcurveto{\pgfqpoint{5.914083in}{6.006600in}}{\pgfqpoint{5.903484in}{6.002209in}}{\pgfqpoint{5.895670in}{5.994396in}}%
\pgfpathcurveto{\pgfqpoint{5.887856in}{5.986582in}}{\pgfqpoint{5.883466in}{5.975983in}}{\pgfqpoint{5.883466in}{5.964933in}}%
\pgfpathcurveto{\pgfqpoint{5.883466in}{5.953883in}}{\pgfqpoint{5.887856in}{5.943284in}}{\pgfqpoint{5.895670in}{5.935470in}}%
\pgfpathcurveto{\pgfqpoint{5.903484in}{5.927657in}}{\pgfqpoint{5.914083in}{5.923266in}}{\pgfqpoint{5.925133in}{5.923266in}}%
\pgfpathclose%
\pgfusepath{stroke,fill}%
\end{pgfscope}%
\begin{pgfscope}%
\pgfpathrectangle{\pgfqpoint{0.970666in}{4.121437in}}{\pgfqpoint{5.699255in}{2.685432in}}%
\pgfusepath{clip}%
\pgfsetbuttcap%
\pgfsetroundjoin%
\definecolor{currentfill}{rgb}{0.000000,0.000000,0.000000}%
\pgfsetfillcolor{currentfill}%
\pgfsetlinewidth{1.003750pt}%
\definecolor{currentstroke}{rgb}{0.000000,0.000000,0.000000}%
\pgfsetstrokecolor{currentstroke}%
\pgfsetdash{}{0pt}%
\pgfpathmoveto{\pgfqpoint{5.471783in}{4.671317in}}%
\pgfpathcurveto{\pgfqpoint{5.482833in}{4.671317in}}{\pgfqpoint{5.493432in}{4.675707in}}{\pgfqpoint{5.501246in}{4.683521in}}%
\pgfpathcurveto{\pgfqpoint{5.509059in}{4.691334in}}{\pgfqpoint{5.513450in}{4.701933in}}{\pgfqpoint{5.513450in}{4.712983in}}%
\pgfpathcurveto{\pgfqpoint{5.513450in}{4.724033in}}{\pgfqpoint{5.509059in}{4.734633in}}{\pgfqpoint{5.501246in}{4.742446in}}%
\pgfpathcurveto{\pgfqpoint{5.493432in}{4.750260in}}{\pgfqpoint{5.482833in}{4.754650in}}{\pgfqpoint{5.471783in}{4.754650in}}%
\pgfpathcurveto{\pgfqpoint{5.460733in}{4.754650in}}{\pgfqpoint{5.450134in}{4.750260in}}{\pgfqpoint{5.442320in}{4.742446in}}%
\pgfpathcurveto{\pgfqpoint{5.434507in}{4.734633in}}{\pgfqpoint{5.430116in}{4.724033in}}{\pgfqpoint{5.430116in}{4.712983in}}%
\pgfpathcurveto{\pgfqpoint{5.430116in}{4.701933in}}{\pgfqpoint{5.434507in}{4.691334in}}{\pgfqpoint{5.442320in}{4.683521in}}%
\pgfpathcurveto{\pgfqpoint{5.450134in}{4.675707in}}{\pgfqpoint{5.460733in}{4.671317in}}{\pgfqpoint{5.471783in}{4.671317in}}%
\pgfpathclose%
\pgfusepath{stroke,fill}%
\end{pgfscope}%
\begin{pgfscope}%
\pgfpathrectangle{\pgfqpoint{0.970666in}{4.121437in}}{\pgfqpoint{5.699255in}{2.685432in}}%
\pgfusepath{clip}%
\pgfsetbuttcap%
\pgfsetroundjoin%
\definecolor{currentfill}{rgb}{0.000000,0.000000,0.000000}%
\pgfsetfillcolor{currentfill}%
\pgfsetlinewidth{1.003750pt}%
\definecolor{currentstroke}{rgb}{0.000000,0.000000,0.000000}%
\pgfsetstrokecolor{currentstroke}%
\pgfsetdash{}{0pt}%
\pgfpathmoveto{\pgfqpoint{5.309872in}{4.201836in}}%
\pgfpathcurveto{\pgfqpoint{5.320922in}{4.201836in}}{\pgfqpoint{5.331521in}{4.206226in}}{\pgfqpoint{5.339335in}{4.214039in}}%
\pgfpathcurveto{\pgfqpoint{5.347149in}{4.221853in}}{\pgfqpoint{5.351539in}{4.232452in}}{\pgfqpoint{5.351539in}{4.243502in}}%
\pgfpathcurveto{\pgfqpoint{5.351539in}{4.254552in}}{\pgfqpoint{5.347149in}{4.265151in}}{\pgfqpoint{5.339335in}{4.272965in}}%
\pgfpathcurveto{\pgfqpoint{5.331521in}{4.280779in}}{\pgfqpoint{5.320922in}{4.285169in}}{\pgfqpoint{5.309872in}{4.285169in}}%
\pgfpathcurveto{\pgfqpoint{5.298822in}{4.285169in}}{\pgfqpoint{5.288223in}{4.280779in}}{\pgfqpoint{5.280410in}{4.272965in}}%
\pgfpathcurveto{\pgfqpoint{5.272596in}{4.265151in}}{\pgfqpoint{5.268206in}{4.254552in}}{\pgfqpoint{5.268206in}{4.243502in}}%
\pgfpathcurveto{\pgfqpoint{5.268206in}{4.232452in}}{\pgfqpoint{5.272596in}{4.221853in}}{\pgfqpoint{5.280410in}{4.214039in}}%
\pgfpathcurveto{\pgfqpoint{5.288223in}{4.206226in}}{\pgfqpoint{5.298822in}{4.201836in}}{\pgfqpoint{5.309872in}{4.201836in}}%
\pgfpathclose%
\pgfusepath{stroke,fill}%
\end{pgfscope}%
\begin{pgfscope}%
\pgfpathrectangle{\pgfqpoint{0.970666in}{4.121437in}}{\pgfqpoint{5.699255in}{2.685432in}}%
\pgfusepath{clip}%
\pgfsetbuttcap%
\pgfsetroundjoin%
\definecolor{currentfill}{rgb}{0.000000,0.000000,0.000000}%
\pgfsetfillcolor{currentfill}%
\pgfsetlinewidth{1.003750pt}%
\definecolor{currentstroke}{rgb}{0.000000,0.000000,0.000000}%
\pgfsetstrokecolor{currentstroke}%
\pgfsetdash{}{0pt}%
\pgfpathmoveto{\pgfqpoint{3.949823in}{4.311381in}}%
\pgfpathcurveto{\pgfqpoint{3.960873in}{4.311381in}}{\pgfqpoint{3.971472in}{4.315771in}}{\pgfqpoint{3.979285in}{4.323585in}}%
\pgfpathcurveto{\pgfqpoint{3.987099in}{4.331399in}}{\pgfqpoint{3.991489in}{4.341998in}}{\pgfqpoint{3.991489in}{4.353048in}}%
\pgfpathcurveto{\pgfqpoint{3.991489in}{4.364098in}}{\pgfqpoint{3.987099in}{4.374697in}}{\pgfqpoint{3.979285in}{4.382511in}}%
\pgfpathcurveto{\pgfqpoint{3.971472in}{4.390324in}}{\pgfqpoint{3.960873in}{4.394714in}}{\pgfqpoint{3.949823in}{4.394714in}}%
\pgfpathcurveto{\pgfqpoint{3.938773in}{4.394714in}}{\pgfqpoint{3.928174in}{4.390324in}}{\pgfqpoint{3.920360in}{4.382511in}}%
\pgfpathcurveto{\pgfqpoint{3.912546in}{4.374697in}}{\pgfqpoint{3.908156in}{4.364098in}}{\pgfqpoint{3.908156in}{4.353048in}}%
\pgfpathcurveto{\pgfqpoint{3.908156in}{4.341998in}}{\pgfqpoint{3.912546in}{4.331399in}}{\pgfqpoint{3.920360in}{4.323585in}}%
\pgfpathcurveto{\pgfqpoint{3.928174in}{4.315771in}}{\pgfqpoint{3.938773in}{4.311381in}}{\pgfqpoint{3.949823in}{4.311381in}}%
\pgfpathclose%
\pgfusepath{stroke,fill}%
\end{pgfscope}%
\begin{pgfscope}%
\pgfpathrectangle{\pgfqpoint{0.970666in}{4.121437in}}{\pgfqpoint{5.699255in}{2.685432in}}%
\pgfusepath{clip}%
\pgfsetbuttcap%
\pgfsetroundjoin%
\definecolor{currentfill}{rgb}{0.000000,0.000000,0.000000}%
\pgfsetfillcolor{currentfill}%
\pgfsetlinewidth{1.003750pt}%
\definecolor{currentstroke}{rgb}{0.000000,0.000000,0.000000}%
\pgfsetstrokecolor{currentstroke}%
\pgfsetdash{}{0pt}%
\pgfpathmoveto{\pgfqpoint{2.201188in}{5.610279in}}%
\pgfpathcurveto{\pgfqpoint{2.212238in}{5.610279in}}{\pgfqpoint{2.222837in}{5.614669in}}{\pgfqpoint{2.230650in}{5.622483in}}%
\pgfpathcurveto{\pgfqpoint{2.238464in}{5.630296in}}{\pgfqpoint{2.242854in}{5.640895in}}{\pgfqpoint{2.242854in}{5.651946in}}%
\pgfpathcurveto{\pgfqpoint{2.242854in}{5.662996in}}{\pgfqpoint{2.238464in}{5.673595in}}{\pgfqpoint{2.230650in}{5.681408in}}%
\pgfpathcurveto{\pgfqpoint{2.222837in}{5.689222in}}{\pgfqpoint{2.212238in}{5.693612in}}{\pgfqpoint{2.201188in}{5.693612in}}%
\pgfpathcurveto{\pgfqpoint{2.190137in}{5.693612in}}{\pgfqpoint{2.179538in}{5.689222in}}{\pgfqpoint{2.171725in}{5.681408in}}%
\pgfpathcurveto{\pgfqpoint{2.163911in}{5.673595in}}{\pgfqpoint{2.159521in}{5.662996in}}{\pgfqpoint{2.159521in}{5.651946in}}%
\pgfpathcurveto{\pgfqpoint{2.159521in}{5.640895in}}{\pgfqpoint{2.163911in}{5.630296in}}{\pgfqpoint{2.171725in}{5.622483in}}%
\pgfpathcurveto{\pgfqpoint{2.179538in}{5.614669in}}{\pgfqpoint{2.190137in}{5.610279in}}{\pgfqpoint{2.201188in}{5.610279in}}%
\pgfpathclose%
\pgfusepath{stroke,fill}%
\end{pgfscope}%
\begin{pgfscope}%
\pgfpathrectangle{\pgfqpoint{0.970666in}{4.121437in}}{\pgfqpoint{5.699255in}{2.685432in}}%
\pgfusepath{clip}%
\pgfsetbuttcap%
\pgfsetroundjoin%
\definecolor{currentfill}{rgb}{0.000000,0.000000,0.000000}%
\pgfsetfillcolor{currentfill}%
\pgfsetlinewidth{1.003750pt}%
\definecolor{currentstroke}{rgb}{0.000000,0.000000,0.000000}%
\pgfsetstrokecolor{currentstroke}%
\pgfsetdash{}{0pt}%
\pgfpathmoveto{\pgfqpoint{6.151808in}{6.392748in}}%
\pgfpathcurveto{\pgfqpoint{6.162858in}{6.392748in}}{\pgfqpoint{6.173457in}{6.397138in}}{\pgfqpoint{6.181271in}{6.404951in}}%
\pgfpathcurveto{\pgfqpoint{6.189084in}{6.412765in}}{\pgfqpoint{6.193474in}{6.423364in}}{\pgfqpoint{6.193474in}{6.434414in}}%
\pgfpathcurveto{\pgfqpoint{6.193474in}{6.445464in}}{\pgfqpoint{6.189084in}{6.456063in}}{\pgfqpoint{6.181271in}{6.463877in}}%
\pgfpathcurveto{\pgfqpoint{6.173457in}{6.471691in}}{\pgfqpoint{6.162858in}{6.476081in}}{\pgfqpoint{6.151808in}{6.476081in}}%
\pgfpathcurveto{\pgfqpoint{6.140758in}{6.476081in}}{\pgfqpoint{6.130159in}{6.471691in}}{\pgfqpoint{6.122345in}{6.463877in}}%
\pgfpathcurveto{\pgfqpoint{6.114531in}{6.456063in}}{\pgfqpoint{6.110141in}{6.445464in}}{\pgfqpoint{6.110141in}{6.434414in}}%
\pgfpathcurveto{\pgfqpoint{6.110141in}{6.423364in}}{\pgfqpoint{6.114531in}{6.412765in}}{\pgfqpoint{6.122345in}{6.404951in}}%
\pgfpathcurveto{\pgfqpoint{6.130159in}{6.397138in}}{\pgfqpoint{6.140758in}{6.392748in}}{\pgfqpoint{6.151808in}{6.392748in}}%
\pgfpathclose%
\pgfusepath{stroke,fill}%
\end{pgfscope}%
\begin{pgfscope}%
\pgfpathrectangle{\pgfqpoint{0.970666in}{4.121437in}}{\pgfqpoint{5.699255in}{2.685432in}}%
\pgfusepath{clip}%
\pgfsetbuttcap%
\pgfsetroundjoin%
\definecolor{currentfill}{rgb}{0.000000,0.000000,0.000000}%
\pgfsetfillcolor{currentfill}%
\pgfsetlinewidth{1.003750pt}%
\definecolor{currentstroke}{rgb}{0.000000,0.000000,0.000000}%
\pgfsetstrokecolor{currentstroke}%
\pgfsetdash{}{0pt}%
\pgfpathmoveto{\pgfqpoint{5.763222in}{6.580540in}}%
\pgfpathcurveto{\pgfqpoint{5.774272in}{6.580540in}}{\pgfqpoint{5.784871in}{6.584930in}}{\pgfqpoint{5.792685in}{6.592744in}}%
\pgfpathcurveto{\pgfqpoint{5.800499in}{6.600557in}}{\pgfqpoint{5.804889in}{6.611157in}}{\pgfqpoint{5.804889in}{6.622207in}}%
\pgfpathcurveto{\pgfqpoint{5.804889in}{6.633257in}}{\pgfqpoint{5.800499in}{6.643856in}}{\pgfqpoint{5.792685in}{6.651669in}}%
\pgfpathcurveto{\pgfqpoint{5.784871in}{6.659483in}}{\pgfqpoint{5.774272in}{6.663873in}}{\pgfqpoint{5.763222in}{6.663873in}}%
\pgfpathcurveto{\pgfqpoint{5.752172in}{6.663873in}}{\pgfqpoint{5.741573in}{6.659483in}}{\pgfqpoint{5.733759in}{6.651669in}}%
\pgfpathcurveto{\pgfqpoint{5.725946in}{6.643856in}}{\pgfqpoint{5.721555in}{6.633257in}}{\pgfqpoint{5.721555in}{6.622207in}}%
\pgfpathcurveto{\pgfqpoint{5.721555in}{6.611157in}}{\pgfqpoint{5.725946in}{6.600557in}}{\pgfqpoint{5.733759in}{6.592744in}}%
\pgfpathcurveto{\pgfqpoint{5.741573in}{6.584930in}}{\pgfqpoint{5.752172in}{6.580540in}}{\pgfqpoint{5.763222in}{6.580540in}}%
\pgfpathclose%
\pgfusepath{stroke,fill}%
\end{pgfscope}%
\begin{pgfscope}%
\pgfpathrectangle{\pgfqpoint{0.970666in}{4.121437in}}{\pgfqpoint{5.699255in}{2.685432in}}%
\pgfusepath{clip}%
\pgfsetbuttcap%
\pgfsetroundjoin%
\definecolor{currentfill}{rgb}{0.000000,0.000000,0.000000}%
\pgfsetfillcolor{currentfill}%
\pgfsetlinewidth{1.003750pt}%
\definecolor{currentstroke}{rgb}{0.000000,0.000000,0.000000}%
\pgfsetstrokecolor{currentstroke}%
\pgfsetdash{}{0pt}%
\pgfpathmoveto{\pgfqpoint{2.201188in}{6.502293in}}%
\pgfpathcurveto{\pgfqpoint{2.212238in}{6.502293in}}{\pgfqpoint{2.222837in}{6.506683in}}{\pgfqpoint{2.230650in}{6.514497in}}%
\pgfpathcurveto{\pgfqpoint{2.238464in}{6.522311in}}{\pgfqpoint{2.242854in}{6.532910in}}{\pgfqpoint{2.242854in}{6.543960in}}%
\pgfpathcurveto{\pgfqpoint{2.242854in}{6.555010in}}{\pgfqpoint{2.238464in}{6.565609in}}{\pgfqpoint{2.230650in}{6.573423in}}%
\pgfpathcurveto{\pgfqpoint{2.222837in}{6.581236in}}{\pgfqpoint{2.212238in}{6.585626in}}{\pgfqpoint{2.201188in}{6.585626in}}%
\pgfpathcurveto{\pgfqpoint{2.190137in}{6.585626in}}{\pgfqpoint{2.179538in}{6.581236in}}{\pgfqpoint{2.171725in}{6.573423in}}%
\pgfpathcurveto{\pgfqpoint{2.163911in}{6.565609in}}{\pgfqpoint{2.159521in}{6.555010in}}{\pgfqpoint{2.159521in}{6.543960in}}%
\pgfpathcurveto{\pgfqpoint{2.159521in}{6.532910in}}{\pgfqpoint{2.163911in}{6.522311in}}{\pgfqpoint{2.171725in}{6.514497in}}%
\pgfpathcurveto{\pgfqpoint{2.179538in}{6.506683in}}{\pgfqpoint{2.190137in}{6.502293in}}{\pgfqpoint{2.201188in}{6.502293in}}%
\pgfpathclose%
\pgfusepath{stroke,fill}%
\end{pgfscope}%
\begin{pgfscope}%
\pgfpathrectangle{\pgfqpoint{0.970666in}{4.121437in}}{\pgfqpoint{5.699255in}{2.685432in}}%
\pgfusepath{clip}%
\pgfsetbuttcap%
\pgfsetroundjoin%
\definecolor{currentfill}{rgb}{0.000000,0.000000,0.000000}%
\pgfsetfillcolor{currentfill}%
\pgfsetlinewidth{1.003750pt}%
\definecolor{currentstroke}{rgb}{0.000000,0.000000,0.000000}%
\pgfsetstrokecolor{currentstroke}%
\pgfsetdash{}{0pt}%
\pgfpathmoveto{\pgfqpoint{1.650691in}{4.452225in}}%
\pgfpathcurveto{\pgfqpoint{1.661741in}{4.452225in}}{\pgfqpoint{1.672340in}{4.456616in}}{\pgfqpoint{1.680154in}{4.464429in}}%
\pgfpathcurveto{\pgfqpoint{1.687968in}{4.472243in}}{\pgfqpoint{1.692358in}{4.482842in}}{\pgfqpoint{1.692358in}{4.493892in}}%
\pgfpathcurveto{\pgfqpoint{1.692358in}{4.504942in}}{\pgfqpoint{1.687968in}{4.515541in}}{\pgfqpoint{1.680154in}{4.523355in}}%
\pgfpathcurveto{\pgfqpoint{1.672340in}{4.531169in}}{\pgfqpoint{1.661741in}{4.535559in}}{\pgfqpoint{1.650691in}{4.535559in}}%
\pgfpathcurveto{\pgfqpoint{1.639641in}{4.535559in}}{\pgfqpoint{1.629042in}{4.531169in}}{\pgfqpoint{1.621228in}{4.523355in}}%
\pgfpathcurveto{\pgfqpoint{1.613415in}{4.515541in}}{\pgfqpoint{1.609025in}{4.504942in}}{\pgfqpoint{1.609025in}{4.493892in}}%
\pgfpathcurveto{\pgfqpoint{1.609025in}{4.482842in}}{\pgfqpoint{1.613415in}{4.472243in}}{\pgfqpoint{1.621228in}{4.464429in}}%
\pgfpathcurveto{\pgfqpoint{1.629042in}{4.456616in}}{\pgfqpoint{1.639641in}{4.452225in}}{\pgfqpoint{1.650691in}{4.452225in}}%
\pgfpathclose%
\pgfusepath{stroke,fill}%
\end{pgfscope}%
\begin{pgfscope}%
\pgfpathrectangle{\pgfqpoint{0.970666in}{4.121437in}}{\pgfqpoint{5.699255in}{2.685432in}}%
\pgfusepath{clip}%
\pgfsetbuttcap%
\pgfsetroundjoin%
\definecolor{currentfill}{rgb}{0.000000,0.000000,0.000000}%
\pgfsetfillcolor{currentfill}%
\pgfsetlinewidth{1.003750pt}%
\definecolor{currentstroke}{rgb}{0.000000,0.000000,0.000000}%
\pgfsetstrokecolor{currentstroke}%
\pgfsetdash{}{0pt}%
\pgfpathmoveto{\pgfqpoint{3.593619in}{4.733914in}}%
\pgfpathcurveto{\pgfqpoint{3.604669in}{4.733914in}}{\pgfqpoint{3.615268in}{4.738304in}}{\pgfqpoint{3.623082in}{4.746118in}}%
\pgfpathcurveto{\pgfqpoint{3.630896in}{4.753932in}}{\pgfqpoint{3.635286in}{4.764531in}}{\pgfqpoint{3.635286in}{4.775581in}}%
\pgfpathcurveto{\pgfqpoint{3.635286in}{4.786631in}}{\pgfqpoint{3.630896in}{4.797230in}}{\pgfqpoint{3.623082in}{4.805044in}}%
\pgfpathcurveto{\pgfqpoint{3.615268in}{4.812857in}}{\pgfqpoint{3.604669in}{4.817247in}}{\pgfqpoint{3.593619in}{4.817247in}}%
\pgfpathcurveto{\pgfqpoint{3.582569in}{4.817247in}}{\pgfqpoint{3.571970in}{4.812857in}}{\pgfqpoint{3.564156in}{4.805044in}}%
\pgfpathcurveto{\pgfqpoint{3.556343in}{4.797230in}}{\pgfqpoint{3.551953in}{4.786631in}}{\pgfqpoint{3.551953in}{4.775581in}}%
\pgfpathcurveto{\pgfqpoint{3.551953in}{4.764531in}}{\pgfqpoint{3.556343in}{4.753932in}}{\pgfqpoint{3.564156in}{4.746118in}}%
\pgfpathcurveto{\pgfqpoint{3.571970in}{4.738304in}}{\pgfqpoint{3.582569in}{4.733914in}}{\pgfqpoint{3.593619in}{4.733914in}}%
\pgfpathclose%
\pgfusepath{stroke,fill}%
\end{pgfscope}%
\begin{pgfscope}%
\pgfpathrectangle{\pgfqpoint{0.970666in}{4.121437in}}{\pgfqpoint{5.699255in}{2.685432in}}%
\pgfusepath{clip}%
\pgfsetbuttcap%
\pgfsetroundjoin%
\definecolor{currentfill}{rgb}{0.000000,0.000000,0.000000}%
\pgfsetfillcolor{currentfill}%
\pgfsetlinewidth{1.003750pt}%
\definecolor{currentstroke}{rgb}{0.000000,0.000000,0.000000}%
\pgfsetstrokecolor{currentstroke}%
\pgfsetdash{}{0pt}%
\pgfpathmoveto{\pgfqpoint{4.759376in}{4.733914in}}%
\pgfpathcurveto{\pgfqpoint{4.770426in}{4.733914in}}{\pgfqpoint{4.781025in}{4.738304in}}{\pgfqpoint{4.788839in}{4.746118in}}%
\pgfpathcurveto{\pgfqpoint{4.796652in}{4.753932in}}{\pgfqpoint{4.801043in}{4.764531in}}{\pgfqpoint{4.801043in}{4.775581in}}%
\pgfpathcurveto{\pgfqpoint{4.801043in}{4.786631in}}{\pgfqpoint{4.796652in}{4.797230in}}{\pgfqpoint{4.788839in}{4.805044in}}%
\pgfpathcurveto{\pgfqpoint{4.781025in}{4.812857in}}{\pgfqpoint{4.770426in}{4.817247in}}{\pgfqpoint{4.759376in}{4.817247in}}%
\pgfpathcurveto{\pgfqpoint{4.748326in}{4.817247in}}{\pgfqpoint{4.737727in}{4.812857in}}{\pgfqpoint{4.729913in}{4.805044in}}%
\pgfpathcurveto{\pgfqpoint{4.722100in}{4.797230in}}{\pgfqpoint{4.717709in}{4.786631in}}{\pgfqpoint{4.717709in}{4.775581in}}%
\pgfpathcurveto{\pgfqpoint{4.717709in}{4.764531in}}{\pgfqpoint{4.722100in}{4.753932in}}{\pgfqpoint{4.729913in}{4.746118in}}%
\pgfpathcurveto{\pgfqpoint{4.737727in}{4.738304in}}{\pgfqpoint{4.748326in}{4.733914in}}{\pgfqpoint{4.759376in}{4.733914in}}%
\pgfpathclose%
\pgfusepath{stroke,fill}%
\end{pgfscope}%
\begin{pgfscope}%
\pgfpathrectangle{\pgfqpoint{0.970666in}{4.121437in}}{\pgfqpoint{5.699255in}{2.685432in}}%
\pgfusepath{clip}%
\pgfsetbuttcap%
\pgfsetroundjoin%
\definecolor{currentfill}{rgb}{0.000000,0.000000,0.000000}%
\pgfsetfillcolor{currentfill}%
\pgfsetlinewidth{1.003750pt}%
\definecolor{currentstroke}{rgb}{0.000000,0.000000,0.000000}%
\pgfsetstrokecolor{currentstroke}%
\pgfsetdash{}{0pt}%
\pgfpathmoveto{\pgfqpoint{2.201188in}{5.453785in}}%
\pgfpathcurveto{\pgfqpoint{2.212238in}{5.453785in}}{\pgfqpoint{2.222837in}{5.458176in}}{\pgfqpoint{2.230650in}{5.465989in}}%
\pgfpathcurveto{\pgfqpoint{2.238464in}{5.473803in}}{\pgfqpoint{2.242854in}{5.484402in}}{\pgfqpoint{2.242854in}{5.495452in}}%
\pgfpathcurveto{\pgfqpoint{2.242854in}{5.506502in}}{\pgfqpoint{2.238464in}{5.517101in}}{\pgfqpoint{2.230650in}{5.524915in}}%
\pgfpathcurveto{\pgfqpoint{2.222837in}{5.532728in}}{\pgfqpoint{2.212238in}{5.537119in}}{\pgfqpoint{2.201188in}{5.537119in}}%
\pgfpathcurveto{\pgfqpoint{2.190137in}{5.537119in}}{\pgfqpoint{2.179538in}{5.532728in}}{\pgfqpoint{2.171725in}{5.524915in}}%
\pgfpathcurveto{\pgfqpoint{2.163911in}{5.517101in}}{\pgfqpoint{2.159521in}{5.506502in}}{\pgfqpoint{2.159521in}{5.495452in}}%
\pgfpathcurveto{\pgfqpoint{2.159521in}{5.484402in}}{\pgfqpoint{2.163911in}{5.473803in}}{\pgfqpoint{2.171725in}{5.465989in}}%
\pgfpathcurveto{\pgfqpoint{2.179538in}{5.458176in}}{\pgfqpoint{2.190137in}{5.453785in}}{\pgfqpoint{2.201188in}{5.453785in}}%
\pgfpathclose%
\pgfusepath{stroke,fill}%
\end{pgfscope}%
\begin{pgfscope}%
\pgfpathrectangle{\pgfqpoint{0.970666in}{4.121437in}}{\pgfqpoint{5.699255in}{2.685432in}}%
\pgfusepath{clip}%
\pgfsetbuttcap%
\pgfsetroundjoin%
\definecolor{currentfill}{rgb}{0.000000,0.000000,0.000000}%
\pgfsetfillcolor{currentfill}%
\pgfsetlinewidth{1.003750pt}%
\definecolor{currentstroke}{rgb}{0.000000,0.000000,0.000000}%
\pgfsetstrokecolor{currentstroke}%
\pgfsetdash{}{0pt}%
\pgfpathmoveto{\pgfqpoint{2.104041in}{6.486644in}}%
\pgfpathcurveto{\pgfqpoint{2.115091in}{6.486644in}}{\pgfqpoint{2.125690in}{6.491034in}}{\pgfqpoint{2.133504in}{6.498848in}}%
\pgfpathcurveto{\pgfqpoint{2.141318in}{6.506661in}}{\pgfqpoint{2.145708in}{6.517260in}}{\pgfqpoint{2.145708in}{6.528310in}}%
\pgfpathcurveto{\pgfqpoint{2.145708in}{6.539361in}}{\pgfqpoint{2.141318in}{6.549960in}}{\pgfqpoint{2.133504in}{6.557773in}}%
\pgfpathcurveto{\pgfqpoint{2.125690in}{6.565587in}}{\pgfqpoint{2.115091in}{6.569977in}}{\pgfqpoint{2.104041in}{6.569977in}}%
\pgfpathcurveto{\pgfqpoint{2.092991in}{6.569977in}}{\pgfqpoint{2.082392in}{6.565587in}}{\pgfqpoint{2.074578in}{6.557773in}}%
\pgfpathcurveto{\pgfqpoint{2.066765in}{6.549960in}}{\pgfqpoint{2.062374in}{6.539361in}}{\pgfqpoint{2.062374in}{6.528310in}}%
\pgfpathcurveto{\pgfqpoint{2.062374in}{6.517260in}}{\pgfqpoint{2.066765in}{6.506661in}}{\pgfqpoint{2.074578in}{6.498848in}}%
\pgfpathcurveto{\pgfqpoint{2.082392in}{6.491034in}}{\pgfqpoint{2.092991in}{6.486644in}}{\pgfqpoint{2.104041in}{6.486644in}}%
\pgfpathclose%
\pgfusepath{stroke,fill}%
\end{pgfscope}%
\begin{pgfscope}%
\pgfpathrectangle{\pgfqpoint{0.970666in}{4.121437in}}{\pgfqpoint{5.699255in}{2.685432in}}%
\pgfusepath{clip}%
\pgfsetbuttcap%
\pgfsetroundjoin%
\definecolor{currentfill}{rgb}{0.000000,0.000000,0.000000}%
\pgfsetfillcolor{currentfill}%
\pgfsetlinewidth{1.003750pt}%
\definecolor{currentstroke}{rgb}{0.000000,0.000000,0.000000}%
\pgfsetstrokecolor{currentstroke}%
\pgfsetdash{}{0pt}%
\pgfpathmoveto{\pgfqpoint{3.626001in}{5.923266in}}%
\pgfpathcurveto{\pgfqpoint{3.637052in}{5.923266in}}{\pgfqpoint{3.647651in}{5.927657in}}{\pgfqpoint{3.655464in}{5.935470in}}%
\pgfpathcurveto{\pgfqpoint{3.663278in}{5.943284in}}{\pgfqpoint{3.667668in}{5.953883in}}{\pgfqpoint{3.667668in}{5.964933in}}%
\pgfpathcurveto{\pgfqpoint{3.667668in}{5.975983in}}{\pgfqpoint{3.663278in}{5.986582in}}{\pgfqpoint{3.655464in}{5.994396in}}%
\pgfpathcurveto{\pgfqpoint{3.647651in}{6.002209in}}{\pgfqpoint{3.637052in}{6.006600in}}{\pgfqpoint{3.626001in}{6.006600in}}%
\pgfpathcurveto{\pgfqpoint{3.614951in}{6.006600in}}{\pgfqpoint{3.604352in}{6.002209in}}{\pgfqpoint{3.596539in}{5.994396in}}%
\pgfpathcurveto{\pgfqpoint{3.588725in}{5.986582in}}{\pgfqpoint{3.584335in}{5.975983in}}{\pgfqpoint{3.584335in}{5.964933in}}%
\pgfpathcurveto{\pgfqpoint{3.584335in}{5.953883in}}{\pgfqpoint{3.588725in}{5.943284in}}{\pgfqpoint{3.596539in}{5.935470in}}%
\pgfpathcurveto{\pgfqpoint{3.604352in}{5.927657in}}{\pgfqpoint{3.614951in}{5.923266in}}{\pgfqpoint{3.626001in}{5.923266in}}%
\pgfpathclose%
\pgfusepath{stroke,fill}%
\end{pgfscope}%
\begin{pgfscope}%
\pgfpathrectangle{\pgfqpoint{0.970666in}{4.121437in}}{\pgfqpoint{5.699255in}{2.685432in}}%
\pgfusepath{clip}%
\pgfsetbuttcap%
\pgfsetroundjoin%
\definecolor{currentfill}{rgb}{0.000000,0.000000,0.000000}%
\pgfsetfillcolor{currentfill}%
\pgfsetlinewidth{1.003750pt}%
\definecolor{currentstroke}{rgb}{0.000000,0.000000,0.000000}%
\pgfsetstrokecolor{currentstroke}%
\pgfsetdash{}{0pt}%
\pgfpathmoveto{\pgfqpoint{3.237416in}{4.686966in}}%
\pgfpathcurveto{\pgfqpoint{3.248466in}{4.686966in}}{\pgfqpoint{3.259065in}{4.691356in}}{\pgfqpoint{3.266879in}{4.699170in}}%
\pgfpathcurveto{\pgfqpoint{3.274692in}{4.706984in}}{\pgfqpoint{3.279082in}{4.717583in}}{\pgfqpoint{3.279082in}{4.728633in}}%
\pgfpathcurveto{\pgfqpoint{3.279082in}{4.739683in}}{\pgfqpoint{3.274692in}{4.750282in}}{\pgfqpoint{3.266879in}{4.758096in}}%
\pgfpathcurveto{\pgfqpoint{3.259065in}{4.765909in}}{\pgfqpoint{3.248466in}{4.770299in}}{\pgfqpoint{3.237416in}{4.770299in}}%
\pgfpathcurveto{\pgfqpoint{3.226366in}{4.770299in}}{\pgfqpoint{3.215767in}{4.765909in}}{\pgfqpoint{3.207953in}{4.758096in}}%
\pgfpathcurveto{\pgfqpoint{3.200139in}{4.750282in}}{\pgfqpoint{3.195749in}{4.739683in}}{\pgfqpoint{3.195749in}{4.728633in}}%
\pgfpathcurveto{\pgfqpoint{3.195749in}{4.717583in}}{\pgfqpoint{3.200139in}{4.706984in}}{\pgfqpoint{3.207953in}{4.699170in}}%
\pgfpathcurveto{\pgfqpoint{3.215767in}{4.691356in}}{\pgfqpoint{3.226366in}{4.686966in}}{\pgfqpoint{3.237416in}{4.686966in}}%
\pgfpathclose%
\pgfusepath{stroke,fill}%
\end{pgfscope}%
\begin{pgfscope}%
\pgfpathrectangle{\pgfqpoint{0.970666in}{4.121437in}}{\pgfqpoint{5.699255in}{2.685432in}}%
\pgfusepath{clip}%
\pgfsetbuttcap%
\pgfsetroundjoin%
\definecolor{currentfill}{rgb}{0.000000,0.000000,0.000000}%
\pgfsetfillcolor{currentfill}%
\pgfsetlinewidth{1.003750pt}%
\definecolor{currentstroke}{rgb}{0.000000,0.000000,0.000000}%
\pgfsetstrokecolor{currentstroke}%
\pgfsetdash{}{0pt}%
\pgfpathmoveto{\pgfqpoint{4.759376in}{5.344240in}}%
\pgfpathcurveto{\pgfqpoint{4.770426in}{5.344240in}}{\pgfqpoint{4.781025in}{5.348630in}}{\pgfqpoint{4.788839in}{5.356444in}}%
\pgfpathcurveto{\pgfqpoint{4.796652in}{5.364257in}}{\pgfqpoint{4.801043in}{5.374856in}}{\pgfqpoint{4.801043in}{5.385906in}}%
\pgfpathcurveto{\pgfqpoint{4.801043in}{5.396956in}}{\pgfqpoint{4.796652in}{5.407555in}}{\pgfqpoint{4.788839in}{5.415369in}}%
\pgfpathcurveto{\pgfqpoint{4.781025in}{5.423183in}}{\pgfqpoint{4.770426in}{5.427573in}}{\pgfqpoint{4.759376in}{5.427573in}}%
\pgfpathcurveto{\pgfqpoint{4.748326in}{5.427573in}}{\pgfqpoint{4.737727in}{5.423183in}}{\pgfqpoint{4.729913in}{5.415369in}}%
\pgfpathcurveto{\pgfqpoint{4.722100in}{5.407555in}}{\pgfqpoint{4.717709in}{5.396956in}}{\pgfqpoint{4.717709in}{5.385906in}}%
\pgfpathcurveto{\pgfqpoint{4.717709in}{5.374856in}}{\pgfqpoint{4.722100in}{5.364257in}}{\pgfqpoint{4.729913in}{5.356444in}}%
\pgfpathcurveto{\pgfqpoint{4.737727in}{5.348630in}}{\pgfqpoint{4.748326in}{5.344240in}}{\pgfqpoint{4.759376in}{5.344240in}}%
\pgfpathclose%
\pgfusepath{stroke,fill}%
\end{pgfscope}%
\begin{pgfscope}%
\pgfpathrectangle{\pgfqpoint{0.970666in}{4.121437in}}{\pgfqpoint{5.699255in}{2.685432in}}%
\pgfusepath{clip}%
\pgfsetbuttcap%
\pgfsetroundjoin%
\definecolor{currentfill}{rgb}{0.000000,0.000000,0.000000}%
\pgfsetfillcolor{currentfill}%
\pgfsetlinewidth{1.003750pt}%
\definecolor{currentstroke}{rgb}{0.000000,0.000000,0.000000}%
\pgfsetstrokecolor{currentstroke}%
\pgfsetdash{}{0pt}%
\pgfpathmoveto{\pgfqpoint{6.184190in}{4.796512in}}%
\pgfpathcurveto{\pgfqpoint{6.195240in}{4.796512in}}{\pgfqpoint{6.205839in}{4.800902in}}{\pgfqpoint{6.213653in}{4.808716in}}%
\pgfpathcurveto{\pgfqpoint{6.221466in}{4.816529in}}{\pgfqpoint{6.225857in}{4.827128in}}{\pgfqpoint{6.225857in}{4.838178in}}%
\pgfpathcurveto{\pgfqpoint{6.225857in}{4.849228in}}{\pgfqpoint{6.221466in}{4.859827in}}{\pgfqpoint{6.213653in}{4.867641in}}%
\pgfpathcurveto{\pgfqpoint{6.205839in}{4.875455in}}{\pgfqpoint{6.195240in}{4.879845in}}{\pgfqpoint{6.184190in}{4.879845in}}%
\pgfpathcurveto{\pgfqpoint{6.173140in}{4.879845in}}{\pgfqpoint{6.162541in}{4.875455in}}{\pgfqpoint{6.154727in}{4.867641in}}%
\pgfpathcurveto{\pgfqpoint{6.146913in}{4.859827in}}{\pgfqpoint{6.142523in}{4.849228in}}{\pgfqpoint{6.142523in}{4.838178in}}%
\pgfpathcurveto{\pgfqpoint{6.142523in}{4.827128in}}{\pgfqpoint{6.146913in}{4.816529in}}{\pgfqpoint{6.154727in}{4.808716in}}%
\pgfpathcurveto{\pgfqpoint{6.162541in}{4.800902in}}{\pgfqpoint{6.173140in}{4.796512in}}{\pgfqpoint{6.184190in}{4.796512in}}%
\pgfpathclose%
\pgfusepath{stroke,fill}%
\end{pgfscope}%
\begin{pgfscope}%
\pgfpathrectangle{\pgfqpoint{0.970666in}{4.121437in}}{\pgfqpoint{5.699255in}{2.685432in}}%
\pgfusepath{clip}%
\pgfsetbuttcap%
\pgfsetroundjoin%
\definecolor{currentfill}{rgb}{0.000000,0.000000,0.000000}%
\pgfsetfillcolor{currentfill}%
\pgfsetlinewidth{1.003750pt}%
\definecolor{currentstroke}{rgb}{0.000000,0.000000,0.000000}%
\pgfsetstrokecolor{currentstroke}%
\pgfsetdash{}{0pt}%
\pgfpathmoveto{\pgfqpoint{5.277490in}{4.780862in}}%
\pgfpathcurveto{\pgfqpoint{5.288540in}{4.780862in}}{\pgfqpoint{5.299139in}{4.785253in}}{\pgfqpoint{5.306953in}{4.793066in}}%
\pgfpathcurveto{\pgfqpoint{5.314767in}{4.800880in}}{\pgfqpoint{5.319157in}{4.811479in}}{\pgfqpoint{5.319157in}{4.822529in}}%
\pgfpathcurveto{\pgfqpoint{5.319157in}{4.833579in}}{\pgfqpoint{5.314767in}{4.844178in}}{\pgfqpoint{5.306953in}{4.851992in}}%
\pgfpathcurveto{\pgfqpoint{5.299139in}{4.859805in}}{\pgfqpoint{5.288540in}{4.864196in}}{\pgfqpoint{5.277490in}{4.864196in}}%
\pgfpathcurveto{\pgfqpoint{5.266440in}{4.864196in}}{\pgfqpoint{5.255841in}{4.859805in}}{\pgfqpoint{5.248027in}{4.851992in}}%
\pgfpathcurveto{\pgfqpoint{5.240214in}{4.844178in}}{\pgfqpoint{5.235823in}{4.833579in}}{\pgfqpoint{5.235823in}{4.822529in}}%
\pgfpathcurveto{\pgfqpoint{5.235823in}{4.811479in}}{\pgfqpoint{5.240214in}{4.800880in}}{\pgfqpoint{5.248027in}{4.793066in}}%
\pgfpathcurveto{\pgfqpoint{5.255841in}{4.785253in}}{\pgfqpoint{5.266440in}{4.780862in}}{\pgfqpoint{5.277490in}{4.780862in}}%
\pgfpathclose%
\pgfusepath{stroke,fill}%
\end{pgfscope}%
\begin{pgfscope}%
\pgfpathrectangle{\pgfqpoint{0.970666in}{4.121437in}}{\pgfqpoint{5.699255in}{2.685432in}}%
\pgfusepath{clip}%
\pgfsetbuttcap%
\pgfsetroundjoin%
\definecolor{currentfill}{rgb}{0.000000,0.000000,0.000000}%
\pgfsetfillcolor{currentfill}%
\pgfsetlinewidth{1.003750pt}%
\definecolor{currentstroke}{rgb}{0.000000,0.000000,0.000000}%
\pgfsetstrokecolor{currentstroke}%
\pgfsetdash{}{0pt}%
\pgfpathmoveto{\pgfqpoint{4.046969in}{4.874759in}}%
\pgfpathcurveto{\pgfqpoint{4.058019in}{4.874759in}}{\pgfqpoint{4.068618in}{4.879149in}}{\pgfqpoint{4.076432in}{4.886962in}}%
\pgfpathcurveto{\pgfqpoint{4.084246in}{4.894776in}}{\pgfqpoint{4.088636in}{4.905375in}}{\pgfqpoint{4.088636in}{4.916425in}}%
\pgfpathcurveto{\pgfqpoint{4.088636in}{4.927475in}}{\pgfqpoint{4.084246in}{4.938074in}}{\pgfqpoint{4.076432in}{4.945888in}}%
\pgfpathcurveto{\pgfqpoint{4.068618in}{4.953702in}}{\pgfqpoint{4.058019in}{4.958092in}}{\pgfqpoint{4.046969in}{4.958092in}}%
\pgfpathcurveto{\pgfqpoint{4.035919in}{4.958092in}}{\pgfqpoint{4.025320in}{4.953702in}}{\pgfqpoint{4.017506in}{4.945888in}}%
\pgfpathcurveto{\pgfqpoint{4.009693in}{4.938074in}}{\pgfqpoint{4.005302in}{4.927475in}}{\pgfqpoint{4.005302in}{4.916425in}}%
\pgfpathcurveto{\pgfqpoint{4.005302in}{4.905375in}}{\pgfqpoint{4.009693in}{4.894776in}}{\pgfqpoint{4.017506in}{4.886962in}}%
\pgfpathcurveto{\pgfqpoint{4.025320in}{4.879149in}}{\pgfqpoint{4.035919in}{4.874759in}}{\pgfqpoint{4.046969in}{4.874759in}}%
\pgfpathclose%
\pgfusepath{stroke,fill}%
\end{pgfscope}%
\begin{pgfscope}%
\pgfpathrectangle{\pgfqpoint{0.970666in}{4.121437in}}{\pgfqpoint{5.699255in}{2.685432in}}%
\pgfusepath{clip}%
\pgfsetbuttcap%
\pgfsetroundjoin%
\definecolor{currentfill}{rgb}{0.000000,0.000000,0.000000}%
\pgfsetfillcolor{currentfill}%
\pgfsetlinewidth{1.003750pt}%
\definecolor{currentstroke}{rgb}{0.000000,0.000000,0.000000}%
\pgfsetstrokecolor{currentstroke}%
\pgfsetdash{}{0pt}%
\pgfpathmoveto{\pgfqpoint{2.363098in}{4.530472in}}%
\pgfpathcurveto{\pgfqpoint{2.374148in}{4.530472in}}{\pgfqpoint{2.384747in}{4.534863in}}{\pgfqpoint{2.392561in}{4.542676in}}%
\pgfpathcurveto{\pgfqpoint{2.400375in}{4.550490in}}{\pgfqpoint{2.404765in}{4.561089in}}{\pgfqpoint{2.404765in}{4.572139in}}%
\pgfpathcurveto{\pgfqpoint{2.404765in}{4.583189in}}{\pgfqpoint{2.400375in}{4.593788in}}{\pgfqpoint{2.392561in}{4.601602in}}%
\pgfpathcurveto{\pgfqpoint{2.384747in}{4.609415in}}{\pgfqpoint{2.374148in}{4.613806in}}{\pgfqpoint{2.363098in}{4.613806in}}%
\pgfpathcurveto{\pgfqpoint{2.352048in}{4.613806in}}{\pgfqpoint{2.341449in}{4.609415in}}{\pgfqpoint{2.333635in}{4.601602in}}%
\pgfpathcurveto{\pgfqpoint{2.325822in}{4.593788in}}{\pgfqpoint{2.321432in}{4.583189in}}{\pgfqpoint{2.321432in}{4.572139in}}%
\pgfpathcurveto{\pgfqpoint{2.321432in}{4.561089in}}{\pgfqpoint{2.325822in}{4.550490in}}{\pgfqpoint{2.333635in}{4.542676in}}%
\pgfpathcurveto{\pgfqpoint{2.341449in}{4.534863in}}{\pgfqpoint{2.352048in}{4.530472in}}{\pgfqpoint{2.363098in}{4.530472in}}%
\pgfpathclose%
\pgfusepath{stroke,fill}%
\end{pgfscope}%
\begin{pgfscope}%
\pgfpathrectangle{\pgfqpoint{0.970666in}{4.121437in}}{\pgfqpoint{5.699255in}{2.685432in}}%
\pgfusepath{clip}%
\pgfsetbuttcap%
\pgfsetroundjoin%
\definecolor{currentfill}{rgb}{0.000000,0.000000,0.000000}%
\pgfsetfillcolor{currentfill}%
\pgfsetlinewidth{1.003750pt}%
\definecolor{currentstroke}{rgb}{0.000000,0.000000,0.000000}%
\pgfsetstrokecolor{currentstroke}%
\pgfsetdash{}{0pt}%
\pgfpathmoveto{\pgfqpoint{2.492627in}{6.611839in}}%
\pgfpathcurveto{\pgfqpoint{2.503677in}{6.611839in}}{\pgfqpoint{2.514276in}{6.616229in}}{\pgfqpoint{2.522090in}{6.624043in}}%
\pgfpathcurveto{\pgfqpoint{2.529903in}{6.631856in}}{\pgfqpoint{2.534293in}{6.642455in}}{\pgfqpoint{2.534293in}{6.653505in}}%
\pgfpathcurveto{\pgfqpoint{2.534293in}{6.664556in}}{\pgfqpoint{2.529903in}{6.675155in}}{\pgfqpoint{2.522090in}{6.682968in}}%
\pgfpathcurveto{\pgfqpoint{2.514276in}{6.690782in}}{\pgfqpoint{2.503677in}{6.695172in}}{\pgfqpoint{2.492627in}{6.695172in}}%
\pgfpathcurveto{\pgfqpoint{2.481577in}{6.695172in}}{\pgfqpoint{2.470978in}{6.690782in}}{\pgfqpoint{2.463164in}{6.682968in}}%
\pgfpathcurveto{\pgfqpoint{2.455350in}{6.675155in}}{\pgfqpoint{2.450960in}{6.664556in}}{\pgfqpoint{2.450960in}{6.653505in}}%
\pgfpathcurveto{\pgfqpoint{2.450960in}{6.642455in}}{\pgfqpoint{2.455350in}{6.631856in}}{\pgfqpoint{2.463164in}{6.624043in}}%
\pgfpathcurveto{\pgfqpoint{2.470978in}{6.616229in}}{\pgfqpoint{2.481577in}{6.611839in}}{\pgfqpoint{2.492627in}{6.611839in}}%
\pgfpathclose%
\pgfusepath{stroke,fill}%
\end{pgfscope}%
\begin{pgfscope}%
\pgfpathrectangle{\pgfqpoint{0.970666in}{4.121437in}}{\pgfqpoint{5.699255in}{2.685432in}}%
\pgfusepath{clip}%
\pgfsetbuttcap%
\pgfsetroundjoin%
\definecolor{currentfill}{rgb}{0.000000,0.000000,0.000000}%
\pgfsetfillcolor{currentfill}%
\pgfsetlinewidth{1.003750pt}%
\definecolor{currentstroke}{rgb}{0.000000,0.000000,0.000000}%
\pgfsetstrokecolor{currentstroke}%
\pgfsetdash{}{0pt}%
\pgfpathmoveto{\pgfqpoint{5.471783in}{6.267553in}}%
\pgfpathcurveto{\pgfqpoint{5.482833in}{6.267553in}}{\pgfqpoint{5.493432in}{6.271943in}}{\pgfqpoint{5.501246in}{6.279756in}}%
\pgfpathcurveto{\pgfqpoint{5.509059in}{6.287570in}}{\pgfqpoint{5.513450in}{6.298169in}}{\pgfqpoint{5.513450in}{6.309219in}}%
\pgfpathcurveto{\pgfqpoint{5.513450in}{6.320269in}}{\pgfqpoint{5.509059in}{6.330868in}}{\pgfqpoint{5.501246in}{6.338682in}}%
\pgfpathcurveto{\pgfqpoint{5.493432in}{6.346496in}}{\pgfqpoint{5.482833in}{6.350886in}}{\pgfqpoint{5.471783in}{6.350886in}}%
\pgfpathcurveto{\pgfqpoint{5.460733in}{6.350886in}}{\pgfqpoint{5.450134in}{6.346496in}}{\pgfqpoint{5.442320in}{6.338682in}}%
\pgfpathcurveto{\pgfqpoint{5.434507in}{6.330868in}}{\pgfqpoint{5.430116in}{6.320269in}}{\pgfqpoint{5.430116in}{6.309219in}}%
\pgfpathcurveto{\pgfqpoint{5.430116in}{6.298169in}}{\pgfqpoint{5.434507in}{6.287570in}}{\pgfqpoint{5.442320in}{6.279756in}}%
\pgfpathcurveto{\pgfqpoint{5.450134in}{6.271943in}}{\pgfqpoint{5.460733in}{6.267553in}}{\pgfqpoint{5.471783in}{6.267553in}}%
\pgfpathclose%
\pgfusepath{stroke,fill}%
\end{pgfscope}%
\begin{pgfscope}%
\pgfpathrectangle{\pgfqpoint{0.970666in}{4.121437in}}{\pgfqpoint{5.699255in}{2.685432in}}%
\pgfusepath{clip}%
\pgfsetbuttcap%
\pgfsetroundjoin%
\definecolor{currentfill}{rgb}{0.000000,0.000000,0.000000}%
\pgfsetfillcolor{currentfill}%
\pgfsetlinewidth{1.003750pt}%
\definecolor{currentstroke}{rgb}{0.000000,0.000000,0.000000}%
\pgfsetstrokecolor{currentstroke}%
\pgfsetdash{}{0pt}%
\pgfpathmoveto{\pgfqpoint{6.022279in}{5.406837in}}%
\pgfpathcurveto{\pgfqpoint{6.033329in}{5.406837in}}{\pgfqpoint{6.043928in}{5.411227in}}{\pgfqpoint{6.051742in}{5.419041in}}%
\pgfpathcurveto{\pgfqpoint{6.059556in}{5.426855in}}{\pgfqpoint{6.063946in}{5.437454in}}{\pgfqpoint{6.063946in}{5.448504in}}%
\pgfpathcurveto{\pgfqpoint{6.063946in}{5.459554in}}{\pgfqpoint{6.059556in}{5.470153in}}{\pgfqpoint{6.051742in}{5.477967in}}%
\pgfpathcurveto{\pgfqpoint{6.043928in}{5.485780in}}{\pgfqpoint{6.033329in}{5.490170in}}{\pgfqpoint{6.022279in}{5.490170in}}%
\pgfpathcurveto{\pgfqpoint{6.011229in}{5.490170in}}{\pgfqpoint{6.000630in}{5.485780in}}{\pgfqpoint{5.992816in}{5.477967in}}%
\pgfpathcurveto{\pgfqpoint{5.985003in}{5.470153in}}{\pgfqpoint{5.980613in}{5.459554in}}{\pgfqpoint{5.980613in}{5.448504in}}%
\pgfpathcurveto{\pgfqpoint{5.980613in}{5.437454in}}{\pgfqpoint{5.985003in}{5.426855in}}{\pgfqpoint{5.992816in}{5.419041in}}%
\pgfpathcurveto{\pgfqpoint{6.000630in}{5.411227in}}{\pgfqpoint{6.011229in}{5.406837in}}{\pgfqpoint{6.022279in}{5.406837in}}%
\pgfpathclose%
\pgfusepath{stroke,fill}%
\end{pgfscope}%
\begin{pgfscope}%
\pgfpathrectangle{\pgfqpoint{0.970666in}{4.121437in}}{\pgfqpoint{5.699255in}{2.685432in}}%
\pgfusepath{clip}%
\pgfsetbuttcap%
\pgfsetroundjoin%
\definecolor{currentfill}{rgb}{0.000000,0.000000,0.000000}%
\pgfsetfillcolor{currentfill}%
\pgfsetlinewidth{1.003750pt}%
\definecolor{currentstroke}{rgb}{0.000000,0.000000,0.000000}%
\pgfsetstrokecolor{currentstroke}%
\pgfsetdash{}{0pt}%
\pgfpathmoveto{\pgfqpoint{4.791758in}{4.201836in}}%
\pgfpathcurveto{\pgfqpoint{4.802808in}{4.201836in}}{\pgfqpoint{4.813407in}{4.206226in}}{\pgfqpoint{4.821221in}{4.214039in}}%
\pgfpathcurveto{\pgfqpoint{4.829035in}{4.221853in}}{\pgfqpoint{4.833425in}{4.232452in}}{\pgfqpoint{4.833425in}{4.243502in}}%
\pgfpathcurveto{\pgfqpoint{4.833425in}{4.254552in}}{\pgfqpoint{4.829035in}{4.265151in}}{\pgfqpoint{4.821221in}{4.272965in}}%
\pgfpathcurveto{\pgfqpoint{4.813407in}{4.280779in}}{\pgfqpoint{4.802808in}{4.285169in}}{\pgfqpoint{4.791758in}{4.285169in}}%
\pgfpathcurveto{\pgfqpoint{4.780708in}{4.285169in}}{\pgfqpoint{4.770109in}{4.280779in}}{\pgfqpoint{4.762295in}{4.272965in}}%
\pgfpathcurveto{\pgfqpoint{4.754482in}{4.265151in}}{\pgfqpoint{4.750092in}{4.254552in}}{\pgfqpoint{4.750092in}{4.243502in}}%
\pgfpathcurveto{\pgfqpoint{4.750092in}{4.232452in}}{\pgfqpoint{4.754482in}{4.221853in}}{\pgfqpoint{4.762295in}{4.214039in}}%
\pgfpathcurveto{\pgfqpoint{4.770109in}{4.206226in}}{\pgfqpoint{4.780708in}{4.201836in}}{\pgfqpoint{4.791758in}{4.201836in}}%
\pgfpathclose%
\pgfusepath{stroke,fill}%
\end{pgfscope}%
\begin{pgfscope}%
\pgfpathrectangle{\pgfqpoint{0.970666in}{4.121437in}}{\pgfqpoint{5.699255in}{2.685432in}}%
\pgfusepath{clip}%
\pgfsetbuttcap%
\pgfsetroundjoin%
\definecolor{currentfill}{rgb}{0.000000,0.000000,0.000000}%
\pgfsetfillcolor{currentfill}%
\pgfsetlinewidth{1.003750pt}%
\definecolor{currentstroke}{rgb}{0.000000,0.000000,0.000000}%
\pgfsetstrokecolor{currentstroke}%
\pgfsetdash{}{0pt}%
\pgfpathmoveto{\pgfqpoint{2.945977in}{5.391188in}}%
\pgfpathcurveto{\pgfqpoint{2.957027in}{5.391188in}}{\pgfqpoint{2.967626in}{5.395578in}}{\pgfqpoint{2.975439in}{5.403392in}}%
\pgfpathcurveto{\pgfqpoint{2.983253in}{5.411205in}}{\pgfqpoint{2.987643in}{5.421804in}}{\pgfqpoint{2.987643in}{5.432854in}}%
\pgfpathcurveto{\pgfqpoint{2.987643in}{5.443905in}}{\pgfqpoint{2.983253in}{5.454504in}}{\pgfqpoint{2.975439in}{5.462317in}}%
\pgfpathcurveto{\pgfqpoint{2.967626in}{5.470131in}}{\pgfqpoint{2.957027in}{5.474521in}}{\pgfqpoint{2.945977in}{5.474521in}}%
\pgfpathcurveto{\pgfqpoint{2.934926in}{5.474521in}}{\pgfqpoint{2.924327in}{5.470131in}}{\pgfqpoint{2.916514in}{5.462317in}}%
\pgfpathcurveto{\pgfqpoint{2.908700in}{5.454504in}}{\pgfqpoint{2.904310in}{5.443905in}}{\pgfqpoint{2.904310in}{5.432854in}}%
\pgfpathcurveto{\pgfqpoint{2.904310in}{5.421804in}}{\pgfqpoint{2.908700in}{5.411205in}}{\pgfqpoint{2.916514in}{5.403392in}}%
\pgfpathcurveto{\pgfqpoint{2.924327in}{5.395578in}}{\pgfqpoint{2.934926in}{5.391188in}}{\pgfqpoint{2.945977in}{5.391188in}}%
\pgfpathclose%
\pgfusepath{stroke,fill}%
\end{pgfscope}%
\begin{pgfscope}%
\pgfpathrectangle{\pgfqpoint{0.970666in}{4.121437in}}{\pgfqpoint{5.699255in}{2.685432in}}%
\pgfusepath{clip}%
\pgfsetbuttcap%
\pgfsetroundjoin%
\definecolor{currentfill}{rgb}{0.000000,0.000000,0.000000}%
\pgfsetfillcolor{currentfill}%
\pgfsetlinewidth{1.003750pt}%
\definecolor{currentstroke}{rgb}{0.000000,0.000000,0.000000}%
\pgfsetstrokecolor{currentstroke}%
\pgfsetdash{}{0pt}%
\pgfpathmoveto{\pgfqpoint{1.262106in}{5.109499in}}%
\pgfpathcurveto{\pgfqpoint{1.273156in}{5.109499in}}{\pgfqpoint{1.283755in}{5.113889in}}{\pgfqpoint{1.291568in}{5.121703in}}%
\pgfpathcurveto{\pgfqpoint{1.299382in}{5.129517in}}{\pgfqpoint{1.303772in}{5.140116in}}{\pgfqpoint{1.303772in}{5.151166in}}%
\pgfpathcurveto{\pgfqpoint{1.303772in}{5.162216in}}{\pgfqpoint{1.299382in}{5.172815in}}{\pgfqpoint{1.291568in}{5.180629in}}%
\pgfpathcurveto{\pgfqpoint{1.283755in}{5.188442in}}{\pgfqpoint{1.273156in}{5.192832in}}{\pgfqpoint{1.262106in}{5.192832in}}%
\pgfpathcurveto{\pgfqpoint{1.251056in}{5.192832in}}{\pgfqpoint{1.240457in}{5.188442in}}{\pgfqpoint{1.232643in}{5.180629in}}%
\pgfpathcurveto{\pgfqpoint{1.224829in}{5.172815in}}{\pgfqpoint{1.220439in}{5.162216in}}{\pgfqpoint{1.220439in}{5.151166in}}%
\pgfpathcurveto{\pgfqpoint{1.220439in}{5.140116in}}{\pgfqpoint{1.224829in}{5.129517in}}{\pgfqpoint{1.232643in}{5.121703in}}%
\pgfpathcurveto{\pgfqpoint{1.240457in}{5.113889in}}{\pgfqpoint{1.251056in}{5.109499in}}{\pgfqpoint{1.262106in}{5.109499in}}%
\pgfpathclose%
\pgfusepath{stroke,fill}%
\end{pgfscope}%
\begin{pgfscope}%
\pgfpathrectangle{\pgfqpoint{0.970666in}{4.121437in}}{\pgfqpoint{5.699255in}{2.685432in}}%
\pgfusepath{clip}%
\pgfsetbuttcap%
\pgfsetroundjoin%
\definecolor{currentfill}{rgb}{0.000000,0.000000,0.000000}%
\pgfsetfillcolor{currentfill}%
\pgfsetlinewidth{1.003750pt}%
\definecolor{currentstroke}{rgb}{0.000000,0.000000,0.000000}%
\pgfsetstrokecolor{currentstroke}%
\pgfsetdash{}{0pt}%
\pgfpathmoveto{\pgfqpoint{2.525009in}{4.483524in}}%
\pgfpathcurveto{\pgfqpoint{2.536059in}{4.483524in}}{\pgfqpoint{2.546658in}{4.487914in}}{\pgfqpoint{2.554472in}{4.495728in}}%
\pgfpathcurveto{\pgfqpoint{2.562285in}{4.503542in}}{\pgfqpoint{2.566676in}{4.514141in}}{\pgfqpoint{2.566676in}{4.525191in}}%
\pgfpathcurveto{\pgfqpoint{2.566676in}{4.536241in}}{\pgfqpoint{2.562285in}{4.546840in}}{\pgfqpoint{2.554472in}{4.554654in}}%
\pgfpathcurveto{\pgfqpoint{2.546658in}{4.562467in}}{\pgfqpoint{2.536059in}{4.566858in}}{\pgfqpoint{2.525009in}{4.566858in}}%
\pgfpathcurveto{\pgfqpoint{2.513959in}{4.566858in}}{\pgfqpoint{2.503360in}{4.562467in}}{\pgfqpoint{2.495546in}{4.554654in}}%
\pgfpathcurveto{\pgfqpoint{2.487732in}{4.546840in}}{\pgfqpoint{2.483342in}{4.536241in}}{\pgfqpoint{2.483342in}{4.525191in}}%
\pgfpathcurveto{\pgfqpoint{2.483342in}{4.514141in}}{\pgfqpoint{2.487732in}{4.503542in}}{\pgfqpoint{2.495546in}{4.495728in}}%
\pgfpathcurveto{\pgfqpoint{2.503360in}{4.487914in}}{\pgfqpoint{2.513959in}{4.483524in}}{\pgfqpoint{2.525009in}{4.483524in}}%
\pgfpathclose%
\pgfusepath{stroke,fill}%
\end{pgfscope}%
\begin{pgfscope}%
\pgfpathrectangle{\pgfqpoint{0.970666in}{4.121437in}}{\pgfqpoint{5.699255in}{2.685432in}}%
\pgfusepath{clip}%
\pgfsetbuttcap%
\pgfsetroundjoin%
\definecolor{currentfill}{rgb}{0.000000,0.000000,0.000000}%
\pgfsetfillcolor{currentfill}%
\pgfsetlinewidth{1.003750pt}%
\definecolor{currentstroke}{rgb}{0.000000,0.000000,0.000000}%
\pgfsetstrokecolor{currentstroke}%
\pgfsetdash{}{0pt}%
\pgfpathmoveto{\pgfqpoint{3.496473in}{5.156447in}}%
\pgfpathcurveto{\pgfqpoint{3.507523in}{5.156447in}}{\pgfqpoint{3.518122in}{5.160837in}}{\pgfqpoint{3.525936in}{5.168651in}}%
\pgfpathcurveto{\pgfqpoint{3.533749in}{5.176465in}}{\pgfqpoint{3.538140in}{5.187064in}}{\pgfqpoint{3.538140in}{5.198114in}}%
\pgfpathcurveto{\pgfqpoint{3.538140in}{5.209164in}}{\pgfqpoint{3.533749in}{5.219763in}}{\pgfqpoint{3.525936in}{5.227577in}}%
\pgfpathcurveto{\pgfqpoint{3.518122in}{5.235390in}}{\pgfqpoint{3.507523in}{5.239781in}}{\pgfqpoint{3.496473in}{5.239781in}}%
\pgfpathcurveto{\pgfqpoint{3.485423in}{5.239781in}}{\pgfqpoint{3.474824in}{5.235390in}}{\pgfqpoint{3.467010in}{5.227577in}}%
\pgfpathcurveto{\pgfqpoint{3.459196in}{5.219763in}}{\pgfqpoint{3.454806in}{5.209164in}}{\pgfqpoint{3.454806in}{5.198114in}}%
\pgfpathcurveto{\pgfqpoint{3.454806in}{5.187064in}}{\pgfqpoint{3.459196in}{5.176465in}}{\pgfqpoint{3.467010in}{5.168651in}}%
\pgfpathcurveto{\pgfqpoint{3.474824in}{5.160837in}}{\pgfqpoint{3.485423in}{5.156447in}}{\pgfqpoint{3.496473in}{5.156447in}}%
\pgfpathclose%
\pgfusepath{stroke,fill}%
\end{pgfscope}%
\begin{pgfscope}%
\pgfpathrectangle{\pgfqpoint{0.970666in}{4.121437in}}{\pgfqpoint{5.699255in}{2.685432in}}%
\pgfusepath{clip}%
\pgfsetbuttcap%
\pgfsetroundjoin%
\definecolor{currentfill}{rgb}{0.000000,0.000000,0.000000}%
\pgfsetfillcolor{currentfill}%
\pgfsetlinewidth{1.003750pt}%
\definecolor{currentstroke}{rgb}{0.000000,0.000000,0.000000}%
\pgfsetstrokecolor{currentstroke}%
\pgfsetdash{}{0pt}%
\pgfpathmoveto{\pgfqpoint{3.431709in}{5.610279in}}%
\pgfpathcurveto{\pgfqpoint{3.442759in}{5.610279in}}{\pgfqpoint{3.453358in}{5.614669in}}{\pgfqpoint{3.461171in}{5.622483in}}%
\pgfpathcurveto{\pgfqpoint{3.468985in}{5.630296in}}{\pgfqpoint{3.473375in}{5.640895in}}{\pgfqpoint{3.473375in}{5.651946in}}%
\pgfpathcurveto{\pgfqpoint{3.473375in}{5.662996in}}{\pgfqpoint{3.468985in}{5.673595in}}{\pgfqpoint{3.461171in}{5.681408in}}%
\pgfpathcurveto{\pgfqpoint{3.453358in}{5.689222in}}{\pgfqpoint{3.442759in}{5.693612in}}{\pgfqpoint{3.431709in}{5.693612in}}%
\pgfpathcurveto{\pgfqpoint{3.420658in}{5.693612in}}{\pgfqpoint{3.410059in}{5.689222in}}{\pgfqpoint{3.402246in}{5.681408in}}%
\pgfpathcurveto{\pgfqpoint{3.394432in}{5.673595in}}{\pgfqpoint{3.390042in}{5.662996in}}{\pgfqpoint{3.390042in}{5.651946in}}%
\pgfpathcurveto{\pgfqpoint{3.390042in}{5.640895in}}{\pgfqpoint{3.394432in}{5.630296in}}{\pgfqpoint{3.402246in}{5.622483in}}%
\pgfpathcurveto{\pgfqpoint{3.410059in}{5.614669in}}{\pgfqpoint{3.420658in}{5.610279in}}{\pgfqpoint{3.431709in}{5.610279in}}%
\pgfpathclose%
\pgfusepath{stroke,fill}%
\end{pgfscope}%
\begin{pgfscope}%
\pgfpathrectangle{\pgfqpoint{0.970666in}{4.121437in}}{\pgfqpoint{5.699255in}{2.685432in}}%
\pgfusepath{clip}%
\pgfsetbuttcap%
\pgfsetroundjoin%
\definecolor{currentfill}{rgb}{0.000000,0.000000,0.000000}%
\pgfsetfillcolor{currentfill}%
\pgfsetlinewidth{1.003750pt}%
\definecolor{currentstroke}{rgb}{0.000000,0.000000,0.000000}%
\pgfsetstrokecolor{currentstroke}%
\pgfsetdash{}{0pt}%
\pgfpathmoveto{\pgfqpoint{4.208880in}{5.312941in}}%
\pgfpathcurveto{\pgfqpoint{4.219930in}{5.312941in}}{\pgfqpoint{4.230529in}{5.317331in}}{\pgfqpoint{4.238343in}{5.325145in}}%
\pgfpathcurveto{\pgfqpoint{4.246156in}{5.332958in}}{\pgfqpoint{4.250546in}{5.343557in}}{\pgfqpoint{4.250546in}{5.354608in}}%
\pgfpathcurveto{\pgfqpoint{4.250546in}{5.365658in}}{\pgfqpoint{4.246156in}{5.376257in}}{\pgfqpoint{4.238343in}{5.384070in}}%
\pgfpathcurveto{\pgfqpoint{4.230529in}{5.391884in}}{\pgfqpoint{4.219930in}{5.396274in}}{\pgfqpoint{4.208880in}{5.396274in}}%
\pgfpathcurveto{\pgfqpoint{4.197830in}{5.396274in}}{\pgfqpoint{4.187231in}{5.391884in}}{\pgfqpoint{4.179417in}{5.384070in}}%
\pgfpathcurveto{\pgfqpoint{4.171603in}{5.376257in}}{\pgfqpoint{4.167213in}{5.365658in}}{\pgfqpoint{4.167213in}{5.354608in}}%
\pgfpathcurveto{\pgfqpoint{4.167213in}{5.343557in}}{\pgfqpoint{4.171603in}{5.332958in}}{\pgfqpoint{4.179417in}{5.325145in}}%
\pgfpathcurveto{\pgfqpoint{4.187231in}{5.317331in}}{\pgfqpoint{4.197830in}{5.312941in}}{\pgfqpoint{4.208880in}{5.312941in}}%
\pgfpathclose%
\pgfusepath{stroke,fill}%
\end{pgfscope}%
\begin{pgfscope}%
\pgfpathrectangle{\pgfqpoint{0.970666in}{4.121437in}}{\pgfqpoint{5.699255in}{2.685432in}}%
\pgfusepath{clip}%
\pgfsetbuttcap%
\pgfsetroundjoin%
\definecolor{currentfill}{rgb}{0.000000,0.000000,0.000000}%
\pgfsetfillcolor{currentfill}%
\pgfsetlinewidth{1.003750pt}%
\definecolor{currentstroke}{rgb}{0.000000,0.000000,0.000000}%
\pgfsetstrokecolor{currentstroke}%
\pgfsetdash{}{0pt}%
\pgfpathmoveto{\pgfqpoint{5.374637in}{6.001513in}}%
\pgfpathcurveto{\pgfqpoint{5.385687in}{6.001513in}}{\pgfqpoint{5.396286in}{6.005903in}}{\pgfqpoint{5.404099in}{6.013717in}}%
\pgfpathcurveto{\pgfqpoint{5.411913in}{6.021531in}}{\pgfqpoint{5.416303in}{6.032130in}}{\pgfqpoint{5.416303in}{6.043180in}}%
\pgfpathcurveto{\pgfqpoint{5.416303in}{6.054230in}}{\pgfqpoint{5.411913in}{6.064829in}}{\pgfqpoint{5.404099in}{6.072643in}}%
\pgfpathcurveto{\pgfqpoint{5.396286in}{6.080456in}}{\pgfqpoint{5.385687in}{6.084847in}}{\pgfqpoint{5.374637in}{6.084847in}}%
\pgfpathcurveto{\pgfqpoint{5.363586in}{6.084847in}}{\pgfqpoint{5.352987in}{6.080456in}}{\pgfqpoint{5.345174in}{6.072643in}}%
\pgfpathcurveto{\pgfqpoint{5.337360in}{6.064829in}}{\pgfqpoint{5.332970in}{6.054230in}}{\pgfqpoint{5.332970in}{6.043180in}}%
\pgfpathcurveto{\pgfqpoint{5.332970in}{6.032130in}}{\pgfqpoint{5.337360in}{6.021531in}}{\pgfqpoint{5.345174in}{6.013717in}}%
\pgfpathcurveto{\pgfqpoint{5.352987in}{6.005903in}}{\pgfqpoint{5.363586in}{6.001513in}}{\pgfqpoint{5.374637in}{6.001513in}}%
\pgfpathclose%
\pgfusepath{stroke,fill}%
\end{pgfscope}%
\begin{pgfscope}%
\pgfpathrectangle{\pgfqpoint{0.970666in}{4.121437in}}{\pgfqpoint{5.699255in}{2.685432in}}%
\pgfusepath{clip}%
\pgfsetbuttcap%
\pgfsetroundjoin%
\definecolor{currentfill}{rgb}{0.000000,0.000000,0.000000}%
\pgfsetfillcolor{currentfill}%
\pgfsetlinewidth{1.003750pt}%
\definecolor{currentstroke}{rgb}{0.000000,0.000000,0.000000}%
\pgfsetstrokecolor{currentstroke}%
\pgfsetdash{}{0pt}%
\pgfpathmoveto{\pgfqpoint{5.730840in}{5.735474in}}%
\pgfpathcurveto{\pgfqpoint{5.741890in}{5.735474in}}{\pgfqpoint{5.752489in}{5.739864in}}{\pgfqpoint{5.760303in}{5.747678in}}%
\pgfpathcurveto{\pgfqpoint{5.768116in}{5.755491in}}{\pgfqpoint{5.772507in}{5.766090in}}{\pgfqpoint{5.772507in}{5.777141in}}%
\pgfpathcurveto{\pgfqpoint{5.772507in}{5.788191in}}{\pgfqpoint{5.768116in}{5.798790in}}{\pgfqpoint{5.760303in}{5.806603in}}%
\pgfpathcurveto{\pgfqpoint{5.752489in}{5.814417in}}{\pgfqpoint{5.741890in}{5.818807in}}{\pgfqpoint{5.730840in}{5.818807in}}%
\pgfpathcurveto{\pgfqpoint{5.719790in}{5.818807in}}{\pgfqpoint{5.709191in}{5.814417in}}{\pgfqpoint{5.701377in}{5.806603in}}%
\pgfpathcurveto{\pgfqpoint{5.693564in}{5.798790in}}{\pgfqpoint{5.689173in}{5.788191in}}{\pgfqpoint{5.689173in}{5.777141in}}%
\pgfpathcurveto{\pgfqpoint{5.689173in}{5.766090in}}{\pgfqpoint{5.693564in}{5.755491in}}{\pgfqpoint{5.701377in}{5.747678in}}%
\pgfpathcurveto{\pgfqpoint{5.709191in}{5.739864in}}{\pgfqpoint{5.719790in}{5.735474in}}{\pgfqpoint{5.730840in}{5.735474in}}%
\pgfpathclose%
\pgfusepath{stroke,fill}%
\end{pgfscope}%
\begin{pgfscope}%
\pgfpathrectangle{\pgfqpoint{0.970666in}{4.121437in}}{\pgfqpoint{5.699255in}{2.685432in}}%
\pgfusepath{clip}%
\pgfsetbuttcap%
\pgfsetroundjoin%
\definecolor{currentfill}{rgb}{0.000000,0.000000,0.000000}%
\pgfsetfillcolor{currentfill}%
\pgfsetlinewidth{1.003750pt}%
\definecolor{currentstroke}{rgb}{0.000000,0.000000,0.000000}%
\pgfsetstrokecolor{currentstroke}%
\pgfsetdash{}{0pt}%
\pgfpathmoveto{\pgfqpoint{5.795604in}{4.968655in}}%
\pgfpathcurveto{\pgfqpoint{5.806654in}{4.968655in}}{\pgfqpoint{5.817253in}{4.973045in}}{\pgfqpoint{5.825067in}{4.980859in}}%
\pgfpathcurveto{\pgfqpoint{5.832881in}{4.988672in}}{\pgfqpoint{5.837271in}{4.999271in}}{\pgfqpoint{5.837271in}{5.010321in}}%
\pgfpathcurveto{\pgfqpoint{5.837271in}{5.021372in}}{\pgfqpoint{5.832881in}{5.031971in}}{\pgfqpoint{5.825067in}{5.039784in}}%
\pgfpathcurveto{\pgfqpoint{5.817253in}{5.047598in}}{\pgfqpoint{5.806654in}{5.051988in}}{\pgfqpoint{5.795604in}{5.051988in}}%
\pgfpathcurveto{\pgfqpoint{5.784554in}{5.051988in}}{\pgfqpoint{5.773955in}{5.047598in}}{\pgfqpoint{5.766142in}{5.039784in}}%
\pgfpathcurveto{\pgfqpoint{5.758328in}{5.031971in}}{\pgfqpoint{5.753938in}{5.021372in}}{\pgfqpoint{5.753938in}{5.010321in}}%
\pgfpathcurveto{\pgfqpoint{5.753938in}{4.999271in}}{\pgfqpoint{5.758328in}{4.988672in}}{\pgfqpoint{5.766142in}{4.980859in}}%
\pgfpathcurveto{\pgfqpoint{5.773955in}{4.973045in}}{\pgfqpoint{5.784554in}{4.968655in}}{\pgfqpoint{5.795604in}{4.968655in}}%
\pgfpathclose%
\pgfusepath{stroke,fill}%
\end{pgfscope}%
\begin{pgfscope}%
\pgfpathrectangle{\pgfqpoint{0.970666in}{4.121437in}}{\pgfqpoint{5.699255in}{2.685432in}}%
\pgfusepath{clip}%
\pgfsetbuttcap%
\pgfsetroundjoin%
\definecolor{currentfill}{rgb}{0.000000,0.000000,0.000000}%
\pgfsetfillcolor{currentfill}%
\pgfsetlinewidth{1.003750pt}%
\definecolor{currentstroke}{rgb}{0.000000,0.000000,0.000000}%
\pgfsetstrokecolor{currentstroke}%
\pgfsetdash{}{0pt}%
\pgfpathmoveto{\pgfqpoint{2.686920in}{4.389628in}}%
\pgfpathcurveto{\pgfqpoint{2.697970in}{4.389628in}}{\pgfqpoint{2.708569in}{4.394018in}}{\pgfqpoint{2.716382in}{4.401832in}}%
\pgfpathcurveto{\pgfqpoint{2.724196in}{4.409646in}}{\pgfqpoint{2.728586in}{4.420245in}}{\pgfqpoint{2.728586in}{4.431295in}}%
\pgfpathcurveto{\pgfqpoint{2.728586in}{4.442345in}}{\pgfqpoint{2.724196in}{4.452944in}}{\pgfqpoint{2.716382in}{4.460757in}}%
\pgfpathcurveto{\pgfqpoint{2.708569in}{4.468571in}}{\pgfqpoint{2.697970in}{4.472961in}}{\pgfqpoint{2.686920in}{4.472961in}}%
\pgfpathcurveto{\pgfqpoint{2.675869in}{4.472961in}}{\pgfqpoint{2.665270in}{4.468571in}}{\pgfqpoint{2.657457in}{4.460757in}}%
\pgfpathcurveto{\pgfqpoint{2.649643in}{4.452944in}}{\pgfqpoint{2.645253in}{4.442345in}}{\pgfqpoint{2.645253in}{4.431295in}}%
\pgfpathcurveto{\pgfqpoint{2.645253in}{4.420245in}}{\pgfqpoint{2.649643in}{4.409646in}}{\pgfqpoint{2.657457in}{4.401832in}}%
\pgfpathcurveto{\pgfqpoint{2.665270in}{4.394018in}}{\pgfqpoint{2.675869in}{4.389628in}}{\pgfqpoint{2.686920in}{4.389628in}}%
\pgfpathclose%
\pgfusepath{stroke,fill}%
\end{pgfscope}%
\begin{pgfscope}%
\pgfpathrectangle{\pgfqpoint{0.970666in}{4.121437in}}{\pgfqpoint{5.699255in}{2.685432in}}%
\pgfusepath{clip}%
\pgfsetbuttcap%
\pgfsetroundjoin%
\definecolor{currentfill}{rgb}{0.000000,0.000000,0.000000}%
\pgfsetfillcolor{currentfill}%
\pgfsetlinewidth{1.003750pt}%
\definecolor{currentstroke}{rgb}{0.000000,0.000000,0.000000}%
\pgfsetstrokecolor{currentstroke}%
\pgfsetdash{}{0pt}%
\pgfpathmoveto{\pgfqpoint{4.500319in}{5.688526in}}%
\pgfpathcurveto{\pgfqpoint{4.511369in}{5.688526in}}{\pgfqpoint{4.521968in}{5.692916in}}{\pgfqpoint{4.529782in}{5.700730in}}%
\pgfpathcurveto{\pgfqpoint{4.537595in}{5.708543in}}{\pgfqpoint{4.541986in}{5.719142in}}{\pgfqpoint{4.541986in}{5.730192in}}%
\pgfpathcurveto{\pgfqpoint{4.541986in}{5.741243in}}{\pgfqpoint{4.537595in}{5.751842in}}{\pgfqpoint{4.529782in}{5.759655in}}%
\pgfpathcurveto{\pgfqpoint{4.521968in}{5.767469in}}{\pgfqpoint{4.511369in}{5.771859in}}{\pgfqpoint{4.500319in}{5.771859in}}%
\pgfpathcurveto{\pgfqpoint{4.489269in}{5.771859in}}{\pgfqpoint{4.478670in}{5.767469in}}{\pgfqpoint{4.470856in}{5.759655in}}%
\pgfpathcurveto{\pgfqpoint{4.463043in}{5.751842in}}{\pgfqpoint{4.458652in}{5.741243in}}{\pgfqpoint{4.458652in}{5.730192in}}%
\pgfpathcurveto{\pgfqpoint{4.458652in}{5.719142in}}{\pgfqpoint{4.463043in}{5.708543in}}{\pgfqpoint{4.470856in}{5.700730in}}%
\pgfpathcurveto{\pgfqpoint{4.478670in}{5.692916in}}{\pgfqpoint{4.489269in}{5.688526in}}{\pgfqpoint{4.500319in}{5.688526in}}%
\pgfpathclose%
\pgfusepath{stroke,fill}%
\end{pgfscope}%
\begin{pgfscope}%
\pgfpathrectangle{\pgfqpoint{0.970666in}{4.121437in}}{\pgfqpoint{5.699255in}{2.685432in}}%
\pgfusepath{clip}%
\pgfsetbuttcap%
\pgfsetroundjoin%
\definecolor{currentfill}{rgb}{0.000000,0.000000,0.000000}%
\pgfsetfillcolor{currentfill}%
\pgfsetlinewidth{1.003750pt}%
\definecolor{currentstroke}{rgb}{0.000000,0.000000,0.000000}%
\pgfsetstrokecolor{currentstroke}%
\pgfsetdash{}{0pt}%
\pgfpathmoveto{\pgfqpoint{2.945977in}{6.111059in}}%
\pgfpathcurveto{\pgfqpoint{2.957027in}{6.111059in}}{\pgfqpoint{2.967626in}{6.115449in}}{\pgfqpoint{2.975439in}{6.123263in}}%
\pgfpathcurveto{\pgfqpoint{2.983253in}{6.131076in}}{\pgfqpoint{2.987643in}{6.141675in}}{\pgfqpoint{2.987643in}{6.152725in}}%
\pgfpathcurveto{\pgfqpoint{2.987643in}{6.163776in}}{\pgfqpoint{2.983253in}{6.174375in}}{\pgfqpoint{2.975439in}{6.182188in}}%
\pgfpathcurveto{\pgfqpoint{2.967626in}{6.190002in}}{\pgfqpoint{2.957027in}{6.194392in}}{\pgfqpoint{2.945977in}{6.194392in}}%
\pgfpathcurveto{\pgfqpoint{2.934926in}{6.194392in}}{\pgfqpoint{2.924327in}{6.190002in}}{\pgfqpoint{2.916514in}{6.182188in}}%
\pgfpathcurveto{\pgfqpoint{2.908700in}{6.174375in}}{\pgfqpoint{2.904310in}{6.163776in}}{\pgfqpoint{2.904310in}{6.152725in}}%
\pgfpathcurveto{\pgfqpoint{2.904310in}{6.141675in}}{\pgfqpoint{2.908700in}{6.131076in}}{\pgfqpoint{2.916514in}{6.123263in}}%
\pgfpathcurveto{\pgfqpoint{2.924327in}{6.115449in}}{\pgfqpoint{2.934926in}{6.111059in}}{\pgfqpoint{2.945977in}{6.111059in}}%
\pgfpathclose%
\pgfusepath{stroke,fill}%
\end{pgfscope}%
\begin{pgfscope}%
\pgfpathrectangle{\pgfqpoint{0.970666in}{4.121437in}}{\pgfqpoint{5.699255in}{2.685432in}}%
\pgfusepath{clip}%
\pgfsetbuttcap%
\pgfsetroundjoin%
\definecolor{currentfill}{rgb}{0.000000,0.000000,0.000000}%
\pgfsetfillcolor{currentfill}%
\pgfsetlinewidth{1.003750pt}%
\definecolor{currentstroke}{rgb}{0.000000,0.000000,0.000000}%
\pgfsetstrokecolor{currentstroke}%
\pgfsetdash{}{0pt}%
\pgfpathmoveto{\pgfqpoint{3.172652in}{6.236254in}}%
\pgfpathcurveto{\pgfqpoint{3.183702in}{6.236254in}}{\pgfqpoint{3.194301in}{6.240644in}}{\pgfqpoint{3.202114in}{6.248458in}}%
\pgfpathcurveto{\pgfqpoint{3.209928in}{6.256271in}}{\pgfqpoint{3.214318in}{6.266870in}}{\pgfqpoint{3.214318in}{6.277920in}}%
\pgfpathcurveto{\pgfqpoint{3.214318in}{6.288971in}}{\pgfqpoint{3.209928in}{6.299570in}}{\pgfqpoint{3.202114in}{6.307383in}}%
\pgfpathcurveto{\pgfqpoint{3.194301in}{6.315197in}}{\pgfqpoint{3.183702in}{6.319587in}}{\pgfqpoint{3.172652in}{6.319587in}}%
\pgfpathcurveto{\pgfqpoint{3.161601in}{6.319587in}}{\pgfqpoint{3.151002in}{6.315197in}}{\pgfqpoint{3.143189in}{6.307383in}}%
\pgfpathcurveto{\pgfqpoint{3.135375in}{6.299570in}}{\pgfqpoint{3.130985in}{6.288971in}}{\pgfqpoint{3.130985in}{6.277920in}}%
\pgfpathcurveto{\pgfqpoint{3.130985in}{6.266870in}}{\pgfqpoint{3.135375in}{6.256271in}}{\pgfqpoint{3.143189in}{6.248458in}}%
\pgfpathcurveto{\pgfqpoint{3.151002in}{6.240644in}}{\pgfqpoint{3.161601in}{6.236254in}}{\pgfqpoint{3.172652in}{6.236254in}}%
\pgfpathclose%
\pgfusepath{stroke,fill}%
\end{pgfscope}%
\begin{pgfscope}%
\pgfpathrectangle{\pgfqpoint{0.970666in}{4.121437in}}{\pgfqpoint{5.699255in}{2.685432in}}%
\pgfusepath{clip}%
\pgfsetbuttcap%
\pgfsetroundjoin%
\definecolor{currentfill}{rgb}{0.000000,0.000000,0.000000}%
\pgfsetfillcolor{currentfill}%
\pgfsetlinewidth{1.003750pt}%
\definecolor{currentstroke}{rgb}{0.000000,0.000000,0.000000}%
\pgfsetstrokecolor{currentstroke}%
\pgfsetdash{}{0pt}%
\pgfpathmoveto{\pgfqpoint{4.888905in}{6.392748in}}%
\pgfpathcurveto{\pgfqpoint{4.899955in}{6.392748in}}{\pgfqpoint{4.910554in}{6.397138in}}{\pgfqpoint{4.918367in}{6.404951in}}%
\pgfpathcurveto{\pgfqpoint{4.926181in}{6.412765in}}{\pgfqpoint{4.930571in}{6.423364in}}{\pgfqpoint{4.930571in}{6.434414in}}%
\pgfpathcurveto{\pgfqpoint{4.930571in}{6.445464in}}{\pgfqpoint{4.926181in}{6.456063in}}{\pgfqpoint{4.918367in}{6.463877in}}%
\pgfpathcurveto{\pgfqpoint{4.910554in}{6.471691in}}{\pgfqpoint{4.899955in}{6.476081in}}{\pgfqpoint{4.888905in}{6.476081in}}%
\pgfpathcurveto{\pgfqpoint{4.877854in}{6.476081in}}{\pgfqpoint{4.867255in}{6.471691in}}{\pgfqpoint{4.859442in}{6.463877in}}%
\pgfpathcurveto{\pgfqpoint{4.851628in}{6.456063in}}{\pgfqpoint{4.847238in}{6.445464in}}{\pgfqpoint{4.847238in}{6.434414in}}%
\pgfpathcurveto{\pgfqpoint{4.847238in}{6.423364in}}{\pgfqpoint{4.851628in}{6.412765in}}{\pgfqpoint{4.859442in}{6.404951in}}%
\pgfpathcurveto{\pgfqpoint{4.867255in}{6.397138in}}{\pgfqpoint{4.877854in}{6.392748in}}{\pgfqpoint{4.888905in}{6.392748in}}%
\pgfpathclose%
\pgfusepath{stroke,fill}%
\end{pgfscope}%
\begin{pgfscope}%
\pgfpathrectangle{\pgfqpoint{0.970666in}{4.121437in}}{\pgfqpoint{5.699255in}{2.685432in}}%
\pgfusepath{clip}%
\pgfsetbuttcap%
\pgfsetroundjoin%
\definecolor{currentfill}{rgb}{0.000000,0.000000,0.000000}%
\pgfsetfillcolor{currentfill}%
\pgfsetlinewidth{1.003750pt}%
\definecolor{currentstroke}{rgb}{0.000000,0.000000,0.000000}%
\pgfsetstrokecolor{currentstroke}%
\pgfsetdash{}{0pt}%
\pgfpathmoveto{\pgfqpoint{4.824140in}{6.345799in}}%
\pgfpathcurveto{\pgfqpoint{4.835190in}{6.345799in}}{\pgfqpoint{4.845789in}{6.350190in}}{\pgfqpoint{4.853603in}{6.358003in}}%
\pgfpathcurveto{\pgfqpoint{4.861417in}{6.365817in}}{\pgfqpoint{4.865807in}{6.376416in}}{\pgfqpoint{4.865807in}{6.387466in}}%
\pgfpathcurveto{\pgfqpoint{4.865807in}{6.398516in}}{\pgfqpoint{4.861417in}{6.409115in}}{\pgfqpoint{4.853603in}{6.416929in}}%
\pgfpathcurveto{\pgfqpoint{4.845789in}{6.424742in}}{\pgfqpoint{4.835190in}{6.429133in}}{\pgfqpoint{4.824140in}{6.429133in}}%
\pgfpathcurveto{\pgfqpoint{4.813090in}{6.429133in}}{\pgfqpoint{4.802491in}{6.424742in}}{\pgfqpoint{4.794678in}{6.416929in}}%
\pgfpathcurveto{\pgfqpoint{4.786864in}{6.409115in}}{\pgfqpoint{4.782474in}{6.398516in}}{\pgfqpoint{4.782474in}{6.387466in}}%
\pgfpathcurveto{\pgfqpoint{4.782474in}{6.376416in}}{\pgfqpoint{4.786864in}{6.365817in}}{\pgfqpoint{4.794678in}{6.358003in}}%
\pgfpathcurveto{\pgfqpoint{4.802491in}{6.350190in}}{\pgfqpoint{4.813090in}{6.345799in}}{\pgfqpoint{4.824140in}{6.345799in}}%
\pgfpathclose%
\pgfusepath{stroke,fill}%
\end{pgfscope}%
\begin{pgfscope}%
\pgfpathrectangle{\pgfqpoint{0.970666in}{4.121437in}}{\pgfqpoint{5.699255in}{2.685432in}}%
\pgfusepath{clip}%
\pgfsetbuttcap%
\pgfsetroundjoin%
\definecolor{currentfill}{rgb}{0.000000,0.000000,0.000000}%
\pgfsetfillcolor{currentfill}%
\pgfsetlinewidth{1.003750pt}%
\definecolor{currentstroke}{rgb}{0.000000,0.000000,0.000000}%
\pgfsetstrokecolor{currentstroke}%
\pgfsetdash{}{0pt}%
\pgfpathmoveto{\pgfqpoint{3.464091in}{5.422486in}}%
\pgfpathcurveto{\pgfqpoint{3.475141in}{5.422486in}}{\pgfqpoint{3.485740in}{5.426877in}}{\pgfqpoint{3.493554in}{5.434690in}}%
\pgfpathcurveto{\pgfqpoint{3.501367in}{5.442504in}}{\pgfqpoint{3.505757in}{5.453103in}}{\pgfqpoint{3.505757in}{5.464153in}}%
\pgfpathcurveto{\pgfqpoint{3.505757in}{5.475203in}}{\pgfqpoint{3.501367in}{5.485802in}}{\pgfqpoint{3.493554in}{5.493616in}}%
\pgfpathcurveto{\pgfqpoint{3.485740in}{5.501430in}}{\pgfqpoint{3.475141in}{5.505820in}}{\pgfqpoint{3.464091in}{5.505820in}}%
\pgfpathcurveto{\pgfqpoint{3.453041in}{5.505820in}}{\pgfqpoint{3.442442in}{5.501430in}}{\pgfqpoint{3.434628in}{5.493616in}}%
\pgfpathcurveto{\pgfqpoint{3.426814in}{5.485802in}}{\pgfqpoint{3.422424in}{5.475203in}}{\pgfqpoint{3.422424in}{5.464153in}}%
\pgfpathcurveto{\pgfqpoint{3.422424in}{5.453103in}}{\pgfqpoint{3.426814in}{5.442504in}}{\pgfqpoint{3.434628in}{5.434690in}}%
\pgfpathcurveto{\pgfqpoint{3.442442in}{5.426877in}}{\pgfqpoint{3.453041in}{5.422486in}}{\pgfqpoint{3.464091in}{5.422486in}}%
\pgfpathclose%
\pgfusepath{stroke,fill}%
\end{pgfscope}%
\begin{pgfscope}%
\pgfpathrectangle{\pgfqpoint{0.970666in}{4.121437in}}{\pgfqpoint{5.699255in}{2.685432in}}%
\pgfusepath{clip}%
\pgfsetbuttcap%
\pgfsetroundjoin%
\definecolor{currentfill}{rgb}{0.000000,0.000000,0.000000}%
\pgfsetfillcolor{currentfill}%
\pgfsetlinewidth{1.003750pt}%
\definecolor{currentstroke}{rgb}{0.000000,0.000000,0.000000}%
\pgfsetstrokecolor{currentstroke}%
\pgfsetdash{}{0pt}%
\pgfpathmoveto{\pgfqpoint{2.006895in}{4.514823in}}%
\pgfpathcurveto{\pgfqpoint{2.017945in}{4.514823in}}{\pgfqpoint{2.028544in}{4.519213in}}{\pgfqpoint{2.036358in}{4.527027in}}%
\pgfpathcurveto{\pgfqpoint{2.044171in}{4.534840in}}{\pgfqpoint{2.048561in}{4.545440in}}{\pgfqpoint{2.048561in}{4.556490in}}%
\pgfpathcurveto{\pgfqpoint{2.048561in}{4.567540in}}{\pgfqpoint{2.044171in}{4.578139in}}{\pgfqpoint{2.036358in}{4.585952in}}%
\pgfpathcurveto{\pgfqpoint{2.028544in}{4.593766in}}{\pgfqpoint{2.017945in}{4.598156in}}{\pgfqpoint{2.006895in}{4.598156in}}%
\pgfpathcurveto{\pgfqpoint{1.995845in}{4.598156in}}{\pgfqpoint{1.985246in}{4.593766in}}{\pgfqpoint{1.977432in}{4.585952in}}%
\pgfpathcurveto{\pgfqpoint{1.969618in}{4.578139in}}{\pgfqpoint{1.965228in}{4.567540in}}{\pgfqpoint{1.965228in}{4.556490in}}%
\pgfpathcurveto{\pgfqpoint{1.965228in}{4.545440in}}{\pgfqpoint{1.969618in}{4.534840in}}{\pgfqpoint{1.977432in}{4.527027in}}%
\pgfpathcurveto{\pgfqpoint{1.985246in}{4.519213in}}{\pgfqpoint{1.995845in}{4.514823in}}{\pgfqpoint{2.006895in}{4.514823in}}%
\pgfpathclose%
\pgfusepath{stroke,fill}%
\end{pgfscope}%
\begin{pgfscope}%
\pgfpathrectangle{\pgfqpoint{0.970666in}{4.121437in}}{\pgfqpoint{5.699255in}{2.685432in}}%
\pgfusepath{clip}%
\pgfsetbuttcap%
\pgfsetroundjoin%
\definecolor{currentfill}{rgb}{0.000000,0.000000,0.000000}%
\pgfsetfillcolor{currentfill}%
\pgfsetlinewidth{1.003750pt}%
\definecolor{currentstroke}{rgb}{0.000000,0.000000,0.000000}%
\pgfsetstrokecolor{currentstroke}%
\pgfsetdash{}{0pt}%
\pgfpathmoveto{\pgfqpoint{2.945977in}{4.264433in}}%
\pgfpathcurveto{\pgfqpoint{2.957027in}{4.264433in}}{\pgfqpoint{2.967626in}{4.268823in}}{\pgfqpoint{2.975439in}{4.276637in}}%
\pgfpathcurveto{\pgfqpoint{2.983253in}{4.284451in}}{\pgfqpoint{2.987643in}{4.295050in}}{\pgfqpoint{2.987643in}{4.306100in}}%
\pgfpathcurveto{\pgfqpoint{2.987643in}{4.317150in}}{\pgfqpoint{2.983253in}{4.327749in}}{\pgfqpoint{2.975439in}{4.335562in}}%
\pgfpathcurveto{\pgfqpoint{2.967626in}{4.343376in}}{\pgfqpoint{2.957027in}{4.347766in}}{\pgfqpoint{2.945977in}{4.347766in}}%
\pgfpathcurveto{\pgfqpoint{2.934926in}{4.347766in}}{\pgfqpoint{2.924327in}{4.343376in}}{\pgfqpoint{2.916514in}{4.335562in}}%
\pgfpathcurveto{\pgfqpoint{2.908700in}{4.327749in}}{\pgfqpoint{2.904310in}{4.317150in}}{\pgfqpoint{2.904310in}{4.306100in}}%
\pgfpathcurveto{\pgfqpoint{2.904310in}{4.295050in}}{\pgfqpoint{2.908700in}{4.284451in}}{\pgfqpoint{2.916514in}{4.276637in}}%
\pgfpathcurveto{\pgfqpoint{2.924327in}{4.268823in}}{\pgfqpoint{2.934926in}{4.264433in}}{\pgfqpoint{2.945977in}{4.264433in}}%
\pgfpathclose%
\pgfusepath{stroke,fill}%
\end{pgfscope}%
\begin{pgfscope}%
\pgfpathrectangle{\pgfqpoint{0.970666in}{4.121437in}}{\pgfqpoint{5.699255in}{2.685432in}}%
\pgfusepath{clip}%
\pgfsetbuttcap%
\pgfsetroundjoin%
\definecolor{currentfill}{rgb}{0.000000,0.000000,0.000000}%
\pgfsetfillcolor{currentfill}%
\pgfsetlinewidth{1.003750pt}%
\definecolor{currentstroke}{rgb}{0.000000,0.000000,0.000000}%
\pgfsetstrokecolor{currentstroke}%
\pgfsetdash{}{0pt}%
\pgfpathmoveto{\pgfqpoint{5.374637in}{4.843460in}}%
\pgfpathcurveto{\pgfqpoint{5.385687in}{4.843460in}}{\pgfqpoint{5.396286in}{4.847850in}}{\pgfqpoint{5.404099in}{4.855664in}}%
\pgfpathcurveto{\pgfqpoint{5.411913in}{4.863477in}}{\pgfqpoint{5.416303in}{4.874076in}}{\pgfqpoint{5.416303in}{4.885126in}}%
\pgfpathcurveto{\pgfqpoint{5.416303in}{4.896177in}}{\pgfqpoint{5.411913in}{4.906776in}}{\pgfqpoint{5.404099in}{4.914589in}}%
\pgfpathcurveto{\pgfqpoint{5.396286in}{4.922403in}}{\pgfqpoint{5.385687in}{4.926793in}}{\pgfqpoint{5.374637in}{4.926793in}}%
\pgfpathcurveto{\pgfqpoint{5.363586in}{4.926793in}}{\pgfqpoint{5.352987in}{4.922403in}}{\pgfqpoint{5.345174in}{4.914589in}}%
\pgfpathcurveto{\pgfqpoint{5.337360in}{4.906776in}}{\pgfqpoint{5.332970in}{4.896177in}}{\pgfqpoint{5.332970in}{4.885126in}}%
\pgfpathcurveto{\pgfqpoint{5.332970in}{4.874076in}}{\pgfqpoint{5.337360in}{4.863477in}}{\pgfqpoint{5.345174in}{4.855664in}}%
\pgfpathcurveto{\pgfqpoint{5.352987in}{4.847850in}}{\pgfqpoint{5.363586in}{4.843460in}}{\pgfqpoint{5.374637in}{4.843460in}}%
\pgfpathclose%
\pgfusepath{stroke,fill}%
\end{pgfscope}%
\begin{pgfscope}%
\pgfpathrectangle{\pgfqpoint{0.970666in}{4.121437in}}{\pgfqpoint{5.699255in}{2.685432in}}%
\pgfusepath{clip}%
\pgfsetbuttcap%
\pgfsetroundjoin%
\definecolor{currentfill}{rgb}{0.000000,0.000000,0.000000}%
\pgfsetfillcolor{currentfill}%
\pgfsetlinewidth{1.003750pt}%
\definecolor{currentstroke}{rgb}{0.000000,0.000000,0.000000}%
\pgfsetstrokecolor{currentstroke}%
\pgfsetdash{}{0pt}%
\pgfpathmoveto{\pgfqpoint{1.488781in}{4.671317in}}%
\pgfpathcurveto{\pgfqpoint{1.499831in}{4.671317in}}{\pgfqpoint{1.510430in}{4.675707in}}{\pgfqpoint{1.518243in}{4.683521in}}%
\pgfpathcurveto{\pgfqpoint{1.526057in}{4.691334in}}{\pgfqpoint{1.530447in}{4.701933in}}{\pgfqpoint{1.530447in}{4.712983in}}%
\pgfpathcurveto{\pgfqpoint{1.530447in}{4.724033in}}{\pgfqpoint{1.526057in}{4.734633in}}{\pgfqpoint{1.518243in}{4.742446in}}%
\pgfpathcurveto{\pgfqpoint{1.510430in}{4.750260in}}{\pgfqpoint{1.499831in}{4.754650in}}{\pgfqpoint{1.488781in}{4.754650in}}%
\pgfpathcurveto{\pgfqpoint{1.477730in}{4.754650in}}{\pgfqpoint{1.467131in}{4.750260in}}{\pgfqpoint{1.459318in}{4.742446in}}%
\pgfpathcurveto{\pgfqpoint{1.451504in}{4.734633in}}{\pgfqpoint{1.447114in}{4.724033in}}{\pgfqpoint{1.447114in}{4.712983in}}%
\pgfpathcurveto{\pgfqpoint{1.447114in}{4.701933in}}{\pgfqpoint{1.451504in}{4.691334in}}{\pgfqpoint{1.459318in}{4.683521in}}%
\pgfpathcurveto{\pgfqpoint{1.467131in}{4.675707in}}{\pgfqpoint{1.477730in}{4.671317in}}{\pgfqpoint{1.488781in}{4.671317in}}%
\pgfpathclose%
\pgfusepath{stroke,fill}%
\end{pgfscope}%
\begin{pgfscope}%
\pgfpathrectangle{\pgfqpoint{0.970666in}{4.121437in}}{\pgfqpoint{5.699255in}{2.685432in}}%
\pgfusepath{clip}%
\pgfsetbuttcap%
\pgfsetroundjoin%
\definecolor{currentfill}{rgb}{0.000000,0.000000,0.000000}%
\pgfsetfillcolor{currentfill}%
\pgfsetlinewidth{1.003750pt}%
\definecolor{currentstroke}{rgb}{0.000000,0.000000,0.000000}%
\pgfsetstrokecolor{currentstroke}%
\pgfsetdash{}{0pt}%
\pgfpathmoveto{\pgfqpoint{2.654537in}{5.297292in}}%
\pgfpathcurveto{\pgfqpoint{2.665588in}{5.297292in}}{\pgfqpoint{2.676187in}{5.301682in}}{\pgfqpoint{2.684000in}{5.309495in}}%
\pgfpathcurveto{\pgfqpoint{2.691814in}{5.317309in}}{\pgfqpoint{2.696204in}{5.327908in}}{\pgfqpoint{2.696204in}{5.338958in}}%
\pgfpathcurveto{\pgfqpoint{2.696204in}{5.350008in}}{\pgfqpoint{2.691814in}{5.360607in}}{\pgfqpoint{2.684000in}{5.368421in}}%
\pgfpathcurveto{\pgfqpoint{2.676187in}{5.376235in}}{\pgfqpoint{2.665588in}{5.380625in}}{\pgfqpoint{2.654537in}{5.380625in}}%
\pgfpathcurveto{\pgfqpoint{2.643487in}{5.380625in}}{\pgfqpoint{2.632888in}{5.376235in}}{\pgfqpoint{2.625075in}{5.368421in}}%
\pgfpathcurveto{\pgfqpoint{2.617261in}{5.360607in}}{\pgfqpoint{2.612871in}{5.350008in}}{\pgfqpoint{2.612871in}{5.338958in}}%
\pgfpathcurveto{\pgfqpoint{2.612871in}{5.327908in}}{\pgfqpoint{2.617261in}{5.317309in}}{\pgfqpoint{2.625075in}{5.309495in}}%
\pgfpathcurveto{\pgfqpoint{2.632888in}{5.301682in}}{\pgfqpoint{2.643487in}{5.297292in}}{\pgfqpoint{2.654537in}{5.297292in}}%
\pgfpathclose%
\pgfusepath{stroke,fill}%
\end{pgfscope}%
\begin{pgfscope}%
\pgfpathrectangle{\pgfqpoint{0.970666in}{4.121437in}}{\pgfqpoint{5.699255in}{2.685432in}}%
\pgfusepath{clip}%
\pgfsetbuttcap%
\pgfsetroundjoin%
\definecolor{currentfill}{rgb}{0.000000,0.000000,0.000000}%
\pgfsetfillcolor{currentfill}%
\pgfsetlinewidth{1.003750pt}%
\definecolor{currentstroke}{rgb}{0.000000,0.000000,0.000000}%
\pgfsetstrokecolor{currentstroke}%
\pgfsetdash{}{0pt}%
\pgfpathmoveto{\pgfqpoint{2.363098in}{5.672876in}}%
\pgfpathcurveto{\pgfqpoint{2.374148in}{5.672876in}}{\pgfqpoint{2.384747in}{5.677267in}}{\pgfqpoint{2.392561in}{5.685080in}}%
\pgfpathcurveto{\pgfqpoint{2.400375in}{5.692894in}}{\pgfqpoint{2.404765in}{5.703493in}}{\pgfqpoint{2.404765in}{5.714543in}}%
\pgfpathcurveto{\pgfqpoint{2.404765in}{5.725593in}}{\pgfqpoint{2.400375in}{5.736192in}}{\pgfqpoint{2.392561in}{5.744006in}}%
\pgfpathcurveto{\pgfqpoint{2.384747in}{5.751820in}}{\pgfqpoint{2.374148in}{5.756210in}}{\pgfqpoint{2.363098in}{5.756210in}}%
\pgfpathcurveto{\pgfqpoint{2.352048in}{5.756210in}}{\pgfqpoint{2.341449in}{5.751820in}}{\pgfqpoint{2.333635in}{5.744006in}}%
\pgfpathcurveto{\pgfqpoint{2.325822in}{5.736192in}}{\pgfqpoint{2.321432in}{5.725593in}}{\pgfqpoint{2.321432in}{5.714543in}}%
\pgfpathcurveto{\pgfqpoint{2.321432in}{5.703493in}}{\pgfqpoint{2.325822in}{5.692894in}}{\pgfqpoint{2.333635in}{5.685080in}}%
\pgfpathcurveto{\pgfqpoint{2.341449in}{5.677267in}}{\pgfqpoint{2.352048in}{5.672876in}}{\pgfqpoint{2.363098in}{5.672876in}}%
\pgfpathclose%
\pgfusepath{stroke,fill}%
\end{pgfscope}%
\begin{pgfscope}%
\pgfpathrectangle{\pgfqpoint{0.970666in}{4.121437in}}{\pgfqpoint{5.699255in}{2.685432in}}%
\pgfusepath{clip}%
\pgfsetbuttcap%
\pgfsetroundjoin%
\definecolor{currentfill}{rgb}{0.000000,0.000000,0.000000}%
\pgfsetfillcolor{currentfill}%
\pgfsetlinewidth{1.003750pt}%
\definecolor{currentstroke}{rgb}{0.000000,0.000000,0.000000}%
\pgfsetstrokecolor{currentstroke}%
\pgfsetdash{}{0pt}%
\pgfpathmoveto{\pgfqpoint{3.431709in}{5.547681in}}%
\pgfpathcurveto{\pgfqpoint{3.442759in}{5.547681in}}{\pgfqpoint{3.453358in}{5.552072in}}{\pgfqpoint{3.461171in}{5.559885in}}%
\pgfpathcurveto{\pgfqpoint{3.468985in}{5.567699in}}{\pgfqpoint{3.473375in}{5.578298in}}{\pgfqpoint{3.473375in}{5.589348in}}%
\pgfpathcurveto{\pgfqpoint{3.473375in}{5.600398in}}{\pgfqpoint{3.468985in}{5.610997in}}{\pgfqpoint{3.461171in}{5.618811in}}%
\pgfpathcurveto{\pgfqpoint{3.453358in}{5.626625in}}{\pgfqpoint{3.442759in}{5.631015in}}{\pgfqpoint{3.431709in}{5.631015in}}%
\pgfpathcurveto{\pgfqpoint{3.420658in}{5.631015in}}{\pgfqpoint{3.410059in}{5.626625in}}{\pgfqpoint{3.402246in}{5.618811in}}%
\pgfpathcurveto{\pgfqpoint{3.394432in}{5.610997in}}{\pgfqpoint{3.390042in}{5.600398in}}{\pgfqpoint{3.390042in}{5.589348in}}%
\pgfpathcurveto{\pgfqpoint{3.390042in}{5.578298in}}{\pgfqpoint{3.394432in}{5.567699in}}{\pgfqpoint{3.402246in}{5.559885in}}%
\pgfpathcurveto{\pgfqpoint{3.410059in}{5.552072in}}{\pgfqpoint{3.420658in}{5.547681in}}{\pgfqpoint{3.431709in}{5.547681in}}%
\pgfpathclose%
\pgfusepath{stroke,fill}%
\end{pgfscope}%
\begin{pgfscope}%
\pgfpathrectangle{\pgfqpoint{0.970666in}{4.121437in}}{\pgfqpoint{5.699255in}{2.685432in}}%
\pgfusepath{clip}%
\pgfsetbuttcap%
\pgfsetroundjoin%
\definecolor{currentfill}{rgb}{0.000000,0.000000,0.000000}%
\pgfsetfillcolor{currentfill}%
\pgfsetlinewidth{1.003750pt}%
\definecolor{currentstroke}{rgb}{0.000000,0.000000,0.000000}%
\pgfsetstrokecolor{currentstroke}%
\pgfsetdash{}{0pt}%
\pgfpathmoveto{\pgfqpoint{3.982205in}{4.342680in}}%
\pgfpathcurveto{\pgfqpoint{3.993255in}{4.342680in}}{\pgfqpoint{4.003854in}{4.347070in}}{\pgfqpoint{4.011668in}{4.354884in}}%
\pgfpathcurveto{\pgfqpoint{4.019481in}{4.362697in}}{\pgfqpoint{4.023872in}{4.373296in}}{\pgfqpoint{4.023872in}{4.384347in}}%
\pgfpathcurveto{\pgfqpoint{4.023872in}{4.395397in}}{\pgfqpoint{4.019481in}{4.405996in}}{\pgfqpoint{4.011668in}{4.413809in}}%
\pgfpathcurveto{\pgfqpoint{4.003854in}{4.421623in}}{\pgfqpoint{3.993255in}{4.426013in}}{\pgfqpoint{3.982205in}{4.426013in}}%
\pgfpathcurveto{\pgfqpoint{3.971155in}{4.426013in}}{\pgfqpoint{3.960556in}{4.421623in}}{\pgfqpoint{3.952742in}{4.413809in}}%
\pgfpathcurveto{\pgfqpoint{3.944928in}{4.405996in}}{\pgfqpoint{3.940538in}{4.395397in}}{\pgfqpoint{3.940538in}{4.384347in}}%
\pgfpathcurveto{\pgfqpoint{3.940538in}{4.373296in}}{\pgfqpoint{3.944928in}{4.362697in}}{\pgfqpoint{3.952742in}{4.354884in}}%
\pgfpathcurveto{\pgfqpoint{3.960556in}{4.347070in}}{\pgfqpoint{3.971155in}{4.342680in}}{\pgfqpoint{3.982205in}{4.342680in}}%
\pgfpathclose%
\pgfusepath{stroke,fill}%
\end{pgfscope}%
\begin{pgfscope}%
\pgfpathrectangle{\pgfqpoint{0.970666in}{4.121437in}}{\pgfqpoint{5.699255in}{2.685432in}}%
\pgfusepath{clip}%
\pgfsetbuttcap%
\pgfsetroundjoin%
\definecolor{currentfill}{rgb}{0.000000,0.000000,0.000000}%
\pgfsetfillcolor{currentfill}%
\pgfsetlinewidth{1.003750pt}%
\definecolor{currentstroke}{rgb}{0.000000,0.000000,0.000000}%
\pgfsetstrokecolor{currentstroke}%
\pgfsetdash{}{0pt}%
\pgfpathmoveto{\pgfqpoint{3.140269in}{5.391188in}}%
\pgfpathcurveto{\pgfqpoint{3.151320in}{5.391188in}}{\pgfqpoint{3.161919in}{5.395578in}}{\pgfqpoint{3.169732in}{5.403392in}}%
\pgfpathcurveto{\pgfqpoint{3.177546in}{5.411205in}}{\pgfqpoint{3.181936in}{5.421804in}}{\pgfqpoint{3.181936in}{5.432854in}}%
\pgfpathcurveto{\pgfqpoint{3.181936in}{5.443905in}}{\pgfqpoint{3.177546in}{5.454504in}}{\pgfqpoint{3.169732in}{5.462317in}}%
\pgfpathcurveto{\pgfqpoint{3.161919in}{5.470131in}}{\pgfqpoint{3.151320in}{5.474521in}}{\pgfqpoint{3.140269in}{5.474521in}}%
\pgfpathcurveto{\pgfqpoint{3.129219in}{5.474521in}}{\pgfqpoint{3.118620in}{5.470131in}}{\pgfqpoint{3.110807in}{5.462317in}}%
\pgfpathcurveto{\pgfqpoint{3.102993in}{5.454504in}}{\pgfqpoint{3.098603in}{5.443905in}}{\pgfqpoint{3.098603in}{5.432854in}}%
\pgfpathcurveto{\pgfqpoint{3.098603in}{5.421804in}}{\pgfqpoint{3.102993in}{5.411205in}}{\pgfqpoint{3.110807in}{5.403392in}}%
\pgfpathcurveto{\pgfqpoint{3.118620in}{5.395578in}}{\pgfqpoint{3.129219in}{5.391188in}}{\pgfqpoint{3.140269in}{5.391188in}}%
\pgfpathclose%
\pgfusepath{stroke,fill}%
\end{pgfscope}%
\begin{pgfscope}%
\pgfpathrectangle{\pgfqpoint{0.970666in}{4.121437in}}{\pgfqpoint{5.699255in}{2.685432in}}%
\pgfusepath{clip}%
\pgfsetbuttcap%
\pgfsetroundjoin%
\definecolor{currentfill}{rgb}{0.000000,0.000000,0.000000}%
\pgfsetfillcolor{currentfill}%
\pgfsetlinewidth{1.003750pt}%
\definecolor{currentstroke}{rgb}{0.000000,0.000000,0.000000}%
\pgfsetstrokecolor{currentstroke}%
\pgfsetdash{}{0pt}%
\pgfpathmoveto{\pgfqpoint{3.334562in}{5.813721in}}%
\pgfpathcurveto{\pgfqpoint{3.345612in}{5.813721in}}{\pgfqpoint{3.356211in}{5.818111in}}{\pgfqpoint{3.364025in}{5.825925in}}%
\pgfpathcurveto{\pgfqpoint{3.371839in}{5.833738in}}{\pgfqpoint{3.376229in}{5.844337in}}{\pgfqpoint{3.376229in}{5.855387in}}%
\pgfpathcurveto{\pgfqpoint{3.376229in}{5.866438in}}{\pgfqpoint{3.371839in}{5.877037in}}{\pgfqpoint{3.364025in}{5.884850in}}%
\pgfpathcurveto{\pgfqpoint{3.356211in}{5.892664in}}{\pgfqpoint{3.345612in}{5.897054in}}{\pgfqpoint{3.334562in}{5.897054in}}%
\pgfpathcurveto{\pgfqpoint{3.323512in}{5.897054in}}{\pgfqpoint{3.312913in}{5.892664in}}{\pgfqpoint{3.305099in}{5.884850in}}%
\pgfpathcurveto{\pgfqpoint{3.297286in}{5.877037in}}{\pgfqpoint{3.292896in}{5.866438in}}{\pgfqpoint{3.292896in}{5.855387in}}%
\pgfpathcurveto{\pgfqpoint{3.292896in}{5.844337in}}{\pgfqpoint{3.297286in}{5.833738in}}{\pgfqpoint{3.305099in}{5.825925in}}%
\pgfpathcurveto{\pgfqpoint{3.312913in}{5.818111in}}{\pgfqpoint{3.323512in}{5.813721in}}{\pgfqpoint{3.334562in}{5.813721in}}%
\pgfpathclose%
\pgfusepath{stroke,fill}%
\end{pgfscope}%
\begin{pgfscope}%
\pgfpathrectangle{\pgfqpoint{0.970666in}{4.121437in}}{\pgfqpoint{5.699255in}{2.685432in}}%
\pgfusepath{clip}%
\pgfsetbuttcap%
\pgfsetroundjoin%
\definecolor{currentfill}{rgb}{0.000000,0.000000,0.000000}%
\pgfsetfillcolor{currentfill}%
\pgfsetlinewidth{1.003750pt}%
\definecolor{currentstroke}{rgb}{0.000000,0.000000,0.000000}%
\pgfsetstrokecolor{currentstroke}%
\pgfsetdash{}{0pt}%
\pgfpathmoveto{\pgfqpoint{5.601311in}{5.876318in}}%
\pgfpathcurveto{\pgfqpoint{5.612362in}{5.876318in}}{\pgfqpoint{5.622961in}{5.880709in}}{\pgfqpoint{5.630774in}{5.888522in}}%
\pgfpathcurveto{\pgfqpoint{5.638588in}{5.896336in}}{\pgfqpoint{5.642978in}{5.906935in}}{\pgfqpoint{5.642978in}{5.917985in}}%
\pgfpathcurveto{\pgfqpoint{5.642978in}{5.929035in}}{\pgfqpoint{5.638588in}{5.939634in}}{\pgfqpoint{5.630774in}{5.947448in}}%
\pgfpathcurveto{\pgfqpoint{5.622961in}{5.955261in}}{\pgfqpoint{5.612362in}{5.959652in}}{\pgfqpoint{5.601311in}{5.959652in}}%
\pgfpathcurveto{\pgfqpoint{5.590261in}{5.959652in}}{\pgfqpoint{5.579662in}{5.955261in}}{\pgfqpoint{5.571849in}{5.947448in}}%
\pgfpathcurveto{\pgfqpoint{5.564035in}{5.939634in}}{\pgfqpoint{5.559645in}{5.929035in}}{\pgfqpoint{5.559645in}{5.917985in}}%
\pgfpathcurveto{\pgfqpoint{5.559645in}{5.906935in}}{\pgfqpoint{5.564035in}{5.896336in}}{\pgfqpoint{5.571849in}{5.888522in}}%
\pgfpathcurveto{\pgfqpoint{5.579662in}{5.880709in}}{\pgfqpoint{5.590261in}{5.876318in}}{\pgfqpoint{5.601311in}{5.876318in}}%
\pgfpathclose%
\pgfusepath{stroke,fill}%
\end{pgfscope}%
\begin{pgfscope}%
\pgfpathrectangle{\pgfqpoint{0.970666in}{4.121437in}}{\pgfqpoint{5.699255in}{2.685432in}}%
\pgfusepath{clip}%
\pgfsetbuttcap%
\pgfsetroundjoin%
\definecolor{currentfill}{rgb}{0.000000,0.000000,0.000000}%
\pgfsetfillcolor{currentfill}%
\pgfsetlinewidth{1.003750pt}%
\definecolor{currentstroke}{rgb}{0.000000,0.000000,0.000000}%
\pgfsetstrokecolor{currentstroke}%
\pgfsetdash{}{0pt}%
\pgfpathmoveto{\pgfqpoint{5.147962in}{5.375538in}}%
\pgfpathcurveto{\pgfqpoint{5.159012in}{5.375538in}}{\pgfqpoint{5.169611in}{5.379929in}}{\pgfqpoint{5.177424in}{5.387742in}}%
\pgfpathcurveto{\pgfqpoint{5.185238in}{5.395556in}}{\pgfqpoint{5.189628in}{5.406155in}}{\pgfqpoint{5.189628in}{5.417205in}}%
\pgfpathcurveto{\pgfqpoint{5.189628in}{5.428255in}}{\pgfqpoint{5.185238in}{5.438854in}}{\pgfqpoint{5.177424in}{5.446668in}}%
\pgfpathcurveto{\pgfqpoint{5.169611in}{5.454481in}}{\pgfqpoint{5.159012in}{5.458872in}}{\pgfqpoint{5.147962in}{5.458872in}}%
\pgfpathcurveto{\pgfqpoint{5.136912in}{5.458872in}}{\pgfqpoint{5.126312in}{5.454481in}}{\pgfqpoint{5.118499in}{5.446668in}}%
\pgfpathcurveto{\pgfqpoint{5.110685in}{5.438854in}}{\pgfqpoint{5.106295in}{5.428255in}}{\pgfqpoint{5.106295in}{5.417205in}}%
\pgfpathcurveto{\pgfqpoint{5.106295in}{5.406155in}}{\pgfqpoint{5.110685in}{5.395556in}}{\pgfqpoint{5.118499in}{5.387742in}}%
\pgfpathcurveto{\pgfqpoint{5.126312in}{5.379929in}}{\pgfqpoint{5.136912in}{5.375538in}}{\pgfqpoint{5.147962in}{5.375538in}}%
\pgfpathclose%
\pgfusepath{stroke,fill}%
\end{pgfscope}%
\begin{pgfscope}%
\pgfpathrectangle{\pgfqpoint{0.970666in}{4.121437in}}{\pgfqpoint{5.699255in}{2.685432in}}%
\pgfusepath{clip}%
\pgfsetbuttcap%
\pgfsetroundjoin%
\definecolor{currentfill}{rgb}{0.000000,0.000000,0.000000}%
\pgfsetfillcolor{currentfill}%
\pgfsetlinewidth{1.003750pt}%
\definecolor{currentstroke}{rgb}{0.000000,0.000000,0.000000}%
\pgfsetstrokecolor{currentstroke}%
\pgfsetdash{}{0pt}%
\pgfpathmoveto{\pgfqpoint{2.233570in}{4.467875in}}%
\pgfpathcurveto{\pgfqpoint{2.244620in}{4.467875in}}{\pgfqpoint{2.255219in}{4.472265in}}{\pgfqpoint{2.263032in}{4.480079in}}%
\pgfpathcurveto{\pgfqpoint{2.270846in}{4.487892in}}{\pgfqpoint{2.275236in}{4.498491in}}{\pgfqpoint{2.275236in}{4.509542in}}%
\pgfpathcurveto{\pgfqpoint{2.275236in}{4.520592in}}{\pgfqpoint{2.270846in}{4.531191in}}{\pgfqpoint{2.263032in}{4.539004in}}%
\pgfpathcurveto{\pgfqpoint{2.255219in}{4.546818in}}{\pgfqpoint{2.244620in}{4.551208in}}{\pgfqpoint{2.233570in}{4.551208in}}%
\pgfpathcurveto{\pgfqpoint{2.222520in}{4.551208in}}{\pgfqpoint{2.211921in}{4.546818in}}{\pgfqpoint{2.204107in}{4.539004in}}%
\pgfpathcurveto{\pgfqpoint{2.196293in}{4.531191in}}{\pgfqpoint{2.191903in}{4.520592in}}{\pgfqpoint{2.191903in}{4.509542in}}%
\pgfpathcurveto{\pgfqpoint{2.191903in}{4.498491in}}{\pgfqpoint{2.196293in}{4.487892in}}{\pgfqpoint{2.204107in}{4.480079in}}%
\pgfpathcurveto{\pgfqpoint{2.211921in}{4.472265in}}{\pgfqpoint{2.222520in}{4.467875in}}{\pgfqpoint{2.233570in}{4.467875in}}%
\pgfpathclose%
\pgfusepath{stroke,fill}%
\end{pgfscope}%
\begin{pgfscope}%
\pgfpathrectangle{\pgfqpoint{0.970666in}{4.121437in}}{\pgfqpoint{5.699255in}{2.685432in}}%
\pgfusepath{clip}%
\pgfsetbuttcap%
\pgfsetroundjoin%
\definecolor{currentfill}{rgb}{0.000000,0.000000,0.000000}%
\pgfsetfillcolor{currentfill}%
\pgfsetlinewidth{1.003750pt}%
\definecolor{currentstroke}{rgb}{0.000000,0.000000,0.000000}%
\pgfsetstrokecolor{currentstroke}%
\pgfsetdash{}{0pt}%
\pgfpathmoveto{\pgfqpoint{1.456398in}{4.295732in}}%
\pgfpathcurveto{\pgfqpoint{1.467449in}{4.295732in}}{\pgfqpoint{1.478048in}{4.300122in}}{\pgfqpoint{1.485861in}{4.307936in}}%
\pgfpathcurveto{\pgfqpoint{1.493675in}{4.315749in}}{\pgfqpoint{1.498065in}{4.326348in}}{\pgfqpoint{1.498065in}{4.337398in}}%
\pgfpathcurveto{\pgfqpoint{1.498065in}{4.348449in}}{\pgfqpoint{1.493675in}{4.359048in}}{\pgfqpoint{1.485861in}{4.366861in}}%
\pgfpathcurveto{\pgfqpoint{1.478048in}{4.374675in}}{\pgfqpoint{1.467449in}{4.379065in}}{\pgfqpoint{1.456398in}{4.379065in}}%
\pgfpathcurveto{\pgfqpoint{1.445348in}{4.379065in}}{\pgfqpoint{1.434749in}{4.374675in}}{\pgfqpoint{1.426936in}{4.366861in}}%
\pgfpathcurveto{\pgfqpoint{1.419122in}{4.359048in}}{\pgfqpoint{1.414732in}{4.348449in}}{\pgfqpoint{1.414732in}{4.337398in}}%
\pgfpathcurveto{\pgfqpoint{1.414732in}{4.326348in}}{\pgfqpoint{1.419122in}{4.315749in}}{\pgfqpoint{1.426936in}{4.307936in}}%
\pgfpathcurveto{\pgfqpoint{1.434749in}{4.300122in}}{\pgfqpoint{1.445348in}{4.295732in}}{\pgfqpoint{1.456398in}{4.295732in}}%
\pgfpathclose%
\pgfusepath{stroke,fill}%
\end{pgfscope}%
\begin{pgfscope}%
\pgfpathrectangle{\pgfqpoint{0.970666in}{4.121437in}}{\pgfqpoint{5.699255in}{2.685432in}}%
\pgfusepath{clip}%
\pgfsetbuttcap%
\pgfsetroundjoin%
\definecolor{currentfill}{rgb}{0.000000,0.000000,0.000000}%
\pgfsetfillcolor{currentfill}%
\pgfsetlinewidth{1.003750pt}%
\definecolor{currentstroke}{rgb}{0.000000,0.000000,0.000000}%
\pgfsetstrokecolor{currentstroke}%
\pgfsetdash{}{0pt}%
\pgfpathmoveto{\pgfqpoint{1.229724in}{4.593070in}}%
\pgfpathcurveto{\pgfqpoint{1.240774in}{4.593070in}}{\pgfqpoint{1.251373in}{4.597460in}}{\pgfqpoint{1.259186in}{4.605274in}}%
\pgfpathcurveto{\pgfqpoint{1.267000in}{4.613087in}}{\pgfqpoint{1.271390in}{4.623686in}}{\pgfqpoint{1.271390in}{4.634736in}}%
\pgfpathcurveto{\pgfqpoint{1.271390in}{4.645787in}}{\pgfqpoint{1.267000in}{4.656386in}}{\pgfqpoint{1.259186in}{4.664199in}}%
\pgfpathcurveto{\pgfqpoint{1.251373in}{4.672013in}}{\pgfqpoint{1.240774in}{4.676403in}}{\pgfqpoint{1.229724in}{4.676403in}}%
\pgfpathcurveto{\pgfqpoint{1.218673in}{4.676403in}}{\pgfqpoint{1.208074in}{4.672013in}}{\pgfqpoint{1.200261in}{4.664199in}}%
\pgfpathcurveto{\pgfqpoint{1.192447in}{4.656386in}}{\pgfqpoint{1.188057in}{4.645787in}}{\pgfqpoint{1.188057in}{4.634736in}}%
\pgfpathcurveto{\pgfqpoint{1.188057in}{4.623686in}}{\pgfqpoint{1.192447in}{4.613087in}}{\pgfqpoint{1.200261in}{4.605274in}}%
\pgfpathcurveto{\pgfqpoint{1.208074in}{4.597460in}}{\pgfqpoint{1.218673in}{4.593070in}}{\pgfqpoint{1.229724in}{4.593070in}}%
\pgfpathclose%
\pgfusepath{stroke,fill}%
\end{pgfscope}%
\begin{pgfscope}%
\pgfpathrectangle{\pgfqpoint{0.970666in}{4.121437in}}{\pgfqpoint{5.699255in}{2.685432in}}%
\pgfusepath{clip}%
\pgfsetbuttcap%
\pgfsetroundjoin%
\definecolor{currentfill}{rgb}{0.000000,0.000000,0.000000}%
\pgfsetfillcolor{currentfill}%
\pgfsetlinewidth{1.003750pt}%
\definecolor{currentstroke}{rgb}{0.000000,0.000000,0.000000}%
\pgfsetstrokecolor{currentstroke}%
\pgfsetdash{}{0pt}%
\pgfpathmoveto{\pgfqpoint{5.277490in}{4.702615in}}%
\pgfpathcurveto{\pgfqpoint{5.288540in}{4.702615in}}{\pgfqpoint{5.299139in}{4.707006in}}{\pgfqpoint{5.306953in}{4.714819in}}%
\pgfpathcurveto{\pgfqpoint{5.314767in}{4.722633in}}{\pgfqpoint{5.319157in}{4.733232in}}{\pgfqpoint{5.319157in}{4.744282in}}%
\pgfpathcurveto{\pgfqpoint{5.319157in}{4.755332in}}{\pgfqpoint{5.314767in}{4.765931in}}{\pgfqpoint{5.306953in}{4.773745in}}%
\pgfpathcurveto{\pgfqpoint{5.299139in}{4.781558in}}{\pgfqpoint{5.288540in}{4.785949in}}{\pgfqpoint{5.277490in}{4.785949in}}%
\pgfpathcurveto{\pgfqpoint{5.266440in}{4.785949in}}{\pgfqpoint{5.255841in}{4.781558in}}{\pgfqpoint{5.248027in}{4.773745in}}%
\pgfpathcurveto{\pgfqpoint{5.240214in}{4.765931in}}{\pgfqpoint{5.235823in}{4.755332in}}{\pgfqpoint{5.235823in}{4.744282in}}%
\pgfpathcurveto{\pgfqpoint{5.235823in}{4.733232in}}{\pgfqpoint{5.240214in}{4.722633in}}{\pgfqpoint{5.248027in}{4.714819in}}%
\pgfpathcurveto{\pgfqpoint{5.255841in}{4.707006in}}{\pgfqpoint{5.266440in}{4.702615in}}{\pgfqpoint{5.277490in}{4.702615in}}%
\pgfpathclose%
\pgfusepath{stroke,fill}%
\end{pgfscope}%
\begin{pgfscope}%
\pgfpathrectangle{\pgfqpoint{0.970666in}{4.121437in}}{\pgfqpoint{5.699255in}{2.685432in}}%
\pgfusepath{clip}%
\pgfsetbuttcap%
\pgfsetroundjoin%
\definecolor{currentfill}{rgb}{0.000000,0.000000,0.000000}%
\pgfsetfillcolor{currentfill}%
\pgfsetlinewidth{1.003750pt}%
\definecolor{currentstroke}{rgb}{0.000000,0.000000,0.000000}%
\pgfsetstrokecolor{currentstroke}%
\pgfsetdash{}{0pt}%
\pgfpathmoveto{\pgfqpoint{4.662230in}{6.377098in}}%
\pgfpathcurveto{\pgfqpoint{4.673280in}{6.377098in}}{\pgfqpoint{4.683879in}{6.381488in}}{\pgfqpoint{4.691692in}{6.389302in}}%
\pgfpathcurveto{\pgfqpoint{4.699506in}{6.397116in}}{\pgfqpoint{4.703896in}{6.407715in}}{\pgfqpoint{4.703896in}{6.418765in}}%
\pgfpathcurveto{\pgfqpoint{4.703896in}{6.429815in}}{\pgfqpoint{4.699506in}{6.440414in}}{\pgfqpoint{4.691692in}{6.448228in}}%
\pgfpathcurveto{\pgfqpoint{4.683879in}{6.456041in}}{\pgfqpoint{4.673280in}{6.460431in}}{\pgfqpoint{4.662230in}{6.460431in}}%
\pgfpathcurveto{\pgfqpoint{4.651180in}{6.460431in}}{\pgfqpoint{4.640580in}{6.456041in}}{\pgfqpoint{4.632767in}{6.448228in}}%
\pgfpathcurveto{\pgfqpoint{4.624953in}{6.440414in}}{\pgfqpoint{4.620563in}{6.429815in}}{\pgfqpoint{4.620563in}{6.418765in}}%
\pgfpathcurveto{\pgfqpoint{4.620563in}{6.407715in}}{\pgfqpoint{4.624953in}{6.397116in}}{\pgfqpoint{4.632767in}{6.389302in}}%
\pgfpathcurveto{\pgfqpoint{4.640580in}{6.381488in}}{\pgfqpoint{4.651180in}{6.377098in}}{\pgfqpoint{4.662230in}{6.377098in}}%
\pgfpathclose%
\pgfusepath{stroke,fill}%
\end{pgfscope}%
\begin{pgfscope}%
\pgfpathrectangle{\pgfqpoint{0.970666in}{4.121437in}}{\pgfqpoint{5.699255in}{2.685432in}}%
\pgfusepath{clip}%
\pgfsetbuttcap%
\pgfsetroundjoin%
\definecolor{currentfill}{rgb}{0.000000,0.000000,0.000000}%
\pgfsetfillcolor{currentfill}%
\pgfsetlinewidth{1.003750pt}%
\definecolor{currentstroke}{rgb}{0.000000,0.000000,0.000000}%
\pgfsetstrokecolor{currentstroke}%
\pgfsetdash{}{0pt}%
\pgfpathmoveto{\pgfqpoint{4.338408in}{6.408397in}}%
\pgfpathcurveto{\pgfqpoint{4.349458in}{6.408397in}}{\pgfqpoint{4.360057in}{6.412787in}}{\pgfqpoint{4.367871in}{6.420601in}}%
\pgfpathcurveto{\pgfqpoint{4.375685in}{6.428414in}}{\pgfqpoint{4.380075in}{6.439013in}}{\pgfqpoint{4.380075in}{6.450064in}}%
\pgfpathcurveto{\pgfqpoint{4.380075in}{6.461114in}}{\pgfqpoint{4.375685in}{6.471713in}}{\pgfqpoint{4.367871in}{6.479526in}}%
\pgfpathcurveto{\pgfqpoint{4.360057in}{6.487340in}}{\pgfqpoint{4.349458in}{6.491730in}}{\pgfqpoint{4.338408in}{6.491730in}}%
\pgfpathcurveto{\pgfqpoint{4.327358in}{6.491730in}}{\pgfqpoint{4.316759in}{6.487340in}}{\pgfqpoint{4.308946in}{6.479526in}}%
\pgfpathcurveto{\pgfqpoint{4.301132in}{6.471713in}}{\pgfqpoint{4.296742in}{6.461114in}}{\pgfqpoint{4.296742in}{6.450064in}}%
\pgfpathcurveto{\pgfqpoint{4.296742in}{6.439013in}}{\pgfqpoint{4.301132in}{6.428414in}}{\pgfqpoint{4.308946in}{6.420601in}}%
\pgfpathcurveto{\pgfqpoint{4.316759in}{6.412787in}}{\pgfqpoint{4.327358in}{6.408397in}}{\pgfqpoint{4.338408in}{6.408397in}}%
\pgfpathclose%
\pgfusepath{stroke,fill}%
\end{pgfscope}%
\begin{pgfscope}%
\pgfpathrectangle{\pgfqpoint{0.970666in}{4.121437in}}{\pgfqpoint{5.699255in}{2.685432in}}%
\pgfusepath{clip}%
\pgfsetbuttcap%
\pgfsetroundjoin%
\definecolor{currentfill}{rgb}{0.000000,0.000000,0.000000}%
\pgfsetfillcolor{currentfill}%
\pgfsetlinewidth{1.003750pt}%
\definecolor{currentstroke}{rgb}{0.000000,0.000000,0.000000}%
\pgfsetstrokecolor{currentstroke}%
\pgfsetdash{}{0pt}%
\pgfpathmoveto{\pgfqpoint{4.565083in}{4.405277in}}%
\pgfpathcurveto{\pgfqpoint{4.576133in}{4.405277in}}{\pgfqpoint{4.586732in}{4.409668in}}{\pgfqpoint{4.594546in}{4.417481in}}%
\pgfpathcurveto{\pgfqpoint{4.602360in}{4.425295in}}{\pgfqpoint{4.606750in}{4.435894in}}{\pgfqpoint{4.606750in}{4.446944in}}%
\pgfpathcurveto{\pgfqpoint{4.606750in}{4.457994in}}{\pgfqpoint{4.602360in}{4.468593in}}{\pgfqpoint{4.594546in}{4.476407in}}%
\pgfpathcurveto{\pgfqpoint{4.586732in}{4.484220in}}{\pgfqpoint{4.576133in}{4.488611in}}{\pgfqpoint{4.565083in}{4.488611in}}%
\pgfpathcurveto{\pgfqpoint{4.554033in}{4.488611in}}{\pgfqpoint{4.543434in}{4.484220in}}{\pgfqpoint{4.535620in}{4.476407in}}%
\pgfpathcurveto{\pgfqpoint{4.527807in}{4.468593in}}{\pgfqpoint{4.523417in}{4.457994in}}{\pgfqpoint{4.523417in}{4.446944in}}%
\pgfpathcurveto{\pgfqpoint{4.523417in}{4.435894in}}{\pgfqpoint{4.527807in}{4.425295in}}{\pgfqpoint{4.535620in}{4.417481in}}%
\pgfpathcurveto{\pgfqpoint{4.543434in}{4.409668in}}{\pgfqpoint{4.554033in}{4.405277in}}{\pgfqpoint{4.565083in}{4.405277in}}%
\pgfpathclose%
\pgfusepath{stroke,fill}%
\end{pgfscope}%
\begin{pgfscope}%
\pgfpathrectangle{\pgfqpoint{0.970666in}{4.121437in}}{\pgfqpoint{5.699255in}{2.685432in}}%
\pgfusepath{clip}%
\pgfsetbuttcap%
\pgfsetroundjoin%
\definecolor{currentfill}{rgb}{0.000000,0.000000,0.000000}%
\pgfsetfillcolor{currentfill}%
\pgfsetlinewidth{1.003750pt}%
\definecolor{currentstroke}{rgb}{0.000000,0.000000,0.000000}%
\pgfsetstrokecolor{currentstroke}%
\pgfsetdash{}{0pt}%
\pgfpathmoveto{\pgfqpoint{4.888905in}{4.671317in}}%
\pgfpathcurveto{\pgfqpoint{4.899955in}{4.671317in}}{\pgfqpoint{4.910554in}{4.675707in}}{\pgfqpoint{4.918367in}{4.683521in}}%
\pgfpathcurveto{\pgfqpoint{4.926181in}{4.691334in}}{\pgfqpoint{4.930571in}{4.701933in}}{\pgfqpoint{4.930571in}{4.712983in}}%
\pgfpathcurveto{\pgfqpoint{4.930571in}{4.724033in}}{\pgfqpoint{4.926181in}{4.734633in}}{\pgfqpoint{4.918367in}{4.742446in}}%
\pgfpathcurveto{\pgfqpoint{4.910554in}{4.750260in}}{\pgfqpoint{4.899955in}{4.754650in}}{\pgfqpoint{4.888905in}{4.754650in}}%
\pgfpathcurveto{\pgfqpoint{4.877854in}{4.754650in}}{\pgfqpoint{4.867255in}{4.750260in}}{\pgfqpoint{4.859442in}{4.742446in}}%
\pgfpathcurveto{\pgfqpoint{4.851628in}{4.734633in}}{\pgfqpoint{4.847238in}{4.724033in}}{\pgfqpoint{4.847238in}{4.712983in}}%
\pgfpathcurveto{\pgfqpoint{4.847238in}{4.701933in}}{\pgfqpoint{4.851628in}{4.691334in}}{\pgfqpoint{4.859442in}{4.683521in}}%
\pgfpathcurveto{\pgfqpoint{4.867255in}{4.675707in}}{\pgfqpoint{4.877854in}{4.671317in}}{\pgfqpoint{4.888905in}{4.671317in}}%
\pgfpathclose%
\pgfusepath{stroke,fill}%
\end{pgfscope}%
\begin{pgfscope}%
\pgfpathrectangle{\pgfqpoint{0.970666in}{4.121437in}}{\pgfqpoint{5.699255in}{2.685432in}}%
\pgfusepath{clip}%
\pgfsetbuttcap%
\pgfsetroundjoin%
\definecolor{currentfill}{rgb}{0.000000,0.000000,0.000000}%
\pgfsetfillcolor{currentfill}%
\pgfsetlinewidth{1.003750pt}%
\definecolor{currentstroke}{rgb}{0.000000,0.000000,0.000000}%
\pgfsetstrokecolor{currentstroke}%
\pgfsetdash{}{0pt}%
\pgfpathmoveto{\pgfqpoint{5.342254in}{5.031252in}}%
\pgfpathcurveto{\pgfqpoint{5.353305in}{5.031252in}}{\pgfqpoint{5.363904in}{5.035642in}}{\pgfqpoint{5.371717in}{5.043456in}}%
\pgfpathcurveto{\pgfqpoint{5.379531in}{5.051270in}}{\pgfqpoint{5.383921in}{5.061869in}}{\pgfqpoint{5.383921in}{5.072919in}}%
\pgfpathcurveto{\pgfqpoint{5.383921in}{5.083969in}}{\pgfqpoint{5.379531in}{5.094568in}}{\pgfqpoint{5.371717in}{5.102382in}}%
\pgfpathcurveto{\pgfqpoint{5.363904in}{5.110195in}}{\pgfqpoint{5.353305in}{5.114586in}}{\pgfqpoint{5.342254in}{5.114586in}}%
\pgfpathcurveto{\pgfqpoint{5.331204in}{5.114586in}}{\pgfqpoint{5.320605in}{5.110195in}}{\pgfqpoint{5.312792in}{5.102382in}}%
\pgfpathcurveto{\pgfqpoint{5.304978in}{5.094568in}}{\pgfqpoint{5.300588in}{5.083969in}}{\pgfqpoint{5.300588in}{5.072919in}}%
\pgfpathcurveto{\pgfqpoint{5.300588in}{5.061869in}}{\pgfqpoint{5.304978in}{5.051270in}}{\pgfqpoint{5.312792in}{5.043456in}}%
\pgfpathcurveto{\pgfqpoint{5.320605in}{5.035642in}}{\pgfqpoint{5.331204in}{5.031252in}}{\pgfqpoint{5.342254in}{5.031252in}}%
\pgfpathclose%
\pgfusepath{stroke,fill}%
\end{pgfscope}%
\begin{pgfscope}%
\pgfpathrectangle{\pgfqpoint{0.970666in}{4.121437in}}{\pgfqpoint{5.699255in}{2.685432in}}%
\pgfusepath{clip}%
\pgfsetbuttcap%
\pgfsetroundjoin%
\definecolor{currentfill}{rgb}{0.000000,0.000000,0.000000}%
\pgfsetfillcolor{currentfill}%
\pgfsetlinewidth{1.003750pt}%
\definecolor{currentstroke}{rgb}{0.000000,0.000000,0.000000}%
\pgfsetstrokecolor{currentstroke}%
\pgfsetdash{}{0pt}%
\pgfpathmoveto{\pgfqpoint{2.557391in}{5.438136in}}%
\pgfpathcurveto{\pgfqpoint{2.568441in}{5.438136in}}{\pgfqpoint{2.579040in}{5.442526in}}{\pgfqpoint{2.586854in}{5.450340in}}%
\pgfpathcurveto{\pgfqpoint{2.594667in}{5.458153in}}{\pgfqpoint{2.599058in}{5.468752in}}{\pgfqpoint{2.599058in}{5.479803in}}%
\pgfpathcurveto{\pgfqpoint{2.599058in}{5.490853in}}{\pgfqpoint{2.594667in}{5.501452in}}{\pgfqpoint{2.586854in}{5.509265in}}%
\pgfpathcurveto{\pgfqpoint{2.579040in}{5.517079in}}{\pgfqpoint{2.568441in}{5.521469in}}{\pgfqpoint{2.557391in}{5.521469in}}%
\pgfpathcurveto{\pgfqpoint{2.546341in}{5.521469in}}{\pgfqpoint{2.535742in}{5.517079in}}{\pgfqpoint{2.527928in}{5.509265in}}%
\pgfpathcurveto{\pgfqpoint{2.520115in}{5.501452in}}{\pgfqpoint{2.515724in}{5.490853in}}{\pgfqpoint{2.515724in}{5.479803in}}%
\pgfpathcurveto{\pgfqpoint{2.515724in}{5.468752in}}{\pgfqpoint{2.520115in}{5.458153in}}{\pgfqpoint{2.527928in}{5.450340in}}%
\pgfpathcurveto{\pgfqpoint{2.535742in}{5.442526in}}{\pgfqpoint{2.546341in}{5.438136in}}{\pgfqpoint{2.557391in}{5.438136in}}%
\pgfpathclose%
\pgfusepath{stroke,fill}%
\end{pgfscope}%
\begin{pgfscope}%
\pgfpathrectangle{\pgfqpoint{0.970666in}{4.121437in}}{\pgfqpoint{5.699255in}{2.685432in}}%
\pgfusepath{clip}%
\pgfsetbuttcap%
\pgfsetroundjoin%
\definecolor{currentfill}{rgb}{0.000000,0.000000,0.000000}%
\pgfsetfillcolor{currentfill}%
\pgfsetlinewidth{1.003750pt}%
\definecolor{currentstroke}{rgb}{0.000000,0.000000,0.000000}%
\pgfsetstrokecolor{currentstroke}%
\pgfsetdash{}{0pt}%
\pgfpathmoveto{\pgfqpoint{5.536547in}{6.596189in}}%
\pgfpathcurveto{\pgfqpoint{5.547597in}{6.596189in}}{\pgfqpoint{5.558196in}{6.600580in}}{\pgfqpoint{5.566010in}{6.608393in}}%
\pgfpathcurveto{\pgfqpoint{5.573824in}{6.616207in}}{\pgfqpoint{5.578214in}{6.626806in}}{\pgfqpoint{5.578214in}{6.637856in}}%
\pgfpathcurveto{\pgfqpoint{5.578214in}{6.648906in}}{\pgfqpoint{5.573824in}{6.659505in}}{\pgfqpoint{5.566010in}{6.667319in}}%
\pgfpathcurveto{\pgfqpoint{5.558196in}{6.675132in}}{\pgfqpoint{5.547597in}{6.679523in}}{\pgfqpoint{5.536547in}{6.679523in}}%
\pgfpathcurveto{\pgfqpoint{5.525497in}{6.679523in}}{\pgfqpoint{5.514898in}{6.675132in}}{\pgfqpoint{5.507084in}{6.667319in}}%
\pgfpathcurveto{\pgfqpoint{5.499271in}{6.659505in}}{\pgfqpoint{5.494881in}{6.648906in}}{\pgfqpoint{5.494881in}{6.637856in}}%
\pgfpathcurveto{\pgfqpoint{5.494881in}{6.626806in}}{\pgfqpoint{5.499271in}{6.616207in}}{\pgfqpoint{5.507084in}{6.608393in}}%
\pgfpathcurveto{\pgfqpoint{5.514898in}{6.600580in}}{\pgfqpoint{5.525497in}{6.596189in}}{\pgfqpoint{5.536547in}{6.596189in}}%
\pgfpathclose%
\pgfusepath{stroke,fill}%
\end{pgfscope}%
\begin{pgfscope}%
\pgfpathrectangle{\pgfqpoint{0.970666in}{4.121437in}}{\pgfqpoint{5.699255in}{2.685432in}}%
\pgfusepath{clip}%
\pgfsetbuttcap%
\pgfsetroundjoin%
\definecolor{currentfill}{rgb}{0.000000,0.000000,0.000000}%
\pgfsetfillcolor{currentfill}%
\pgfsetlinewidth{1.003750pt}%
\definecolor{currentstroke}{rgb}{0.000000,0.000000,0.000000}%
\pgfsetstrokecolor{currentstroke}%
\pgfsetdash{}{0pt}%
\pgfpathmoveto{\pgfqpoint{5.795604in}{6.611839in}}%
\pgfpathcurveto{\pgfqpoint{5.806654in}{6.611839in}}{\pgfqpoint{5.817253in}{6.616229in}}{\pgfqpoint{5.825067in}{6.624043in}}%
\pgfpathcurveto{\pgfqpoint{5.832881in}{6.631856in}}{\pgfqpoint{5.837271in}{6.642455in}}{\pgfqpoint{5.837271in}{6.653505in}}%
\pgfpathcurveto{\pgfqpoint{5.837271in}{6.664556in}}{\pgfqpoint{5.832881in}{6.675155in}}{\pgfqpoint{5.825067in}{6.682968in}}%
\pgfpathcurveto{\pgfqpoint{5.817253in}{6.690782in}}{\pgfqpoint{5.806654in}{6.695172in}}{\pgfqpoint{5.795604in}{6.695172in}}%
\pgfpathcurveto{\pgfqpoint{5.784554in}{6.695172in}}{\pgfqpoint{5.773955in}{6.690782in}}{\pgfqpoint{5.766142in}{6.682968in}}%
\pgfpathcurveto{\pgfqpoint{5.758328in}{6.675155in}}{\pgfqpoint{5.753938in}{6.664556in}}{\pgfqpoint{5.753938in}{6.653505in}}%
\pgfpathcurveto{\pgfqpoint{5.753938in}{6.642455in}}{\pgfqpoint{5.758328in}{6.631856in}}{\pgfqpoint{5.766142in}{6.624043in}}%
\pgfpathcurveto{\pgfqpoint{5.773955in}{6.616229in}}{\pgfqpoint{5.784554in}{6.611839in}}{\pgfqpoint{5.795604in}{6.611839in}}%
\pgfpathclose%
\pgfusepath{stroke,fill}%
\end{pgfscope}%
\begin{pgfscope}%
\pgfpathrectangle{\pgfqpoint{0.970666in}{4.121437in}}{\pgfqpoint{5.699255in}{2.685432in}}%
\pgfusepath{clip}%
\pgfsetbuttcap%
\pgfsetroundjoin%
\definecolor{currentfill}{rgb}{0.000000,0.000000,0.000000}%
\pgfsetfillcolor{currentfill}%
\pgfsetlinewidth{1.003750pt}%
\definecolor{currentstroke}{rgb}{0.000000,0.000000,0.000000}%
\pgfsetstrokecolor{currentstroke}%
\pgfsetdash{}{0pt}%
\pgfpathmoveto{\pgfqpoint{6.410865in}{6.298851in}}%
\pgfpathcurveto{\pgfqpoint{6.421915in}{6.298851in}}{\pgfqpoint{6.432514in}{6.303242in}}{\pgfqpoint{6.440328in}{6.311055in}}%
\pgfpathcurveto{\pgfqpoint{6.448141in}{6.318869in}}{\pgfqpoint{6.452531in}{6.329468in}}{\pgfqpoint{6.452531in}{6.340518in}}%
\pgfpathcurveto{\pgfqpoint{6.452531in}{6.351568in}}{\pgfqpoint{6.448141in}{6.362167in}}{\pgfqpoint{6.440328in}{6.369981in}}%
\pgfpathcurveto{\pgfqpoint{6.432514in}{6.377794in}}{\pgfqpoint{6.421915in}{6.382185in}}{\pgfqpoint{6.410865in}{6.382185in}}%
\pgfpathcurveto{\pgfqpoint{6.399815in}{6.382185in}}{\pgfqpoint{6.389216in}{6.377794in}}{\pgfqpoint{6.381402in}{6.369981in}}%
\pgfpathcurveto{\pgfqpoint{6.373588in}{6.362167in}}{\pgfqpoint{6.369198in}{6.351568in}}{\pgfqpoint{6.369198in}{6.340518in}}%
\pgfpathcurveto{\pgfqpoint{6.369198in}{6.329468in}}{\pgfqpoint{6.373588in}{6.318869in}}{\pgfqpoint{6.381402in}{6.311055in}}%
\pgfpathcurveto{\pgfqpoint{6.389216in}{6.303242in}}{\pgfqpoint{6.399815in}{6.298851in}}{\pgfqpoint{6.410865in}{6.298851in}}%
\pgfpathclose%
\pgfusepath{stroke,fill}%
\end{pgfscope}%
\begin{pgfscope}%
\pgfpathrectangle{\pgfqpoint{0.970666in}{4.121437in}}{\pgfqpoint{5.699255in}{2.685432in}}%
\pgfusepath{clip}%
\pgfsetbuttcap%
\pgfsetroundjoin%
\definecolor{currentfill}{rgb}{0.000000,0.000000,0.000000}%
\pgfsetfillcolor{currentfill}%
\pgfsetlinewidth{1.003750pt}%
\definecolor{currentstroke}{rgb}{0.000000,0.000000,0.000000}%
\pgfsetstrokecolor{currentstroke}%
\pgfsetdash{}{0pt}%
\pgfpathmoveto{\pgfqpoint{2.686920in}{6.001513in}}%
\pgfpathcurveto{\pgfqpoint{2.697970in}{6.001513in}}{\pgfqpoint{2.708569in}{6.005903in}}{\pgfqpoint{2.716382in}{6.013717in}}%
\pgfpathcurveto{\pgfqpoint{2.724196in}{6.021531in}}{\pgfqpoint{2.728586in}{6.032130in}}{\pgfqpoint{2.728586in}{6.043180in}}%
\pgfpathcurveto{\pgfqpoint{2.728586in}{6.054230in}}{\pgfqpoint{2.724196in}{6.064829in}}{\pgfqpoint{2.716382in}{6.072643in}}%
\pgfpathcurveto{\pgfqpoint{2.708569in}{6.080456in}}{\pgfqpoint{2.697970in}{6.084847in}}{\pgfqpoint{2.686920in}{6.084847in}}%
\pgfpathcurveto{\pgfqpoint{2.675869in}{6.084847in}}{\pgfqpoint{2.665270in}{6.080456in}}{\pgfqpoint{2.657457in}{6.072643in}}%
\pgfpathcurveto{\pgfqpoint{2.649643in}{6.064829in}}{\pgfqpoint{2.645253in}{6.054230in}}{\pgfqpoint{2.645253in}{6.043180in}}%
\pgfpathcurveto{\pgfqpoint{2.645253in}{6.032130in}}{\pgfqpoint{2.649643in}{6.021531in}}{\pgfqpoint{2.657457in}{6.013717in}}%
\pgfpathcurveto{\pgfqpoint{2.665270in}{6.005903in}}{\pgfqpoint{2.675869in}{6.001513in}}{\pgfqpoint{2.686920in}{6.001513in}}%
\pgfpathclose%
\pgfusepath{stroke,fill}%
\end{pgfscope}%
\begin{pgfscope}%
\pgfpathrectangle{\pgfqpoint{0.970666in}{4.121437in}}{\pgfqpoint{5.699255in}{2.685432in}}%
\pgfusepath{clip}%
\pgfsetbuttcap%
\pgfsetroundjoin%
\definecolor{currentfill}{rgb}{0.000000,0.000000,0.000000}%
\pgfsetfillcolor{currentfill}%
\pgfsetlinewidth{1.003750pt}%
\definecolor{currentstroke}{rgb}{0.000000,0.000000,0.000000}%
\pgfsetstrokecolor{currentstroke}%
\pgfsetdash{}{0pt}%
\pgfpathmoveto{\pgfqpoint{1.456398in}{5.876318in}}%
\pgfpathcurveto{\pgfqpoint{1.467449in}{5.876318in}}{\pgfqpoint{1.478048in}{5.880709in}}{\pgfqpoint{1.485861in}{5.888522in}}%
\pgfpathcurveto{\pgfqpoint{1.493675in}{5.896336in}}{\pgfqpoint{1.498065in}{5.906935in}}{\pgfqpoint{1.498065in}{5.917985in}}%
\pgfpathcurveto{\pgfqpoint{1.498065in}{5.929035in}}{\pgfqpoint{1.493675in}{5.939634in}}{\pgfqpoint{1.485861in}{5.947448in}}%
\pgfpathcurveto{\pgfqpoint{1.478048in}{5.955261in}}{\pgfqpoint{1.467449in}{5.959652in}}{\pgfqpoint{1.456398in}{5.959652in}}%
\pgfpathcurveto{\pgfqpoint{1.445348in}{5.959652in}}{\pgfqpoint{1.434749in}{5.955261in}}{\pgfqpoint{1.426936in}{5.947448in}}%
\pgfpathcurveto{\pgfqpoint{1.419122in}{5.939634in}}{\pgfqpoint{1.414732in}{5.929035in}}{\pgfqpoint{1.414732in}{5.917985in}}%
\pgfpathcurveto{\pgfqpoint{1.414732in}{5.906935in}}{\pgfqpoint{1.419122in}{5.896336in}}{\pgfqpoint{1.426936in}{5.888522in}}%
\pgfpathcurveto{\pgfqpoint{1.434749in}{5.880709in}}{\pgfqpoint{1.445348in}{5.876318in}}{\pgfqpoint{1.456398in}{5.876318in}}%
\pgfpathclose%
\pgfusepath{stroke,fill}%
\end{pgfscope}%
\begin{pgfscope}%
\pgfpathrectangle{\pgfqpoint{0.970666in}{4.121437in}}{\pgfqpoint{5.699255in}{2.685432in}}%
\pgfusepath{clip}%
\pgfsetbuttcap%
\pgfsetroundjoin%
\definecolor{currentfill}{rgb}{0.000000,0.000000,1.000000}%
\pgfsetfillcolor{currentfill}%
\pgfsetlinewidth{1.003750pt}%
\definecolor{currentstroke}{rgb}{0.000000,0.000000,1.000000}%
\pgfsetstrokecolor{currentstroke}%
\pgfsetdash{}{0pt}%
\pgfpathmoveto{\pgfqpoint{2.136423in}{4.640018in}}%
\pgfpathcurveto{\pgfqpoint{2.147473in}{4.640018in}}{\pgfqpoint{2.158072in}{4.644408in}}{\pgfqpoint{2.165886in}{4.652222in}}%
\pgfpathcurveto{\pgfqpoint{2.173700in}{4.660035in}}{\pgfqpoint{2.178090in}{4.670634in}}{\pgfqpoint{2.178090in}{4.681685in}}%
\pgfpathcurveto{\pgfqpoint{2.178090in}{4.692735in}}{\pgfqpoint{2.173700in}{4.703334in}}{\pgfqpoint{2.165886in}{4.711147in}}%
\pgfpathcurveto{\pgfqpoint{2.158072in}{4.718961in}}{\pgfqpoint{2.147473in}{4.723351in}}{\pgfqpoint{2.136423in}{4.723351in}}%
\pgfpathcurveto{\pgfqpoint{2.125373in}{4.723351in}}{\pgfqpoint{2.114774in}{4.718961in}}{\pgfqpoint{2.106960in}{4.711147in}}%
\pgfpathcurveto{\pgfqpoint{2.099147in}{4.703334in}}{\pgfqpoint{2.094757in}{4.692735in}}{\pgfqpoint{2.094757in}{4.681685in}}%
\pgfpathcurveto{\pgfqpoint{2.094757in}{4.670634in}}{\pgfqpoint{2.099147in}{4.660035in}}{\pgfqpoint{2.106960in}{4.652222in}}%
\pgfpathcurveto{\pgfqpoint{2.114774in}{4.644408in}}{\pgfqpoint{2.125373in}{4.640018in}}{\pgfqpoint{2.136423in}{4.640018in}}%
\pgfpathclose%
\pgfusepath{stroke,fill}%
\end{pgfscope}%
\begin{pgfscope}%
\pgfsetbuttcap%
\pgfsetroundjoin%
\definecolor{currentfill}{rgb}{0.000000,0.000000,0.000000}%
\pgfsetfillcolor{currentfill}%
\pgfsetlinewidth{0.803000pt}%
\definecolor{currentstroke}{rgb}{0.000000,0.000000,0.000000}%
\pgfsetstrokecolor{currentstroke}%
\pgfsetdash{}{0pt}%
\pgfsys@defobject{currentmarker}{\pgfqpoint{0.000000in}{-0.048611in}}{\pgfqpoint{0.000000in}{0.000000in}}{%
\pgfpathmoveto{\pgfqpoint{0.000000in}{0.000000in}}%
\pgfpathlineto{\pgfqpoint{0.000000in}{-0.048611in}}%
\pgfusepath{stroke,fill}%
}%
\begin{pgfscope}%
\pgfsys@transformshift{1.132577in}{4.121437in}%
\pgfsys@useobject{currentmarker}{}%
\end{pgfscope}%
\end{pgfscope}%
\begin{pgfscope}%
\definecolor{textcolor}{rgb}{0.000000,0.000000,0.000000}%
\pgfsetstrokecolor{textcolor}%
\pgfsetfillcolor{textcolor}%
\pgftext[x=1.132577in,y=4.024215in,,top]{\color{textcolor}\rmfamily\fontsize{10.000000}{12.000000}\selectfont \(\displaystyle 0\)}%
\end{pgfscope}%
\begin{pgfscope}%
\pgfsetbuttcap%
\pgfsetroundjoin%
\definecolor{currentfill}{rgb}{0.000000,0.000000,0.000000}%
\pgfsetfillcolor{currentfill}%
\pgfsetlinewidth{0.803000pt}%
\definecolor{currentstroke}{rgb}{0.000000,0.000000,0.000000}%
\pgfsetstrokecolor{currentstroke}%
\pgfsetdash{}{0pt}%
\pgfsys@defobject{currentmarker}{\pgfqpoint{0.000000in}{-0.048611in}}{\pgfqpoint{0.000000in}{0.000000in}}{%
\pgfpathmoveto{\pgfqpoint{0.000000in}{0.000000in}}%
\pgfpathlineto{\pgfqpoint{0.000000in}{-0.048611in}}%
\pgfusepath{stroke,fill}%
}%
\begin{pgfscope}%
\pgfsys@transformshift{1.780220in}{4.121437in}%
\pgfsys@useobject{currentmarker}{}%
\end{pgfscope}%
\end{pgfscope}%
\begin{pgfscope}%
\definecolor{textcolor}{rgb}{0.000000,0.000000,0.000000}%
\pgfsetstrokecolor{textcolor}%
\pgfsetfillcolor{textcolor}%
\pgftext[x=1.780220in,y=4.024215in,,top]{\color{textcolor}\rmfamily\fontsize{10.000000}{12.000000}\selectfont \(\displaystyle 20\)}%
\end{pgfscope}%
\begin{pgfscope}%
\pgfsetbuttcap%
\pgfsetroundjoin%
\definecolor{currentfill}{rgb}{0.000000,0.000000,0.000000}%
\pgfsetfillcolor{currentfill}%
\pgfsetlinewidth{0.803000pt}%
\definecolor{currentstroke}{rgb}{0.000000,0.000000,0.000000}%
\pgfsetstrokecolor{currentstroke}%
\pgfsetdash{}{0pt}%
\pgfsys@defobject{currentmarker}{\pgfqpoint{0.000000in}{-0.048611in}}{\pgfqpoint{0.000000in}{0.000000in}}{%
\pgfpathmoveto{\pgfqpoint{0.000000in}{0.000000in}}%
\pgfpathlineto{\pgfqpoint{0.000000in}{-0.048611in}}%
\pgfusepath{stroke,fill}%
}%
\begin{pgfscope}%
\pgfsys@transformshift{2.427862in}{4.121437in}%
\pgfsys@useobject{currentmarker}{}%
\end{pgfscope}%
\end{pgfscope}%
\begin{pgfscope}%
\definecolor{textcolor}{rgb}{0.000000,0.000000,0.000000}%
\pgfsetstrokecolor{textcolor}%
\pgfsetfillcolor{textcolor}%
\pgftext[x=2.427862in,y=4.024215in,,top]{\color{textcolor}\rmfamily\fontsize{10.000000}{12.000000}\selectfont \(\displaystyle 40\)}%
\end{pgfscope}%
\begin{pgfscope}%
\pgfsetbuttcap%
\pgfsetroundjoin%
\definecolor{currentfill}{rgb}{0.000000,0.000000,0.000000}%
\pgfsetfillcolor{currentfill}%
\pgfsetlinewidth{0.803000pt}%
\definecolor{currentstroke}{rgb}{0.000000,0.000000,0.000000}%
\pgfsetstrokecolor{currentstroke}%
\pgfsetdash{}{0pt}%
\pgfsys@defobject{currentmarker}{\pgfqpoint{0.000000in}{-0.048611in}}{\pgfqpoint{0.000000in}{0.000000in}}{%
\pgfpathmoveto{\pgfqpoint{0.000000in}{0.000000in}}%
\pgfpathlineto{\pgfqpoint{0.000000in}{-0.048611in}}%
\pgfusepath{stroke,fill}%
}%
\begin{pgfscope}%
\pgfsys@transformshift{3.075505in}{4.121437in}%
\pgfsys@useobject{currentmarker}{}%
\end{pgfscope}%
\end{pgfscope}%
\begin{pgfscope}%
\definecolor{textcolor}{rgb}{0.000000,0.000000,0.000000}%
\pgfsetstrokecolor{textcolor}%
\pgfsetfillcolor{textcolor}%
\pgftext[x=3.075505in,y=4.024215in,,top]{\color{textcolor}\rmfamily\fontsize{10.000000}{12.000000}\selectfont \(\displaystyle 60\)}%
\end{pgfscope}%
\begin{pgfscope}%
\pgfsetbuttcap%
\pgfsetroundjoin%
\definecolor{currentfill}{rgb}{0.000000,0.000000,0.000000}%
\pgfsetfillcolor{currentfill}%
\pgfsetlinewidth{0.803000pt}%
\definecolor{currentstroke}{rgb}{0.000000,0.000000,0.000000}%
\pgfsetstrokecolor{currentstroke}%
\pgfsetdash{}{0pt}%
\pgfsys@defobject{currentmarker}{\pgfqpoint{0.000000in}{-0.048611in}}{\pgfqpoint{0.000000in}{0.000000in}}{%
\pgfpathmoveto{\pgfqpoint{0.000000in}{0.000000in}}%
\pgfpathlineto{\pgfqpoint{0.000000in}{-0.048611in}}%
\pgfusepath{stroke,fill}%
}%
\begin{pgfscope}%
\pgfsys@transformshift{3.723148in}{4.121437in}%
\pgfsys@useobject{currentmarker}{}%
\end{pgfscope}%
\end{pgfscope}%
\begin{pgfscope}%
\definecolor{textcolor}{rgb}{0.000000,0.000000,0.000000}%
\pgfsetstrokecolor{textcolor}%
\pgfsetfillcolor{textcolor}%
\pgftext[x=3.723148in,y=4.024215in,,top]{\color{textcolor}\rmfamily\fontsize{10.000000}{12.000000}\selectfont \(\displaystyle 80\)}%
\end{pgfscope}%
\begin{pgfscope}%
\pgfsetbuttcap%
\pgfsetroundjoin%
\definecolor{currentfill}{rgb}{0.000000,0.000000,0.000000}%
\pgfsetfillcolor{currentfill}%
\pgfsetlinewidth{0.803000pt}%
\definecolor{currentstroke}{rgb}{0.000000,0.000000,0.000000}%
\pgfsetstrokecolor{currentstroke}%
\pgfsetdash{}{0pt}%
\pgfsys@defobject{currentmarker}{\pgfqpoint{0.000000in}{-0.048611in}}{\pgfqpoint{0.000000in}{0.000000in}}{%
\pgfpathmoveto{\pgfqpoint{0.000000in}{0.000000in}}%
\pgfpathlineto{\pgfqpoint{0.000000in}{-0.048611in}}%
\pgfusepath{stroke,fill}%
}%
\begin{pgfscope}%
\pgfsys@transformshift{4.370790in}{4.121437in}%
\pgfsys@useobject{currentmarker}{}%
\end{pgfscope}%
\end{pgfscope}%
\begin{pgfscope}%
\definecolor{textcolor}{rgb}{0.000000,0.000000,0.000000}%
\pgfsetstrokecolor{textcolor}%
\pgfsetfillcolor{textcolor}%
\pgftext[x=4.370790in,y=4.024215in,,top]{\color{textcolor}\rmfamily\fontsize{10.000000}{12.000000}\selectfont \(\displaystyle 100\)}%
\end{pgfscope}%
\begin{pgfscope}%
\pgfsetbuttcap%
\pgfsetroundjoin%
\definecolor{currentfill}{rgb}{0.000000,0.000000,0.000000}%
\pgfsetfillcolor{currentfill}%
\pgfsetlinewidth{0.803000pt}%
\definecolor{currentstroke}{rgb}{0.000000,0.000000,0.000000}%
\pgfsetstrokecolor{currentstroke}%
\pgfsetdash{}{0pt}%
\pgfsys@defobject{currentmarker}{\pgfqpoint{0.000000in}{-0.048611in}}{\pgfqpoint{0.000000in}{0.000000in}}{%
\pgfpathmoveto{\pgfqpoint{0.000000in}{0.000000in}}%
\pgfpathlineto{\pgfqpoint{0.000000in}{-0.048611in}}%
\pgfusepath{stroke,fill}%
}%
\begin{pgfscope}%
\pgfsys@transformshift{5.018433in}{4.121437in}%
\pgfsys@useobject{currentmarker}{}%
\end{pgfscope}%
\end{pgfscope}%
\begin{pgfscope}%
\definecolor{textcolor}{rgb}{0.000000,0.000000,0.000000}%
\pgfsetstrokecolor{textcolor}%
\pgfsetfillcolor{textcolor}%
\pgftext[x=5.018433in,y=4.024215in,,top]{\color{textcolor}\rmfamily\fontsize{10.000000}{12.000000}\selectfont \(\displaystyle 120\)}%
\end{pgfscope}%
\begin{pgfscope}%
\pgfsetbuttcap%
\pgfsetroundjoin%
\definecolor{currentfill}{rgb}{0.000000,0.000000,0.000000}%
\pgfsetfillcolor{currentfill}%
\pgfsetlinewidth{0.803000pt}%
\definecolor{currentstroke}{rgb}{0.000000,0.000000,0.000000}%
\pgfsetstrokecolor{currentstroke}%
\pgfsetdash{}{0pt}%
\pgfsys@defobject{currentmarker}{\pgfqpoint{0.000000in}{-0.048611in}}{\pgfqpoint{0.000000in}{0.000000in}}{%
\pgfpathmoveto{\pgfqpoint{0.000000in}{0.000000in}}%
\pgfpathlineto{\pgfqpoint{0.000000in}{-0.048611in}}%
\pgfusepath{stroke,fill}%
}%
\begin{pgfscope}%
\pgfsys@transformshift{5.666076in}{4.121437in}%
\pgfsys@useobject{currentmarker}{}%
\end{pgfscope}%
\end{pgfscope}%
\begin{pgfscope}%
\definecolor{textcolor}{rgb}{0.000000,0.000000,0.000000}%
\pgfsetstrokecolor{textcolor}%
\pgfsetfillcolor{textcolor}%
\pgftext[x=5.666076in,y=4.024215in,,top]{\color{textcolor}\rmfamily\fontsize{10.000000}{12.000000}\selectfont \(\displaystyle 140\)}%
\end{pgfscope}%
\begin{pgfscope}%
\pgfsetbuttcap%
\pgfsetroundjoin%
\definecolor{currentfill}{rgb}{0.000000,0.000000,0.000000}%
\pgfsetfillcolor{currentfill}%
\pgfsetlinewidth{0.803000pt}%
\definecolor{currentstroke}{rgb}{0.000000,0.000000,0.000000}%
\pgfsetstrokecolor{currentstroke}%
\pgfsetdash{}{0pt}%
\pgfsys@defobject{currentmarker}{\pgfqpoint{0.000000in}{-0.048611in}}{\pgfqpoint{0.000000in}{0.000000in}}{%
\pgfpathmoveto{\pgfqpoint{0.000000in}{0.000000in}}%
\pgfpathlineto{\pgfqpoint{0.000000in}{-0.048611in}}%
\pgfusepath{stroke,fill}%
}%
\begin{pgfscope}%
\pgfsys@transformshift{6.313718in}{4.121437in}%
\pgfsys@useobject{currentmarker}{}%
\end{pgfscope}%
\end{pgfscope}%
\begin{pgfscope}%
\definecolor{textcolor}{rgb}{0.000000,0.000000,0.000000}%
\pgfsetstrokecolor{textcolor}%
\pgfsetfillcolor{textcolor}%
\pgftext[x=6.313718in,y=4.024215in,,top]{\color{textcolor}\rmfamily\fontsize{10.000000}{12.000000}\selectfont \(\displaystyle 160\)}%
\end{pgfscope}%
\begin{pgfscope}%
\pgfsetbuttcap%
\pgfsetroundjoin%
\definecolor{currentfill}{rgb}{0.000000,0.000000,0.000000}%
\pgfsetfillcolor{currentfill}%
\pgfsetlinewidth{0.803000pt}%
\definecolor{currentstroke}{rgb}{0.000000,0.000000,0.000000}%
\pgfsetstrokecolor{currentstroke}%
\pgfsetdash{}{0pt}%
\pgfsys@defobject{currentmarker}{\pgfqpoint{-0.048611in}{0.000000in}}{\pgfqpoint{0.000000in}{0.000000in}}{%
\pgfpathmoveto{\pgfqpoint{0.000000in}{0.000000in}}%
\pgfpathlineto{\pgfqpoint{-0.048611in}{0.000000in}}%
\pgfusepath{stroke,fill}%
}%
\begin{pgfscope}%
\pgfsys@transformshift{0.970666in}{4.243502in}%
\pgfsys@useobject{currentmarker}{}%
\end{pgfscope}%
\end{pgfscope}%
\begin{pgfscope}%
\definecolor{textcolor}{rgb}{0.000000,0.000000,0.000000}%
\pgfsetstrokecolor{textcolor}%
\pgfsetfillcolor{textcolor}%
\pgftext[x=0.804000in, y=4.190741in, left, base]{\color{textcolor}\rmfamily\fontsize{10.000000}{12.000000}\selectfont \(\displaystyle 0\)}%
\end{pgfscope}%
\begin{pgfscope}%
\pgfsetbuttcap%
\pgfsetroundjoin%
\definecolor{currentfill}{rgb}{0.000000,0.000000,0.000000}%
\pgfsetfillcolor{currentfill}%
\pgfsetlinewidth{0.803000pt}%
\definecolor{currentstroke}{rgb}{0.000000,0.000000,0.000000}%
\pgfsetstrokecolor{currentstroke}%
\pgfsetdash{}{0pt}%
\pgfsys@defobject{currentmarker}{\pgfqpoint{-0.048611in}{0.000000in}}{\pgfqpoint{0.000000in}{0.000000in}}{%
\pgfpathmoveto{\pgfqpoint{0.000000in}{0.000000in}}%
\pgfpathlineto{\pgfqpoint{-0.048611in}{0.000000in}}%
\pgfusepath{stroke,fill}%
}%
\begin{pgfscope}%
\pgfsys@transformshift{0.970666in}{4.556490in}%
\pgfsys@useobject{currentmarker}{}%
\end{pgfscope}%
\end{pgfscope}%
\begin{pgfscope}%
\definecolor{textcolor}{rgb}{0.000000,0.000000,0.000000}%
\pgfsetstrokecolor{textcolor}%
\pgfsetfillcolor{textcolor}%
\pgftext[x=0.734555in, y=4.503728in, left, base]{\color{textcolor}\rmfamily\fontsize{10.000000}{12.000000}\selectfont \(\displaystyle 20\)}%
\end{pgfscope}%
\begin{pgfscope}%
\pgfsetbuttcap%
\pgfsetroundjoin%
\definecolor{currentfill}{rgb}{0.000000,0.000000,0.000000}%
\pgfsetfillcolor{currentfill}%
\pgfsetlinewidth{0.803000pt}%
\definecolor{currentstroke}{rgb}{0.000000,0.000000,0.000000}%
\pgfsetstrokecolor{currentstroke}%
\pgfsetdash{}{0pt}%
\pgfsys@defobject{currentmarker}{\pgfqpoint{-0.048611in}{0.000000in}}{\pgfqpoint{0.000000in}{0.000000in}}{%
\pgfpathmoveto{\pgfqpoint{0.000000in}{0.000000in}}%
\pgfpathlineto{\pgfqpoint{-0.048611in}{0.000000in}}%
\pgfusepath{stroke,fill}%
}%
\begin{pgfscope}%
\pgfsys@transformshift{0.970666in}{4.869477in}%
\pgfsys@useobject{currentmarker}{}%
\end{pgfscope}%
\end{pgfscope}%
\begin{pgfscope}%
\definecolor{textcolor}{rgb}{0.000000,0.000000,0.000000}%
\pgfsetstrokecolor{textcolor}%
\pgfsetfillcolor{textcolor}%
\pgftext[x=0.734555in, y=4.816716in, left, base]{\color{textcolor}\rmfamily\fontsize{10.000000}{12.000000}\selectfont \(\displaystyle 40\)}%
\end{pgfscope}%
\begin{pgfscope}%
\pgfsetbuttcap%
\pgfsetroundjoin%
\definecolor{currentfill}{rgb}{0.000000,0.000000,0.000000}%
\pgfsetfillcolor{currentfill}%
\pgfsetlinewidth{0.803000pt}%
\definecolor{currentstroke}{rgb}{0.000000,0.000000,0.000000}%
\pgfsetstrokecolor{currentstroke}%
\pgfsetdash{}{0pt}%
\pgfsys@defobject{currentmarker}{\pgfqpoint{-0.048611in}{0.000000in}}{\pgfqpoint{0.000000in}{0.000000in}}{%
\pgfpathmoveto{\pgfqpoint{0.000000in}{0.000000in}}%
\pgfpathlineto{\pgfqpoint{-0.048611in}{0.000000in}}%
\pgfusepath{stroke,fill}%
}%
\begin{pgfscope}%
\pgfsys@transformshift{0.970666in}{5.182464in}%
\pgfsys@useobject{currentmarker}{}%
\end{pgfscope}%
\end{pgfscope}%
\begin{pgfscope}%
\definecolor{textcolor}{rgb}{0.000000,0.000000,0.000000}%
\pgfsetstrokecolor{textcolor}%
\pgfsetfillcolor{textcolor}%
\pgftext[x=0.734555in, y=5.129703in, left, base]{\color{textcolor}\rmfamily\fontsize{10.000000}{12.000000}\selectfont \(\displaystyle 60\)}%
\end{pgfscope}%
\begin{pgfscope}%
\pgfsetbuttcap%
\pgfsetroundjoin%
\definecolor{currentfill}{rgb}{0.000000,0.000000,0.000000}%
\pgfsetfillcolor{currentfill}%
\pgfsetlinewidth{0.803000pt}%
\definecolor{currentstroke}{rgb}{0.000000,0.000000,0.000000}%
\pgfsetstrokecolor{currentstroke}%
\pgfsetdash{}{0pt}%
\pgfsys@defobject{currentmarker}{\pgfqpoint{-0.048611in}{0.000000in}}{\pgfqpoint{0.000000in}{0.000000in}}{%
\pgfpathmoveto{\pgfqpoint{0.000000in}{0.000000in}}%
\pgfpathlineto{\pgfqpoint{-0.048611in}{0.000000in}}%
\pgfusepath{stroke,fill}%
}%
\begin{pgfscope}%
\pgfsys@transformshift{0.970666in}{5.495452in}%
\pgfsys@useobject{currentmarker}{}%
\end{pgfscope}%
\end{pgfscope}%
\begin{pgfscope}%
\definecolor{textcolor}{rgb}{0.000000,0.000000,0.000000}%
\pgfsetstrokecolor{textcolor}%
\pgfsetfillcolor{textcolor}%
\pgftext[x=0.734555in, y=5.442690in, left, base]{\color{textcolor}\rmfamily\fontsize{10.000000}{12.000000}\selectfont \(\displaystyle 80\)}%
\end{pgfscope}%
\begin{pgfscope}%
\pgfsetbuttcap%
\pgfsetroundjoin%
\definecolor{currentfill}{rgb}{0.000000,0.000000,0.000000}%
\pgfsetfillcolor{currentfill}%
\pgfsetlinewidth{0.803000pt}%
\definecolor{currentstroke}{rgb}{0.000000,0.000000,0.000000}%
\pgfsetstrokecolor{currentstroke}%
\pgfsetdash{}{0pt}%
\pgfsys@defobject{currentmarker}{\pgfqpoint{-0.048611in}{0.000000in}}{\pgfqpoint{0.000000in}{0.000000in}}{%
\pgfpathmoveto{\pgfqpoint{0.000000in}{0.000000in}}%
\pgfpathlineto{\pgfqpoint{-0.048611in}{0.000000in}}%
\pgfusepath{stroke,fill}%
}%
\begin{pgfscope}%
\pgfsys@transformshift{0.970666in}{5.808439in}%
\pgfsys@useobject{currentmarker}{}%
\end{pgfscope}%
\end{pgfscope}%
\begin{pgfscope}%
\definecolor{textcolor}{rgb}{0.000000,0.000000,0.000000}%
\pgfsetstrokecolor{textcolor}%
\pgfsetfillcolor{textcolor}%
\pgftext[x=0.665110in, y=5.755678in, left, base]{\color{textcolor}\rmfamily\fontsize{10.000000}{12.000000}\selectfont \(\displaystyle 100\)}%
\end{pgfscope}%
\begin{pgfscope}%
\pgfsetbuttcap%
\pgfsetroundjoin%
\definecolor{currentfill}{rgb}{0.000000,0.000000,0.000000}%
\pgfsetfillcolor{currentfill}%
\pgfsetlinewidth{0.803000pt}%
\definecolor{currentstroke}{rgb}{0.000000,0.000000,0.000000}%
\pgfsetstrokecolor{currentstroke}%
\pgfsetdash{}{0pt}%
\pgfsys@defobject{currentmarker}{\pgfqpoint{-0.048611in}{0.000000in}}{\pgfqpoint{0.000000in}{0.000000in}}{%
\pgfpathmoveto{\pgfqpoint{0.000000in}{0.000000in}}%
\pgfpathlineto{\pgfqpoint{-0.048611in}{0.000000in}}%
\pgfusepath{stroke,fill}%
}%
\begin{pgfscope}%
\pgfsys@transformshift{0.970666in}{6.121427in}%
\pgfsys@useobject{currentmarker}{}%
\end{pgfscope}%
\end{pgfscope}%
\begin{pgfscope}%
\definecolor{textcolor}{rgb}{0.000000,0.000000,0.000000}%
\pgfsetstrokecolor{textcolor}%
\pgfsetfillcolor{textcolor}%
\pgftext[x=0.665110in, y=6.068665in, left, base]{\color{textcolor}\rmfamily\fontsize{10.000000}{12.000000}\selectfont \(\displaystyle 120\)}%
\end{pgfscope}%
\begin{pgfscope}%
\pgfsetbuttcap%
\pgfsetroundjoin%
\definecolor{currentfill}{rgb}{0.000000,0.000000,0.000000}%
\pgfsetfillcolor{currentfill}%
\pgfsetlinewidth{0.803000pt}%
\definecolor{currentstroke}{rgb}{0.000000,0.000000,0.000000}%
\pgfsetstrokecolor{currentstroke}%
\pgfsetdash{}{0pt}%
\pgfsys@defobject{currentmarker}{\pgfqpoint{-0.048611in}{0.000000in}}{\pgfqpoint{0.000000in}{0.000000in}}{%
\pgfpathmoveto{\pgfqpoint{0.000000in}{0.000000in}}%
\pgfpathlineto{\pgfqpoint{-0.048611in}{0.000000in}}%
\pgfusepath{stroke,fill}%
}%
\begin{pgfscope}%
\pgfsys@transformshift{0.970666in}{6.434414in}%
\pgfsys@useobject{currentmarker}{}%
\end{pgfscope}%
\end{pgfscope}%
\begin{pgfscope}%
\definecolor{textcolor}{rgb}{0.000000,0.000000,0.000000}%
\pgfsetstrokecolor{textcolor}%
\pgfsetfillcolor{textcolor}%
\pgftext[x=0.665110in, y=6.381653in, left, base]{\color{textcolor}\rmfamily\fontsize{10.000000}{12.000000}\selectfont \(\displaystyle 140\)}%
\end{pgfscope}%
\begin{pgfscope}%
\pgfsetbuttcap%
\pgfsetroundjoin%
\definecolor{currentfill}{rgb}{0.000000,0.000000,0.000000}%
\pgfsetfillcolor{currentfill}%
\pgfsetlinewidth{0.803000pt}%
\definecolor{currentstroke}{rgb}{0.000000,0.000000,0.000000}%
\pgfsetstrokecolor{currentstroke}%
\pgfsetdash{}{0pt}%
\pgfsys@defobject{currentmarker}{\pgfqpoint{-0.048611in}{0.000000in}}{\pgfqpoint{0.000000in}{0.000000in}}{%
\pgfpathmoveto{\pgfqpoint{0.000000in}{0.000000in}}%
\pgfpathlineto{\pgfqpoint{-0.048611in}{0.000000in}}%
\pgfusepath{stroke,fill}%
}%
\begin{pgfscope}%
\pgfsys@transformshift{0.970666in}{6.747402in}%
\pgfsys@useobject{currentmarker}{}%
\end{pgfscope}%
\end{pgfscope}%
\begin{pgfscope}%
\definecolor{textcolor}{rgb}{0.000000,0.000000,0.000000}%
\pgfsetstrokecolor{textcolor}%
\pgfsetfillcolor{textcolor}%
\pgftext[x=0.665110in, y=6.694640in, left, base]{\color{textcolor}\rmfamily\fontsize{10.000000}{12.000000}\selectfont \(\displaystyle 160\)}%
\end{pgfscope}%
\begin{pgfscope}%
\pgfsetrectcap%
\pgfsetmiterjoin%
\pgfsetlinewidth{0.803000pt}%
\definecolor{currentstroke}{rgb}{0.000000,0.000000,0.000000}%
\pgfsetstrokecolor{currentstroke}%
\pgfsetdash{}{0pt}%
\pgfpathmoveto{\pgfqpoint{0.970666in}{4.121437in}}%
\pgfpathlineto{\pgfqpoint{0.970666in}{6.806869in}}%
\pgfusepath{stroke}%
\end{pgfscope}%
\begin{pgfscope}%
\pgfsetrectcap%
\pgfsetmiterjoin%
\pgfsetlinewidth{0.803000pt}%
\definecolor{currentstroke}{rgb}{0.000000,0.000000,0.000000}%
\pgfsetstrokecolor{currentstroke}%
\pgfsetdash{}{0pt}%
\pgfpathmoveto{\pgfqpoint{6.669922in}{4.121437in}}%
\pgfpathlineto{\pgfqpoint{6.669922in}{6.806869in}}%
\pgfusepath{stroke}%
\end{pgfscope}%
\begin{pgfscope}%
\pgfsetrectcap%
\pgfsetmiterjoin%
\pgfsetlinewidth{0.803000pt}%
\definecolor{currentstroke}{rgb}{0.000000,0.000000,0.000000}%
\pgfsetstrokecolor{currentstroke}%
\pgfsetdash{}{0pt}%
\pgfpathmoveto{\pgfqpoint{0.970666in}{4.121437in}}%
\pgfpathlineto{\pgfqpoint{6.669922in}{4.121437in}}%
\pgfusepath{stroke}%
\end{pgfscope}%
\begin{pgfscope}%
\pgfsetrectcap%
\pgfsetmiterjoin%
\pgfsetlinewidth{0.803000pt}%
\definecolor{currentstroke}{rgb}{0.000000,0.000000,0.000000}%
\pgfsetstrokecolor{currentstroke}%
\pgfsetdash{}{0pt}%
\pgfpathmoveto{\pgfqpoint{0.970666in}{6.806869in}}%
\pgfpathlineto{\pgfqpoint{6.669922in}{6.806869in}}%
\pgfusepath{stroke}%
\end{pgfscope}%
\begin{pgfscope}%
\definecolor{textcolor}{rgb}{0.000000,0.000000,0.000000}%
\pgfsetstrokecolor{textcolor}%
\pgfsetfillcolor{textcolor}%
\pgftext[x=3.820294in,y=6.890203in,,base]{\color{textcolor}\rmfamily\fontsize{12.000000}{14.400000}\selectfont swap 5329.8}%
\end{pgfscope}%
\begin{pgfscope}%
\pgfsetbuttcap%
\pgfsetmiterjoin%
\definecolor{currentfill}{rgb}{1.000000,1.000000,1.000000}%
\pgfsetfillcolor{currentfill}%
\pgfsetlinewidth{0.000000pt}%
\definecolor{currentstroke}{rgb}{0.000000,0.000000,0.000000}%
\pgfsetstrokecolor{currentstroke}%
\pgfsetstrokeopacity{0.000000}%
\pgfsetdash{}{0pt}%
\pgfpathmoveto{\pgfqpoint{7.640588in}{4.121437in}}%
\pgfpathlineto{\pgfqpoint{13.339844in}{4.121437in}}%
\pgfpathlineto{\pgfqpoint{13.339844in}{6.806869in}}%
\pgfpathlineto{\pgfqpoint{7.640588in}{6.806869in}}%
\pgfpathclose%
\pgfusepath{fill}%
\end{pgfscope}%
\begin{pgfscope}%
\pgfpathrectangle{\pgfqpoint{7.640588in}{4.121437in}}{\pgfqpoint{5.699255in}{2.685432in}}%
\pgfusepath{clip}%
\pgfsetrectcap%
\pgfsetroundjoin%
\pgfsetlinewidth{1.505625pt}%
\definecolor{currentstroke}{rgb}{0.000000,0.000000,0.000000}%
\pgfsetstrokecolor{currentstroke}%
\pgfsetdash{}{0pt}%
\pgfpathmoveto{\pgfqpoint{8.288231in}{5.949284in}}%
\pgfpathlineto{\pgfqpoint{8.773963in}{6.528310in}}%
\pgfpathlineto{\pgfqpoint{8.773963in}{6.669155in}}%
\pgfpathlineto{\pgfqpoint{9.162549in}{6.653505in}}%
\pgfpathlineto{\pgfqpoint{8.871109in}{6.543960in}}%
\pgfpathlineto{\pgfqpoint{8.871109in}{6.497012in}}%
\pgfpathlineto{\pgfqpoint{8.935874in}{6.184024in}}%
\pgfpathlineto{\pgfqpoint{9.356841in}{6.043180in}}%
\pgfpathlineto{\pgfqpoint{9.615898in}{6.152725in}}%
\pgfpathlineto{\pgfqpoint{9.842573in}{6.277920in}}%
\pgfpathlineto{\pgfqpoint{10.198777in}{6.324869in}}%
\pgfpathlineto{\pgfqpoint{10.814037in}{6.418765in}}%
\pgfpathlineto{\pgfqpoint{11.008330in}{6.450064in}}%
\pgfpathlineto{\pgfqpoint{11.332152in}{6.418765in}}%
\pgfpathlineto{\pgfqpoint{11.494062in}{6.387466in}}%
\pgfpathlineto{\pgfqpoint{11.558826in}{6.434414in}}%
\pgfpathlineto{\pgfqpoint{11.785501in}{6.309219in}}%
\pgfpathlineto{\pgfqpoint{12.206469in}{6.637856in}}%
\pgfpathlineto{\pgfqpoint{12.433144in}{6.622207in}}%
\pgfpathlineto{\pgfqpoint{12.465526in}{6.653505in}}%
\pgfpathlineto{\pgfqpoint{12.789347in}{6.684804in}}%
\pgfpathlineto{\pgfqpoint{12.821730in}{6.434414in}}%
\pgfpathlineto{\pgfqpoint{13.080787in}{6.340518in}}%
\pgfpathlineto{\pgfqpoint{12.141705in}{6.309219in}}%
\pgfpathlineto{\pgfqpoint{12.044558in}{6.043180in}}%
\pgfpathlineto{\pgfqpoint{12.271233in}{5.917985in}}%
\pgfpathlineto{\pgfqpoint{12.595055in}{5.964933in}}%
\pgfpathlineto{\pgfqpoint{12.400762in}{5.777141in}}%
\pgfpathlineto{\pgfqpoint{12.692201in}{5.448504in}}%
\pgfpathlineto{\pgfqpoint{12.465526in}{5.010321in}}%
\pgfpathlineto{\pgfqpoint{12.854112in}{4.838178in}}%
\pgfpathlineto{\pgfqpoint{12.141705in}{4.712983in}}%
\pgfpathlineto{\pgfqpoint{11.947412in}{4.744282in}}%
\pgfpathlineto{\pgfqpoint{11.979794in}{4.243502in}}%
\pgfpathlineto{\pgfqpoint{11.461680in}{4.243502in}}%
\pgfpathlineto{\pgfqpoint{10.619745in}{4.353048in}}%
\pgfpathlineto{\pgfqpoint{10.652127in}{4.384347in}}%
\pgfpathlineto{\pgfqpoint{11.235005in}{4.446944in}}%
\pgfpathlineto{\pgfqpoint{11.429298in}{4.775581in}}%
\pgfpathlineto{\pgfqpoint{11.558826in}{4.712983in}}%
\pgfpathlineto{\pgfqpoint{11.947412in}{4.822529in}}%
\pgfpathlineto{\pgfqpoint{12.044558in}{4.885126in}}%
\pgfpathlineto{\pgfqpoint{12.012176in}{5.072919in}}%
\pgfpathlineto{\pgfqpoint{11.817884in}{5.417205in}}%
\pgfpathlineto{\pgfqpoint{11.429298in}{5.385906in}}%
\pgfpathlineto{\pgfqpoint{11.170241in}{5.730192in}}%
\pgfpathlineto{\pgfqpoint{11.105477in}{5.745842in}}%
\pgfpathlineto{\pgfqpoint{10.814037in}{6.011881in}}%
\pgfpathlineto{\pgfqpoint{10.295923in}{5.964933in}}%
\pgfpathlineto{\pgfqpoint{10.004484in}{5.855387in}}%
\pgfpathlineto{\pgfqpoint{10.101630in}{5.651946in}}%
\pgfpathlineto{\pgfqpoint{10.101630in}{5.589348in}}%
\pgfpathlineto{\pgfqpoint{10.134013in}{5.464153in}}%
\pgfpathlineto{\pgfqpoint{10.198777in}{5.292010in}}%
\pgfpathlineto{\pgfqpoint{10.166395in}{5.198114in}}%
\pgfpathlineto{\pgfqpoint{10.878802in}{5.354608in}}%
\pgfpathlineto{\pgfqpoint{10.716891in}{4.916425in}}%
\pgfpathlineto{\pgfqpoint{10.263541in}{4.775581in}}%
\pgfpathlineto{\pgfqpoint{9.907338in}{4.728633in}}%
\pgfpathlineto{\pgfqpoint{9.810191in}{4.509542in}}%
\pgfpathlineto{\pgfqpoint{9.615898in}{4.306100in}}%
\pgfpathlineto{\pgfqpoint{9.356841in}{4.431295in}}%
\pgfpathlineto{\pgfqpoint{9.194931in}{4.525191in}}%
\pgfpathlineto{\pgfqpoint{9.033020in}{4.572139in}}%
\pgfpathlineto{\pgfqpoint{8.903492in}{4.509542in}}%
\pgfpathlineto{\pgfqpoint{8.676817in}{4.556490in}}%
\pgfpathlineto{\pgfqpoint{8.806345in}{4.681685in}}%
\pgfpathlineto{\pgfqpoint{8.547288in}{4.650386in}}%
\pgfpathlineto{\pgfqpoint{8.417760in}{4.619087in}}%
\pgfpathlineto{\pgfqpoint{8.320613in}{4.493892in}}%
\pgfpathlineto{\pgfqpoint{8.126320in}{4.337398in}}%
\pgfpathlineto{\pgfqpoint{7.899645in}{4.634736in}}%
\pgfpathlineto{\pgfqpoint{8.158702in}{4.712983in}}%
\pgfpathlineto{\pgfqpoint{7.932028in}{5.151166in}}%
\pgfpathlineto{\pgfqpoint{9.324459in}{5.338958in}}%
\pgfpathlineto{\pgfqpoint{9.810191in}{5.432854in}}%
\pgfpathlineto{\pgfqpoint{9.615898in}{5.432854in}}%
\pgfpathlineto{\pgfqpoint{9.227313in}{5.479803in}}%
\pgfpathlineto{\pgfqpoint{8.871109in}{5.495452in}}%
\pgfpathlineto{\pgfqpoint{8.871109in}{5.651946in}}%
\pgfpathlineto{\pgfqpoint{9.033020in}{5.714543in}}%
\pgfpathlineto{\pgfqpoint{8.126320in}{5.917985in}}%
\pgfpathlineto{\pgfqpoint{8.288231in}{5.949284in}}%
\pgfusepath{stroke}%
\end{pgfscope}%
\begin{pgfscope}%
\pgfpathrectangle{\pgfqpoint{7.640588in}{4.121437in}}{\pgfqpoint{5.699255in}{2.685432in}}%
\pgfusepath{clip}%
\pgfsetbuttcap%
\pgfsetroundjoin%
\definecolor{currentfill}{rgb}{1.000000,0.000000,0.000000}%
\pgfsetfillcolor{currentfill}%
\pgfsetlinewidth{1.003750pt}%
\definecolor{currentstroke}{rgb}{1.000000,0.000000,0.000000}%
\pgfsetstrokecolor{currentstroke}%
\pgfsetdash{}{0pt}%
\pgfpathmoveto{\pgfqpoint{8.288231in}{5.907617in}}%
\pgfpathcurveto{\pgfqpoint{8.299281in}{5.907617in}}{\pgfqpoint{8.309880in}{5.912007in}}{\pgfqpoint{8.317694in}{5.919821in}}%
\pgfpathcurveto{\pgfqpoint{8.325507in}{5.927635in}}{\pgfqpoint{8.329898in}{5.938234in}}{\pgfqpoint{8.329898in}{5.949284in}}%
\pgfpathcurveto{\pgfqpoint{8.329898in}{5.960334in}}{\pgfqpoint{8.325507in}{5.970933in}}{\pgfqpoint{8.317694in}{5.978746in}}%
\pgfpathcurveto{\pgfqpoint{8.309880in}{5.986560in}}{\pgfqpoint{8.299281in}{5.990950in}}{\pgfqpoint{8.288231in}{5.990950in}}%
\pgfpathcurveto{\pgfqpoint{8.277181in}{5.990950in}}{\pgfqpoint{8.266582in}{5.986560in}}{\pgfqpoint{8.258768in}{5.978746in}}%
\pgfpathcurveto{\pgfqpoint{8.250955in}{5.970933in}}{\pgfqpoint{8.246564in}{5.960334in}}{\pgfqpoint{8.246564in}{5.949284in}}%
\pgfpathcurveto{\pgfqpoint{8.246564in}{5.938234in}}{\pgfqpoint{8.250955in}{5.927635in}}{\pgfqpoint{8.258768in}{5.919821in}}%
\pgfpathcurveto{\pgfqpoint{8.266582in}{5.912007in}}{\pgfqpoint{8.277181in}{5.907617in}}{\pgfqpoint{8.288231in}{5.907617in}}%
\pgfpathclose%
\pgfusepath{stroke,fill}%
\end{pgfscope}%
\begin{pgfscope}%
\pgfpathrectangle{\pgfqpoint{7.640588in}{4.121437in}}{\pgfqpoint{5.699255in}{2.685432in}}%
\pgfusepath{clip}%
\pgfsetbuttcap%
\pgfsetroundjoin%
\definecolor{currentfill}{rgb}{0.750000,0.750000,0.000000}%
\pgfsetfillcolor{currentfill}%
\pgfsetlinewidth{1.003750pt}%
\definecolor{currentstroke}{rgb}{0.750000,0.750000,0.000000}%
\pgfsetstrokecolor{currentstroke}%
\pgfsetdash{}{0pt}%
\pgfpathmoveto{\pgfqpoint{8.773963in}{6.486644in}}%
\pgfpathcurveto{\pgfqpoint{8.785013in}{6.486644in}}{\pgfqpoint{8.795612in}{6.491034in}}{\pgfqpoint{8.803426in}{6.498848in}}%
\pgfpathcurveto{\pgfqpoint{8.811239in}{6.506661in}}{\pgfqpoint{8.815630in}{6.517260in}}{\pgfqpoint{8.815630in}{6.528310in}}%
\pgfpathcurveto{\pgfqpoint{8.815630in}{6.539361in}}{\pgfqpoint{8.811239in}{6.549960in}}{\pgfqpoint{8.803426in}{6.557773in}}%
\pgfpathcurveto{\pgfqpoint{8.795612in}{6.565587in}}{\pgfqpoint{8.785013in}{6.569977in}}{\pgfqpoint{8.773963in}{6.569977in}}%
\pgfpathcurveto{\pgfqpoint{8.762913in}{6.569977in}}{\pgfqpoint{8.752314in}{6.565587in}}{\pgfqpoint{8.744500in}{6.557773in}}%
\pgfpathcurveto{\pgfqpoint{8.736687in}{6.549960in}}{\pgfqpoint{8.732296in}{6.539361in}}{\pgfqpoint{8.732296in}{6.528310in}}%
\pgfpathcurveto{\pgfqpoint{8.732296in}{6.517260in}}{\pgfqpoint{8.736687in}{6.506661in}}{\pgfqpoint{8.744500in}{6.498848in}}%
\pgfpathcurveto{\pgfqpoint{8.752314in}{6.491034in}}{\pgfqpoint{8.762913in}{6.486644in}}{\pgfqpoint{8.773963in}{6.486644in}}%
\pgfpathclose%
\pgfusepath{stroke,fill}%
\end{pgfscope}%
\begin{pgfscope}%
\pgfpathrectangle{\pgfqpoint{7.640588in}{4.121437in}}{\pgfqpoint{5.699255in}{2.685432in}}%
\pgfusepath{clip}%
\pgfsetbuttcap%
\pgfsetroundjoin%
\definecolor{currentfill}{rgb}{0.000000,0.000000,0.000000}%
\pgfsetfillcolor{currentfill}%
\pgfsetlinewidth{1.003750pt}%
\definecolor{currentstroke}{rgb}{0.000000,0.000000,0.000000}%
\pgfsetstrokecolor{currentstroke}%
\pgfsetdash{}{0pt}%
\pgfpathmoveto{\pgfqpoint{8.773963in}{6.627488in}}%
\pgfpathcurveto{\pgfqpoint{8.785013in}{6.627488in}}{\pgfqpoint{8.795612in}{6.631878in}}{\pgfqpoint{8.803426in}{6.639692in}}%
\pgfpathcurveto{\pgfqpoint{8.811239in}{6.647506in}}{\pgfqpoint{8.815630in}{6.658105in}}{\pgfqpoint{8.815630in}{6.669155in}}%
\pgfpathcurveto{\pgfqpoint{8.815630in}{6.680205in}}{\pgfqpoint{8.811239in}{6.690804in}}{\pgfqpoint{8.803426in}{6.698618in}}%
\pgfpathcurveto{\pgfqpoint{8.795612in}{6.706431in}}{\pgfqpoint{8.785013in}{6.710821in}}{\pgfqpoint{8.773963in}{6.710821in}}%
\pgfpathcurveto{\pgfqpoint{8.762913in}{6.710821in}}{\pgfqpoint{8.752314in}{6.706431in}}{\pgfqpoint{8.744500in}{6.698618in}}%
\pgfpathcurveto{\pgfqpoint{8.736687in}{6.690804in}}{\pgfqpoint{8.732296in}{6.680205in}}{\pgfqpoint{8.732296in}{6.669155in}}%
\pgfpathcurveto{\pgfqpoint{8.732296in}{6.658105in}}{\pgfqpoint{8.736687in}{6.647506in}}{\pgfqpoint{8.744500in}{6.639692in}}%
\pgfpathcurveto{\pgfqpoint{8.752314in}{6.631878in}}{\pgfqpoint{8.762913in}{6.627488in}}{\pgfqpoint{8.773963in}{6.627488in}}%
\pgfpathclose%
\pgfusepath{stroke,fill}%
\end{pgfscope}%
\begin{pgfscope}%
\pgfpathrectangle{\pgfqpoint{7.640588in}{4.121437in}}{\pgfqpoint{5.699255in}{2.685432in}}%
\pgfusepath{clip}%
\pgfsetbuttcap%
\pgfsetroundjoin%
\definecolor{currentfill}{rgb}{0.000000,0.000000,0.000000}%
\pgfsetfillcolor{currentfill}%
\pgfsetlinewidth{1.003750pt}%
\definecolor{currentstroke}{rgb}{0.000000,0.000000,0.000000}%
\pgfsetstrokecolor{currentstroke}%
\pgfsetdash{}{0pt}%
\pgfpathmoveto{\pgfqpoint{9.162549in}{6.611839in}}%
\pgfpathcurveto{\pgfqpoint{9.173599in}{6.611839in}}{\pgfqpoint{9.184198in}{6.616229in}}{\pgfqpoint{9.192011in}{6.624043in}}%
\pgfpathcurveto{\pgfqpoint{9.199825in}{6.631856in}}{\pgfqpoint{9.204215in}{6.642455in}}{\pgfqpoint{9.204215in}{6.653505in}}%
\pgfpathcurveto{\pgfqpoint{9.204215in}{6.664556in}}{\pgfqpoint{9.199825in}{6.675155in}}{\pgfqpoint{9.192011in}{6.682968in}}%
\pgfpathcurveto{\pgfqpoint{9.184198in}{6.690782in}}{\pgfqpoint{9.173599in}{6.695172in}}{\pgfqpoint{9.162549in}{6.695172in}}%
\pgfpathcurveto{\pgfqpoint{9.151498in}{6.695172in}}{\pgfqpoint{9.140899in}{6.690782in}}{\pgfqpoint{9.133086in}{6.682968in}}%
\pgfpathcurveto{\pgfqpoint{9.125272in}{6.675155in}}{\pgfqpoint{9.120882in}{6.664556in}}{\pgfqpoint{9.120882in}{6.653505in}}%
\pgfpathcurveto{\pgfqpoint{9.120882in}{6.642455in}}{\pgfqpoint{9.125272in}{6.631856in}}{\pgfqpoint{9.133086in}{6.624043in}}%
\pgfpathcurveto{\pgfqpoint{9.140899in}{6.616229in}}{\pgfqpoint{9.151498in}{6.611839in}}{\pgfqpoint{9.162549in}{6.611839in}}%
\pgfpathclose%
\pgfusepath{stroke,fill}%
\end{pgfscope}%
\begin{pgfscope}%
\pgfpathrectangle{\pgfqpoint{7.640588in}{4.121437in}}{\pgfqpoint{5.699255in}{2.685432in}}%
\pgfusepath{clip}%
\pgfsetbuttcap%
\pgfsetroundjoin%
\definecolor{currentfill}{rgb}{0.000000,0.000000,0.000000}%
\pgfsetfillcolor{currentfill}%
\pgfsetlinewidth{1.003750pt}%
\definecolor{currentstroke}{rgb}{0.000000,0.000000,0.000000}%
\pgfsetstrokecolor{currentstroke}%
\pgfsetdash{}{0pt}%
\pgfpathmoveto{\pgfqpoint{8.871109in}{6.502293in}}%
\pgfpathcurveto{\pgfqpoint{8.882160in}{6.502293in}}{\pgfqpoint{8.892759in}{6.506683in}}{\pgfqpoint{8.900572in}{6.514497in}}%
\pgfpathcurveto{\pgfqpoint{8.908386in}{6.522311in}}{\pgfqpoint{8.912776in}{6.532910in}}{\pgfqpoint{8.912776in}{6.543960in}}%
\pgfpathcurveto{\pgfqpoint{8.912776in}{6.555010in}}{\pgfqpoint{8.908386in}{6.565609in}}{\pgfqpoint{8.900572in}{6.573423in}}%
\pgfpathcurveto{\pgfqpoint{8.892759in}{6.581236in}}{\pgfqpoint{8.882160in}{6.585626in}}{\pgfqpoint{8.871109in}{6.585626in}}%
\pgfpathcurveto{\pgfqpoint{8.860059in}{6.585626in}}{\pgfqpoint{8.849460in}{6.581236in}}{\pgfqpoint{8.841647in}{6.573423in}}%
\pgfpathcurveto{\pgfqpoint{8.833833in}{6.565609in}}{\pgfqpoint{8.829443in}{6.555010in}}{\pgfqpoint{8.829443in}{6.543960in}}%
\pgfpathcurveto{\pgfqpoint{8.829443in}{6.532910in}}{\pgfqpoint{8.833833in}{6.522311in}}{\pgfqpoint{8.841647in}{6.514497in}}%
\pgfpathcurveto{\pgfqpoint{8.849460in}{6.506683in}}{\pgfqpoint{8.860059in}{6.502293in}}{\pgfqpoint{8.871109in}{6.502293in}}%
\pgfpathclose%
\pgfusepath{stroke,fill}%
\end{pgfscope}%
\begin{pgfscope}%
\pgfpathrectangle{\pgfqpoint{7.640588in}{4.121437in}}{\pgfqpoint{5.699255in}{2.685432in}}%
\pgfusepath{clip}%
\pgfsetbuttcap%
\pgfsetroundjoin%
\definecolor{currentfill}{rgb}{0.000000,0.000000,0.000000}%
\pgfsetfillcolor{currentfill}%
\pgfsetlinewidth{1.003750pt}%
\definecolor{currentstroke}{rgb}{0.000000,0.000000,0.000000}%
\pgfsetstrokecolor{currentstroke}%
\pgfsetdash{}{0pt}%
\pgfpathmoveto{\pgfqpoint{8.871109in}{6.455345in}}%
\pgfpathcurveto{\pgfqpoint{8.882160in}{6.455345in}}{\pgfqpoint{8.892759in}{6.459735in}}{\pgfqpoint{8.900572in}{6.467549in}}%
\pgfpathcurveto{\pgfqpoint{8.908386in}{6.475363in}}{\pgfqpoint{8.912776in}{6.485962in}}{\pgfqpoint{8.912776in}{6.497012in}}%
\pgfpathcurveto{\pgfqpoint{8.912776in}{6.508062in}}{\pgfqpoint{8.908386in}{6.518661in}}{\pgfqpoint{8.900572in}{6.526474in}}%
\pgfpathcurveto{\pgfqpoint{8.892759in}{6.534288in}}{\pgfqpoint{8.882160in}{6.538678in}}{\pgfqpoint{8.871109in}{6.538678in}}%
\pgfpathcurveto{\pgfqpoint{8.860059in}{6.538678in}}{\pgfqpoint{8.849460in}{6.534288in}}{\pgfqpoint{8.841647in}{6.526474in}}%
\pgfpathcurveto{\pgfqpoint{8.833833in}{6.518661in}}{\pgfqpoint{8.829443in}{6.508062in}}{\pgfqpoint{8.829443in}{6.497012in}}%
\pgfpathcurveto{\pgfqpoint{8.829443in}{6.485962in}}{\pgfqpoint{8.833833in}{6.475363in}}{\pgfqpoint{8.841647in}{6.467549in}}%
\pgfpathcurveto{\pgfqpoint{8.849460in}{6.459735in}}{\pgfqpoint{8.860059in}{6.455345in}}{\pgfqpoint{8.871109in}{6.455345in}}%
\pgfpathclose%
\pgfusepath{stroke,fill}%
\end{pgfscope}%
\begin{pgfscope}%
\pgfpathrectangle{\pgfqpoint{7.640588in}{4.121437in}}{\pgfqpoint{5.699255in}{2.685432in}}%
\pgfusepath{clip}%
\pgfsetbuttcap%
\pgfsetroundjoin%
\definecolor{currentfill}{rgb}{0.000000,0.000000,0.000000}%
\pgfsetfillcolor{currentfill}%
\pgfsetlinewidth{1.003750pt}%
\definecolor{currentstroke}{rgb}{0.000000,0.000000,0.000000}%
\pgfsetstrokecolor{currentstroke}%
\pgfsetdash{}{0pt}%
\pgfpathmoveto{\pgfqpoint{8.935874in}{6.142358in}}%
\pgfpathcurveto{\pgfqpoint{8.946924in}{6.142358in}}{\pgfqpoint{8.957523in}{6.146748in}}{\pgfqpoint{8.965336in}{6.154561in}}%
\pgfpathcurveto{\pgfqpoint{8.973150in}{6.162375in}}{\pgfqpoint{8.977540in}{6.172974in}}{\pgfqpoint{8.977540in}{6.184024in}}%
\pgfpathcurveto{\pgfqpoint{8.977540in}{6.195074in}}{\pgfqpoint{8.973150in}{6.205673in}}{\pgfqpoint{8.965336in}{6.213487in}}%
\pgfpathcurveto{\pgfqpoint{8.957523in}{6.221301in}}{\pgfqpoint{8.946924in}{6.225691in}}{\pgfqpoint{8.935874in}{6.225691in}}%
\pgfpathcurveto{\pgfqpoint{8.924824in}{6.225691in}}{\pgfqpoint{8.914225in}{6.221301in}}{\pgfqpoint{8.906411in}{6.213487in}}%
\pgfpathcurveto{\pgfqpoint{8.898597in}{6.205673in}}{\pgfqpoint{8.894207in}{6.195074in}}{\pgfqpoint{8.894207in}{6.184024in}}%
\pgfpathcurveto{\pgfqpoint{8.894207in}{6.172974in}}{\pgfqpoint{8.898597in}{6.162375in}}{\pgfqpoint{8.906411in}{6.154561in}}%
\pgfpathcurveto{\pgfqpoint{8.914225in}{6.146748in}}{\pgfqpoint{8.924824in}{6.142358in}}{\pgfqpoint{8.935874in}{6.142358in}}%
\pgfpathclose%
\pgfusepath{stroke,fill}%
\end{pgfscope}%
\begin{pgfscope}%
\pgfpathrectangle{\pgfqpoint{7.640588in}{4.121437in}}{\pgfqpoint{5.699255in}{2.685432in}}%
\pgfusepath{clip}%
\pgfsetbuttcap%
\pgfsetroundjoin%
\definecolor{currentfill}{rgb}{0.000000,0.000000,0.000000}%
\pgfsetfillcolor{currentfill}%
\pgfsetlinewidth{1.003750pt}%
\definecolor{currentstroke}{rgb}{0.000000,0.000000,0.000000}%
\pgfsetstrokecolor{currentstroke}%
\pgfsetdash{}{0pt}%
\pgfpathmoveto{\pgfqpoint{9.356841in}{6.001513in}}%
\pgfpathcurveto{\pgfqpoint{9.367892in}{6.001513in}}{\pgfqpoint{9.378491in}{6.005903in}}{\pgfqpoint{9.386304in}{6.013717in}}%
\pgfpathcurveto{\pgfqpoint{9.394118in}{6.021531in}}{\pgfqpoint{9.398508in}{6.032130in}}{\pgfqpoint{9.398508in}{6.043180in}}%
\pgfpathcurveto{\pgfqpoint{9.398508in}{6.054230in}}{\pgfqpoint{9.394118in}{6.064829in}}{\pgfqpoint{9.386304in}{6.072643in}}%
\pgfpathcurveto{\pgfqpoint{9.378491in}{6.080456in}}{\pgfqpoint{9.367892in}{6.084847in}}{\pgfqpoint{9.356841in}{6.084847in}}%
\pgfpathcurveto{\pgfqpoint{9.345791in}{6.084847in}}{\pgfqpoint{9.335192in}{6.080456in}}{\pgfqpoint{9.327379in}{6.072643in}}%
\pgfpathcurveto{\pgfqpoint{9.319565in}{6.064829in}}{\pgfqpoint{9.315175in}{6.054230in}}{\pgfqpoint{9.315175in}{6.043180in}}%
\pgfpathcurveto{\pgfqpoint{9.315175in}{6.032130in}}{\pgfqpoint{9.319565in}{6.021531in}}{\pgfqpoint{9.327379in}{6.013717in}}%
\pgfpathcurveto{\pgfqpoint{9.335192in}{6.005903in}}{\pgfqpoint{9.345791in}{6.001513in}}{\pgfqpoint{9.356841in}{6.001513in}}%
\pgfpathclose%
\pgfusepath{stroke,fill}%
\end{pgfscope}%
\begin{pgfscope}%
\pgfpathrectangle{\pgfqpoint{7.640588in}{4.121437in}}{\pgfqpoint{5.699255in}{2.685432in}}%
\pgfusepath{clip}%
\pgfsetbuttcap%
\pgfsetroundjoin%
\definecolor{currentfill}{rgb}{0.000000,0.000000,0.000000}%
\pgfsetfillcolor{currentfill}%
\pgfsetlinewidth{1.003750pt}%
\definecolor{currentstroke}{rgb}{0.000000,0.000000,0.000000}%
\pgfsetstrokecolor{currentstroke}%
\pgfsetdash{}{0pt}%
\pgfpathmoveto{\pgfqpoint{9.615898in}{6.111059in}}%
\pgfpathcurveto{\pgfqpoint{9.626949in}{6.111059in}}{\pgfqpoint{9.637548in}{6.115449in}}{\pgfqpoint{9.645361in}{6.123263in}}%
\pgfpathcurveto{\pgfqpoint{9.653175in}{6.131076in}}{\pgfqpoint{9.657565in}{6.141675in}}{\pgfqpoint{9.657565in}{6.152725in}}%
\pgfpathcurveto{\pgfqpoint{9.657565in}{6.163776in}}{\pgfqpoint{9.653175in}{6.174375in}}{\pgfqpoint{9.645361in}{6.182188in}}%
\pgfpathcurveto{\pgfqpoint{9.637548in}{6.190002in}}{\pgfqpoint{9.626949in}{6.194392in}}{\pgfqpoint{9.615898in}{6.194392in}}%
\pgfpathcurveto{\pgfqpoint{9.604848in}{6.194392in}}{\pgfqpoint{9.594249in}{6.190002in}}{\pgfqpoint{9.586436in}{6.182188in}}%
\pgfpathcurveto{\pgfqpoint{9.578622in}{6.174375in}}{\pgfqpoint{9.574232in}{6.163776in}}{\pgfqpoint{9.574232in}{6.152725in}}%
\pgfpathcurveto{\pgfqpoint{9.574232in}{6.141675in}}{\pgfqpoint{9.578622in}{6.131076in}}{\pgfqpoint{9.586436in}{6.123263in}}%
\pgfpathcurveto{\pgfqpoint{9.594249in}{6.115449in}}{\pgfqpoint{9.604848in}{6.111059in}}{\pgfqpoint{9.615898in}{6.111059in}}%
\pgfpathclose%
\pgfusepath{stroke,fill}%
\end{pgfscope}%
\begin{pgfscope}%
\pgfpathrectangle{\pgfqpoint{7.640588in}{4.121437in}}{\pgfqpoint{5.699255in}{2.685432in}}%
\pgfusepath{clip}%
\pgfsetbuttcap%
\pgfsetroundjoin%
\definecolor{currentfill}{rgb}{0.000000,0.000000,0.000000}%
\pgfsetfillcolor{currentfill}%
\pgfsetlinewidth{1.003750pt}%
\definecolor{currentstroke}{rgb}{0.000000,0.000000,0.000000}%
\pgfsetstrokecolor{currentstroke}%
\pgfsetdash{}{0pt}%
\pgfpathmoveto{\pgfqpoint{9.842573in}{6.236254in}}%
\pgfpathcurveto{\pgfqpoint{9.853624in}{6.236254in}}{\pgfqpoint{9.864223in}{6.240644in}}{\pgfqpoint{9.872036in}{6.248458in}}%
\pgfpathcurveto{\pgfqpoint{9.879850in}{6.256271in}}{\pgfqpoint{9.884240in}{6.266870in}}{\pgfqpoint{9.884240in}{6.277920in}}%
\pgfpathcurveto{\pgfqpoint{9.884240in}{6.288971in}}{\pgfqpoint{9.879850in}{6.299570in}}{\pgfqpoint{9.872036in}{6.307383in}}%
\pgfpathcurveto{\pgfqpoint{9.864223in}{6.315197in}}{\pgfqpoint{9.853624in}{6.319587in}}{\pgfqpoint{9.842573in}{6.319587in}}%
\pgfpathcurveto{\pgfqpoint{9.831523in}{6.319587in}}{\pgfqpoint{9.820924in}{6.315197in}}{\pgfqpoint{9.813111in}{6.307383in}}%
\pgfpathcurveto{\pgfqpoint{9.805297in}{6.299570in}}{\pgfqpoint{9.800907in}{6.288971in}}{\pgfqpoint{9.800907in}{6.277920in}}%
\pgfpathcurveto{\pgfqpoint{9.800907in}{6.266870in}}{\pgfqpoint{9.805297in}{6.256271in}}{\pgfqpoint{9.813111in}{6.248458in}}%
\pgfpathcurveto{\pgfqpoint{9.820924in}{6.240644in}}{\pgfqpoint{9.831523in}{6.236254in}}{\pgfqpoint{9.842573in}{6.236254in}}%
\pgfpathclose%
\pgfusepath{stroke,fill}%
\end{pgfscope}%
\begin{pgfscope}%
\pgfpathrectangle{\pgfqpoint{7.640588in}{4.121437in}}{\pgfqpoint{5.699255in}{2.685432in}}%
\pgfusepath{clip}%
\pgfsetbuttcap%
\pgfsetroundjoin%
\definecolor{currentfill}{rgb}{0.000000,0.000000,0.000000}%
\pgfsetfillcolor{currentfill}%
\pgfsetlinewidth{1.003750pt}%
\definecolor{currentstroke}{rgb}{0.000000,0.000000,0.000000}%
\pgfsetstrokecolor{currentstroke}%
\pgfsetdash{}{0pt}%
\pgfpathmoveto{\pgfqpoint{10.198777in}{6.283202in}}%
\pgfpathcurveto{\pgfqpoint{10.209827in}{6.283202in}}{\pgfqpoint{10.220426in}{6.287592in}}{\pgfqpoint{10.228240in}{6.295406in}}%
\pgfpathcurveto{\pgfqpoint{10.236053in}{6.303219in}}{\pgfqpoint{10.240444in}{6.313818in}}{\pgfqpoint{10.240444in}{6.324869in}}%
\pgfpathcurveto{\pgfqpoint{10.240444in}{6.335919in}}{\pgfqpoint{10.236053in}{6.346518in}}{\pgfqpoint{10.228240in}{6.354331in}}%
\pgfpathcurveto{\pgfqpoint{10.220426in}{6.362145in}}{\pgfqpoint{10.209827in}{6.366535in}}{\pgfqpoint{10.198777in}{6.366535in}}%
\pgfpathcurveto{\pgfqpoint{10.187727in}{6.366535in}}{\pgfqpoint{10.177128in}{6.362145in}}{\pgfqpoint{10.169314in}{6.354331in}}%
\pgfpathcurveto{\pgfqpoint{10.161500in}{6.346518in}}{\pgfqpoint{10.157110in}{6.335919in}}{\pgfqpoint{10.157110in}{6.324869in}}%
\pgfpathcurveto{\pgfqpoint{10.157110in}{6.313818in}}{\pgfqpoint{10.161500in}{6.303219in}}{\pgfqpoint{10.169314in}{6.295406in}}%
\pgfpathcurveto{\pgfqpoint{10.177128in}{6.287592in}}{\pgfqpoint{10.187727in}{6.283202in}}{\pgfqpoint{10.198777in}{6.283202in}}%
\pgfpathclose%
\pgfusepath{stroke,fill}%
\end{pgfscope}%
\begin{pgfscope}%
\pgfpathrectangle{\pgfqpoint{7.640588in}{4.121437in}}{\pgfqpoint{5.699255in}{2.685432in}}%
\pgfusepath{clip}%
\pgfsetbuttcap%
\pgfsetroundjoin%
\definecolor{currentfill}{rgb}{0.000000,0.000000,0.000000}%
\pgfsetfillcolor{currentfill}%
\pgfsetlinewidth{1.003750pt}%
\definecolor{currentstroke}{rgb}{0.000000,0.000000,0.000000}%
\pgfsetstrokecolor{currentstroke}%
\pgfsetdash{}{0pt}%
\pgfpathmoveto{\pgfqpoint{10.814037in}{6.377098in}}%
\pgfpathcurveto{\pgfqpoint{10.825088in}{6.377098in}}{\pgfqpoint{10.835687in}{6.381488in}}{\pgfqpoint{10.843500in}{6.389302in}}%
\pgfpathcurveto{\pgfqpoint{10.851314in}{6.397116in}}{\pgfqpoint{10.855704in}{6.407715in}}{\pgfqpoint{10.855704in}{6.418765in}}%
\pgfpathcurveto{\pgfqpoint{10.855704in}{6.429815in}}{\pgfqpoint{10.851314in}{6.440414in}}{\pgfqpoint{10.843500in}{6.448228in}}%
\pgfpathcurveto{\pgfqpoint{10.835687in}{6.456041in}}{\pgfqpoint{10.825088in}{6.460431in}}{\pgfqpoint{10.814037in}{6.460431in}}%
\pgfpathcurveto{\pgfqpoint{10.802987in}{6.460431in}}{\pgfqpoint{10.792388in}{6.456041in}}{\pgfqpoint{10.784575in}{6.448228in}}%
\pgfpathcurveto{\pgfqpoint{10.776761in}{6.440414in}}{\pgfqpoint{10.772371in}{6.429815in}}{\pgfqpoint{10.772371in}{6.418765in}}%
\pgfpathcurveto{\pgfqpoint{10.772371in}{6.407715in}}{\pgfqpoint{10.776761in}{6.397116in}}{\pgfqpoint{10.784575in}{6.389302in}}%
\pgfpathcurveto{\pgfqpoint{10.792388in}{6.381488in}}{\pgfqpoint{10.802987in}{6.377098in}}{\pgfqpoint{10.814037in}{6.377098in}}%
\pgfpathclose%
\pgfusepath{stroke,fill}%
\end{pgfscope}%
\begin{pgfscope}%
\pgfpathrectangle{\pgfqpoint{7.640588in}{4.121437in}}{\pgfqpoint{5.699255in}{2.685432in}}%
\pgfusepath{clip}%
\pgfsetbuttcap%
\pgfsetroundjoin%
\definecolor{currentfill}{rgb}{0.000000,0.000000,0.000000}%
\pgfsetfillcolor{currentfill}%
\pgfsetlinewidth{1.003750pt}%
\definecolor{currentstroke}{rgb}{0.000000,0.000000,0.000000}%
\pgfsetstrokecolor{currentstroke}%
\pgfsetdash{}{0pt}%
\pgfpathmoveto{\pgfqpoint{11.008330in}{6.408397in}}%
\pgfpathcurveto{\pgfqpoint{11.019380in}{6.408397in}}{\pgfqpoint{11.029979in}{6.412787in}}{\pgfqpoint{11.037793in}{6.420601in}}%
\pgfpathcurveto{\pgfqpoint{11.045607in}{6.428414in}}{\pgfqpoint{11.049997in}{6.439013in}}{\pgfqpoint{11.049997in}{6.450064in}}%
\pgfpathcurveto{\pgfqpoint{11.049997in}{6.461114in}}{\pgfqpoint{11.045607in}{6.471713in}}{\pgfqpoint{11.037793in}{6.479526in}}%
\pgfpathcurveto{\pgfqpoint{11.029979in}{6.487340in}}{\pgfqpoint{11.019380in}{6.491730in}}{\pgfqpoint{11.008330in}{6.491730in}}%
\pgfpathcurveto{\pgfqpoint{10.997280in}{6.491730in}}{\pgfqpoint{10.986681in}{6.487340in}}{\pgfqpoint{10.978867in}{6.479526in}}%
\pgfpathcurveto{\pgfqpoint{10.971054in}{6.471713in}}{\pgfqpoint{10.966664in}{6.461114in}}{\pgfqpoint{10.966664in}{6.450064in}}%
\pgfpathcurveto{\pgfqpoint{10.966664in}{6.439013in}}{\pgfqpoint{10.971054in}{6.428414in}}{\pgfqpoint{10.978867in}{6.420601in}}%
\pgfpathcurveto{\pgfqpoint{10.986681in}{6.412787in}}{\pgfqpoint{10.997280in}{6.408397in}}{\pgfqpoint{11.008330in}{6.408397in}}%
\pgfpathclose%
\pgfusepath{stroke,fill}%
\end{pgfscope}%
\begin{pgfscope}%
\pgfpathrectangle{\pgfqpoint{7.640588in}{4.121437in}}{\pgfqpoint{5.699255in}{2.685432in}}%
\pgfusepath{clip}%
\pgfsetbuttcap%
\pgfsetroundjoin%
\definecolor{currentfill}{rgb}{0.000000,0.000000,0.000000}%
\pgfsetfillcolor{currentfill}%
\pgfsetlinewidth{1.003750pt}%
\definecolor{currentstroke}{rgb}{0.000000,0.000000,0.000000}%
\pgfsetstrokecolor{currentstroke}%
\pgfsetdash{}{0pt}%
\pgfpathmoveto{\pgfqpoint{11.332152in}{6.377098in}}%
\pgfpathcurveto{\pgfqpoint{11.343202in}{6.377098in}}{\pgfqpoint{11.353801in}{6.381488in}}{\pgfqpoint{11.361614in}{6.389302in}}%
\pgfpathcurveto{\pgfqpoint{11.369428in}{6.397116in}}{\pgfqpoint{11.373818in}{6.407715in}}{\pgfqpoint{11.373818in}{6.418765in}}%
\pgfpathcurveto{\pgfqpoint{11.373818in}{6.429815in}}{\pgfqpoint{11.369428in}{6.440414in}}{\pgfqpoint{11.361614in}{6.448228in}}%
\pgfpathcurveto{\pgfqpoint{11.353801in}{6.456041in}}{\pgfqpoint{11.343202in}{6.460431in}}{\pgfqpoint{11.332152in}{6.460431in}}%
\pgfpathcurveto{\pgfqpoint{11.321101in}{6.460431in}}{\pgfqpoint{11.310502in}{6.456041in}}{\pgfqpoint{11.302689in}{6.448228in}}%
\pgfpathcurveto{\pgfqpoint{11.294875in}{6.440414in}}{\pgfqpoint{11.290485in}{6.429815in}}{\pgfqpoint{11.290485in}{6.418765in}}%
\pgfpathcurveto{\pgfqpoint{11.290485in}{6.407715in}}{\pgfqpoint{11.294875in}{6.397116in}}{\pgfqpoint{11.302689in}{6.389302in}}%
\pgfpathcurveto{\pgfqpoint{11.310502in}{6.381488in}}{\pgfqpoint{11.321101in}{6.377098in}}{\pgfqpoint{11.332152in}{6.377098in}}%
\pgfpathclose%
\pgfusepath{stroke,fill}%
\end{pgfscope}%
\begin{pgfscope}%
\pgfpathrectangle{\pgfqpoint{7.640588in}{4.121437in}}{\pgfqpoint{5.699255in}{2.685432in}}%
\pgfusepath{clip}%
\pgfsetbuttcap%
\pgfsetroundjoin%
\definecolor{currentfill}{rgb}{0.000000,0.000000,0.000000}%
\pgfsetfillcolor{currentfill}%
\pgfsetlinewidth{1.003750pt}%
\definecolor{currentstroke}{rgb}{0.000000,0.000000,0.000000}%
\pgfsetstrokecolor{currentstroke}%
\pgfsetdash{}{0pt}%
\pgfpathmoveto{\pgfqpoint{11.494062in}{6.345799in}}%
\pgfpathcurveto{\pgfqpoint{11.505112in}{6.345799in}}{\pgfqpoint{11.515711in}{6.350190in}}{\pgfqpoint{11.523525in}{6.358003in}}%
\pgfpathcurveto{\pgfqpoint{11.531339in}{6.365817in}}{\pgfqpoint{11.535729in}{6.376416in}}{\pgfqpoint{11.535729in}{6.387466in}}%
\pgfpathcurveto{\pgfqpoint{11.535729in}{6.398516in}}{\pgfqpoint{11.531339in}{6.409115in}}{\pgfqpoint{11.523525in}{6.416929in}}%
\pgfpathcurveto{\pgfqpoint{11.515711in}{6.424742in}}{\pgfqpoint{11.505112in}{6.429133in}}{\pgfqpoint{11.494062in}{6.429133in}}%
\pgfpathcurveto{\pgfqpoint{11.483012in}{6.429133in}}{\pgfqpoint{11.472413in}{6.424742in}}{\pgfqpoint{11.464599in}{6.416929in}}%
\pgfpathcurveto{\pgfqpoint{11.456786in}{6.409115in}}{\pgfqpoint{11.452396in}{6.398516in}}{\pgfqpoint{11.452396in}{6.387466in}}%
\pgfpathcurveto{\pgfqpoint{11.452396in}{6.376416in}}{\pgfqpoint{11.456786in}{6.365817in}}{\pgfqpoint{11.464599in}{6.358003in}}%
\pgfpathcurveto{\pgfqpoint{11.472413in}{6.350190in}}{\pgfqpoint{11.483012in}{6.345799in}}{\pgfqpoint{11.494062in}{6.345799in}}%
\pgfpathclose%
\pgfusepath{stroke,fill}%
\end{pgfscope}%
\begin{pgfscope}%
\pgfpathrectangle{\pgfqpoint{7.640588in}{4.121437in}}{\pgfqpoint{5.699255in}{2.685432in}}%
\pgfusepath{clip}%
\pgfsetbuttcap%
\pgfsetroundjoin%
\definecolor{currentfill}{rgb}{0.000000,0.000000,0.000000}%
\pgfsetfillcolor{currentfill}%
\pgfsetlinewidth{1.003750pt}%
\definecolor{currentstroke}{rgb}{0.000000,0.000000,0.000000}%
\pgfsetstrokecolor{currentstroke}%
\pgfsetdash{}{0pt}%
\pgfpathmoveto{\pgfqpoint{11.558826in}{6.392748in}}%
\pgfpathcurveto{\pgfqpoint{11.569877in}{6.392748in}}{\pgfqpoint{11.580476in}{6.397138in}}{\pgfqpoint{11.588289in}{6.404951in}}%
\pgfpathcurveto{\pgfqpoint{11.596103in}{6.412765in}}{\pgfqpoint{11.600493in}{6.423364in}}{\pgfqpoint{11.600493in}{6.434414in}}%
\pgfpathcurveto{\pgfqpoint{11.600493in}{6.445464in}}{\pgfqpoint{11.596103in}{6.456063in}}{\pgfqpoint{11.588289in}{6.463877in}}%
\pgfpathcurveto{\pgfqpoint{11.580476in}{6.471691in}}{\pgfqpoint{11.569877in}{6.476081in}}{\pgfqpoint{11.558826in}{6.476081in}}%
\pgfpathcurveto{\pgfqpoint{11.547776in}{6.476081in}}{\pgfqpoint{11.537177in}{6.471691in}}{\pgfqpoint{11.529364in}{6.463877in}}%
\pgfpathcurveto{\pgfqpoint{11.521550in}{6.456063in}}{\pgfqpoint{11.517160in}{6.445464in}}{\pgfqpoint{11.517160in}{6.434414in}}%
\pgfpathcurveto{\pgfqpoint{11.517160in}{6.423364in}}{\pgfqpoint{11.521550in}{6.412765in}}{\pgfqpoint{11.529364in}{6.404951in}}%
\pgfpathcurveto{\pgfqpoint{11.537177in}{6.397138in}}{\pgfqpoint{11.547776in}{6.392748in}}{\pgfqpoint{11.558826in}{6.392748in}}%
\pgfpathclose%
\pgfusepath{stroke,fill}%
\end{pgfscope}%
\begin{pgfscope}%
\pgfpathrectangle{\pgfqpoint{7.640588in}{4.121437in}}{\pgfqpoint{5.699255in}{2.685432in}}%
\pgfusepath{clip}%
\pgfsetbuttcap%
\pgfsetroundjoin%
\definecolor{currentfill}{rgb}{0.000000,0.000000,0.000000}%
\pgfsetfillcolor{currentfill}%
\pgfsetlinewidth{1.003750pt}%
\definecolor{currentstroke}{rgb}{0.000000,0.000000,0.000000}%
\pgfsetstrokecolor{currentstroke}%
\pgfsetdash{}{0pt}%
\pgfpathmoveto{\pgfqpoint{11.785501in}{6.267553in}}%
\pgfpathcurveto{\pgfqpoint{11.796552in}{6.267553in}}{\pgfqpoint{11.807151in}{6.271943in}}{\pgfqpoint{11.814964in}{6.279756in}}%
\pgfpathcurveto{\pgfqpoint{11.822778in}{6.287570in}}{\pgfqpoint{11.827168in}{6.298169in}}{\pgfqpoint{11.827168in}{6.309219in}}%
\pgfpathcurveto{\pgfqpoint{11.827168in}{6.320269in}}{\pgfqpoint{11.822778in}{6.330868in}}{\pgfqpoint{11.814964in}{6.338682in}}%
\pgfpathcurveto{\pgfqpoint{11.807151in}{6.346496in}}{\pgfqpoint{11.796552in}{6.350886in}}{\pgfqpoint{11.785501in}{6.350886in}}%
\pgfpathcurveto{\pgfqpoint{11.774451in}{6.350886in}}{\pgfqpoint{11.763852in}{6.346496in}}{\pgfqpoint{11.756039in}{6.338682in}}%
\pgfpathcurveto{\pgfqpoint{11.748225in}{6.330868in}}{\pgfqpoint{11.743835in}{6.320269in}}{\pgfqpoint{11.743835in}{6.309219in}}%
\pgfpathcurveto{\pgfqpoint{11.743835in}{6.298169in}}{\pgfqpoint{11.748225in}{6.287570in}}{\pgfqpoint{11.756039in}{6.279756in}}%
\pgfpathcurveto{\pgfqpoint{11.763852in}{6.271943in}}{\pgfqpoint{11.774451in}{6.267553in}}{\pgfqpoint{11.785501in}{6.267553in}}%
\pgfpathclose%
\pgfusepath{stroke,fill}%
\end{pgfscope}%
\begin{pgfscope}%
\pgfpathrectangle{\pgfqpoint{7.640588in}{4.121437in}}{\pgfqpoint{5.699255in}{2.685432in}}%
\pgfusepath{clip}%
\pgfsetbuttcap%
\pgfsetroundjoin%
\definecolor{currentfill}{rgb}{0.000000,0.000000,0.000000}%
\pgfsetfillcolor{currentfill}%
\pgfsetlinewidth{1.003750pt}%
\definecolor{currentstroke}{rgb}{0.000000,0.000000,0.000000}%
\pgfsetstrokecolor{currentstroke}%
\pgfsetdash{}{0pt}%
\pgfpathmoveto{\pgfqpoint{12.206469in}{6.596189in}}%
\pgfpathcurveto{\pgfqpoint{12.217519in}{6.596189in}}{\pgfqpoint{12.228118in}{6.600580in}}{\pgfqpoint{12.235932in}{6.608393in}}%
\pgfpathcurveto{\pgfqpoint{12.243746in}{6.616207in}}{\pgfqpoint{12.248136in}{6.626806in}}{\pgfqpoint{12.248136in}{6.637856in}}%
\pgfpathcurveto{\pgfqpoint{12.248136in}{6.648906in}}{\pgfqpoint{12.243746in}{6.659505in}}{\pgfqpoint{12.235932in}{6.667319in}}%
\pgfpathcurveto{\pgfqpoint{12.228118in}{6.675132in}}{\pgfqpoint{12.217519in}{6.679523in}}{\pgfqpoint{12.206469in}{6.679523in}}%
\pgfpathcurveto{\pgfqpoint{12.195419in}{6.679523in}}{\pgfqpoint{12.184820in}{6.675132in}}{\pgfqpoint{12.177006in}{6.667319in}}%
\pgfpathcurveto{\pgfqpoint{12.169193in}{6.659505in}}{\pgfqpoint{12.164802in}{6.648906in}}{\pgfqpoint{12.164802in}{6.637856in}}%
\pgfpathcurveto{\pgfqpoint{12.164802in}{6.626806in}}{\pgfqpoint{12.169193in}{6.616207in}}{\pgfqpoint{12.177006in}{6.608393in}}%
\pgfpathcurveto{\pgfqpoint{12.184820in}{6.600580in}}{\pgfqpoint{12.195419in}{6.596189in}}{\pgfqpoint{12.206469in}{6.596189in}}%
\pgfpathclose%
\pgfusepath{stroke,fill}%
\end{pgfscope}%
\begin{pgfscope}%
\pgfpathrectangle{\pgfqpoint{7.640588in}{4.121437in}}{\pgfqpoint{5.699255in}{2.685432in}}%
\pgfusepath{clip}%
\pgfsetbuttcap%
\pgfsetroundjoin%
\definecolor{currentfill}{rgb}{0.000000,0.000000,0.000000}%
\pgfsetfillcolor{currentfill}%
\pgfsetlinewidth{1.003750pt}%
\definecolor{currentstroke}{rgb}{0.000000,0.000000,0.000000}%
\pgfsetstrokecolor{currentstroke}%
\pgfsetdash{}{0pt}%
\pgfpathmoveto{\pgfqpoint{12.433144in}{6.580540in}}%
\pgfpathcurveto{\pgfqpoint{12.444194in}{6.580540in}}{\pgfqpoint{12.454793in}{6.584930in}}{\pgfqpoint{12.462607in}{6.592744in}}%
\pgfpathcurveto{\pgfqpoint{12.470420in}{6.600557in}}{\pgfqpoint{12.474811in}{6.611157in}}{\pgfqpoint{12.474811in}{6.622207in}}%
\pgfpathcurveto{\pgfqpoint{12.474811in}{6.633257in}}{\pgfqpoint{12.470420in}{6.643856in}}{\pgfqpoint{12.462607in}{6.651669in}}%
\pgfpathcurveto{\pgfqpoint{12.454793in}{6.659483in}}{\pgfqpoint{12.444194in}{6.663873in}}{\pgfqpoint{12.433144in}{6.663873in}}%
\pgfpathcurveto{\pgfqpoint{12.422094in}{6.663873in}}{\pgfqpoint{12.411495in}{6.659483in}}{\pgfqpoint{12.403681in}{6.651669in}}%
\pgfpathcurveto{\pgfqpoint{12.395868in}{6.643856in}}{\pgfqpoint{12.391477in}{6.633257in}}{\pgfqpoint{12.391477in}{6.622207in}}%
\pgfpathcurveto{\pgfqpoint{12.391477in}{6.611157in}}{\pgfqpoint{12.395868in}{6.600557in}}{\pgfqpoint{12.403681in}{6.592744in}}%
\pgfpathcurveto{\pgfqpoint{12.411495in}{6.584930in}}{\pgfqpoint{12.422094in}{6.580540in}}{\pgfqpoint{12.433144in}{6.580540in}}%
\pgfpathclose%
\pgfusepath{stroke,fill}%
\end{pgfscope}%
\begin{pgfscope}%
\pgfpathrectangle{\pgfqpoint{7.640588in}{4.121437in}}{\pgfqpoint{5.699255in}{2.685432in}}%
\pgfusepath{clip}%
\pgfsetbuttcap%
\pgfsetroundjoin%
\definecolor{currentfill}{rgb}{0.000000,0.000000,0.000000}%
\pgfsetfillcolor{currentfill}%
\pgfsetlinewidth{1.003750pt}%
\definecolor{currentstroke}{rgb}{0.000000,0.000000,0.000000}%
\pgfsetstrokecolor{currentstroke}%
\pgfsetdash{}{0pt}%
\pgfpathmoveto{\pgfqpoint{12.465526in}{6.611839in}}%
\pgfpathcurveto{\pgfqpoint{12.476576in}{6.611839in}}{\pgfqpoint{12.487175in}{6.616229in}}{\pgfqpoint{12.494989in}{6.624043in}}%
\pgfpathcurveto{\pgfqpoint{12.502803in}{6.631856in}}{\pgfqpoint{12.507193in}{6.642455in}}{\pgfqpoint{12.507193in}{6.653505in}}%
\pgfpathcurveto{\pgfqpoint{12.507193in}{6.664556in}}{\pgfqpoint{12.502803in}{6.675155in}}{\pgfqpoint{12.494989in}{6.682968in}}%
\pgfpathcurveto{\pgfqpoint{12.487175in}{6.690782in}}{\pgfqpoint{12.476576in}{6.695172in}}{\pgfqpoint{12.465526in}{6.695172in}}%
\pgfpathcurveto{\pgfqpoint{12.454476in}{6.695172in}}{\pgfqpoint{12.443877in}{6.690782in}}{\pgfqpoint{12.436063in}{6.682968in}}%
\pgfpathcurveto{\pgfqpoint{12.428250in}{6.675155in}}{\pgfqpoint{12.423859in}{6.664556in}}{\pgfqpoint{12.423859in}{6.653505in}}%
\pgfpathcurveto{\pgfqpoint{12.423859in}{6.642455in}}{\pgfqpoint{12.428250in}{6.631856in}}{\pgfqpoint{12.436063in}{6.624043in}}%
\pgfpathcurveto{\pgfqpoint{12.443877in}{6.616229in}}{\pgfqpoint{12.454476in}{6.611839in}}{\pgfqpoint{12.465526in}{6.611839in}}%
\pgfpathclose%
\pgfusepath{stroke,fill}%
\end{pgfscope}%
\begin{pgfscope}%
\pgfpathrectangle{\pgfqpoint{7.640588in}{4.121437in}}{\pgfqpoint{5.699255in}{2.685432in}}%
\pgfusepath{clip}%
\pgfsetbuttcap%
\pgfsetroundjoin%
\definecolor{currentfill}{rgb}{0.000000,0.000000,0.000000}%
\pgfsetfillcolor{currentfill}%
\pgfsetlinewidth{1.003750pt}%
\definecolor{currentstroke}{rgb}{0.000000,0.000000,0.000000}%
\pgfsetstrokecolor{currentstroke}%
\pgfsetdash{}{0pt}%
\pgfpathmoveto{\pgfqpoint{12.789347in}{6.643137in}}%
\pgfpathcurveto{\pgfqpoint{12.800398in}{6.643137in}}{\pgfqpoint{12.810997in}{6.647528in}}{\pgfqpoint{12.818810in}{6.655341in}}%
\pgfpathcurveto{\pgfqpoint{12.826624in}{6.663155in}}{\pgfqpoint{12.831014in}{6.673754in}}{\pgfqpoint{12.831014in}{6.684804in}}%
\pgfpathcurveto{\pgfqpoint{12.831014in}{6.695854in}}{\pgfqpoint{12.826624in}{6.706453in}}{\pgfqpoint{12.818810in}{6.714267in}}%
\pgfpathcurveto{\pgfqpoint{12.810997in}{6.722081in}}{\pgfqpoint{12.800398in}{6.726471in}}{\pgfqpoint{12.789347in}{6.726471in}}%
\pgfpathcurveto{\pgfqpoint{12.778297in}{6.726471in}}{\pgfqpoint{12.767698in}{6.722081in}}{\pgfqpoint{12.759885in}{6.714267in}}%
\pgfpathcurveto{\pgfqpoint{12.752071in}{6.706453in}}{\pgfqpoint{12.747681in}{6.695854in}}{\pgfqpoint{12.747681in}{6.684804in}}%
\pgfpathcurveto{\pgfqpoint{12.747681in}{6.673754in}}{\pgfqpoint{12.752071in}{6.663155in}}{\pgfqpoint{12.759885in}{6.655341in}}%
\pgfpathcurveto{\pgfqpoint{12.767698in}{6.647528in}}{\pgfqpoint{12.778297in}{6.643137in}}{\pgfqpoint{12.789347in}{6.643137in}}%
\pgfpathclose%
\pgfusepath{stroke,fill}%
\end{pgfscope}%
\begin{pgfscope}%
\pgfpathrectangle{\pgfqpoint{7.640588in}{4.121437in}}{\pgfqpoint{5.699255in}{2.685432in}}%
\pgfusepath{clip}%
\pgfsetbuttcap%
\pgfsetroundjoin%
\definecolor{currentfill}{rgb}{0.000000,0.000000,0.000000}%
\pgfsetfillcolor{currentfill}%
\pgfsetlinewidth{1.003750pt}%
\definecolor{currentstroke}{rgb}{0.000000,0.000000,0.000000}%
\pgfsetstrokecolor{currentstroke}%
\pgfsetdash{}{0pt}%
\pgfpathmoveto{\pgfqpoint{12.821730in}{6.392748in}}%
\pgfpathcurveto{\pgfqpoint{12.832780in}{6.392748in}}{\pgfqpoint{12.843379in}{6.397138in}}{\pgfqpoint{12.851192in}{6.404951in}}%
\pgfpathcurveto{\pgfqpoint{12.859006in}{6.412765in}}{\pgfqpoint{12.863396in}{6.423364in}}{\pgfqpoint{12.863396in}{6.434414in}}%
\pgfpathcurveto{\pgfqpoint{12.863396in}{6.445464in}}{\pgfqpoint{12.859006in}{6.456063in}}{\pgfqpoint{12.851192in}{6.463877in}}%
\pgfpathcurveto{\pgfqpoint{12.843379in}{6.471691in}}{\pgfqpoint{12.832780in}{6.476081in}}{\pgfqpoint{12.821730in}{6.476081in}}%
\pgfpathcurveto{\pgfqpoint{12.810679in}{6.476081in}}{\pgfqpoint{12.800080in}{6.471691in}}{\pgfqpoint{12.792267in}{6.463877in}}%
\pgfpathcurveto{\pgfqpoint{12.784453in}{6.456063in}}{\pgfqpoint{12.780063in}{6.445464in}}{\pgfqpoint{12.780063in}{6.434414in}}%
\pgfpathcurveto{\pgfqpoint{12.780063in}{6.423364in}}{\pgfqpoint{12.784453in}{6.412765in}}{\pgfqpoint{12.792267in}{6.404951in}}%
\pgfpathcurveto{\pgfqpoint{12.800080in}{6.397138in}}{\pgfqpoint{12.810679in}{6.392748in}}{\pgfqpoint{12.821730in}{6.392748in}}%
\pgfpathclose%
\pgfusepath{stroke,fill}%
\end{pgfscope}%
\begin{pgfscope}%
\pgfpathrectangle{\pgfqpoint{7.640588in}{4.121437in}}{\pgfqpoint{5.699255in}{2.685432in}}%
\pgfusepath{clip}%
\pgfsetbuttcap%
\pgfsetroundjoin%
\definecolor{currentfill}{rgb}{0.000000,0.000000,0.000000}%
\pgfsetfillcolor{currentfill}%
\pgfsetlinewidth{1.003750pt}%
\definecolor{currentstroke}{rgb}{0.000000,0.000000,0.000000}%
\pgfsetstrokecolor{currentstroke}%
\pgfsetdash{}{0pt}%
\pgfpathmoveto{\pgfqpoint{13.080787in}{6.298851in}}%
\pgfpathcurveto{\pgfqpoint{13.091837in}{6.298851in}}{\pgfqpoint{13.102436in}{6.303242in}}{\pgfqpoint{13.110249in}{6.311055in}}%
\pgfpathcurveto{\pgfqpoint{13.118063in}{6.318869in}}{\pgfqpoint{13.122453in}{6.329468in}}{\pgfqpoint{13.122453in}{6.340518in}}%
\pgfpathcurveto{\pgfqpoint{13.122453in}{6.351568in}}{\pgfqpoint{13.118063in}{6.362167in}}{\pgfqpoint{13.110249in}{6.369981in}}%
\pgfpathcurveto{\pgfqpoint{13.102436in}{6.377794in}}{\pgfqpoint{13.091837in}{6.382185in}}{\pgfqpoint{13.080787in}{6.382185in}}%
\pgfpathcurveto{\pgfqpoint{13.069737in}{6.382185in}}{\pgfqpoint{13.059138in}{6.377794in}}{\pgfqpoint{13.051324in}{6.369981in}}%
\pgfpathcurveto{\pgfqpoint{13.043510in}{6.362167in}}{\pgfqpoint{13.039120in}{6.351568in}}{\pgfqpoint{13.039120in}{6.340518in}}%
\pgfpathcurveto{\pgfqpoint{13.039120in}{6.329468in}}{\pgfqpoint{13.043510in}{6.318869in}}{\pgfqpoint{13.051324in}{6.311055in}}%
\pgfpathcurveto{\pgfqpoint{13.059138in}{6.303242in}}{\pgfqpoint{13.069737in}{6.298851in}}{\pgfqpoint{13.080787in}{6.298851in}}%
\pgfpathclose%
\pgfusepath{stroke,fill}%
\end{pgfscope}%
\begin{pgfscope}%
\pgfpathrectangle{\pgfqpoint{7.640588in}{4.121437in}}{\pgfqpoint{5.699255in}{2.685432in}}%
\pgfusepath{clip}%
\pgfsetbuttcap%
\pgfsetroundjoin%
\definecolor{currentfill}{rgb}{0.000000,0.000000,0.000000}%
\pgfsetfillcolor{currentfill}%
\pgfsetlinewidth{1.003750pt}%
\definecolor{currentstroke}{rgb}{0.000000,0.000000,0.000000}%
\pgfsetstrokecolor{currentstroke}%
\pgfsetdash{}{0pt}%
\pgfpathmoveto{\pgfqpoint{12.141705in}{6.267553in}}%
\pgfpathcurveto{\pgfqpoint{12.152755in}{6.267553in}}{\pgfqpoint{12.163354in}{6.271943in}}{\pgfqpoint{12.171168in}{6.279756in}}%
\pgfpathcurveto{\pgfqpoint{12.178981in}{6.287570in}}{\pgfqpoint{12.183372in}{6.298169in}}{\pgfqpoint{12.183372in}{6.309219in}}%
\pgfpathcurveto{\pgfqpoint{12.183372in}{6.320269in}}{\pgfqpoint{12.178981in}{6.330868in}}{\pgfqpoint{12.171168in}{6.338682in}}%
\pgfpathcurveto{\pgfqpoint{12.163354in}{6.346496in}}{\pgfqpoint{12.152755in}{6.350886in}}{\pgfqpoint{12.141705in}{6.350886in}}%
\pgfpathcurveto{\pgfqpoint{12.130655in}{6.350886in}}{\pgfqpoint{12.120056in}{6.346496in}}{\pgfqpoint{12.112242in}{6.338682in}}%
\pgfpathcurveto{\pgfqpoint{12.104428in}{6.330868in}}{\pgfqpoint{12.100038in}{6.320269in}}{\pgfqpoint{12.100038in}{6.309219in}}%
\pgfpathcurveto{\pgfqpoint{12.100038in}{6.298169in}}{\pgfqpoint{12.104428in}{6.287570in}}{\pgfqpoint{12.112242in}{6.279756in}}%
\pgfpathcurveto{\pgfqpoint{12.120056in}{6.271943in}}{\pgfqpoint{12.130655in}{6.267553in}}{\pgfqpoint{12.141705in}{6.267553in}}%
\pgfpathclose%
\pgfusepath{stroke,fill}%
\end{pgfscope}%
\begin{pgfscope}%
\pgfpathrectangle{\pgfqpoint{7.640588in}{4.121437in}}{\pgfqpoint{5.699255in}{2.685432in}}%
\pgfusepath{clip}%
\pgfsetbuttcap%
\pgfsetroundjoin%
\definecolor{currentfill}{rgb}{0.000000,0.000000,0.000000}%
\pgfsetfillcolor{currentfill}%
\pgfsetlinewidth{1.003750pt}%
\definecolor{currentstroke}{rgb}{0.000000,0.000000,0.000000}%
\pgfsetstrokecolor{currentstroke}%
\pgfsetdash{}{0pt}%
\pgfpathmoveto{\pgfqpoint{12.044558in}{6.001513in}}%
\pgfpathcurveto{\pgfqpoint{12.055609in}{6.001513in}}{\pgfqpoint{12.066208in}{6.005903in}}{\pgfqpoint{12.074021in}{6.013717in}}%
\pgfpathcurveto{\pgfqpoint{12.081835in}{6.021531in}}{\pgfqpoint{12.086225in}{6.032130in}}{\pgfqpoint{12.086225in}{6.043180in}}%
\pgfpathcurveto{\pgfqpoint{12.086225in}{6.054230in}}{\pgfqpoint{12.081835in}{6.064829in}}{\pgfqpoint{12.074021in}{6.072643in}}%
\pgfpathcurveto{\pgfqpoint{12.066208in}{6.080456in}}{\pgfqpoint{12.055609in}{6.084847in}}{\pgfqpoint{12.044558in}{6.084847in}}%
\pgfpathcurveto{\pgfqpoint{12.033508in}{6.084847in}}{\pgfqpoint{12.022909in}{6.080456in}}{\pgfqpoint{12.015096in}{6.072643in}}%
\pgfpathcurveto{\pgfqpoint{12.007282in}{6.064829in}}{\pgfqpoint{12.002892in}{6.054230in}}{\pgfqpoint{12.002892in}{6.043180in}}%
\pgfpathcurveto{\pgfqpoint{12.002892in}{6.032130in}}{\pgfqpoint{12.007282in}{6.021531in}}{\pgfqpoint{12.015096in}{6.013717in}}%
\pgfpathcurveto{\pgfqpoint{12.022909in}{6.005903in}}{\pgfqpoint{12.033508in}{6.001513in}}{\pgfqpoint{12.044558in}{6.001513in}}%
\pgfpathclose%
\pgfusepath{stroke,fill}%
\end{pgfscope}%
\begin{pgfscope}%
\pgfpathrectangle{\pgfqpoint{7.640588in}{4.121437in}}{\pgfqpoint{5.699255in}{2.685432in}}%
\pgfusepath{clip}%
\pgfsetbuttcap%
\pgfsetroundjoin%
\definecolor{currentfill}{rgb}{0.000000,0.000000,0.000000}%
\pgfsetfillcolor{currentfill}%
\pgfsetlinewidth{1.003750pt}%
\definecolor{currentstroke}{rgb}{0.000000,0.000000,0.000000}%
\pgfsetstrokecolor{currentstroke}%
\pgfsetdash{}{0pt}%
\pgfpathmoveto{\pgfqpoint{12.271233in}{5.876318in}}%
\pgfpathcurveto{\pgfqpoint{12.282283in}{5.876318in}}{\pgfqpoint{12.292883in}{5.880709in}}{\pgfqpoint{12.300696in}{5.888522in}}%
\pgfpathcurveto{\pgfqpoint{12.308510in}{5.896336in}}{\pgfqpoint{12.312900in}{5.906935in}}{\pgfqpoint{12.312900in}{5.917985in}}%
\pgfpathcurveto{\pgfqpoint{12.312900in}{5.929035in}}{\pgfqpoint{12.308510in}{5.939634in}}{\pgfqpoint{12.300696in}{5.947448in}}%
\pgfpathcurveto{\pgfqpoint{12.292883in}{5.955261in}}{\pgfqpoint{12.282283in}{5.959652in}}{\pgfqpoint{12.271233in}{5.959652in}}%
\pgfpathcurveto{\pgfqpoint{12.260183in}{5.959652in}}{\pgfqpoint{12.249584in}{5.955261in}}{\pgfqpoint{12.241771in}{5.947448in}}%
\pgfpathcurveto{\pgfqpoint{12.233957in}{5.939634in}}{\pgfqpoint{12.229567in}{5.929035in}}{\pgfqpoint{12.229567in}{5.917985in}}%
\pgfpathcurveto{\pgfqpoint{12.229567in}{5.906935in}}{\pgfqpoint{12.233957in}{5.896336in}}{\pgfqpoint{12.241771in}{5.888522in}}%
\pgfpathcurveto{\pgfqpoint{12.249584in}{5.880709in}}{\pgfqpoint{12.260183in}{5.876318in}}{\pgfqpoint{12.271233in}{5.876318in}}%
\pgfpathclose%
\pgfusepath{stroke,fill}%
\end{pgfscope}%
\begin{pgfscope}%
\pgfpathrectangle{\pgfqpoint{7.640588in}{4.121437in}}{\pgfqpoint{5.699255in}{2.685432in}}%
\pgfusepath{clip}%
\pgfsetbuttcap%
\pgfsetroundjoin%
\definecolor{currentfill}{rgb}{0.000000,0.000000,0.000000}%
\pgfsetfillcolor{currentfill}%
\pgfsetlinewidth{1.003750pt}%
\definecolor{currentstroke}{rgb}{0.000000,0.000000,0.000000}%
\pgfsetstrokecolor{currentstroke}%
\pgfsetdash{}{0pt}%
\pgfpathmoveto{\pgfqpoint{12.595055in}{5.923266in}}%
\pgfpathcurveto{\pgfqpoint{12.606105in}{5.923266in}}{\pgfqpoint{12.616704in}{5.927657in}}{\pgfqpoint{12.624517in}{5.935470in}}%
\pgfpathcurveto{\pgfqpoint{12.632331in}{5.943284in}}{\pgfqpoint{12.636721in}{5.953883in}}{\pgfqpoint{12.636721in}{5.964933in}}%
\pgfpathcurveto{\pgfqpoint{12.636721in}{5.975983in}}{\pgfqpoint{12.632331in}{5.986582in}}{\pgfqpoint{12.624517in}{5.994396in}}%
\pgfpathcurveto{\pgfqpoint{12.616704in}{6.002209in}}{\pgfqpoint{12.606105in}{6.006600in}}{\pgfqpoint{12.595055in}{6.006600in}}%
\pgfpathcurveto{\pgfqpoint{12.584005in}{6.006600in}}{\pgfqpoint{12.573406in}{6.002209in}}{\pgfqpoint{12.565592in}{5.994396in}}%
\pgfpathcurveto{\pgfqpoint{12.557778in}{5.986582in}}{\pgfqpoint{12.553388in}{5.975983in}}{\pgfqpoint{12.553388in}{5.964933in}}%
\pgfpathcurveto{\pgfqpoint{12.553388in}{5.953883in}}{\pgfqpoint{12.557778in}{5.943284in}}{\pgfqpoint{12.565592in}{5.935470in}}%
\pgfpathcurveto{\pgfqpoint{12.573406in}{5.927657in}}{\pgfqpoint{12.584005in}{5.923266in}}{\pgfqpoint{12.595055in}{5.923266in}}%
\pgfpathclose%
\pgfusepath{stroke,fill}%
\end{pgfscope}%
\begin{pgfscope}%
\pgfpathrectangle{\pgfqpoint{7.640588in}{4.121437in}}{\pgfqpoint{5.699255in}{2.685432in}}%
\pgfusepath{clip}%
\pgfsetbuttcap%
\pgfsetroundjoin%
\definecolor{currentfill}{rgb}{0.000000,0.000000,0.000000}%
\pgfsetfillcolor{currentfill}%
\pgfsetlinewidth{1.003750pt}%
\definecolor{currentstroke}{rgb}{0.000000,0.000000,0.000000}%
\pgfsetstrokecolor{currentstroke}%
\pgfsetdash{}{0pt}%
\pgfpathmoveto{\pgfqpoint{12.400762in}{5.735474in}}%
\pgfpathcurveto{\pgfqpoint{12.411812in}{5.735474in}}{\pgfqpoint{12.422411in}{5.739864in}}{\pgfqpoint{12.430225in}{5.747678in}}%
\pgfpathcurveto{\pgfqpoint{12.438038in}{5.755491in}}{\pgfqpoint{12.442429in}{5.766090in}}{\pgfqpoint{12.442429in}{5.777141in}}%
\pgfpathcurveto{\pgfqpoint{12.442429in}{5.788191in}}{\pgfqpoint{12.438038in}{5.798790in}}{\pgfqpoint{12.430225in}{5.806603in}}%
\pgfpathcurveto{\pgfqpoint{12.422411in}{5.814417in}}{\pgfqpoint{12.411812in}{5.818807in}}{\pgfqpoint{12.400762in}{5.818807in}}%
\pgfpathcurveto{\pgfqpoint{12.389712in}{5.818807in}}{\pgfqpoint{12.379113in}{5.814417in}}{\pgfqpoint{12.371299in}{5.806603in}}%
\pgfpathcurveto{\pgfqpoint{12.363485in}{5.798790in}}{\pgfqpoint{12.359095in}{5.788191in}}{\pgfqpoint{12.359095in}{5.777141in}}%
\pgfpathcurveto{\pgfqpoint{12.359095in}{5.766090in}}{\pgfqpoint{12.363485in}{5.755491in}}{\pgfqpoint{12.371299in}{5.747678in}}%
\pgfpathcurveto{\pgfqpoint{12.379113in}{5.739864in}}{\pgfqpoint{12.389712in}{5.735474in}}{\pgfqpoint{12.400762in}{5.735474in}}%
\pgfpathclose%
\pgfusepath{stroke,fill}%
\end{pgfscope}%
\begin{pgfscope}%
\pgfpathrectangle{\pgfqpoint{7.640588in}{4.121437in}}{\pgfqpoint{5.699255in}{2.685432in}}%
\pgfusepath{clip}%
\pgfsetbuttcap%
\pgfsetroundjoin%
\definecolor{currentfill}{rgb}{0.000000,0.000000,0.000000}%
\pgfsetfillcolor{currentfill}%
\pgfsetlinewidth{1.003750pt}%
\definecolor{currentstroke}{rgb}{0.000000,0.000000,0.000000}%
\pgfsetstrokecolor{currentstroke}%
\pgfsetdash{}{0pt}%
\pgfpathmoveto{\pgfqpoint{12.692201in}{5.406837in}}%
\pgfpathcurveto{\pgfqpoint{12.703251in}{5.406837in}}{\pgfqpoint{12.713850in}{5.411227in}}{\pgfqpoint{12.721664in}{5.419041in}}%
\pgfpathcurveto{\pgfqpoint{12.729477in}{5.426855in}}{\pgfqpoint{12.733868in}{5.437454in}}{\pgfqpoint{12.733868in}{5.448504in}}%
\pgfpathcurveto{\pgfqpoint{12.733868in}{5.459554in}}{\pgfqpoint{12.729477in}{5.470153in}}{\pgfqpoint{12.721664in}{5.477967in}}%
\pgfpathcurveto{\pgfqpoint{12.713850in}{5.485780in}}{\pgfqpoint{12.703251in}{5.490170in}}{\pgfqpoint{12.692201in}{5.490170in}}%
\pgfpathcurveto{\pgfqpoint{12.681151in}{5.490170in}}{\pgfqpoint{12.670552in}{5.485780in}}{\pgfqpoint{12.662738in}{5.477967in}}%
\pgfpathcurveto{\pgfqpoint{12.654925in}{5.470153in}}{\pgfqpoint{12.650534in}{5.459554in}}{\pgfqpoint{12.650534in}{5.448504in}}%
\pgfpathcurveto{\pgfqpoint{12.650534in}{5.437454in}}{\pgfqpoint{12.654925in}{5.426855in}}{\pgfqpoint{12.662738in}{5.419041in}}%
\pgfpathcurveto{\pgfqpoint{12.670552in}{5.411227in}}{\pgfqpoint{12.681151in}{5.406837in}}{\pgfqpoint{12.692201in}{5.406837in}}%
\pgfpathclose%
\pgfusepath{stroke,fill}%
\end{pgfscope}%
\begin{pgfscope}%
\pgfpathrectangle{\pgfqpoint{7.640588in}{4.121437in}}{\pgfqpoint{5.699255in}{2.685432in}}%
\pgfusepath{clip}%
\pgfsetbuttcap%
\pgfsetroundjoin%
\definecolor{currentfill}{rgb}{0.000000,0.000000,0.000000}%
\pgfsetfillcolor{currentfill}%
\pgfsetlinewidth{1.003750pt}%
\definecolor{currentstroke}{rgb}{0.000000,0.000000,0.000000}%
\pgfsetstrokecolor{currentstroke}%
\pgfsetdash{}{0pt}%
\pgfpathmoveto{\pgfqpoint{12.465526in}{4.968655in}}%
\pgfpathcurveto{\pgfqpoint{12.476576in}{4.968655in}}{\pgfqpoint{12.487175in}{4.973045in}}{\pgfqpoint{12.494989in}{4.980859in}}%
\pgfpathcurveto{\pgfqpoint{12.502803in}{4.988672in}}{\pgfqpoint{12.507193in}{4.999271in}}{\pgfqpoint{12.507193in}{5.010321in}}%
\pgfpathcurveto{\pgfqpoint{12.507193in}{5.021372in}}{\pgfqpoint{12.502803in}{5.031971in}}{\pgfqpoint{12.494989in}{5.039784in}}%
\pgfpathcurveto{\pgfqpoint{12.487175in}{5.047598in}}{\pgfqpoint{12.476576in}{5.051988in}}{\pgfqpoint{12.465526in}{5.051988in}}%
\pgfpathcurveto{\pgfqpoint{12.454476in}{5.051988in}}{\pgfqpoint{12.443877in}{5.047598in}}{\pgfqpoint{12.436063in}{5.039784in}}%
\pgfpathcurveto{\pgfqpoint{12.428250in}{5.031971in}}{\pgfqpoint{12.423859in}{5.021372in}}{\pgfqpoint{12.423859in}{5.010321in}}%
\pgfpathcurveto{\pgfqpoint{12.423859in}{4.999271in}}{\pgfqpoint{12.428250in}{4.988672in}}{\pgfqpoint{12.436063in}{4.980859in}}%
\pgfpathcurveto{\pgfqpoint{12.443877in}{4.973045in}}{\pgfqpoint{12.454476in}{4.968655in}}{\pgfqpoint{12.465526in}{4.968655in}}%
\pgfpathclose%
\pgfusepath{stroke,fill}%
\end{pgfscope}%
\begin{pgfscope}%
\pgfpathrectangle{\pgfqpoint{7.640588in}{4.121437in}}{\pgfqpoint{5.699255in}{2.685432in}}%
\pgfusepath{clip}%
\pgfsetbuttcap%
\pgfsetroundjoin%
\definecolor{currentfill}{rgb}{0.000000,0.000000,0.000000}%
\pgfsetfillcolor{currentfill}%
\pgfsetlinewidth{1.003750pt}%
\definecolor{currentstroke}{rgb}{0.000000,0.000000,0.000000}%
\pgfsetstrokecolor{currentstroke}%
\pgfsetdash{}{0pt}%
\pgfpathmoveto{\pgfqpoint{12.854112in}{4.796512in}}%
\pgfpathcurveto{\pgfqpoint{12.865162in}{4.796512in}}{\pgfqpoint{12.875761in}{4.800902in}}{\pgfqpoint{12.883575in}{4.808716in}}%
\pgfpathcurveto{\pgfqpoint{12.891388in}{4.816529in}}{\pgfqpoint{12.895778in}{4.827128in}}{\pgfqpoint{12.895778in}{4.838178in}}%
\pgfpathcurveto{\pgfqpoint{12.895778in}{4.849228in}}{\pgfqpoint{12.891388in}{4.859827in}}{\pgfqpoint{12.883575in}{4.867641in}}%
\pgfpathcurveto{\pgfqpoint{12.875761in}{4.875455in}}{\pgfqpoint{12.865162in}{4.879845in}}{\pgfqpoint{12.854112in}{4.879845in}}%
\pgfpathcurveto{\pgfqpoint{12.843062in}{4.879845in}}{\pgfqpoint{12.832463in}{4.875455in}}{\pgfqpoint{12.824649in}{4.867641in}}%
\pgfpathcurveto{\pgfqpoint{12.816835in}{4.859827in}}{\pgfqpoint{12.812445in}{4.849228in}}{\pgfqpoint{12.812445in}{4.838178in}}%
\pgfpathcurveto{\pgfqpoint{12.812445in}{4.827128in}}{\pgfqpoint{12.816835in}{4.816529in}}{\pgfqpoint{12.824649in}{4.808716in}}%
\pgfpathcurveto{\pgfqpoint{12.832463in}{4.800902in}}{\pgfqpoint{12.843062in}{4.796512in}}{\pgfqpoint{12.854112in}{4.796512in}}%
\pgfpathclose%
\pgfusepath{stroke,fill}%
\end{pgfscope}%
\begin{pgfscope}%
\pgfpathrectangle{\pgfqpoint{7.640588in}{4.121437in}}{\pgfqpoint{5.699255in}{2.685432in}}%
\pgfusepath{clip}%
\pgfsetbuttcap%
\pgfsetroundjoin%
\definecolor{currentfill}{rgb}{0.000000,0.000000,0.000000}%
\pgfsetfillcolor{currentfill}%
\pgfsetlinewidth{1.003750pt}%
\definecolor{currentstroke}{rgb}{0.000000,0.000000,0.000000}%
\pgfsetstrokecolor{currentstroke}%
\pgfsetdash{}{0pt}%
\pgfpathmoveto{\pgfqpoint{12.141705in}{4.671317in}}%
\pgfpathcurveto{\pgfqpoint{12.152755in}{4.671317in}}{\pgfqpoint{12.163354in}{4.675707in}}{\pgfqpoint{12.171168in}{4.683521in}}%
\pgfpathcurveto{\pgfqpoint{12.178981in}{4.691334in}}{\pgfqpoint{12.183372in}{4.701933in}}{\pgfqpoint{12.183372in}{4.712983in}}%
\pgfpathcurveto{\pgfqpoint{12.183372in}{4.724033in}}{\pgfqpoint{12.178981in}{4.734633in}}{\pgfqpoint{12.171168in}{4.742446in}}%
\pgfpathcurveto{\pgfqpoint{12.163354in}{4.750260in}}{\pgfqpoint{12.152755in}{4.754650in}}{\pgfqpoint{12.141705in}{4.754650in}}%
\pgfpathcurveto{\pgfqpoint{12.130655in}{4.754650in}}{\pgfqpoint{12.120056in}{4.750260in}}{\pgfqpoint{12.112242in}{4.742446in}}%
\pgfpathcurveto{\pgfqpoint{12.104428in}{4.734633in}}{\pgfqpoint{12.100038in}{4.724033in}}{\pgfqpoint{12.100038in}{4.712983in}}%
\pgfpathcurveto{\pgfqpoint{12.100038in}{4.701933in}}{\pgfqpoint{12.104428in}{4.691334in}}{\pgfqpoint{12.112242in}{4.683521in}}%
\pgfpathcurveto{\pgfqpoint{12.120056in}{4.675707in}}{\pgfqpoint{12.130655in}{4.671317in}}{\pgfqpoint{12.141705in}{4.671317in}}%
\pgfpathclose%
\pgfusepath{stroke,fill}%
\end{pgfscope}%
\begin{pgfscope}%
\pgfpathrectangle{\pgfqpoint{7.640588in}{4.121437in}}{\pgfqpoint{5.699255in}{2.685432in}}%
\pgfusepath{clip}%
\pgfsetbuttcap%
\pgfsetroundjoin%
\definecolor{currentfill}{rgb}{0.000000,0.000000,0.000000}%
\pgfsetfillcolor{currentfill}%
\pgfsetlinewidth{1.003750pt}%
\definecolor{currentstroke}{rgb}{0.000000,0.000000,0.000000}%
\pgfsetstrokecolor{currentstroke}%
\pgfsetdash{}{0pt}%
\pgfpathmoveto{\pgfqpoint{11.947412in}{4.702615in}}%
\pgfpathcurveto{\pgfqpoint{11.958462in}{4.702615in}}{\pgfqpoint{11.969061in}{4.707006in}}{\pgfqpoint{11.976875in}{4.714819in}}%
\pgfpathcurveto{\pgfqpoint{11.984688in}{4.722633in}}{\pgfqpoint{11.989079in}{4.733232in}}{\pgfqpoint{11.989079in}{4.744282in}}%
\pgfpathcurveto{\pgfqpoint{11.989079in}{4.755332in}}{\pgfqpoint{11.984688in}{4.765931in}}{\pgfqpoint{11.976875in}{4.773745in}}%
\pgfpathcurveto{\pgfqpoint{11.969061in}{4.781558in}}{\pgfqpoint{11.958462in}{4.785949in}}{\pgfqpoint{11.947412in}{4.785949in}}%
\pgfpathcurveto{\pgfqpoint{11.936362in}{4.785949in}}{\pgfqpoint{11.925763in}{4.781558in}}{\pgfqpoint{11.917949in}{4.773745in}}%
\pgfpathcurveto{\pgfqpoint{11.910136in}{4.765931in}}{\pgfqpoint{11.905745in}{4.755332in}}{\pgfqpoint{11.905745in}{4.744282in}}%
\pgfpathcurveto{\pgfqpoint{11.905745in}{4.733232in}}{\pgfqpoint{11.910136in}{4.722633in}}{\pgfqpoint{11.917949in}{4.714819in}}%
\pgfpathcurveto{\pgfqpoint{11.925763in}{4.707006in}}{\pgfqpoint{11.936362in}{4.702615in}}{\pgfqpoint{11.947412in}{4.702615in}}%
\pgfpathclose%
\pgfusepath{stroke,fill}%
\end{pgfscope}%
\begin{pgfscope}%
\pgfpathrectangle{\pgfqpoint{7.640588in}{4.121437in}}{\pgfqpoint{5.699255in}{2.685432in}}%
\pgfusepath{clip}%
\pgfsetbuttcap%
\pgfsetroundjoin%
\definecolor{currentfill}{rgb}{0.000000,0.000000,0.000000}%
\pgfsetfillcolor{currentfill}%
\pgfsetlinewidth{1.003750pt}%
\definecolor{currentstroke}{rgb}{0.000000,0.000000,0.000000}%
\pgfsetstrokecolor{currentstroke}%
\pgfsetdash{}{0pt}%
\pgfpathmoveto{\pgfqpoint{11.979794in}{4.201836in}}%
\pgfpathcurveto{\pgfqpoint{11.990844in}{4.201836in}}{\pgfqpoint{12.001443in}{4.206226in}}{\pgfqpoint{12.009257in}{4.214039in}}%
\pgfpathcurveto{\pgfqpoint{12.017071in}{4.221853in}}{\pgfqpoint{12.021461in}{4.232452in}}{\pgfqpoint{12.021461in}{4.243502in}}%
\pgfpathcurveto{\pgfqpoint{12.021461in}{4.254552in}}{\pgfqpoint{12.017071in}{4.265151in}}{\pgfqpoint{12.009257in}{4.272965in}}%
\pgfpathcurveto{\pgfqpoint{12.001443in}{4.280779in}}{\pgfqpoint{11.990844in}{4.285169in}}{\pgfqpoint{11.979794in}{4.285169in}}%
\pgfpathcurveto{\pgfqpoint{11.968744in}{4.285169in}}{\pgfqpoint{11.958145in}{4.280779in}}{\pgfqpoint{11.950331in}{4.272965in}}%
\pgfpathcurveto{\pgfqpoint{11.942518in}{4.265151in}}{\pgfqpoint{11.938128in}{4.254552in}}{\pgfqpoint{11.938128in}{4.243502in}}%
\pgfpathcurveto{\pgfqpoint{11.938128in}{4.232452in}}{\pgfqpoint{11.942518in}{4.221853in}}{\pgfqpoint{11.950331in}{4.214039in}}%
\pgfpathcurveto{\pgfqpoint{11.958145in}{4.206226in}}{\pgfqpoint{11.968744in}{4.201836in}}{\pgfqpoint{11.979794in}{4.201836in}}%
\pgfpathclose%
\pgfusepath{stroke,fill}%
\end{pgfscope}%
\begin{pgfscope}%
\pgfpathrectangle{\pgfqpoint{7.640588in}{4.121437in}}{\pgfqpoint{5.699255in}{2.685432in}}%
\pgfusepath{clip}%
\pgfsetbuttcap%
\pgfsetroundjoin%
\definecolor{currentfill}{rgb}{0.000000,0.000000,0.000000}%
\pgfsetfillcolor{currentfill}%
\pgfsetlinewidth{1.003750pt}%
\definecolor{currentstroke}{rgb}{0.000000,0.000000,0.000000}%
\pgfsetstrokecolor{currentstroke}%
\pgfsetdash{}{0pt}%
\pgfpathmoveto{\pgfqpoint{11.461680in}{4.201836in}}%
\pgfpathcurveto{\pgfqpoint{11.472730in}{4.201836in}}{\pgfqpoint{11.483329in}{4.206226in}}{\pgfqpoint{11.491143in}{4.214039in}}%
\pgfpathcurveto{\pgfqpoint{11.498956in}{4.221853in}}{\pgfqpoint{11.503347in}{4.232452in}}{\pgfqpoint{11.503347in}{4.243502in}}%
\pgfpathcurveto{\pgfqpoint{11.503347in}{4.254552in}}{\pgfqpoint{11.498956in}{4.265151in}}{\pgfqpoint{11.491143in}{4.272965in}}%
\pgfpathcurveto{\pgfqpoint{11.483329in}{4.280779in}}{\pgfqpoint{11.472730in}{4.285169in}}{\pgfqpoint{11.461680in}{4.285169in}}%
\pgfpathcurveto{\pgfqpoint{11.450630in}{4.285169in}}{\pgfqpoint{11.440031in}{4.280779in}}{\pgfqpoint{11.432217in}{4.272965in}}%
\pgfpathcurveto{\pgfqpoint{11.424404in}{4.265151in}}{\pgfqpoint{11.420013in}{4.254552in}}{\pgfqpoint{11.420013in}{4.243502in}}%
\pgfpathcurveto{\pgfqpoint{11.420013in}{4.232452in}}{\pgfqpoint{11.424404in}{4.221853in}}{\pgfqpoint{11.432217in}{4.214039in}}%
\pgfpathcurveto{\pgfqpoint{11.440031in}{4.206226in}}{\pgfqpoint{11.450630in}{4.201836in}}{\pgfqpoint{11.461680in}{4.201836in}}%
\pgfpathclose%
\pgfusepath{stroke,fill}%
\end{pgfscope}%
\begin{pgfscope}%
\pgfpathrectangle{\pgfqpoint{7.640588in}{4.121437in}}{\pgfqpoint{5.699255in}{2.685432in}}%
\pgfusepath{clip}%
\pgfsetbuttcap%
\pgfsetroundjoin%
\definecolor{currentfill}{rgb}{0.000000,0.000000,0.000000}%
\pgfsetfillcolor{currentfill}%
\pgfsetlinewidth{1.003750pt}%
\definecolor{currentstroke}{rgb}{0.000000,0.000000,0.000000}%
\pgfsetstrokecolor{currentstroke}%
\pgfsetdash{}{0pt}%
\pgfpathmoveto{\pgfqpoint{10.619745in}{4.311381in}}%
\pgfpathcurveto{\pgfqpoint{10.630795in}{4.311381in}}{\pgfqpoint{10.641394in}{4.315771in}}{\pgfqpoint{10.649207in}{4.323585in}}%
\pgfpathcurveto{\pgfqpoint{10.657021in}{4.331399in}}{\pgfqpoint{10.661411in}{4.341998in}}{\pgfqpoint{10.661411in}{4.353048in}}%
\pgfpathcurveto{\pgfqpoint{10.661411in}{4.364098in}}{\pgfqpoint{10.657021in}{4.374697in}}{\pgfqpoint{10.649207in}{4.382511in}}%
\pgfpathcurveto{\pgfqpoint{10.641394in}{4.390324in}}{\pgfqpoint{10.630795in}{4.394714in}}{\pgfqpoint{10.619745in}{4.394714in}}%
\pgfpathcurveto{\pgfqpoint{10.608694in}{4.394714in}}{\pgfqpoint{10.598095in}{4.390324in}}{\pgfqpoint{10.590282in}{4.382511in}}%
\pgfpathcurveto{\pgfqpoint{10.582468in}{4.374697in}}{\pgfqpoint{10.578078in}{4.364098in}}{\pgfqpoint{10.578078in}{4.353048in}}%
\pgfpathcurveto{\pgfqpoint{10.578078in}{4.341998in}}{\pgfqpoint{10.582468in}{4.331399in}}{\pgfqpoint{10.590282in}{4.323585in}}%
\pgfpathcurveto{\pgfqpoint{10.598095in}{4.315771in}}{\pgfqpoint{10.608694in}{4.311381in}}{\pgfqpoint{10.619745in}{4.311381in}}%
\pgfpathclose%
\pgfusepath{stroke,fill}%
\end{pgfscope}%
\begin{pgfscope}%
\pgfpathrectangle{\pgfqpoint{7.640588in}{4.121437in}}{\pgfqpoint{5.699255in}{2.685432in}}%
\pgfusepath{clip}%
\pgfsetbuttcap%
\pgfsetroundjoin%
\definecolor{currentfill}{rgb}{0.000000,0.000000,0.000000}%
\pgfsetfillcolor{currentfill}%
\pgfsetlinewidth{1.003750pt}%
\definecolor{currentstroke}{rgb}{0.000000,0.000000,0.000000}%
\pgfsetstrokecolor{currentstroke}%
\pgfsetdash{}{0pt}%
\pgfpathmoveto{\pgfqpoint{10.652127in}{4.342680in}}%
\pgfpathcurveto{\pgfqpoint{10.663177in}{4.342680in}}{\pgfqpoint{10.673776in}{4.347070in}}{\pgfqpoint{10.681590in}{4.354884in}}%
\pgfpathcurveto{\pgfqpoint{10.689403in}{4.362697in}}{\pgfqpoint{10.693793in}{4.373296in}}{\pgfqpoint{10.693793in}{4.384347in}}%
\pgfpathcurveto{\pgfqpoint{10.693793in}{4.395397in}}{\pgfqpoint{10.689403in}{4.405996in}}{\pgfqpoint{10.681590in}{4.413809in}}%
\pgfpathcurveto{\pgfqpoint{10.673776in}{4.421623in}}{\pgfqpoint{10.663177in}{4.426013in}}{\pgfqpoint{10.652127in}{4.426013in}}%
\pgfpathcurveto{\pgfqpoint{10.641077in}{4.426013in}}{\pgfqpoint{10.630478in}{4.421623in}}{\pgfqpoint{10.622664in}{4.413809in}}%
\pgfpathcurveto{\pgfqpoint{10.614850in}{4.405996in}}{\pgfqpoint{10.610460in}{4.395397in}}{\pgfqpoint{10.610460in}{4.384347in}}%
\pgfpathcurveto{\pgfqpoint{10.610460in}{4.373296in}}{\pgfqpoint{10.614850in}{4.362697in}}{\pgfqpoint{10.622664in}{4.354884in}}%
\pgfpathcurveto{\pgfqpoint{10.630478in}{4.347070in}}{\pgfqpoint{10.641077in}{4.342680in}}{\pgfqpoint{10.652127in}{4.342680in}}%
\pgfpathclose%
\pgfusepath{stroke,fill}%
\end{pgfscope}%
\begin{pgfscope}%
\pgfpathrectangle{\pgfqpoint{7.640588in}{4.121437in}}{\pgfqpoint{5.699255in}{2.685432in}}%
\pgfusepath{clip}%
\pgfsetbuttcap%
\pgfsetroundjoin%
\definecolor{currentfill}{rgb}{0.000000,0.000000,0.000000}%
\pgfsetfillcolor{currentfill}%
\pgfsetlinewidth{1.003750pt}%
\definecolor{currentstroke}{rgb}{0.000000,0.000000,0.000000}%
\pgfsetstrokecolor{currentstroke}%
\pgfsetdash{}{0pt}%
\pgfpathmoveto{\pgfqpoint{11.235005in}{4.405277in}}%
\pgfpathcurveto{\pgfqpoint{11.246055in}{4.405277in}}{\pgfqpoint{11.256654in}{4.409668in}}{\pgfqpoint{11.264468in}{4.417481in}}%
\pgfpathcurveto{\pgfqpoint{11.272282in}{4.425295in}}{\pgfqpoint{11.276672in}{4.435894in}}{\pgfqpoint{11.276672in}{4.446944in}}%
\pgfpathcurveto{\pgfqpoint{11.276672in}{4.457994in}}{\pgfqpoint{11.272282in}{4.468593in}}{\pgfqpoint{11.264468in}{4.476407in}}%
\pgfpathcurveto{\pgfqpoint{11.256654in}{4.484220in}}{\pgfqpoint{11.246055in}{4.488611in}}{\pgfqpoint{11.235005in}{4.488611in}}%
\pgfpathcurveto{\pgfqpoint{11.223955in}{4.488611in}}{\pgfqpoint{11.213356in}{4.484220in}}{\pgfqpoint{11.205542in}{4.476407in}}%
\pgfpathcurveto{\pgfqpoint{11.197729in}{4.468593in}}{\pgfqpoint{11.193338in}{4.457994in}}{\pgfqpoint{11.193338in}{4.446944in}}%
\pgfpathcurveto{\pgfqpoint{11.193338in}{4.435894in}}{\pgfqpoint{11.197729in}{4.425295in}}{\pgfqpoint{11.205542in}{4.417481in}}%
\pgfpathcurveto{\pgfqpoint{11.213356in}{4.409668in}}{\pgfqpoint{11.223955in}{4.405277in}}{\pgfqpoint{11.235005in}{4.405277in}}%
\pgfpathclose%
\pgfusepath{stroke,fill}%
\end{pgfscope}%
\begin{pgfscope}%
\pgfpathrectangle{\pgfqpoint{7.640588in}{4.121437in}}{\pgfqpoint{5.699255in}{2.685432in}}%
\pgfusepath{clip}%
\pgfsetbuttcap%
\pgfsetroundjoin%
\definecolor{currentfill}{rgb}{0.000000,0.000000,0.000000}%
\pgfsetfillcolor{currentfill}%
\pgfsetlinewidth{1.003750pt}%
\definecolor{currentstroke}{rgb}{0.000000,0.000000,0.000000}%
\pgfsetstrokecolor{currentstroke}%
\pgfsetdash{}{0pt}%
\pgfpathmoveto{\pgfqpoint{11.429298in}{4.733914in}}%
\pgfpathcurveto{\pgfqpoint{11.440348in}{4.733914in}}{\pgfqpoint{11.450947in}{4.738304in}}{\pgfqpoint{11.458761in}{4.746118in}}%
\pgfpathcurveto{\pgfqpoint{11.466574in}{4.753932in}}{\pgfqpoint{11.470965in}{4.764531in}}{\pgfqpoint{11.470965in}{4.775581in}}%
\pgfpathcurveto{\pgfqpoint{11.470965in}{4.786631in}}{\pgfqpoint{11.466574in}{4.797230in}}{\pgfqpoint{11.458761in}{4.805044in}}%
\pgfpathcurveto{\pgfqpoint{11.450947in}{4.812857in}}{\pgfqpoint{11.440348in}{4.817247in}}{\pgfqpoint{11.429298in}{4.817247in}}%
\pgfpathcurveto{\pgfqpoint{11.418248in}{4.817247in}}{\pgfqpoint{11.407649in}{4.812857in}}{\pgfqpoint{11.399835in}{4.805044in}}%
\pgfpathcurveto{\pgfqpoint{11.392022in}{4.797230in}}{\pgfqpoint{11.387631in}{4.786631in}}{\pgfqpoint{11.387631in}{4.775581in}}%
\pgfpathcurveto{\pgfqpoint{11.387631in}{4.764531in}}{\pgfqpoint{11.392022in}{4.753932in}}{\pgfqpoint{11.399835in}{4.746118in}}%
\pgfpathcurveto{\pgfqpoint{11.407649in}{4.738304in}}{\pgfqpoint{11.418248in}{4.733914in}}{\pgfqpoint{11.429298in}{4.733914in}}%
\pgfpathclose%
\pgfusepath{stroke,fill}%
\end{pgfscope}%
\begin{pgfscope}%
\pgfpathrectangle{\pgfqpoint{7.640588in}{4.121437in}}{\pgfqpoint{5.699255in}{2.685432in}}%
\pgfusepath{clip}%
\pgfsetbuttcap%
\pgfsetroundjoin%
\definecolor{currentfill}{rgb}{0.000000,0.000000,0.000000}%
\pgfsetfillcolor{currentfill}%
\pgfsetlinewidth{1.003750pt}%
\definecolor{currentstroke}{rgb}{0.000000,0.000000,0.000000}%
\pgfsetstrokecolor{currentstroke}%
\pgfsetdash{}{0pt}%
\pgfpathmoveto{\pgfqpoint{11.558826in}{4.671317in}}%
\pgfpathcurveto{\pgfqpoint{11.569877in}{4.671317in}}{\pgfqpoint{11.580476in}{4.675707in}}{\pgfqpoint{11.588289in}{4.683521in}}%
\pgfpathcurveto{\pgfqpoint{11.596103in}{4.691334in}}{\pgfqpoint{11.600493in}{4.701933in}}{\pgfqpoint{11.600493in}{4.712983in}}%
\pgfpathcurveto{\pgfqpoint{11.600493in}{4.724033in}}{\pgfqpoint{11.596103in}{4.734633in}}{\pgfqpoint{11.588289in}{4.742446in}}%
\pgfpathcurveto{\pgfqpoint{11.580476in}{4.750260in}}{\pgfqpoint{11.569877in}{4.754650in}}{\pgfqpoint{11.558826in}{4.754650in}}%
\pgfpathcurveto{\pgfqpoint{11.547776in}{4.754650in}}{\pgfqpoint{11.537177in}{4.750260in}}{\pgfqpoint{11.529364in}{4.742446in}}%
\pgfpathcurveto{\pgfqpoint{11.521550in}{4.734633in}}{\pgfqpoint{11.517160in}{4.724033in}}{\pgfqpoint{11.517160in}{4.712983in}}%
\pgfpathcurveto{\pgfqpoint{11.517160in}{4.701933in}}{\pgfqpoint{11.521550in}{4.691334in}}{\pgfqpoint{11.529364in}{4.683521in}}%
\pgfpathcurveto{\pgfqpoint{11.537177in}{4.675707in}}{\pgfqpoint{11.547776in}{4.671317in}}{\pgfqpoint{11.558826in}{4.671317in}}%
\pgfpathclose%
\pgfusepath{stroke,fill}%
\end{pgfscope}%
\begin{pgfscope}%
\pgfpathrectangle{\pgfqpoint{7.640588in}{4.121437in}}{\pgfqpoint{5.699255in}{2.685432in}}%
\pgfusepath{clip}%
\pgfsetbuttcap%
\pgfsetroundjoin%
\definecolor{currentfill}{rgb}{0.000000,0.000000,0.000000}%
\pgfsetfillcolor{currentfill}%
\pgfsetlinewidth{1.003750pt}%
\definecolor{currentstroke}{rgb}{0.000000,0.000000,0.000000}%
\pgfsetstrokecolor{currentstroke}%
\pgfsetdash{}{0pt}%
\pgfpathmoveto{\pgfqpoint{11.947412in}{4.780862in}}%
\pgfpathcurveto{\pgfqpoint{11.958462in}{4.780862in}}{\pgfqpoint{11.969061in}{4.785253in}}{\pgfqpoint{11.976875in}{4.793066in}}%
\pgfpathcurveto{\pgfqpoint{11.984688in}{4.800880in}}{\pgfqpoint{11.989079in}{4.811479in}}{\pgfqpoint{11.989079in}{4.822529in}}%
\pgfpathcurveto{\pgfqpoint{11.989079in}{4.833579in}}{\pgfqpoint{11.984688in}{4.844178in}}{\pgfqpoint{11.976875in}{4.851992in}}%
\pgfpathcurveto{\pgfqpoint{11.969061in}{4.859805in}}{\pgfqpoint{11.958462in}{4.864196in}}{\pgfqpoint{11.947412in}{4.864196in}}%
\pgfpathcurveto{\pgfqpoint{11.936362in}{4.864196in}}{\pgfqpoint{11.925763in}{4.859805in}}{\pgfqpoint{11.917949in}{4.851992in}}%
\pgfpathcurveto{\pgfqpoint{11.910136in}{4.844178in}}{\pgfqpoint{11.905745in}{4.833579in}}{\pgfqpoint{11.905745in}{4.822529in}}%
\pgfpathcurveto{\pgfqpoint{11.905745in}{4.811479in}}{\pgfqpoint{11.910136in}{4.800880in}}{\pgfqpoint{11.917949in}{4.793066in}}%
\pgfpathcurveto{\pgfqpoint{11.925763in}{4.785253in}}{\pgfqpoint{11.936362in}{4.780862in}}{\pgfqpoint{11.947412in}{4.780862in}}%
\pgfpathclose%
\pgfusepath{stroke,fill}%
\end{pgfscope}%
\begin{pgfscope}%
\pgfpathrectangle{\pgfqpoint{7.640588in}{4.121437in}}{\pgfqpoint{5.699255in}{2.685432in}}%
\pgfusepath{clip}%
\pgfsetbuttcap%
\pgfsetroundjoin%
\definecolor{currentfill}{rgb}{0.000000,0.000000,0.000000}%
\pgfsetfillcolor{currentfill}%
\pgfsetlinewidth{1.003750pt}%
\definecolor{currentstroke}{rgb}{0.000000,0.000000,0.000000}%
\pgfsetstrokecolor{currentstroke}%
\pgfsetdash{}{0pt}%
\pgfpathmoveto{\pgfqpoint{12.044558in}{4.843460in}}%
\pgfpathcurveto{\pgfqpoint{12.055609in}{4.843460in}}{\pgfqpoint{12.066208in}{4.847850in}}{\pgfqpoint{12.074021in}{4.855664in}}%
\pgfpathcurveto{\pgfqpoint{12.081835in}{4.863477in}}{\pgfqpoint{12.086225in}{4.874076in}}{\pgfqpoint{12.086225in}{4.885126in}}%
\pgfpathcurveto{\pgfqpoint{12.086225in}{4.896177in}}{\pgfqpoint{12.081835in}{4.906776in}}{\pgfqpoint{12.074021in}{4.914589in}}%
\pgfpathcurveto{\pgfqpoint{12.066208in}{4.922403in}}{\pgfqpoint{12.055609in}{4.926793in}}{\pgfqpoint{12.044558in}{4.926793in}}%
\pgfpathcurveto{\pgfqpoint{12.033508in}{4.926793in}}{\pgfqpoint{12.022909in}{4.922403in}}{\pgfqpoint{12.015096in}{4.914589in}}%
\pgfpathcurveto{\pgfqpoint{12.007282in}{4.906776in}}{\pgfqpoint{12.002892in}{4.896177in}}{\pgfqpoint{12.002892in}{4.885126in}}%
\pgfpathcurveto{\pgfqpoint{12.002892in}{4.874076in}}{\pgfqpoint{12.007282in}{4.863477in}}{\pgfqpoint{12.015096in}{4.855664in}}%
\pgfpathcurveto{\pgfqpoint{12.022909in}{4.847850in}}{\pgfqpoint{12.033508in}{4.843460in}}{\pgfqpoint{12.044558in}{4.843460in}}%
\pgfpathclose%
\pgfusepath{stroke,fill}%
\end{pgfscope}%
\begin{pgfscope}%
\pgfpathrectangle{\pgfqpoint{7.640588in}{4.121437in}}{\pgfqpoint{5.699255in}{2.685432in}}%
\pgfusepath{clip}%
\pgfsetbuttcap%
\pgfsetroundjoin%
\definecolor{currentfill}{rgb}{0.000000,0.000000,0.000000}%
\pgfsetfillcolor{currentfill}%
\pgfsetlinewidth{1.003750pt}%
\definecolor{currentstroke}{rgb}{0.000000,0.000000,0.000000}%
\pgfsetstrokecolor{currentstroke}%
\pgfsetdash{}{0pt}%
\pgfpathmoveto{\pgfqpoint{12.012176in}{5.031252in}}%
\pgfpathcurveto{\pgfqpoint{12.023226in}{5.031252in}}{\pgfqpoint{12.033825in}{5.035642in}}{\pgfqpoint{12.041639in}{5.043456in}}%
\pgfpathcurveto{\pgfqpoint{12.049453in}{5.051270in}}{\pgfqpoint{12.053843in}{5.061869in}}{\pgfqpoint{12.053843in}{5.072919in}}%
\pgfpathcurveto{\pgfqpoint{12.053843in}{5.083969in}}{\pgfqpoint{12.049453in}{5.094568in}}{\pgfqpoint{12.041639in}{5.102382in}}%
\pgfpathcurveto{\pgfqpoint{12.033825in}{5.110195in}}{\pgfqpoint{12.023226in}{5.114586in}}{\pgfqpoint{12.012176in}{5.114586in}}%
\pgfpathcurveto{\pgfqpoint{12.001126in}{5.114586in}}{\pgfqpoint{11.990527in}{5.110195in}}{\pgfqpoint{11.982714in}{5.102382in}}%
\pgfpathcurveto{\pgfqpoint{11.974900in}{5.094568in}}{\pgfqpoint{11.970510in}{5.083969in}}{\pgfqpoint{11.970510in}{5.072919in}}%
\pgfpathcurveto{\pgfqpoint{11.970510in}{5.061869in}}{\pgfqpoint{11.974900in}{5.051270in}}{\pgfqpoint{11.982714in}{5.043456in}}%
\pgfpathcurveto{\pgfqpoint{11.990527in}{5.035642in}}{\pgfqpoint{12.001126in}{5.031252in}}{\pgfqpoint{12.012176in}{5.031252in}}%
\pgfpathclose%
\pgfusepath{stroke,fill}%
\end{pgfscope}%
\begin{pgfscope}%
\pgfpathrectangle{\pgfqpoint{7.640588in}{4.121437in}}{\pgfqpoint{5.699255in}{2.685432in}}%
\pgfusepath{clip}%
\pgfsetbuttcap%
\pgfsetroundjoin%
\definecolor{currentfill}{rgb}{0.000000,0.000000,0.000000}%
\pgfsetfillcolor{currentfill}%
\pgfsetlinewidth{1.003750pt}%
\definecolor{currentstroke}{rgb}{0.000000,0.000000,0.000000}%
\pgfsetstrokecolor{currentstroke}%
\pgfsetdash{}{0pt}%
\pgfpathmoveto{\pgfqpoint{11.817884in}{5.375538in}}%
\pgfpathcurveto{\pgfqpoint{11.828934in}{5.375538in}}{\pgfqpoint{11.839533in}{5.379929in}}{\pgfqpoint{11.847346in}{5.387742in}}%
\pgfpathcurveto{\pgfqpoint{11.855160in}{5.395556in}}{\pgfqpoint{11.859550in}{5.406155in}}{\pgfqpoint{11.859550in}{5.417205in}}%
\pgfpathcurveto{\pgfqpoint{11.859550in}{5.428255in}}{\pgfqpoint{11.855160in}{5.438854in}}{\pgfqpoint{11.847346in}{5.446668in}}%
\pgfpathcurveto{\pgfqpoint{11.839533in}{5.454481in}}{\pgfqpoint{11.828934in}{5.458872in}}{\pgfqpoint{11.817884in}{5.458872in}}%
\pgfpathcurveto{\pgfqpoint{11.806833in}{5.458872in}}{\pgfqpoint{11.796234in}{5.454481in}}{\pgfqpoint{11.788421in}{5.446668in}}%
\pgfpathcurveto{\pgfqpoint{11.780607in}{5.438854in}}{\pgfqpoint{11.776217in}{5.428255in}}{\pgfqpoint{11.776217in}{5.417205in}}%
\pgfpathcurveto{\pgfqpoint{11.776217in}{5.406155in}}{\pgfqpoint{11.780607in}{5.395556in}}{\pgfqpoint{11.788421in}{5.387742in}}%
\pgfpathcurveto{\pgfqpoint{11.796234in}{5.379929in}}{\pgfqpoint{11.806833in}{5.375538in}}{\pgfqpoint{11.817884in}{5.375538in}}%
\pgfpathclose%
\pgfusepath{stroke,fill}%
\end{pgfscope}%
\begin{pgfscope}%
\pgfpathrectangle{\pgfqpoint{7.640588in}{4.121437in}}{\pgfqpoint{5.699255in}{2.685432in}}%
\pgfusepath{clip}%
\pgfsetbuttcap%
\pgfsetroundjoin%
\definecolor{currentfill}{rgb}{0.000000,0.000000,0.000000}%
\pgfsetfillcolor{currentfill}%
\pgfsetlinewidth{1.003750pt}%
\definecolor{currentstroke}{rgb}{0.000000,0.000000,0.000000}%
\pgfsetstrokecolor{currentstroke}%
\pgfsetdash{}{0pt}%
\pgfpathmoveto{\pgfqpoint{11.429298in}{5.344240in}}%
\pgfpathcurveto{\pgfqpoint{11.440348in}{5.344240in}}{\pgfqpoint{11.450947in}{5.348630in}}{\pgfqpoint{11.458761in}{5.356444in}}%
\pgfpathcurveto{\pgfqpoint{11.466574in}{5.364257in}}{\pgfqpoint{11.470965in}{5.374856in}}{\pgfqpoint{11.470965in}{5.385906in}}%
\pgfpathcurveto{\pgfqpoint{11.470965in}{5.396956in}}{\pgfqpoint{11.466574in}{5.407555in}}{\pgfqpoint{11.458761in}{5.415369in}}%
\pgfpathcurveto{\pgfqpoint{11.450947in}{5.423183in}}{\pgfqpoint{11.440348in}{5.427573in}}{\pgfqpoint{11.429298in}{5.427573in}}%
\pgfpathcurveto{\pgfqpoint{11.418248in}{5.427573in}}{\pgfqpoint{11.407649in}{5.423183in}}{\pgfqpoint{11.399835in}{5.415369in}}%
\pgfpathcurveto{\pgfqpoint{11.392022in}{5.407555in}}{\pgfqpoint{11.387631in}{5.396956in}}{\pgfqpoint{11.387631in}{5.385906in}}%
\pgfpathcurveto{\pgfqpoint{11.387631in}{5.374856in}}{\pgfqpoint{11.392022in}{5.364257in}}{\pgfqpoint{11.399835in}{5.356444in}}%
\pgfpathcurveto{\pgfqpoint{11.407649in}{5.348630in}}{\pgfqpoint{11.418248in}{5.344240in}}{\pgfqpoint{11.429298in}{5.344240in}}%
\pgfpathclose%
\pgfusepath{stroke,fill}%
\end{pgfscope}%
\begin{pgfscope}%
\pgfpathrectangle{\pgfqpoint{7.640588in}{4.121437in}}{\pgfqpoint{5.699255in}{2.685432in}}%
\pgfusepath{clip}%
\pgfsetbuttcap%
\pgfsetroundjoin%
\definecolor{currentfill}{rgb}{0.000000,0.000000,0.000000}%
\pgfsetfillcolor{currentfill}%
\pgfsetlinewidth{1.003750pt}%
\definecolor{currentstroke}{rgb}{0.000000,0.000000,0.000000}%
\pgfsetstrokecolor{currentstroke}%
\pgfsetdash{}{0pt}%
\pgfpathmoveto{\pgfqpoint{11.170241in}{5.688526in}}%
\pgfpathcurveto{\pgfqpoint{11.181291in}{5.688526in}}{\pgfqpoint{11.191890in}{5.692916in}}{\pgfqpoint{11.199704in}{5.700730in}}%
\pgfpathcurveto{\pgfqpoint{11.207517in}{5.708543in}}{\pgfqpoint{11.211908in}{5.719142in}}{\pgfqpoint{11.211908in}{5.730192in}}%
\pgfpathcurveto{\pgfqpoint{11.211908in}{5.741243in}}{\pgfqpoint{11.207517in}{5.751842in}}{\pgfqpoint{11.199704in}{5.759655in}}%
\pgfpathcurveto{\pgfqpoint{11.191890in}{5.767469in}}{\pgfqpoint{11.181291in}{5.771859in}}{\pgfqpoint{11.170241in}{5.771859in}}%
\pgfpathcurveto{\pgfqpoint{11.159191in}{5.771859in}}{\pgfqpoint{11.148592in}{5.767469in}}{\pgfqpoint{11.140778in}{5.759655in}}%
\pgfpathcurveto{\pgfqpoint{11.132964in}{5.751842in}}{\pgfqpoint{11.128574in}{5.741243in}}{\pgfqpoint{11.128574in}{5.730192in}}%
\pgfpathcurveto{\pgfqpoint{11.128574in}{5.719142in}}{\pgfqpoint{11.132964in}{5.708543in}}{\pgfqpoint{11.140778in}{5.700730in}}%
\pgfpathcurveto{\pgfqpoint{11.148592in}{5.692916in}}{\pgfqpoint{11.159191in}{5.688526in}}{\pgfqpoint{11.170241in}{5.688526in}}%
\pgfpathclose%
\pgfusepath{stroke,fill}%
\end{pgfscope}%
\begin{pgfscope}%
\pgfpathrectangle{\pgfqpoint{7.640588in}{4.121437in}}{\pgfqpoint{5.699255in}{2.685432in}}%
\pgfusepath{clip}%
\pgfsetbuttcap%
\pgfsetroundjoin%
\definecolor{currentfill}{rgb}{0.000000,0.000000,0.000000}%
\pgfsetfillcolor{currentfill}%
\pgfsetlinewidth{1.003750pt}%
\definecolor{currentstroke}{rgb}{0.000000,0.000000,0.000000}%
\pgfsetstrokecolor{currentstroke}%
\pgfsetdash{}{0pt}%
\pgfpathmoveto{\pgfqpoint{11.105477in}{5.704175in}}%
\pgfpathcurveto{\pgfqpoint{11.116527in}{5.704175in}}{\pgfqpoint{11.127126in}{5.708565in}}{\pgfqpoint{11.134939in}{5.716379in}}%
\pgfpathcurveto{\pgfqpoint{11.142753in}{5.724193in}}{\pgfqpoint{11.147143in}{5.734792in}}{\pgfqpoint{11.147143in}{5.745842in}}%
\pgfpathcurveto{\pgfqpoint{11.147143in}{5.756892in}}{\pgfqpoint{11.142753in}{5.767491in}}{\pgfqpoint{11.134939in}{5.775305in}}%
\pgfpathcurveto{\pgfqpoint{11.127126in}{5.783118in}}{\pgfqpoint{11.116527in}{5.787509in}}{\pgfqpoint{11.105477in}{5.787509in}}%
\pgfpathcurveto{\pgfqpoint{11.094426in}{5.787509in}}{\pgfqpoint{11.083827in}{5.783118in}}{\pgfqpoint{11.076014in}{5.775305in}}%
\pgfpathcurveto{\pgfqpoint{11.068200in}{5.767491in}}{\pgfqpoint{11.063810in}{5.756892in}}{\pgfqpoint{11.063810in}{5.745842in}}%
\pgfpathcurveto{\pgfqpoint{11.063810in}{5.734792in}}{\pgfqpoint{11.068200in}{5.724193in}}{\pgfqpoint{11.076014in}{5.716379in}}%
\pgfpathcurveto{\pgfqpoint{11.083827in}{5.708565in}}{\pgfqpoint{11.094426in}{5.704175in}}{\pgfqpoint{11.105477in}{5.704175in}}%
\pgfpathclose%
\pgfusepath{stroke,fill}%
\end{pgfscope}%
\begin{pgfscope}%
\pgfpathrectangle{\pgfqpoint{7.640588in}{4.121437in}}{\pgfqpoint{5.699255in}{2.685432in}}%
\pgfusepath{clip}%
\pgfsetbuttcap%
\pgfsetroundjoin%
\definecolor{currentfill}{rgb}{0.000000,0.000000,0.000000}%
\pgfsetfillcolor{currentfill}%
\pgfsetlinewidth{1.003750pt}%
\definecolor{currentstroke}{rgb}{0.000000,0.000000,0.000000}%
\pgfsetstrokecolor{currentstroke}%
\pgfsetdash{}{0pt}%
\pgfpathmoveto{\pgfqpoint{10.814037in}{5.970214in}}%
\pgfpathcurveto{\pgfqpoint{10.825088in}{5.970214in}}{\pgfqpoint{10.835687in}{5.974605in}}{\pgfqpoint{10.843500in}{5.982418in}}%
\pgfpathcurveto{\pgfqpoint{10.851314in}{5.990232in}}{\pgfqpoint{10.855704in}{6.000831in}}{\pgfqpoint{10.855704in}{6.011881in}}%
\pgfpathcurveto{\pgfqpoint{10.855704in}{6.022931in}}{\pgfqpoint{10.851314in}{6.033530in}}{\pgfqpoint{10.843500in}{6.041344in}}%
\pgfpathcurveto{\pgfqpoint{10.835687in}{6.049158in}}{\pgfqpoint{10.825088in}{6.053548in}}{\pgfqpoint{10.814037in}{6.053548in}}%
\pgfpathcurveto{\pgfqpoint{10.802987in}{6.053548in}}{\pgfqpoint{10.792388in}{6.049158in}}{\pgfqpoint{10.784575in}{6.041344in}}%
\pgfpathcurveto{\pgfqpoint{10.776761in}{6.033530in}}{\pgfqpoint{10.772371in}{6.022931in}}{\pgfqpoint{10.772371in}{6.011881in}}%
\pgfpathcurveto{\pgfqpoint{10.772371in}{6.000831in}}{\pgfqpoint{10.776761in}{5.990232in}}{\pgfqpoint{10.784575in}{5.982418in}}%
\pgfpathcurveto{\pgfqpoint{10.792388in}{5.974605in}}{\pgfqpoint{10.802987in}{5.970214in}}{\pgfqpoint{10.814037in}{5.970214in}}%
\pgfpathclose%
\pgfusepath{stroke,fill}%
\end{pgfscope}%
\begin{pgfscope}%
\pgfpathrectangle{\pgfqpoint{7.640588in}{4.121437in}}{\pgfqpoint{5.699255in}{2.685432in}}%
\pgfusepath{clip}%
\pgfsetbuttcap%
\pgfsetroundjoin%
\definecolor{currentfill}{rgb}{0.000000,0.000000,0.000000}%
\pgfsetfillcolor{currentfill}%
\pgfsetlinewidth{1.003750pt}%
\definecolor{currentstroke}{rgb}{0.000000,0.000000,0.000000}%
\pgfsetstrokecolor{currentstroke}%
\pgfsetdash{}{0pt}%
\pgfpathmoveto{\pgfqpoint{10.295923in}{5.923266in}}%
\pgfpathcurveto{\pgfqpoint{10.306973in}{5.923266in}}{\pgfqpoint{10.317572in}{5.927657in}}{\pgfqpoint{10.325386in}{5.935470in}}%
\pgfpathcurveto{\pgfqpoint{10.333200in}{5.943284in}}{\pgfqpoint{10.337590in}{5.953883in}}{\pgfqpoint{10.337590in}{5.964933in}}%
\pgfpathcurveto{\pgfqpoint{10.337590in}{5.975983in}}{\pgfqpoint{10.333200in}{5.986582in}}{\pgfqpoint{10.325386in}{5.994396in}}%
\pgfpathcurveto{\pgfqpoint{10.317572in}{6.002209in}}{\pgfqpoint{10.306973in}{6.006600in}}{\pgfqpoint{10.295923in}{6.006600in}}%
\pgfpathcurveto{\pgfqpoint{10.284873in}{6.006600in}}{\pgfqpoint{10.274274in}{6.002209in}}{\pgfqpoint{10.266460in}{5.994396in}}%
\pgfpathcurveto{\pgfqpoint{10.258647in}{5.986582in}}{\pgfqpoint{10.254257in}{5.975983in}}{\pgfqpoint{10.254257in}{5.964933in}}%
\pgfpathcurveto{\pgfqpoint{10.254257in}{5.953883in}}{\pgfqpoint{10.258647in}{5.943284in}}{\pgfqpoint{10.266460in}{5.935470in}}%
\pgfpathcurveto{\pgfqpoint{10.274274in}{5.927657in}}{\pgfqpoint{10.284873in}{5.923266in}}{\pgfqpoint{10.295923in}{5.923266in}}%
\pgfpathclose%
\pgfusepath{stroke,fill}%
\end{pgfscope}%
\begin{pgfscope}%
\pgfpathrectangle{\pgfqpoint{7.640588in}{4.121437in}}{\pgfqpoint{5.699255in}{2.685432in}}%
\pgfusepath{clip}%
\pgfsetbuttcap%
\pgfsetroundjoin%
\definecolor{currentfill}{rgb}{0.000000,0.000000,0.000000}%
\pgfsetfillcolor{currentfill}%
\pgfsetlinewidth{1.003750pt}%
\definecolor{currentstroke}{rgb}{0.000000,0.000000,0.000000}%
\pgfsetstrokecolor{currentstroke}%
\pgfsetdash{}{0pt}%
\pgfpathmoveto{\pgfqpoint{10.004484in}{5.813721in}}%
\pgfpathcurveto{\pgfqpoint{10.015534in}{5.813721in}}{\pgfqpoint{10.026133in}{5.818111in}}{\pgfqpoint{10.033947in}{5.825925in}}%
\pgfpathcurveto{\pgfqpoint{10.041760in}{5.833738in}}{\pgfqpoint{10.046151in}{5.844337in}}{\pgfqpoint{10.046151in}{5.855387in}}%
\pgfpathcurveto{\pgfqpoint{10.046151in}{5.866438in}}{\pgfqpoint{10.041760in}{5.877037in}}{\pgfqpoint{10.033947in}{5.884850in}}%
\pgfpathcurveto{\pgfqpoint{10.026133in}{5.892664in}}{\pgfqpoint{10.015534in}{5.897054in}}{\pgfqpoint{10.004484in}{5.897054in}}%
\pgfpathcurveto{\pgfqpoint{9.993434in}{5.897054in}}{\pgfqpoint{9.982835in}{5.892664in}}{\pgfqpoint{9.975021in}{5.884850in}}%
\pgfpathcurveto{\pgfqpoint{9.967208in}{5.877037in}}{\pgfqpoint{9.962817in}{5.866438in}}{\pgfqpoint{9.962817in}{5.855387in}}%
\pgfpathcurveto{\pgfqpoint{9.962817in}{5.844337in}}{\pgfqpoint{9.967208in}{5.833738in}}{\pgfqpoint{9.975021in}{5.825925in}}%
\pgfpathcurveto{\pgfqpoint{9.982835in}{5.818111in}}{\pgfqpoint{9.993434in}{5.813721in}}{\pgfqpoint{10.004484in}{5.813721in}}%
\pgfpathclose%
\pgfusepath{stroke,fill}%
\end{pgfscope}%
\begin{pgfscope}%
\pgfpathrectangle{\pgfqpoint{7.640588in}{4.121437in}}{\pgfqpoint{5.699255in}{2.685432in}}%
\pgfusepath{clip}%
\pgfsetbuttcap%
\pgfsetroundjoin%
\definecolor{currentfill}{rgb}{0.000000,0.000000,0.000000}%
\pgfsetfillcolor{currentfill}%
\pgfsetlinewidth{1.003750pt}%
\definecolor{currentstroke}{rgb}{0.000000,0.000000,0.000000}%
\pgfsetstrokecolor{currentstroke}%
\pgfsetdash{}{0pt}%
\pgfpathmoveto{\pgfqpoint{10.101630in}{5.610279in}}%
\pgfpathcurveto{\pgfqpoint{10.112681in}{5.610279in}}{\pgfqpoint{10.123280in}{5.614669in}}{\pgfqpoint{10.131093in}{5.622483in}}%
\pgfpathcurveto{\pgfqpoint{10.138907in}{5.630296in}}{\pgfqpoint{10.143297in}{5.640895in}}{\pgfqpoint{10.143297in}{5.651946in}}%
\pgfpathcurveto{\pgfqpoint{10.143297in}{5.662996in}}{\pgfqpoint{10.138907in}{5.673595in}}{\pgfqpoint{10.131093in}{5.681408in}}%
\pgfpathcurveto{\pgfqpoint{10.123280in}{5.689222in}}{\pgfqpoint{10.112681in}{5.693612in}}{\pgfqpoint{10.101630in}{5.693612in}}%
\pgfpathcurveto{\pgfqpoint{10.090580in}{5.693612in}}{\pgfqpoint{10.079981in}{5.689222in}}{\pgfqpoint{10.072168in}{5.681408in}}%
\pgfpathcurveto{\pgfqpoint{10.064354in}{5.673595in}}{\pgfqpoint{10.059964in}{5.662996in}}{\pgfqpoint{10.059964in}{5.651946in}}%
\pgfpathcurveto{\pgfqpoint{10.059964in}{5.640895in}}{\pgfqpoint{10.064354in}{5.630296in}}{\pgfqpoint{10.072168in}{5.622483in}}%
\pgfpathcurveto{\pgfqpoint{10.079981in}{5.614669in}}{\pgfqpoint{10.090580in}{5.610279in}}{\pgfqpoint{10.101630in}{5.610279in}}%
\pgfpathclose%
\pgfusepath{stroke,fill}%
\end{pgfscope}%
\begin{pgfscope}%
\pgfpathrectangle{\pgfqpoint{7.640588in}{4.121437in}}{\pgfqpoint{5.699255in}{2.685432in}}%
\pgfusepath{clip}%
\pgfsetbuttcap%
\pgfsetroundjoin%
\definecolor{currentfill}{rgb}{0.000000,0.000000,0.000000}%
\pgfsetfillcolor{currentfill}%
\pgfsetlinewidth{1.003750pt}%
\definecolor{currentstroke}{rgb}{0.000000,0.000000,0.000000}%
\pgfsetstrokecolor{currentstroke}%
\pgfsetdash{}{0pt}%
\pgfpathmoveto{\pgfqpoint{10.101630in}{5.547681in}}%
\pgfpathcurveto{\pgfqpoint{10.112681in}{5.547681in}}{\pgfqpoint{10.123280in}{5.552072in}}{\pgfqpoint{10.131093in}{5.559885in}}%
\pgfpathcurveto{\pgfqpoint{10.138907in}{5.567699in}}{\pgfqpoint{10.143297in}{5.578298in}}{\pgfqpoint{10.143297in}{5.589348in}}%
\pgfpathcurveto{\pgfqpoint{10.143297in}{5.600398in}}{\pgfqpoint{10.138907in}{5.610997in}}{\pgfqpoint{10.131093in}{5.618811in}}%
\pgfpathcurveto{\pgfqpoint{10.123280in}{5.626625in}}{\pgfqpoint{10.112681in}{5.631015in}}{\pgfqpoint{10.101630in}{5.631015in}}%
\pgfpathcurveto{\pgfqpoint{10.090580in}{5.631015in}}{\pgfqpoint{10.079981in}{5.626625in}}{\pgfqpoint{10.072168in}{5.618811in}}%
\pgfpathcurveto{\pgfqpoint{10.064354in}{5.610997in}}{\pgfqpoint{10.059964in}{5.600398in}}{\pgfqpoint{10.059964in}{5.589348in}}%
\pgfpathcurveto{\pgfqpoint{10.059964in}{5.578298in}}{\pgfqpoint{10.064354in}{5.567699in}}{\pgfqpoint{10.072168in}{5.559885in}}%
\pgfpathcurveto{\pgfqpoint{10.079981in}{5.552072in}}{\pgfqpoint{10.090580in}{5.547681in}}{\pgfqpoint{10.101630in}{5.547681in}}%
\pgfpathclose%
\pgfusepath{stroke,fill}%
\end{pgfscope}%
\begin{pgfscope}%
\pgfpathrectangle{\pgfqpoint{7.640588in}{4.121437in}}{\pgfqpoint{5.699255in}{2.685432in}}%
\pgfusepath{clip}%
\pgfsetbuttcap%
\pgfsetroundjoin%
\definecolor{currentfill}{rgb}{0.000000,0.000000,0.000000}%
\pgfsetfillcolor{currentfill}%
\pgfsetlinewidth{1.003750pt}%
\definecolor{currentstroke}{rgb}{0.000000,0.000000,0.000000}%
\pgfsetstrokecolor{currentstroke}%
\pgfsetdash{}{0pt}%
\pgfpathmoveto{\pgfqpoint{10.134013in}{5.422486in}}%
\pgfpathcurveto{\pgfqpoint{10.145063in}{5.422486in}}{\pgfqpoint{10.155662in}{5.426877in}}{\pgfqpoint{10.163475in}{5.434690in}}%
\pgfpathcurveto{\pgfqpoint{10.171289in}{5.442504in}}{\pgfqpoint{10.175679in}{5.453103in}}{\pgfqpoint{10.175679in}{5.464153in}}%
\pgfpathcurveto{\pgfqpoint{10.175679in}{5.475203in}}{\pgfqpoint{10.171289in}{5.485802in}}{\pgfqpoint{10.163475in}{5.493616in}}%
\pgfpathcurveto{\pgfqpoint{10.155662in}{5.501430in}}{\pgfqpoint{10.145063in}{5.505820in}}{\pgfqpoint{10.134013in}{5.505820in}}%
\pgfpathcurveto{\pgfqpoint{10.122962in}{5.505820in}}{\pgfqpoint{10.112363in}{5.501430in}}{\pgfqpoint{10.104550in}{5.493616in}}%
\pgfpathcurveto{\pgfqpoint{10.096736in}{5.485802in}}{\pgfqpoint{10.092346in}{5.475203in}}{\pgfqpoint{10.092346in}{5.464153in}}%
\pgfpathcurveto{\pgfqpoint{10.092346in}{5.453103in}}{\pgfqpoint{10.096736in}{5.442504in}}{\pgfqpoint{10.104550in}{5.434690in}}%
\pgfpathcurveto{\pgfqpoint{10.112363in}{5.426877in}}{\pgfqpoint{10.122962in}{5.422486in}}{\pgfqpoint{10.134013in}{5.422486in}}%
\pgfpathclose%
\pgfusepath{stroke,fill}%
\end{pgfscope}%
\begin{pgfscope}%
\pgfpathrectangle{\pgfqpoint{7.640588in}{4.121437in}}{\pgfqpoint{5.699255in}{2.685432in}}%
\pgfusepath{clip}%
\pgfsetbuttcap%
\pgfsetroundjoin%
\definecolor{currentfill}{rgb}{0.000000,0.000000,0.000000}%
\pgfsetfillcolor{currentfill}%
\pgfsetlinewidth{1.003750pt}%
\definecolor{currentstroke}{rgb}{0.000000,0.000000,0.000000}%
\pgfsetstrokecolor{currentstroke}%
\pgfsetdash{}{0pt}%
\pgfpathmoveto{\pgfqpoint{10.198777in}{5.250343in}}%
\pgfpathcurveto{\pgfqpoint{10.209827in}{5.250343in}}{\pgfqpoint{10.220426in}{5.254734in}}{\pgfqpoint{10.228240in}{5.262547in}}%
\pgfpathcurveto{\pgfqpoint{10.236053in}{5.270361in}}{\pgfqpoint{10.240444in}{5.280960in}}{\pgfqpoint{10.240444in}{5.292010in}}%
\pgfpathcurveto{\pgfqpoint{10.240444in}{5.303060in}}{\pgfqpoint{10.236053in}{5.313659in}}{\pgfqpoint{10.228240in}{5.321473in}}%
\pgfpathcurveto{\pgfqpoint{10.220426in}{5.329286in}}{\pgfqpoint{10.209827in}{5.333677in}}{\pgfqpoint{10.198777in}{5.333677in}}%
\pgfpathcurveto{\pgfqpoint{10.187727in}{5.333677in}}{\pgfqpoint{10.177128in}{5.329286in}}{\pgfqpoint{10.169314in}{5.321473in}}%
\pgfpathcurveto{\pgfqpoint{10.161500in}{5.313659in}}{\pgfqpoint{10.157110in}{5.303060in}}{\pgfqpoint{10.157110in}{5.292010in}}%
\pgfpathcurveto{\pgfqpoint{10.157110in}{5.280960in}}{\pgfqpoint{10.161500in}{5.270361in}}{\pgfqpoint{10.169314in}{5.262547in}}%
\pgfpathcurveto{\pgfqpoint{10.177128in}{5.254734in}}{\pgfqpoint{10.187727in}{5.250343in}}{\pgfqpoint{10.198777in}{5.250343in}}%
\pgfpathclose%
\pgfusepath{stroke,fill}%
\end{pgfscope}%
\begin{pgfscope}%
\pgfpathrectangle{\pgfqpoint{7.640588in}{4.121437in}}{\pgfqpoint{5.699255in}{2.685432in}}%
\pgfusepath{clip}%
\pgfsetbuttcap%
\pgfsetroundjoin%
\definecolor{currentfill}{rgb}{0.000000,0.000000,0.000000}%
\pgfsetfillcolor{currentfill}%
\pgfsetlinewidth{1.003750pt}%
\definecolor{currentstroke}{rgb}{0.000000,0.000000,0.000000}%
\pgfsetstrokecolor{currentstroke}%
\pgfsetdash{}{0pt}%
\pgfpathmoveto{\pgfqpoint{10.166395in}{5.156447in}}%
\pgfpathcurveto{\pgfqpoint{10.177445in}{5.156447in}}{\pgfqpoint{10.188044in}{5.160837in}}{\pgfqpoint{10.195858in}{5.168651in}}%
\pgfpathcurveto{\pgfqpoint{10.203671in}{5.176465in}}{\pgfqpoint{10.208061in}{5.187064in}}{\pgfqpoint{10.208061in}{5.198114in}}%
\pgfpathcurveto{\pgfqpoint{10.208061in}{5.209164in}}{\pgfqpoint{10.203671in}{5.219763in}}{\pgfqpoint{10.195858in}{5.227577in}}%
\pgfpathcurveto{\pgfqpoint{10.188044in}{5.235390in}}{\pgfqpoint{10.177445in}{5.239781in}}{\pgfqpoint{10.166395in}{5.239781in}}%
\pgfpathcurveto{\pgfqpoint{10.155345in}{5.239781in}}{\pgfqpoint{10.144746in}{5.235390in}}{\pgfqpoint{10.136932in}{5.227577in}}%
\pgfpathcurveto{\pgfqpoint{10.129118in}{5.219763in}}{\pgfqpoint{10.124728in}{5.209164in}}{\pgfqpoint{10.124728in}{5.198114in}}%
\pgfpathcurveto{\pgfqpoint{10.124728in}{5.187064in}}{\pgfqpoint{10.129118in}{5.176465in}}{\pgfqpoint{10.136932in}{5.168651in}}%
\pgfpathcurveto{\pgfqpoint{10.144746in}{5.160837in}}{\pgfqpoint{10.155345in}{5.156447in}}{\pgfqpoint{10.166395in}{5.156447in}}%
\pgfpathclose%
\pgfusepath{stroke,fill}%
\end{pgfscope}%
\begin{pgfscope}%
\pgfpathrectangle{\pgfqpoint{7.640588in}{4.121437in}}{\pgfqpoint{5.699255in}{2.685432in}}%
\pgfusepath{clip}%
\pgfsetbuttcap%
\pgfsetroundjoin%
\definecolor{currentfill}{rgb}{0.000000,0.000000,0.000000}%
\pgfsetfillcolor{currentfill}%
\pgfsetlinewidth{1.003750pt}%
\definecolor{currentstroke}{rgb}{0.000000,0.000000,0.000000}%
\pgfsetstrokecolor{currentstroke}%
\pgfsetdash{}{0pt}%
\pgfpathmoveto{\pgfqpoint{10.878802in}{5.312941in}}%
\pgfpathcurveto{\pgfqpoint{10.889852in}{5.312941in}}{\pgfqpoint{10.900451in}{5.317331in}}{\pgfqpoint{10.908264in}{5.325145in}}%
\pgfpathcurveto{\pgfqpoint{10.916078in}{5.332958in}}{\pgfqpoint{10.920468in}{5.343557in}}{\pgfqpoint{10.920468in}{5.354608in}}%
\pgfpathcurveto{\pgfqpoint{10.920468in}{5.365658in}}{\pgfqpoint{10.916078in}{5.376257in}}{\pgfqpoint{10.908264in}{5.384070in}}%
\pgfpathcurveto{\pgfqpoint{10.900451in}{5.391884in}}{\pgfqpoint{10.889852in}{5.396274in}}{\pgfqpoint{10.878802in}{5.396274in}}%
\pgfpathcurveto{\pgfqpoint{10.867752in}{5.396274in}}{\pgfqpoint{10.857152in}{5.391884in}}{\pgfqpoint{10.849339in}{5.384070in}}%
\pgfpathcurveto{\pgfqpoint{10.841525in}{5.376257in}}{\pgfqpoint{10.837135in}{5.365658in}}{\pgfqpoint{10.837135in}{5.354608in}}%
\pgfpathcurveto{\pgfqpoint{10.837135in}{5.343557in}}{\pgfqpoint{10.841525in}{5.332958in}}{\pgfqpoint{10.849339in}{5.325145in}}%
\pgfpathcurveto{\pgfqpoint{10.857152in}{5.317331in}}{\pgfqpoint{10.867752in}{5.312941in}}{\pgfqpoint{10.878802in}{5.312941in}}%
\pgfpathclose%
\pgfusepath{stroke,fill}%
\end{pgfscope}%
\begin{pgfscope}%
\pgfpathrectangle{\pgfqpoint{7.640588in}{4.121437in}}{\pgfqpoint{5.699255in}{2.685432in}}%
\pgfusepath{clip}%
\pgfsetbuttcap%
\pgfsetroundjoin%
\definecolor{currentfill}{rgb}{0.000000,0.000000,0.000000}%
\pgfsetfillcolor{currentfill}%
\pgfsetlinewidth{1.003750pt}%
\definecolor{currentstroke}{rgb}{0.000000,0.000000,0.000000}%
\pgfsetstrokecolor{currentstroke}%
\pgfsetdash{}{0pt}%
\pgfpathmoveto{\pgfqpoint{10.716891in}{4.874759in}}%
\pgfpathcurveto{\pgfqpoint{10.727941in}{4.874759in}}{\pgfqpoint{10.738540in}{4.879149in}}{\pgfqpoint{10.746354in}{4.886962in}}%
\pgfpathcurveto{\pgfqpoint{10.754167in}{4.894776in}}{\pgfqpoint{10.758558in}{4.905375in}}{\pgfqpoint{10.758558in}{4.916425in}}%
\pgfpathcurveto{\pgfqpoint{10.758558in}{4.927475in}}{\pgfqpoint{10.754167in}{4.938074in}}{\pgfqpoint{10.746354in}{4.945888in}}%
\pgfpathcurveto{\pgfqpoint{10.738540in}{4.953702in}}{\pgfqpoint{10.727941in}{4.958092in}}{\pgfqpoint{10.716891in}{4.958092in}}%
\pgfpathcurveto{\pgfqpoint{10.705841in}{4.958092in}}{\pgfqpoint{10.695242in}{4.953702in}}{\pgfqpoint{10.687428in}{4.945888in}}%
\pgfpathcurveto{\pgfqpoint{10.679615in}{4.938074in}}{\pgfqpoint{10.675224in}{4.927475in}}{\pgfqpoint{10.675224in}{4.916425in}}%
\pgfpathcurveto{\pgfqpoint{10.675224in}{4.905375in}}{\pgfqpoint{10.679615in}{4.894776in}}{\pgfqpoint{10.687428in}{4.886962in}}%
\pgfpathcurveto{\pgfqpoint{10.695242in}{4.879149in}}{\pgfqpoint{10.705841in}{4.874759in}}{\pgfqpoint{10.716891in}{4.874759in}}%
\pgfpathclose%
\pgfusepath{stroke,fill}%
\end{pgfscope}%
\begin{pgfscope}%
\pgfpathrectangle{\pgfqpoint{7.640588in}{4.121437in}}{\pgfqpoint{5.699255in}{2.685432in}}%
\pgfusepath{clip}%
\pgfsetbuttcap%
\pgfsetroundjoin%
\definecolor{currentfill}{rgb}{0.000000,0.000000,0.000000}%
\pgfsetfillcolor{currentfill}%
\pgfsetlinewidth{1.003750pt}%
\definecolor{currentstroke}{rgb}{0.000000,0.000000,0.000000}%
\pgfsetstrokecolor{currentstroke}%
\pgfsetdash{}{0pt}%
\pgfpathmoveto{\pgfqpoint{10.263541in}{4.733914in}}%
\pgfpathcurveto{\pgfqpoint{10.274591in}{4.733914in}}{\pgfqpoint{10.285190in}{4.738304in}}{\pgfqpoint{10.293004in}{4.746118in}}%
\pgfpathcurveto{\pgfqpoint{10.300818in}{4.753932in}}{\pgfqpoint{10.305208in}{4.764531in}}{\pgfqpoint{10.305208in}{4.775581in}}%
\pgfpathcurveto{\pgfqpoint{10.305208in}{4.786631in}}{\pgfqpoint{10.300818in}{4.797230in}}{\pgfqpoint{10.293004in}{4.805044in}}%
\pgfpathcurveto{\pgfqpoint{10.285190in}{4.812857in}}{\pgfqpoint{10.274591in}{4.817247in}}{\pgfqpoint{10.263541in}{4.817247in}}%
\pgfpathcurveto{\pgfqpoint{10.252491in}{4.817247in}}{\pgfqpoint{10.241892in}{4.812857in}}{\pgfqpoint{10.234078in}{4.805044in}}%
\pgfpathcurveto{\pgfqpoint{10.226265in}{4.797230in}}{\pgfqpoint{10.221874in}{4.786631in}}{\pgfqpoint{10.221874in}{4.775581in}}%
\pgfpathcurveto{\pgfqpoint{10.221874in}{4.764531in}}{\pgfqpoint{10.226265in}{4.753932in}}{\pgfqpoint{10.234078in}{4.746118in}}%
\pgfpathcurveto{\pgfqpoint{10.241892in}{4.738304in}}{\pgfqpoint{10.252491in}{4.733914in}}{\pgfqpoint{10.263541in}{4.733914in}}%
\pgfpathclose%
\pgfusepath{stroke,fill}%
\end{pgfscope}%
\begin{pgfscope}%
\pgfpathrectangle{\pgfqpoint{7.640588in}{4.121437in}}{\pgfqpoint{5.699255in}{2.685432in}}%
\pgfusepath{clip}%
\pgfsetbuttcap%
\pgfsetroundjoin%
\definecolor{currentfill}{rgb}{0.000000,0.000000,0.000000}%
\pgfsetfillcolor{currentfill}%
\pgfsetlinewidth{1.003750pt}%
\definecolor{currentstroke}{rgb}{0.000000,0.000000,0.000000}%
\pgfsetstrokecolor{currentstroke}%
\pgfsetdash{}{0pt}%
\pgfpathmoveto{\pgfqpoint{9.907338in}{4.686966in}}%
\pgfpathcurveto{\pgfqpoint{9.918388in}{4.686966in}}{\pgfqpoint{9.928987in}{4.691356in}}{\pgfqpoint{9.936800in}{4.699170in}}%
\pgfpathcurveto{\pgfqpoint{9.944614in}{4.706984in}}{\pgfqpoint{9.949004in}{4.717583in}}{\pgfqpoint{9.949004in}{4.728633in}}%
\pgfpathcurveto{\pgfqpoint{9.949004in}{4.739683in}}{\pgfqpoint{9.944614in}{4.750282in}}{\pgfqpoint{9.936800in}{4.758096in}}%
\pgfpathcurveto{\pgfqpoint{9.928987in}{4.765909in}}{\pgfqpoint{9.918388in}{4.770299in}}{\pgfqpoint{9.907338in}{4.770299in}}%
\pgfpathcurveto{\pgfqpoint{9.896288in}{4.770299in}}{\pgfqpoint{9.885689in}{4.765909in}}{\pgfqpoint{9.877875in}{4.758096in}}%
\pgfpathcurveto{\pgfqpoint{9.870061in}{4.750282in}}{\pgfqpoint{9.865671in}{4.739683in}}{\pgfqpoint{9.865671in}{4.728633in}}%
\pgfpathcurveto{\pgfqpoint{9.865671in}{4.717583in}}{\pgfqpoint{9.870061in}{4.706984in}}{\pgfqpoint{9.877875in}{4.699170in}}%
\pgfpathcurveto{\pgfqpoint{9.885689in}{4.691356in}}{\pgfqpoint{9.896288in}{4.686966in}}{\pgfqpoint{9.907338in}{4.686966in}}%
\pgfpathclose%
\pgfusepath{stroke,fill}%
\end{pgfscope}%
\begin{pgfscope}%
\pgfpathrectangle{\pgfqpoint{7.640588in}{4.121437in}}{\pgfqpoint{5.699255in}{2.685432in}}%
\pgfusepath{clip}%
\pgfsetbuttcap%
\pgfsetroundjoin%
\definecolor{currentfill}{rgb}{0.000000,0.000000,0.000000}%
\pgfsetfillcolor{currentfill}%
\pgfsetlinewidth{1.003750pt}%
\definecolor{currentstroke}{rgb}{0.000000,0.000000,0.000000}%
\pgfsetstrokecolor{currentstroke}%
\pgfsetdash{}{0pt}%
\pgfpathmoveto{\pgfqpoint{9.810191in}{4.467875in}}%
\pgfpathcurveto{\pgfqpoint{9.821241in}{4.467875in}}{\pgfqpoint{9.831840in}{4.472265in}}{\pgfqpoint{9.839654in}{4.480079in}}%
\pgfpathcurveto{\pgfqpoint{9.847468in}{4.487892in}}{\pgfqpoint{9.851858in}{4.498491in}}{\pgfqpoint{9.851858in}{4.509542in}}%
\pgfpathcurveto{\pgfqpoint{9.851858in}{4.520592in}}{\pgfqpoint{9.847468in}{4.531191in}}{\pgfqpoint{9.839654in}{4.539004in}}%
\pgfpathcurveto{\pgfqpoint{9.831840in}{4.546818in}}{\pgfqpoint{9.821241in}{4.551208in}}{\pgfqpoint{9.810191in}{4.551208in}}%
\pgfpathcurveto{\pgfqpoint{9.799141in}{4.551208in}}{\pgfqpoint{9.788542in}{4.546818in}}{\pgfqpoint{9.780728in}{4.539004in}}%
\pgfpathcurveto{\pgfqpoint{9.772915in}{4.531191in}}{\pgfqpoint{9.768525in}{4.520592in}}{\pgfqpoint{9.768525in}{4.509542in}}%
\pgfpathcurveto{\pgfqpoint{9.768525in}{4.498491in}}{\pgfqpoint{9.772915in}{4.487892in}}{\pgfqpoint{9.780728in}{4.480079in}}%
\pgfpathcurveto{\pgfqpoint{9.788542in}{4.472265in}}{\pgfqpoint{9.799141in}{4.467875in}}{\pgfqpoint{9.810191in}{4.467875in}}%
\pgfpathclose%
\pgfusepath{stroke,fill}%
\end{pgfscope}%
\begin{pgfscope}%
\pgfpathrectangle{\pgfqpoint{7.640588in}{4.121437in}}{\pgfqpoint{5.699255in}{2.685432in}}%
\pgfusepath{clip}%
\pgfsetbuttcap%
\pgfsetroundjoin%
\definecolor{currentfill}{rgb}{0.000000,0.000000,0.000000}%
\pgfsetfillcolor{currentfill}%
\pgfsetlinewidth{1.003750pt}%
\definecolor{currentstroke}{rgb}{0.000000,0.000000,0.000000}%
\pgfsetstrokecolor{currentstroke}%
\pgfsetdash{}{0pt}%
\pgfpathmoveto{\pgfqpoint{9.615898in}{4.264433in}}%
\pgfpathcurveto{\pgfqpoint{9.626949in}{4.264433in}}{\pgfqpoint{9.637548in}{4.268823in}}{\pgfqpoint{9.645361in}{4.276637in}}%
\pgfpathcurveto{\pgfqpoint{9.653175in}{4.284451in}}{\pgfqpoint{9.657565in}{4.295050in}}{\pgfqpoint{9.657565in}{4.306100in}}%
\pgfpathcurveto{\pgfqpoint{9.657565in}{4.317150in}}{\pgfqpoint{9.653175in}{4.327749in}}{\pgfqpoint{9.645361in}{4.335562in}}%
\pgfpathcurveto{\pgfqpoint{9.637548in}{4.343376in}}{\pgfqpoint{9.626949in}{4.347766in}}{\pgfqpoint{9.615898in}{4.347766in}}%
\pgfpathcurveto{\pgfqpoint{9.604848in}{4.347766in}}{\pgfqpoint{9.594249in}{4.343376in}}{\pgfqpoint{9.586436in}{4.335562in}}%
\pgfpathcurveto{\pgfqpoint{9.578622in}{4.327749in}}{\pgfqpoint{9.574232in}{4.317150in}}{\pgfqpoint{9.574232in}{4.306100in}}%
\pgfpathcurveto{\pgfqpoint{9.574232in}{4.295050in}}{\pgfqpoint{9.578622in}{4.284451in}}{\pgfqpoint{9.586436in}{4.276637in}}%
\pgfpathcurveto{\pgfqpoint{9.594249in}{4.268823in}}{\pgfqpoint{9.604848in}{4.264433in}}{\pgfqpoint{9.615898in}{4.264433in}}%
\pgfpathclose%
\pgfusepath{stroke,fill}%
\end{pgfscope}%
\begin{pgfscope}%
\pgfpathrectangle{\pgfqpoint{7.640588in}{4.121437in}}{\pgfqpoint{5.699255in}{2.685432in}}%
\pgfusepath{clip}%
\pgfsetbuttcap%
\pgfsetroundjoin%
\definecolor{currentfill}{rgb}{0.000000,0.000000,0.000000}%
\pgfsetfillcolor{currentfill}%
\pgfsetlinewidth{1.003750pt}%
\definecolor{currentstroke}{rgb}{0.000000,0.000000,0.000000}%
\pgfsetstrokecolor{currentstroke}%
\pgfsetdash{}{0pt}%
\pgfpathmoveto{\pgfqpoint{9.356841in}{4.389628in}}%
\pgfpathcurveto{\pgfqpoint{9.367892in}{4.389628in}}{\pgfqpoint{9.378491in}{4.394018in}}{\pgfqpoint{9.386304in}{4.401832in}}%
\pgfpathcurveto{\pgfqpoint{9.394118in}{4.409646in}}{\pgfqpoint{9.398508in}{4.420245in}}{\pgfqpoint{9.398508in}{4.431295in}}%
\pgfpathcurveto{\pgfqpoint{9.398508in}{4.442345in}}{\pgfqpoint{9.394118in}{4.452944in}}{\pgfqpoint{9.386304in}{4.460757in}}%
\pgfpathcurveto{\pgfqpoint{9.378491in}{4.468571in}}{\pgfqpoint{9.367892in}{4.472961in}}{\pgfqpoint{9.356841in}{4.472961in}}%
\pgfpathcurveto{\pgfqpoint{9.345791in}{4.472961in}}{\pgfqpoint{9.335192in}{4.468571in}}{\pgfqpoint{9.327379in}{4.460757in}}%
\pgfpathcurveto{\pgfqpoint{9.319565in}{4.452944in}}{\pgfqpoint{9.315175in}{4.442345in}}{\pgfqpoint{9.315175in}{4.431295in}}%
\pgfpathcurveto{\pgfqpoint{9.315175in}{4.420245in}}{\pgfqpoint{9.319565in}{4.409646in}}{\pgfqpoint{9.327379in}{4.401832in}}%
\pgfpathcurveto{\pgfqpoint{9.335192in}{4.394018in}}{\pgfqpoint{9.345791in}{4.389628in}}{\pgfqpoint{9.356841in}{4.389628in}}%
\pgfpathclose%
\pgfusepath{stroke,fill}%
\end{pgfscope}%
\begin{pgfscope}%
\pgfpathrectangle{\pgfqpoint{7.640588in}{4.121437in}}{\pgfqpoint{5.699255in}{2.685432in}}%
\pgfusepath{clip}%
\pgfsetbuttcap%
\pgfsetroundjoin%
\definecolor{currentfill}{rgb}{0.000000,0.000000,0.000000}%
\pgfsetfillcolor{currentfill}%
\pgfsetlinewidth{1.003750pt}%
\definecolor{currentstroke}{rgb}{0.000000,0.000000,0.000000}%
\pgfsetstrokecolor{currentstroke}%
\pgfsetdash{}{0pt}%
\pgfpathmoveto{\pgfqpoint{9.194931in}{4.483524in}}%
\pgfpathcurveto{\pgfqpoint{9.205981in}{4.483524in}}{\pgfqpoint{9.216580in}{4.487914in}}{\pgfqpoint{9.224394in}{4.495728in}}%
\pgfpathcurveto{\pgfqpoint{9.232207in}{4.503542in}}{\pgfqpoint{9.236597in}{4.514141in}}{\pgfqpoint{9.236597in}{4.525191in}}%
\pgfpathcurveto{\pgfqpoint{9.236597in}{4.536241in}}{\pgfqpoint{9.232207in}{4.546840in}}{\pgfqpoint{9.224394in}{4.554654in}}%
\pgfpathcurveto{\pgfqpoint{9.216580in}{4.562467in}}{\pgfqpoint{9.205981in}{4.566858in}}{\pgfqpoint{9.194931in}{4.566858in}}%
\pgfpathcurveto{\pgfqpoint{9.183881in}{4.566858in}}{\pgfqpoint{9.173282in}{4.562467in}}{\pgfqpoint{9.165468in}{4.554654in}}%
\pgfpathcurveto{\pgfqpoint{9.157654in}{4.546840in}}{\pgfqpoint{9.153264in}{4.536241in}}{\pgfqpoint{9.153264in}{4.525191in}}%
\pgfpathcurveto{\pgfqpoint{9.153264in}{4.514141in}}{\pgfqpoint{9.157654in}{4.503542in}}{\pgfqpoint{9.165468in}{4.495728in}}%
\pgfpathcurveto{\pgfqpoint{9.173282in}{4.487914in}}{\pgfqpoint{9.183881in}{4.483524in}}{\pgfqpoint{9.194931in}{4.483524in}}%
\pgfpathclose%
\pgfusepath{stroke,fill}%
\end{pgfscope}%
\begin{pgfscope}%
\pgfpathrectangle{\pgfqpoint{7.640588in}{4.121437in}}{\pgfqpoint{5.699255in}{2.685432in}}%
\pgfusepath{clip}%
\pgfsetbuttcap%
\pgfsetroundjoin%
\definecolor{currentfill}{rgb}{0.000000,0.000000,0.000000}%
\pgfsetfillcolor{currentfill}%
\pgfsetlinewidth{1.003750pt}%
\definecolor{currentstroke}{rgb}{0.000000,0.000000,0.000000}%
\pgfsetstrokecolor{currentstroke}%
\pgfsetdash{}{0pt}%
\pgfpathmoveto{\pgfqpoint{9.033020in}{4.530472in}}%
\pgfpathcurveto{\pgfqpoint{9.044070in}{4.530472in}}{\pgfqpoint{9.054669in}{4.534863in}}{\pgfqpoint{9.062483in}{4.542676in}}%
\pgfpathcurveto{\pgfqpoint{9.070296in}{4.550490in}}{\pgfqpoint{9.074687in}{4.561089in}}{\pgfqpoint{9.074687in}{4.572139in}}%
\pgfpathcurveto{\pgfqpoint{9.074687in}{4.583189in}}{\pgfqpoint{9.070296in}{4.593788in}}{\pgfqpoint{9.062483in}{4.601602in}}%
\pgfpathcurveto{\pgfqpoint{9.054669in}{4.609415in}}{\pgfqpoint{9.044070in}{4.613806in}}{\pgfqpoint{9.033020in}{4.613806in}}%
\pgfpathcurveto{\pgfqpoint{9.021970in}{4.613806in}}{\pgfqpoint{9.011371in}{4.609415in}}{\pgfqpoint{9.003557in}{4.601602in}}%
\pgfpathcurveto{\pgfqpoint{8.995744in}{4.593788in}}{\pgfqpoint{8.991353in}{4.583189in}}{\pgfqpoint{8.991353in}{4.572139in}}%
\pgfpathcurveto{\pgfqpoint{8.991353in}{4.561089in}}{\pgfqpoint{8.995744in}{4.550490in}}{\pgfqpoint{9.003557in}{4.542676in}}%
\pgfpathcurveto{\pgfqpoint{9.011371in}{4.534863in}}{\pgfqpoint{9.021970in}{4.530472in}}{\pgfqpoint{9.033020in}{4.530472in}}%
\pgfpathclose%
\pgfusepath{stroke,fill}%
\end{pgfscope}%
\begin{pgfscope}%
\pgfpathrectangle{\pgfqpoint{7.640588in}{4.121437in}}{\pgfqpoint{5.699255in}{2.685432in}}%
\pgfusepath{clip}%
\pgfsetbuttcap%
\pgfsetroundjoin%
\definecolor{currentfill}{rgb}{0.000000,0.000000,0.000000}%
\pgfsetfillcolor{currentfill}%
\pgfsetlinewidth{1.003750pt}%
\definecolor{currentstroke}{rgb}{0.000000,0.000000,0.000000}%
\pgfsetstrokecolor{currentstroke}%
\pgfsetdash{}{0pt}%
\pgfpathmoveto{\pgfqpoint{8.903492in}{4.467875in}}%
\pgfpathcurveto{\pgfqpoint{8.914542in}{4.467875in}}{\pgfqpoint{8.925141in}{4.472265in}}{\pgfqpoint{8.932954in}{4.480079in}}%
\pgfpathcurveto{\pgfqpoint{8.940768in}{4.487892in}}{\pgfqpoint{8.945158in}{4.498491in}}{\pgfqpoint{8.945158in}{4.509542in}}%
\pgfpathcurveto{\pgfqpoint{8.945158in}{4.520592in}}{\pgfqpoint{8.940768in}{4.531191in}}{\pgfqpoint{8.932954in}{4.539004in}}%
\pgfpathcurveto{\pgfqpoint{8.925141in}{4.546818in}}{\pgfqpoint{8.914542in}{4.551208in}}{\pgfqpoint{8.903492in}{4.551208in}}%
\pgfpathcurveto{\pgfqpoint{8.892441in}{4.551208in}}{\pgfqpoint{8.881842in}{4.546818in}}{\pgfqpoint{8.874029in}{4.539004in}}%
\pgfpathcurveto{\pgfqpoint{8.866215in}{4.531191in}}{\pgfqpoint{8.861825in}{4.520592in}}{\pgfqpoint{8.861825in}{4.509542in}}%
\pgfpathcurveto{\pgfqpoint{8.861825in}{4.498491in}}{\pgfqpoint{8.866215in}{4.487892in}}{\pgfqpoint{8.874029in}{4.480079in}}%
\pgfpathcurveto{\pgfqpoint{8.881842in}{4.472265in}}{\pgfqpoint{8.892441in}{4.467875in}}{\pgfqpoint{8.903492in}{4.467875in}}%
\pgfpathclose%
\pgfusepath{stroke,fill}%
\end{pgfscope}%
\begin{pgfscope}%
\pgfpathrectangle{\pgfqpoint{7.640588in}{4.121437in}}{\pgfqpoint{5.699255in}{2.685432in}}%
\pgfusepath{clip}%
\pgfsetbuttcap%
\pgfsetroundjoin%
\definecolor{currentfill}{rgb}{0.000000,0.000000,0.000000}%
\pgfsetfillcolor{currentfill}%
\pgfsetlinewidth{1.003750pt}%
\definecolor{currentstroke}{rgb}{0.000000,0.000000,0.000000}%
\pgfsetstrokecolor{currentstroke}%
\pgfsetdash{}{0pt}%
\pgfpathmoveto{\pgfqpoint{8.676817in}{4.514823in}}%
\pgfpathcurveto{\pgfqpoint{8.687867in}{4.514823in}}{\pgfqpoint{8.698466in}{4.519213in}}{\pgfqpoint{8.706279in}{4.527027in}}%
\pgfpathcurveto{\pgfqpoint{8.714093in}{4.534840in}}{\pgfqpoint{8.718483in}{4.545440in}}{\pgfqpoint{8.718483in}{4.556490in}}%
\pgfpathcurveto{\pgfqpoint{8.718483in}{4.567540in}}{\pgfqpoint{8.714093in}{4.578139in}}{\pgfqpoint{8.706279in}{4.585952in}}%
\pgfpathcurveto{\pgfqpoint{8.698466in}{4.593766in}}{\pgfqpoint{8.687867in}{4.598156in}}{\pgfqpoint{8.676817in}{4.598156in}}%
\pgfpathcurveto{\pgfqpoint{8.665766in}{4.598156in}}{\pgfqpoint{8.655167in}{4.593766in}}{\pgfqpoint{8.647354in}{4.585952in}}%
\pgfpathcurveto{\pgfqpoint{8.639540in}{4.578139in}}{\pgfqpoint{8.635150in}{4.567540in}}{\pgfqpoint{8.635150in}{4.556490in}}%
\pgfpathcurveto{\pgfqpoint{8.635150in}{4.545440in}}{\pgfqpoint{8.639540in}{4.534840in}}{\pgfqpoint{8.647354in}{4.527027in}}%
\pgfpathcurveto{\pgfqpoint{8.655167in}{4.519213in}}{\pgfqpoint{8.665766in}{4.514823in}}{\pgfqpoint{8.676817in}{4.514823in}}%
\pgfpathclose%
\pgfusepath{stroke,fill}%
\end{pgfscope}%
\begin{pgfscope}%
\pgfpathrectangle{\pgfqpoint{7.640588in}{4.121437in}}{\pgfqpoint{5.699255in}{2.685432in}}%
\pgfusepath{clip}%
\pgfsetbuttcap%
\pgfsetroundjoin%
\definecolor{currentfill}{rgb}{0.000000,0.000000,0.000000}%
\pgfsetfillcolor{currentfill}%
\pgfsetlinewidth{1.003750pt}%
\definecolor{currentstroke}{rgb}{0.000000,0.000000,0.000000}%
\pgfsetstrokecolor{currentstroke}%
\pgfsetdash{}{0pt}%
\pgfpathmoveto{\pgfqpoint{8.806345in}{4.640018in}}%
\pgfpathcurveto{\pgfqpoint{8.817395in}{4.640018in}}{\pgfqpoint{8.827994in}{4.644408in}}{\pgfqpoint{8.835808in}{4.652222in}}%
\pgfpathcurveto{\pgfqpoint{8.843622in}{4.660035in}}{\pgfqpoint{8.848012in}{4.670634in}}{\pgfqpoint{8.848012in}{4.681685in}}%
\pgfpathcurveto{\pgfqpoint{8.848012in}{4.692735in}}{\pgfqpoint{8.843622in}{4.703334in}}{\pgfqpoint{8.835808in}{4.711147in}}%
\pgfpathcurveto{\pgfqpoint{8.827994in}{4.718961in}}{\pgfqpoint{8.817395in}{4.723351in}}{\pgfqpoint{8.806345in}{4.723351in}}%
\pgfpathcurveto{\pgfqpoint{8.795295in}{4.723351in}}{\pgfqpoint{8.784696in}{4.718961in}}{\pgfqpoint{8.776882in}{4.711147in}}%
\pgfpathcurveto{\pgfqpoint{8.769069in}{4.703334in}}{\pgfqpoint{8.764678in}{4.692735in}}{\pgfqpoint{8.764678in}{4.681685in}}%
\pgfpathcurveto{\pgfqpoint{8.764678in}{4.670634in}}{\pgfqpoint{8.769069in}{4.660035in}}{\pgfqpoint{8.776882in}{4.652222in}}%
\pgfpathcurveto{\pgfqpoint{8.784696in}{4.644408in}}{\pgfqpoint{8.795295in}{4.640018in}}{\pgfqpoint{8.806345in}{4.640018in}}%
\pgfpathclose%
\pgfusepath{stroke,fill}%
\end{pgfscope}%
\begin{pgfscope}%
\pgfpathrectangle{\pgfqpoint{7.640588in}{4.121437in}}{\pgfqpoint{5.699255in}{2.685432in}}%
\pgfusepath{clip}%
\pgfsetbuttcap%
\pgfsetroundjoin%
\definecolor{currentfill}{rgb}{0.000000,0.000000,0.000000}%
\pgfsetfillcolor{currentfill}%
\pgfsetlinewidth{1.003750pt}%
\definecolor{currentstroke}{rgb}{0.000000,0.000000,0.000000}%
\pgfsetstrokecolor{currentstroke}%
\pgfsetdash{}{0pt}%
\pgfpathmoveto{\pgfqpoint{8.547288in}{4.608719in}}%
\pgfpathcurveto{\pgfqpoint{8.558338in}{4.608719in}}{\pgfqpoint{8.568937in}{4.613109in}}{\pgfqpoint{8.576751in}{4.620923in}}%
\pgfpathcurveto{\pgfqpoint{8.584564in}{4.628737in}}{\pgfqpoint{8.588955in}{4.639336in}}{\pgfqpoint{8.588955in}{4.650386in}}%
\pgfpathcurveto{\pgfqpoint{8.588955in}{4.661436in}}{\pgfqpoint{8.584564in}{4.672035in}}{\pgfqpoint{8.576751in}{4.679849in}}%
\pgfpathcurveto{\pgfqpoint{8.568937in}{4.687662in}}{\pgfqpoint{8.558338in}{4.692053in}}{\pgfqpoint{8.547288in}{4.692053in}}%
\pgfpathcurveto{\pgfqpoint{8.536238in}{4.692053in}}{\pgfqpoint{8.525639in}{4.687662in}}{\pgfqpoint{8.517825in}{4.679849in}}%
\pgfpathcurveto{\pgfqpoint{8.510012in}{4.672035in}}{\pgfqpoint{8.505621in}{4.661436in}}{\pgfqpoint{8.505621in}{4.650386in}}%
\pgfpathcurveto{\pgfqpoint{8.505621in}{4.639336in}}{\pgfqpoint{8.510012in}{4.628737in}}{\pgfqpoint{8.517825in}{4.620923in}}%
\pgfpathcurveto{\pgfqpoint{8.525639in}{4.613109in}}{\pgfqpoint{8.536238in}{4.608719in}}{\pgfqpoint{8.547288in}{4.608719in}}%
\pgfpathclose%
\pgfusepath{stroke,fill}%
\end{pgfscope}%
\begin{pgfscope}%
\pgfpathrectangle{\pgfqpoint{7.640588in}{4.121437in}}{\pgfqpoint{5.699255in}{2.685432in}}%
\pgfusepath{clip}%
\pgfsetbuttcap%
\pgfsetroundjoin%
\definecolor{currentfill}{rgb}{0.000000,0.000000,0.000000}%
\pgfsetfillcolor{currentfill}%
\pgfsetlinewidth{1.003750pt}%
\definecolor{currentstroke}{rgb}{0.000000,0.000000,0.000000}%
\pgfsetstrokecolor{currentstroke}%
\pgfsetdash{}{0pt}%
\pgfpathmoveto{\pgfqpoint{8.417760in}{4.577420in}}%
\pgfpathcurveto{\pgfqpoint{8.428810in}{4.577420in}}{\pgfqpoint{8.439409in}{4.581811in}}{\pgfqpoint{8.447222in}{4.589624in}}%
\pgfpathcurveto{\pgfqpoint{8.455036in}{4.597438in}}{\pgfqpoint{8.459426in}{4.608037in}}{\pgfqpoint{8.459426in}{4.619087in}}%
\pgfpathcurveto{\pgfqpoint{8.459426in}{4.630137in}}{\pgfqpoint{8.455036in}{4.640736in}}{\pgfqpoint{8.447222in}{4.648550in}}%
\pgfpathcurveto{\pgfqpoint{8.439409in}{4.656364in}}{\pgfqpoint{8.428810in}{4.660754in}}{\pgfqpoint{8.417760in}{4.660754in}}%
\pgfpathcurveto{\pgfqpoint{8.406709in}{4.660754in}}{\pgfqpoint{8.396110in}{4.656364in}}{\pgfqpoint{8.388297in}{4.648550in}}%
\pgfpathcurveto{\pgfqpoint{8.380483in}{4.640736in}}{\pgfqpoint{8.376093in}{4.630137in}}{\pgfqpoint{8.376093in}{4.619087in}}%
\pgfpathcurveto{\pgfqpoint{8.376093in}{4.608037in}}{\pgfqpoint{8.380483in}{4.597438in}}{\pgfqpoint{8.388297in}{4.589624in}}%
\pgfpathcurveto{\pgfqpoint{8.396110in}{4.581811in}}{\pgfqpoint{8.406709in}{4.577420in}}{\pgfqpoint{8.417760in}{4.577420in}}%
\pgfpathclose%
\pgfusepath{stroke,fill}%
\end{pgfscope}%
\begin{pgfscope}%
\pgfpathrectangle{\pgfqpoint{7.640588in}{4.121437in}}{\pgfqpoint{5.699255in}{2.685432in}}%
\pgfusepath{clip}%
\pgfsetbuttcap%
\pgfsetroundjoin%
\definecolor{currentfill}{rgb}{0.000000,0.000000,0.000000}%
\pgfsetfillcolor{currentfill}%
\pgfsetlinewidth{1.003750pt}%
\definecolor{currentstroke}{rgb}{0.000000,0.000000,0.000000}%
\pgfsetstrokecolor{currentstroke}%
\pgfsetdash{}{0pt}%
\pgfpathmoveto{\pgfqpoint{8.320613in}{4.452225in}}%
\pgfpathcurveto{\pgfqpoint{8.331663in}{4.452225in}}{\pgfqpoint{8.342262in}{4.456616in}}{\pgfqpoint{8.350076in}{4.464429in}}%
\pgfpathcurveto{\pgfqpoint{8.357890in}{4.472243in}}{\pgfqpoint{8.362280in}{4.482842in}}{\pgfqpoint{8.362280in}{4.493892in}}%
\pgfpathcurveto{\pgfqpoint{8.362280in}{4.504942in}}{\pgfqpoint{8.357890in}{4.515541in}}{\pgfqpoint{8.350076in}{4.523355in}}%
\pgfpathcurveto{\pgfqpoint{8.342262in}{4.531169in}}{\pgfqpoint{8.331663in}{4.535559in}}{\pgfqpoint{8.320613in}{4.535559in}}%
\pgfpathcurveto{\pgfqpoint{8.309563in}{4.535559in}}{\pgfqpoint{8.298964in}{4.531169in}}{\pgfqpoint{8.291150in}{4.523355in}}%
\pgfpathcurveto{\pgfqpoint{8.283337in}{4.515541in}}{\pgfqpoint{8.278946in}{4.504942in}}{\pgfqpoint{8.278946in}{4.493892in}}%
\pgfpathcurveto{\pgfqpoint{8.278946in}{4.482842in}}{\pgfqpoint{8.283337in}{4.472243in}}{\pgfqpoint{8.291150in}{4.464429in}}%
\pgfpathcurveto{\pgfqpoint{8.298964in}{4.456616in}}{\pgfqpoint{8.309563in}{4.452225in}}{\pgfqpoint{8.320613in}{4.452225in}}%
\pgfpathclose%
\pgfusepath{stroke,fill}%
\end{pgfscope}%
\begin{pgfscope}%
\pgfpathrectangle{\pgfqpoint{7.640588in}{4.121437in}}{\pgfqpoint{5.699255in}{2.685432in}}%
\pgfusepath{clip}%
\pgfsetbuttcap%
\pgfsetroundjoin%
\definecolor{currentfill}{rgb}{0.000000,0.000000,0.000000}%
\pgfsetfillcolor{currentfill}%
\pgfsetlinewidth{1.003750pt}%
\definecolor{currentstroke}{rgb}{0.000000,0.000000,0.000000}%
\pgfsetstrokecolor{currentstroke}%
\pgfsetdash{}{0pt}%
\pgfpathmoveto{\pgfqpoint{8.126320in}{4.295732in}}%
\pgfpathcurveto{\pgfqpoint{8.137370in}{4.295732in}}{\pgfqpoint{8.147970in}{4.300122in}}{\pgfqpoint{8.155783in}{4.307936in}}%
\pgfpathcurveto{\pgfqpoint{8.163597in}{4.315749in}}{\pgfqpoint{8.167987in}{4.326348in}}{\pgfqpoint{8.167987in}{4.337398in}}%
\pgfpathcurveto{\pgfqpoint{8.167987in}{4.348449in}}{\pgfqpoint{8.163597in}{4.359048in}}{\pgfqpoint{8.155783in}{4.366861in}}%
\pgfpathcurveto{\pgfqpoint{8.147970in}{4.374675in}}{\pgfqpoint{8.137370in}{4.379065in}}{\pgfqpoint{8.126320in}{4.379065in}}%
\pgfpathcurveto{\pgfqpoint{8.115270in}{4.379065in}}{\pgfqpoint{8.104671in}{4.374675in}}{\pgfqpoint{8.096858in}{4.366861in}}%
\pgfpathcurveto{\pgfqpoint{8.089044in}{4.359048in}}{\pgfqpoint{8.084654in}{4.348449in}}{\pgfqpoint{8.084654in}{4.337398in}}%
\pgfpathcurveto{\pgfqpoint{8.084654in}{4.326348in}}{\pgfqpoint{8.089044in}{4.315749in}}{\pgfqpoint{8.096858in}{4.307936in}}%
\pgfpathcurveto{\pgfqpoint{8.104671in}{4.300122in}}{\pgfqpoint{8.115270in}{4.295732in}}{\pgfqpoint{8.126320in}{4.295732in}}%
\pgfpathclose%
\pgfusepath{stroke,fill}%
\end{pgfscope}%
\begin{pgfscope}%
\pgfpathrectangle{\pgfqpoint{7.640588in}{4.121437in}}{\pgfqpoint{5.699255in}{2.685432in}}%
\pgfusepath{clip}%
\pgfsetbuttcap%
\pgfsetroundjoin%
\definecolor{currentfill}{rgb}{0.000000,0.000000,0.000000}%
\pgfsetfillcolor{currentfill}%
\pgfsetlinewidth{1.003750pt}%
\definecolor{currentstroke}{rgb}{0.000000,0.000000,0.000000}%
\pgfsetstrokecolor{currentstroke}%
\pgfsetdash{}{0pt}%
\pgfpathmoveto{\pgfqpoint{7.899645in}{4.593070in}}%
\pgfpathcurveto{\pgfqpoint{7.910696in}{4.593070in}}{\pgfqpoint{7.921295in}{4.597460in}}{\pgfqpoint{7.929108in}{4.605274in}}%
\pgfpathcurveto{\pgfqpoint{7.936922in}{4.613087in}}{\pgfqpoint{7.941312in}{4.623686in}}{\pgfqpoint{7.941312in}{4.634736in}}%
\pgfpathcurveto{\pgfqpoint{7.941312in}{4.645787in}}{\pgfqpoint{7.936922in}{4.656386in}}{\pgfqpoint{7.929108in}{4.664199in}}%
\pgfpathcurveto{\pgfqpoint{7.921295in}{4.672013in}}{\pgfqpoint{7.910696in}{4.676403in}}{\pgfqpoint{7.899645in}{4.676403in}}%
\pgfpathcurveto{\pgfqpoint{7.888595in}{4.676403in}}{\pgfqpoint{7.877996in}{4.672013in}}{\pgfqpoint{7.870183in}{4.664199in}}%
\pgfpathcurveto{\pgfqpoint{7.862369in}{4.656386in}}{\pgfqpoint{7.857979in}{4.645787in}}{\pgfqpoint{7.857979in}{4.634736in}}%
\pgfpathcurveto{\pgfqpoint{7.857979in}{4.623686in}}{\pgfqpoint{7.862369in}{4.613087in}}{\pgfqpoint{7.870183in}{4.605274in}}%
\pgfpathcurveto{\pgfqpoint{7.877996in}{4.597460in}}{\pgfqpoint{7.888595in}{4.593070in}}{\pgfqpoint{7.899645in}{4.593070in}}%
\pgfpathclose%
\pgfusepath{stroke,fill}%
\end{pgfscope}%
\begin{pgfscope}%
\pgfpathrectangle{\pgfqpoint{7.640588in}{4.121437in}}{\pgfqpoint{5.699255in}{2.685432in}}%
\pgfusepath{clip}%
\pgfsetbuttcap%
\pgfsetroundjoin%
\definecolor{currentfill}{rgb}{0.000000,0.000000,0.000000}%
\pgfsetfillcolor{currentfill}%
\pgfsetlinewidth{1.003750pt}%
\definecolor{currentstroke}{rgb}{0.000000,0.000000,0.000000}%
\pgfsetstrokecolor{currentstroke}%
\pgfsetdash{}{0pt}%
\pgfpathmoveto{\pgfqpoint{8.158702in}{4.671317in}}%
\pgfpathcurveto{\pgfqpoint{8.169753in}{4.671317in}}{\pgfqpoint{8.180352in}{4.675707in}}{\pgfqpoint{8.188165in}{4.683521in}}%
\pgfpathcurveto{\pgfqpoint{8.195979in}{4.691334in}}{\pgfqpoint{8.200369in}{4.701933in}}{\pgfqpoint{8.200369in}{4.712983in}}%
\pgfpathcurveto{\pgfqpoint{8.200369in}{4.724033in}}{\pgfqpoint{8.195979in}{4.734633in}}{\pgfqpoint{8.188165in}{4.742446in}}%
\pgfpathcurveto{\pgfqpoint{8.180352in}{4.750260in}}{\pgfqpoint{8.169753in}{4.754650in}}{\pgfqpoint{8.158702in}{4.754650in}}%
\pgfpathcurveto{\pgfqpoint{8.147652in}{4.754650in}}{\pgfqpoint{8.137053in}{4.750260in}}{\pgfqpoint{8.129240in}{4.742446in}}%
\pgfpathcurveto{\pgfqpoint{8.121426in}{4.734633in}}{\pgfqpoint{8.117036in}{4.724033in}}{\pgfqpoint{8.117036in}{4.712983in}}%
\pgfpathcurveto{\pgfqpoint{8.117036in}{4.701933in}}{\pgfqpoint{8.121426in}{4.691334in}}{\pgfqpoint{8.129240in}{4.683521in}}%
\pgfpathcurveto{\pgfqpoint{8.137053in}{4.675707in}}{\pgfqpoint{8.147652in}{4.671317in}}{\pgfqpoint{8.158702in}{4.671317in}}%
\pgfpathclose%
\pgfusepath{stroke,fill}%
\end{pgfscope}%
\begin{pgfscope}%
\pgfpathrectangle{\pgfqpoint{7.640588in}{4.121437in}}{\pgfqpoint{5.699255in}{2.685432in}}%
\pgfusepath{clip}%
\pgfsetbuttcap%
\pgfsetroundjoin%
\definecolor{currentfill}{rgb}{0.000000,0.000000,0.000000}%
\pgfsetfillcolor{currentfill}%
\pgfsetlinewidth{1.003750pt}%
\definecolor{currentstroke}{rgb}{0.000000,0.000000,0.000000}%
\pgfsetstrokecolor{currentstroke}%
\pgfsetdash{}{0pt}%
\pgfpathmoveto{\pgfqpoint{7.932028in}{5.109499in}}%
\pgfpathcurveto{\pgfqpoint{7.943078in}{5.109499in}}{\pgfqpoint{7.953677in}{5.113889in}}{\pgfqpoint{7.961490in}{5.121703in}}%
\pgfpathcurveto{\pgfqpoint{7.969304in}{5.129517in}}{\pgfqpoint{7.973694in}{5.140116in}}{\pgfqpoint{7.973694in}{5.151166in}}%
\pgfpathcurveto{\pgfqpoint{7.973694in}{5.162216in}}{\pgfqpoint{7.969304in}{5.172815in}}{\pgfqpoint{7.961490in}{5.180629in}}%
\pgfpathcurveto{\pgfqpoint{7.953677in}{5.188442in}}{\pgfqpoint{7.943078in}{5.192832in}}{\pgfqpoint{7.932028in}{5.192832in}}%
\pgfpathcurveto{\pgfqpoint{7.920977in}{5.192832in}}{\pgfqpoint{7.910378in}{5.188442in}}{\pgfqpoint{7.902565in}{5.180629in}}%
\pgfpathcurveto{\pgfqpoint{7.894751in}{5.172815in}}{\pgfqpoint{7.890361in}{5.162216in}}{\pgfqpoint{7.890361in}{5.151166in}}%
\pgfpathcurveto{\pgfqpoint{7.890361in}{5.140116in}}{\pgfqpoint{7.894751in}{5.129517in}}{\pgfqpoint{7.902565in}{5.121703in}}%
\pgfpathcurveto{\pgfqpoint{7.910378in}{5.113889in}}{\pgfqpoint{7.920977in}{5.109499in}}{\pgfqpoint{7.932028in}{5.109499in}}%
\pgfpathclose%
\pgfusepath{stroke,fill}%
\end{pgfscope}%
\begin{pgfscope}%
\pgfpathrectangle{\pgfqpoint{7.640588in}{4.121437in}}{\pgfqpoint{5.699255in}{2.685432in}}%
\pgfusepath{clip}%
\pgfsetbuttcap%
\pgfsetroundjoin%
\definecolor{currentfill}{rgb}{0.000000,0.000000,0.000000}%
\pgfsetfillcolor{currentfill}%
\pgfsetlinewidth{1.003750pt}%
\definecolor{currentstroke}{rgb}{0.000000,0.000000,0.000000}%
\pgfsetstrokecolor{currentstroke}%
\pgfsetdash{}{0pt}%
\pgfpathmoveto{\pgfqpoint{9.324459in}{5.297292in}}%
\pgfpathcurveto{\pgfqpoint{9.335509in}{5.297292in}}{\pgfqpoint{9.346108in}{5.301682in}}{\pgfqpoint{9.353922in}{5.309495in}}%
\pgfpathcurveto{\pgfqpoint{9.361736in}{5.317309in}}{\pgfqpoint{9.366126in}{5.327908in}}{\pgfqpoint{9.366126in}{5.338958in}}%
\pgfpathcurveto{\pgfqpoint{9.366126in}{5.350008in}}{\pgfqpoint{9.361736in}{5.360607in}}{\pgfqpoint{9.353922in}{5.368421in}}%
\pgfpathcurveto{\pgfqpoint{9.346108in}{5.376235in}}{\pgfqpoint{9.335509in}{5.380625in}}{\pgfqpoint{9.324459in}{5.380625in}}%
\pgfpathcurveto{\pgfqpoint{9.313409in}{5.380625in}}{\pgfqpoint{9.302810in}{5.376235in}}{\pgfqpoint{9.294996in}{5.368421in}}%
\pgfpathcurveto{\pgfqpoint{9.287183in}{5.360607in}}{\pgfqpoint{9.282793in}{5.350008in}}{\pgfqpoint{9.282793in}{5.338958in}}%
\pgfpathcurveto{\pgfqpoint{9.282793in}{5.327908in}}{\pgfqpoint{9.287183in}{5.317309in}}{\pgfqpoint{9.294996in}{5.309495in}}%
\pgfpathcurveto{\pgfqpoint{9.302810in}{5.301682in}}{\pgfqpoint{9.313409in}{5.297292in}}{\pgfqpoint{9.324459in}{5.297292in}}%
\pgfpathclose%
\pgfusepath{stroke,fill}%
\end{pgfscope}%
\begin{pgfscope}%
\pgfpathrectangle{\pgfqpoint{7.640588in}{4.121437in}}{\pgfqpoint{5.699255in}{2.685432in}}%
\pgfusepath{clip}%
\pgfsetbuttcap%
\pgfsetroundjoin%
\definecolor{currentfill}{rgb}{0.000000,0.000000,0.000000}%
\pgfsetfillcolor{currentfill}%
\pgfsetlinewidth{1.003750pt}%
\definecolor{currentstroke}{rgb}{0.000000,0.000000,0.000000}%
\pgfsetstrokecolor{currentstroke}%
\pgfsetdash{}{0pt}%
\pgfpathmoveto{\pgfqpoint{9.810191in}{5.391188in}}%
\pgfpathcurveto{\pgfqpoint{9.821241in}{5.391188in}}{\pgfqpoint{9.831840in}{5.395578in}}{\pgfqpoint{9.839654in}{5.403392in}}%
\pgfpathcurveto{\pgfqpoint{9.847468in}{5.411205in}}{\pgfqpoint{9.851858in}{5.421804in}}{\pgfqpoint{9.851858in}{5.432854in}}%
\pgfpathcurveto{\pgfqpoint{9.851858in}{5.443905in}}{\pgfqpoint{9.847468in}{5.454504in}}{\pgfqpoint{9.839654in}{5.462317in}}%
\pgfpathcurveto{\pgfqpoint{9.831840in}{5.470131in}}{\pgfqpoint{9.821241in}{5.474521in}}{\pgfqpoint{9.810191in}{5.474521in}}%
\pgfpathcurveto{\pgfqpoint{9.799141in}{5.474521in}}{\pgfqpoint{9.788542in}{5.470131in}}{\pgfqpoint{9.780728in}{5.462317in}}%
\pgfpathcurveto{\pgfqpoint{9.772915in}{5.454504in}}{\pgfqpoint{9.768525in}{5.443905in}}{\pgfqpoint{9.768525in}{5.432854in}}%
\pgfpathcurveto{\pgfqpoint{9.768525in}{5.421804in}}{\pgfqpoint{9.772915in}{5.411205in}}{\pgfqpoint{9.780728in}{5.403392in}}%
\pgfpathcurveto{\pgfqpoint{9.788542in}{5.395578in}}{\pgfqpoint{9.799141in}{5.391188in}}{\pgfqpoint{9.810191in}{5.391188in}}%
\pgfpathclose%
\pgfusepath{stroke,fill}%
\end{pgfscope}%
\begin{pgfscope}%
\pgfpathrectangle{\pgfqpoint{7.640588in}{4.121437in}}{\pgfqpoint{5.699255in}{2.685432in}}%
\pgfusepath{clip}%
\pgfsetbuttcap%
\pgfsetroundjoin%
\definecolor{currentfill}{rgb}{0.000000,0.000000,0.000000}%
\pgfsetfillcolor{currentfill}%
\pgfsetlinewidth{1.003750pt}%
\definecolor{currentstroke}{rgb}{0.000000,0.000000,0.000000}%
\pgfsetstrokecolor{currentstroke}%
\pgfsetdash{}{0pt}%
\pgfpathmoveto{\pgfqpoint{9.615898in}{5.391188in}}%
\pgfpathcurveto{\pgfqpoint{9.626949in}{5.391188in}}{\pgfqpoint{9.637548in}{5.395578in}}{\pgfqpoint{9.645361in}{5.403392in}}%
\pgfpathcurveto{\pgfqpoint{9.653175in}{5.411205in}}{\pgfqpoint{9.657565in}{5.421804in}}{\pgfqpoint{9.657565in}{5.432854in}}%
\pgfpathcurveto{\pgfqpoint{9.657565in}{5.443905in}}{\pgfqpoint{9.653175in}{5.454504in}}{\pgfqpoint{9.645361in}{5.462317in}}%
\pgfpathcurveto{\pgfqpoint{9.637548in}{5.470131in}}{\pgfqpoint{9.626949in}{5.474521in}}{\pgfqpoint{9.615898in}{5.474521in}}%
\pgfpathcurveto{\pgfqpoint{9.604848in}{5.474521in}}{\pgfqpoint{9.594249in}{5.470131in}}{\pgfqpoint{9.586436in}{5.462317in}}%
\pgfpathcurveto{\pgfqpoint{9.578622in}{5.454504in}}{\pgfqpoint{9.574232in}{5.443905in}}{\pgfqpoint{9.574232in}{5.432854in}}%
\pgfpathcurveto{\pgfqpoint{9.574232in}{5.421804in}}{\pgfqpoint{9.578622in}{5.411205in}}{\pgfqpoint{9.586436in}{5.403392in}}%
\pgfpathcurveto{\pgfqpoint{9.594249in}{5.395578in}}{\pgfqpoint{9.604848in}{5.391188in}}{\pgfqpoint{9.615898in}{5.391188in}}%
\pgfpathclose%
\pgfusepath{stroke,fill}%
\end{pgfscope}%
\begin{pgfscope}%
\pgfpathrectangle{\pgfqpoint{7.640588in}{4.121437in}}{\pgfqpoint{5.699255in}{2.685432in}}%
\pgfusepath{clip}%
\pgfsetbuttcap%
\pgfsetroundjoin%
\definecolor{currentfill}{rgb}{0.000000,0.000000,0.000000}%
\pgfsetfillcolor{currentfill}%
\pgfsetlinewidth{1.003750pt}%
\definecolor{currentstroke}{rgb}{0.000000,0.000000,0.000000}%
\pgfsetstrokecolor{currentstroke}%
\pgfsetdash{}{0pt}%
\pgfpathmoveto{\pgfqpoint{9.227313in}{5.438136in}}%
\pgfpathcurveto{\pgfqpoint{9.238363in}{5.438136in}}{\pgfqpoint{9.248962in}{5.442526in}}{\pgfqpoint{9.256776in}{5.450340in}}%
\pgfpathcurveto{\pgfqpoint{9.264589in}{5.458153in}}{\pgfqpoint{9.268980in}{5.468752in}}{\pgfqpoint{9.268980in}{5.479803in}}%
\pgfpathcurveto{\pgfqpoint{9.268980in}{5.490853in}}{\pgfqpoint{9.264589in}{5.501452in}}{\pgfqpoint{9.256776in}{5.509265in}}%
\pgfpathcurveto{\pgfqpoint{9.248962in}{5.517079in}}{\pgfqpoint{9.238363in}{5.521469in}}{\pgfqpoint{9.227313in}{5.521469in}}%
\pgfpathcurveto{\pgfqpoint{9.216263in}{5.521469in}}{\pgfqpoint{9.205664in}{5.517079in}}{\pgfqpoint{9.197850in}{5.509265in}}%
\pgfpathcurveto{\pgfqpoint{9.190036in}{5.501452in}}{\pgfqpoint{9.185646in}{5.490853in}}{\pgfqpoint{9.185646in}{5.479803in}}%
\pgfpathcurveto{\pgfqpoint{9.185646in}{5.468752in}}{\pgfqpoint{9.190036in}{5.458153in}}{\pgfqpoint{9.197850in}{5.450340in}}%
\pgfpathcurveto{\pgfqpoint{9.205664in}{5.442526in}}{\pgfqpoint{9.216263in}{5.438136in}}{\pgfqpoint{9.227313in}{5.438136in}}%
\pgfpathclose%
\pgfusepath{stroke,fill}%
\end{pgfscope}%
\begin{pgfscope}%
\pgfpathrectangle{\pgfqpoint{7.640588in}{4.121437in}}{\pgfqpoint{5.699255in}{2.685432in}}%
\pgfusepath{clip}%
\pgfsetbuttcap%
\pgfsetroundjoin%
\definecolor{currentfill}{rgb}{0.000000,0.000000,0.000000}%
\pgfsetfillcolor{currentfill}%
\pgfsetlinewidth{1.003750pt}%
\definecolor{currentstroke}{rgb}{0.000000,0.000000,0.000000}%
\pgfsetstrokecolor{currentstroke}%
\pgfsetdash{}{0pt}%
\pgfpathmoveto{\pgfqpoint{8.871109in}{5.453785in}}%
\pgfpathcurveto{\pgfqpoint{8.882160in}{5.453785in}}{\pgfqpoint{8.892759in}{5.458176in}}{\pgfqpoint{8.900572in}{5.465989in}}%
\pgfpathcurveto{\pgfqpoint{8.908386in}{5.473803in}}{\pgfqpoint{8.912776in}{5.484402in}}{\pgfqpoint{8.912776in}{5.495452in}}%
\pgfpathcurveto{\pgfqpoint{8.912776in}{5.506502in}}{\pgfqpoint{8.908386in}{5.517101in}}{\pgfqpoint{8.900572in}{5.524915in}}%
\pgfpathcurveto{\pgfqpoint{8.892759in}{5.532728in}}{\pgfqpoint{8.882160in}{5.537119in}}{\pgfqpoint{8.871109in}{5.537119in}}%
\pgfpathcurveto{\pgfqpoint{8.860059in}{5.537119in}}{\pgfqpoint{8.849460in}{5.532728in}}{\pgfqpoint{8.841647in}{5.524915in}}%
\pgfpathcurveto{\pgfqpoint{8.833833in}{5.517101in}}{\pgfqpoint{8.829443in}{5.506502in}}{\pgfqpoint{8.829443in}{5.495452in}}%
\pgfpathcurveto{\pgfqpoint{8.829443in}{5.484402in}}{\pgfqpoint{8.833833in}{5.473803in}}{\pgfqpoint{8.841647in}{5.465989in}}%
\pgfpathcurveto{\pgfqpoint{8.849460in}{5.458176in}}{\pgfqpoint{8.860059in}{5.453785in}}{\pgfqpoint{8.871109in}{5.453785in}}%
\pgfpathclose%
\pgfusepath{stroke,fill}%
\end{pgfscope}%
\begin{pgfscope}%
\pgfpathrectangle{\pgfqpoint{7.640588in}{4.121437in}}{\pgfqpoint{5.699255in}{2.685432in}}%
\pgfusepath{clip}%
\pgfsetbuttcap%
\pgfsetroundjoin%
\definecolor{currentfill}{rgb}{0.000000,0.000000,0.000000}%
\pgfsetfillcolor{currentfill}%
\pgfsetlinewidth{1.003750pt}%
\definecolor{currentstroke}{rgb}{0.000000,0.000000,0.000000}%
\pgfsetstrokecolor{currentstroke}%
\pgfsetdash{}{0pt}%
\pgfpathmoveto{\pgfqpoint{8.871109in}{5.610279in}}%
\pgfpathcurveto{\pgfqpoint{8.882160in}{5.610279in}}{\pgfqpoint{8.892759in}{5.614669in}}{\pgfqpoint{8.900572in}{5.622483in}}%
\pgfpathcurveto{\pgfqpoint{8.908386in}{5.630296in}}{\pgfqpoint{8.912776in}{5.640895in}}{\pgfqpoint{8.912776in}{5.651946in}}%
\pgfpathcurveto{\pgfqpoint{8.912776in}{5.662996in}}{\pgfqpoint{8.908386in}{5.673595in}}{\pgfqpoint{8.900572in}{5.681408in}}%
\pgfpathcurveto{\pgfqpoint{8.892759in}{5.689222in}}{\pgfqpoint{8.882160in}{5.693612in}}{\pgfqpoint{8.871109in}{5.693612in}}%
\pgfpathcurveto{\pgfqpoint{8.860059in}{5.693612in}}{\pgfqpoint{8.849460in}{5.689222in}}{\pgfqpoint{8.841647in}{5.681408in}}%
\pgfpathcurveto{\pgfqpoint{8.833833in}{5.673595in}}{\pgfqpoint{8.829443in}{5.662996in}}{\pgfqpoint{8.829443in}{5.651946in}}%
\pgfpathcurveto{\pgfqpoint{8.829443in}{5.640895in}}{\pgfqpoint{8.833833in}{5.630296in}}{\pgfqpoint{8.841647in}{5.622483in}}%
\pgfpathcurveto{\pgfqpoint{8.849460in}{5.614669in}}{\pgfqpoint{8.860059in}{5.610279in}}{\pgfqpoint{8.871109in}{5.610279in}}%
\pgfpathclose%
\pgfusepath{stroke,fill}%
\end{pgfscope}%
\begin{pgfscope}%
\pgfpathrectangle{\pgfqpoint{7.640588in}{4.121437in}}{\pgfqpoint{5.699255in}{2.685432in}}%
\pgfusepath{clip}%
\pgfsetbuttcap%
\pgfsetroundjoin%
\definecolor{currentfill}{rgb}{0.000000,0.000000,0.000000}%
\pgfsetfillcolor{currentfill}%
\pgfsetlinewidth{1.003750pt}%
\definecolor{currentstroke}{rgb}{0.000000,0.000000,0.000000}%
\pgfsetstrokecolor{currentstroke}%
\pgfsetdash{}{0pt}%
\pgfpathmoveto{\pgfqpoint{9.033020in}{5.672876in}}%
\pgfpathcurveto{\pgfqpoint{9.044070in}{5.672876in}}{\pgfqpoint{9.054669in}{5.677267in}}{\pgfqpoint{9.062483in}{5.685080in}}%
\pgfpathcurveto{\pgfqpoint{9.070296in}{5.692894in}}{\pgfqpoint{9.074687in}{5.703493in}}{\pgfqpoint{9.074687in}{5.714543in}}%
\pgfpathcurveto{\pgfqpoint{9.074687in}{5.725593in}}{\pgfqpoint{9.070296in}{5.736192in}}{\pgfqpoint{9.062483in}{5.744006in}}%
\pgfpathcurveto{\pgfqpoint{9.054669in}{5.751820in}}{\pgfqpoint{9.044070in}{5.756210in}}{\pgfqpoint{9.033020in}{5.756210in}}%
\pgfpathcurveto{\pgfqpoint{9.021970in}{5.756210in}}{\pgfqpoint{9.011371in}{5.751820in}}{\pgfqpoint{9.003557in}{5.744006in}}%
\pgfpathcurveto{\pgfqpoint{8.995744in}{5.736192in}}{\pgfqpoint{8.991353in}{5.725593in}}{\pgfqpoint{8.991353in}{5.714543in}}%
\pgfpathcurveto{\pgfqpoint{8.991353in}{5.703493in}}{\pgfqpoint{8.995744in}{5.692894in}}{\pgfqpoint{9.003557in}{5.685080in}}%
\pgfpathcurveto{\pgfqpoint{9.011371in}{5.677267in}}{\pgfqpoint{9.021970in}{5.672876in}}{\pgfqpoint{9.033020in}{5.672876in}}%
\pgfpathclose%
\pgfusepath{stroke,fill}%
\end{pgfscope}%
\begin{pgfscope}%
\pgfpathrectangle{\pgfqpoint{7.640588in}{4.121437in}}{\pgfqpoint{5.699255in}{2.685432in}}%
\pgfusepath{clip}%
\pgfsetbuttcap%
\pgfsetroundjoin%
\definecolor{currentfill}{rgb}{0.000000,0.000000,1.000000}%
\pgfsetfillcolor{currentfill}%
\pgfsetlinewidth{1.003750pt}%
\definecolor{currentstroke}{rgb}{0.000000,0.000000,1.000000}%
\pgfsetstrokecolor{currentstroke}%
\pgfsetdash{}{0pt}%
\pgfpathmoveto{\pgfqpoint{8.126320in}{5.876318in}}%
\pgfpathcurveto{\pgfqpoint{8.137370in}{5.876318in}}{\pgfqpoint{8.147970in}{5.880709in}}{\pgfqpoint{8.155783in}{5.888522in}}%
\pgfpathcurveto{\pgfqpoint{8.163597in}{5.896336in}}{\pgfqpoint{8.167987in}{5.906935in}}{\pgfqpoint{8.167987in}{5.917985in}}%
\pgfpathcurveto{\pgfqpoint{8.167987in}{5.929035in}}{\pgfqpoint{8.163597in}{5.939634in}}{\pgfqpoint{8.155783in}{5.947448in}}%
\pgfpathcurveto{\pgfqpoint{8.147970in}{5.955261in}}{\pgfqpoint{8.137370in}{5.959652in}}{\pgfqpoint{8.126320in}{5.959652in}}%
\pgfpathcurveto{\pgfqpoint{8.115270in}{5.959652in}}{\pgfqpoint{8.104671in}{5.955261in}}{\pgfqpoint{8.096858in}{5.947448in}}%
\pgfpathcurveto{\pgfqpoint{8.089044in}{5.939634in}}{\pgfqpoint{8.084654in}{5.929035in}}{\pgfqpoint{8.084654in}{5.917985in}}%
\pgfpathcurveto{\pgfqpoint{8.084654in}{5.906935in}}{\pgfqpoint{8.089044in}{5.896336in}}{\pgfqpoint{8.096858in}{5.888522in}}%
\pgfpathcurveto{\pgfqpoint{8.104671in}{5.880709in}}{\pgfqpoint{8.115270in}{5.876318in}}{\pgfqpoint{8.126320in}{5.876318in}}%
\pgfpathclose%
\pgfusepath{stroke,fill}%
\end{pgfscope}%
\begin{pgfscope}%
\pgfsetbuttcap%
\pgfsetroundjoin%
\definecolor{currentfill}{rgb}{0.000000,0.000000,0.000000}%
\pgfsetfillcolor{currentfill}%
\pgfsetlinewidth{0.803000pt}%
\definecolor{currentstroke}{rgb}{0.000000,0.000000,0.000000}%
\pgfsetstrokecolor{currentstroke}%
\pgfsetdash{}{0pt}%
\pgfsys@defobject{currentmarker}{\pgfqpoint{0.000000in}{-0.048611in}}{\pgfqpoint{0.000000in}{0.000000in}}{%
\pgfpathmoveto{\pgfqpoint{0.000000in}{0.000000in}}%
\pgfpathlineto{\pgfqpoint{0.000000in}{-0.048611in}}%
\pgfusepath{stroke,fill}%
}%
\begin{pgfscope}%
\pgfsys@transformshift{7.802499in}{4.121437in}%
\pgfsys@useobject{currentmarker}{}%
\end{pgfscope}%
\end{pgfscope}%
\begin{pgfscope}%
\definecolor{textcolor}{rgb}{0.000000,0.000000,0.000000}%
\pgfsetstrokecolor{textcolor}%
\pgfsetfillcolor{textcolor}%
\pgftext[x=7.802499in,y=4.024215in,,top]{\color{textcolor}\rmfamily\fontsize{10.000000}{12.000000}\selectfont \(\displaystyle 0\)}%
\end{pgfscope}%
\begin{pgfscope}%
\pgfsetbuttcap%
\pgfsetroundjoin%
\definecolor{currentfill}{rgb}{0.000000,0.000000,0.000000}%
\pgfsetfillcolor{currentfill}%
\pgfsetlinewidth{0.803000pt}%
\definecolor{currentstroke}{rgb}{0.000000,0.000000,0.000000}%
\pgfsetstrokecolor{currentstroke}%
\pgfsetdash{}{0pt}%
\pgfsys@defobject{currentmarker}{\pgfqpoint{0.000000in}{-0.048611in}}{\pgfqpoint{0.000000in}{0.000000in}}{%
\pgfpathmoveto{\pgfqpoint{0.000000in}{0.000000in}}%
\pgfpathlineto{\pgfqpoint{0.000000in}{-0.048611in}}%
\pgfusepath{stroke,fill}%
}%
\begin{pgfscope}%
\pgfsys@transformshift{8.450142in}{4.121437in}%
\pgfsys@useobject{currentmarker}{}%
\end{pgfscope}%
\end{pgfscope}%
\begin{pgfscope}%
\definecolor{textcolor}{rgb}{0.000000,0.000000,0.000000}%
\pgfsetstrokecolor{textcolor}%
\pgfsetfillcolor{textcolor}%
\pgftext[x=8.450142in,y=4.024215in,,top]{\color{textcolor}\rmfamily\fontsize{10.000000}{12.000000}\selectfont \(\displaystyle 20\)}%
\end{pgfscope}%
\begin{pgfscope}%
\pgfsetbuttcap%
\pgfsetroundjoin%
\definecolor{currentfill}{rgb}{0.000000,0.000000,0.000000}%
\pgfsetfillcolor{currentfill}%
\pgfsetlinewidth{0.803000pt}%
\definecolor{currentstroke}{rgb}{0.000000,0.000000,0.000000}%
\pgfsetstrokecolor{currentstroke}%
\pgfsetdash{}{0pt}%
\pgfsys@defobject{currentmarker}{\pgfqpoint{0.000000in}{-0.048611in}}{\pgfqpoint{0.000000in}{0.000000in}}{%
\pgfpathmoveto{\pgfqpoint{0.000000in}{0.000000in}}%
\pgfpathlineto{\pgfqpoint{0.000000in}{-0.048611in}}%
\pgfusepath{stroke,fill}%
}%
\begin{pgfscope}%
\pgfsys@transformshift{9.097784in}{4.121437in}%
\pgfsys@useobject{currentmarker}{}%
\end{pgfscope}%
\end{pgfscope}%
\begin{pgfscope}%
\definecolor{textcolor}{rgb}{0.000000,0.000000,0.000000}%
\pgfsetstrokecolor{textcolor}%
\pgfsetfillcolor{textcolor}%
\pgftext[x=9.097784in,y=4.024215in,,top]{\color{textcolor}\rmfamily\fontsize{10.000000}{12.000000}\selectfont \(\displaystyle 40\)}%
\end{pgfscope}%
\begin{pgfscope}%
\pgfsetbuttcap%
\pgfsetroundjoin%
\definecolor{currentfill}{rgb}{0.000000,0.000000,0.000000}%
\pgfsetfillcolor{currentfill}%
\pgfsetlinewidth{0.803000pt}%
\definecolor{currentstroke}{rgb}{0.000000,0.000000,0.000000}%
\pgfsetstrokecolor{currentstroke}%
\pgfsetdash{}{0pt}%
\pgfsys@defobject{currentmarker}{\pgfqpoint{0.000000in}{-0.048611in}}{\pgfqpoint{0.000000in}{0.000000in}}{%
\pgfpathmoveto{\pgfqpoint{0.000000in}{0.000000in}}%
\pgfpathlineto{\pgfqpoint{0.000000in}{-0.048611in}}%
\pgfusepath{stroke,fill}%
}%
\begin{pgfscope}%
\pgfsys@transformshift{9.745427in}{4.121437in}%
\pgfsys@useobject{currentmarker}{}%
\end{pgfscope}%
\end{pgfscope}%
\begin{pgfscope}%
\definecolor{textcolor}{rgb}{0.000000,0.000000,0.000000}%
\pgfsetstrokecolor{textcolor}%
\pgfsetfillcolor{textcolor}%
\pgftext[x=9.745427in,y=4.024215in,,top]{\color{textcolor}\rmfamily\fontsize{10.000000}{12.000000}\selectfont \(\displaystyle 60\)}%
\end{pgfscope}%
\begin{pgfscope}%
\pgfsetbuttcap%
\pgfsetroundjoin%
\definecolor{currentfill}{rgb}{0.000000,0.000000,0.000000}%
\pgfsetfillcolor{currentfill}%
\pgfsetlinewidth{0.803000pt}%
\definecolor{currentstroke}{rgb}{0.000000,0.000000,0.000000}%
\pgfsetstrokecolor{currentstroke}%
\pgfsetdash{}{0pt}%
\pgfsys@defobject{currentmarker}{\pgfqpoint{0.000000in}{-0.048611in}}{\pgfqpoint{0.000000in}{0.000000in}}{%
\pgfpathmoveto{\pgfqpoint{0.000000in}{0.000000in}}%
\pgfpathlineto{\pgfqpoint{0.000000in}{-0.048611in}}%
\pgfusepath{stroke,fill}%
}%
\begin{pgfscope}%
\pgfsys@transformshift{10.393070in}{4.121437in}%
\pgfsys@useobject{currentmarker}{}%
\end{pgfscope}%
\end{pgfscope}%
\begin{pgfscope}%
\definecolor{textcolor}{rgb}{0.000000,0.000000,0.000000}%
\pgfsetstrokecolor{textcolor}%
\pgfsetfillcolor{textcolor}%
\pgftext[x=10.393070in,y=4.024215in,,top]{\color{textcolor}\rmfamily\fontsize{10.000000}{12.000000}\selectfont \(\displaystyle 80\)}%
\end{pgfscope}%
\begin{pgfscope}%
\pgfsetbuttcap%
\pgfsetroundjoin%
\definecolor{currentfill}{rgb}{0.000000,0.000000,0.000000}%
\pgfsetfillcolor{currentfill}%
\pgfsetlinewidth{0.803000pt}%
\definecolor{currentstroke}{rgb}{0.000000,0.000000,0.000000}%
\pgfsetstrokecolor{currentstroke}%
\pgfsetdash{}{0pt}%
\pgfsys@defobject{currentmarker}{\pgfqpoint{0.000000in}{-0.048611in}}{\pgfqpoint{0.000000in}{0.000000in}}{%
\pgfpathmoveto{\pgfqpoint{0.000000in}{0.000000in}}%
\pgfpathlineto{\pgfqpoint{0.000000in}{-0.048611in}}%
\pgfusepath{stroke,fill}%
}%
\begin{pgfscope}%
\pgfsys@transformshift{11.040712in}{4.121437in}%
\pgfsys@useobject{currentmarker}{}%
\end{pgfscope}%
\end{pgfscope}%
\begin{pgfscope}%
\definecolor{textcolor}{rgb}{0.000000,0.000000,0.000000}%
\pgfsetstrokecolor{textcolor}%
\pgfsetfillcolor{textcolor}%
\pgftext[x=11.040712in,y=4.024215in,,top]{\color{textcolor}\rmfamily\fontsize{10.000000}{12.000000}\selectfont \(\displaystyle 100\)}%
\end{pgfscope}%
\begin{pgfscope}%
\pgfsetbuttcap%
\pgfsetroundjoin%
\definecolor{currentfill}{rgb}{0.000000,0.000000,0.000000}%
\pgfsetfillcolor{currentfill}%
\pgfsetlinewidth{0.803000pt}%
\definecolor{currentstroke}{rgb}{0.000000,0.000000,0.000000}%
\pgfsetstrokecolor{currentstroke}%
\pgfsetdash{}{0pt}%
\pgfsys@defobject{currentmarker}{\pgfqpoint{0.000000in}{-0.048611in}}{\pgfqpoint{0.000000in}{0.000000in}}{%
\pgfpathmoveto{\pgfqpoint{0.000000in}{0.000000in}}%
\pgfpathlineto{\pgfqpoint{0.000000in}{-0.048611in}}%
\pgfusepath{stroke,fill}%
}%
\begin{pgfscope}%
\pgfsys@transformshift{11.688355in}{4.121437in}%
\pgfsys@useobject{currentmarker}{}%
\end{pgfscope}%
\end{pgfscope}%
\begin{pgfscope}%
\definecolor{textcolor}{rgb}{0.000000,0.000000,0.000000}%
\pgfsetstrokecolor{textcolor}%
\pgfsetfillcolor{textcolor}%
\pgftext[x=11.688355in,y=4.024215in,,top]{\color{textcolor}\rmfamily\fontsize{10.000000}{12.000000}\selectfont \(\displaystyle 120\)}%
\end{pgfscope}%
\begin{pgfscope}%
\pgfsetbuttcap%
\pgfsetroundjoin%
\definecolor{currentfill}{rgb}{0.000000,0.000000,0.000000}%
\pgfsetfillcolor{currentfill}%
\pgfsetlinewidth{0.803000pt}%
\definecolor{currentstroke}{rgb}{0.000000,0.000000,0.000000}%
\pgfsetstrokecolor{currentstroke}%
\pgfsetdash{}{0pt}%
\pgfsys@defobject{currentmarker}{\pgfqpoint{0.000000in}{-0.048611in}}{\pgfqpoint{0.000000in}{0.000000in}}{%
\pgfpathmoveto{\pgfqpoint{0.000000in}{0.000000in}}%
\pgfpathlineto{\pgfqpoint{0.000000in}{-0.048611in}}%
\pgfusepath{stroke,fill}%
}%
\begin{pgfscope}%
\pgfsys@transformshift{12.335998in}{4.121437in}%
\pgfsys@useobject{currentmarker}{}%
\end{pgfscope}%
\end{pgfscope}%
\begin{pgfscope}%
\definecolor{textcolor}{rgb}{0.000000,0.000000,0.000000}%
\pgfsetstrokecolor{textcolor}%
\pgfsetfillcolor{textcolor}%
\pgftext[x=12.335998in,y=4.024215in,,top]{\color{textcolor}\rmfamily\fontsize{10.000000}{12.000000}\selectfont \(\displaystyle 140\)}%
\end{pgfscope}%
\begin{pgfscope}%
\pgfsetbuttcap%
\pgfsetroundjoin%
\definecolor{currentfill}{rgb}{0.000000,0.000000,0.000000}%
\pgfsetfillcolor{currentfill}%
\pgfsetlinewidth{0.803000pt}%
\definecolor{currentstroke}{rgb}{0.000000,0.000000,0.000000}%
\pgfsetstrokecolor{currentstroke}%
\pgfsetdash{}{0pt}%
\pgfsys@defobject{currentmarker}{\pgfqpoint{0.000000in}{-0.048611in}}{\pgfqpoint{0.000000in}{0.000000in}}{%
\pgfpathmoveto{\pgfqpoint{0.000000in}{0.000000in}}%
\pgfpathlineto{\pgfqpoint{0.000000in}{-0.048611in}}%
\pgfusepath{stroke,fill}%
}%
\begin{pgfscope}%
\pgfsys@transformshift{12.983640in}{4.121437in}%
\pgfsys@useobject{currentmarker}{}%
\end{pgfscope}%
\end{pgfscope}%
\begin{pgfscope}%
\definecolor{textcolor}{rgb}{0.000000,0.000000,0.000000}%
\pgfsetstrokecolor{textcolor}%
\pgfsetfillcolor{textcolor}%
\pgftext[x=12.983640in,y=4.024215in,,top]{\color{textcolor}\rmfamily\fontsize{10.000000}{12.000000}\selectfont \(\displaystyle 160\)}%
\end{pgfscope}%
\begin{pgfscope}%
\pgfsetbuttcap%
\pgfsetroundjoin%
\definecolor{currentfill}{rgb}{0.000000,0.000000,0.000000}%
\pgfsetfillcolor{currentfill}%
\pgfsetlinewidth{0.803000pt}%
\definecolor{currentstroke}{rgb}{0.000000,0.000000,0.000000}%
\pgfsetstrokecolor{currentstroke}%
\pgfsetdash{}{0pt}%
\pgfsys@defobject{currentmarker}{\pgfqpoint{-0.048611in}{0.000000in}}{\pgfqpoint{0.000000in}{0.000000in}}{%
\pgfpathmoveto{\pgfqpoint{0.000000in}{0.000000in}}%
\pgfpathlineto{\pgfqpoint{-0.048611in}{0.000000in}}%
\pgfusepath{stroke,fill}%
}%
\begin{pgfscope}%
\pgfsys@transformshift{7.640588in}{4.243502in}%
\pgfsys@useobject{currentmarker}{}%
\end{pgfscope}%
\end{pgfscope}%
\begin{pgfscope}%
\definecolor{textcolor}{rgb}{0.000000,0.000000,0.000000}%
\pgfsetstrokecolor{textcolor}%
\pgfsetfillcolor{textcolor}%
\pgftext[x=7.473921in, y=4.190741in, left, base]{\color{textcolor}\rmfamily\fontsize{10.000000}{12.000000}\selectfont \(\displaystyle 0\)}%
\end{pgfscope}%
\begin{pgfscope}%
\pgfsetbuttcap%
\pgfsetroundjoin%
\definecolor{currentfill}{rgb}{0.000000,0.000000,0.000000}%
\pgfsetfillcolor{currentfill}%
\pgfsetlinewidth{0.803000pt}%
\definecolor{currentstroke}{rgb}{0.000000,0.000000,0.000000}%
\pgfsetstrokecolor{currentstroke}%
\pgfsetdash{}{0pt}%
\pgfsys@defobject{currentmarker}{\pgfqpoint{-0.048611in}{0.000000in}}{\pgfqpoint{0.000000in}{0.000000in}}{%
\pgfpathmoveto{\pgfqpoint{0.000000in}{0.000000in}}%
\pgfpathlineto{\pgfqpoint{-0.048611in}{0.000000in}}%
\pgfusepath{stroke,fill}%
}%
\begin{pgfscope}%
\pgfsys@transformshift{7.640588in}{4.556490in}%
\pgfsys@useobject{currentmarker}{}%
\end{pgfscope}%
\end{pgfscope}%
\begin{pgfscope}%
\definecolor{textcolor}{rgb}{0.000000,0.000000,0.000000}%
\pgfsetstrokecolor{textcolor}%
\pgfsetfillcolor{textcolor}%
\pgftext[x=7.404477in, y=4.503728in, left, base]{\color{textcolor}\rmfamily\fontsize{10.000000}{12.000000}\selectfont \(\displaystyle 20\)}%
\end{pgfscope}%
\begin{pgfscope}%
\pgfsetbuttcap%
\pgfsetroundjoin%
\definecolor{currentfill}{rgb}{0.000000,0.000000,0.000000}%
\pgfsetfillcolor{currentfill}%
\pgfsetlinewidth{0.803000pt}%
\definecolor{currentstroke}{rgb}{0.000000,0.000000,0.000000}%
\pgfsetstrokecolor{currentstroke}%
\pgfsetdash{}{0pt}%
\pgfsys@defobject{currentmarker}{\pgfqpoint{-0.048611in}{0.000000in}}{\pgfqpoint{0.000000in}{0.000000in}}{%
\pgfpathmoveto{\pgfqpoint{0.000000in}{0.000000in}}%
\pgfpathlineto{\pgfqpoint{-0.048611in}{0.000000in}}%
\pgfusepath{stroke,fill}%
}%
\begin{pgfscope}%
\pgfsys@transformshift{7.640588in}{4.869477in}%
\pgfsys@useobject{currentmarker}{}%
\end{pgfscope}%
\end{pgfscope}%
\begin{pgfscope}%
\definecolor{textcolor}{rgb}{0.000000,0.000000,0.000000}%
\pgfsetstrokecolor{textcolor}%
\pgfsetfillcolor{textcolor}%
\pgftext[x=7.404477in, y=4.816716in, left, base]{\color{textcolor}\rmfamily\fontsize{10.000000}{12.000000}\selectfont \(\displaystyle 40\)}%
\end{pgfscope}%
\begin{pgfscope}%
\pgfsetbuttcap%
\pgfsetroundjoin%
\definecolor{currentfill}{rgb}{0.000000,0.000000,0.000000}%
\pgfsetfillcolor{currentfill}%
\pgfsetlinewidth{0.803000pt}%
\definecolor{currentstroke}{rgb}{0.000000,0.000000,0.000000}%
\pgfsetstrokecolor{currentstroke}%
\pgfsetdash{}{0pt}%
\pgfsys@defobject{currentmarker}{\pgfqpoint{-0.048611in}{0.000000in}}{\pgfqpoint{0.000000in}{0.000000in}}{%
\pgfpathmoveto{\pgfqpoint{0.000000in}{0.000000in}}%
\pgfpathlineto{\pgfqpoint{-0.048611in}{0.000000in}}%
\pgfusepath{stroke,fill}%
}%
\begin{pgfscope}%
\pgfsys@transformshift{7.640588in}{5.182464in}%
\pgfsys@useobject{currentmarker}{}%
\end{pgfscope}%
\end{pgfscope}%
\begin{pgfscope}%
\definecolor{textcolor}{rgb}{0.000000,0.000000,0.000000}%
\pgfsetstrokecolor{textcolor}%
\pgfsetfillcolor{textcolor}%
\pgftext[x=7.404477in, y=5.129703in, left, base]{\color{textcolor}\rmfamily\fontsize{10.000000}{12.000000}\selectfont \(\displaystyle 60\)}%
\end{pgfscope}%
\begin{pgfscope}%
\pgfsetbuttcap%
\pgfsetroundjoin%
\definecolor{currentfill}{rgb}{0.000000,0.000000,0.000000}%
\pgfsetfillcolor{currentfill}%
\pgfsetlinewidth{0.803000pt}%
\definecolor{currentstroke}{rgb}{0.000000,0.000000,0.000000}%
\pgfsetstrokecolor{currentstroke}%
\pgfsetdash{}{0pt}%
\pgfsys@defobject{currentmarker}{\pgfqpoint{-0.048611in}{0.000000in}}{\pgfqpoint{0.000000in}{0.000000in}}{%
\pgfpathmoveto{\pgfqpoint{0.000000in}{0.000000in}}%
\pgfpathlineto{\pgfqpoint{-0.048611in}{0.000000in}}%
\pgfusepath{stroke,fill}%
}%
\begin{pgfscope}%
\pgfsys@transformshift{7.640588in}{5.495452in}%
\pgfsys@useobject{currentmarker}{}%
\end{pgfscope}%
\end{pgfscope}%
\begin{pgfscope}%
\definecolor{textcolor}{rgb}{0.000000,0.000000,0.000000}%
\pgfsetstrokecolor{textcolor}%
\pgfsetfillcolor{textcolor}%
\pgftext[x=7.404477in, y=5.442690in, left, base]{\color{textcolor}\rmfamily\fontsize{10.000000}{12.000000}\selectfont \(\displaystyle 80\)}%
\end{pgfscope}%
\begin{pgfscope}%
\pgfsetbuttcap%
\pgfsetroundjoin%
\definecolor{currentfill}{rgb}{0.000000,0.000000,0.000000}%
\pgfsetfillcolor{currentfill}%
\pgfsetlinewidth{0.803000pt}%
\definecolor{currentstroke}{rgb}{0.000000,0.000000,0.000000}%
\pgfsetstrokecolor{currentstroke}%
\pgfsetdash{}{0pt}%
\pgfsys@defobject{currentmarker}{\pgfqpoint{-0.048611in}{0.000000in}}{\pgfqpoint{0.000000in}{0.000000in}}{%
\pgfpathmoveto{\pgfqpoint{0.000000in}{0.000000in}}%
\pgfpathlineto{\pgfqpoint{-0.048611in}{0.000000in}}%
\pgfusepath{stroke,fill}%
}%
\begin{pgfscope}%
\pgfsys@transformshift{7.640588in}{5.808439in}%
\pgfsys@useobject{currentmarker}{}%
\end{pgfscope}%
\end{pgfscope}%
\begin{pgfscope}%
\definecolor{textcolor}{rgb}{0.000000,0.000000,0.000000}%
\pgfsetstrokecolor{textcolor}%
\pgfsetfillcolor{textcolor}%
\pgftext[x=7.335032in, y=5.755678in, left, base]{\color{textcolor}\rmfamily\fontsize{10.000000}{12.000000}\selectfont \(\displaystyle 100\)}%
\end{pgfscope}%
\begin{pgfscope}%
\pgfsetbuttcap%
\pgfsetroundjoin%
\definecolor{currentfill}{rgb}{0.000000,0.000000,0.000000}%
\pgfsetfillcolor{currentfill}%
\pgfsetlinewidth{0.803000pt}%
\definecolor{currentstroke}{rgb}{0.000000,0.000000,0.000000}%
\pgfsetstrokecolor{currentstroke}%
\pgfsetdash{}{0pt}%
\pgfsys@defobject{currentmarker}{\pgfqpoint{-0.048611in}{0.000000in}}{\pgfqpoint{0.000000in}{0.000000in}}{%
\pgfpathmoveto{\pgfqpoint{0.000000in}{0.000000in}}%
\pgfpathlineto{\pgfqpoint{-0.048611in}{0.000000in}}%
\pgfusepath{stroke,fill}%
}%
\begin{pgfscope}%
\pgfsys@transformshift{7.640588in}{6.121427in}%
\pgfsys@useobject{currentmarker}{}%
\end{pgfscope}%
\end{pgfscope}%
\begin{pgfscope}%
\definecolor{textcolor}{rgb}{0.000000,0.000000,0.000000}%
\pgfsetstrokecolor{textcolor}%
\pgfsetfillcolor{textcolor}%
\pgftext[x=7.335032in, y=6.068665in, left, base]{\color{textcolor}\rmfamily\fontsize{10.000000}{12.000000}\selectfont \(\displaystyle 120\)}%
\end{pgfscope}%
\begin{pgfscope}%
\pgfsetbuttcap%
\pgfsetroundjoin%
\definecolor{currentfill}{rgb}{0.000000,0.000000,0.000000}%
\pgfsetfillcolor{currentfill}%
\pgfsetlinewidth{0.803000pt}%
\definecolor{currentstroke}{rgb}{0.000000,0.000000,0.000000}%
\pgfsetstrokecolor{currentstroke}%
\pgfsetdash{}{0pt}%
\pgfsys@defobject{currentmarker}{\pgfqpoint{-0.048611in}{0.000000in}}{\pgfqpoint{0.000000in}{0.000000in}}{%
\pgfpathmoveto{\pgfqpoint{0.000000in}{0.000000in}}%
\pgfpathlineto{\pgfqpoint{-0.048611in}{0.000000in}}%
\pgfusepath{stroke,fill}%
}%
\begin{pgfscope}%
\pgfsys@transformshift{7.640588in}{6.434414in}%
\pgfsys@useobject{currentmarker}{}%
\end{pgfscope}%
\end{pgfscope}%
\begin{pgfscope}%
\definecolor{textcolor}{rgb}{0.000000,0.000000,0.000000}%
\pgfsetstrokecolor{textcolor}%
\pgfsetfillcolor{textcolor}%
\pgftext[x=7.335032in, y=6.381653in, left, base]{\color{textcolor}\rmfamily\fontsize{10.000000}{12.000000}\selectfont \(\displaystyle 140\)}%
\end{pgfscope}%
\begin{pgfscope}%
\pgfsetbuttcap%
\pgfsetroundjoin%
\definecolor{currentfill}{rgb}{0.000000,0.000000,0.000000}%
\pgfsetfillcolor{currentfill}%
\pgfsetlinewidth{0.803000pt}%
\definecolor{currentstroke}{rgb}{0.000000,0.000000,0.000000}%
\pgfsetstrokecolor{currentstroke}%
\pgfsetdash{}{0pt}%
\pgfsys@defobject{currentmarker}{\pgfqpoint{-0.048611in}{0.000000in}}{\pgfqpoint{0.000000in}{0.000000in}}{%
\pgfpathmoveto{\pgfqpoint{0.000000in}{0.000000in}}%
\pgfpathlineto{\pgfqpoint{-0.048611in}{0.000000in}}%
\pgfusepath{stroke,fill}%
}%
\begin{pgfscope}%
\pgfsys@transformshift{7.640588in}{6.747402in}%
\pgfsys@useobject{currentmarker}{}%
\end{pgfscope}%
\end{pgfscope}%
\begin{pgfscope}%
\definecolor{textcolor}{rgb}{0.000000,0.000000,0.000000}%
\pgfsetstrokecolor{textcolor}%
\pgfsetfillcolor{textcolor}%
\pgftext[x=7.335032in, y=6.694640in, left, base]{\color{textcolor}\rmfamily\fontsize{10.000000}{12.000000}\selectfont \(\displaystyle 160\)}%
\end{pgfscope}%
\begin{pgfscope}%
\pgfsetrectcap%
\pgfsetmiterjoin%
\pgfsetlinewidth{0.803000pt}%
\definecolor{currentstroke}{rgb}{0.000000,0.000000,0.000000}%
\pgfsetstrokecolor{currentstroke}%
\pgfsetdash{}{0pt}%
\pgfpathmoveto{\pgfqpoint{7.640588in}{4.121437in}}%
\pgfpathlineto{\pgfqpoint{7.640588in}{6.806869in}}%
\pgfusepath{stroke}%
\end{pgfscope}%
\begin{pgfscope}%
\pgfsetrectcap%
\pgfsetmiterjoin%
\pgfsetlinewidth{0.803000pt}%
\definecolor{currentstroke}{rgb}{0.000000,0.000000,0.000000}%
\pgfsetstrokecolor{currentstroke}%
\pgfsetdash{}{0pt}%
\pgfpathmoveto{\pgfqpoint{13.339844in}{4.121437in}}%
\pgfpathlineto{\pgfqpoint{13.339844in}{6.806869in}}%
\pgfusepath{stroke}%
\end{pgfscope}%
\begin{pgfscope}%
\pgfsetrectcap%
\pgfsetmiterjoin%
\pgfsetlinewidth{0.803000pt}%
\definecolor{currentstroke}{rgb}{0.000000,0.000000,0.000000}%
\pgfsetstrokecolor{currentstroke}%
\pgfsetdash{}{0pt}%
\pgfpathmoveto{\pgfqpoint{7.640588in}{4.121437in}}%
\pgfpathlineto{\pgfqpoint{13.339844in}{4.121437in}}%
\pgfusepath{stroke}%
\end{pgfscope}%
\begin{pgfscope}%
\pgfsetrectcap%
\pgfsetmiterjoin%
\pgfsetlinewidth{0.803000pt}%
\definecolor{currentstroke}{rgb}{0.000000,0.000000,0.000000}%
\pgfsetstrokecolor{currentstroke}%
\pgfsetdash{}{0pt}%
\pgfpathmoveto{\pgfqpoint{7.640588in}{6.806869in}}%
\pgfpathlineto{\pgfqpoint{13.339844in}{6.806869in}}%
\pgfusepath{stroke}%
\end{pgfscope}%
\begin{pgfscope}%
\definecolor{textcolor}{rgb}{0.000000,0.000000,0.000000}%
\pgfsetstrokecolor{textcolor}%
\pgfsetfillcolor{textcolor}%
\pgftext[x=10.490216in,y=6.890203in,,base]{\color{textcolor}\rmfamily\fontsize{12.000000}{14.400000}\selectfont twoopt 1143.8}%
\end{pgfscope}%
\begin{pgfscope}%
\pgfsetbuttcap%
\pgfsetmiterjoin%
\definecolor{currentfill}{rgb}{1.000000,1.000000,1.000000}%
\pgfsetfillcolor{currentfill}%
\pgfsetlinewidth{0.000000pt}%
\definecolor{currentstroke}{rgb}{0.000000,0.000000,0.000000}%
\pgfsetstrokecolor{currentstroke}%
\pgfsetstrokeopacity{0.000000}%
\pgfsetdash{}{0pt}%
\pgfpathmoveto{\pgfqpoint{0.970666in}{0.566125in}}%
\pgfpathlineto{\pgfqpoint{6.669922in}{0.566125in}}%
\pgfpathlineto{\pgfqpoint{6.669922in}{3.251557in}}%
\pgfpathlineto{\pgfqpoint{0.970666in}{3.251557in}}%
\pgfpathclose%
\pgfusepath{fill}%
\end{pgfscope}%
\begin{pgfscope}%
\pgfpathrectangle{\pgfqpoint{0.970666in}{0.566125in}}{\pgfqpoint{5.699255in}{2.685432in}}%
\pgfusepath{clip}%
\pgfsetrectcap%
\pgfsetroundjoin%
\pgfsetlinewidth{1.505625pt}%
\definecolor{currentstroke}{rgb}{0.000000,0.000000,0.000000}%
\pgfsetstrokecolor{currentstroke}%
\pgfsetdash{}{0pt}%
\pgfpathmoveto{\pgfqpoint{4.435555in}{2.190529in}}%
\pgfpathlineto{\pgfqpoint{4.500319in}{2.174880in}}%
\pgfpathlineto{\pgfqpoint{4.144116in}{2.456569in}}%
\pgfpathlineto{\pgfqpoint{3.626001in}{2.409621in}}%
\pgfpathlineto{\pgfqpoint{3.334562in}{2.300075in}}%
\pgfpathlineto{\pgfqpoint{3.431709in}{2.096633in}}%
\pgfpathlineto{\pgfqpoint{3.431709in}{2.034036in}}%
\pgfpathlineto{\pgfqpoint{3.464091in}{1.908841in}}%
\pgfpathlineto{\pgfqpoint{3.140269in}{1.877542in}}%
\pgfpathlineto{\pgfqpoint{2.945977in}{1.877542in}}%
\pgfpathlineto{\pgfqpoint{2.654537in}{1.783646in}}%
\pgfpathlineto{\pgfqpoint{2.557391in}{1.924490in}}%
\pgfpathlineto{\pgfqpoint{2.201188in}{1.940139in}}%
\pgfpathlineto{\pgfqpoint{2.201188in}{2.096633in}}%
\pgfpathlineto{\pgfqpoint{2.363098in}{2.159231in}}%
\pgfpathlineto{\pgfqpoint{2.686920in}{2.487867in}}%
\pgfpathlineto{\pgfqpoint{2.945977in}{2.597413in}}%
\pgfpathlineto{\pgfqpoint{3.172652in}{2.722608in}}%
\pgfpathlineto{\pgfqpoint{3.528855in}{2.769556in}}%
\pgfpathlineto{\pgfqpoint{4.144116in}{2.863452in}}%
\pgfpathlineto{\pgfqpoint{4.338408in}{2.894751in}}%
\pgfpathlineto{\pgfqpoint{4.662230in}{2.863452in}}%
\pgfpathlineto{\pgfqpoint{4.824140in}{2.832154in}}%
\pgfpathlineto{\pgfqpoint{4.888905in}{2.879102in}}%
\pgfpathlineto{\pgfqpoint{5.115579in}{2.753907in}}%
\pgfpathlineto{\pgfqpoint{5.471783in}{2.753907in}}%
\pgfpathlineto{\pgfqpoint{5.374637in}{2.487867in}}%
\pgfpathlineto{\pgfqpoint{5.601311in}{2.362672in}}%
\pgfpathlineto{\pgfqpoint{5.730840in}{2.221828in}}%
\pgfpathlineto{\pgfqpoint{5.925133in}{2.409621in}}%
\pgfpathlineto{\pgfqpoint{6.410865in}{2.785205in}}%
\pgfpathlineto{\pgfqpoint{6.151808in}{2.879102in}}%
\pgfpathlineto{\pgfqpoint{6.119426in}{3.129492in}}%
\pgfpathlineto{\pgfqpoint{5.795604in}{3.098193in}}%
\pgfpathlineto{\pgfqpoint{5.763222in}{3.066894in}}%
\pgfpathlineto{\pgfqpoint{5.536547in}{3.082544in}}%
\pgfpathlineto{\pgfqpoint{6.022279in}{1.893191in}}%
\pgfpathlineto{\pgfqpoint{5.147962in}{1.861893in}}%
\pgfpathlineto{\pgfqpoint{4.759376in}{1.830594in}}%
\pgfpathlineto{\pgfqpoint{4.208880in}{1.799295in}}%
\pgfpathlineto{\pgfqpoint{3.528855in}{1.736698in}}%
\pgfpathlineto{\pgfqpoint{3.496473in}{1.642801in}}%
\pgfpathlineto{\pgfqpoint{4.046969in}{1.361113in}}%
\pgfpathlineto{\pgfqpoint{3.593619in}{1.220268in}}%
\pgfpathlineto{\pgfqpoint{3.237416in}{1.173320in}}%
\pgfpathlineto{\pgfqpoint{3.140269in}{0.954229in}}%
\pgfpathlineto{\pgfqpoint{2.945977in}{0.750787in}}%
\pgfpathlineto{\pgfqpoint{2.686920in}{0.875982in}}%
\pgfpathlineto{\pgfqpoint{2.525009in}{0.969878in}}%
\pgfpathlineto{\pgfqpoint{2.363098in}{1.016827in}}%
\pgfpathlineto{\pgfqpoint{2.233570in}{0.954229in}}%
\pgfpathlineto{\pgfqpoint{2.006895in}{1.001177in}}%
\pgfpathlineto{\pgfqpoint{1.877366in}{1.095073in}}%
\pgfpathlineto{\pgfqpoint{1.747838in}{1.063775in}}%
\pgfpathlineto{\pgfqpoint{1.650691in}{0.938580in}}%
\pgfpathlineto{\pgfqpoint{1.456398in}{0.782086in}}%
\pgfpathlineto{\pgfqpoint{1.229724in}{1.079424in}}%
\pgfpathlineto{\pgfqpoint{1.488781in}{1.157671in}}%
\pgfpathlineto{\pgfqpoint{2.136423in}{1.126372in}}%
\pgfpathlineto{\pgfqpoint{1.262106in}{1.595853in}}%
\pgfpathlineto{\pgfqpoint{1.456398in}{2.362672in}}%
\pgfpathlineto{\pgfqpoint{1.618309in}{2.393971in}}%
\pgfpathlineto{\pgfqpoint{2.265952in}{2.628712in}}%
\pgfpathlineto{\pgfqpoint{2.201188in}{2.941699in}}%
\pgfpathlineto{\pgfqpoint{2.201188in}{2.988647in}}%
\pgfpathlineto{\pgfqpoint{2.104041in}{2.972998in}}%
\pgfpathlineto{\pgfqpoint{2.104041in}{3.113842in}}%
\pgfpathlineto{\pgfqpoint{2.492627in}{3.098193in}}%
\pgfpathlineto{\pgfqpoint{5.342254in}{1.517606in}}%
\pgfpathlineto{\pgfqpoint{5.374637in}{1.329814in}}%
\pgfpathlineto{\pgfqpoint{5.277490in}{1.267216in}}%
\pgfpathlineto{\pgfqpoint{5.277490in}{1.188970in}}%
\pgfpathlineto{\pgfqpoint{5.471783in}{1.157671in}}%
\pgfpathlineto{\pgfqpoint{4.888905in}{1.157671in}}%
\pgfpathlineto{\pgfqpoint{4.759376in}{1.220268in}}%
\pgfpathlineto{\pgfqpoint{4.565083in}{0.891632in}}%
\pgfpathlineto{\pgfqpoint{4.791758in}{0.688190in}}%
\pgfpathlineto{\pgfqpoint{5.309872in}{0.688190in}}%
\pgfpathlineto{\pgfqpoint{3.982205in}{0.829034in}}%
\pgfpathlineto{\pgfqpoint{3.949823in}{0.797735in}}%
\pgfpathlineto{\pgfqpoint{5.795604in}{1.455009in}}%
\pgfpathlineto{\pgfqpoint{6.184190in}{1.282866in}}%
\pgfpathlineto{\pgfqpoint{4.435555in}{2.190529in}}%
\pgfusepath{stroke}%
\end{pgfscope}%
\begin{pgfscope}%
\pgfpathrectangle{\pgfqpoint{0.970666in}{0.566125in}}{\pgfqpoint{5.699255in}{2.685432in}}%
\pgfusepath{clip}%
\pgfsetbuttcap%
\pgfsetroundjoin%
\definecolor{currentfill}{rgb}{1.000000,0.000000,0.000000}%
\pgfsetfillcolor{currentfill}%
\pgfsetlinewidth{1.003750pt}%
\definecolor{currentstroke}{rgb}{1.000000,0.000000,0.000000}%
\pgfsetstrokecolor{currentstroke}%
\pgfsetdash{}{0pt}%
\pgfpathmoveto{\pgfqpoint{4.435555in}{2.148863in}}%
\pgfpathcurveto{\pgfqpoint{4.446605in}{2.148863in}}{\pgfqpoint{4.457204in}{2.153253in}}{\pgfqpoint{4.465017in}{2.161067in}}%
\pgfpathcurveto{\pgfqpoint{4.472831in}{2.168880in}}{\pgfqpoint{4.477221in}{2.179479in}}{\pgfqpoint{4.477221in}{2.190529in}}%
\pgfpathcurveto{\pgfqpoint{4.477221in}{2.201579in}}{\pgfqpoint{4.472831in}{2.212179in}}{\pgfqpoint{4.465017in}{2.219992in}}%
\pgfpathcurveto{\pgfqpoint{4.457204in}{2.227806in}}{\pgfqpoint{4.446605in}{2.232196in}}{\pgfqpoint{4.435555in}{2.232196in}}%
\pgfpathcurveto{\pgfqpoint{4.424505in}{2.232196in}}{\pgfqpoint{4.413906in}{2.227806in}}{\pgfqpoint{4.406092in}{2.219992in}}%
\pgfpathcurveto{\pgfqpoint{4.398278in}{2.212179in}}{\pgfqpoint{4.393888in}{2.201579in}}{\pgfqpoint{4.393888in}{2.190529in}}%
\pgfpathcurveto{\pgfqpoint{4.393888in}{2.179479in}}{\pgfqpoint{4.398278in}{2.168880in}}{\pgfqpoint{4.406092in}{2.161067in}}%
\pgfpathcurveto{\pgfqpoint{4.413906in}{2.153253in}}{\pgfqpoint{4.424505in}{2.148863in}}{\pgfqpoint{4.435555in}{2.148863in}}%
\pgfpathclose%
\pgfusepath{stroke,fill}%
\end{pgfscope}%
\begin{pgfscope}%
\pgfpathrectangle{\pgfqpoint{0.970666in}{0.566125in}}{\pgfqpoint{5.699255in}{2.685432in}}%
\pgfusepath{clip}%
\pgfsetbuttcap%
\pgfsetroundjoin%
\definecolor{currentfill}{rgb}{0.750000,0.750000,0.000000}%
\pgfsetfillcolor{currentfill}%
\pgfsetlinewidth{1.003750pt}%
\definecolor{currentstroke}{rgb}{0.750000,0.750000,0.000000}%
\pgfsetstrokecolor{currentstroke}%
\pgfsetdash{}{0pt}%
\pgfpathmoveto{\pgfqpoint{4.500319in}{2.133213in}}%
\pgfpathcurveto{\pgfqpoint{4.511369in}{2.133213in}}{\pgfqpoint{4.521968in}{2.137604in}}{\pgfqpoint{4.529782in}{2.145417in}}%
\pgfpathcurveto{\pgfqpoint{4.537595in}{2.153231in}}{\pgfqpoint{4.541986in}{2.163830in}}{\pgfqpoint{4.541986in}{2.174880in}}%
\pgfpathcurveto{\pgfqpoint{4.541986in}{2.185930in}}{\pgfqpoint{4.537595in}{2.196529in}}{\pgfqpoint{4.529782in}{2.204343in}}%
\pgfpathcurveto{\pgfqpoint{4.521968in}{2.212156in}}{\pgfqpoint{4.511369in}{2.216547in}}{\pgfqpoint{4.500319in}{2.216547in}}%
\pgfpathcurveto{\pgfqpoint{4.489269in}{2.216547in}}{\pgfqpoint{4.478670in}{2.212156in}}{\pgfqpoint{4.470856in}{2.204343in}}%
\pgfpathcurveto{\pgfqpoint{4.463043in}{2.196529in}}{\pgfqpoint{4.458652in}{2.185930in}}{\pgfqpoint{4.458652in}{2.174880in}}%
\pgfpathcurveto{\pgfqpoint{4.458652in}{2.163830in}}{\pgfqpoint{4.463043in}{2.153231in}}{\pgfqpoint{4.470856in}{2.145417in}}%
\pgfpathcurveto{\pgfqpoint{4.478670in}{2.137604in}}{\pgfqpoint{4.489269in}{2.133213in}}{\pgfqpoint{4.500319in}{2.133213in}}%
\pgfpathclose%
\pgfusepath{stroke,fill}%
\end{pgfscope}%
\begin{pgfscope}%
\pgfpathrectangle{\pgfqpoint{0.970666in}{0.566125in}}{\pgfqpoint{5.699255in}{2.685432in}}%
\pgfusepath{clip}%
\pgfsetbuttcap%
\pgfsetroundjoin%
\definecolor{currentfill}{rgb}{0.000000,0.000000,0.000000}%
\pgfsetfillcolor{currentfill}%
\pgfsetlinewidth{1.003750pt}%
\definecolor{currentstroke}{rgb}{0.000000,0.000000,0.000000}%
\pgfsetstrokecolor{currentstroke}%
\pgfsetdash{}{0pt}%
\pgfpathmoveto{\pgfqpoint{4.144116in}{2.414902in}}%
\pgfpathcurveto{\pgfqpoint{4.155166in}{2.414902in}}{\pgfqpoint{4.165765in}{2.419292in}}{\pgfqpoint{4.173578in}{2.427106in}}%
\pgfpathcurveto{\pgfqpoint{4.181392in}{2.434919in}}{\pgfqpoint{4.185782in}{2.445519in}}{\pgfqpoint{4.185782in}{2.456569in}}%
\pgfpathcurveto{\pgfqpoint{4.185782in}{2.467619in}}{\pgfqpoint{4.181392in}{2.478218in}}{\pgfqpoint{4.173578in}{2.486031in}}%
\pgfpathcurveto{\pgfqpoint{4.165765in}{2.493845in}}{\pgfqpoint{4.155166in}{2.498235in}}{\pgfqpoint{4.144116in}{2.498235in}}%
\pgfpathcurveto{\pgfqpoint{4.133065in}{2.498235in}}{\pgfqpoint{4.122466in}{2.493845in}}{\pgfqpoint{4.114653in}{2.486031in}}%
\pgfpathcurveto{\pgfqpoint{4.106839in}{2.478218in}}{\pgfqpoint{4.102449in}{2.467619in}}{\pgfqpoint{4.102449in}{2.456569in}}%
\pgfpathcurveto{\pgfqpoint{4.102449in}{2.445519in}}{\pgfqpoint{4.106839in}{2.434919in}}{\pgfqpoint{4.114653in}{2.427106in}}%
\pgfpathcurveto{\pgfqpoint{4.122466in}{2.419292in}}{\pgfqpoint{4.133065in}{2.414902in}}{\pgfqpoint{4.144116in}{2.414902in}}%
\pgfpathclose%
\pgfusepath{stroke,fill}%
\end{pgfscope}%
\begin{pgfscope}%
\pgfpathrectangle{\pgfqpoint{0.970666in}{0.566125in}}{\pgfqpoint{5.699255in}{2.685432in}}%
\pgfusepath{clip}%
\pgfsetbuttcap%
\pgfsetroundjoin%
\definecolor{currentfill}{rgb}{0.000000,0.000000,0.000000}%
\pgfsetfillcolor{currentfill}%
\pgfsetlinewidth{1.003750pt}%
\definecolor{currentstroke}{rgb}{0.000000,0.000000,0.000000}%
\pgfsetstrokecolor{currentstroke}%
\pgfsetdash{}{0pt}%
\pgfpathmoveto{\pgfqpoint{3.626001in}{2.367954in}}%
\pgfpathcurveto{\pgfqpoint{3.637052in}{2.367954in}}{\pgfqpoint{3.647651in}{2.372344in}}{\pgfqpoint{3.655464in}{2.380158in}}%
\pgfpathcurveto{\pgfqpoint{3.663278in}{2.387971in}}{\pgfqpoint{3.667668in}{2.398570in}}{\pgfqpoint{3.667668in}{2.409621in}}%
\pgfpathcurveto{\pgfqpoint{3.667668in}{2.420671in}}{\pgfqpoint{3.663278in}{2.431270in}}{\pgfqpoint{3.655464in}{2.439083in}}%
\pgfpathcurveto{\pgfqpoint{3.647651in}{2.446897in}}{\pgfqpoint{3.637052in}{2.451287in}}{\pgfqpoint{3.626001in}{2.451287in}}%
\pgfpathcurveto{\pgfqpoint{3.614951in}{2.451287in}}{\pgfqpoint{3.604352in}{2.446897in}}{\pgfqpoint{3.596539in}{2.439083in}}%
\pgfpathcurveto{\pgfqpoint{3.588725in}{2.431270in}}{\pgfqpoint{3.584335in}{2.420671in}}{\pgfqpoint{3.584335in}{2.409621in}}%
\pgfpathcurveto{\pgfqpoint{3.584335in}{2.398570in}}{\pgfqpoint{3.588725in}{2.387971in}}{\pgfqpoint{3.596539in}{2.380158in}}%
\pgfpathcurveto{\pgfqpoint{3.604352in}{2.372344in}}{\pgfqpoint{3.614951in}{2.367954in}}{\pgfqpoint{3.626001in}{2.367954in}}%
\pgfpathclose%
\pgfusepath{stroke,fill}%
\end{pgfscope}%
\begin{pgfscope}%
\pgfpathrectangle{\pgfqpoint{0.970666in}{0.566125in}}{\pgfqpoint{5.699255in}{2.685432in}}%
\pgfusepath{clip}%
\pgfsetbuttcap%
\pgfsetroundjoin%
\definecolor{currentfill}{rgb}{0.000000,0.000000,0.000000}%
\pgfsetfillcolor{currentfill}%
\pgfsetlinewidth{1.003750pt}%
\definecolor{currentstroke}{rgb}{0.000000,0.000000,0.000000}%
\pgfsetstrokecolor{currentstroke}%
\pgfsetdash{}{0pt}%
\pgfpathmoveto{\pgfqpoint{3.334562in}{2.258408in}}%
\pgfpathcurveto{\pgfqpoint{3.345612in}{2.258408in}}{\pgfqpoint{3.356211in}{2.262799in}}{\pgfqpoint{3.364025in}{2.270612in}}%
\pgfpathcurveto{\pgfqpoint{3.371839in}{2.278426in}}{\pgfqpoint{3.376229in}{2.289025in}}{\pgfqpoint{3.376229in}{2.300075in}}%
\pgfpathcurveto{\pgfqpoint{3.376229in}{2.311125in}}{\pgfqpoint{3.371839in}{2.321724in}}{\pgfqpoint{3.364025in}{2.329538in}}%
\pgfpathcurveto{\pgfqpoint{3.356211in}{2.337351in}}{\pgfqpoint{3.345612in}{2.341742in}}{\pgfqpoint{3.334562in}{2.341742in}}%
\pgfpathcurveto{\pgfqpoint{3.323512in}{2.341742in}}{\pgfqpoint{3.312913in}{2.337351in}}{\pgfqpoint{3.305099in}{2.329538in}}%
\pgfpathcurveto{\pgfqpoint{3.297286in}{2.321724in}}{\pgfqpoint{3.292896in}{2.311125in}}{\pgfqpoint{3.292896in}{2.300075in}}%
\pgfpathcurveto{\pgfqpoint{3.292896in}{2.289025in}}{\pgfqpoint{3.297286in}{2.278426in}}{\pgfqpoint{3.305099in}{2.270612in}}%
\pgfpathcurveto{\pgfqpoint{3.312913in}{2.262799in}}{\pgfqpoint{3.323512in}{2.258408in}}{\pgfqpoint{3.334562in}{2.258408in}}%
\pgfpathclose%
\pgfusepath{stroke,fill}%
\end{pgfscope}%
\begin{pgfscope}%
\pgfpathrectangle{\pgfqpoint{0.970666in}{0.566125in}}{\pgfqpoint{5.699255in}{2.685432in}}%
\pgfusepath{clip}%
\pgfsetbuttcap%
\pgfsetroundjoin%
\definecolor{currentfill}{rgb}{0.000000,0.000000,0.000000}%
\pgfsetfillcolor{currentfill}%
\pgfsetlinewidth{1.003750pt}%
\definecolor{currentstroke}{rgb}{0.000000,0.000000,0.000000}%
\pgfsetstrokecolor{currentstroke}%
\pgfsetdash{}{0pt}%
\pgfpathmoveto{\pgfqpoint{3.431709in}{2.054966in}}%
\pgfpathcurveto{\pgfqpoint{3.442759in}{2.054966in}}{\pgfqpoint{3.453358in}{2.059357in}}{\pgfqpoint{3.461171in}{2.067170in}}%
\pgfpathcurveto{\pgfqpoint{3.468985in}{2.074984in}}{\pgfqpoint{3.473375in}{2.085583in}}{\pgfqpoint{3.473375in}{2.096633in}}%
\pgfpathcurveto{\pgfqpoint{3.473375in}{2.107683in}}{\pgfqpoint{3.468985in}{2.118282in}}{\pgfqpoint{3.461171in}{2.126096in}}%
\pgfpathcurveto{\pgfqpoint{3.453358in}{2.133910in}}{\pgfqpoint{3.442759in}{2.138300in}}{\pgfqpoint{3.431709in}{2.138300in}}%
\pgfpathcurveto{\pgfqpoint{3.420658in}{2.138300in}}{\pgfqpoint{3.410059in}{2.133910in}}{\pgfqpoint{3.402246in}{2.126096in}}%
\pgfpathcurveto{\pgfqpoint{3.394432in}{2.118282in}}{\pgfqpoint{3.390042in}{2.107683in}}{\pgfqpoint{3.390042in}{2.096633in}}%
\pgfpathcurveto{\pgfqpoint{3.390042in}{2.085583in}}{\pgfqpoint{3.394432in}{2.074984in}}{\pgfqpoint{3.402246in}{2.067170in}}%
\pgfpathcurveto{\pgfqpoint{3.410059in}{2.059357in}}{\pgfqpoint{3.420658in}{2.054966in}}{\pgfqpoint{3.431709in}{2.054966in}}%
\pgfpathclose%
\pgfusepath{stroke,fill}%
\end{pgfscope}%
\begin{pgfscope}%
\pgfpathrectangle{\pgfqpoint{0.970666in}{0.566125in}}{\pgfqpoint{5.699255in}{2.685432in}}%
\pgfusepath{clip}%
\pgfsetbuttcap%
\pgfsetroundjoin%
\definecolor{currentfill}{rgb}{0.000000,0.000000,0.000000}%
\pgfsetfillcolor{currentfill}%
\pgfsetlinewidth{1.003750pt}%
\definecolor{currentstroke}{rgb}{0.000000,0.000000,0.000000}%
\pgfsetstrokecolor{currentstroke}%
\pgfsetdash{}{0pt}%
\pgfpathmoveto{\pgfqpoint{3.431709in}{1.992369in}}%
\pgfpathcurveto{\pgfqpoint{3.442759in}{1.992369in}}{\pgfqpoint{3.453358in}{1.996759in}}{\pgfqpoint{3.461171in}{2.004573in}}%
\pgfpathcurveto{\pgfqpoint{3.468985in}{2.012386in}}{\pgfqpoint{3.473375in}{2.022986in}}{\pgfqpoint{3.473375in}{2.034036in}}%
\pgfpathcurveto{\pgfqpoint{3.473375in}{2.045086in}}{\pgfqpoint{3.468985in}{2.055685in}}{\pgfqpoint{3.461171in}{2.063498in}}%
\pgfpathcurveto{\pgfqpoint{3.453358in}{2.071312in}}{\pgfqpoint{3.442759in}{2.075702in}}{\pgfqpoint{3.431709in}{2.075702in}}%
\pgfpathcurveto{\pgfqpoint{3.420658in}{2.075702in}}{\pgfqpoint{3.410059in}{2.071312in}}{\pgfqpoint{3.402246in}{2.063498in}}%
\pgfpathcurveto{\pgfqpoint{3.394432in}{2.055685in}}{\pgfqpoint{3.390042in}{2.045086in}}{\pgfqpoint{3.390042in}{2.034036in}}%
\pgfpathcurveto{\pgfqpoint{3.390042in}{2.022986in}}{\pgfqpoint{3.394432in}{2.012386in}}{\pgfqpoint{3.402246in}{2.004573in}}%
\pgfpathcurveto{\pgfqpoint{3.410059in}{1.996759in}}{\pgfqpoint{3.420658in}{1.992369in}}{\pgfqpoint{3.431709in}{1.992369in}}%
\pgfpathclose%
\pgfusepath{stroke,fill}%
\end{pgfscope}%
\begin{pgfscope}%
\pgfpathrectangle{\pgfqpoint{0.970666in}{0.566125in}}{\pgfqpoint{5.699255in}{2.685432in}}%
\pgfusepath{clip}%
\pgfsetbuttcap%
\pgfsetroundjoin%
\definecolor{currentfill}{rgb}{0.000000,0.000000,0.000000}%
\pgfsetfillcolor{currentfill}%
\pgfsetlinewidth{1.003750pt}%
\definecolor{currentstroke}{rgb}{0.000000,0.000000,0.000000}%
\pgfsetstrokecolor{currentstroke}%
\pgfsetdash{}{0pt}%
\pgfpathmoveto{\pgfqpoint{3.464091in}{1.867174in}}%
\pgfpathcurveto{\pgfqpoint{3.475141in}{1.867174in}}{\pgfqpoint{3.485740in}{1.871564in}}{\pgfqpoint{3.493554in}{1.879378in}}%
\pgfpathcurveto{\pgfqpoint{3.501367in}{1.887192in}}{\pgfqpoint{3.505757in}{1.897791in}}{\pgfqpoint{3.505757in}{1.908841in}}%
\pgfpathcurveto{\pgfqpoint{3.505757in}{1.919891in}}{\pgfqpoint{3.501367in}{1.930490in}}{\pgfqpoint{3.493554in}{1.938303in}}%
\pgfpathcurveto{\pgfqpoint{3.485740in}{1.946117in}}{\pgfqpoint{3.475141in}{1.950507in}}{\pgfqpoint{3.464091in}{1.950507in}}%
\pgfpathcurveto{\pgfqpoint{3.453041in}{1.950507in}}{\pgfqpoint{3.442442in}{1.946117in}}{\pgfqpoint{3.434628in}{1.938303in}}%
\pgfpathcurveto{\pgfqpoint{3.426814in}{1.930490in}}{\pgfqpoint{3.422424in}{1.919891in}}{\pgfqpoint{3.422424in}{1.908841in}}%
\pgfpathcurveto{\pgfqpoint{3.422424in}{1.897791in}}{\pgfqpoint{3.426814in}{1.887192in}}{\pgfqpoint{3.434628in}{1.879378in}}%
\pgfpathcurveto{\pgfqpoint{3.442442in}{1.871564in}}{\pgfqpoint{3.453041in}{1.867174in}}{\pgfqpoint{3.464091in}{1.867174in}}%
\pgfpathclose%
\pgfusepath{stroke,fill}%
\end{pgfscope}%
\begin{pgfscope}%
\pgfpathrectangle{\pgfqpoint{0.970666in}{0.566125in}}{\pgfqpoint{5.699255in}{2.685432in}}%
\pgfusepath{clip}%
\pgfsetbuttcap%
\pgfsetroundjoin%
\definecolor{currentfill}{rgb}{0.000000,0.000000,0.000000}%
\pgfsetfillcolor{currentfill}%
\pgfsetlinewidth{1.003750pt}%
\definecolor{currentstroke}{rgb}{0.000000,0.000000,0.000000}%
\pgfsetstrokecolor{currentstroke}%
\pgfsetdash{}{0pt}%
\pgfpathmoveto{\pgfqpoint{3.140269in}{1.835875in}}%
\pgfpathcurveto{\pgfqpoint{3.151320in}{1.835875in}}{\pgfqpoint{3.161919in}{1.840266in}}{\pgfqpoint{3.169732in}{1.848079in}}%
\pgfpathcurveto{\pgfqpoint{3.177546in}{1.855893in}}{\pgfqpoint{3.181936in}{1.866492in}}{\pgfqpoint{3.181936in}{1.877542in}}%
\pgfpathcurveto{\pgfqpoint{3.181936in}{1.888592in}}{\pgfqpoint{3.177546in}{1.899191in}}{\pgfqpoint{3.169732in}{1.907005in}}%
\pgfpathcurveto{\pgfqpoint{3.161919in}{1.914818in}}{\pgfqpoint{3.151320in}{1.919209in}}{\pgfqpoint{3.140269in}{1.919209in}}%
\pgfpathcurveto{\pgfqpoint{3.129219in}{1.919209in}}{\pgfqpoint{3.118620in}{1.914818in}}{\pgfqpoint{3.110807in}{1.907005in}}%
\pgfpathcurveto{\pgfqpoint{3.102993in}{1.899191in}}{\pgfqpoint{3.098603in}{1.888592in}}{\pgfqpoint{3.098603in}{1.877542in}}%
\pgfpathcurveto{\pgfqpoint{3.098603in}{1.866492in}}{\pgfqpoint{3.102993in}{1.855893in}}{\pgfqpoint{3.110807in}{1.848079in}}%
\pgfpathcurveto{\pgfqpoint{3.118620in}{1.840266in}}{\pgfqpoint{3.129219in}{1.835875in}}{\pgfqpoint{3.140269in}{1.835875in}}%
\pgfpathclose%
\pgfusepath{stroke,fill}%
\end{pgfscope}%
\begin{pgfscope}%
\pgfpathrectangle{\pgfqpoint{0.970666in}{0.566125in}}{\pgfqpoint{5.699255in}{2.685432in}}%
\pgfusepath{clip}%
\pgfsetbuttcap%
\pgfsetroundjoin%
\definecolor{currentfill}{rgb}{0.000000,0.000000,0.000000}%
\pgfsetfillcolor{currentfill}%
\pgfsetlinewidth{1.003750pt}%
\definecolor{currentstroke}{rgb}{0.000000,0.000000,0.000000}%
\pgfsetstrokecolor{currentstroke}%
\pgfsetdash{}{0pt}%
\pgfpathmoveto{\pgfqpoint{2.945977in}{1.835875in}}%
\pgfpathcurveto{\pgfqpoint{2.957027in}{1.835875in}}{\pgfqpoint{2.967626in}{1.840266in}}{\pgfqpoint{2.975439in}{1.848079in}}%
\pgfpathcurveto{\pgfqpoint{2.983253in}{1.855893in}}{\pgfqpoint{2.987643in}{1.866492in}}{\pgfqpoint{2.987643in}{1.877542in}}%
\pgfpathcurveto{\pgfqpoint{2.987643in}{1.888592in}}{\pgfqpoint{2.983253in}{1.899191in}}{\pgfqpoint{2.975439in}{1.907005in}}%
\pgfpathcurveto{\pgfqpoint{2.967626in}{1.914818in}}{\pgfqpoint{2.957027in}{1.919209in}}{\pgfqpoint{2.945977in}{1.919209in}}%
\pgfpathcurveto{\pgfqpoint{2.934926in}{1.919209in}}{\pgfqpoint{2.924327in}{1.914818in}}{\pgfqpoint{2.916514in}{1.907005in}}%
\pgfpathcurveto{\pgfqpoint{2.908700in}{1.899191in}}{\pgfqpoint{2.904310in}{1.888592in}}{\pgfqpoint{2.904310in}{1.877542in}}%
\pgfpathcurveto{\pgfqpoint{2.904310in}{1.866492in}}{\pgfqpoint{2.908700in}{1.855893in}}{\pgfqpoint{2.916514in}{1.848079in}}%
\pgfpathcurveto{\pgfqpoint{2.924327in}{1.840266in}}{\pgfqpoint{2.934926in}{1.835875in}}{\pgfqpoint{2.945977in}{1.835875in}}%
\pgfpathclose%
\pgfusepath{stroke,fill}%
\end{pgfscope}%
\begin{pgfscope}%
\pgfpathrectangle{\pgfqpoint{0.970666in}{0.566125in}}{\pgfqpoint{5.699255in}{2.685432in}}%
\pgfusepath{clip}%
\pgfsetbuttcap%
\pgfsetroundjoin%
\definecolor{currentfill}{rgb}{0.000000,0.000000,0.000000}%
\pgfsetfillcolor{currentfill}%
\pgfsetlinewidth{1.003750pt}%
\definecolor{currentstroke}{rgb}{0.000000,0.000000,0.000000}%
\pgfsetstrokecolor{currentstroke}%
\pgfsetdash{}{0pt}%
\pgfpathmoveto{\pgfqpoint{2.654537in}{1.741979in}}%
\pgfpathcurveto{\pgfqpoint{2.665588in}{1.741979in}}{\pgfqpoint{2.676187in}{1.746369in}}{\pgfqpoint{2.684000in}{1.754183in}}%
\pgfpathcurveto{\pgfqpoint{2.691814in}{1.761997in}}{\pgfqpoint{2.696204in}{1.772596in}}{\pgfqpoint{2.696204in}{1.783646in}}%
\pgfpathcurveto{\pgfqpoint{2.696204in}{1.794696in}}{\pgfqpoint{2.691814in}{1.805295in}}{\pgfqpoint{2.684000in}{1.813108in}}%
\pgfpathcurveto{\pgfqpoint{2.676187in}{1.820922in}}{\pgfqpoint{2.665588in}{1.825312in}}{\pgfqpoint{2.654537in}{1.825312in}}%
\pgfpathcurveto{\pgfqpoint{2.643487in}{1.825312in}}{\pgfqpoint{2.632888in}{1.820922in}}{\pgfqpoint{2.625075in}{1.813108in}}%
\pgfpathcurveto{\pgfqpoint{2.617261in}{1.805295in}}{\pgfqpoint{2.612871in}{1.794696in}}{\pgfqpoint{2.612871in}{1.783646in}}%
\pgfpathcurveto{\pgfqpoint{2.612871in}{1.772596in}}{\pgfqpoint{2.617261in}{1.761997in}}{\pgfqpoint{2.625075in}{1.754183in}}%
\pgfpathcurveto{\pgfqpoint{2.632888in}{1.746369in}}{\pgfqpoint{2.643487in}{1.741979in}}{\pgfqpoint{2.654537in}{1.741979in}}%
\pgfpathclose%
\pgfusepath{stroke,fill}%
\end{pgfscope}%
\begin{pgfscope}%
\pgfpathrectangle{\pgfqpoint{0.970666in}{0.566125in}}{\pgfqpoint{5.699255in}{2.685432in}}%
\pgfusepath{clip}%
\pgfsetbuttcap%
\pgfsetroundjoin%
\definecolor{currentfill}{rgb}{0.000000,0.000000,0.000000}%
\pgfsetfillcolor{currentfill}%
\pgfsetlinewidth{1.003750pt}%
\definecolor{currentstroke}{rgb}{0.000000,0.000000,0.000000}%
\pgfsetstrokecolor{currentstroke}%
\pgfsetdash{}{0pt}%
\pgfpathmoveto{\pgfqpoint{2.557391in}{1.882823in}}%
\pgfpathcurveto{\pgfqpoint{2.568441in}{1.882823in}}{\pgfqpoint{2.579040in}{1.887214in}}{\pgfqpoint{2.586854in}{1.895027in}}%
\pgfpathcurveto{\pgfqpoint{2.594667in}{1.902841in}}{\pgfqpoint{2.599058in}{1.913440in}}{\pgfqpoint{2.599058in}{1.924490in}}%
\pgfpathcurveto{\pgfqpoint{2.599058in}{1.935540in}}{\pgfqpoint{2.594667in}{1.946139in}}{\pgfqpoint{2.586854in}{1.953953in}}%
\pgfpathcurveto{\pgfqpoint{2.579040in}{1.961766in}}{\pgfqpoint{2.568441in}{1.966157in}}{\pgfqpoint{2.557391in}{1.966157in}}%
\pgfpathcurveto{\pgfqpoint{2.546341in}{1.966157in}}{\pgfqpoint{2.535742in}{1.961766in}}{\pgfqpoint{2.527928in}{1.953953in}}%
\pgfpathcurveto{\pgfqpoint{2.520115in}{1.946139in}}{\pgfqpoint{2.515724in}{1.935540in}}{\pgfqpoint{2.515724in}{1.924490in}}%
\pgfpathcurveto{\pgfqpoint{2.515724in}{1.913440in}}{\pgfqpoint{2.520115in}{1.902841in}}{\pgfqpoint{2.527928in}{1.895027in}}%
\pgfpathcurveto{\pgfqpoint{2.535742in}{1.887214in}}{\pgfqpoint{2.546341in}{1.882823in}}{\pgfqpoint{2.557391in}{1.882823in}}%
\pgfpathclose%
\pgfusepath{stroke,fill}%
\end{pgfscope}%
\begin{pgfscope}%
\pgfpathrectangle{\pgfqpoint{0.970666in}{0.566125in}}{\pgfqpoint{5.699255in}{2.685432in}}%
\pgfusepath{clip}%
\pgfsetbuttcap%
\pgfsetroundjoin%
\definecolor{currentfill}{rgb}{0.000000,0.000000,0.000000}%
\pgfsetfillcolor{currentfill}%
\pgfsetlinewidth{1.003750pt}%
\definecolor{currentstroke}{rgb}{0.000000,0.000000,0.000000}%
\pgfsetstrokecolor{currentstroke}%
\pgfsetdash{}{0pt}%
\pgfpathmoveto{\pgfqpoint{2.201188in}{1.898473in}}%
\pgfpathcurveto{\pgfqpoint{2.212238in}{1.898473in}}{\pgfqpoint{2.222837in}{1.902863in}}{\pgfqpoint{2.230650in}{1.910677in}}%
\pgfpathcurveto{\pgfqpoint{2.238464in}{1.918490in}}{\pgfqpoint{2.242854in}{1.929089in}}{\pgfqpoint{2.242854in}{1.940139in}}%
\pgfpathcurveto{\pgfqpoint{2.242854in}{1.951190in}}{\pgfqpoint{2.238464in}{1.961789in}}{\pgfqpoint{2.230650in}{1.969602in}}%
\pgfpathcurveto{\pgfqpoint{2.222837in}{1.977416in}}{\pgfqpoint{2.212238in}{1.981806in}}{\pgfqpoint{2.201188in}{1.981806in}}%
\pgfpathcurveto{\pgfqpoint{2.190137in}{1.981806in}}{\pgfqpoint{2.179538in}{1.977416in}}{\pgfqpoint{2.171725in}{1.969602in}}%
\pgfpathcurveto{\pgfqpoint{2.163911in}{1.961789in}}{\pgfqpoint{2.159521in}{1.951190in}}{\pgfqpoint{2.159521in}{1.940139in}}%
\pgfpathcurveto{\pgfqpoint{2.159521in}{1.929089in}}{\pgfqpoint{2.163911in}{1.918490in}}{\pgfqpoint{2.171725in}{1.910677in}}%
\pgfpathcurveto{\pgfqpoint{2.179538in}{1.902863in}}{\pgfqpoint{2.190137in}{1.898473in}}{\pgfqpoint{2.201188in}{1.898473in}}%
\pgfpathclose%
\pgfusepath{stroke,fill}%
\end{pgfscope}%
\begin{pgfscope}%
\pgfpathrectangle{\pgfqpoint{0.970666in}{0.566125in}}{\pgfqpoint{5.699255in}{2.685432in}}%
\pgfusepath{clip}%
\pgfsetbuttcap%
\pgfsetroundjoin%
\definecolor{currentfill}{rgb}{0.000000,0.000000,0.000000}%
\pgfsetfillcolor{currentfill}%
\pgfsetlinewidth{1.003750pt}%
\definecolor{currentstroke}{rgb}{0.000000,0.000000,0.000000}%
\pgfsetstrokecolor{currentstroke}%
\pgfsetdash{}{0pt}%
\pgfpathmoveto{\pgfqpoint{2.201188in}{2.054966in}}%
\pgfpathcurveto{\pgfqpoint{2.212238in}{2.054966in}}{\pgfqpoint{2.222837in}{2.059357in}}{\pgfqpoint{2.230650in}{2.067170in}}%
\pgfpathcurveto{\pgfqpoint{2.238464in}{2.074984in}}{\pgfqpoint{2.242854in}{2.085583in}}{\pgfqpoint{2.242854in}{2.096633in}}%
\pgfpathcurveto{\pgfqpoint{2.242854in}{2.107683in}}{\pgfqpoint{2.238464in}{2.118282in}}{\pgfqpoint{2.230650in}{2.126096in}}%
\pgfpathcurveto{\pgfqpoint{2.222837in}{2.133910in}}{\pgfqpoint{2.212238in}{2.138300in}}{\pgfqpoint{2.201188in}{2.138300in}}%
\pgfpathcurveto{\pgfqpoint{2.190137in}{2.138300in}}{\pgfqpoint{2.179538in}{2.133910in}}{\pgfqpoint{2.171725in}{2.126096in}}%
\pgfpathcurveto{\pgfqpoint{2.163911in}{2.118282in}}{\pgfqpoint{2.159521in}{2.107683in}}{\pgfqpoint{2.159521in}{2.096633in}}%
\pgfpathcurveto{\pgfqpoint{2.159521in}{2.085583in}}{\pgfqpoint{2.163911in}{2.074984in}}{\pgfqpoint{2.171725in}{2.067170in}}%
\pgfpathcurveto{\pgfqpoint{2.179538in}{2.059357in}}{\pgfqpoint{2.190137in}{2.054966in}}{\pgfqpoint{2.201188in}{2.054966in}}%
\pgfpathclose%
\pgfusepath{stroke,fill}%
\end{pgfscope}%
\begin{pgfscope}%
\pgfpathrectangle{\pgfqpoint{0.970666in}{0.566125in}}{\pgfqpoint{5.699255in}{2.685432in}}%
\pgfusepath{clip}%
\pgfsetbuttcap%
\pgfsetroundjoin%
\definecolor{currentfill}{rgb}{0.000000,0.000000,0.000000}%
\pgfsetfillcolor{currentfill}%
\pgfsetlinewidth{1.003750pt}%
\definecolor{currentstroke}{rgb}{0.000000,0.000000,0.000000}%
\pgfsetstrokecolor{currentstroke}%
\pgfsetdash{}{0pt}%
\pgfpathmoveto{\pgfqpoint{2.363098in}{2.117564in}}%
\pgfpathcurveto{\pgfqpoint{2.374148in}{2.117564in}}{\pgfqpoint{2.384747in}{2.121954in}}{\pgfqpoint{2.392561in}{2.129768in}}%
\pgfpathcurveto{\pgfqpoint{2.400375in}{2.137581in}}{\pgfqpoint{2.404765in}{2.148180in}}{\pgfqpoint{2.404765in}{2.159231in}}%
\pgfpathcurveto{\pgfqpoint{2.404765in}{2.170281in}}{\pgfqpoint{2.400375in}{2.180880in}}{\pgfqpoint{2.392561in}{2.188693in}}%
\pgfpathcurveto{\pgfqpoint{2.384747in}{2.196507in}}{\pgfqpoint{2.374148in}{2.200897in}}{\pgfqpoint{2.363098in}{2.200897in}}%
\pgfpathcurveto{\pgfqpoint{2.352048in}{2.200897in}}{\pgfqpoint{2.341449in}{2.196507in}}{\pgfqpoint{2.333635in}{2.188693in}}%
\pgfpathcurveto{\pgfqpoint{2.325822in}{2.180880in}}{\pgfqpoint{2.321432in}{2.170281in}}{\pgfqpoint{2.321432in}{2.159231in}}%
\pgfpathcurveto{\pgfqpoint{2.321432in}{2.148180in}}{\pgfqpoint{2.325822in}{2.137581in}}{\pgfqpoint{2.333635in}{2.129768in}}%
\pgfpathcurveto{\pgfqpoint{2.341449in}{2.121954in}}{\pgfqpoint{2.352048in}{2.117564in}}{\pgfqpoint{2.363098in}{2.117564in}}%
\pgfpathclose%
\pgfusepath{stroke,fill}%
\end{pgfscope}%
\begin{pgfscope}%
\pgfpathrectangle{\pgfqpoint{0.970666in}{0.566125in}}{\pgfqpoint{5.699255in}{2.685432in}}%
\pgfusepath{clip}%
\pgfsetbuttcap%
\pgfsetroundjoin%
\definecolor{currentfill}{rgb}{0.000000,0.000000,0.000000}%
\pgfsetfillcolor{currentfill}%
\pgfsetlinewidth{1.003750pt}%
\definecolor{currentstroke}{rgb}{0.000000,0.000000,0.000000}%
\pgfsetstrokecolor{currentstroke}%
\pgfsetdash{}{0pt}%
\pgfpathmoveto{\pgfqpoint{2.686920in}{2.446201in}}%
\pgfpathcurveto{\pgfqpoint{2.697970in}{2.446201in}}{\pgfqpoint{2.708569in}{2.450591in}}{\pgfqpoint{2.716382in}{2.458405in}}%
\pgfpathcurveto{\pgfqpoint{2.724196in}{2.466218in}}{\pgfqpoint{2.728586in}{2.476817in}}{\pgfqpoint{2.728586in}{2.487867in}}%
\pgfpathcurveto{\pgfqpoint{2.728586in}{2.498918in}}{\pgfqpoint{2.724196in}{2.509517in}}{\pgfqpoint{2.716382in}{2.517330in}}%
\pgfpathcurveto{\pgfqpoint{2.708569in}{2.525144in}}{\pgfqpoint{2.697970in}{2.529534in}}{\pgfqpoint{2.686920in}{2.529534in}}%
\pgfpathcurveto{\pgfqpoint{2.675869in}{2.529534in}}{\pgfqpoint{2.665270in}{2.525144in}}{\pgfqpoint{2.657457in}{2.517330in}}%
\pgfpathcurveto{\pgfqpoint{2.649643in}{2.509517in}}{\pgfqpoint{2.645253in}{2.498918in}}{\pgfqpoint{2.645253in}{2.487867in}}%
\pgfpathcurveto{\pgfqpoint{2.645253in}{2.476817in}}{\pgfqpoint{2.649643in}{2.466218in}}{\pgfqpoint{2.657457in}{2.458405in}}%
\pgfpathcurveto{\pgfqpoint{2.665270in}{2.450591in}}{\pgfqpoint{2.675869in}{2.446201in}}{\pgfqpoint{2.686920in}{2.446201in}}%
\pgfpathclose%
\pgfusepath{stroke,fill}%
\end{pgfscope}%
\begin{pgfscope}%
\pgfpathrectangle{\pgfqpoint{0.970666in}{0.566125in}}{\pgfqpoint{5.699255in}{2.685432in}}%
\pgfusepath{clip}%
\pgfsetbuttcap%
\pgfsetroundjoin%
\definecolor{currentfill}{rgb}{0.000000,0.000000,0.000000}%
\pgfsetfillcolor{currentfill}%
\pgfsetlinewidth{1.003750pt}%
\definecolor{currentstroke}{rgb}{0.000000,0.000000,0.000000}%
\pgfsetstrokecolor{currentstroke}%
\pgfsetdash{}{0pt}%
\pgfpathmoveto{\pgfqpoint{2.945977in}{2.555746in}}%
\pgfpathcurveto{\pgfqpoint{2.957027in}{2.555746in}}{\pgfqpoint{2.967626in}{2.560137in}}{\pgfqpoint{2.975439in}{2.567950in}}%
\pgfpathcurveto{\pgfqpoint{2.983253in}{2.575764in}}{\pgfqpoint{2.987643in}{2.586363in}}{\pgfqpoint{2.987643in}{2.597413in}}%
\pgfpathcurveto{\pgfqpoint{2.987643in}{2.608463in}}{\pgfqpoint{2.983253in}{2.619062in}}{\pgfqpoint{2.975439in}{2.626876in}}%
\pgfpathcurveto{\pgfqpoint{2.967626in}{2.634689in}}{\pgfqpoint{2.957027in}{2.639080in}}{\pgfqpoint{2.945977in}{2.639080in}}%
\pgfpathcurveto{\pgfqpoint{2.934926in}{2.639080in}}{\pgfqpoint{2.924327in}{2.634689in}}{\pgfqpoint{2.916514in}{2.626876in}}%
\pgfpathcurveto{\pgfqpoint{2.908700in}{2.619062in}}{\pgfqpoint{2.904310in}{2.608463in}}{\pgfqpoint{2.904310in}{2.597413in}}%
\pgfpathcurveto{\pgfqpoint{2.904310in}{2.586363in}}{\pgfqpoint{2.908700in}{2.575764in}}{\pgfqpoint{2.916514in}{2.567950in}}%
\pgfpathcurveto{\pgfqpoint{2.924327in}{2.560137in}}{\pgfqpoint{2.934926in}{2.555746in}}{\pgfqpoint{2.945977in}{2.555746in}}%
\pgfpathclose%
\pgfusepath{stroke,fill}%
\end{pgfscope}%
\begin{pgfscope}%
\pgfpathrectangle{\pgfqpoint{0.970666in}{0.566125in}}{\pgfqpoint{5.699255in}{2.685432in}}%
\pgfusepath{clip}%
\pgfsetbuttcap%
\pgfsetroundjoin%
\definecolor{currentfill}{rgb}{0.000000,0.000000,0.000000}%
\pgfsetfillcolor{currentfill}%
\pgfsetlinewidth{1.003750pt}%
\definecolor{currentstroke}{rgb}{0.000000,0.000000,0.000000}%
\pgfsetstrokecolor{currentstroke}%
\pgfsetdash{}{0pt}%
\pgfpathmoveto{\pgfqpoint{3.172652in}{2.680941in}}%
\pgfpathcurveto{\pgfqpoint{3.183702in}{2.680941in}}{\pgfqpoint{3.194301in}{2.685332in}}{\pgfqpoint{3.202114in}{2.693145in}}%
\pgfpathcurveto{\pgfqpoint{3.209928in}{2.700959in}}{\pgfqpoint{3.214318in}{2.711558in}}{\pgfqpoint{3.214318in}{2.722608in}}%
\pgfpathcurveto{\pgfqpoint{3.214318in}{2.733658in}}{\pgfqpoint{3.209928in}{2.744257in}}{\pgfqpoint{3.202114in}{2.752071in}}%
\pgfpathcurveto{\pgfqpoint{3.194301in}{2.759884in}}{\pgfqpoint{3.183702in}{2.764275in}}{\pgfqpoint{3.172652in}{2.764275in}}%
\pgfpathcurveto{\pgfqpoint{3.161601in}{2.764275in}}{\pgfqpoint{3.151002in}{2.759884in}}{\pgfqpoint{3.143189in}{2.752071in}}%
\pgfpathcurveto{\pgfqpoint{3.135375in}{2.744257in}}{\pgfqpoint{3.130985in}{2.733658in}}{\pgfqpoint{3.130985in}{2.722608in}}%
\pgfpathcurveto{\pgfqpoint{3.130985in}{2.711558in}}{\pgfqpoint{3.135375in}{2.700959in}}{\pgfqpoint{3.143189in}{2.693145in}}%
\pgfpathcurveto{\pgfqpoint{3.151002in}{2.685332in}}{\pgfqpoint{3.161601in}{2.680941in}}{\pgfqpoint{3.172652in}{2.680941in}}%
\pgfpathclose%
\pgfusepath{stroke,fill}%
\end{pgfscope}%
\begin{pgfscope}%
\pgfpathrectangle{\pgfqpoint{0.970666in}{0.566125in}}{\pgfqpoint{5.699255in}{2.685432in}}%
\pgfusepath{clip}%
\pgfsetbuttcap%
\pgfsetroundjoin%
\definecolor{currentfill}{rgb}{0.000000,0.000000,0.000000}%
\pgfsetfillcolor{currentfill}%
\pgfsetlinewidth{1.003750pt}%
\definecolor{currentstroke}{rgb}{0.000000,0.000000,0.000000}%
\pgfsetstrokecolor{currentstroke}%
\pgfsetdash{}{0pt}%
\pgfpathmoveto{\pgfqpoint{3.528855in}{2.727889in}}%
\pgfpathcurveto{\pgfqpoint{3.539905in}{2.727889in}}{\pgfqpoint{3.550504in}{2.732280in}}{\pgfqpoint{3.558318in}{2.740093in}}%
\pgfpathcurveto{\pgfqpoint{3.566131in}{2.747907in}}{\pgfqpoint{3.570522in}{2.758506in}}{\pgfqpoint{3.570522in}{2.769556in}}%
\pgfpathcurveto{\pgfqpoint{3.570522in}{2.780606in}}{\pgfqpoint{3.566131in}{2.791205in}}{\pgfqpoint{3.558318in}{2.799019in}}%
\pgfpathcurveto{\pgfqpoint{3.550504in}{2.806832in}}{\pgfqpoint{3.539905in}{2.811223in}}{\pgfqpoint{3.528855in}{2.811223in}}%
\pgfpathcurveto{\pgfqpoint{3.517805in}{2.811223in}}{\pgfqpoint{3.507206in}{2.806832in}}{\pgfqpoint{3.499392in}{2.799019in}}%
\pgfpathcurveto{\pgfqpoint{3.491579in}{2.791205in}}{\pgfqpoint{3.487188in}{2.780606in}}{\pgfqpoint{3.487188in}{2.769556in}}%
\pgfpathcurveto{\pgfqpoint{3.487188in}{2.758506in}}{\pgfqpoint{3.491579in}{2.747907in}}{\pgfqpoint{3.499392in}{2.740093in}}%
\pgfpathcurveto{\pgfqpoint{3.507206in}{2.732280in}}{\pgfqpoint{3.517805in}{2.727889in}}{\pgfqpoint{3.528855in}{2.727889in}}%
\pgfpathclose%
\pgfusepath{stroke,fill}%
\end{pgfscope}%
\begin{pgfscope}%
\pgfpathrectangle{\pgfqpoint{0.970666in}{0.566125in}}{\pgfqpoint{5.699255in}{2.685432in}}%
\pgfusepath{clip}%
\pgfsetbuttcap%
\pgfsetroundjoin%
\definecolor{currentfill}{rgb}{0.000000,0.000000,0.000000}%
\pgfsetfillcolor{currentfill}%
\pgfsetlinewidth{1.003750pt}%
\definecolor{currentstroke}{rgb}{0.000000,0.000000,0.000000}%
\pgfsetstrokecolor{currentstroke}%
\pgfsetdash{}{0pt}%
\pgfpathmoveto{\pgfqpoint{4.144116in}{2.821786in}}%
\pgfpathcurveto{\pgfqpoint{4.155166in}{2.821786in}}{\pgfqpoint{4.165765in}{2.826176in}}{\pgfqpoint{4.173578in}{2.833990in}}%
\pgfpathcurveto{\pgfqpoint{4.181392in}{2.841803in}}{\pgfqpoint{4.185782in}{2.852402in}}{\pgfqpoint{4.185782in}{2.863452in}}%
\pgfpathcurveto{\pgfqpoint{4.185782in}{2.874502in}}{\pgfqpoint{4.181392in}{2.885101in}}{\pgfqpoint{4.173578in}{2.892915in}}%
\pgfpathcurveto{\pgfqpoint{4.165765in}{2.900729in}}{\pgfqpoint{4.155166in}{2.905119in}}{\pgfqpoint{4.144116in}{2.905119in}}%
\pgfpathcurveto{\pgfqpoint{4.133065in}{2.905119in}}{\pgfqpoint{4.122466in}{2.900729in}}{\pgfqpoint{4.114653in}{2.892915in}}%
\pgfpathcurveto{\pgfqpoint{4.106839in}{2.885101in}}{\pgfqpoint{4.102449in}{2.874502in}}{\pgfqpoint{4.102449in}{2.863452in}}%
\pgfpathcurveto{\pgfqpoint{4.102449in}{2.852402in}}{\pgfqpoint{4.106839in}{2.841803in}}{\pgfqpoint{4.114653in}{2.833990in}}%
\pgfpathcurveto{\pgfqpoint{4.122466in}{2.826176in}}{\pgfqpoint{4.133065in}{2.821786in}}{\pgfqpoint{4.144116in}{2.821786in}}%
\pgfpathclose%
\pgfusepath{stroke,fill}%
\end{pgfscope}%
\begin{pgfscope}%
\pgfpathrectangle{\pgfqpoint{0.970666in}{0.566125in}}{\pgfqpoint{5.699255in}{2.685432in}}%
\pgfusepath{clip}%
\pgfsetbuttcap%
\pgfsetroundjoin%
\definecolor{currentfill}{rgb}{0.000000,0.000000,0.000000}%
\pgfsetfillcolor{currentfill}%
\pgfsetlinewidth{1.003750pt}%
\definecolor{currentstroke}{rgb}{0.000000,0.000000,0.000000}%
\pgfsetstrokecolor{currentstroke}%
\pgfsetdash{}{0pt}%
\pgfpathmoveto{\pgfqpoint{4.338408in}{2.853084in}}%
\pgfpathcurveto{\pgfqpoint{4.349458in}{2.853084in}}{\pgfqpoint{4.360057in}{2.857475in}}{\pgfqpoint{4.367871in}{2.865288in}}%
\pgfpathcurveto{\pgfqpoint{4.375685in}{2.873102in}}{\pgfqpoint{4.380075in}{2.883701in}}{\pgfqpoint{4.380075in}{2.894751in}}%
\pgfpathcurveto{\pgfqpoint{4.380075in}{2.905801in}}{\pgfqpoint{4.375685in}{2.916400in}}{\pgfqpoint{4.367871in}{2.924214in}}%
\pgfpathcurveto{\pgfqpoint{4.360057in}{2.932027in}}{\pgfqpoint{4.349458in}{2.936418in}}{\pgfqpoint{4.338408in}{2.936418in}}%
\pgfpathcurveto{\pgfqpoint{4.327358in}{2.936418in}}{\pgfqpoint{4.316759in}{2.932027in}}{\pgfqpoint{4.308946in}{2.924214in}}%
\pgfpathcurveto{\pgfqpoint{4.301132in}{2.916400in}}{\pgfqpoint{4.296742in}{2.905801in}}{\pgfqpoint{4.296742in}{2.894751in}}%
\pgfpathcurveto{\pgfqpoint{4.296742in}{2.883701in}}{\pgfqpoint{4.301132in}{2.873102in}}{\pgfqpoint{4.308946in}{2.865288in}}%
\pgfpathcurveto{\pgfqpoint{4.316759in}{2.857475in}}{\pgfqpoint{4.327358in}{2.853084in}}{\pgfqpoint{4.338408in}{2.853084in}}%
\pgfpathclose%
\pgfusepath{stroke,fill}%
\end{pgfscope}%
\begin{pgfscope}%
\pgfpathrectangle{\pgfqpoint{0.970666in}{0.566125in}}{\pgfqpoint{5.699255in}{2.685432in}}%
\pgfusepath{clip}%
\pgfsetbuttcap%
\pgfsetroundjoin%
\definecolor{currentfill}{rgb}{0.000000,0.000000,0.000000}%
\pgfsetfillcolor{currentfill}%
\pgfsetlinewidth{1.003750pt}%
\definecolor{currentstroke}{rgb}{0.000000,0.000000,0.000000}%
\pgfsetstrokecolor{currentstroke}%
\pgfsetdash{}{0pt}%
\pgfpathmoveto{\pgfqpoint{4.662230in}{2.821786in}}%
\pgfpathcurveto{\pgfqpoint{4.673280in}{2.821786in}}{\pgfqpoint{4.683879in}{2.826176in}}{\pgfqpoint{4.691692in}{2.833990in}}%
\pgfpathcurveto{\pgfqpoint{4.699506in}{2.841803in}}{\pgfqpoint{4.703896in}{2.852402in}}{\pgfqpoint{4.703896in}{2.863452in}}%
\pgfpathcurveto{\pgfqpoint{4.703896in}{2.874502in}}{\pgfqpoint{4.699506in}{2.885101in}}{\pgfqpoint{4.691692in}{2.892915in}}%
\pgfpathcurveto{\pgfqpoint{4.683879in}{2.900729in}}{\pgfqpoint{4.673280in}{2.905119in}}{\pgfqpoint{4.662230in}{2.905119in}}%
\pgfpathcurveto{\pgfqpoint{4.651180in}{2.905119in}}{\pgfqpoint{4.640580in}{2.900729in}}{\pgfqpoint{4.632767in}{2.892915in}}%
\pgfpathcurveto{\pgfqpoint{4.624953in}{2.885101in}}{\pgfqpoint{4.620563in}{2.874502in}}{\pgfqpoint{4.620563in}{2.863452in}}%
\pgfpathcurveto{\pgfqpoint{4.620563in}{2.852402in}}{\pgfqpoint{4.624953in}{2.841803in}}{\pgfqpoint{4.632767in}{2.833990in}}%
\pgfpathcurveto{\pgfqpoint{4.640580in}{2.826176in}}{\pgfqpoint{4.651180in}{2.821786in}}{\pgfqpoint{4.662230in}{2.821786in}}%
\pgfpathclose%
\pgfusepath{stroke,fill}%
\end{pgfscope}%
\begin{pgfscope}%
\pgfpathrectangle{\pgfqpoint{0.970666in}{0.566125in}}{\pgfqpoint{5.699255in}{2.685432in}}%
\pgfusepath{clip}%
\pgfsetbuttcap%
\pgfsetroundjoin%
\definecolor{currentfill}{rgb}{0.000000,0.000000,0.000000}%
\pgfsetfillcolor{currentfill}%
\pgfsetlinewidth{1.003750pt}%
\definecolor{currentstroke}{rgb}{0.000000,0.000000,0.000000}%
\pgfsetstrokecolor{currentstroke}%
\pgfsetdash{}{0pt}%
\pgfpathmoveto{\pgfqpoint{4.824140in}{2.790487in}}%
\pgfpathcurveto{\pgfqpoint{4.835190in}{2.790487in}}{\pgfqpoint{4.845789in}{2.794877in}}{\pgfqpoint{4.853603in}{2.802691in}}%
\pgfpathcurveto{\pgfqpoint{4.861417in}{2.810504in}}{\pgfqpoint{4.865807in}{2.821103in}}{\pgfqpoint{4.865807in}{2.832154in}}%
\pgfpathcurveto{\pgfqpoint{4.865807in}{2.843204in}}{\pgfqpoint{4.861417in}{2.853803in}}{\pgfqpoint{4.853603in}{2.861616in}}%
\pgfpathcurveto{\pgfqpoint{4.845789in}{2.869430in}}{\pgfqpoint{4.835190in}{2.873820in}}{\pgfqpoint{4.824140in}{2.873820in}}%
\pgfpathcurveto{\pgfqpoint{4.813090in}{2.873820in}}{\pgfqpoint{4.802491in}{2.869430in}}{\pgfqpoint{4.794678in}{2.861616in}}%
\pgfpathcurveto{\pgfqpoint{4.786864in}{2.853803in}}{\pgfqpoint{4.782474in}{2.843204in}}{\pgfqpoint{4.782474in}{2.832154in}}%
\pgfpathcurveto{\pgfqpoint{4.782474in}{2.821103in}}{\pgfqpoint{4.786864in}{2.810504in}}{\pgfqpoint{4.794678in}{2.802691in}}%
\pgfpathcurveto{\pgfqpoint{4.802491in}{2.794877in}}{\pgfqpoint{4.813090in}{2.790487in}}{\pgfqpoint{4.824140in}{2.790487in}}%
\pgfpathclose%
\pgfusepath{stroke,fill}%
\end{pgfscope}%
\begin{pgfscope}%
\pgfpathrectangle{\pgfqpoint{0.970666in}{0.566125in}}{\pgfqpoint{5.699255in}{2.685432in}}%
\pgfusepath{clip}%
\pgfsetbuttcap%
\pgfsetroundjoin%
\definecolor{currentfill}{rgb}{0.000000,0.000000,0.000000}%
\pgfsetfillcolor{currentfill}%
\pgfsetlinewidth{1.003750pt}%
\definecolor{currentstroke}{rgb}{0.000000,0.000000,0.000000}%
\pgfsetstrokecolor{currentstroke}%
\pgfsetdash{}{0pt}%
\pgfpathmoveto{\pgfqpoint{4.888905in}{2.837435in}}%
\pgfpathcurveto{\pgfqpoint{4.899955in}{2.837435in}}{\pgfqpoint{4.910554in}{2.841825in}}{\pgfqpoint{4.918367in}{2.849639in}}%
\pgfpathcurveto{\pgfqpoint{4.926181in}{2.857453in}}{\pgfqpoint{4.930571in}{2.868052in}}{\pgfqpoint{4.930571in}{2.879102in}}%
\pgfpathcurveto{\pgfqpoint{4.930571in}{2.890152in}}{\pgfqpoint{4.926181in}{2.900751in}}{\pgfqpoint{4.918367in}{2.908564in}}%
\pgfpathcurveto{\pgfqpoint{4.910554in}{2.916378in}}{\pgfqpoint{4.899955in}{2.920768in}}{\pgfqpoint{4.888905in}{2.920768in}}%
\pgfpathcurveto{\pgfqpoint{4.877854in}{2.920768in}}{\pgfqpoint{4.867255in}{2.916378in}}{\pgfqpoint{4.859442in}{2.908564in}}%
\pgfpathcurveto{\pgfqpoint{4.851628in}{2.900751in}}{\pgfqpoint{4.847238in}{2.890152in}}{\pgfqpoint{4.847238in}{2.879102in}}%
\pgfpathcurveto{\pgfqpoint{4.847238in}{2.868052in}}{\pgfqpoint{4.851628in}{2.857453in}}{\pgfqpoint{4.859442in}{2.849639in}}%
\pgfpathcurveto{\pgfqpoint{4.867255in}{2.841825in}}{\pgfqpoint{4.877854in}{2.837435in}}{\pgfqpoint{4.888905in}{2.837435in}}%
\pgfpathclose%
\pgfusepath{stroke,fill}%
\end{pgfscope}%
\begin{pgfscope}%
\pgfpathrectangle{\pgfqpoint{0.970666in}{0.566125in}}{\pgfqpoint{5.699255in}{2.685432in}}%
\pgfusepath{clip}%
\pgfsetbuttcap%
\pgfsetroundjoin%
\definecolor{currentfill}{rgb}{0.000000,0.000000,0.000000}%
\pgfsetfillcolor{currentfill}%
\pgfsetlinewidth{1.003750pt}%
\definecolor{currentstroke}{rgb}{0.000000,0.000000,0.000000}%
\pgfsetstrokecolor{currentstroke}%
\pgfsetdash{}{0pt}%
\pgfpathmoveto{\pgfqpoint{5.115579in}{2.712240in}}%
\pgfpathcurveto{\pgfqpoint{5.126630in}{2.712240in}}{\pgfqpoint{5.137229in}{2.716630in}}{\pgfqpoint{5.145042in}{2.724444in}}%
\pgfpathcurveto{\pgfqpoint{5.152856in}{2.732258in}}{\pgfqpoint{5.157246in}{2.742857in}}{\pgfqpoint{5.157246in}{2.753907in}}%
\pgfpathcurveto{\pgfqpoint{5.157246in}{2.764957in}}{\pgfqpoint{5.152856in}{2.775556in}}{\pgfqpoint{5.145042in}{2.783369in}}%
\pgfpathcurveto{\pgfqpoint{5.137229in}{2.791183in}}{\pgfqpoint{5.126630in}{2.795573in}}{\pgfqpoint{5.115579in}{2.795573in}}%
\pgfpathcurveto{\pgfqpoint{5.104529in}{2.795573in}}{\pgfqpoint{5.093930in}{2.791183in}}{\pgfqpoint{5.086117in}{2.783369in}}%
\pgfpathcurveto{\pgfqpoint{5.078303in}{2.775556in}}{\pgfqpoint{5.073913in}{2.764957in}}{\pgfqpoint{5.073913in}{2.753907in}}%
\pgfpathcurveto{\pgfqpoint{5.073913in}{2.742857in}}{\pgfqpoint{5.078303in}{2.732258in}}{\pgfqpoint{5.086117in}{2.724444in}}%
\pgfpathcurveto{\pgfqpoint{5.093930in}{2.716630in}}{\pgfqpoint{5.104529in}{2.712240in}}{\pgfqpoint{5.115579in}{2.712240in}}%
\pgfpathclose%
\pgfusepath{stroke,fill}%
\end{pgfscope}%
\begin{pgfscope}%
\pgfpathrectangle{\pgfqpoint{0.970666in}{0.566125in}}{\pgfqpoint{5.699255in}{2.685432in}}%
\pgfusepath{clip}%
\pgfsetbuttcap%
\pgfsetroundjoin%
\definecolor{currentfill}{rgb}{0.000000,0.000000,0.000000}%
\pgfsetfillcolor{currentfill}%
\pgfsetlinewidth{1.003750pt}%
\definecolor{currentstroke}{rgb}{0.000000,0.000000,0.000000}%
\pgfsetstrokecolor{currentstroke}%
\pgfsetdash{}{0pt}%
\pgfpathmoveto{\pgfqpoint{5.471783in}{2.712240in}}%
\pgfpathcurveto{\pgfqpoint{5.482833in}{2.712240in}}{\pgfqpoint{5.493432in}{2.716630in}}{\pgfqpoint{5.501246in}{2.724444in}}%
\pgfpathcurveto{\pgfqpoint{5.509059in}{2.732258in}}{\pgfqpoint{5.513450in}{2.742857in}}{\pgfqpoint{5.513450in}{2.753907in}}%
\pgfpathcurveto{\pgfqpoint{5.513450in}{2.764957in}}{\pgfqpoint{5.509059in}{2.775556in}}{\pgfqpoint{5.501246in}{2.783369in}}%
\pgfpathcurveto{\pgfqpoint{5.493432in}{2.791183in}}{\pgfqpoint{5.482833in}{2.795573in}}{\pgfqpoint{5.471783in}{2.795573in}}%
\pgfpathcurveto{\pgfqpoint{5.460733in}{2.795573in}}{\pgfqpoint{5.450134in}{2.791183in}}{\pgfqpoint{5.442320in}{2.783369in}}%
\pgfpathcurveto{\pgfqpoint{5.434507in}{2.775556in}}{\pgfqpoint{5.430116in}{2.764957in}}{\pgfqpoint{5.430116in}{2.753907in}}%
\pgfpathcurveto{\pgfqpoint{5.430116in}{2.742857in}}{\pgfqpoint{5.434507in}{2.732258in}}{\pgfqpoint{5.442320in}{2.724444in}}%
\pgfpathcurveto{\pgfqpoint{5.450134in}{2.716630in}}{\pgfqpoint{5.460733in}{2.712240in}}{\pgfqpoint{5.471783in}{2.712240in}}%
\pgfpathclose%
\pgfusepath{stroke,fill}%
\end{pgfscope}%
\begin{pgfscope}%
\pgfpathrectangle{\pgfqpoint{0.970666in}{0.566125in}}{\pgfqpoint{5.699255in}{2.685432in}}%
\pgfusepath{clip}%
\pgfsetbuttcap%
\pgfsetroundjoin%
\definecolor{currentfill}{rgb}{0.000000,0.000000,0.000000}%
\pgfsetfillcolor{currentfill}%
\pgfsetlinewidth{1.003750pt}%
\definecolor{currentstroke}{rgb}{0.000000,0.000000,0.000000}%
\pgfsetstrokecolor{currentstroke}%
\pgfsetdash{}{0pt}%
\pgfpathmoveto{\pgfqpoint{5.374637in}{2.446201in}}%
\pgfpathcurveto{\pgfqpoint{5.385687in}{2.446201in}}{\pgfqpoint{5.396286in}{2.450591in}}{\pgfqpoint{5.404099in}{2.458405in}}%
\pgfpathcurveto{\pgfqpoint{5.411913in}{2.466218in}}{\pgfqpoint{5.416303in}{2.476817in}}{\pgfqpoint{5.416303in}{2.487867in}}%
\pgfpathcurveto{\pgfqpoint{5.416303in}{2.498918in}}{\pgfqpoint{5.411913in}{2.509517in}}{\pgfqpoint{5.404099in}{2.517330in}}%
\pgfpathcurveto{\pgfqpoint{5.396286in}{2.525144in}}{\pgfqpoint{5.385687in}{2.529534in}}{\pgfqpoint{5.374637in}{2.529534in}}%
\pgfpathcurveto{\pgfqpoint{5.363586in}{2.529534in}}{\pgfqpoint{5.352987in}{2.525144in}}{\pgfqpoint{5.345174in}{2.517330in}}%
\pgfpathcurveto{\pgfqpoint{5.337360in}{2.509517in}}{\pgfqpoint{5.332970in}{2.498918in}}{\pgfqpoint{5.332970in}{2.487867in}}%
\pgfpathcurveto{\pgfqpoint{5.332970in}{2.476817in}}{\pgfqpoint{5.337360in}{2.466218in}}{\pgfqpoint{5.345174in}{2.458405in}}%
\pgfpathcurveto{\pgfqpoint{5.352987in}{2.450591in}}{\pgfqpoint{5.363586in}{2.446201in}}{\pgfqpoint{5.374637in}{2.446201in}}%
\pgfpathclose%
\pgfusepath{stroke,fill}%
\end{pgfscope}%
\begin{pgfscope}%
\pgfpathrectangle{\pgfqpoint{0.970666in}{0.566125in}}{\pgfqpoint{5.699255in}{2.685432in}}%
\pgfusepath{clip}%
\pgfsetbuttcap%
\pgfsetroundjoin%
\definecolor{currentfill}{rgb}{0.000000,0.000000,0.000000}%
\pgfsetfillcolor{currentfill}%
\pgfsetlinewidth{1.003750pt}%
\definecolor{currentstroke}{rgb}{0.000000,0.000000,0.000000}%
\pgfsetstrokecolor{currentstroke}%
\pgfsetdash{}{0pt}%
\pgfpathmoveto{\pgfqpoint{5.601311in}{2.321006in}}%
\pgfpathcurveto{\pgfqpoint{5.612362in}{2.321006in}}{\pgfqpoint{5.622961in}{2.325396in}}{\pgfqpoint{5.630774in}{2.333210in}}%
\pgfpathcurveto{\pgfqpoint{5.638588in}{2.341023in}}{\pgfqpoint{5.642978in}{2.351622in}}{\pgfqpoint{5.642978in}{2.362672in}}%
\pgfpathcurveto{\pgfqpoint{5.642978in}{2.373723in}}{\pgfqpoint{5.638588in}{2.384322in}}{\pgfqpoint{5.630774in}{2.392135in}}%
\pgfpathcurveto{\pgfqpoint{5.622961in}{2.399949in}}{\pgfqpoint{5.612362in}{2.404339in}}{\pgfqpoint{5.601311in}{2.404339in}}%
\pgfpathcurveto{\pgfqpoint{5.590261in}{2.404339in}}{\pgfqpoint{5.579662in}{2.399949in}}{\pgfqpoint{5.571849in}{2.392135in}}%
\pgfpathcurveto{\pgfqpoint{5.564035in}{2.384322in}}{\pgfqpoint{5.559645in}{2.373723in}}{\pgfqpoint{5.559645in}{2.362672in}}%
\pgfpathcurveto{\pgfqpoint{5.559645in}{2.351622in}}{\pgfqpoint{5.564035in}{2.341023in}}{\pgfqpoint{5.571849in}{2.333210in}}%
\pgfpathcurveto{\pgfqpoint{5.579662in}{2.325396in}}{\pgfqpoint{5.590261in}{2.321006in}}{\pgfqpoint{5.601311in}{2.321006in}}%
\pgfpathclose%
\pgfusepath{stroke,fill}%
\end{pgfscope}%
\begin{pgfscope}%
\pgfpathrectangle{\pgfqpoint{0.970666in}{0.566125in}}{\pgfqpoint{5.699255in}{2.685432in}}%
\pgfusepath{clip}%
\pgfsetbuttcap%
\pgfsetroundjoin%
\definecolor{currentfill}{rgb}{0.000000,0.000000,0.000000}%
\pgfsetfillcolor{currentfill}%
\pgfsetlinewidth{1.003750pt}%
\definecolor{currentstroke}{rgb}{0.000000,0.000000,0.000000}%
\pgfsetstrokecolor{currentstroke}%
\pgfsetdash{}{0pt}%
\pgfpathmoveto{\pgfqpoint{5.730840in}{2.180161in}}%
\pgfpathcurveto{\pgfqpoint{5.741890in}{2.180161in}}{\pgfqpoint{5.752489in}{2.184552in}}{\pgfqpoint{5.760303in}{2.192365in}}%
\pgfpathcurveto{\pgfqpoint{5.768116in}{2.200179in}}{\pgfqpoint{5.772507in}{2.210778in}}{\pgfqpoint{5.772507in}{2.221828in}}%
\pgfpathcurveto{\pgfqpoint{5.772507in}{2.232878in}}{\pgfqpoint{5.768116in}{2.243477in}}{\pgfqpoint{5.760303in}{2.251291in}}%
\pgfpathcurveto{\pgfqpoint{5.752489in}{2.259104in}}{\pgfqpoint{5.741890in}{2.263495in}}{\pgfqpoint{5.730840in}{2.263495in}}%
\pgfpathcurveto{\pgfqpoint{5.719790in}{2.263495in}}{\pgfqpoint{5.709191in}{2.259104in}}{\pgfqpoint{5.701377in}{2.251291in}}%
\pgfpathcurveto{\pgfqpoint{5.693564in}{2.243477in}}{\pgfqpoint{5.689173in}{2.232878in}}{\pgfqpoint{5.689173in}{2.221828in}}%
\pgfpathcurveto{\pgfqpoint{5.689173in}{2.210778in}}{\pgfqpoint{5.693564in}{2.200179in}}{\pgfqpoint{5.701377in}{2.192365in}}%
\pgfpathcurveto{\pgfqpoint{5.709191in}{2.184552in}}{\pgfqpoint{5.719790in}{2.180161in}}{\pgfqpoint{5.730840in}{2.180161in}}%
\pgfpathclose%
\pgfusepath{stroke,fill}%
\end{pgfscope}%
\begin{pgfscope}%
\pgfpathrectangle{\pgfqpoint{0.970666in}{0.566125in}}{\pgfqpoint{5.699255in}{2.685432in}}%
\pgfusepath{clip}%
\pgfsetbuttcap%
\pgfsetroundjoin%
\definecolor{currentfill}{rgb}{0.000000,0.000000,0.000000}%
\pgfsetfillcolor{currentfill}%
\pgfsetlinewidth{1.003750pt}%
\definecolor{currentstroke}{rgb}{0.000000,0.000000,0.000000}%
\pgfsetstrokecolor{currentstroke}%
\pgfsetdash{}{0pt}%
\pgfpathmoveto{\pgfqpoint{5.925133in}{2.367954in}}%
\pgfpathcurveto{\pgfqpoint{5.936183in}{2.367954in}}{\pgfqpoint{5.946782in}{2.372344in}}{\pgfqpoint{5.954596in}{2.380158in}}%
\pgfpathcurveto{\pgfqpoint{5.962409in}{2.387971in}}{\pgfqpoint{5.966799in}{2.398570in}}{\pgfqpoint{5.966799in}{2.409621in}}%
\pgfpathcurveto{\pgfqpoint{5.966799in}{2.420671in}}{\pgfqpoint{5.962409in}{2.431270in}}{\pgfqpoint{5.954596in}{2.439083in}}%
\pgfpathcurveto{\pgfqpoint{5.946782in}{2.446897in}}{\pgfqpoint{5.936183in}{2.451287in}}{\pgfqpoint{5.925133in}{2.451287in}}%
\pgfpathcurveto{\pgfqpoint{5.914083in}{2.451287in}}{\pgfqpoint{5.903484in}{2.446897in}}{\pgfqpoint{5.895670in}{2.439083in}}%
\pgfpathcurveto{\pgfqpoint{5.887856in}{2.431270in}}{\pgfqpoint{5.883466in}{2.420671in}}{\pgfqpoint{5.883466in}{2.409621in}}%
\pgfpathcurveto{\pgfqpoint{5.883466in}{2.398570in}}{\pgfqpoint{5.887856in}{2.387971in}}{\pgfqpoint{5.895670in}{2.380158in}}%
\pgfpathcurveto{\pgfqpoint{5.903484in}{2.372344in}}{\pgfqpoint{5.914083in}{2.367954in}}{\pgfqpoint{5.925133in}{2.367954in}}%
\pgfpathclose%
\pgfusepath{stroke,fill}%
\end{pgfscope}%
\begin{pgfscope}%
\pgfpathrectangle{\pgfqpoint{0.970666in}{0.566125in}}{\pgfqpoint{5.699255in}{2.685432in}}%
\pgfusepath{clip}%
\pgfsetbuttcap%
\pgfsetroundjoin%
\definecolor{currentfill}{rgb}{0.000000,0.000000,0.000000}%
\pgfsetfillcolor{currentfill}%
\pgfsetlinewidth{1.003750pt}%
\definecolor{currentstroke}{rgb}{0.000000,0.000000,0.000000}%
\pgfsetstrokecolor{currentstroke}%
\pgfsetdash{}{0pt}%
\pgfpathmoveto{\pgfqpoint{6.410865in}{2.743539in}}%
\pgfpathcurveto{\pgfqpoint{6.421915in}{2.743539in}}{\pgfqpoint{6.432514in}{2.747929in}}{\pgfqpoint{6.440328in}{2.755743in}}%
\pgfpathcurveto{\pgfqpoint{6.448141in}{2.763556in}}{\pgfqpoint{6.452531in}{2.774155in}}{\pgfqpoint{6.452531in}{2.785205in}}%
\pgfpathcurveto{\pgfqpoint{6.452531in}{2.796256in}}{\pgfqpoint{6.448141in}{2.806855in}}{\pgfqpoint{6.440328in}{2.814668in}}%
\pgfpathcurveto{\pgfqpoint{6.432514in}{2.822482in}}{\pgfqpoint{6.421915in}{2.826872in}}{\pgfqpoint{6.410865in}{2.826872in}}%
\pgfpathcurveto{\pgfqpoint{6.399815in}{2.826872in}}{\pgfqpoint{6.389216in}{2.822482in}}{\pgfqpoint{6.381402in}{2.814668in}}%
\pgfpathcurveto{\pgfqpoint{6.373588in}{2.806855in}}{\pgfqpoint{6.369198in}{2.796256in}}{\pgfqpoint{6.369198in}{2.785205in}}%
\pgfpathcurveto{\pgfqpoint{6.369198in}{2.774155in}}{\pgfqpoint{6.373588in}{2.763556in}}{\pgfqpoint{6.381402in}{2.755743in}}%
\pgfpathcurveto{\pgfqpoint{6.389216in}{2.747929in}}{\pgfqpoint{6.399815in}{2.743539in}}{\pgfqpoint{6.410865in}{2.743539in}}%
\pgfpathclose%
\pgfusepath{stroke,fill}%
\end{pgfscope}%
\begin{pgfscope}%
\pgfpathrectangle{\pgfqpoint{0.970666in}{0.566125in}}{\pgfqpoint{5.699255in}{2.685432in}}%
\pgfusepath{clip}%
\pgfsetbuttcap%
\pgfsetroundjoin%
\definecolor{currentfill}{rgb}{0.000000,0.000000,0.000000}%
\pgfsetfillcolor{currentfill}%
\pgfsetlinewidth{1.003750pt}%
\definecolor{currentstroke}{rgb}{0.000000,0.000000,0.000000}%
\pgfsetstrokecolor{currentstroke}%
\pgfsetdash{}{0pt}%
\pgfpathmoveto{\pgfqpoint{6.151808in}{2.837435in}}%
\pgfpathcurveto{\pgfqpoint{6.162858in}{2.837435in}}{\pgfqpoint{6.173457in}{2.841825in}}{\pgfqpoint{6.181271in}{2.849639in}}%
\pgfpathcurveto{\pgfqpoint{6.189084in}{2.857453in}}{\pgfqpoint{6.193474in}{2.868052in}}{\pgfqpoint{6.193474in}{2.879102in}}%
\pgfpathcurveto{\pgfqpoint{6.193474in}{2.890152in}}{\pgfqpoint{6.189084in}{2.900751in}}{\pgfqpoint{6.181271in}{2.908564in}}%
\pgfpathcurveto{\pgfqpoint{6.173457in}{2.916378in}}{\pgfqpoint{6.162858in}{2.920768in}}{\pgfqpoint{6.151808in}{2.920768in}}%
\pgfpathcurveto{\pgfqpoint{6.140758in}{2.920768in}}{\pgfqpoint{6.130159in}{2.916378in}}{\pgfqpoint{6.122345in}{2.908564in}}%
\pgfpathcurveto{\pgfqpoint{6.114531in}{2.900751in}}{\pgfqpoint{6.110141in}{2.890152in}}{\pgfqpoint{6.110141in}{2.879102in}}%
\pgfpathcurveto{\pgfqpoint{6.110141in}{2.868052in}}{\pgfqpoint{6.114531in}{2.857453in}}{\pgfqpoint{6.122345in}{2.849639in}}%
\pgfpathcurveto{\pgfqpoint{6.130159in}{2.841825in}}{\pgfqpoint{6.140758in}{2.837435in}}{\pgfqpoint{6.151808in}{2.837435in}}%
\pgfpathclose%
\pgfusepath{stroke,fill}%
\end{pgfscope}%
\begin{pgfscope}%
\pgfpathrectangle{\pgfqpoint{0.970666in}{0.566125in}}{\pgfqpoint{5.699255in}{2.685432in}}%
\pgfusepath{clip}%
\pgfsetbuttcap%
\pgfsetroundjoin%
\definecolor{currentfill}{rgb}{0.000000,0.000000,0.000000}%
\pgfsetfillcolor{currentfill}%
\pgfsetlinewidth{1.003750pt}%
\definecolor{currentstroke}{rgb}{0.000000,0.000000,0.000000}%
\pgfsetstrokecolor{currentstroke}%
\pgfsetdash{}{0pt}%
\pgfpathmoveto{\pgfqpoint{6.119426in}{3.087825in}}%
\pgfpathcurveto{\pgfqpoint{6.130476in}{3.087825in}}{\pgfqpoint{6.141075in}{3.092215in}}{\pgfqpoint{6.148888in}{3.100029in}}%
\pgfpathcurveto{\pgfqpoint{6.156702in}{3.107842in}}{\pgfqpoint{6.161092in}{3.118441in}}{\pgfqpoint{6.161092in}{3.129492in}}%
\pgfpathcurveto{\pgfqpoint{6.161092in}{3.140542in}}{\pgfqpoint{6.156702in}{3.151141in}}{\pgfqpoint{6.148888in}{3.158954in}}%
\pgfpathcurveto{\pgfqpoint{6.141075in}{3.166768in}}{\pgfqpoint{6.130476in}{3.171158in}}{\pgfqpoint{6.119426in}{3.171158in}}%
\pgfpathcurveto{\pgfqpoint{6.108375in}{3.171158in}}{\pgfqpoint{6.097776in}{3.166768in}}{\pgfqpoint{6.089963in}{3.158954in}}%
\pgfpathcurveto{\pgfqpoint{6.082149in}{3.151141in}}{\pgfqpoint{6.077759in}{3.140542in}}{\pgfqpoint{6.077759in}{3.129492in}}%
\pgfpathcurveto{\pgfqpoint{6.077759in}{3.118441in}}{\pgfqpoint{6.082149in}{3.107842in}}{\pgfqpoint{6.089963in}{3.100029in}}%
\pgfpathcurveto{\pgfqpoint{6.097776in}{3.092215in}}{\pgfqpoint{6.108375in}{3.087825in}}{\pgfqpoint{6.119426in}{3.087825in}}%
\pgfpathclose%
\pgfusepath{stroke,fill}%
\end{pgfscope}%
\begin{pgfscope}%
\pgfpathrectangle{\pgfqpoint{0.970666in}{0.566125in}}{\pgfqpoint{5.699255in}{2.685432in}}%
\pgfusepath{clip}%
\pgfsetbuttcap%
\pgfsetroundjoin%
\definecolor{currentfill}{rgb}{0.000000,0.000000,0.000000}%
\pgfsetfillcolor{currentfill}%
\pgfsetlinewidth{1.003750pt}%
\definecolor{currentstroke}{rgb}{0.000000,0.000000,0.000000}%
\pgfsetstrokecolor{currentstroke}%
\pgfsetdash{}{0pt}%
\pgfpathmoveto{\pgfqpoint{5.795604in}{3.056526in}}%
\pgfpathcurveto{\pgfqpoint{5.806654in}{3.056526in}}{\pgfqpoint{5.817253in}{3.060916in}}{\pgfqpoint{5.825067in}{3.068730in}}%
\pgfpathcurveto{\pgfqpoint{5.832881in}{3.076544in}}{\pgfqpoint{5.837271in}{3.087143in}}{\pgfqpoint{5.837271in}{3.098193in}}%
\pgfpathcurveto{\pgfqpoint{5.837271in}{3.109243in}}{\pgfqpoint{5.832881in}{3.119842in}}{\pgfqpoint{5.825067in}{3.127656in}}%
\pgfpathcurveto{\pgfqpoint{5.817253in}{3.135469in}}{\pgfqpoint{5.806654in}{3.139860in}}{\pgfqpoint{5.795604in}{3.139860in}}%
\pgfpathcurveto{\pgfqpoint{5.784554in}{3.139860in}}{\pgfqpoint{5.773955in}{3.135469in}}{\pgfqpoint{5.766142in}{3.127656in}}%
\pgfpathcurveto{\pgfqpoint{5.758328in}{3.119842in}}{\pgfqpoint{5.753938in}{3.109243in}}{\pgfqpoint{5.753938in}{3.098193in}}%
\pgfpathcurveto{\pgfqpoint{5.753938in}{3.087143in}}{\pgfqpoint{5.758328in}{3.076544in}}{\pgfqpoint{5.766142in}{3.068730in}}%
\pgfpathcurveto{\pgfqpoint{5.773955in}{3.060916in}}{\pgfqpoint{5.784554in}{3.056526in}}{\pgfqpoint{5.795604in}{3.056526in}}%
\pgfpathclose%
\pgfusepath{stroke,fill}%
\end{pgfscope}%
\begin{pgfscope}%
\pgfpathrectangle{\pgfqpoint{0.970666in}{0.566125in}}{\pgfqpoint{5.699255in}{2.685432in}}%
\pgfusepath{clip}%
\pgfsetbuttcap%
\pgfsetroundjoin%
\definecolor{currentfill}{rgb}{0.000000,0.000000,0.000000}%
\pgfsetfillcolor{currentfill}%
\pgfsetlinewidth{1.003750pt}%
\definecolor{currentstroke}{rgb}{0.000000,0.000000,0.000000}%
\pgfsetstrokecolor{currentstroke}%
\pgfsetdash{}{0pt}%
\pgfpathmoveto{\pgfqpoint{5.763222in}{3.025227in}}%
\pgfpathcurveto{\pgfqpoint{5.774272in}{3.025227in}}{\pgfqpoint{5.784871in}{3.029618in}}{\pgfqpoint{5.792685in}{3.037431in}}%
\pgfpathcurveto{\pgfqpoint{5.800499in}{3.045245in}}{\pgfqpoint{5.804889in}{3.055844in}}{\pgfqpoint{5.804889in}{3.066894in}}%
\pgfpathcurveto{\pgfqpoint{5.804889in}{3.077944in}}{\pgfqpoint{5.800499in}{3.088543in}}{\pgfqpoint{5.792685in}{3.096357in}}%
\pgfpathcurveto{\pgfqpoint{5.784871in}{3.104171in}}{\pgfqpoint{5.774272in}{3.108561in}}{\pgfqpoint{5.763222in}{3.108561in}}%
\pgfpathcurveto{\pgfqpoint{5.752172in}{3.108561in}}{\pgfqpoint{5.741573in}{3.104171in}}{\pgfqpoint{5.733759in}{3.096357in}}%
\pgfpathcurveto{\pgfqpoint{5.725946in}{3.088543in}}{\pgfqpoint{5.721555in}{3.077944in}}{\pgfqpoint{5.721555in}{3.066894in}}%
\pgfpathcurveto{\pgfqpoint{5.721555in}{3.055844in}}{\pgfqpoint{5.725946in}{3.045245in}}{\pgfqpoint{5.733759in}{3.037431in}}%
\pgfpathcurveto{\pgfqpoint{5.741573in}{3.029618in}}{\pgfqpoint{5.752172in}{3.025227in}}{\pgfqpoint{5.763222in}{3.025227in}}%
\pgfpathclose%
\pgfusepath{stroke,fill}%
\end{pgfscope}%
\begin{pgfscope}%
\pgfpathrectangle{\pgfqpoint{0.970666in}{0.566125in}}{\pgfqpoint{5.699255in}{2.685432in}}%
\pgfusepath{clip}%
\pgfsetbuttcap%
\pgfsetroundjoin%
\definecolor{currentfill}{rgb}{0.000000,0.000000,0.000000}%
\pgfsetfillcolor{currentfill}%
\pgfsetlinewidth{1.003750pt}%
\definecolor{currentstroke}{rgb}{0.000000,0.000000,0.000000}%
\pgfsetstrokecolor{currentstroke}%
\pgfsetdash{}{0pt}%
\pgfpathmoveto{\pgfqpoint{5.536547in}{3.040877in}}%
\pgfpathcurveto{\pgfqpoint{5.547597in}{3.040877in}}{\pgfqpoint{5.558196in}{3.045267in}}{\pgfqpoint{5.566010in}{3.053081in}}%
\pgfpathcurveto{\pgfqpoint{5.573824in}{3.060894in}}{\pgfqpoint{5.578214in}{3.071493in}}{\pgfqpoint{5.578214in}{3.082544in}}%
\pgfpathcurveto{\pgfqpoint{5.578214in}{3.093594in}}{\pgfqpoint{5.573824in}{3.104193in}}{\pgfqpoint{5.566010in}{3.112006in}}%
\pgfpathcurveto{\pgfqpoint{5.558196in}{3.119820in}}{\pgfqpoint{5.547597in}{3.124210in}}{\pgfqpoint{5.536547in}{3.124210in}}%
\pgfpathcurveto{\pgfqpoint{5.525497in}{3.124210in}}{\pgfqpoint{5.514898in}{3.119820in}}{\pgfqpoint{5.507084in}{3.112006in}}%
\pgfpathcurveto{\pgfqpoint{5.499271in}{3.104193in}}{\pgfqpoint{5.494881in}{3.093594in}}{\pgfqpoint{5.494881in}{3.082544in}}%
\pgfpathcurveto{\pgfqpoint{5.494881in}{3.071493in}}{\pgfqpoint{5.499271in}{3.060894in}}{\pgfqpoint{5.507084in}{3.053081in}}%
\pgfpathcurveto{\pgfqpoint{5.514898in}{3.045267in}}{\pgfqpoint{5.525497in}{3.040877in}}{\pgfqpoint{5.536547in}{3.040877in}}%
\pgfpathclose%
\pgfusepath{stroke,fill}%
\end{pgfscope}%
\begin{pgfscope}%
\pgfpathrectangle{\pgfqpoint{0.970666in}{0.566125in}}{\pgfqpoint{5.699255in}{2.685432in}}%
\pgfusepath{clip}%
\pgfsetbuttcap%
\pgfsetroundjoin%
\definecolor{currentfill}{rgb}{0.000000,0.000000,0.000000}%
\pgfsetfillcolor{currentfill}%
\pgfsetlinewidth{1.003750pt}%
\definecolor{currentstroke}{rgb}{0.000000,0.000000,0.000000}%
\pgfsetstrokecolor{currentstroke}%
\pgfsetdash{}{0pt}%
\pgfpathmoveto{\pgfqpoint{6.022279in}{1.851525in}}%
\pgfpathcurveto{\pgfqpoint{6.033329in}{1.851525in}}{\pgfqpoint{6.043928in}{1.855915in}}{\pgfqpoint{6.051742in}{1.863729in}}%
\pgfpathcurveto{\pgfqpoint{6.059556in}{1.871542in}}{\pgfqpoint{6.063946in}{1.882141in}}{\pgfqpoint{6.063946in}{1.893191in}}%
\pgfpathcurveto{\pgfqpoint{6.063946in}{1.904241in}}{\pgfqpoint{6.059556in}{1.914840in}}{\pgfqpoint{6.051742in}{1.922654in}}%
\pgfpathcurveto{\pgfqpoint{6.043928in}{1.930468in}}{\pgfqpoint{6.033329in}{1.934858in}}{\pgfqpoint{6.022279in}{1.934858in}}%
\pgfpathcurveto{\pgfqpoint{6.011229in}{1.934858in}}{\pgfqpoint{6.000630in}{1.930468in}}{\pgfqpoint{5.992816in}{1.922654in}}%
\pgfpathcurveto{\pgfqpoint{5.985003in}{1.914840in}}{\pgfqpoint{5.980613in}{1.904241in}}{\pgfqpoint{5.980613in}{1.893191in}}%
\pgfpathcurveto{\pgfqpoint{5.980613in}{1.882141in}}{\pgfqpoint{5.985003in}{1.871542in}}{\pgfqpoint{5.992816in}{1.863729in}}%
\pgfpathcurveto{\pgfqpoint{6.000630in}{1.855915in}}{\pgfqpoint{6.011229in}{1.851525in}}{\pgfqpoint{6.022279in}{1.851525in}}%
\pgfpathclose%
\pgfusepath{stroke,fill}%
\end{pgfscope}%
\begin{pgfscope}%
\pgfpathrectangle{\pgfqpoint{0.970666in}{0.566125in}}{\pgfqpoint{5.699255in}{2.685432in}}%
\pgfusepath{clip}%
\pgfsetbuttcap%
\pgfsetroundjoin%
\definecolor{currentfill}{rgb}{0.000000,0.000000,0.000000}%
\pgfsetfillcolor{currentfill}%
\pgfsetlinewidth{1.003750pt}%
\definecolor{currentstroke}{rgb}{0.000000,0.000000,0.000000}%
\pgfsetstrokecolor{currentstroke}%
\pgfsetdash{}{0pt}%
\pgfpathmoveto{\pgfqpoint{5.147962in}{1.820226in}}%
\pgfpathcurveto{\pgfqpoint{5.159012in}{1.820226in}}{\pgfqpoint{5.169611in}{1.824616in}}{\pgfqpoint{5.177424in}{1.832430in}}%
\pgfpathcurveto{\pgfqpoint{5.185238in}{1.840243in}}{\pgfqpoint{5.189628in}{1.850842in}}{\pgfqpoint{5.189628in}{1.861893in}}%
\pgfpathcurveto{\pgfqpoint{5.189628in}{1.872943in}}{\pgfqpoint{5.185238in}{1.883542in}}{\pgfqpoint{5.177424in}{1.891355in}}%
\pgfpathcurveto{\pgfqpoint{5.169611in}{1.899169in}}{\pgfqpoint{5.159012in}{1.903559in}}{\pgfqpoint{5.147962in}{1.903559in}}%
\pgfpathcurveto{\pgfqpoint{5.136912in}{1.903559in}}{\pgfqpoint{5.126312in}{1.899169in}}{\pgfqpoint{5.118499in}{1.891355in}}%
\pgfpathcurveto{\pgfqpoint{5.110685in}{1.883542in}}{\pgfqpoint{5.106295in}{1.872943in}}{\pgfqpoint{5.106295in}{1.861893in}}%
\pgfpathcurveto{\pgfqpoint{5.106295in}{1.850842in}}{\pgfqpoint{5.110685in}{1.840243in}}{\pgfqpoint{5.118499in}{1.832430in}}%
\pgfpathcurveto{\pgfqpoint{5.126312in}{1.824616in}}{\pgfqpoint{5.136912in}{1.820226in}}{\pgfqpoint{5.147962in}{1.820226in}}%
\pgfpathclose%
\pgfusepath{stroke,fill}%
\end{pgfscope}%
\begin{pgfscope}%
\pgfpathrectangle{\pgfqpoint{0.970666in}{0.566125in}}{\pgfqpoint{5.699255in}{2.685432in}}%
\pgfusepath{clip}%
\pgfsetbuttcap%
\pgfsetroundjoin%
\definecolor{currentfill}{rgb}{0.000000,0.000000,0.000000}%
\pgfsetfillcolor{currentfill}%
\pgfsetlinewidth{1.003750pt}%
\definecolor{currentstroke}{rgb}{0.000000,0.000000,0.000000}%
\pgfsetstrokecolor{currentstroke}%
\pgfsetdash{}{0pt}%
\pgfpathmoveto{\pgfqpoint{4.759376in}{1.788927in}}%
\pgfpathcurveto{\pgfqpoint{4.770426in}{1.788927in}}{\pgfqpoint{4.781025in}{1.793317in}}{\pgfqpoint{4.788839in}{1.801131in}}%
\pgfpathcurveto{\pgfqpoint{4.796652in}{1.808945in}}{\pgfqpoint{4.801043in}{1.819544in}}{\pgfqpoint{4.801043in}{1.830594in}}%
\pgfpathcurveto{\pgfqpoint{4.801043in}{1.841644in}}{\pgfqpoint{4.796652in}{1.852243in}}{\pgfqpoint{4.788839in}{1.860057in}}%
\pgfpathcurveto{\pgfqpoint{4.781025in}{1.867870in}}{\pgfqpoint{4.770426in}{1.872260in}}{\pgfqpoint{4.759376in}{1.872260in}}%
\pgfpathcurveto{\pgfqpoint{4.748326in}{1.872260in}}{\pgfqpoint{4.737727in}{1.867870in}}{\pgfqpoint{4.729913in}{1.860057in}}%
\pgfpathcurveto{\pgfqpoint{4.722100in}{1.852243in}}{\pgfqpoint{4.717709in}{1.841644in}}{\pgfqpoint{4.717709in}{1.830594in}}%
\pgfpathcurveto{\pgfqpoint{4.717709in}{1.819544in}}{\pgfqpoint{4.722100in}{1.808945in}}{\pgfqpoint{4.729913in}{1.801131in}}%
\pgfpathcurveto{\pgfqpoint{4.737727in}{1.793317in}}{\pgfqpoint{4.748326in}{1.788927in}}{\pgfqpoint{4.759376in}{1.788927in}}%
\pgfpathclose%
\pgfusepath{stroke,fill}%
\end{pgfscope}%
\begin{pgfscope}%
\pgfpathrectangle{\pgfqpoint{0.970666in}{0.566125in}}{\pgfqpoint{5.699255in}{2.685432in}}%
\pgfusepath{clip}%
\pgfsetbuttcap%
\pgfsetroundjoin%
\definecolor{currentfill}{rgb}{0.000000,0.000000,0.000000}%
\pgfsetfillcolor{currentfill}%
\pgfsetlinewidth{1.003750pt}%
\definecolor{currentstroke}{rgb}{0.000000,0.000000,0.000000}%
\pgfsetstrokecolor{currentstroke}%
\pgfsetdash{}{0pt}%
\pgfpathmoveto{\pgfqpoint{4.208880in}{1.757628in}}%
\pgfpathcurveto{\pgfqpoint{4.219930in}{1.757628in}}{\pgfqpoint{4.230529in}{1.762019in}}{\pgfqpoint{4.238343in}{1.769832in}}%
\pgfpathcurveto{\pgfqpoint{4.246156in}{1.777646in}}{\pgfqpoint{4.250546in}{1.788245in}}{\pgfqpoint{4.250546in}{1.799295in}}%
\pgfpathcurveto{\pgfqpoint{4.250546in}{1.810345in}}{\pgfqpoint{4.246156in}{1.820944in}}{\pgfqpoint{4.238343in}{1.828758in}}%
\pgfpathcurveto{\pgfqpoint{4.230529in}{1.836571in}}{\pgfqpoint{4.219930in}{1.840962in}}{\pgfqpoint{4.208880in}{1.840962in}}%
\pgfpathcurveto{\pgfqpoint{4.197830in}{1.840962in}}{\pgfqpoint{4.187231in}{1.836571in}}{\pgfqpoint{4.179417in}{1.828758in}}%
\pgfpathcurveto{\pgfqpoint{4.171603in}{1.820944in}}{\pgfqpoint{4.167213in}{1.810345in}}{\pgfqpoint{4.167213in}{1.799295in}}%
\pgfpathcurveto{\pgfqpoint{4.167213in}{1.788245in}}{\pgfqpoint{4.171603in}{1.777646in}}{\pgfqpoint{4.179417in}{1.769832in}}%
\pgfpathcurveto{\pgfqpoint{4.187231in}{1.762019in}}{\pgfqpoint{4.197830in}{1.757628in}}{\pgfqpoint{4.208880in}{1.757628in}}%
\pgfpathclose%
\pgfusepath{stroke,fill}%
\end{pgfscope}%
\begin{pgfscope}%
\pgfpathrectangle{\pgfqpoint{0.970666in}{0.566125in}}{\pgfqpoint{5.699255in}{2.685432in}}%
\pgfusepath{clip}%
\pgfsetbuttcap%
\pgfsetroundjoin%
\definecolor{currentfill}{rgb}{0.000000,0.000000,0.000000}%
\pgfsetfillcolor{currentfill}%
\pgfsetlinewidth{1.003750pt}%
\definecolor{currentstroke}{rgb}{0.000000,0.000000,0.000000}%
\pgfsetstrokecolor{currentstroke}%
\pgfsetdash{}{0pt}%
\pgfpathmoveto{\pgfqpoint{3.528855in}{1.695031in}}%
\pgfpathcurveto{\pgfqpoint{3.539905in}{1.695031in}}{\pgfqpoint{3.550504in}{1.699421in}}{\pgfqpoint{3.558318in}{1.707235in}}%
\pgfpathcurveto{\pgfqpoint{3.566131in}{1.715048in}}{\pgfqpoint{3.570522in}{1.725647in}}{\pgfqpoint{3.570522in}{1.736698in}}%
\pgfpathcurveto{\pgfqpoint{3.570522in}{1.747748in}}{\pgfqpoint{3.566131in}{1.758347in}}{\pgfqpoint{3.558318in}{1.766160in}}%
\pgfpathcurveto{\pgfqpoint{3.550504in}{1.773974in}}{\pgfqpoint{3.539905in}{1.778364in}}{\pgfqpoint{3.528855in}{1.778364in}}%
\pgfpathcurveto{\pgfqpoint{3.517805in}{1.778364in}}{\pgfqpoint{3.507206in}{1.773974in}}{\pgfqpoint{3.499392in}{1.766160in}}%
\pgfpathcurveto{\pgfqpoint{3.491579in}{1.758347in}}{\pgfqpoint{3.487188in}{1.747748in}}{\pgfqpoint{3.487188in}{1.736698in}}%
\pgfpathcurveto{\pgfqpoint{3.487188in}{1.725647in}}{\pgfqpoint{3.491579in}{1.715048in}}{\pgfqpoint{3.499392in}{1.707235in}}%
\pgfpathcurveto{\pgfqpoint{3.507206in}{1.699421in}}{\pgfqpoint{3.517805in}{1.695031in}}{\pgfqpoint{3.528855in}{1.695031in}}%
\pgfpathclose%
\pgfusepath{stroke,fill}%
\end{pgfscope}%
\begin{pgfscope}%
\pgfpathrectangle{\pgfqpoint{0.970666in}{0.566125in}}{\pgfqpoint{5.699255in}{2.685432in}}%
\pgfusepath{clip}%
\pgfsetbuttcap%
\pgfsetroundjoin%
\definecolor{currentfill}{rgb}{0.000000,0.000000,0.000000}%
\pgfsetfillcolor{currentfill}%
\pgfsetlinewidth{1.003750pt}%
\definecolor{currentstroke}{rgb}{0.000000,0.000000,0.000000}%
\pgfsetstrokecolor{currentstroke}%
\pgfsetdash{}{0pt}%
\pgfpathmoveto{\pgfqpoint{3.496473in}{1.601135in}}%
\pgfpathcurveto{\pgfqpoint{3.507523in}{1.601135in}}{\pgfqpoint{3.518122in}{1.605525in}}{\pgfqpoint{3.525936in}{1.613339in}}%
\pgfpathcurveto{\pgfqpoint{3.533749in}{1.621152in}}{\pgfqpoint{3.538140in}{1.631751in}}{\pgfqpoint{3.538140in}{1.642801in}}%
\pgfpathcurveto{\pgfqpoint{3.538140in}{1.653851in}}{\pgfqpoint{3.533749in}{1.664451in}}{\pgfqpoint{3.525936in}{1.672264in}}%
\pgfpathcurveto{\pgfqpoint{3.518122in}{1.680078in}}{\pgfqpoint{3.507523in}{1.684468in}}{\pgfqpoint{3.496473in}{1.684468in}}%
\pgfpathcurveto{\pgfqpoint{3.485423in}{1.684468in}}{\pgfqpoint{3.474824in}{1.680078in}}{\pgfqpoint{3.467010in}{1.672264in}}%
\pgfpathcurveto{\pgfqpoint{3.459196in}{1.664451in}}{\pgfqpoint{3.454806in}{1.653851in}}{\pgfqpoint{3.454806in}{1.642801in}}%
\pgfpathcurveto{\pgfqpoint{3.454806in}{1.631751in}}{\pgfqpoint{3.459196in}{1.621152in}}{\pgfqpoint{3.467010in}{1.613339in}}%
\pgfpathcurveto{\pgfqpoint{3.474824in}{1.605525in}}{\pgfqpoint{3.485423in}{1.601135in}}{\pgfqpoint{3.496473in}{1.601135in}}%
\pgfpathclose%
\pgfusepath{stroke,fill}%
\end{pgfscope}%
\begin{pgfscope}%
\pgfpathrectangle{\pgfqpoint{0.970666in}{0.566125in}}{\pgfqpoint{5.699255in}{2.685432in}}%
\pgfusepath{clip}%
\pgfsetbuttcap%
\pgfsetroundjoin%
\definecolor{currentfill}{rgb}{0.000000,0.000000,0.000000}%
\pgfsetfillcolor{currentfill}%
\pgfsetlinewidth{1.003750pt}%
\definecolor{currentstroke}{rgb}{0.000000,0.000000,0.000000}%
\pgfsetstrokecolor{currentstroke}%
\pgfsetdash{}{0pt}%
\pgfpathmoveto{\pgfqpoint{4.046969in}{1.319446in}}%
\pgfpathcurveto{\pgfqpoint{4.058019in}{1.319446in}}{\pgfqpoint{4.068618in}{1.323836in}}{\pgfqpoint{4.076432in}{1.331650in}}%
\pgfpathcurveto{\pgfqpoint{4.084246in}{1.339464in}}{\pgfqpoint{4.088636in}{1.350063in}}{\pgfqpoint{4.088636in}{1.361113in}}%
\pgfpathcurveto{\pgfqpoint{4.088636in}{1.372163in}}{\pgfqpoint{4.084246in}{1.382762in}}{\pgfqpoint{4.076432in}{1.390575in}}%
\pgfpathcurveto{\pgfqpoint{4.068618in}{1.398389in}}{\pgfqpoint{4.058019in}{1.402779in}}{\pgfqpoint{4.046969in}{1.402779in}}%
\pgfpathcurveto{\pgfqpoint{4.035919in}{1.402779in}}{\pgfqpoint{4.025320in}{1.398389in}}{\pgfqpoint{4.017506in}{1.390575in}}%
\pgfpathcurveto{\pgfqpoint{4.009693in}{1.382762in}}{\pgfqpoint{4.005302in}{1.372163in}}{\pgfqpoint{4.005302in}{1.361113in}}%
\pgfpathcurveto{\pgfqpoint{4.005302in}{1.350063in}}{\pgfqpoint{4.009693in}{1.339464in}}{\pgfqpoint{4.017506in}{1.331650in}}%
\pgfpathcurveto{\pgfqpoint{4.025320in}{1.323836in}}{\pgfqpoint{4.035919in}{1.319446in}}{\pgfqpoint{4.046969in}{1.319446in}}%
\pgfpathclose%
\pgfusepath{stroke,fill}%
\end{pgfscope}%
\begin{pgfscope}%
\pgfpathrectangle{\pgfqpoint{0.970666in}{0.566125in}}{\pgfqpoint{5.699255in}{2.685432in}}%
\pgfusepath{clip}%
\pgfsetbuttcap%
\pgfsetroundjoin%
\definecolor{currentfill}{rgb}{0.000000,0.000000,0.000000}%
\pgfsetfillcolor{currentfill}%
\pgfsetlinewidth{1.003750pt}%
\definecolor{currentstroke}{rgb}{0.000000,0.000000,0.000000}%
\pgfsetstrokecolor{currentstroke}%
\pgfsetdash{}{0pt}%
\pgfpathmoveto{\pgfqpoint{3.593619in}{1.178602in}}%
\pgfpathcurveto{\pgfqpoint{3.604669in}{1.178602in}}{\pgfqpoint{3.615268in}{1.182992in}}{\pgfqpoint{3.623082in}{1.190806in}}%
\pgfpathcurveto{\pgfqpoint{3.630896in}{1.198619in}}{\pgfqpoint{3.635286in}{1.209218in}}{\pgfqpoint{3.635286in}{1.220268in}}%
\pgfpathcurveto{\pgfqpoint{3.635286in}{1.231318in}}{\pgfqpoint{3.630896in}{1.241917in}}{\pgfqpoint{3.623082in}{1.249731in}}%
\pgfpathcurveto{\pgfqpoint{3.615268in}{1.257545in}}{\pgfqpoint{3.604669in}{1.261935in}}{\pgfqpoint{3.593619in}{1.261935in}}%
\pgfpathcurveto{\pgfqpoint{3.582569in}{1.261935in}}{\pgfqpoint{3.571970in}{1.257545in}}{\pgfqpoint{3.564156in}{1.249731in}}%
\pgfpathcurveto{\pgfqpoint{3.556343in}{1.241917in}}{\pgfqpoint{3.551953in}{1.231318in}}{\pgfqpoint{3.551953in}{1.220268in}}%
\pgfpathcurveto{\pgfqpoint{3.551953in}{1.209218in}}{\pgfqpoint{3.556343in}{1.198619in}}{\pgfqpoint{3.564156in}{1.190806in}}%
\pgfpathcurveto{\pgfqpoint{3.571970in}{1.182992in}}{\pgfqpoint{3.582569in}{1.178602in}}{\pgfqpoint{3.593619in}{1.178602in}}%
\pgfpathclose%
\pgfusepath{stroke,fill}%
\end{pgfscope}%
\begin{pgfscope}%
\pgfpathrectangle{\pgfqpoint{0.970666in}{0.566125in}}{\pgfqpoint{5.699255in}{2.685432in}}%
\pgfusepath{clip}%
\pgfsetbuttcap%
\pgfsetroundjoin%
\definecolor{currentfill}{rgb}{0.000000,0.000000,0.000000}%
\pgfsetfillcolor{currentfill}%
\pgfsetlinewidth{1.003750pt}%
\definecolor{currentstroke}{rgb}{0.000000,0.000000,0.000000}%
\pgfsetstrokecolor{currentstroke}%
\pgfsetdash{}{0pt}%
\pgfpathmoveto{\pgfqpoint{3.237416in}{1.131654in}}%
\pgfpathcurveto{\pgfqpoint{3.248466in}{1.131654in}}{\pgfqpoint{3.259065in}{1.136044in}}{\pgfqpoint{3.266879in}{1.143857in}}%
\pgfpathcurveto{\pgfqpoint{3.274692in}{1.151671in}}{\pgfqpoint{3.279082in}{1.162270in}}{\pgfqpoint{3.279082in}{1.173320in}}%
\pgfpathcurveto{\pgfqpoint{3.279082in}{1.184370in}}{\pgfqpoint{3.274692in}{1.194969in}}{\pgfqpoint{3.266879in}{1.202783in}}%
\pgfpathcurveto{\pgfqpoint{3.259065in}{1.210597in}}{\pgfqpoint{3.248466in}{1.214987in}}{\pgfqpoint{3.237416in}{1.214987in}}%
\pgfpathcurveto{\pgfqpoint{3.226366in}{1.214987in}}{\pgfqpoint{3.215767in}{1.210597in}}{\pgfqpoint{3.207953in}{1.202783in}}%
\pgfpathcurveto{\pgfqpoint{3.200139in}{1.194969in}}{\pgfqpoint{3.195749in}{1.184370in}}{\pgfqpoint{3.195749in}{1.173320in}}%
\pgfpathcurveto{\pgfqpoint{3.195749in}{1.162270in}}{\pgfqpoint{3.200139in}{1.151671in}}{\pgfqpoint{3.207953in}{1.143857in}}%
\pgfpathcurveto{\pgfqpoint{3.215767in}{1.136044in}}{\pgfqpoint{3.226366in}{1.131654in}}{\pgfqpoint{3.237416in}{1.131654in}}%
\pgfpathclose%
\pgfusepath{stroke,fill}%
\end{pgfscope}%
\begin{pgfscope}%
\pgfpathrectangle{\pgfqpoint{0.970666in}{0.566125in}}{\pgfqpoint{5.699255in}{2.685432in}}%
\pgfusepath{clip}%
\pgfsetbuttcap%
\pgfsetroundjoin%
\definecolor{currentfill}{rgb}{0.000000,0.000000,0.000000}%
\pgfsetfillcolor{currentfill}%
\pgfsetlinewidth{1.003750pt}%
\definecolor{currentstroke}{rgb}{0.000000,0.000000,0.000000}%
\pgfsetstrokecolor{currentstroke}%
\pgfsetdash{}{0pt}%
\pgfpathmoveto{\pgfqpoint{3.140269in}{0.912562in}}%
\pgfpathcurveto{\pgfqpoint{3.151320in}{0.912562in}}{\pgfqpoint{3.161919in}{0.916953in}}{\pgfqpoint{3.169732in}{0.924766in}}%
\pgfpathcurveto{\pgfqpoint{3.177546in}{0.932580in}}{\pgfqpoint{3.181936in}{0.943179in}}{\pgfqpoint{3.181936in}{0.954229in}}%
\pgfpathcurveto{\pgfqpoint{3.181936in}{0.965279in}}{\pgfqpoint{3.177546in}{0.975878in}}{\pgfqpoint{3.169732in}{0.983692in}}%
\pgfpathcurveto{\pgfqpoint{3.161919in}{0.991505in}}{\pgfqpoint{3.151320in}{0.995896in}}{\pgfqpoint{3.140269in}{0.995896in}}%
\pgfpathcurveto{\pgfqpoint{3.129219in}{0.995896in}}{\pgfqpoint{3.118620in}{0.991505in}}{\pgfqpoint{3.110807in}{0.983692in}}%
\pgfpathcurveto{\pgfqpoint{3.102993in}{0.975878in}}{\pgfqpoint{3.098603in}{0.965279in}}{\pgfqpoint{3.098603in}{0.954229in}}%
\pgfpathcurveto{\pgfqpoint{3.098603in}{0.943179in}}{\pgfqpoint{3.102993in}{0.932580in}}{\pgfqpoint{3.110807in}{0.924766in}}%
\pgfpathcurveto{\pgfqpoint{3.118620in}{0.916953in}}{\pgfqpoint{3.129219in}{0.912562in}}{\pgfqpoint{3.140269in}{0.912562in}}%
\pgfpathclose%
\pgfusepath{stroke,fill}%
\end{pgfscope}%
\begin{pgfscope}%
\pgfpathrectangle{\pgfqpoint{0.970666in}{0.566125in}}{\pgfqpoint{5.699255in}{2.685432in}}%
\pgfusepath{clip}%
\pgfsetbuttcap%
\pgfsetroundjoin%
\definecolor{currentfill}{rgb}{0.000000,0.000000,0.000000}%
\pgfsetfillcolor{currentfill}%
\pgfsetlinewidth{1.003750pt}%
\definecolor{currentstroke}{rgb}{0.000000,0.000000,0.000000}%
\pgfsetstrokecolor{currentstroke}%
\pgfsetdash{}{0pt}%
\pgfpathmoveto{\pgfqpoint{2.945977in}{0.709121in}}%
\pgfpathcurveto{\pgfqpoint{2.957027in}{0.709121in}}{\pgfqpoint{2.967626in}{0.713511in}}{\pgfqpoint{2.975439in}{0.721324in}}%
\pgfpathcurveto{\pgfqpoint{2.983253in}{0.729138in}}{\pgfqpoint{2.987643in}{0.739737in}}{\pgfqpoint{2.987643in}{0.750787in}}%
\pgfpathcurveto{\pgfqpoint{2.987643in}{0.761837in}}{\pgfqpoint{2.983253in}{0.772436in}}{\pgfqpoint{2.975439in}{0.780250in}}%
\pgfpathcurveto{\pgfqpoint{2.967626in}{0.788064in}}{\pgfqpoint{2.957027in}{0.792454in}}{\pgfqpoint{2.945977in}{0.792454in}}%
\pgfpathcurveto{\pgfqpoint{2.934926in}{0.792454in}}{\pgfqpoint{2.924327in}{0.788064in}}{\pgfqpoint{2.916514in}{0.780250in}}%
\pgfpathcurveto{\pgfqpoint{2.908700in}{0.772436in}}{\pgfqpoint{2.904310in}{0.761837in}}{\pgfqpoint{2.904310in}{0.750787in}}%
\pgfpathcurveto{\pgfqpoint{2.904310in}{0.739737in}}{\pgfqpoint{2.908700in}{0.729138in}}{\pgfqpoint{2.916514in}{0.721324in}}%
\pgfpathcurveto{\pgfqpoint{2.924327in}{0.713511in}}{\pgfqpoint{2.934926in}{0.709121in}}{\pgfqpoint{2.945977in}{0.709121in}}%
\pgfpathclose%
\pgfusepath{stroke,fill}%
\end{pgfscope}%
\begin{pgfscope}%
\pgfpathrectangle{\pgfqpoint{0.970666in}{0.566125in}}{\pgfqpoint{5.699255in}{2.685432in}}%
\pgfusepath{clip}%
\pgfsetbuttcap%
\pgfsetroundjoin%
\definecolor{currentfill}{rgb}{0.000000,0.000000,0.000000}%
\pgfsetfillcolor{currentfill}%
\pgfsetlinewidth{1.003750pt}%
\definecolor{currentstroke}{rgb}{0.000000,0.000000,0.000000}%
\pgfsetstrokecolor{currentstroke}%
\pgfsetdash{}{0pt}%
\pgfpathmoveto{\pgfqpoint{2.686920in}{0.834315in}}%
\pgfpathcurveto{\pgfqpoint{2.697970in}{0.834315in}}{\pgfqpoint{2.708569in}{0.838706in}}{\pgfqpoint{2.716382in}{0.846519in}}%
\pgfpathcurveto{\pgfqpoint{2.724196in}{0.854333in}}{\pgfqpoint{2.728586in}{0.864932in}}{\pgfqpoint{2.728586in}{0.875982in}}%
\pgfpathcurveto{\pgfqpoint{2.728586in}{0.887032in}}{\pgfqpoint{2.724196in}{0.897631in}}{\pgfqpoint{2.716382in}{0.905445in}}%
\pgfpathcurveto{\pgfqpoint{2.708569in}{0.913259in}}{\pgfqpoint{2.697970in}{0.917649in}}{\pgfqpoint{2.686920in}{0.917649in}}%
\pgfpathcurveto{\pgfqpoint{2.675869in}{0.917649in}}{\pgfqpoint{2.665270in}{0.913259in}}{\pgfqpoint{2.657457in}{0.905445in}}%
\pgfpathcurveto{\pgfqpoint{2.649643in}{0.897631in}}{\pgfqpoint{2.645253in}{0.887032in}}{\pgfqpoint{2.645253in}{0.875982in}}%
\pgfpathcurveto{\pgfqpoint{2.645253in}{0.864932in}}{\pgfqpoint{2.649643in}{0.854333in}}{\pgfqpoint{2.657457in}{0.846519in}}%
\pgfpathcurveto{\pgfqpoint{2.665270in}{0.838706in}}{\pgfqpoint{2.675869in}{0.834315in}}{\pgfqpoint{2.686920in}{0.834315in}}%
\pgfpathclose%
\pgfusepath{stroke,fill}%
\end{pgfscope}%
\begin{pgfscope}%
\pgfpathrectangle{\pgfqpoint{0.970666in}{0.566125in}}{\pgfqpoint{5.699255in}{2.685432in}}%
\pgfusepath{clip}%
\pgfsetbuttcap%
\pgfsetroundjoin%
\definecolor{currentfill}{rgb}{0.000000,0.000000,0.000000}%
\pgfsetfillcolor{currentfill}%
\pgfsetlinewidth{1.003750pt}%
\definecolor{currentstroke}{rgb}{0.000000,0.000000,0.000000}%
\pgfsetstrokecolor{currentstroke}%
\pgfsetdash{}{0pt}%
\pgfpathmoveto{\pgfqpoint{2.525009in}{0.928212in}}%
\pgfpathcurveto{\pgfqpoint{2.536059in}{0.928212in}}{\pgfqpoint{2.546658in}{0.932602in}}{\pgfqpoint{2.554472in}{0.940416in}}%
\pgfpathcurveto{\pgfqpoint{2.562285in}{0.948229in}}{\pgfqpoint{2.566676in}{0.958828in}}{\pgfqpoint{2.566676in}{0.969878in}}%
\pgfpathcurveto{\pgfqpoint{2.566676in}{0.980929in}}{\pgfqpoint{2.562285in}{0.991528in}}{\pgfqpoint{2.554472in}{0.999341in}}%
\pgfpathcurveto{\pgfqpoint{2.546658in}{1.007155in}}{\pgfqpoint{2.536059in}{1.011545in}}{\pgfqpoint{2.525009in}{1.011545in}}%
\pgfpathcurveto{\pgfqpoint{2.513959in}{1.011545in}}{\pgfqpoint{2.503360in}{1.007155in}}{\pgfqpoint{2.495546in}{0.999341in}}%
\pgfpathcurveto{\pgfqpoint{2.487732in}{0.991528in}}{\pgfqpoint{2.483342in}{0.980929in}}{\pgfqpoint{2.483342in}{0.969878in}}%
\pgfpathcurveto{\pgfqpoint{2.483342in}{0.958828in}}{\pgfqpoint{2.487732in}{0.948229in}}{\pgfqpoint{2.495546in}{0.940416in}}%
\pgfpathcurveto{\pgfqpoint{2.503360in}{0.932602in}}{\pgfqpoint{2.513959in}{0.928212in}}{\pgfqpoint{2.525009in}{0.928212in}}%
\pgfpathclose%
\pgfusepath{stroke,fill}%
\end{pgfscope}%
\begin{pgfscope}%
\pgfpathrectangle{\pgfqpoint{0.970666in}{0.566125in}}{\pgfqpoint{5.699255in}{2.685432in}}%
\pgfusepath{clip}%
\pgfsetbuttcap%
\pgfsetroundjoin%
\definecolor{currentfill}{rgb}{0.000000,0.000000,0.000000}%
\pgfsetfillcolor{currentfill}%
\pgfsetlinewidth{1.003750pt}%
\definecolor{currentstroke}{rgb}{0.000000,0.000000,0.000000}%
\pgfsetstrokecolor{currentstroke}%
\pgfsetdash{}{0pt}%
\pgfpathmoveto{\pgfqpoint{2.363098in}{0.975160in}}%
\pgfpathcurveto{\pgfqpoint{2.374148in}{0.975160in}}{\pgfqpoint{2.384747in}{0.979550in}}{\pgfqpoint{2.392561in}{0.987364in}}%
\pgfpathcurveto{\pgfqpoint{2.400375in}{0.995177in}}{\pgfqpoint{2.404765in}{1.005776in}}{\pgfqpoint{2.404765in}{1.016827in}}%
\pgfpathcurveto{\pgfqpoint{2.404765in}{1.027877in}}{\pgfqpoint{2.400375in}{1.038476in}}{\pgfqpoint{2.392561in}{1.046289in}}%
\pgfpathcurveto{\pgfqpoint{2.384747in}{1.054103in}}{\pgfqpoint{2.374148in}{1.058493in}}{\pgfqpoint{2.363098in}{1.058493in}}%
\pgfpathcurveto{\pgfqpoint{2.352048in}{1.058493in}}{\pgfqpoint{2.341449in}{1.054103in}}{\pgfqpoint{2.333635in}{1.046289in}}%
\pgfpathcurveto{\pgfqpoint{2.325822in}{1.038476in}}{\pgfqpoint{2.321432in}{1.027877in}}{\pgfqpoint{2.321432in}{1.016827in}}%
\pgfpathcurveto{\pgfqpoint{2.321432in}{1.005776in}}{\pgfqpoint{2.325822in}{0.995177in}}{\pgfqpoint{2.333635in}{0.987364in}}%
\pgfpathcurveto{\pgfqpoint{2.341449in}{0.979550in}}{\pgfqpoint{2.352048in}{0.975160in}}{\pgfqpoint{2.363098in}{0.975160in}}%
\pgfpathclose%
\pgfusepath{stroke,fill}%
\end{pgfscope}%
\begin{pgfscope}%
\pgfpathrectangle{\pgfqpoint{0.970666in}{0.566125in}}{\pgfqpoint{5.699255in}{2.685432in}}%
\pgfusepath{clip}%
\pgfsetbuttcap%
\pgfsetroundjoin%
\definecolor{currentfill}{rgb}{0.000000,0.000000,0.000000}%
\pgfsetfillcolor{currentfill}%
\pgfsetlinewidth{1.003750pt}%
\definecolor{currentstroke}{rgb}{0.000000,0.000000,0.000000}%
\pgfsetstrokecolor{currentstroke}%
\pgfsetdash{}{0pt}%
\pgfpathmoveto{\pgfqpoint{2.233570in}{0.912562in}}%
\pgfpathcurveto{\pgfqpoint{2.244620in}{0.912562in}}{\pgfqpoint{2.255219in}{0.916953in}}{\pgfqpoint{2.263032in}{0.924766in}}%
\pgfpathcurveto{\pgfqpoint{2.270846in}{0.932580in}}{\pgfqpoint{2.275236in}{0.943179in}}{\pgfqpoint{2.275236in}{0.954229in}}%
\pgfpathcurveto{\pgfqpoint{2.275236in}{0.965279in}}{\pgfqpoint{2.270846in}{0.975878in}}{\pgfqpoint{2.263032in}{0.983692in}}%
\pgfpathcurveto{\pgfqpoint{2.255219in}{0.991505in}}{\pgfqpoint{2.244620in}{0.995896in}}{\pgfqpoint{2.233570in}{0.995896in}}%
\pgfpathcurveto{\pgfqpoint{2.222520in}{0.995896in}}{\pgfqpoint{2.211921in}{0.991505in}}{\pgfqpoint{2.204107in}{0.983692in}}%
\pgfpathcurveto{\pgfqpoint{2.196293in}{0.975878in}}{\pgfqpoint{2.191903in}{0.965279in}}{\pgfqpoint{2.191903in}{0.954229in}}%
\pgfpathcurveto{\pgfqpoint{2.191903in}{0.943179in}}{\pgfqpoint{2.196293in}{0.932580in}}{\pgfqpoint{2.204107in}{0.924766in}}%
\pgfpathcurveto{\pgfqpoint{2.211921in}{0.916953in}}{\pgfqpoint{2.222520in}{0.912562in}}{\pgfqpoint{2.233570in}{0.912562in}}%
\pgfpathclose%
\pgfusepath{stroke,fill}%
\end{pgfscope}%
\begin{pgfscope}%
\pgfpathrectangle{\pgfqpoint{0.970666in}{0.566125in}}{\pgfqpoint{5.699255in}{2.685432in}}%
\pgfusepath{clip}%
\pgfsetbuttcap%
\pgfsetroundjoin%
\definecolor{currentfill}{rgb}{0.000000,0.000000,0.000000}%
\pgfsetfillcolor{currentfill}%
\pgfsetlinewidth{1.003750pt}%
\definecolor{currentstroke}{rgb}{0.000000,0.000000,0.000000}%
\pgfsetstrokecolor{currentstroke}%
\pgfsetdash{}{0pt}%
\pgfpathmoveto{\pgfqpoint{2.006895in}{0.959510in}}%
\pgfpathcurveto{\pgfqpoint{2.017945in}{0.959510in}}{\pgfqpoint{2.028544in}{0.963901in}}{\pgfqpoint{2.036358in}{0.971714in}}%
\pgfpathcurveto{\pgfqpoint{2.044171in}{0.979528in}}{\pgfqpoint{2.048561in}{0.990127in}}{\pgfqpoint{2.048561in}{1.001177in}}%
\pgfpathcurveto{\pgfqpoint{2.048561in}{1.012227in}}{\pgfqpoint{2.044171in}{1.022826in}}{\pgfqpoint{2.036358in}{1.030640in}}%
\pgfpathcurveto{\pgfqpoint{2.028544in}{1.038454in}}{\pgfqpoint{2.017945in}{1.042844in}}{\pgfqpoint{2.006895in}{1.042844in}}%
\pgfpathcurveto{\pgfqpoint{1.995845in}{1.042844in}}{\pgfqpoint{1.985246in}{1.038454in}}{\pgfqpoint{1.977432in}{1.030640in}}%
\pgfpathcurveto{\pgfqpoint{1.969618in}{1.022826in}}{\pgfqpoint{1.965228in}{1.012227in}}{\pgfqpoint{1.965228in}{1.001177in}}%
\pgfpathcurveto{\pgfqpoint{1.965228in}{0.990127in}}{\pgfqpoint{1.969618in}{0.979528in}}{\pgfqpoint{1.977432in}{0.971714in}}%
\pgfpathcurveto{\pgfqpoint{1.985246in}{0.963901in}}{\pgfqpoint{1.995845in}{0.959510in}}{\pgfqpoint{2.006895in}{0.959510in}}%
\pgfpathclose%
\pgfusepath{stroke,fill}%
\end{pgfscope}%
\begin{pgfscope}%
\pgfpathrectangle{\pgfqpoint{0.970666in}{0.566125in}}{\pgfqpoint{5.699255in}{2.685432in}}%
\pgfusepath{clip}%
\pgfsetbuttcap%
\pgfsetroundjoin%
\definecolor{currentfill}{rgb}{0.000000,0.000000,0.000000}%
\pgfsetfillcolor{currentfill}%
\pgfsetlinewidth{1.003750pt}%
\definecolor{currentstroke}{rgb}{0.000000,0.000000,0.000000}%
\pgfsetstrokecolor{currentstroke}%
\pgfsetdash{}{0pt}%
\pgfpathmoveto{\pgfqpoint{1.877366in}{1.053407in}}%
\pgfpathcurveto{\pgfqpoint{1.888416in}{1.053407in}}{\pgfqpoint{1.899015in}{1.057797in}}{\pgfqpoint{1.906829in}{1.065611in}}%
\pgfpathcurveto{\pgfqpoint{1.914643in}{1.073424in}}{\pgfqpoint{1.919033in}{1.084023in}}{\pgfqpoint{1.919033in}{1.095073in}}%
\pgfpathcurveto{\pgfqpoint{1.919033in}{1.106123in}}{\pgfqpoint{1.914643in}{1.116723in}}{\pgfqpoint{1.906829in}{1.124536in}}%
\pgfpathcurveto{\pgfqpoint{1.899015in}{1.132350in}}{\pgfqpoint{1.888416in}{1.136740in}}{\pgfqpoint{1.877366in}{1.136740in}}%
\pgfpathcurveto{\pgfqpoint{1.866316in}{1.136740in}}{\pgfqpoint{1.855717in}{1.132350in}}{\pgfqpoint{1.847903in}{1.124536in}}%
\pgfpathcurveto{\pgfqpoint{1.840090in}{1.116723in}}{\pgfqpoint{1.835700in}{1.106123in}}{\pgfqpoint{1.835700in}{1.095073in}}%
\pgfpathcurveto{\pgfqpoint{1.835700in}{1.084023in}}{\pgfqpoint{1.840090in}{1.073424in}}{\pgfqpoint{1.847903in}{1.065611in}}%
\pgfpathcurveto{\pgfqpoint{1.855717in}{1.057797in}}{\pgfqpoint{1.866316in}{1.053407in}}{\pgfqpoint{1.877366in}{1.053407in}}%
\pgfpathclose%
\pgfusepath{stroke,fill}%
\end{pgfscope}%
\begin{pgfscope}%
\pgfpathrectangle{\pgfqpoint{0.970666in}{0.566125in}}{\pgfqpoint{5.699255in}{2.685432in}}%
\pgfusepath{clip}%
\pgfsetbuttcap%
\pgfsetroundjoin%
\definecolor{currentfill}{rgb}{0.000000,0.000000,0.000000}%
\pgfsetfillcolor{currentfill}%
\pgfsetlinewidth{1.003750pt}%
\definecolor{currentstroke}{rgb}{0.000000,0.000000,0.000000}%
\pgfsetstrokecolor{currentstroke}%
\pgfsetdash{}{0pt}%
\pgfpathmoveto{\pgfqpoint{1.747838in}{1.022108in}}%
\pgfpathcurveto{\pgfqpoint{1.758888in}{1.022108in}}{\pgfqpoint{1.769487in}{1.026498in}}{\pgfqpoint{1.777300in}{1.034312in}}%
\pgfpathcurveto{\pgfqpoint{1.785114in}{1.042125in}}{\pgfqpoint{1.789504in}{1.052724in}}{\pgfqpoint{1.789504in}{1.063775in}}%
\pgfpathcurveto{\pgfqpoint{1.789504in}{1.074825in}}{\pgfqpoint{1.785114in}{1.085424in}}{\pgfqpoint{1.777300in}{1.093237in}}%
\pgfpathcurveto{\pgfqpoint{1.769487in}{1.101051in}}{\pgfqpoint{1.758888in}{1.105441in}}{\pgfqpoint{1.747838in}{1.105441in}}%
\pgfpathcurveto{\pgfqpoint{1.736788in}{1.105441in}}{\pgfqpoint{1.726189in}{1.101051in}}{\pgfqpoint{1.718375in}{1.093237in}}%
\pgfpathcurveto{\pgfqpoint{1.710561in}{1.085424in}}{\pgfqpoint{1.706171in}{1.074825in}}{\pgfqpoint{1.706171in}{1.063775in}}%
\pgfpathcurveto{\pgfqpoint{1.706171in}{1.052724in}}{\pgfqpoint{1.710561in}{1.042125in}}{\pgfqpoint{1.718375in}{1.034312in}}%
\pgfpathcurveto{\pgfqpoint{1.726189in}{1.026498in}}{\pgfqpoint{1.736788in}{1.022108in}}{\pgfqpoint{1.747838in}{1.022108in}}%
\pgfpathclose%
\pgfusepath{stroke,fill}%
\end{pgfscope}%
\begin{pgfscope}%
\pgfpathrectangle{\pgfqpoint{0.970666in}{0.566125in}}{\pgfqpoint{5.699255in}{2.685432in}}%
\pgfusepath{clip}%
\pgfsetbuttcap%
\pgfsetroundjoin%
\definecolor{currentfill}{rgb}{0.000000,0.000000,0.000000}%
\pgfsetfillcolor{currentfill}%
\pgfsetlinewidth{1.003750pt}%
\definecolor{currentstroke}{rgb}{0.000000,0.000000,0.000000}%
\pgfsetstrokecolor{currentstroke}%
\pgfsetdash{}{0pt}%
\pgfpathmoveto{\pgfqpoint{1.650691in}{0.896913in}}%
\pgfpathcurveto{\pgfqpoint{1.661741in}{0.896913in}}{\pgfqpoint{1.672340in}{0.901303in}}{\pgfqpoint{1.680154in}{0.909117in}}%
\pgfpathcurveto{\pgfqpoint{1.687968in}{0.916930in}}{\pgfqpoint{1.692358in}{0.927530in}}{\pgfqpoint{1.692358in}{0.938580in}}%
\pgfpathcurveto{\pgfqpoint{1.692358in}{0.949630in}}{\pgfqpoint{1.687968in}{0.960229in}}{\pgfqpoint{1.680154in}{0.968042in}}%
\pgfpathcurveto{\pgfqpoint{1.672340in}{0.975856in}}{\pgfqpoint{1.661741in}{0.980246in}}{\pgfqpoint{1.650691in}{0.980246in}}%
\pgfpathcurveto{\pgfqpoint{1.639641in}{0.980246in}}{\pgfqpoint{1.629042in}{0.975856in}}{\pgfqpoint{1.621228in}{0.968042in}}%
\pgfpathcurveto{\pgfqpoint{1.613415in}{0.960229in}}{\pgfqpoint{1.609025in}{0.949630in}}{\pgfqpoint{1.609025in}{0.938580in}}%
\pgfpathcurveto{\pgfqpoint{1.609025in}{0.927530in}}{\pgfqpoint{1.613415in}{0.916930in}}{\pgfqpoint{1.621228in}{0.909117in}}%
\pgfpathcurveto{\pgfqpoint{1.629042in}{0.901303in}}{\pgfqpoint{1.639641in}{0.896913in}}{\pgfqpoint{1.650691in}{0.896913in}}%
\pgfpathclose%
\pgfusepath{stroke,fill}%
\end{pgfscope}%
\begin{pgfscope}%
\pgfpathrectangle{\pgfqpoint{0.970666in}{0.566125in}}{\pgfqpoint{5.699255in}{2.685432in}}%
\pgfusepath{clip}%
\pgfsetbuttcap%
\pgfsetroundjoin%
\definecolor{currentfill}{rgb}{0.000000,0.000000,0.000000}%
\pgfsetfillcolor{currentfill}%
\pgfsetlinewidth{1.003750pt}%
\definecolor{currentstroke}{rgb}{0.000000,0.000000,0.000000}%
\pgfsetstrokecolor{currentstroke}%
\pgfsetdash{}{0pt}%
\pgfpathmoveto{\pgfqpoint{1.456398in}{0.740419in}}%
\pgfpathcurveto{\pgfqpoint{1.467449in}{0.740419in}}{\pgfqpoint{1.478048in}{0.744810in}}{\pgfqpoint{1.485861in}{0.752623in}}%
\pgfpathcurveto{\pgfqpoint{1.493675in}{0.760437in}}{\pgfqpoint{1.498065in}{0.771036in}}{\pgfqpoint{1.498065in}{0.782086in}}%
\pgfpathcurveto{\pgfqpoint{1.498065in}{0.793136in}}{\pgfqpoint{1.493675in}{0.803735in}}{\pgfqpoint{1.485861in}{0.811549in}}%
\pgfpathcurveto{\pgfqpoint{1.478048in}{0.819362in}}{\pgfqpoint{1.467449in}{0.823753in}}{\pgfqpoint{1.456398in}{0.823753in}}%
\pgfpathcurveto{\pgfqpoint{1.445348in}{0.823753in}}{\pgfqpoint{1.434749in}{0.819362in}}{\pgfqpoint{1.426936in}{0.811549in}}%
\pgfpathcurveto{\pgfqpoint{1.419122in}{0.803735in}}{\pgfqpoint{1.414732in}{0.793136in}}{\pgfqpoint{1.414732in}{0.782086in}}%
\pgfpathcurveto{\pgfqpoint{1.414732in}{0.771036in}}{\pgfqpoint{1.419122in}{0.760437in}}{\pgfqpoint{1.426936in}{0.752623in}}%
\pgfpathcurveto{\pgfqpoint{1.434749in}{0.744810in}}{\pgfqpoint{1.445348in}{0.740419in}}{\pgfqpoint{1.456398in}{0.740419in}}%
\pgfpathclose%
\pgfusepath{stroke,fill}%
\end{pgfscope}%
\begin{pgfscope}%
\pgfpathrectangle{\pgfqpoint{0.970666in}{0.566125in}}{\pgfqpoint{5.699255in}{2.685432in}}%
\pgfusepath{clip}%
\pgfsetbuttcap%
\pgfsetroundjoin%
\definecolor{currentfill}{rgb}{0.000000,0.000000,0.000000}%
\pgfsetfillcolor{currentfill}%
\pgfsetlinewidth{1.003750pt}%
\definecolor{currentstroke}{rgb}{0.000000,0.000000,0.000000}%
\pgfsetstrokecolor{currentstroke}%
\pgfsetdash{}{0pt}%
\pgfpathmoveto{\pgfqpoint{1.229724in}{1.037757in}}%
\pgfpathcurveto{\pgfqpoint{1.240774in}{1.037757in}}{\pgfqpoint{1.251373in}{1.042148in}}{\pgfqpoint{1.259186in}{1.049961in}}%
\pgfpathcurveto{\pgfqpoint{1.267000in}{1.057775in}}{\pgfqpoint{1.271390in}{1.068374in}}{\pgfqpoint{1.271390in}{1.079424in}}%
\pgfpathcurveto{\pgfqpoint{1.271390in}{1.090474in}}{\pgfqpoint{1.267000in}{1.101073in}}{\pgfqpoint{1.259186in}{1.108887in}}%
\pgfpathcurveto{\pgfqpoint{1.251373in}{1.116700in}}{\pgfqpoint{1.240774in}{1.121091in}}{\pgfqpoint{1.229724in}{1.121091in}}%
\pgfpathcurveto{\pgfqpoint{1.218673in}{1.121091in}}{\pgfqpoint{1.208074in}{1.116700in}}{\pgfqpoint{1.200261in}{1.108887in}}%
\pgfpathcurveto{\pgfqpoint{1.192447in}{1.101073in}}{\pgfqpoint{1.188057in}{1.090474in}}{\pgfqpoint{1.188057in}{1.079424in}}%
\pgfpathcurveto{\pgfqpoint{1.188057in}{1.068374in}}{\pgfqpoint{1.192447in}{1.057775in}}{\pgfqpoint{1.200261in}{1.049961in}}%
\pgfpathcurveto{\pgfqpoint{1.208074in}{1.042148in}}{\pgfqpoint{1.218673in}{1.037757in}}{\pgfqpoint{1.229724in}{1.037757in}}%
\pgfpathclose%
\pgfusepath{stroke,fill}%
\end{pgfscope}%
\begin{pgfscope}%
\pgfpathrectangle{\pgfqpoint{0.970666in}{0.566125in}}{\pgfqpoint{5.699255in}{2.685432in}}%
\pgfusepath{clip}%
\pgfsetbuttcap%
\pgfsetroundjoin%
\definecolor{currentfill}{rgb}{0.000000,0.000000,0.000000}%
\pgfsetfillcolor{currentfill}%
\pgfsetlinewidth{1.003750pt}%
\definecolor{currentstroke}{rgb}{0.000000,0.000000,0.000000}%
\pgfsetstrokecolor{currentstroke}%
\pgfsetdash{}{0pt}%
\pgfpathmoveto{\pgfqpoint{1.488781in}{1.116004in}}%
\pgfpathcurveto{\pgfqpoint{1.499831in}{1.116004in}}{\pgfqpoint{1.510430in}{1.120394in}}{\pgfqpoint{1.518243in}{1.128208in}}%
\pgfpathcurveto{\pgfqpoint{1.526057in}{1.136022in}}{\pgfqpoint{1.530447in}{1.146621in}}{\pgfqpoint{1.530447in}{1.157671in}}%
\pgfpathcurveto{\pgfqpoint{1.530447in}{1.168721in}}{\pgfqpoint{1.526057in}{1.179320in}}{\pgfqpoint{1.518243in}{1.187134in}}%
\pgfpathcurveto{\pgfqpoint{1.510430in}{1.194947in}}{\pgfqpoint{1.499831in}{1.199338in}}{\pgfqpoint{1.488781in}{1.199338in}}%
\pgfpathcurveto{\pgfqpoint{1.477730in}{1.199338in}}{\pgfqpoint{1.467131in}{1.194947in}}{\pgfqpoint{1.459318in}{1.187134in}}%
\pgfpathcurveto{\pgfqpoint{1.451504in}{1.179320in}}{\pgfqpoint{1.447114in}{1.168721in}}{\pgfqpoint{1.447114in}{1.157671in}}%
\pgfpathcurveto{\pgfqpoint{1.447114in}{1.146621in}}{\pgfqpoint{1.451504in}{1.136022in}}{\pgfqpoint{1.459318in}{1.128208in}}%
\pgfpathcurveto{\pgfqpoint{1.467131in}{1.120394in}}{\pgfqpoint{1.477730in}{1.116004in}}{\pgfqpoint{1.488781in}{1.116004in}}%
\pgfpathclose%
\pgfusepath{stroke,fill}%
\end{pgfscope}%
\begin{pgfscope}%
\pgfpathrectangle{\pgfqpoint{0.970666in}{0.566125in}}{\pgfqpoint{5.699255in}{2.685432in}}%
\pgfusepath{clip}%
\pgfsetbuttcap%
\pgfsetroundjoin%
\definecolor{currentfill}{rgb}{0.000000,0.000000,0.000000}%
\pgfsetfillcolor{currentfill}%
\pgfsetlinewidth{1.003750pt}%
\definecolor{currentstroke}{rgb}{0.000000,0.000000,0.000000}%
\pgfsetstrokecolor{currentstroke}%
\pgfsetdash{}{0pt}%
\pgfpathmoveto{\pgfqpoint{2.136423in}{1.084705in}}%
\pgfpathcurveto{\pgfqpoint{2.147473in}{1.084705in}}{\pgfqpoint{2.158072in}{1.089096in}}{\pgfqpoint{2.165886in}{1.096909in}}%
\pgfpathcurveto{\pgfqpoint{2.173700in}{1.104723in}}{\pgfqpoint{2.178090in}{1.115322in}}{\pgfqpoint{2.178090in}{1.126372in}}%
\pgfpathcurveto{\pgfqpoint{2.178090in}{1.137422in}}{\pgfqpoint{2.173700in}{1.148021in}}{\pgfqpoint{2.165886in}{1.155835in}}%
\pgfpathcurveto{\pgfqpoint{2.158072in}{1.163649in}}{\pgfqpoint{2.147473in}{1.168039in}}{\pgfqpoint{2.136423in}{1.168039in}}%
\pgfpathcurveto{\pgfqpoint{2.125373in}{1.168039in}}{\pgfqpoint{2.114774in}{1.163649in}}{\pgfqpoint{2.106960in}{1.155835in}}%
\pgfpathcurveto{\pgfqpoint{2.099147in}{1.148021in}}{\pgfqpoint{2.094757in}{1.137422in}}{\pgfqpoint{2.094757in}{1.126372in}}%
\pgfpathcurveto{\pgfqpoint{2.094757in}{1.115322in}}{\pgfqpoint{2.099147in}{1.104723in}}{\pgfqpoint{2.106960in}{1.096909in}}%
\pgfpathcurveto{\pgfqpoint{2.114774in}{1.089096in}}{\pgfqpoint{2.125373in}{1.084705in}}{\pgfqpoint{2.136423in}{1.084705in}}%
\pgfpathclose%
\pgfusepath{stroke,fill}%
\end{pgfscope}%
\begin{pgfscope}%
\pgfpathrectangle{\pgfqpoint{0.970666in}{0.566125in}}{\pgfqpoint{5.699255in}{2.685432in}}%
\pgfusepath{clip}%
\pgfsetbuttcap%
\pgfsetroundjoin%
\definecolor{currentfill}{rgb}{0.000000,0.000000,0.000000}%
\pgfsetfillcolor{currentfill}%
\pgfsetlinewidth{1.003750pt}%
\definecolor{currentstroke}{rgb}{0.000000,0.000000,0.000000}%
\pgfsetstrokecolor{currentstroke}%
\pgfsetdash{}{0pt}%
\pgfpathmoveto{\pgfqpoint{1.262106in}{1.554187in}}%
\pgfpathcurveto{\pgfqpoint{1.273156in}{1.554187in}}{\pgfqpoint{1.283755in}{1.558577in}}{\pgfqpoint{1.291568in}{1.566390in}}%
\pgfpathcurveto{\pgfqpoint{1.299382in}{1.574204in}}{\pgfqpoint{1.303772in}{1.584803in}}{\pgfqpoint{1.303772in}{1.595853in}}%
\pgfpathcurveto{\pgfqpoint{1.303772in}{1.606903in}}{\pgfqpoint{1.299382in}{1.617502in}}{\pgfqpoint{1.291568in}{1.625316in}}%
\pgfpathcurveto{\pgfqpoint{1.283755in}{1.633130in}}{\pgfqpoint{1.273156in}{1.637520in}}{\pgfqpoint{1.262106in}{1.637520in}}%
\pgfpathcurveto{\pgfqpoint{1.251056in}{1.637520in}}{\pgfqpoint{1.240457in}{1.633130in}}{\pgfqpoint{1.232643in}{1.625316in}}%
\pgfpathcurveto{\pgfqpoint{1.224829in}{1.617502in}}{\pgfqpoint{1.220439in}{1.606903in}}{\pgfqpoint{1.220439in}{1.595853in}}%
\pgfpathcurveto{\pgfqpoint{1.220439in}{1.584803in}}{\pgfqpoint{1.224829in}{1.574204in}}{\pgfqpoint{1.232643in}{1.566390in}}%
\pgfpathcurveto{\pgfqpoint{1.240457in}{1.558577in}}{\pgfqpoint{1.251056in}{1.554187in}}{\pgfqpoint{1.262106in}{1.554187in}}%
\pgfpathclose%
\pgfusepath{stroke,fill}%
\end{pgfscope}%
\begin{pgfscope}%
\pgfpathrectangle{\pgfqpoint{0.970666in}{0.566125in}}{\pgfqpoint{5.699255in}{2.685432in}}%
\pgfusepath{clip}%
\pgfsetbuttcap%
\pgfsetroundjoin%
\definecolor{currentfill}{rgb}{0.000000,0.000000,0.000000}%
\pgfsetfillcolor{currentfill}%
\pgfsetlinewidth{1.003750pt}%
\definecolor{currentstroke}{rgb}{0.000000,0.000000,0.000000}%
\pgfsetstrokecolor{currentstroke}%
\pgfsetdash{}{0pt}%
\pgfpathmoveto{\pgfqpoint{1.456398in}{2.321006in}}%
\pgfpathcurveto{\pgfqpoint{1.467449in}{2.321006in}}{\pgfqpoint{1.478048in}{2.325396in}}{\pgfqpoint{1.485861in}{2.333210in}}%
\pgfpathcurveto{\pgfqpoint{1.493675in}{2.341023in}}{\pgfqpoint{1.498065in}{2.351622in}}{\pgfqpoint{1.498065in}{2.362672in}}%
\pgfpathcurveto{\pgfqpoint{1.498065in}{2.373723in}}{\pgfqpoint{1.493675in}{2.384322in}}{\pgfqpoint{1.485861in}{2.392135in}}%
\pgfpathcurveto{\pgfqpoint{1.478048in}{2.399949in}}{\pgfqpoint{1.467449in}{2.404339in}}{\pgfqpoint{1.456398in}{2.404339in}}%
\pgfpathcurveto{\pgfqpoint{1.445348in}{2.404339in}}{\pgfqpoint{1.434749in}{2.399949in}}{\pgfqpoint{1.426936in}{2.392135in}}%
\pgfpathcurveto{\pgfqpoint{1.419122in}{2.384322in}}{\pgfqpoint{1.414732in}{2.373723in}}{\pgfqpoint{1.414732in}{2.362672in}}%
\pgfpathcurveto{\pgfqpoint{1.414732in}{2.351622in}}{\pgfqpoint{1.419122in}{2.341023in}}{\pgfqpoint{1.426936in}{2.333210in}}%
\pgfpathcurveto{\pgfqpoint{1.434749in}{2.325396in}}{\pgfqpoint{1.445348in}{2.321006in}}{\pgfqpoint{1.456398in}{2.321006in}}%
\pgfpathclose%
\pgfusepath{stroke,fill}%
\end{pgfscope}%
\begin{pgfscope}%
\pgfpathrectangle{\pgfqpoint{0.970666in}{0.566125in}}{\pgfqpoint{5.699255in}{2.685432in}}%
\pgfusepath{clip}%
\pgfsetbuttcap%
\pgfsetroundjoin%
\definecolor{currentfill}{rgb}{0.000000,0.000000,0.000000}%
\pgfsetfillcolor{currentfill}%
\pgfsetlinewidth{1.003750pt}%
\definecolor{currentstroke}{rgb}{0.000000,0.000000,0.000000}%
\pgfsetstrokecolor{currentstroke}%
\pgfsetdash{}{0pt}%
\pgfpathmoveto{\pgfqpoint{1.618309in}{2.352305in}}%
\pgfpathcurveto{\pgfqpoint{1.629359in}{2.352305in}}{\pgfqpoint{1.639958in}{2.356695in}}{\pgfqpoint{1.647772in}{2.364508in}}%
\pgfpathcurveto{\pgfqpoint{1.655586in}{2.372322in}}{\pgfqpoint{1.659976in}{2.382921in}}{\pgfqpoint{1.659976in}{2.393971in}}%
\pgfpathcurveto{\pgfqpoint{1.659976in}{2.405021in}}{\pgfqpoint{1.655586in}{2.415620in}}{\pgfqpoint{1.647772in}{2.423434in}}%
\pgfpathcurveto{\pgfqpoint{1.639958in}{2.431248in}}{\pgfqpoint{1.629359in}{2.435638in}}{\pgfqpoint{1.618309in}{2.435638in}}%
\pgfpathcurveto{\pgfqpoint{1.607259in}{2.435638in}}{\pgfqpoint{1.596660in}{2.431248in}}{\pgfqpoint{1.588846in}{2.423434in}}%
\pgfpathcurveto{\pgfqpoint{1.581033in}{2.415620in}}{\pgfqpoint{1.576642in}{2.405021in}}{\pgfqpoint{1.576642in}{2.393971in}}%
\pgfpathcurveto{\pgfqpoint{1.576642in}{2.382921in}}{\pgfqpoint{1.581033in}{2.372322in}}{\pgfqpoint{1.588846in}{2.364508in}}%
\pgfpathcurveto{\pgfqpoint{1.596660in}{2.356695in}}{\pgfqpoint{1.607259in}{2.352305in}}{\pgfqpoint{1.618309in}{2.352305in}}%
\pgfpathclose%
\pgfusepath{stroke,fill}%
\end{pgfscope}%
\begin{pgfscope}%
\pgfpathrectangle{\pgfqpoint{0.970666in}{0.566125in}}{\pgfqpoint{5.699255in}{2.685432in}}%
\pgfusepath{clip}%
\pgfsetbuttcap%
\pgfsetroundjoin%
\definecolor{currentfill}{rgb}{0.000000,0.000000,0.000000}%
\pgfsetfillcolor{currentfill}%
\pgfsetlinewidth{1.003750pt}%
\definecolor{currentstroke}{rgb}{0.000000,0.000000,0.000000}%
\pgfsetstrokecolor{currentstroke}%
\pgfsetdash{}{0pt}%
\pgfpathmoveto{\pgfqpoint{2.265952in}{2.587045in}}%
\pgfpathcurveto{\pgfqpoint{2.277002in}{2.587045in}}{\pgfqpoint{2.287601in}{2.591435in}}{\pgfqpoint{2.295415in}{2.599249in}}%
\pgfpathcurveto{\pgfqpoint{2.303228in}{2.607063in}}{\pgfqpoint{2.307618in}{2.617662in}}{\pgfqpoint{2.307618in}{2.628712in}}%
\pgfpathcurveto{\pgfqpoint{2.307618in}{2.639762in}}{\pgfqpoint{2.303228in}{2.650361in}}{\pgfqpoint{2.295415in}{2.658175in}}%
\pgfpathcurveto{\pgfqpoint{2.287601in}{2.665988in}}{\pgfqpoint{2.277002in}{2.670378in}}{\pgfqpoint{2.265952in}{2.670378in}}%
\pgfpathcurveto{\pgfqpoint{2.254902in}{2.670378in}}{\pgfqpoint{2.244303in}{2.665988in}}{\pgfqpoint{2.236489in}{2.658175in}}%
\pgfpathcurveto{\pgfqpoint{2.228675in}{2.650361in}}{\pgfqpoint{2.224285in}{2.639762in}}{\pgfqpoint{2.224285in}{2.628712in}}%
\pgfpathcurveto{\pgfqpoint{2.224285in}{2.617662in}}{\pgfqpoint{2.228675in}{2.607063in}}{\pgfqpoint{2.236489in}{2.599249in}}%
\pgfpathcurveto{\pgfqpoint{2.244303in}{2.591435in}}{\pgfqpoint{2.254902in}{2.587045in}}{\pgfqpoint{2.265952in}{2.587045in}}%
\pgfpathclose%
\pgfusepath{stroke,fill}%
\end{pgfscope}%
\begin{pgfscope}%
\pgfpathrectangle{\pgfqpoint{0.970666in}{0.566125in}}{\pgfqpoint{5.699255in}{2.685432in}}%
\pgfusepath{clip}%
\pgfsetbuttcap%
\pgfsetroundjoin%
\definecolor{currentfill}{rgb}{0.000000,0.000000,0.000000}%
\pgfsetfillcolor{currentfill}%
\pgfsetlinewidth{1.003750pt}%
\definecolor{currentstroke}{rgb}{0.000000,0.000000,0.000000}%
\pgfsetstrokecolor{currentstroke}%
\pgfsetdash{}{0pt}%
\pgfpathmoveto{\pgfqpoint{2.201188in}{2.900032in}}%
\pgfpathcurveto{\pgfqpoint{2.212238in}{2.900032in}}{\pgfqpoint{2.222837in}{2.904423in}}{\pgfqpoint{2.230650in}{2.912236in}}%
\pgfpathcurveto{\pgfqpoint{2.238464in}{2.920050in}}{\pgfqpoint{2.242854in}{2.930649in}}{\pgfqpoint{2.242854in}{2.941699in}}%
\pgfpathcurveto{\pgfqpoint{2.242854in}{2.952749in}}{\pgfqpoint{2.238464in}{2.963348in}}{\pgfqpoint{2.230650in}{2.971162in}}%
\pgfpathcurveto{\pgfqpoint{2.222837in}{2.978976in}}{\pgfqpoint{2.212238in}{2.983366in}}{\pgfqpoint{2.201188in}{2.983366in}}%
\pgfpathcurveto{\pgfqpoint{2.190137in}{2.983366in}}{\pgfqpoint{2.179538in}{2.978976in}}{\pgfqpoint{2.171725in}{2.971162in}}%
\pgfpathcurveto{\pgfqpoint{2.163911in}{2.963348in}}{\pgfqpoint{2.159521in}{2.952749in}}{\pgfqpoint{2.159521in}{2.941699in}}%
\pgfpathcurveto{\pgfqpoint{2.159521in}{2.930649in}}{\pgfqpoint{2.163911in}{2.920050in}}{\pgfqpoint{2.171725in}{2.912236in}}%
\pgfpathcurveto{\pgfqpoint{2.179538in}{2.904423in}}{\pgfqpoint{2.190137in}{2.900032in}}{\pgfqpoint{2.201188in}{2.900032in}}%
\pgfpathclose%
\pgfusepath{stroke,fill}%
\end{pgfscope}%
\begin{pgfscope}%
\pgfpathrectangle{\pgfqpoint{0.970666in}{0.566125in}}{\pgfqpoint{5.699255in}{2.685432in}}%
\pgfusepath{clip}%
\pgfsetbuttcap%
\pgfsetroundjoin%
\definecolor{currentfill}{rgb}{0.000000,0.000000,0.000000}%
\pgfsetfillcolor{currentfill}%
\pgfsetlinewidth{1.003750pt}%
\definecolor{currentstroke}{rgb}{0.000000,0.000000,0.000000}%
\pgfsetstrokecolor{currentstroke}%
\pgfsetdash{}{0pt}%
\pgfpathmoveto{\pgfqpoint{2.201188in}{2.946981in}}%
\pgfpathcurveto{\pgfqpoint{2.212238in}{2.946981in}}{\pgfqpoint{2.222837in}{2.951371in}}{\pgfqpoint{2.230650in}{2.959184in}}%
\pgfpathcurveto{\pgfqpoint{2.238464in}{2.966998in}}{\pgfqpoint{2.242854in}{2.977597in}}{\pgfqpoint{2.242854in}{2.988647in}}%
\pgfpathcurveto{\pgfqpoint{2.242854in}{2.999697in}}{\pgfqpoint{2.238464in}{3.010296in}}{\pgfqpoint{2.230650in}{3.018110in}}%
\pgfpathcurveto{\pgfqpoint{2.222837in}{3.025924in}}{\pgfqpoint{2.212238in}{3.030314in}}{\pgfqpoint{2.201188in}{3.030314in}}%
\pgfpathcurveto{\pgfqpoint{2.190137in}{3.030314in}}{\pgfqpoint{2.179538in}{3.025924in}}{\pgfqpoint{2.171725in}{3.018110in}}%
\pgfpathcurveto{\pgfqpoint{2.163911in}{3.010296in}}{\pgfqpoint{2.159521in}{2.999697in}}{\pgfqpoint{2.159521in}{2.988647in}}%
\pgfpathcurveto{\pgfqpoint{2.159521in}{2.977597in}}{\pgfqpoint{2.163911in}{2.966998in}}{\pgfqpoint{2.171725in}{2.959184in}}%
\pgfpathcurveto{\pgfqpoint{2.179538in}{2.951371in}}{\pgfqpoint{2.190137in}{2.946981in}}{\pgfqpoint{2.201188in}{2.946981in}}%
\pgfpathclose%
\pgfusepath{stroke,fill}%
\end{pgfscope}%
\begin{pgfscope}%
\pgfpathrectangle{\pgfqpoint{0.970666in}{0.566125in}}{\pgfqpoint{5.699255in}{2.685432in}}%
\pgfusepath{clip}%
\pgfsetbuttcap%
\pgfsetroundjoin%
\definecolor{currentfill}{rgb}{0.000000,0.000000,0.000000}%
\pgfsetfillcolor{currentfill}%
\pgfsetlinewidth{1.003750pt}%
\definecolor{currentstroke}{rgb}{0.000000,0.000000,0.000000}%
\pgfsetstrokecolor{currentstroke}%
\pgfsetdash{}{0pt}%
\pgfpathmoveto{\pgfqpoint{2.104041in}{2.931331in}}%
\pgfpathcurveto{\pgfqpoint{2.115091in}{2.931331in}}{\pgfqpoint{2.125690in}{2.935722in}}{\pgfqpoint{2.133504in}{2.943535in}}%
\pgfpathcurveto{\pgfqpoint{2.141318in}{2.951349in}}{\pgfqpoint{2.145708in}{2.961948in}}{\pgfqpoint{2.145708in}{2.972998in}}%
\pgfpathcurveto{\pgfqpoint{2.145708in}{2.984048in}}{\pgfqpoint{2.141318in}{2.994647in}}{\pgfqpoint{2.133504in}{3.002461in}}%
\pgfpathcurveto{\pgfqpoint{2.125690in}{3.010274in}}{\pgfqpoint{2.115091in}{3.014665in}}{\pgfqpoint{2.104041in}{3.014665in}}%
\pgfpathcurveto{\pgfqpoint{2.092991in}{3.014665in}}{\pgfqpoint{2.082392in}{3.010274in}}{\pgfqpoint{2.074578in}{3.002461in}}%
\pgfpathcurveto{\pgfqpoint{2.066765in}{2.994647in}}{\pgfqpoint{2.062374in}{2.984048in}}{\pgfqpoint{2.062374in}{2.972998in}}%
\pgfpathcurveto{\pgfqpoint{2.062374in}{2.961948in}}{\pgfqpoint{2.066765in}{2.951349in}}{\pgfqpoint{2.074578in}{2.943535in}}%
\pgfpathcurveto{\pgfqpoint{2.082392in}{2.935722in}}{\pgfqpoint{2.092991in}{2.931331in}}{\pgfqpoint{2.104041in}{2.931331in}}%
\pgfpathclose%
\pgfusepath{stroke,fill}%
\end{pgfscope}%
\begin{pgfscope}%
\pgfpathrectangle{\pgfqpoint{0.970666in}{0.566125in}}{\pgfqpoint{5.699255in}{2.685432in}}%
\pgfusepath{clip}%
\pgfsetbuttcap%
\pgfsetroundjoin%
\definecolor{currentfill}{rgb}{0.000000,0.000000,0.000000}%
\pgfsetfillcolor{currentfill}%
\pgfsetlinewidth{1.003750pt}%
\definecolor{currentstroke}{rgb}{0.000000,0.000000,0.000000}%
\pgfsetstrokecolor{currentstroke}%
\pgfsetdash{}{0pt}%
\pgfpathmoveto{\pgfqpoint{2.104041in}{3.072176in}}%
\pgfpathcurveto{\pgfqpoint{2.115091in}{3.072176in}}{\pgfqpoint{2.125690in}{3.076566in}}{\pgfqpoint{2.133504in}{3.084379in}}%
\pgfpathcurveto{\pgfqpoint{2.141318in}{3.092193in}}{\pgfqpoint{2.145708in}{3.102792in}}{\pgfqpoint{2.145708in}{3.113842in}}%
\pgfpathcurveto{\pgfqpoint{2.145708in}{3.124892in}}{\pgfqpoint{2.141318in}{3.135491in}}{\pgfqpoint{2.133504in}{3.143305in}}%
\pgfpathcurveto{\pgfqpoint{2.125690in}{3.151119in}}{\pgfqpoint{2.115091in}{3.155509in}}{\pgfqpoint{2.104041in}{3.155509in}}%
\pgfpathcurveto{\pgfqpoint{2.092991in}{3.155509in}}{\pgfqpoint{2.082392in}{3.151119in}}{\pgfqpoint{2.074578in}{3.143305in}}%
\pgfpathcurveto{\pgfqpoint{2.066765in}{3.135491in}}{\pgfqpoint{2.062374in}{3.124892in}}{\pgfqpoint{2.062374in}{3.113842in}}%
\pgfpathcurveto{\pgfqpoint{2.062374in}{3.102792in}}{\pgfqpoint{2.066765in}{3.092193in}}{\pgfqpoint{2.074578in}{3.084379in}}%
\pgfpathcurveto{\pgfqpoint{2.082392in}{3.076566in}}{\pgfqpoint{2.092991in}{3.072176in}}{\pgfqpoint{2.104041in}{3.072176in}}%
\pgfpathclose%
\pgfusepath{stroke,fill}%
\end{pgfscope}%
\begin{pgfscope}%
\pgfpathrectangle{\pgfqpoint{0.970666in}{0.566125in}}{\pgfqpoint{5.699255in}{2.685432in}}%
\pgfusepath{clip}%
\pgfsetbuttcap%
\pgfsetroundjoin%
\definecolor{currentfill}{rgb}{0.000000,0.000000,0.000000}%
\pgfsetfillcolor{currentfill}%
\pgfsetlinewidth{1.003750pt}%
\definecolor{currentstroke}{rgb}{0.000000,0.000000,0.000000}%
\pgfsetstrokecolor{currentstroke}%
\pgfsetdash{}{0pt}%
\pgfpathmoveto{\pgfqpoint{2.492627in}{3.056526in}}%
\pgfpathcurveto{\pgfqpoint{2.503677in}{3.056526in}}{\pgfqpoint{2.514276in}{3.060916in}}{\pgfqpoint{2.522090in}{3.068730in}}%
\pgfpathcurveto{\pgfqpoint{2.529903in}{3.076544in}}{\pgfqpoint{2.534293in}{3.087143in}}{\pgfqpoint{2.534293in}{3.098193in}}%
\pgfpathcurveto{\pgfqpoint{2.534293in}{3.109243in}}{\pgfqpoint{2.529903in}{3.119842in}}{\pgfqpoint{2.522090in}{3.127656in}}%
\pgfpathcurveto{\pgfqpoint{2.514276in}{3.135469in}}{\pgfqpoint{2.503677in}{3.139860in}}{\pgfqpoint{2.492627in}{3.139860in}}%
\pgfpathcurveto{\pgfqpoint{2.481577in}{3.139860in}}{\pgfqpoint{2.470978in}{3.135469in}}{\pgfqpoint{2.463164in}{3.127656in}}%
\pgfpathcurveto{\pgfqpoint{2.455350in}{3.119842in}}{\pgfqpoint{2.450960in}{3.109243in}}{\pgfqpoint{2.450960in}{3.098193in}}%
\pgfpathcurveto{\pgfqpoint{2.450960in}{3.087143in}}{\pgfqpoint{2.455350in}{3.076544in}}{\pgfqpoint{2.463164in}{3.068730in}}%
\pgfpathcurveto{\pgfqpoint{2.470978in}{3.060916in}}{\pgfqpoint{2.481577in}{3.056526in}}{\pgfqpoint{2.492627in}{3.056526in}}%
\pgfpathclose%
\pgfusepath{stroke,fill}%
\end{pgfscope}%
\begin{pgfscope}%
\pgfpathrectangle{\pgfqpoint{0.970666in}{0.566125in}}{\pgfqpoint{5.699255in}{2.685432in}}%
\pgfusepath{clip}%
\pgfsetbuttcap%
\pgfsetroundjoin%
\definecolor{currentfill}{rgb}{0.000000,0.000000,0.000000}%
\pgfsetfillcolor{currentfill}%
\pgfsetlinewidth{1.003750pt}%
\definecolor{currentstroke}{rgb}{0.000000,0.000000,0.000000}%
\pgfsetstrokecolor{currentstroke}%
\pgfsetdash{}{0pt}%
\pgfpathmoveto{\pgfqpoint{5.342254in}{1.475940in}}%
\pgfpathcurveto{\pgfqpoint{5.353305in}{1.475940in}}{\pgfqpoint{5.363904in}{1.480330in}}{\pgfqpoint{5.371717in}{1.488144in}}%
\pgfpathcurveto{\pgfqpoint{5.379531in}{1.495957in}}{\pgfqpoint{5.383921in}{1.506556in}}{\pgfqpoint{5.383921in}{1.517606in}}%
\pgfpathcurveto{\pgfqpoint{5.383921in}{1.528657in}}{\pgfqpoint{5.379531in}{1.539256in}}{\pgfqpoint{5.371717in}{1.547069in}}%
\pgfpathcurveto{\pgfqpoint{5.363904in}{1.554883in}}{\pgfqpoint{5.353305in}{1.559273in}}{\pgfqpoint{5.342254in}{1.559273in}}%
\pgfpathcurveto{\pgfqpoint{5.331204in}{1.559273in}}{\pgfqpoint{5.320605in}{1.554883in}}{\pgfqpoint{5.312792in}{1.547069in}}%
\pgfpathcurveto{\pgfqpoint{5.304978in}{1.539256in}}{\pgfqpoint{5.300588in}{1.528657in}}{\pgfqpoint{5.300588in}{1.517606in}}%
\pgfpathcurveto{\pgfqpoint{5.300588in}{1.506556in}}{\pgfqpoint{5.304978in}{1.495957in}}{\pgfqpoint{5.312792in}{1.488144in}}%
\pgfpathcurveto{\pgfqpoint{5.320605in}{1.480330in}}{\pgfqpoint{5.331204in}{1.475940in}}{\pgfqpoint{5.342254in}{1.475940in}}%
\pgfpathclose%
\pgfusepath{stroke,fill}%
\end{pgfscope}%
\begin{pgfscope}%
\pgfpathrectangle{\pgfqpoint{0.970666in}{0.566125in}}{\pgfqpoint{5.699255in}{2.685432in}}%
\pgfusepath{clip}%
\pgfsetbuttcap%
\pgfsetroundjoin%
\definecolor{currentfill}{rgb}{0.000000,0.000000,0.000000}%
\pgfsetfillcolor{currentfill}%
\pgfsetlinewidth{1.003750pt}%
\definecolor{currentstroke}{rgb}{0.000000,0.000000,0.000000}%
\pgfsetstrokecolor{currentstroke}%
\pgfsetdash{}{0pt}%
\pgfpathmoveto{\pgfqpoint{5.374637in}{1.288147in}}%
\pgfpathcurveto{\pgfqpoint{5.385687in}{1.288147in}}{\pgfqpoint{5.396286in}{1.292538in}}{\pgfqpoint{5.404099in}{1.300351in}}%
\pgfpathcurveto{\pgfqpoint{5.411913in}{1.308165in}}{\pgfqpoint{5.416303in}{1.318764in}}{\pgfqpoint{5.416303in}{1.329814in}}%
\pgfpathcurveto{\pgfqpoint{5.416303in}{1.340864in}}{\pgfqpoint{5.411913in}{1.351463in}}{\pgfqpoint{5.404099in}{1.359277in}}%
\pgfpathcurveto{\pgfqpoint{5.396286in}{1.367090in}}{\pgfqpoint{5.385687in}{1.371481in}}{\pgfqpoint{5.374637in}{1.371481in}}%
\pgfpathcurveto{\pgfqpoint{5.363586in}{1.371481in}}{\pgfqpoint{5.352987in}{1.367090in}}{\pgfqpoint{5.345174in}{1.359277in}}%
\pgfpathcurveto{\pgfqpoint{5.337360in}{1.351463in}}{\pgfqpoint{5.332970in}{1.340864in}}{\pgfqpoint{5.332970in}{1.329814in}}%
\pgfpathcurveto{\pgfqpoint{5.332970in}{1.318764in}}{\pgfqpoint{5.337360in}{1.308165in}}{\pgfqpoint{5.345174in}{1.300351in}}%
\pgfpathcurveto{\pgfqpoint{5.352987in}{1.292538in}}{\pgfqpoint{5.363586in}{1.288147in}}{\pgfqpoint{5.374637in}{1.288147in}}%
\pgfpathclose%
\pgfusepath{stroke,fill}%
\end{pgfscope}%
\begin{pgfscope}%
\pgfpathrectangle{\pgfqpoint{0.970666in}{0.566125in}}{\pgfqpoint{5.699255in}{2.685432in}}%
\pgfusepath{clip}%
\pgfsetbuttcap%
\pgfsetroundjoin%
\definecolor{currentfill}{rgb}{0.000000,0.000000,0.000000}%
\pgfsetfillcolor{currentfill}%
\pgfsetlinewidth{1.003750pt}%
\definecolor{currentstroke}{rgb}{0.000000,0.000000,0.000000}%
\pgfsetstrokecolor{currentstroke}%
\pgfsetdash{}{0pt}%
\pgfpathmoveto{\pgfqpoint{5.277490in}{1.225550in}}%
\pgfpathcurveto{\pgfqpoint{5.288540in}{1.225550in}}{\pgfqpoint{5.299139in}{1.229940in}}{\pgfqpoint{5.306953in}{1.237754in}}%
\pgfpathcurveto{\pgfqpoint{5.314767in}{1.245567in}}{\pgfqpoint{5.319157in}{1.256166in}}{\pgfqpoint{5.319157in}{1.267216in}}%
\pgfpathcurveto{\pgfqpoint{5.319157in}{1.278267in}}{\pgfqpoint{5.314767in}{1.288866in}}{\pgfqpoint{5.306953in}{1.296679in}}%
\pgfpathcurveto{\pgfqpoint{5.299139in}{1.304493in}}{\pgfqpoint{5.288540in}{1.308883in}}{\pgfqpoint{5.277490in}{1.308883in}}%
\pgfpathcurveto{\pgfqpoint{5.266440in}{1.308883in}}{\pgfqpoint{5.255841in}{1.304493in}}{\pgfqpoint{5.248027in}{1.296679in}}%
\pgfpathcurveto{\pgfqpoint{5.240214in}{1.288866in}}{\pgfqpoint{5.235823in}{1.278267in}}{\pgfqpoint{5.235823in}{1.267216in}}%
\pgfpathcurveto{\pgfqpoint{5.235823in}{1.256166in}}{\pgfqpoint{5.240214in}{1.245567in}}{\pgfqpoint{5.248027in}{1.237754in}}%
\pgfpathcurveto{\pgfqpoint{5.255841in}{1.229940in}}{\pgfqpoint{5.266440in}{1.225550in}}{\pgfqpoint{5.277490in}{1.225550in}}%
\pgfpathclose%
\pgfusepath{stroke,fill}%
\end{pgfscope}%
\begin{pgfscope}%
\pgfpathrectangle{\pgfqpoint{0.970666in}{0.566125in}}{\pgfqpoint{5.699255in}{2.685432in}}%
\pgfusepath{clip}%
\pgfsetbuttcap%
\pgfsetroundjoin%
\definecolor{currentfill}{rgb}{0.000000,0.000000,0.000000}%
\pgfsetfillcolor{currentfill}%
\pgfsetlinewidth{1.003750pt}%
\definecolor{currentstroke}{rgb}{0.000000,0.000000,0.000000}%
\pgfsetstrokecolor{currentstroke}%
\pgfsetdash{}{0pt}%
\pgfpathmoveto{\pgfqpoint{5.277490in}{1.147303in}}%
\pgfpathcurveto{\pgfqpoint{5.288540in}{1.147303in}}{\pgfqpoint{5.299139in}{1.151693in}}{\pgfqpoint{5.306953in}{1.159507in}}%
\pgfpathcurveto{\pgfqpoint{5.314767in}{1.167320in}}{\pgfqpoint{5.319157in}{1.177919in}}{\pgfqpoint{5.319157in}{1.188970in}}%
\pgfpathcurveto{\pgfqpoint{5.319157in}{1.200020in}}{\pgfqpoint{5.314767in}{1.210619in}}{\pgfqpoint{5.306953in}{1.218432in}}%
\pgfpathcurveto{\pgfqpoint{5.299139in}{1.226246in}}{\pgfqpoint{5.288540in}{1.230636in}}{\pgfqpoint{5.277490in}{1.230636in}}%
\pgfpathcurveto{\pgfqpoint{5.266440in}{1.230636in}}{\pgfqpoint{5.255841in}{1.226246in}}{\pgfqpoint{5.248027in}{1.218432in}}%
\pgfpathcurveto{\pgfqpoint{5.240214in}{1.210619in}}{\pgfqpoint{5.235823in}{1.200020in}}{\pgfqpoint{5.235823in}{1.188970in}}%
\pgfpathcurveto{\pgfqpoint{5.235823in}{1.177919in}}{\pgfqpoint{5.240214in}{1.167320in}}{\pgfqpoint{5.248027in}{1.159507in}}%
\pgfpathcurveto{\pgfqpoint{5.255841in}{1.151693in}}{\pgfqpoint{5.266440in}{1.147303in}}{\pgfqpoint{5.277490in}{1.147303in}}%
\pgfpathclose%
\pgfusepath{stroke,fill}%
\end{pgfscope}%
\begin{pgfscope}%
\pgfpathrectangle{\pgfqpoint{0.970666in}{0.566125in}}{\pgfqpoint{5.699255in}{2.685432in}}%
\pgfusepath{clip}%
\pgfsetbuttcap%
\pgfsetroundjoin%
\definecolor{currentfill}{rgb}{0.000000,0.000000,0.000000}%
\pgfsetfillcolor{currentfill}%
\pgfsetlinewidth{1.003750pt}%
\definecolor{currentstroke}{rgb}{0.000000,0.000000,0.000000}%
\pgfsetstrokecolor{currentstroke}%
\pgfsetdash{}{0pt}%
\pgfpathmoveto{\pgfqpoint{5.471783in}{1.116004in}}%
\pgfpathcurveto{\pgfqpoint{5.482833in}{1.116004in}}{\pgfqpoint{5.493432in}{1.120394in}}{\pgfqpoint{5.501246in}{1.128208in}}%
\pgfpathcurveto{\pgfqpoint{5.509059in}{1.136022in}}{\pgfqpoint{5.513450in}{1.146621in}}{\pgfqpoint{5.513450in}{1.157671in}}%
\pgfpathcurveto{\pgfqpoint{5.513450in}{1.168721in}}{\pgfqpoint{5.509059in}{1.179320in}}{\pgfqpoint{5.501246in}{1.187134in}}%
\pgfpathcurveto{\pgfqpoint{5.493432in}{1.194947in}}{\pgfqpoint{5.482833in}{1.199338in}}{\pgfqpoint{5.471783in}{1.199338in}}%
\pgfpathcurveto{\pgfqpoint{5.460733in}{1.199338in}}{\pgfqpoint{5.450134in}{1.194947in}}{\pgfqpoint{5.442320in}{1.187134in}}%
\pgfpathcurveto{\pgfqpoint{5.434507in}{1.179320in}}{\pgfqpoint{5.430116in}{1.168721in}}{\pgfqpoint{5.430116in}{1.157671in}}%
\pgfpathcurveto{\pgfqpoint{5.430116in}{1.146621in}}{\pgfqpoint{5.434507in}{1.136022in}}{\pgfqpoint{5.442320in}{1.128208in}}%
\pgfpathcurveto{\pgfqpoint{5.450134in}{1.120394in}}{\pgfqpoint{5.460733in}{1.116004in}}{\pgfqpoint{5.471783in}{1.116004in}}%
\pgfpathclose%
\pgfusepath{stroke,fill}%
\end{pgfscope}%
\begin{pgfscope}%
\pgfpathrectangle{\pgfqpoint{0.970666in}{0.566125in}}{\pgfqpoint{5.699255in}{2.685432in}}%
\pgfusepath{clip}%
\pgfsetbuttcap%
\pgfsetroundjoin%
\definecolor{currentfill}{rgb}{0.000000,0.000000,0.000000}%
\pgfsetfillcolor{currentfill}%
\pgfsetlinewidth{1.003750pt}%
\definecolor{currentstroke}{rgb}{0.000000,0.000000,0.000000}%
\pgfsetstrokecolor{currentstroke}%
\pgfsetdash{}{0pt}%
\pgfpathmoveto{\pgfqpoint{4.888905in}{1.116004in}}%
\pgfpathcurveto{\pgfqpoint{4.899955in}{1.116004in}}{\pgfqpoint{4.910554in}{1.120394in}}{\pgfqpoint{4.918367in}{1.128208in}}%
\pgfpathcurveto{\pgfqpoint{4.926181in}{1.136022in}}{\pgfqpoint{4.930571in}{1.146621in}}{\pgfqpoint{4.930571in}{1.157671in}}%
\pgfpathcurveto{\pgfqpoint{4.930571in}{1.168721in}}{\pgfqpoint{4.926181in}{1.179320in}}{\pgfqpoint{4.918367in}{1.187134in}}%
\pgfpathcurveto{\pgfqpoint{4.910554in}{1.194947in}}{\pgfqpoint{4.899955in}{1.199338in}}{\pgfqpoint{4.888905in}{1.199338in}}%
\pgfpathcurveto{\pgfqpoint{4.877854in}{1.199338in}}{\pgfqpoint{4.867255in}{1.194947in}}{\pgfqpoint{4.859442in}{1.187134in}}%
\pgfpathcurveto{\pgfqpoint{4.851628in}{1.179320in}}{\pgfqpoint{4.847238in}{1.168721in}}{\pgfqpoint{4.847238in}{1.157671in}}%
\pgfpathcurveto{\pgfqpoint{4.847238in}{1.146621in}}{\pgfqpoint{4.851628in}{1.136022in}}{\pgfqpoint{4.859442in}{1.128208in}}%
\pgfpathcurveto{\pgfqpoint{4.867255in}{1.120394in}}{\pgfqpoint{4.877854in}{1.116004in}}{\pgfqpoint{4.888905in}{1.116004in}}%
\pgfpathclose%
\pgfusepath{stroke,fill}%
\end{pgfscope}%
\begin{pgfscope}%
\pgfpathrectangle{\pgfqpoint{0.970666in}{0.566125in}}{\pgfqpoint{5.699255in}{2.685432in}}%
\pgfusepath{clip}%
\pgfsetbuttcap%
\pgfsetroundjoin%
\definecolor{currentfill}{rgb}{0.000000,0.000000,0.000000}%
\pgfsetfillcolor{currentfill}%
\pgfsetlinewidth{1.003750pt}%
\definecolor{currentstroke}{rgb}{0.000000,0.000000,0.000000}%
\pgfsetstrokecolor{currentstroke}%
\pgfsetdash{}{0pt}%
\pgfpathmoveto{\pgfqpoint{4.759376in}{1.178602in}}%
\pgfpathcurveto{\pgfqpoint{4.770426in}{1.178602in}}{\pgfqpoint{4.781025in}{1.182992in}}{\pgfqpoint{4.788839in}{1.190806in}}%
\pgfpathcurveto{\pgfqpoint{4.796652in}{1.198619in}}{\pgfqpoint{4.801043in}{1.209218in}}{\pgfqpoint{4.801043in}{1.220268in}}%
\pgfpathcurveto{\pgfqpoint{4.801043in}{1.231318in}}{\pgfqpoint{4.796652in}{1.241917in}}{\pgfqpoint{4.788839in}{1.249731in}}%
\pgfpathcurveto{\pgfqpoint{4.781025in}{1.257545in}}{\pgfqpoint{4.770426in}{1.261935in}}{\pgfqpoint{4.759376in}{1.261935in}}%
\pgfpathcurveto{\pgfqpoint{4.748326in}{1.261935in}}{\pgfqpoint{4.737727in}{1.257545in}}{\pgfqpoint{4.729913in}{1.249731in}}%
\pgfpathcurveto{\pgfqpoint{4.722100in}{1.241917in}}{\pgfqpoint{4.717709in}{1.231318in}}{\pgfqpoint{4.717709in}{1.220268in}}%
\pgfpathcurveto{\pgfqpoint{4.717709in}{1.209218in}}{\pgfqpoint{4.722100in}{1.198619in}}{\pgfqpoint{4.729913in}{1.190806in}}%
\pgfpathcurveto{\pgfqpoint{4.737727in}{1.182992in}}{\pgfqpoint{4.748326in}{1.178602in}}{\pgfqpoint{4.759376in}{1.178602in}}%
\pgfpathclose%
\pgfusepath{stroke,fill}%
\end{pgfscope}%
\begin{pgfscope}%
\pgfpathrectangle{\pgfqpoint{0.970666in}{0.566125in}}{\pgfqpoint{5.699255in}{2.685432in}}%
\pgfusepath{clip}%
\pgfsetbuttcap%
\pgfsetroundjoin%
\definecolor{currentfill}{rgb}{0.000000,0.000000,0.000000}%
\pgfsetfillcolor{currentfill}%
\pgfsetlinewidth{1.003750pt}%
\definecolor{currentstroke}{rgb}{0.000000,0.000000,0.000000}%
\pgfsetstrokecolor{currentstroke}%
\pgfsetdash{}{0pt}%
\pgfpathmoveto{\pgfqpoint{4.565083in}{0.849965in}}%
\pgfpathcurveto{\pgfqpoint{4.576133in}{0.849965in}}{\pgfqpoint{4.586732in}{0.854355in}}{\pgfqpoint{4.594546in}{0.862169in}}%
\pgfpathcurveto{\pgfqpoint{4.602360in}{0.869982in}}{\pgfqpoint{4.606750in}{0.880581in}}{\pgfqpoint{4.606750in}{0.891632in}}%
\pgfpathcurveto{\pgfqpoint{4.606750in}{0.902682in}}{\pgfqpoint{4.602360in}{0.913281in}}{\pgfqpoint{4.594546in}{0.921094in}}%
\pgfpathcurveto{\pgfqpoint{4.586732in}{0.928908in}}{\pgfqpoint{4.576133in}{0.933298in}}{\pgfqpoint{4.565083in}{0.933298in}}%
\pgfpathcurveto{\pgfqpoint{4.554033in}{0.933298in}}{\pgfqpoint{4.543434in}{0.928908in}}{\pgfqpoint{4.535620in}{0.921094in}}%
\pgfpathcurveto{\pgfqpoint{4.527807in}{0.913281in}}{\pgfqpoint{4.523417in}{0.902682in}}{\pgfqpoint{4.523417in}{0.891632in}}%
\pgfpathcurveto{\pgfqpoint{4.523417in}{0.880581in}}{\pgfqpoint{4.527807in}{0.869982in}}{\pgfqpoint{4.535620in}{0.862169in}}%
\pgfpathcurveto{\pgfqpoint{4.543434in}{0.854355in}}{\pgfqpoint{4.554033in}{0.849965in}}{\pgfqpoint{4.565083in}{0.849965in}}%
\pgfpathclose%
\pgfusepath{stroke,fill}%
\end{pgfscope}%
\begin{pgfscope}%
\pgfpathrectangle{\pgfqpoint{0.970666in}{0.566125in}}{\pgfqpoint{5.699255in}{2.685432in}}%
\pgfusepath{clip}%
\pgfsetbuttcap%
\pgfsetroundjoin%
\definecolor{currentfill}{rgb}{0.000000,0.000000,0.000000}%
\pgfsetfillcolor{currentfill}%
\pgfsetlinewidth{1.003750pt}%
\definecolor{currentstroke}{rgb}{0.000000,0.000000,0.000000}%
\pgfsetstrokecolor{currentstroke}%
\pgfsetdash{}{0pt}%
\pgfpathmoveto{\pgfqpoint{4.791758in}{0.646523in}}%
\pgfpathcurveto{\pgfqpoint{4.802808in}{0.646523in}}{\pgfqpoint{4.813407in}{0.650913in}}{\pgfqpoint{4.821221in}{0.658727in}}%
\pgfpathcurveto{\pgfqpoint{4.829035in}{0.666541in}}{\pgfqpoint{4.833425in}{0.677140in}}{\pgfqpoint{4.833425in}{0.688190in}}%
\pgfpathcurveto{\pgfqpoint{4.833425in}{0.699240in}}{\pgfqpoint{4.829035in}{0.709839in}}{\pgfqpoint{4.821221in}{0.717652in}}%
\pgfpathcurveto{\pgfqpoint{4.813407in}{0.725466in}}{\pgfqpoint{4.802808in}{0.729856in}}{\pgfqpoint{4.791758in}{0.729856in}}%
\pgfpathcurveto{\pgfqpoint{4.780708in}{0.729856in}}{\pgfqpoint{4.770109in}{0.725466in}}{\pgfqpoint{4.762295in}{0.717652in}}%
\pgfpathcurveto{\pgfqpoint{4.754482in}{0.709839in}}{\pgfqpoint{4.750092in}{0.699240in}}{\pgfqpoint{4.750092in}{0.688190in}}%
\pgfpathcurveto{\pgfqpoint{4.750092in}{0.677140in}}{\pgfqpoint{4.754482in}{0.666541in}}{\pgfqpoint{4.762295in}{0.658727in}}%
\pgfpathcurveto{\pgfqpoint{4.770109in}{0.650913in}}{\pgfqpoint{4.780708in}{0.646523in}}{\pgfqpoint{4.791758in}{0.646523in}}%
\pgfpathclose%
\pgfusepath{stroke,fill}%
\end{pgfscope}%
\begin{pgfscope}%
\pgfpathrectangle{\pgfqpoint{0.970666in}{0.566125in}}{\pgfqpoint{5.699255in}{2.685432in}}%
\pgfusepath{clip}%
\pgfsetbuttcap%
\pgfsetroundjoin%
\definecolor{currentfill}{rgb}{0.000000,0.000000,0.000000}%
\pgfsetfillcolor{currentfill}%
\pgfsetlinewidth{1.003750pt}%
\definecolor{currentstroke}{rgb}{0.000000,0.000000,0.000000}%
\pgfsetstrokecolor{currentstroke}%
\pgfsetdash{}{0pt}%
\pgfpathmoveto{\pgfqpoint{5.309872in}{0.646523in}}%
\pgfpathcurveto{\pgfqpoint{5.320922in}{0.646523in}}{\pgfqpoint{5.331521in}{0.650913in}}{\pgfqpoint{5.339335in}{0.658727in}}%
\pgfpathcurveto{\pgfqpoint{5.347149in}{0.666541in}}{\pgfqpoint{5.351539in}{0.677140in}}{\pgfqpoint{5.351539in}{0.688190in}}%
\pgfpathcurveto{\pgfqpoint{5.351539in}{0.699240in}}{\pgfqpoint{5.347149in}{0.709839in}}{\pgfqpoint{5.339335in}{0.717652in}}%
\pgfpathcurveto{\pgfqpoint{5.331521in}{0.725466in}}{\pgfqpoint{5.320922in}{0.729856in}}{\pgfqpoint{5.309872in}{0.729856in}}%
\pgfpathcurveto{\pgfqpoint{5.298822in}{0.729856in}}{\pgfqpoint{5.288223in}{0.725466in}}{\pgfqpoint{5.280410in}{0.717652in}}%
\pgfpathcurveto{\pgfqpoint{5.272596in}{0.709839in}}{\pgfqpoint{5.268206in}{0.699240in}}{\pgfqpoint{5.268206in}{0.688190in}}%
\pgfpathcurveto{\pgfqpoint{5.268206in}{0.677140in}}{\pgfqpoint{5.272596in}{0.666541in}}{\pgfqpoint{5.280410in}{0.658727in}}%
\pgfpathcurveto{\pgfqpoint{5.288223in}{0.650913in}}{\pgfqpoint{5.298822in}{0.646523in}}{\pgfqpoint{5.309872in}{0.646523in}}%
\pgfpathclose%
\pgfusepath{stroke,fill}%
\end{pgfscope}%
\begin{pgfscope}%
\pgfpathrectangle{\pgfqpoint{0.970666in}{0.566125in}}{\pgfqpoint{5.699255in}{2.685432in}}%
\pgfusepath{clip}%
\pgfsetbuttcap%
\pgfsetroundjoin%
\definecolor{currentfill}{rgb}{0.000000,0.000000,0.000000}%
\pgfsetfillcolor{currentfill}%
\pgfsetlinewidth{1.003750pt}%
\definecolor{currentstroke}{rgb}{0.000000,0.000000,0.000000}%
\pgfsetstrokecolor{currentstroke}%
\pgfsetdash{}{0pt}%
\pgfpathmoveto{\pgfqpoint{3.982205in}{0.787367in}}%
\pgfpathcurveto{\pgfqpoint{3.993255in}{0.787367in}}{\pgfqpoint{4.003854in}{0.791758in}}{\pgfqpoint{4.011668in}{0.799571in}}%
\pgfpathcurveto{\pgfqpoint{4.019481in}{0.807385in}}{\pgfqpoint{4.023872in}{0.817984in}}{\pgfqpoint{4.023872in}{0.829034in}}%
\pgfpathcurveto{\pgfqpoint{4.023872in}{0.840084in}}{\pgfqpoint{4.019481in}{0.850683in}}{\pgfqpoint{4.011668in}{0.858497in}}%
\pgfpathcurveto{\pgfqpoint{4.003854in}{0.866310in}}{\pgfqpoint{3.993255in}{0.870701in}}{\pgfqpoint{3.982205in}{0.870701in}}%
\pgfpathcurveto{\pgfqpoint{3.971155in}{0.870701in}}{\pgfqpoint{3.960556in}{0.866310in}}{\pgfqpoint{3.952742in}{0.858497in}}%
\pgfpathcurveto{\pgfqpoint{3.944928in}{0.850683in}}{\pgfqpoint{3.940538in}{0.840084in}}{\pgfqpoint{3.940538in}{0.829034in}}%
\pgfpathcurveto{\pgfqpoint{3.940538in}{0.817984in}}{\pgfqpoint{3.944928in}{0.807385in}}{\pgfqpoint{3.952742in}{0.799571in}}%
\pgfpathcurveto{\pgfqpoint{3.960556in}{0.791758in}}{\pgfqpoint{3.971155in}{0.787367in}}{\pgfqpoint{3.982205in}{0.787367in}}%
\pgfpathclose%
\pgfusepath{stroke,fill}%
\end{pgfscope}%
\begin{pgfscope}%
\pgfpathrectangle{\pgfqpoint{0.970666in}{0.566125in}}{\pgfqpoint{5.699255in}{2.685432in}}%
\pgfusepath{clip}%
\pgfsetbuttcap%
\pgfsetroundjoin%
\definecolor{currentfill}{rgb}{0.000000,0.000000,0.000000}%
\pgfsetfillcolor{currentfill}%
\pgfsetlinewidth{1.003750pt}%
\definecolor{currentstroke}{rgb}{0.000000,0.000000,0.000000}%
\pgfsetstrokecolor{currentstroke}%
\pgfsetdash{}{0pt}%
\pgfpathmoveto{\pgfqpoint{3.949823in}{0.756069in}}%
\pgfpathcurveto{\pgfqpoint{3.960873in}{0.756069in}}{\pgfqpoint{3.971472in}{0.760459in}}{\pgfqpoint{3.979285in}{0.768273in}}%
\pgfpathcurveto{\pgfqpoint{3.987099in}{0.776086in}}{\pgfqpoint{3.991489in}{0.786685in}}{\pgfqpoint{3.991489in}{0.797735in}}%
\pgfpathcurveto{\pgfqpoint{3.991489in}{0.808785in}}{\pgfqpoint{3.987099in}{0.819384in}}{\pgfqpoint{3.979285in}{0.827198in}}%
\pgfpathcurveto{\pgfqpoint{3.971472in}{0.835012in}}{\pgfqpoint{3.960873in}{0.839402in}}{\pgfqpoint{3.949823in}{0.839402in}}%
\pgfpathcurveto{\pgfqpoint{3.938773in}{0.839402in}}{\pgfqpoint{3.928174in}{0.835012in}}{\pgfqpoint{3.920360in}{0.827198in}}%
\pgfpathcurveto{\pgfqpoint{3.912546in}{0.819384in}}{\pgfqpoint{3.908156in}{0.808785in}}{\pgfqpoint{3.908156in}{0.797735in}}%
\pgfpathcurveto{\pgfqpoint{3.908156in}{0.786685in}}{\pgfqpoint{3.912546in}{0.776086in}}{\pgfqpoint{3.920360in}{0.768273in}}%
\pgfpathcurveto{\pgfqpoint{3.928174in}{0.760459in}}{\pgfqpoint{3.938773in}{0.756069in}}{\pgfqpoint{3.949823in}{0.756069in}}%
\pgfpathclose%
\pgfusepath{stroke,fill}%
\end{pgfscope}%
\begin{pgfscope}%
\pgfpathrectangle{\pgfqpoint{0.970666in}{0.566125in}}{\pgfqpoint{5.699255in}{2.685432in}}%
\pgfusepath{clip}%
\pgfsetbuttcap%
\pgfsetroundjoin%
\definecolor{currentfill}{rgb}{0.000000,0.000000,0.000000}%
\pgfsetfillcolor{currentfill}%
\pgfsetlinewidth{1.003750pt}%
\definecolor{currentstroke}{rgb}{0.000000,0.000000,0.000000}%
\pgfsetstrokecolor{currentstroke}%
\pgfsetdash{}{0pt}%
\pgfpathmoveto{\pgfqpoint{5.795604in}{1.413342in}}%
\pgfpathcurveto{\pgfqpoint{5.806654in}{1.413342in}}{\pgfqpoint{5.817253in}{1.417732in}}{\pgfqpoint{5.825067in}{1.425546in}}%
\pgfpathcurveto{\pgfqpoint{5.832881in}{1.433360in}}{\pgfqpoint{5.837271in}{1.443959in}}{\pgfqpoint{5.837271in}{1.455009in}}%
\pgfpathcurveto{\pgfqpoint{5.837271in}{1.466059in}}{\pgfqpoint{5.832881in}{1.476658in}}{\pgfqpoint{5.825067in}{1.484472in}}%
\pgfpathcurveto{\pgfqpoint{5.817253in}{1.492285in}}{\pgfqpoint{5.806654in}{1.496676in}}{\pgfqpoint{5.795604in}{1.496676in}}%
\pgfpathcurveto{\pgfqpoint{5.784554in}{1.496676in}}{\pgfqpoint{5.773955in}{1.492285in}}{\pgfqpoint{5.766142in}{1.484472in}}%
\pgfpathcurveto{\pgfqpoint{5.758328in}{1.476658in}}{\pgfqpoint{5.753938in}{1.466059in}}{\pgfqpoint{5.753938in}{1.455009in}}%
\pgfpathcurveto{\pgfqpoint{5.753938in}{1.443959in}}{\pgfqpoint{5.758328in}{1.433360in}}{\pgfqpoint{5.766142in}{1.425546in}}%
\pgfpathcurveto{\pgfqpoint{5.773955in}{1.417732in}}{\pgfqpoint{5.784554in}{1.413342in}}{\pgfqpoint{5.795604in}{1.413342in}}%
\pgfpathclose%
\pgfusepath{stroke,fill}%
\end{pgfscope}%
\begin{pgfscope}%
\pgfpathrectangle{\pgfqpoint{0.970666in}{0.566125in}}{\pgfqpoint{5.699255in}{2.685432in}}%
\pgfusepath{clip}%
\pgfsetbuttcap%
\pgfsetroundjoin%
\definecolor{currentfill}{rgb}{0.000000,0.000000,1.000000}%
\pgfsetfillcolor{currentfill}%
\pgfsetlinewidth{1.003750pt}%
\definecolor{currentstroke}{rgb}{0.000000,0.000000,1.000000}%
\pgfsetstrokecolor{currentstroke}%
\pgfsetdash{}{0pt}%
\pgfpathmoveto{\pgfqpoint{6.184190in}{1.241199in}}%
\pgfpathcurveto{\pgfqpoint{6.195240in}{1.241199in}}{\pgfqpoint{6.205839in}{1.245589in}}{\pgfqpoint{6.213653in}{1.253403in}}%
\pgfpathcurveto{\pgfqpoint{6.221466in}{1.261217in}}{\pgfqpoint{6.225857in}{1.271816in}}{\pgfqpoint{6.225857in}{1.282866in}}%
\pgfpathcurveto{\pgfqpoint{6.225857in}{1.293916in}}{\pgfqpoint{6.221466in}{1.304515in}}{\pgfqpoint{6.213653in}{1.312329in}}%
\pgfpathcurveto{\pgfqpoint{6.205839in}{1.320142in}}{\pgfqpoint{6.195240in}{1.324532in}}{\pgfqpoint{6.184190in}{1.324532in}}%
\pgfpathcurveto{\pgfqpoint{6.173140in}{1.324532in}}{\pgfqpoint{6.162541in}{1.320142in}}{\pgfqpoint{6.154727in}{1.312329in}}%
\pgfpathcurveto{\pgfqpoint{6.146913in}{1.304515in}}{\pgfqpoint{6.142523in}{1.293916in}}{\pgfqpoint{6.142523in}{1.282866in}}%
\pgfpathcurveto{\pgfqpoint{6.142523in}{1.271816in}}{\pgfqpoint{6.146913in}{1.261217in}}{\pgfqpoint{6.154727in}{1.253403in}}%
\pgfpathcurveto{\pgfqpoint{6.162541in}{1.245589in}}{\pgfqpoint{6.173140in}{1.241199in}}{\pgfqpoint{6.184190in}{1.241199in}}%
\pgfpathclose%
\pgfusepath{stroke,fill}%
\end{pgfscope}%
\begin{pgfscope}%
\pgfsetbuttcap%
\pgfsetroundjoin%
\definecolor{currentfill}{rgb}{0.000000,0.000000,0.000000}%
\pgfsetfillcolor{currentfill}%
\pgfsetlinewidth{0.803000pt}%
\definecolor{currentstroke}{rgb}{0.000000,0.000000,0.000000}%
\pgfsetstrokecolor{currentstroke}%
\pgfsetdash{}{0pt}%
\pgfsys@defobject{currentmarker}{\pgfqpoint{0.000000in}{-0.048611in}}{\pgfqpoint{0.000000in}{0.000000in}}{%
\pgfpathmoveto{\pgfqpoint{0.000000in}{0.000000in}}%
\pgfpathlineto{\pgfqpoint{0.000000in}{-0.048611in}}%
\pgfusepath{stroke,fill}%
}%
\begin{pgfscope}%
\pgfsys@transformshift{1.132577in}{0.566125in}%
\pgfsys@useobject{currentmarker}{}%
\end{pgfscope}%
\end{pgfscope}%
\begin{pgfscope}%
\definecolor{textcolor}{rgb}{0.000000,0.000000,0.000000}%
\pgfsetstrokecolor{textcolor}%
\pgfsetfillcolor{textcolor}%
\pgftext[x=1.132577in,y=0.468902in,,top]{\color{textcolor}\rmfamily\fontsize{10.000000}{12.000000}\selectfont \(\displaystyle 0\)}%
\end{pgfscope}%
\begin{pgfscope}%
\pgfsetbuttcap%
\pgfsetroundjoin%
\definecolor{currentfill}{rgb}{0.000000,0.000000,0.000000}%
\pgfsetfillcolor{currentfill}%
\pgfsetlinewidth{0.803000pt}%
\definecolor{currentstroke}{rgb}{0.000000,0.000000,0.000000}%
\pgfsetstrokecolor{currentstroke}%
\pgfsetdash{}{0pt}%
\pgfsys@defobject{currentmarker}{\pgfqpoint{0.000000in}{-0.048611in}}{\pgfqpoint{0.000000in}{0.000000in}}{%
\pgfpathmoveto{\pgfqpoint{0.000000in}{0.000000in}}%
\pgfpathlineto{\pgfqpoint{0.000000in}{-0.048611in}}%
\pgfusepath{stroke,fill}%
}%
\begin{pgfscope}%
\pgfsys@transformshift{1.780220in}{0.566125in}%
\pgfsys@useobject{currentmarker}{}%
\end{pgfscope}%
\end{pgfscope}%
\begin{pgfscope}%
\definecolor{textcolor}{rgb}{0.000000,0.000000,0.000000}%
\pgfsetstrokecolor{textcolor}%
\pgfsetfillcolor{textcolor}%
\pgftext[x=1.780220in,y=0.468902in,,top]{\color{textcolor}\rmfamily\fontsize{10.000000}{12.000000}\selectfont \(\displaystyle 20\)}%
\end{pgfscope}%
\begin{pgfscope}%
\pgfsetbuttcap%
\pgfsetroundjoin%
\definecolor{currentfill}{rgb}{0.000000,0.000000,0.000000}%
\pgfsetfillcolor{currentfill}%
\pgfsetlinewidth{0.803000pt}%
\definecolor{currentstroke}{rgb}{0.000000,0.000000,0.000000}%
\pgfsetstrokecolor{currentstroke}%
\pgfsetdash{}{0pt}%
\pgfsys@defobject{currentmarker}{\pgfqpoint{0.000000in}{-0.048611in}}{\pgfqpoint{0.000000in}{0.000000in}}{%
\pgfpathmoveto{\pgfqpoint{0.000000in}{0.000000in}}%
\pgfpathlineto{\pgfqpoint{0.000000in}{-0.048611in}}%
\pgfusepath{stroke,fill}%
}%
\begin{pgfscope}%
\pgfsys@transformshift{2.427862in}{0.566125in}%
\pgfsys@useobject{currentmarker}{}%
\end{pgfscope}%
\end{pgfscope}%
\begin{pgfscope}%
\definecolor{textcolor}{rgb}{0.000000,0.000000,0.000000}%
\pgfsetstrokecolor{textcolor}%
\pgfsetfillcolor{textcolor}%
\pgftext[x=2.427862in,y=0.468902in,,top]{\color{textcolor}\rmfamily\fontsize{10.000000}{12.000000}\selectfont \(\displaystyle 40\)}%
\end{pgfscope}%
\begin{pgfscope}%
\pgfsetbuttcap%
\pgfsetroundjoin%
\definecolor{currentfill}{rgb}{0.000000,0.000000,0.000000}%
\pgfsetfillcolor{currentfill}%
\pgfsetlinewidth{0.803000pt}%
\definecolor{currentstroke}{rgb}{0.000000,0.000000,0.000000}%
\pgfsetstrokecolor{currentstroke}%
\pgfsetdash{}{0pt}%
\pgfsys@defobject{currentmarker}{\pgfqpoint{0.000000in}{-0.048611in}}{\pgfqpoint{0.000000in}{0.000000in}}{%
\pgfpathmoveto{\pgfqpoint{0.000000in}{0.000000in}}%
\pgfpathlineto{\pgfqpoint{0.000000in}{-0.048611in}}%
\pgfusepath{stroke,fill}%
}%
\begin{pgfscope}%
\pgfsys@transformshift{3.075505in}{0.566125in}%
\pgfsys@useobject{currentmarker}{}%
\end{pgfscope}%
\end{pgfscope}%
\begin{pgfscope}%
\definecolor{textcolor}{rgb}{0.000000,0.000000,0.000000}%
\pgfsetstrokecolor{textcolor}%
\pgfsetfillcolor{textcolor}%
\pgftext[x=3.075505in,y=0.468902in,,top]{\color{textcolor}\rmfamily\fontsize{10.000000}{12.000000}\selectfont \(\displaystyle 60\)}%
\end{pgfscope}%
\begin{pgfscope}%
\pgfsetbuttcap%
\pgfsetroundjoin%
\definecolor{currentfill}{rgb}{0.000000,0.000000,0.000000}%
\pgfsetfillcolor{currentfill}%
\pgfsetlinewidth{0.803000pt}%
\definecolor{currentstroke}{rgb}{0.000000,0.000000,0.000000}%
\pgfsetstrokecolor{currentstroke}%
\pgfsetdash{}{0pt}%
\pgfsys@defobject{currentmarker}{\pgfqpoint{0.000000in}{-0.048611in}}{\pgfqpoint{0.000000in}{0.000000in}}{%
\pgfpathmoveto{\pgfqpoint{0.000000in}{0.000000in}}%
\pgfpathlineto{\pgfqpoint{0.000000in}{-0.048611in}}%
\pgfusepath{stroke,fill}%
}%
\begin{pgfscope}%
\pgfsys@transformshift{3.723148in}{0.566125in}%
\pgfsys@useobject{currentmarker}{}%
\end{pgfscope}%
\end{pgfscope}%
\begin{pgfscope}%
\definecolor{textcolor}{rgb}{0.000000,0.000000,0.000000}%
\pgfsetstrokecolor{textcolor}%
\pgfsetfillcolor{textcolor}%
\pgftext[x=3.723148in,y=0.468902in,,top]{\color{textcolor}\rmfamily\fontsize{10.000000}{12.000000}\selectfont \(\displaystyle 80\)}%
\end{pgfscope}%
\begin{pgfscope}%
\pgfsetbuttcap%
\pgfsetroundjoin%
\definecolor{currentfill}{rgb}{0.000000,0.000000,0.000000}%
\pgfsetfillcolor{currentfill}%
\pgfsetlinewidth{0.803000pt}%
\definecolor{currentstroke}{rgb}{0.000000,0.000000,0.000000}%
\pgfsetstrokecolor{currentstroke}%
\pgfsetdash{}{0pt}%
\pgfsys@defobject{currentmarker}{\pgfqpoint{0.000000in}{-0.048611in}}{\pgfqpoint{0.000000in}{0.000000in}}{%
\pgfpathmoveto{\pgfqpoint{0.000000in}{0.000000in}}%
\pgfpathlineto{\pgfqpoint{0.000000in}{-0.048611in}}%
\pgfusepath{stroke,fill}%
}%
\begin{pgfscope}%
\pgfsys@transformshift{4.370790in}{0.566125in}%
\pgfsys@useobject{currentmarker}{}%
\end{pgfscope}%
\end{pgfscope}%
\begin{pgfscope}%
\definecolor{textcolor}{rgb}{0.000000,0.000000,0.000000}%
\pgfsetstrokecolor{textcolor}%
\pgfsetfillcolor{textcolor}%
\pgftext[x=4.370790in,y=0.468902in,,top]{\color{textcolor}\rmfamily\fontsize{10.000000}{12.000000}\selectfont \(\displaystyle 100\)}%
\end{pgfscope}%
\begin{pgfscope}%
\pgfsetbuttcap%
\pgfsetroundjoin%
\definecolor{currentfill}{rgb}{0.000000,0.000000,0.000000}%
\pgfsetfillcolor{currentfill}%
\pgfsetlinewidth{0.803000pt}%
\definecolor{currentstroke}{rgb}{0.000000,0.000000,0.000000}%
\pgfsetstrokecolor{currentstroke}%
\pgfsetdash{}{0pt}%
\pgfsys@defobject{currentmarker}{\pgfqpoint{0.000000in}{-0.048611in}}{\pgfqpoint{0.000000in}{0.000000in}}{%
\pgfpathmoveto{\pgfqpoint{0.000000in}{0.000000in}}%
\pgfpathlineto{\pgfqpoint{0.000000in}{-0.048611in}}%
\pgfusepath{stroke,fill}%
}%
\begin{pgfscope}%
\pgfsys@transformshift{5.018433in}{0.566125in}%
\pgfsys@useobject{currentmarker}{}%
\end{pgfscope}%
\end{pgfscope}%
\begin{pgfscope}%
\definecolor{textcolor}{rgb}{0.000000,0.000000,0.000000}%
\pgfsetstrokecolor{textcolor}%
\pgfsetfillcolor{textcolor}%
\pgftext[x=5.018433in,y=0.468902in,,top]{\color{textcolor}\rmfamily\fontsize{10.000000}{12.000000}\selectfont \(\displaystyle 120\)}%
\end{pgfscope}%
\begin{pgfscope}%
\pgfsetbuttcap%
\pgfsetroundjoin%
\definecolor{currentfill}{rgb}{0.000000,0.000000,0.000000}%
\pgfsetfillcolor{currentfill}%
\pgfsetlinewidth{0.803000pt}%
\definecolor{currentstroke}{rgb}{0.000000,0.000000,0.000000}%
\pgfsetstrokecolor{currentstroke}%
\pgfsetdash{}{0pt}%
\pgfsys@defobject{currentmarker}{\pgfqpoint{0.000000in}{-0.048611in}}{\pgfqpoint{0.000000in}{0.000000in}}{%
\pgfpathmoveto{\pgfqpoint{0.000000in}{0.000000in}}%
\pgfpathlineto{\pgfqpoint{0.000000in}{-0.048611in}}%
\pgfusepath{stroke,fill}%
}%
\begin{pgfscope}%
\pgfsys@transformshift{5.666076in}{0.566125in}%
\pgfsys@useobject{currentmarker}{}%
\end{pgfscope}%
\end{pgfscope}%
\begin{pgfscope}%
\definecolor{textcolor}{rgb}{0.000000,0.000000,0.000000}%
\pgfsetstrokecolor{textcolor}%
\pgfsetfillcolor{textcolor}%
\pgftext[x=5.666076in,y=0.468902in,,top]{\color{textcolor}\rmfamily\fontsize{10.000000}{12.000000}\selectfont \(\displaystyle 140\)}%
\end{pgfscope}%
\begin{pgfscope}%
\pgfsetbuttcap%
\pgfsetroundjoin%
\definecolor{currentfill}{rgb}{0.000000,0.000000,0.000000}%
\pgfsetfillcolor{currentfill}%
\pgfsetlinewidth{0.803000pt}%
\definecolor{currentstroke}{rgb}{0.000000,0.000000,0.000000}%
\pgfsetstrokecolor{currentstroke}%
\pgfsetdash{}{0pt}%
\pgfsys@defobject{currentmarker}{\pgfqpoint{0.000000in}{-0.048611in}}{\pgfqpoint{0.000000in}{0.000000in}}{%
\pgfpathmoveto{\pgfqpoint{0.000000in}{0.000000in}}%
\pgfpathlineto{\pgfqpoint{0.000000in}{-0.048611in}}%
\pgfusepath{stroke,fill}%
}%
\begin{pgfscope}%
\pgfsys@transformshift{6.313718in}{0.566125in}%
\pgfsys@useobject{currentmarker}{}%
\end{pgfscope}%
\end{pgfscope}%
\begin{pgfscope}%
\definecolor{textcolor}{rgb}{0.000000,0.000000,0.000000}%
\pgfsetstrokecolor{textcolor}%
\pgfsetfillcolor{textcolor}%
\pgftext[x=6.313718in,y=0.468902in,,top]{\color{textcolor}\rmfamily\fontsize{10.000000}{12.000000}\selectfont \(\displaystyle 160\)}%
\end{pgfscope}%
\begin{pgfscope}%
\pgfsetbuttcap%
\pgfsetroundjoin%
\definecolor{currentfill}{rgb}{0.000000,0.000000,0.000000}%
\pgfsetfillcolor{currentfill}%
\pgfsetlinewidth{0.803000pt}%
\definecolor{currentstroke}{rgb}{0.000000,0.000000,0.000000}%
\pgfsetstrokecolor{currentstroke}%
\pgfsetdash{}{0pt}%
\pgfsys@defobject{currentmarker}{\pgfqpoint{-0.048611in}{0.000000in}}{\pgfqpoint{0.000000in}{0.000000in}}{%
\pgfpathmoveto{\pgfqpoint{0.000000in}{0.000000in}}%
\pgfpathlineto{\pgfqpoint{-0.048611in}{0.000000in}}%
\pgfusepath{stroke,fill}%
}%
\begin{pgfscope}%
\pgfsys@transformshift{0.970666in}{0.688190in}%
\pgfsys@useobject{currentmarker}{}%
\end{pgfscope}%
\end{pgfscope}%
\begin{pgfscope}%
\definecolor{textcolor}{rgb}{0.000000,0.000000,0.000000}%
\pgfsetstrokecolor{textcolor}%
\pgfsetfillcolor{textcolor}%
\pgftext[x=0.804000in, y=0.635428in, left, base]{\color{textcolor}\rmfamily\fontsize{10.000000}{12.000000}\selectfont \(\displaystyle 0\)}%
\end{pgfscope}%
\begin{pgfscope}%
\pgfsetbuttcap%
\pgfsetroundjoin%
\definecolor{currentfill}{rgb}{0.000000,0.000000,0.000000}%
\pgfsetfillcolor{currentfill}%
\pgfsetlinewidth{0.803000pt}%
\definecolor{currentstroke}{rgb}{0.000000,0.000000,0.000000}%
\pgfsetstrokecolor{currentstroke}%
\pgfsetdash{}{0pt}%
\pgfsys@defobject{currentmarker}{\pgfqpoint{-0.048611in}{0.000000in}}{\pgfqpoint{0.000000in}{0.000000in}}{%
\pgfpathmoveto{\pgfqpoint{0.000000in}{0.000000in}}%
\pgfpathlineto{\pgfqpoint{-0.048611in}{0.000000in}}%
\pgfusepath{stroke,fill}%
}%
\begin{pgfscope}%
\pgfsys@transformshift{0.970666in}{1.001177in}%
\pgfsys@useobject{currentmarker}{}%
\end{pgfscope}%
\end{pgfscope}%
\begin{pgfscope}%
\definecolor{textcolor}{rgb}{0.000000,0.000000,0.000000}%
\pgfsetstrokecolor{textcolor}%
\pgfsetfillcolor{textcolor}%
\pgftext[x=0.734555in, y=0.948416in, left, base]{\color{textcolor}\rmfamily\fontsize{10.000000}{12.000000}\selectfont \(\displaystyle 20\)}%
\end{pgfscope}%
\begin{pgfscope}%
\pgfsetbuttcap%
\pgfsetroundjoin%
\definecolor{currentfill}{rgb}{0.000000,0.000000,0.000000}%
\pgfsetfillcolor{currentfill}%
\pgfsetlinewidth{0.803000pt}%
\definecolor{currentstroke}{rgb}{0.000000,0.000000,0.000000}%
\pgfsetstrokecolor{currentstroke}%
\pgfsetdash{}{0pt}%
\pgfsys@defobject{currentmarker}{\pgfqpoint{-0.048611in}{0.000000in}}{\pgfqpoint{0.000000in}{0.000000in}}{%
\pgfpathmoveto{\pgfqpoint{0.000000in}{0.000000in}}%
\pgfpathlineto{\pgfqpoint{-0.048611in}{0.000000in}}%
\pgfusepath{stroke,fill}%
}%
\begin{pgfscope}%
\pgfsys@transformshift{0.970666in}{1.314165in}%
\pgfsys@useobject{currentmarker}{}%
\end{pgfscope}%
\end{pgfscope}%
\begin{pgfscope}%
\definecolor{textcolor}{rgb}{0.000000,0.000000,0.000000}%
\pgfsetstrokecolor{textcolor}%
\pgfsetfillcolor{textcolor}%
\pgftext[x=0.734555in, y=1.261403in, left, base]{\color{textcolor}\rmfamily\fontsize{10.000000}{12.000000}\selectfont \(\displaystyle 40\)}%
\end{pgfscope}%
\begin{pgfscope}%
\pgfsetbuttcap%
\pgfsetroundjoin%
\definecolor{currentfill}{rgb}{0.000000,0.000000,0.000000}%
\pgfsetfillcolor{currentfill}%
\pgfsetlinewidth{0.803000pt}%
\definecolor{currentstroke}{rgb}{0.000000,0.000000,0.000000}%
\pgfsetstrokecolor{currentstroke}%
\pgfsetdash{}{0pt}%
\pgfsys@defobject{currentmarker}{\pgfqpoint{-0.048611in}{0.000000in}}{\pgfqpoint{0.000000in}{0.000000in}}{%
\pgfpathmoveto{\pgfqpoint{0.000000in}{0.000000in}}%
\pgfpathlineto{\pgfqpoint{-0.048611in}{0.000000in}}%
\pgfusepath{stroke,fill}%
}%
\begin{pgfscope}%
\pgfsys@transformshift{0.970666in}{1.627152in}%
\pgfsys@useobject{currentmarker}{}%
\end{pgfscope}%
\end{pgfscope}%
\begin{pgfscope}%
\definecolor{textcolor}{rgb}{0.000000,0.000000,0.000000}%
\pgfsetstrokecolor{textcolor}%
\pgfsetfillcolor{textcolor}%
\pgftext[x=0.734555in, y=1.574390in, left, base]{\color{textcolor}\rmfamily\fontsize{10.000000}{12.000000}\selectfont \(\displaystyle 60\)}%
\end{pgfscope}%
\begin{pgfscope}%
\pgfsetbuttcap%
\pgfsetroundjoin%
\definecolor{currentfill}{rgb}{0.000000,0.000000,0.000000}%
\pgfsetfillcolor{currentfill}%
\pgfsetlinewidth{0.803000pt}%
\definecolor{currentstroke}{rgb}{0.000000,0.000000,0.000000}%
\pgfsetstrokecolor{currentstroke}%
\pgfsetdash{}{0pt}%
\pgfsys@defobject{currentmarker}{\pgfqpoint{-0.048611in}{0.000000in}}{\pgfqpoint{0.000000in}{0.000000in}}{%
\pgfpathmoveto{\pgfqpoint{0.000000in}{0.000000in}}%
\pgfpathlineto{\pgfqpoint{-0.048611in}{0.000000in}}%
\pgfusepath{stroke,fill}%
}%
\begin{pgfscope}%
\pgfsys@transformshift{0.970666in}{1.940139in}%
\pgfsys@useobject{currentmarker}{}%
\end{pgfscope}%
\end{pgfscope}%
\begin{pgfscope}%
\definecolor{textcolor}{rgb}{0.000000,0.000000,0.000000}%
\pgfsetstrokecolor{textcolor}%
\pgfsetfillcolor{textcolor}%
\pgftext[x=0.734555in, y=1.887378in, left, base]{\color{textcolor}\rmfamily\fontsize{10.000000}{12.000000}\selectfont \(\displaystyle 80\)}%
\end{pgfscope}%
\begin{pgfscope}%
\pgfsetbuttcap%
\pgfsetroundjoin%
\definecolor{currentfill}{rgb}{0.000000,0.000000,0.000000}%
\pgfsetfillcolor{currentfill}%
\pgfsetlinewidth{0.803000pt}%
\definecolor{currentstroke}{rgb}{0.000000,0.000000,0.000000}%
\pgfsetstrokecolor{currentstroke}%
\pgfsetdash{}{0pt}%
\pgfsys@defobject{currentmarker}{\pgfqpoint{-0.048611in}{0.000000in}}{\pgfqpoint{0.000000in}{0.000000in}}{%
\pgfpathmoveto{\pgfqpoint{0.000000in}{0.000000in}}%
\pgfpathlineto{\pgfqpoint{-0.048611in}{0.000000in}}%
\pgfusepath{stroke,fill}%
}%
\begin{pgfscope}%
\pgfsys@transformshift{0.970666in}{2.253127in}%
\pgfsys@useobject{currentmarker}{}%
\end{pgfscope}%
\end{pgfscope}%
\begin{pgfscope}%
\definecolor{textcolor}{rgb}{0.000000,0.000000,0.000000}%
\pgfsetstrokecolor{textcolor}%
\pgfsetfillcolor{textcolor}%
\pgftext[x=0.665110in, y=2.200365in, left, base]{\color{textcolor}\rmfamily\fontsize{10.000000}{12.000000}\selectfont \(\displaystyle 100\)}%
\end{pgfscope}%
\begin{pgfscope}%
\pgfsetbuttcap%
\pgfsetroundjoin%
\definecolor{currentfill}{rgb}{0.000000,0.000000,0.000000}%
\pgfsetfillcolor{currentfill}%
\pgfsetlinewidth{0.803000pt}%
\definecolor{currentstroke}{rgb}{0.000000,0.000000,0.000000}%
\pgfsetstrokecolor{currentstroke}%
\pgfsetdash{}{0pt}%
\pgfsys@defobject{currentmarker}{\pgfqpoint{-0.048611in}{0.000000in}}{\pgfqpoint{0.000000in}{0.000000in}}{%
\pgfpathmoveto{\pgfqpoint{0.000000in}{0.000000in}}%
\pgfpathlineto{\pgfqpoint{-0.048611in}{0.000000in}}%
\pgfusepath{stroke,fill}%
}%
\begin{pgfscope}%
\pgfsys@transformshift{0.970666in}{2.566114in}%
\pgfsys@useobject{currentmarker}{}%
\end{pgfscope}%
\end{pgfscope}%
\begin{pgfscope}%
\definecolor{textcolor}{rgb}{0.000000,0.000000,0.000000}%
\pgfsetstrokecolor{textcolor}%
\pgfsetfillcolor{textcolor}%
\pgftext[x=0.665110in, y=2.513353in, left, base]{\color{textcolor}\rmfamily\fontsize{10.000000}{12.000000}\selectfont \(\displaystyle 120\)}%
\end{pgfscope}%
\begin{pgfscope}%
\pgfsetbuttcap%
\pgfsetroundjoin%
\definecolor{currentfill}{rgb}{0.000000,0.000000,0.000000}%
\pgfsetfillcolor{currentfill}%
\pgfsetlinewidth{0.803000pt}%
\definecolor{currentstroke}{rgb}{0.000000,0.000000,0.000000}%
\pgfsetstrokecolor{currentstroke}%
\pgfsetdash{}{0pt}%
\pgfsys@defobject{currentmarker}{\pgfqpoint{-0.048611in}{0.000000in}}{\pgfqpoint{0.000000in}{0.000000in}}{%
\pgfpathmoveto{\pgfqpoint{0.000000in}{0.000000in}}%
\pgfpathlineto{\pgfqpoint{-0.048611in}{0.000000in}}%
\pgfusepath{stroke,fill}%
}%
\begin{pgfscope}%
\pgfsys@transformshift{0.970666in}{2.879102in}%
\pgfsys@useobject{currentmarker}{}%
\end{pgfscope}%
\end{pgfscope}%
\begin{pgfscope}%
\definecolor{textcolor}{rgb}{0.000000,0.000000,0.000000}%
\pgfsetstrokecolor{textcolor}%
\pgfsetfillcolor{textcolor}%
\pgftext[x=0.665110in, y=2.826340in, left, base]{\color{textcolor}\rmfamily\fontsize{10.000000}{12.000000}\selectfont \(\displaystyle 140\)}%
\end{pgfscope}%
\begin{pgfscope}%
\pgfsetbuttcap%
\pgfsetroundjoin%
\definecolor{currentfill}{rgb}{0.000000,0.000000,0.000000}%
\pgfsetfillcolor{currentfill}%
\pgfsetlinewidth{0.803000pt}%
\definecolor{currentstroke}{rgb}{0.000000,0.000000,0.000000}%
\pgfsetstrokecolor{currentstroke}%
\pgfsetdash{}{0pt}%
\pgfsys@defobject{currentmarker}{\pgfqpoint{-0.048611in}{0.000000in}}{\pgfqpoint{0.000000in}{0.000000in}}{%
\pgfpathmoveto{\pgfqpoint{0.000000in}{0.000000in}}%
\pgfpathlineto{\pgfqpoint{-0.048611in}{0.000000in}}%
\pgfusepath{stroke,fill}%
}%
\begin{pgfscope}%
\pgfsys@transformshift{0.970666in}{3.192089in}%
\pgfsys@useobject{currentmarker}{}%
\end{pgfscope}%
\end{pgfscope}%
\begin{pgfscope}%
\definecolor{textcolor}{rgb}{0.000000,0.000000,0.000000}%
\pgfsetstrokecolor{textcolor}%
\pgfsetfillcolor{textcolor}%
\pgftext[x=0.665110in, y=3.139328in, left, base]{\color{textcolor}\rmfamily\fontsize{10.000000}{12.000000}\selectfont \(\displaystyle 160\)}%
\end{pgfscope}%
\begin{pgfscope}%
\pgfsetrectcap%
\pgfsetmiterjoin%
\pgfsetlinewidth{0.803000pt}%
\definecolor{currentstroke}{rgb}{0.000000,0.000000,0.000000}%
\pgfsetstrokecolor{currentstroke}%
\pgfsetdash{}{0pt}%
\pgfpathmoveto{\pgfqpoint{0.970666in}{0.566125in}}%
\pgfpathlineto{\pgfqpoint{0.970666in}{3.251557in}}%
\pgfusepath{stroke}%
\end{pgfscope}%
\begin{pgfscope}%
\pgfsetrectcap%
\pgfsetmiterjoin%
\pgfsetlinewidth{0.803000pt}%
\definecolor{currentstroke}{rgb}{0.000000,0.000000,0.000000}%
\pgfsetstrokecolor{currentstroke}%
\pgfsetdash{}{0pt}%
\pgfpathmoveto{\pgfqpoint{6.669922in}{0.566125in}}%
\pgfpathlineto{\pgfqpoint{6.669922in}{3.251557in}}%
\pgfusepath{stroke}%
\end{pgfscope}%
\begin{pgfscope}%
\pgfsetrectcap%
\pgfsetmiterjoin%
\pgfsetlinewidth{0.803000pt}%
\definecolor{currentstroke}{rgb}{0.000000,0.000000,0.000000}%
\pgfsetstrokecolor{currentstroke}%
\pgfsetdash{}{0pt}%
\pgfpathmoveto{\pgfqpoint{0.970666in}{0.566125in}}%
\pgfpathlineto{\pgfqpoint{6.669922in}{0.566125in}}%
\pgfusepath{stroke}%
\end{pgfscope}%
\begin{pgfscope}%
\pgfsetrectcap%
\pgfsetmiterjoin%
\pgfsetlinewidth{0.803000pt}%
\definecolor{currentstroke}{rgb}{0.000000,0.000000,0.000000}%
\pgfsetstrokecolor{currentstroke}%
\pgfsetdash{}{0pt}%
\pgfpathmoveto{\pgfqpoint{0.970666in}{3.251557in}}%
\pgfpathlineto{\pgfqpoint{6.669922in}{3.251557in}}%
\pgfusepath{stroke}%
\end{pgfscope}%
\begin{pgfscope}%
\definecolor{textcolor}{rgb}{0.000000,0.000000,0.000000}%
\pgfsetstrokecolor{textcolor}%
\pgfsetfillcolor{textcolor}%
\pgftext[x=3.820294in,y=3.334890in,,base]{\color{textcolor}\rmfamily\fontsize{12.000000}{14.400000}\selectfont greedy 1367.5}%
\end{pgfscope}%
\begin{pgfscope}%
\pgfsetbuttcap%
\pgfsetmiterjoin%
\definecolor{currentfill}{rgb}{1.000000,1.000000,1.000000}%
\pgfsetfillcolor{currentfill}%
\pgfsetlinewidth{0.000000pt}%
\definecolor{currentstroke}{rgb}{0.000000,0.000000,0.000000}%
\pgfsetstrokecolor{currentstroke}%
\pgfsetstrokeopacity{0.000000}%
\pgfsetdash{}{0pt}%
\pgfpathmoveto{\pgfqpoint{7.640588in}{0.566125in}}%
\pgfpathlineto{\pgfqpoint{13.339844in}{0.566125in}}%
\pgfpathlineto{\pgfqpoint{13.339844in}{3.251557in}}%
\pgfpathlineto{\pgfqpoint{7.640588in}{3.251557in}}%
\pgfpathclose%
\pgfusepath{fill}%
\end{pgfscope}%
\begin{pgfscope}%
\pgfpathrectangle{\pgfqpoint{7.640588in}{0.566125in}}{\pgfqpoint{5.699255in}{2.685432in}}%
\pgfusepath{clip}%
\pgfsetrectcap%
\pgfsetroundjoin%
\pgfsetlinewidth{1.505625pt}%
\definecolor{currentstroke}{rgb}{0.000000,0.000000,0.000000}%
\pgfsetstrokecolor{currentstroke}%
\pgfsetdash{}{0pt}%
\pgfpathmoveto{\pgfqpoint{12.400762in}{2.221828in}}%
\pgfpathlineto{\pgfqpoint{12.595055in}{2.409621in}}%
\pgfpathlineto{\pgfqpoint{13.080787in}{2.785205in}}%
\pgfpathlineto{\pgfqpoint{12.789347in}{3.129492in}}%
\pgfpathlineto{\pgfqpoint{12.465526in}{3.098193in}}%
\pgfpathlineto{\pgfqpoint{12.206469in}{3.082544in}}%
\pgfpathlineto{\pgfqpoint{12.433144in}{3.066894in}}%
\pgfpathlineto{\pgfqpoint{12.821730in}{2.879102in}}%
\pgfpathlineto{\pgfqpoint{12.271233in}{2.362672in}}%
\pgfpathlineto{\pgfqpoint{12.044558in}{2.487867in}}%
\pgfpathlineto{\pgfqpoint{12.141705in}{2.753907in}}%
\pgfpathlineto{\pgfqpoint{11.785501in}{2.753907in}}%
\pgfpathlineto{\pgfqpoint{11.558826in}{2.879102in}}%
\pgfpathlineto{\pgfqpoint{11.494062in}{2.832154in}}%
\pgfpathlineto{\pgfqpoint{11.332152in}{2.863452in}}%
\pgfpathlineto{\pgfqpoint{11.008330in}{2.894751in}}%
\pgfpathlineto{\pgfqpoint{10.814037in}{2.863452in}}%
\pgfpathlineto{\pgfqpoint{10.198777in}{2.769556in}}%
\pgfpathlineto{\pgfqpoint{9.842573in}{2.722608in}}%
\pgfpathlineto{\pgfqpoint{9.615898in}{2.597413in}}%
\pgfpathlineto{\pgfqpoint{9.356841in}{2.487867in}}%
\pgfpathlineto{\pgfqpoint{7.932028in}{1.595853in}}%
\pgfpathlineto{\pgfqpoint{8.126320in}{2.362672in}}%
\pgfpathlineto{\pgfqpoint{8.288231in}{2.393971in}}%
\pgfpathlineto{\pgfqpoint{9.033020in}{2.159231in}}%
\pgfpathlineto{\pgfqpoint{8.871109in}{2.096633in}}%
\pgfpathlineto{\pgfqpoint{8.871109in}{1.940139in}}%
\pgfpathlineto{\pgfqpoint{9.227313in}{1.924490in}}%
\pgfpathlineto{\pgfqpoint{9.324459in}{1.783646in}}%
\pgfpathlineto{\pgfqpoint{8.158702in}{1.157671in}}%
\pgfpathlineto{\pgfqpoint{7.899645in}{1.079424in}}%
\pgfpathlineto{\pgfqpoint{8.126320in}{0.782086in}}%
\pgfpathlineto{\pgfqpoint{8.320613in}{0.938580in}}%
\pgfpathlineto{\pgfqpoint{8.417760in}{1.063775in}}%
\pgfpathlineto{\pgfqpoint{8.547288in}{1.095073in}}%
\pgfpathlineto{\pgfqpoint{8.806345in}{1.126372in}}%
\pgfpathlineto{\pgfqpoint{8.676817in}{1.001177in}}%
\pgfpathlineto{\pgfqpoint{8.903492in}{0.954229in}}%
\pgfpathlineto{\pgfqpoint{9.033020in}{1.016827in}}%
\pgfpathlineto{\pgfqpoint{9.194931in}{0.969878in}}%
\pgfpathlineto{\pgfqpoint{9.356841in}{0.875982in}}%
\pgfpathlineto{\pgfqpoint{9.615898in}{0.750787in}}%
\pgfpathlineto{\pgfqpoint{9.810191in}{0.954229in}}%
\pgfpathlineto{\pgfqpoint{9.907338in}{1.173320in}}%
\pgfpathlineto{\pgfqpoint{10.263541in}{1.220268in}}%
\pgfpathlineto{\pgfqpoint{10.166395in}{1.642801in}}%
\pgfpathlineto{\pgfqpoint{10.198777in}{1.736698in}}%
\pgfpathlineto{\pgfqpoint{9.615898in}{1.877542in}}%
\pgfpathlineto{\pgfqpoint{9.810191in}{1.877542in}}%
\pgfpathlineto{\pgfqpoint{10.134013in}{1.908841in}}%
\pgfpathlineto{\pgfqpoint{10.101630in}{2.034036in}}%
\pgfpathlineto{\pgfqpoint{10.101630in}{2.096633in}}%
\pgfpathlineto{\pgfqpoint{10.004484in}{2.300075in}}%
\pgfpathlineto{\pgfqpoint{8.935874in}{2.628712in}}%
\pgfpathlineto{\pgfqpoint{8.871109in}{2.941699in}}%
\pgfpathlineto{\pgfqpoint{8.871109in}{2.988647in}}%
\pgfpathlineto{\pgfqpoint{8.773963in}{2.972998in}}%
\pgfpathlineto{\pgfqpoint{8.773963in}{3.113842in}}%
\pgfpathlineto{\pgfqpoint{9.162549in}{3.098193in}}%
\pgfpathlineto{\pgfqpoint{10.295923in}{2.409621in}}%
\pgfpathlineto{\pgfqpoint{10.814037in}{2.456569in}}%
\pgfpathlineto{\pgfqpoint{11.105477in}{2.190529in}}%
\pgfpathlineto{\pgfqpoint{11.170241in}{2.174880in}}%
\pgfpathlineto{\pgfqpoint{11.429298in}{1.830594in}}%
\pgfpathlineto{\pgfqpoint{10.878802in}{1.799295in}}%
\pgfpathlineto{\pgfqpoint{10.716891in}{1.361113in}}%
\pgfpathlineto{\pgfqpoint{11.429298in}{1.220268in}}%
\pgfpathlineto{\pgfqpoint{11.558826in}{1.157671in}}%
\pgfpathlineto{\pgfqpoint{11.235005in}{0.891632in}}%
\pgfpathlineto{\pgfqpoint{10.652127in}{0.829034in}}%
\pgfpathlineto{\pgfqpoint{10.619745in}{0.797735in}}%
\pgfpathlineto{\pgfqpoint{11.461680in}{0.688190in}}%
\pgfpathlineto{\pgfqpoint{11.979794in}{0.688190in}}%
\pgfpathlineto{\pgfqpoint{12.141705in}{1.157671in}}%
\pgfpathlineto{\pgfqpoint{11.947412in}{1.188970in}}%
\pgfpathlineto{\pgfqpoint{11.947412in}{1.267216in}}%
\pgfpathlineto{\pgfqpoint{12.044558in}{1.329814in}}%
\pgfpathlineto{\pgfqpoint{12.854112in}{1.282866in}}%
\pgfpathlineto{\pgfqpoint{12.465526in}{1.455009in}}%
\pgfpathlineto{\pgfqpoint{12.012176in}{1.517606in}}%
\pgfpathlineto{\pgfqpoint{11.817884in}{1.861893in}}%
\pgfpathlineto{\pgfqpoint{12.692201in}{1.893191in}}%
\pgfpathlineto{\pgfqpoint{12.400762in}{2.221828in}}%
\pgfusepath{stroke}%
\end{pgfscope}%
\begin{pgfscope}%
\pgfpathrectangle{\pgfqpoint{7.640588in}{0.566125in}}{\pgfqpoint{5.699255in}{2.685432in}}%
\pgfusepath{clip}%
\pgfsetbuttcap%
\pgfsetroundjoin%
\definecolor{currentfill}{rgb}{1.000000,0.000000,0.000000}%
\pgfsetfillcolor{currentfill}%
\pgfsetlinewidth{1.003750pt}%
\definecolor{currentstroke}{rgb}{1.000000,0.000000,0.000000}%
\pgfsetstrokecolor{currentstroke}%
\pgfsetdash{}{0pt}%
\pgfpathmoveto{\pgfqpoint{12.400762in}{2.180161in}}%
\pgfpathcurveto{\pgfqpoint{12.411812in}{2.180161in}}{\pgfqpoint{12.422411in}{2.184552in}}{\pgfqpoint{12.430225in}{2.192365in}}%
\pgfpathcurveto{\pgfqpoint{12.438038in}{2.200179in}}{\pgfqpoint{12.442429in}{2.210778in}}{\pgfqpoint{12.442429in}{2.221828in}}%
\pgfpathcurveto{\pgfqpoint{12.442429in}{2.232878in}}{\pgfqpoint{12.438038in}{2.243477in}}{\pgfqpoint{12.430225in}{2.251291in}}%
\pgfpathcurveto{\pgfqpoint{12.422411in}{2.259104in}}{\pgfqpoint{12.411812in}{2.263495in}}{\pgfqpoint{12.400762in}{2.263495in}}%
\pgfpathcurveto{\pgfqpoint{12.389712in}{2.263495in}}{\pgfqpoint{12.379113in}{2.259104in}}{\pgfqpoint{12.371299in}{2.251291in}}%
\pgfpathcurveto{\pgfqpoint{12.363485in}{2.243477in}}{\pgfqpoint{12.359095in}{2.232878in}}{\pgfqpoint{12.359095in}{2.221828in}}%
\pgfpathcurveto{\pgfqpoint{12.359095in}{2.210778in}}{\pgfqpoint{12.363485in}{2.200179in}}{\pgfqpoint{12.371299in}{2.192365in}}%
\pgfpathcurveto{\pgfqpoint{12.379113in}{2.184552in}}{\pgfqpoint{12.389712in}{2.180161in}}{\pgfqpoint{12.400762in}{2.180161in}}%
\pgfpathclose%
\pgfusepath{stroke,fill}%
\end{pgfscope}%
\begin{pgfscope}%
\pgfpathrectangle{\pgfqpoint{7.640588in}{0.566125in}}{\pgfqpoint{5.699255in}{2.685432in}}%
\pgfusepath{clip}%
\pgfsetbuttcap%
\pgfsetroundjoin%
\definecolor{currentfill}{rgb}{0.750000,0.750000,0.000000}%
\pgfsetfillcolor{currentfill}%
\pgfsetlinewidth{1.003750pt}%
\definecolor{currentstroke}{rgb}{0.750000,0.750000,0.000000}%
\pgfsetstrokecolor{currentstroke}%
\pgfsetdash{}{0pt}%
\pgfpathmoveto{\pgfqpoint{12.595055in}{2.367954in}}%
\pgfpathcurveto{\pgfqpoint{12.606105in}{2.367954in}}{\pgfqpoint{12.616704in}{2.372344in}}{\pgfqpoint{12.624517in}{2.380158in}}%
\pgfpathcurveto{\pgfqpoint{12.632331in}{2.387971in}}{\pgfqpoint{12.636721in}{2.398570in}}{\pgfqpoint{12.636721in}{2.409621in}}%
\pgfpathcurveto{\pgfqpoint{12.636721in}{2.420671in}}{\pgfqpoint{12.632331in}{2.431270in}}{\pgfqpoint{12.624517in}{2.439083in}}%
\pgfpathcurveto{\pgfqpoint{12.616704in}{2.446897in}}{\pgfqpoint{12.606105in}{2.451287in}}{\pgfqpoint{12.595055in}{2.451287in}}%
\pgfpathcurveto{\pgfqpoint{12.584005in}{2.451287in}}{\pgfqpoint{12.573406in}{2.446897in}}{\pgfqpoint{12.565592in}{2.439083in}}%
\pgfpathcurveto{\pgfqpoint{12.557778in}{2.431270in}}{\pgfqpoint{12.553388in}{2.420671in}}{\pgfqpoint{12.553388in}{2.409621in}}%
\pgfpathcurveto{\pgfqpoint{12.553388in}{2.398570in}}{\pgfqpoint{12.557778in}{2.387971in}}{\pgfqpoint{12.565592in}{2.380158in}}%
\pgfpathcurveto{\pgfqpoint{12.573406in}{2.372344in}}{\pgfqpoint{12.584005in}{2.367954in}}{\pgfqpoint{12.595055in}{2.367954in}}%
\pgfpathclose%
\pgfusepath{stroke,fill}%
\end{pgfscope}%
\begin{pgfscope}%
\pgfpathrectangle{\pgfqpoint{7.640588in}{0.566125in}}{\pgfqpoint{5.699255in}{2.685432in}}%
\pgfusepath{clip}%
\pgfsetbuttcap%
\pgfsetroundjoin%
\definecolor{currentfill}{rgb}{0.000000,0.000000,0.000000}%
\pgfsetfillcolor{currentfill}%
\pgfsetlinewidth{1.003750pt}%
\definecolor{currentstroke}{rgb}{0.000000,0.000000,0.000000}%
\pgfsetstrokecolor{currentstroke}%
\pgfsetdash{}{0pt}%
\pgfpathmoveto{\pgfqpoint{13.080787in}{2.743539in}}%
\pgfpathcurveto{\pgfqpoint{13.091837in}{2.743539in}}{\pgfqpoint{13.102436in}{2.747929in}}{\pgfqpoint{13.110249in}{2.755743in}}%
\pgfpathcurveto{\pgfqpoint{13.118063in}{2.763556in}}{\pgfqpoint{13.122453in}{2.774155in}}{\pgfqpoint{13.122453in}{2.785205in}}%
\pgfpathcurveto{\pgfqpoint{13.122453in}{2.796256in}}{\pgfqpoint{13.118063in}{2.806855in}}{\pgfqpoint{13.110249in}{2.814668in}}%
\pgfpathcurveto{\pgfqpoint{13.102436in}{2.822482in}}{\pgfqpoint{13.091837in}{2.826872in}}{\pgfqpoint{13.080787in}{2.826872in}}%
\pgfpathcurveto{\pgfqpoint{13.069737in}{2.826872in}}{\pgfqpoint{13.059138in}{2.822482in}}{\pgfqpoint{13.051324in}{2.814668in}}%
\pgfpathcurveto{\pgfqpoint{13.043510in}{2.806855in}}{\pgfqpoint{13.039120in}{2.796256in}}{\pgfqpoint{13.039120in}{2.785205in}}%
\pgfpathcurveto{\pgfqpoint{13.039120in}{2.774155in}}{\pgfqpoint{13.043510in}{2.763556in}}{\pgfqpoint{13.051324in}{2.755743in}}%
\pgfpathcurveto{\pgfqpoint{13.059138in}{2.747929in}}{\pgfqpoint{13.069737in}{2.743539in}}{\pgfqpoint{13.080787in}{2.743539in}}%
\pgfpathclose%
\pgfusepath{stroke,fill}%
\end{pgfscope}%
\begin{pgfscope}%
\pgfpathrectangle{\pgfqpoint{7.640588in}{0.566125in}}{\pgfqpoint{5.699255in}{2.685432in}}%
\pgfusepath{clip}%
\pgfsetbuttcap%
\pgfsetroundjoin%
\definecolor{currentfill}{rgb}{0.000000,0.000000,0.000000}%
\pgfsetfillcolor{currentfill}%
\pgfsetlinewidth{1.003750pt}%
\definecolor{currentstroke}{rgb}{0.000000,0.000000,0.000000}%
\pgfsetstrokecolor{currentstroke}%
\pgfsetdash{}{0pt}%
\pgfpathmoveto{\pgfqpoint{12.789347in}{3.087825in}}%
\pgfpathcurveto{\pgfqpoint{12.800398in}{3.087825in}}{\pgfqpoint{12.810997in}{3.092215in}}{\pgfqpoint{12.818810in}{3.100029in}}%
\pgfpathcurveto{\pgfqpoint{12.826624in}{3.107842in}}{\pgfqpoint{12.831014in}{3.118441in}}{\pgfqpoint{12.831014in}{3.129492in}}%
\pgfpathcurveto{\pgfqpoint{12.831014in}{3.140542in}}{\pgfqpoint{12.826624in}{3.151141in}}{\pgfqpoint{12.818810in}{3.158954in}}%
\pgfpathcurveto{\pgfqpoint{12.810997in}{3.166768in}}{\pgfqpoint{12.800398in}{3.171158in}}{\pgfqpoint{12.789347in}{3.171158in}}%
\pgfpathcurveto{\pgfqpoint{12.778297in}{3.171158in}}{\pgfqpoint{12.767698in}{3.166768in}}{\pgfqpoint{12.759885in}{3.158954in}}%
\pgfpathcurveto{\pgfqpoint{12.752071in}{3.151141in}}{\pgfqpoint{12.747681in}{3.140542in}}{\pgfqpoint{12.747681in}{3.129492in}}%
\pgfpathcurveto{\pgfqpoint{12.747681in}{3.118441in}}{\pgfqpoint{12.752071in}{3.107842in}}{\pgfqpoint{12.759885in}{3.100029in}}%
\pgfpathcurveto{\pgfqpoint{12.767698in}{3.092215in}}{\pgfqpoint{12.778297in}{3.087825in}}{\pgfqpoint{12.789347in}{3.087825in}}%
\pgfpathclose%
\pgfusepath{stroke,fill}%
\end{pgfscope}%
\begin{pgfscope}%
\pgfpathrectangle{\pgfqpoint{7.640588in}{0.566125in}}{\pgfqpoint{5.699255in}{2.685432in}}%
\pgfusepath{clip}%
\pgfsetbuttcap%
\pgfsetroundjoin%
\definecolor{currentfill}{rgb}{0.000000,0.000000,0.000000}%
\pgfsetfillcolor{currentfill}%
\pgfsetlinewidth{1.003750pt}%
\definecolor{currentstroke}{rgb}{0.000000,0.000000,0.000000}%
\pgfsetstrokecolor{currentstroke}%
\pgfsetdash{}{0pt}%
\pgfpathmoveto{\pgfqpoint{12.465526in}{3.056526in}}%
\pgfpathcurveto{\pgfqpoint{12.476576in}{3.056526in}}{\pgfqpoint{12.487175in}{3.060916in}}{\pgfqpoint{12.494989in}{3.068730in}}%
\pgfpathcurveto{\pgfqpoint{12.502803in}{3.076544in}}{\pgfqpoint{12.507193in}{3.087143in}}{\pgfqpoint{12.507193in}{3.098193in}}%
\pgfpathcurveto{\pgfqpoint{12.507193in}{3.109243in}}{\pgfqpoint{12.502803in}{3.119842in}}{\pgfqpoint{12.494989in}{3.127656in}}%
\pgfpathcurveto{\pgfqpoint{12.487175in}{3.135469in}}{\pgfqpoint{12.476576in}{3.139860in}}{\pgfqpoint{12.465526in}{3.139860in}}%
\pgfpathcurveto{\pgfqpoint{12.454476in}{3.139860in}}{\pgfqpoint{12.443877in}{3.135469in}}{\pgfqpoint{12.436063in}{3.127656in}}%
\pgfpathcurveto{\pgfqpoint{12.428250in}{3.119842in}}{\pgfqpoint{12.423859in}{3.109243in}}{\pgfqpoint{12.423859in}{3.098193in}}%
\pgfpathcurveto{\pgfqpoint{12.423859in}{3.087143in}}{\pgfqpoint{12.428250in}{3.076544in}}{\pgfqpoint{12.436063in}{3.068730in}}%
\pgfpathcurveto{\pgfqpoint{12.443877in}{3.060916in}}{\pgfqpoint{12.454476in}{3.056526in}}{\pgfqpoint{12.465526in}{3.056526in}}%
\pgfpathclose%
\pgfusepath{stroke,fill}%
\end{pgfscope}%
\begin{pgfscope}%
\pgfpathrectangle{\pgfqpoint{7.640588in}{0.566125in}}{\pgfqpoint{5.699255in}{2.685432in}}%
\pgfusepath{clip}%
\pgfsetbuttcap%
\pgfsetroundjoin%
\definecolor{currentfill}{rgb}{0.000000,0.000000,0.000000}%
\pgfsetfillcolor{currentfill}%
\pgfsetlinewidth{1.003750pt}%
\definecolor{currentstroke}{rgb}{0.000000,0.000000,0.000000}%
\pgfsetstrokecolor{currentstroke}%
\pgfsetdash{}{0pt}%
\pgfpathmoveto{\pgfqpoint{12.206469in}{3.040877in}}%
\pgfpathcurveto{\pgfqpoint{12.217519in}{3.040877in}}{\pgfqpoint{12.228118in}{3.045267in}}{\pgfqpoint{12.235932in}{3.053081in}}%
\pgfpathcurveto{\pgfqpoint{12.243746in}{3.060894in}}{\pgfqpoint{12.248136in}{3.071493in}}{\pgfqpoint{12.248136in}{3.082544in}}%
\pgfpathcurveto{\pgfqpoint{12.248136in}{3.093594in}}{\pgfqpoint{12.243746in}{3.104193in}}{\pgfqpoint{12.235932in}{3.112006in}}%
\pgfpathcurveto{\pgfqpoint{12.228118in}{3.119820in}}{\pgfqpoint{12.217519in}{3.124210in}}{\pgfqpoint{12.206469in}{3.124210in}}%
\pgfpathcurveto{\pgfqpoint{12.195419in}{3.124210in}}{\pgfqpoint{12.184820in}{3.119820in}}{\pgfqpoint{12.177006in}{3.112006in}}%
\pgfpathcurveto{\pgfqpoint{12.169193in}{3.104193in}}{\pgfqpoint{12.164802in}{3.093594in}}{\pgfqpoint{12.164802in}{3.082544in}}%
\pgfpathcurveto{\pgfqpoint{12.164802in}{3.071493in}}{\pgfqpoint{12.169193in}{3.060894in}}{\pgfqpoint{12.177006in}{3.053081in}}%
\pgfpathcurveto{\pgfqpoint{12.184820in}{3.045267in}}{\pgfqpoint{12.195419in}{3.040877in}}{\pgfqpoint{12.206469in}{3.040877in}}%
\pgfpathclose%
\pgfusepath{stroke,fill}%
\end{pgfscope}%
\begin{pgfscope}%
\pgfpathrectangle{\pgfqpoint{7.640588in}{0.566125in}}{\pgfqpoint{5.699255in}{2.685432in}}%
\pgfusepath{clip}%
\pgfsetbuttcap%
\pgfsetroundjoin%
\definecolor{currentfill}{rgb}{0.000000,0.000000,0.000000}%
\pgfsetfillcolor{currentfill}%
\pgfsetlinewidth{1.003750pt}%
\definecolor{currentstroke}{rgb}{0.000000,0.000000,0.000000}%
\pgfsetstrokecolor{currentstroke}%
\pgfsetdash{}{0pt}%
\pgfpathmoveto{\pgfqpoint{12.433144in}{3.025227in}}%
\pgfpathcurveto{\pgfqpoint{12.444194in}{3.025227in}}{\pgfqpoint{12.454793in}{3.029618in}}{\pgfqpoint{12.462607in}{3.037431in}}%
\pgfpathcurveto{\pgfqpoint{12.470420in}{3.045245in}}{\pgfqpoint{12.474811in}{3.055844in}}{\pgfqpoint{12.474811in}{3.066894in}}%
\pgfpathcurveto{\pgfqpoint{12.474811in}{3.077944in}}{\pgfqpoint{12.470420in}{3.088543in}}{\pgfqpoint{12.462607in}{3.096357in}}%
\pgfpathcurveto{\pgfqpoint{12.454793in}{3.104171in}}{\pgfqpoint{12.444194in}{3.108561in}}{\pgfqpoint{12.433144in}{3.108561in}}%
\pgfpathcurveto{\pgfqpoint{12.422094in}{3.108561in}}{\pgfqpoint{12.411495in}{3.104171in}}{\pgfqpoint{12.403681in}{3.096357in}}%
\pgfpathcurveto{\pgfqpoint{12.395868in}{3.088543in}}{\pgfqpoint{12.391477in}{3.077944in}}{\pgfqpoint{12.391477in}{3.066894in}}%
\pgfpathcurveto{\pgfqpoint{12.391477in}{3.055844in}}{\pgfqpoint{12.395868in}{3.045245in}}{\pgfqpoint{12.403681in}{3.037431in}}%
\pgfpathcurveto{\pgfqpoint{12.411495in}{3.029618in}}{\pgfqpoint{12.422094in}{3.025227in}}{\pgfqpoint{12.433144in}{3.025227in}}%
\pgfpathclose%
\pgfusepath{stroke,fill}%
\end{pgfscope}%
\begin{pgfscope}%
\pgfpathrectangle{\pgfqpoint{7.640588in}{0.566125in}}{\pgfqpoint{5.699255in}{2.685432in}}%
\pgfusepath{clip}%
\pgfsetbuttcap%
\pgfsetroundjoin%
\definecolor{currentfill}{rgb}{0.000000,0.000000,0.000000}%
\pgfsetfillcolor{currentfill}%
\pgfsetlinewidth{1.003750pt}%
\definecolor{currentstroke}{rgb}{0.000000,0.000000,0.000000}%
\pgfsetstrokecolor{currentstroke}%
\pgfsetdash{}{0pt}%
\pgfpathmoveto{\pgfqpoint{12.821730in}{2.837435in}}%
\pgfpathcurveto{\pgfqpoint{12.832780in}{2.837435in}}{\pgfqpoint{12.843379in}{2.841825in}}{\pgfqpoint{12.851192in}{2.849639in}}%
\pgfpathcurveto{\pgfqpoint{12.859006in}{2.857453in}}{\pgfqpoint{12.863396in}{2.868052in}}{\pgfqpoint{12.863396in}{2.879102in}}%
\pgfpathcurveto{\pgfqpoint{12.863396in}{2.890152in}}{\pgfqpoint{12.859006in}{2.900751in}}{\pgfqpoint{12.851192in}{2.908564in}}%
\pgfpathcurveto{\pgfqpoint{12.843379in}{2.916378in}}{\pgfqpoint{12.832780in}{2.920768in}}{\pgfqpoint{12.821730in}{2.920768in}}%
\pgfpathcurveto{\pgfqpoint{12.810679in}{2.920768in}}{\pgfqpoint{12.800080in}{2.916378in}}{\pgfqpoint{12.792267in}{2.908564in}}%
\pgfpathcurveto{\pgfqpoint{12.784453in}{2.900751in}}{\pgfqpoint{12.780063in}{2.890152in}}{\pgfqpoint{12.780063in}{2.879102in}}%
\pgfpathcurveto{\pgfqpoint{12.780063in}{2.868052in}}{\pgfqpoint{12.784453in}{2.857453in}}{\pgfqpoint{12.792267in}{2.849639in}}%
\pgfpathcurveto{\pgfqpoint{12.800080in}{2.841825in}}{\pgfqpoint{12.810679in}{2.837435in}}{\pgfqpoint{12.821730in}{2.837435in}}%
\pgfpathclose%
\pgfusepath{stroke,fill}%
\end{pgfscope}%
\begin{pgfscope}%
\pgfpathrectangle{\pgfqpoint{7.640588in}{0.566125in}}{\pgfqpoint{5.699255in}{2.685432in}}%
\pgfusepath{clip}%
\pgfsetbuttcap%
\pgfsetroundjoin%
\definecolor{currentfill}{rgb}{0.000000,0.000000,0.000000}%
\pgfsetfillcolor{currentfill}%
\pgfsetlinewidth{1.003750pt}%
\definecolor{currentstroke}{rgb}{0.000000,0.000000,0.000000}%
\pgfsetstrokecolor{currentstroke}%
\pgfsetdash{}{0pt}%
\pgfpathmoveto{\pgfqpoint{12.271233in}{2.321006in}}%
\pgfpathcurveto{\pgfqpoint{12.282283in}{2.321006in}}{\pgfqpoint{12.292883in}{2.325396in}}{\pgfqpoint{12.300696in}{2.333210in}}%
\pgfpathcurveto{\pgfqpoint{12.308510in}{2.341023in}}{\pgfqpoint{12.312900in}{2.351622in}}{\pgfqpoint{12.312900in}{2.362672in}}%
\pgfpathcurveto{\pgfqpoint{12.312900in}{2.373723in}}{\pgfqpoint{12.308510in}{2.384322in}}{\pgfqpoint{12.300696in}{2.392135in}}%
\pgfpathcurveto{\pgfqpoint{12.292883in}{2.399949in}}{\pgfqpoint{12.282283in}{2.404339in}}{\pgfqpoint{12.271233in}{2.404339in}}%
\pgfpathcurveto{\pgfqpoint{12.260183in}{2.404339in}}{\pgfqpoint{12.249584in}{2.399949in}}{\pgfqpoint{12.241771in}{2.392135in}}%
\pgfpathcurveto{\pgfqpoint{12.233957in}{2.384322in}}{\pgfqpoint{12.229567in}{2.373723in}}{\pgfqpoint{12.229567in}{2.362672in}}%
\pgfpathcurveto{\pgfqpoint{12.229567in}{2.351622in}}{\pgfqpoint{12.233957in}{2.341023in}}{\pgfqpoint{12.241771in}{2.333210in}}%
\pgfpathcurveto{\pgfqpoint{12.249584in}{2.325396in}}{\pgfqpoint{12.260183in}{2.321006in}}{\pgfqpoint{12.271233in}{2.321006in}}%
\pgfpathclose%
\pgfusepath{stroke,fill}%
\end{pgfscope}%
\begin{pgfscope}%
\pgfpathrectangle{\pgfqpoint{7.640588in}{0.566125in}}{\pgfqpoint{5.699255in}{2.685432in}}%
\pgfusepath{clip}%
\pgfsetbuttcap%
\pgfsetroundjoin%
\definecolor{currentfill}{rgb}{0.000000,0.000000,0.000000}%
\pgfsetfillcolor{currentfill}%
\pgfsetlinewidth{1.003750pt}%
\definecolor{currentstroke}{rgb}{0.000000,0.000000,0.000000}%
\pgfsetstrokecolor{currentstroke}%
\pgfsetdash{}{0pt}%
\pgfpathmoveto{\pgfqpoint{12.044558in}{2.446201in}}%
\pgfpathcurveto{\pgfqpoint{12.055609in}{2.446201in}}{\pgfqpoint{12.066208in}{2.450591in}}{\pgfqpoint{12.074021in}{2.458405in}}%
\pgfpathcurveto{\pgfqpoint{12.081835in}{2.466218in}}{\pgfqpoint{12.086225in}{2.476817in}}{\pgfqpoint{12.086225in}{2.487867in}}%
\pgfpathcurveto{\pgfqpoint{12.086225in}{2.498918in}}{\pgfqpoint{12.081835in}{2.509517in}}{\pgfqpoint{12.074021in}{2.517330in}}%
\pgfpathcurveto{\pgfqpoint{12.066208in}{2.525144in}}{\pgfqpoint{12.055609in}{2.529534in}}{\pgfqpoint{12.044558in}{2.529534in}}%
\pgfpathcurveto{\pgfqpoint{12.033508in}{2.529534in}}{\pgfqpoint{12.022909in}{2.525144in}}{\pgfqpoint{12.015096in}{2.517330in}}%
\pgfpathcurveto{\pgfqpoint{12.007282in}{2.509517in}}{\pgfqpoint{12.002892in}{2.498918in}}{\pgfqpoint{12.002892in}{2.487867in}}%
\pgfpathcurveto{\pgfqpoint{12.002892in}{2.476817in}}{\pgfqpoint{12.007282in}{2.466218in}}{\pgfqpoint{12.015096in}{2.458405in}}%
\pgfpathcurveto{\pgfqpoint{12.022909in}{2.450591in}}{\pgfqpoint{12.033508in}{2.446201in}}{\pgfqpoint{12.044558in}{2.446201in}}%
\pgfpathclose%
\pgfusepath{stroke,fill}%
\end{pgfscope}%
\begin{pgfscope}%
\pgfpathrectangle{\pgfqpoint{7.640588in}{0.566125in}}{\pgfqpoint{5.699255in}{2.685432in}}%
\pgfusepath{clip}%
\pgfsetbuttcap%
\pgfsetroundjoin%
\definecolor{currentfill}{rgb}{0.000000,0.000000,0.000000}%
\pgfsetfillcolor{currentfill}%
\pgfsetlinewidth{1.003750pt}%
\definecolor{currentstroke}{rgb}{0.000000,0.000000,0.000000}%
\pgfsetstrokecolor{currentstroke}%
\pgfsetdash{}{0pt}%
\pgfpathmoveto{\pgfqpoint{12.141705in}{2.712240in}}%
\pgfpathcurveto{\pgfqpoint{12.152755in}{2.712240in}}{\pgfqpoint{12.163354in}{2.716630in}}{\pgfqpoint{12.171168in}{2.724444in}}%
\pgfpathcurveto{\pgfqpoint{12.178981in}{2.732258in}}{\pgfqpoint{12.183372in}{2.742857in}}{\pgfqpoint{12.183372in}{2.753907in}}%
\pgfpathcurveto{\pgfqpoint{12.183372in}{2.764957in}}{\pgfqpoint{12.178981in}{2.775556in}}{\pgfqpoint{12.171168in}{2.783369in}}%
\pgfpathcurveto{\pgfqpoint{12.163354in}{2.791183in}}{\pgfqpoint{12.152755in}{2.795573in}}{\pgfqpoint{12.141705in}{2.795573in}}%
\pgfpathcurveto{\pgfqpoint{12.130655in}{2.795573in}}{\pgfqpoint{12.120056in}{2.791183in}}{\pgfqpoint{12.112242in}{2.783369in}}%
\pgfpathcurveto{\pgfqpoint{12.104428in}{2.775556in}}{\pgfqpoint{12.100038in}{2.764957in}}{\pgfqpoint{12.100038in}{2.753907in}}%
\pgfpathcurveto{\pgfqpoint{12.100038in}{2.742857in}}{\pgfqpoint{12.104428in}{2.732258in}}{\pgfqpoint{12.112242in}{2.724444in}}%
\pgfpathcurveto{\pgfqpoint{12.120056in}{2.716630in}}{\pgfqpoint{12.130655in}{2.712240in}}{\pgfqpoint{12.141705in}{2.712240in}}%
\pgfpathclose%
\pgfusepath{stroke,fill}%
\end{pgfscope}%
\begin{pgfscope}%
\pgfpathrectangle{\pgfqpoint{7.640588in}{0.566125in}}{\pgfqpoint{5.699255in}{2.685432in}}%
\pgfusepath{clip}%
\pgfsetbuttcap%
\pgfsetroundjoin%
\definecolor{currentfill}{rgb}{0.000000,0.000000,0.000000}%
\pgfsetfillcolor{currentfill}%
\pgfsetlinewidth{1.003750pt}%
\definecolor{currentstroke}{rgb}{0.000000,0.000000,0.000000}%
\pgfsetstrokecolor{currentstroke}%
\pgfsetdash{}{0pt}%
\pgfpathmoveto{\pgfqpoint{11.785501in}{2.712240in}}%
\pgfpathcurveto{\pgfqpoint{11.796552in}{2.712240in}}{\pgfqpoint{11.807151in}{2.716630in}}{\pgfqpoint{11.814964in}{2.724444in}}%
\pgfpathcurveto{\pgfqpoint{11.822778in}{2.732258in}}{\pgfqpoint{11.827168in}{2.742857in}}{\pgfqpoint{11.827168in}{2.753907in}}%
\pgfpathcurveto{\pgfqpoint{11.827168in}{2.764957in}}{\pgfqpoint{11.822778in}{2.775556in}}{\pgfqpoint{11.814964in}{2.783369in}}%
\pgfpathcurveto{\pgfqpoint{11.807151in}{2.791183in}}{\pgfqpoint{11.796552in}{2.795573in}}{\pgfqpoint{11.785501in}{2.795573in}}%
\pgfpathcurveto{\pgfqpoint{11.774451in}{2.795573in}}{\pgfqpoint{11.763852in}{2.791183in}}{\pgfqpoint{11.756039in}{2.783369in}}%
\pgfpathcurveto{\pgfqpoint{11.748225in}{2.775556in}}{\pgfqpoint{11.743835in}{2.764957in}}{\pgfqpoint{11.743835in}{2.753907in}}%
\pgfpathcurveto{\pgfqpoint{11.743835in}{2.742857in}}{\pgfqpoint{11.748225in}{2.732258in}}{\pgfqpoint{11.756039in}{2.724444in}}%
\pgfpathcurveto{\pgfqpoint{11.763852in}{2.716630in}}{\pgfqpoint{11.774451in}{2.712240in}}{\pgfqpoint{11.785501in}{2.712240in}}%
\pgfpathclose%
\pgfusepath{stroke,fill}%
\end{pgfscope}%
\begin{pgfscope}%
\pgfpathrectangle{\pgfqpoint{7.640588in}{0.566125in}}{\pgfqpoint{5.699255in}{2.685432in}}%
\pgfusepath{clip}%
\pgfsetbuttcap%
\pgfsetroundjoin%
\definecolor{currentfill}{rgb}{0.000000,0.000000,0.000000}%
\pgfsetfillcolor{currentfill}%
\pgfsetlinewidth{1.003750pt}%
\definecolor{currentstroke}{rgb}{0.000000,0.000000,0.000000}%
\pgfsetstrokecolor{currentstroke}%
\pgfsetdash{}{0pt}%
\pgfpathmoveto{\pgfqpoint{11.558826in}{2.837435in}}%
\pgfpathcurveto{\pgfqpoint{11.569877in}{2.837435in}}{\pgfqpoint{11.580476in}{2.841825in}}{\pgfqpoint{11.588289in}{2.849639in}}%
\pgfpathcurveto{\pgfqpoint{11.596103in}{2.857453in}}{\pgfqpoint{11.600493in}{2.868052in}}{\pgfqpoint{11.600493in}{2.879102in}}%
\pgfpathcurveto{\pgfqpoint{11.600493in}{2.890152in}}{\pgfqpoint{11.596103in}{2.900751in}}{\pgfqpoint{11.588289in}{2.908564in}}%
\pgfpathcurveto{\pgfqpoint{11.580476in}{2.916378in}}{\pgfqpoint{11.569877in}{2.920768in}}{\pgfqpoint{11.558826in}{2.920768in}}%
\pgfpathcurveto{\pgfqpoint{11.547776in}{2.920768in}}{\pgfqpoint{11.537177in}{2.916378in}}{\pgfqpoint{11.529364in}{2.908564in}}%
\pgfpathcurveto{\pgfqpoint{11.521550in}{2.900751in}}{\pgfqpoint{11.517160in}{2.890152in}}{\pgfqpoint{11.517160in}{2.879102in}}%
\pgfpathcurveto{\pgfqpoint{11.517160in}{2.868052in}}{\pgfqpoint{11.521550in}{2.857453in}}{\pgfqpoint{11.529364in}{2.849639in}}%
\pgfpathcurveto{\pgfqpoint{11.537177in}{2.841825in}}{\pgfqpoint{11.547776in}{2.837435in}}{\pgfqpoint{11.558826in}{2.837435in}}%
\pgfpathclose%
\pgfusepath{stroke,fill}%
\end{pgfscope}%
\begin{pgfscope}%
\pgfpathrectangle{\pgfqpoint{7.640588in}{0.566125in}}{\pgfqpoint{5.699255in}{2.685432in}}%
\pgfusepath{clip}%
\pgfsetbuttcap%
\pgfsetroundjoin%
\definecolor{currentfill}{rgb}{0.000000,0.000000,0.000000}%
\pgfsetfillcolor{currentfill}%
\pgfsetlinewidth{1.003750pt}%
\definecolor{currentstroke}{rgb}{0.000000,0.000000,0.000000}%
\pgfsetstrokecolor{currentstroke}%
\pgfsetdash{}{0pt}%
\pgfpathmoveto{\pgfqpoint{11.494062in}{2.790487in}}%
\pgfpathcurveto{\pgfqpoint{11.505112in}{2.790487in}}{\pgfqpoint{11.515711in}{2.794877in}}{\pgfqpoint{11.523525in}{2.802691in}}%
\pgfpathcurveto{\pgfqpoint{11.531339in}{2.810504in}}{\pgfqpoint{11.535729in}{2.821103in}}{\pgfqpoint{11.535729in}{2.832154in}}%
\pgfpathcurveto{\pgfqpoint{11.535729in}{2.843204in}}{\pgfqpoint{11.531339in}{2.853803in}}{\pgfqpoint{11.523525in}{2.861616in}}%
\pgfpathcurveto{\pgfqpoint{11.515711in}{2.869430in}}{\pgfqpoint{11.505112in}{2.873820in}}{\pgfqpoint{11.494062in}{2.873820in}}%
\pgfpathcurveto{\pgfqpoint{11.483012in}{2.873820in}}{\pgfqpoint{11.472413in}{2.869430in}}{\pgfqpoint{11.464599in}{2.861616in}}%
\pgfpathcurveto{\pgfqpoint{11.456786in}{2.853803in}}{\pgfqpoint{11.452396in}{2.843204in}}{\pgfqpoint{11.452396in}{2.832154in}}%
\pgfpathcurveto{\pgfqpoint{11.452396in}{2.821103in}}{\pgfqpoint{11.456786in}{2.810504in}}{\pgfqpoint{11.464599in}{2.802691in}}%
\pgfpathcurveto{\pgfqpoint{11.472413in}{2.794877in}}{\pgfqpoint{11.483012in}{2.790487in}}{\pgfqpoint{11.494062in}{2.790487in}}%
\pgfpathclose%
\pgfusepath{stroke,fill}%
\end{pgfscope}%
\begin{pgfscope}%
\pgfpathrectangle{\pgfqpoint{7.640588in}{0.566125in}}{\pgfqpoint{5.699255in}{2.685432in}}%
\pgfusepath{clip}%
\pgfsetbuttcap%
\pgfsetroundjoin%
\definecolor{currentfill}{rgb}{0.000000,0.000000,0.000000}%
\pgfsetfillcolor{currentfill}%
\pgfsetlinewidth{1.003750pt}%
\definecolor{currentstroke}{rgb}{0.000000,0.000000,0.000000}%
\pgfsetstrokecolor{currentstroke}%
\pgfsetdash{}{0pt}%
\pgfpathmoveto{\pgfqpoint{11.332152in}{2.821786in}}%
\pgfpathcurveto{\pgfqpoint{11.343202in}{2.821786in}}{\pgfqpoint{11.353801in}{2.826176in}}{\pgfqpoint{11.361614in}{2.833990in}}%
\pgfpathcurveto{\pgfqpoint{11.369428in}{2.841803in}}{\pgfqpoint{11.373818in}{2.852402in}}{\pgfqpoint{11.373818in}{2.863452in}}%
\pgfpathcurveto{\pgfqpoint{11.373818in}{2.874502in}}{\pgfqpoint{11.369428in}{2.885101in}}{\pgfqpoint{11.361614in}{2.892915in}}%
\pgfpathcurveto{\pgfqpoint{11.353801in}{2.900729in}}{\pgfqpoint{11.343202in}{2.905119in}}{\pgfqpoint{11.332152in}{2.905119in}}%
\pgfpathcurveto{\pgfqpoint{11.321101in}{2.905119in}}{\pgfqpoint{11.310502in}{2.900729in}}{\pgfqpoint{11.302689in}{2.892915in}}%
\pgfpathcurveto{\pgfqpoint{11.294875in}{2.885101in}}{\pgfqpoint{11.290485in}{2.874502in}}{\pgfqpoint{11.290485in}{2.863452in}}%
\pgfpathcurveto{\pgfqpoint{11.290485in}{2.852402in}}{\pgfqpoint{11.294875in}{2.841803in}}{\pgfqpoint{11.302689in}{2.833990in}}%
\pgfpathcurveto{\pgfqpoint{11.310502in}{2.826176in}}{\pgfqpoint{11.321101in}{2.821786in}}{\pgfqpoint{11.332152in}{2.821786in}}%
\pgfpathclose%
\pgfusepath{stroke,fill}%
\end{pgfscope}%
\begin{pgfscope}%
\pgfpathrectangle{\pgfqpoint{7.640588in}{0.566125in}}{\pgfqpoint{5.699255in}{2.685432in}}%
\pgfusepath{clip}%
\pgfsetbuttcap%
\pgfsetroundjoin%
\definecolor{currentfill}{rgb}{0.000000,0.000000,0.000000}%
\pgfsetfillcolor{currentfill}%
\pgfsetlinewidth{1.003750pt}%
\definecolor{currentstroke}{rgb}{0.000000,0.000000,0.000000}%
\pgfsetstrokecolor{currentstroke}%
\pgfsetdash{}{0pt}%
\pgfpathmoveto{\pgfqpoint{11.008330in}{2.853084in}}%
\pgfpathcurveto{\pgfqpoint{11.019380in}{2.853084in}}{\pgfqpoint{11.029979in}{2.857475in}}{\pgfqpoint{11.037793in}{2.865288in}}%
\pgfpathcurveto{\pgfqpoint{11.045607in}{2.873102in}}{\pgfqpoint{11.049997in}{2.883701in}}{\pgfqpoint{11.049997in}{2.894751in}}%
\pgfpathcurveto{\pgfqpoint{11.049997in}{2.905801in}}{\pgfqpoint{11.045607in}{2.916400in}}{\pgfqpoint{11.037793in}{2.924214in}}%
\pgfpathcurveto{\pgfqpoint{11.029979in}{2.932027in}}{\pgfqpoint{11.019380in}{2.936418in}}{\pgfqpoint{11.008330in}{2.936418in}}%
\pgfpathcurveto{\pgfqpoint{10.997280in}{2.936418in}}{\pgfqpoint{10.986681in}{2.932027in}}{\pgfqpoint{10.978867in}{2.924214in}}%
\pgfpathcurveto{\pgfqpoint{10.971054in}{2.916400in}}{\pgfqpoint{10.966664in}{2.905801in}}{\pgfqpoint{10.966664in}{2.894751in}}%
\pgfpathcurveto{\pgfqpoint{10.966664in}{2.883701in}}{\pgfqpoint{10.971054in}{2.873102in}}{\pgfqpoint{10.978867in}{2.865288in}}%
\pgfpathcurveto{\pgfqpoint{10.986681in}{2.857475in}}{\pgfqpoint{10.997280in}{2.853084in}}{\pgfqpoint{11.008330in}{2.853084in}}%
\pgfpathclose%
\pgfusepath{stroke,fill}%
\end{pgfscope}%
\begin{pgfscope}%
\pgfpathrectangle{\pgfqpoint{7.640588in}{0.566125in}}{\pgfqpoint{5.699255in}{2.685432in}}%
\pgfusepath{clip}%
\pgfsetbuttcap%
\pgfsetroundjoin%
\definecolor{currentfill}{rgb}{0.000000,0.000000,0.000000}%
\pgfsetfillcolor{currentfill}%
\pgfsetlinewidth{1.003750pt}%
\definecolor{currentstroke}{rgb}{0.000000,0.000000,0.000000}%
\pgfsetstrokecolor{currentstroke}%
\pgfsetdash{}{0pt}%
\pgfpathmoveto{\pgfqpoint{10.814037in}{2.821786in}}%
\pgfpathcurveto{\pgfqpoint{10.825088in}{2.821786in}}{\pgfqpoint{10.835687in}{2.826176in}}{\pgfqpoint{10.843500in}{2.833990in}}%
\pgfpathcurveto{\pgfqpoint{10.851314in}{2.841803in}}{\pgfqpoint{10.855704in}{2.852402in}}{\pgfqpoint{10.855704in}{2.863452in}}%
\pgfpathcurveto{\pgfqpoint{10.855704in}{2.874502in}}{\pgfqpoint{10.851314in}{2.885101in}}{\pgfqpoint{10.843500in}{2.892915in}}%
\pgfpathcurveto{\pgfqpoint{10.835687in}{2.900729in}}{\pgfqpoint{10.825088in}{2.905119in}}{\pgfqpoint{10.814037in}{2.905119in}}%
\pgfpathcurveto{\pgfqpoint{10.802987in}{2.905119in}}{\pgfqpoint{10.792388in}{2.900729in}}{\pgfqpoint{10.784575in}{2.892915in}}%
\pgfpathcurveto{\pgfqpoint{10.776761in}{2.885101in}}{\pgfqpoint{10.772371in}{2.874502in}}{\pgfqpoint{10.772371in}{2.863452in}}%
\pgfpathcurveto{\pgfqpoint{10.772371in}{2.852402in}}{\pgfqpoint{10.776761in}{2.841803in}}{\pgfqpoint{10.784575in}{2.833990in}}%
\pgfpathcurveto{\pgfqpoint{10.792388in}{2.826176in}}{\pgfqpoint{10.802987in}{2.821786in}}{\pgfqpoint{10.814037in}{2.821786in}}%
\pgfpathclose%
\pgfusepath{stroke,fill}%
\end{pgfscope}%
\begin{pgfscope}%
\pgfpathrectangle{\pgfqpoint{7.640588in}{0.566125in}}{\pgfqpoint{5.699255in}{2.685432in}}%
\pgfusepath{clip}%
\pgfsetbuttcap%
\pgfsetroundjoin%
\definecolor{currentfill}{rgb}{0.000000,0.000000,0.000000}%
\pgfsetfillcolor{currentfill}%
\pgfsetlinewidth{1.003750pt}%
\definecolor{currentstroke}{rgb}{0.000000,0.000000,0.000000}%
\pgfsetstrokecolor{currentstroke}%
\pgfsetdash{}{0pt}%
\pgfpathmoveto{\pgfqpoint{10.198777in}{2.727889in}}%
\pgfpathcurveto{\pgfqpoint{10.209827in}{2.727889in}}{\pgfqpoint{10.220426in}{2.732280in}}{\pgfqpoint{10.228240in}{2.740093in}}%
\pgfpathcurveto{\pgfqpoint{10.236053in}{2.747907in}}{\pgfqpoint{10.240444in}{2.758506in}}{\pgfqpoint{10.240444in}{2.769556in}}%
\pgfpathcurveto{\pgfqpoint{10.240444in}{2.780606in}}{\pgfqpoint{10.236053in}{2.791205in}}{\pgfqpoint{10.228240in}{2.799019in}}%
\pgfpathcurveto{\pgfqpoint{10.220426in}{2.806832in}}{\pgfqpoint{10.209827in}{2.811223in}}{\pgfqpoint{10.198777in}{2.811223in}}%
\pgfpathcurveto{\pgfqpoint{10.187727in}{2.811223in}}{\pgfqpoint{10.177128in}{2.806832in}}{\pgfqpoint{10.169314in}{2.799019in}}%
\pgfpathcurveto{\pgfqpoint{10.161500in}{2.791205in}}{\pgfqpoint{10.157110in}{2.780606in}}{\pgfqpoint{10.157110in}{2.769556in}}%
\pgfpathcurveto{\pgfqpoint{10.157110in}{2.758506in}}{\pgfqpoint{10.161500in}{2.747907in}}{\pgfqpoint{10.169314in}{2.740093in}}%
\pgfpathcurveto{\pgfqpoint{10.177128in}{2.732280in}}{\pgfqpoint{10.187727in}{2.727889in}}{\pgfqpoint{10.198777in}{2.727889in}}%
\pgfpathclose%
\pgfusepath{stroke,fill}%
\end{pgfscope}%
\begin{pgfscope}%
\pgfpathrectangle{\pgfqpoint{7.640588in}{0.566125in}}{\pgfqpoint{5.699255in}{2.685432in}}%
\pgfusepath{clip}%
\pgfsetbuttcap%
\pgfsetroundjoin%
\definecolor{currentfill}{rgb}{0.000000,0.000000,0.000000}%
\pgfsetfillcolor{currentfill}%
\pgfsetlinewidth{1.003750pt}%
\definecolor{currentstroke}{rgb}{0.000000,0.000000,0.000000}%
\pgfsetstrokecolor{currentstroke}%
\pgfsetdash{}{0pt}%
\pgfpathmoveto{\pgfqpoint{9.842573in}{2.680941in}}%
\pgfpathcurveto{\pgfqpoint{9.853624in}{2.680941in}}{\pgfqpoint{9.864223in}{2.685332in}}{\pgfqpoint{9.872036in}{2.693145in}}%
\pgfpathcurveto{\pgfqpoint{9.879850in}{2.700959in}}{\pgfqpoint{9.884240in}{2.711558in}}{\pgfqpoint{9.884240in}{2.722608in}}%
\pgfpathcurveto{\pgfqpoint{9.884240in}{2.733658in}}{\pgfqpoint{9.879850in}{2.744257in}}{\pgfqpoint{9.872036in}{2.752071in}}%
\pgfpathcurveto{\pgfqpoint{9.864223in}{2.759884in}}{\pgfqpoint{9.853624in}{2.764275in}}{\pgfqpoint{9.842573in}{2.764275in}}%
\pgfpathcurveto{\pgfqpoint{9.831523in}{2.764275in}}{\pgfqpoint{9.820924in}{2.759884in}}{\pgfqpoint{9.813111in}{2.752071in}}%
\pgfpathcurveto{\pgfqpoint{9.805297in}{2.744257in}}{\pgfqpoint{9.800907in}{2.733658in}}{\pgfqpoint{9.800907in}{2.722608in}}%
\pgfpathcurveto{\pgfqpoint{9.800907in}{2.711558in}}{\pgfqpoint{9.805297in}{2.700959in}}{\pgfqpoint{9.813111in}{2.693145in}}%
\pgfpathcurveto{\pgfqpoint{9.820924in}{2.685332in}}{\pgfqpoint{9.831523in}{2.680941in}}{\pgfqpoint{9.842573in}{2.680941in}}%
\pgfpathclose%
\pgfusepath{stroke,fill}%
\end{pgfscope}%
\begin{pgfscope}%
\pgfpathrectangle{\pgfqpoint{7.640588in}{0.566125in}}{\pgfqpoint{5.699255in}{2.685432in}}%
\pgfusepath{clip}%
\pgfsetbuttcap%
\pgfsetroundjoin%
\definecolor{currentfill}{rgb}{0.000000,0.000000,0.000000}%
\pgfsetfillcolor{currentfill}%
\pgfsetlinewidth{1.003750pt}%
\definecolor{currentstroke}{rgb}{0.000000,0.000000,0.000000}%
\pgfsetstrokecolor{currentstroke}%
\pgfsetdash{}{0pt}%
\pgfpathmoveto{\pgfqpoint{9.615898in}{2.555746in}}%
\pgfpathcurveto{\pgfqpoint{9.626949in}{2.555746in}}{\pgfqpoint{9.637548in}{2.560137in}}{\pgfqpoint{9.645361in}{2.567950in}}%
\pgfpathcurveto{\pgfqpoint{9.653175in}{2.575764in}}{\pgfqpoint{9.657565in}{2.586363in}}{\pgfqpoint{9.657565in}{2.597413in}}%
\pgfpathcurveto{\pgfqpoint{9.657565in}{2.608463in}}{\pgfqpoint{9.653175in}{2.619062in}}{\pgfqpoint{9.645361in}{2.626876in}}%
\pgfpathcurveto{\pgfqpoint{9.637548in}{2.634689in}}{\pgfqpoint{9.626949in}{2.639080in}}{\pgfqpoint{9.615898in}{2.639080in}}%
\pgfpathcurveto{\pgfqpoint{9.604848in}{2.639080in}}{\pgfqpoint{9.594249in}{2.634689in}}{\pgfqpoint{9.586436in}{2.626876in}}%
\pgfpathcurveto{\pgfqpoint{9.578622in}{2.619062in}}{\pgfqpoint{9.574232in}{2.608463in}}{\pgfqpoint{9.574232in}{2.597413in}}%
\pgfpathcurveto{\pgfqpoint{9.574232in}{2.586363in}}{\pgfqpoint{9.578622in}{2.575764in}}{\pgfqpoint{9.586436in}{2.567950in}}%
\pgfpathcurveto{\pgfqpoint{9.594249in}{2.560137in}}{\pgfqpoint{9.604848in}{2.555746in}}{\pgfqpoint{9.615898in}{2.555746in}}%
\pgfpathclose%
\pgfusepath{stroke,fill}%
\end{pgfscope}%
\begin{pgfscope}%
\pgfpathrectangle{\pgfqpoint{7.640588in}{0.566125in}}{\pgfqpoint{5.699255in}{2.685432in}}%
\pgfusepath{clip}%
\pgfsetbuttcap%
\pgfsetroundjoin%
\definecolor{currentfill}{rgb}{0.000000,0.000000,0.000000}%
\pgfsetfillcolor{currentfill}%
\pgfsetlinewidth{1.003750pt}%
\definecolor{currentstroke}{rgb}{0.000000,0.000000,0.000000}%
\pgfsetstrokecolor{currentstroke}%
\pgfsetdash{}{0pt}%
\pgfpathmoveto{\pgfqpoint{9.356841in}{2.446201in}}%
\pgfpathcurveto{\pgfqpoint{9.367892in}{2.446201in}}{\pgfqpoint{9.378491in}{2.450591in}}{\pgfqpoint{9.386304in}{2.458405in}}%
\pgfpathcurveto{\pgfqpoint{9.394118in}{2.466218in}}{\pgfqpoint{9.398508in}{2.476817in}}{\pgfqpoint{9.398508in}{2.487867in}}%
\pgfpathcurveto{\pgfqpoint{9.398508in}{2.498918in}}{\pgfqpoint{9.394118in}{2.509517in}}{\pgfqpoint{9.386304in}{2.517330in}}%
\pgfpathcurveto{\pgfqpoint{9.378491in}{2.525144in}}{\pgfqpoint{9.367892in}{2.529534in}}{\pgfqpoint{9.356841in}{2.529534in}}%
\pgfpathcurveto{\pgfqpoint{9.345791in}{2.529534in}}{\pgfqpoint{9.335192in}{2.525144in}}{\pgfqpoint{9.327379in}{2.517330in}}%
\pgfpathcurveto{\pgfqpoint{9.319565in}{2.509517in}}{\pgfqpoint{9.315175in}{2.498918in}}{\pgfqpoint{9.315175in}{2.487867in}}%
\pgfpathcurveto{\pgfqpoint{9.315175in}{2.476817in}}{\pgfqpoint{9.319565in}{2.466218in}}{\pgfqpoint{9.327379in}{2.458405in}}%
\pgfpathcurveto{\pgfqpoint{9.335192in}{2.450591in}}{\pgfqpoint{9.345791in}{2.446201in}}{\pgfqpoint{9.356841in}{2.446201in}}%
\pgfpathclose%
\pgfusepath{stroke,fill}%
\end{pgfscope}%
\begin{pgfscope}%
\pgfpathrectangle{\pgfqpoint{7.640588in}{0.566125in}}{\pgfqpoint{5.699255in}{2.685432in}}%
\pgfusepath{clip}%
\pgfsetbuttcap%
\pgfsetroundjoin%
\definecolor{currentfill}{rgb}{0.000000,0.000000,0.000000}%
\pgfsetfillcolor{currentfill}%
\pgfsetlinewidth{1.003750pt}%
\definecolor{currentstroke}{rgb}{0.000000,0.000000,0.000000}%
\pgfsetstrokecolor{currentstroke}%
\pgfsetdash{}{0pt}%
\pgfpathmoveto{\pgfqpoint{7.932028in}{1.554187in}}%
\pgfpathcurveto{\pgfqpoint{7.943078in}{1.554187in}}{\pgfqpoint{7.953677in}{1.558577in}}{\pgfqpoint{7.961490in}{1.566390in}}%
\pgfpathcurveto{\pgfqpoint{7.969304in}{1.574204in}}{\pgfqpoint{7.973694in}{1.584803in}}{\pgfqpoint{7.973694in}{1.595853in}}%
\pgfpathcurveto{\pgfqpoint{7.973694in}{1.606903in}}{\pgfqpoint{7.969304in}{1.617502in}}{\pgfqpoint{7.961490in}{1.625316in}}%
\pgfpathcurveto{\pgfqpoint{7.953677in}{1.633130in}}{\pgfqpoint{7.943078in}{1.637520in}}{\pgfqpoint{7.932028in}{1.637520in}}%
\pgfpathcurveto{\pgfqpoint{7.920977in}{1.637520in}}{\pgfqpoint{7.910378in}{1.633130in}}{\pgfqpoint{7.902565in}{1.625316in}}%
\pgfpathcurveto{\pgfqpoint{7.894751in}{1.617502in}}{\pgfqpoint{7.890361in}{1.606903in}}{\pgfqpoint{7.890361in}{1.595853in}}%
\pgfpathcurveto{\pgfqpoint{7.890361in}{1.584803in}}{\pgfqpoint{7.894751in}{1.574204in}}{\pgfqpoint{7.902565in}{1.566390in}}%
\pgfpathcurveto{\pgfqpoint{7.910378in}{1.558577in}}{\pgfqpoint{7.920977in}{1.554187in}}{\pgfqpoint{7.932028in}{1.554187in}}%
\pgfpathclose%
\pgfusepath{stroke,fill}%
\end{pgfscope}%
\begin{pgfscope}%
\pgfpathrectangle{\pgfqpoint{7.640588in}{0.566125in}}{\pgfqpoint{5.699255in}{2.685432in}}%
\pgfusepath{clip}%
\pgfsetbuttcap%
\pgfsetroundjoin%
\definecolor{currentfill}{rgb}{0.000000,0.000000,0.000000}%
\pgfsetfillcolor{currentfill}%
\pgfsetlinewidth{1.003750pt}%
\definecolor{currentstroke}{rgb}{0.000000,0.000000,0.000000}%
\pgfsetstrokecolor{currentstroke}%
\pgfsetdash{}{0pt}%
\pgfpathmoveto{\pgfqpoint{8.126320in}{2.321006in}}%
\pgfpathcurveto{\pgfqpoint{8.137370in}{2.321006in}}{\pgfqpoint{8.147970in}{2.325396in}}{\pgfqpoint{8.155783in}{2.333210in}}%
\pgfpathcurveto{\pgfqpoint{8.163597in}{2.341023in}}{\pgfqpoint{8.167987in}{2.351622in}}{\pgfqpoint{8.167987in}{2.362672in}}%
\pgfpathcurveto{\pgfqpoint{8.167987in}{2.373723in}}{\pgfqpoint{8.163597in}{2.384322in}}{\pgfqpoint{8.155783in}{2.392135in}}%
\pgfpathcurveto{\pgfqpoint{8.147970in}{2.399949in}}{\pgfqpoint{8.137370in}{2.404339in}}{\pgfqpoint{8.126320in}{2.404339in}}%
\pgfpathcurveto{\pgfqpoint{8.115270in}{2.404339in}}{\pgfqpoint{8.104671in}{2.399949in}}{\pgfqpoint{8.096858in}{2.392135in}}%
\pgfpathcurveto{\pgfqpoint{8.089044in}{2.384322in}}{\pgfqpoint{8.084654in}{2.373723in}}{\pgfqpoint{8.084654in}{2.362672in}}%
\pgfpathcurveto{\pgfqpoint{8.084654in}{2.351622in}}{\pgfqpoint{8.089044in}{2.341023in}}{\pgfqpoint{8.096858in}{2.333210in}}%
\pgfpathcurveto{\pgfqpoint{8.104671in}{2.325396in}}{\pgfqpoint{8.115270in}{2.321006in}}{\pgfqpoint{8.126320in}{2.321006in}}%
\pgfpathclose%
\pgfusepath{stroke,fill}%
\end{pgfscope}%
\begin{pgfscope}%
\pgfpathrectangle{\pgfqpoint{7.640588in}{0.566125in}}{\pgfqpoint{5.699255in}{2.685432in}}%
\pgfusepath{clip}%
\pgfsetbuttcap%
\pgfsetroundjoin%
\definecolor{currentfill}{rgb}{0.000000,0.000000,0.000000}%
\pgfsetfillcolor{currentfill}%
\pgfsetlinewidth{1.003750pt}%
\definecolor{currentstroke}{rgb}{0.000000,0.000000,0.000000}%
\pgfsetstrokecolor{currentstroke}%
\pgfsetdash{}{0pt}%
\pgfpathmoveto{\pgfqpoint{8.288231in}{2.352305in}}%
\pgfpathcurveto{\pgfqpoint{8.299281in}{2.352305in}}{\pgfqpoint{8.309880in}{2.356695in}}{\pgfqpoint{8.317694in}{2.364508in}}%
\pgfpathcurveto{\pgfqpoint{8.325507in}{2.372322in}}{\pgfqpoint{8.329898in}{2.382921in}}{\pgfqpoint{8.329898in}{2.393971in}}%
\pgfpathcurveto{\pgfqpoint{8.329898in}{2.405021in}}{\pgfqpoint{8.325507in}{2.415620in}}{\pgfqpoint{8.317694in}{2.423434in}}%
\pgfpathcurveto{\pgfqpoint{8.309880in}{2.431248in}}{\pgfqpoint{8.299281in}{2.435638in}}{\pgfqpoint{8.288231in}{2.435638in}}%
\pgfpathcurveto{\pgfqpoint{8.277181in}{2.435638in}}{\pgfqpoint{8.266582in}{2.431248in}}{\pgfqpoint{8.258768in}{2.423434in}}%
\pgfpathcurveto{\pgfqpoint{8.250955in}{2.415620in}}{\pgfqpoint{8.246564in}{2.405021in}}{\pgfqpoint{8.246564in}{2.393971in}}%
\pgfpathcurveto{\pgfqpoint{8.246564in}{2.382921in}}{\pgfqpoint{8.250955in}{2.372322in}}{\pgfqpoint{8.258768in}{2.364508in}}%
\pgfpathcurveto{\pgfqpoint{8.266582in}{2.356695in}}{\pgfqpoint{8.277181in}{2.352305in}}{\pgfqpoint{8.288231in}{2.352305in}}%
\pgfpathclose%
\pgfusepath{stroke,fill}%
\end{pgfscope}%
\begin{pgfscope}%
\pgfpathrectangle{\pgfqpoint{7.640588in}{0.566125in}}{\pgfqpoint{5.699255in}{2.685432in}}%
\pgfusepath{clip}%
\pgfsetbuttcap%
\pgfsetroundjoin%
\definecolor{currentfill}{rgb}{0.000000,0.000000,0.000000}%
\pgfsetfillcolor{currentfill}%
\pgfsetlinewidth{1.003750pt}%
\definecolor{currentstroke}{rgb}{0.000000,0.000000,0.000000}%
\pgfsetstrokecolor{currentstroke}%
\pgfsetdash{}{0pt}%
\pgfpathmoveto{\pgfqpoint{9.033020in}{2.117564in}}%
\pgfpathcurveto{\pgfqpoint{9.044070in}{2.117564in}}{\pgfqpoint{9.054669in}{2.121954in}}{\pgfqpoint{9.062483in}{2.129768in}}%
\pgfpathcurveto{\pgfqpoint{9.070296in}{2.137581in}}{\pgfqpoint{9.074687in}{2.148180in}}{\pgfqpoint{9.074687in}{2.159231in}}%
\pgfpathcurveto{\pgfqpoint{9.074687in}{2.170281in}}{\pgfqpoint{9.070296in}{2.180880in}}{\pgfqpoint{9.062483in}{2.188693in}}%
\pgfpathcurveto{\pgfqpoint{9.054669in}{2.196507in}}{\pgfqpoint{9.044070in}{2.200897in}}{\pgfqpoint{9.033020in}{2.200897in}}%
\pgfpathcurveto{\pgfqpoint{9.021970in}{2.200897in}}{\pgfqpoint{9.011371in}{2.196507in}}{\pgfqpoint{9.003557in}{2.188693in}}%
\pgfpathcurveto{\pgfqpoint{8.995744in}{2.180880in}}{\pgfqpoint{8.991353in}{2.170281in}}{\pgfqpoint{8.991353in}{2.159231in}}%
\pgfpathcurveto{\pgfqpoint{8.991353in}{2.148180in}}{\pgfqpoint{8.995744in}{2.137581in}}{\pgfqpoint{9.003557in}{2.129768in}}%
\pgfpathcurveto{\pgfqpoint{9.011371in}{2.121954in}}{\pgfqpoint{9.021970in}{2.117564in}}{\pgfqpoint{9.033020in}{2.117564in}}%
\pgfpathclose%
\pgfusepath{stroke,fill}%
\end{pgfscope}%
\begin{pgfscope}%
\pgfpathrectangle{\pgfqpoint{7.640588in}{0.566125in}}{\pgfqpoint{5.699255in}{2.685432in}}%
\pgfusepath{clip}%
\pgfsetbuttcap%
\pgfsetroundjoin%
\definecolor{currentfill}{rgb}{0.000000,0.000000,0.000000}%
\pgfsetfillcolor{currentfill}%
\pgfsetlinewidth{1.003750pt}%
\definecolor{currentstroke}{rgb}{0.000000,0.000000,0.000000}%
\pgfsetstrokecolor{currentstroke}%
\pgfsetdash{}{0pt}%
\pgfpathmoveto{\pgfqpoint{8.871109in}{2.054966in}}%
\pgfpathcurveto{\pgfqpoint{8.882160in}{2.054966in}}{\pgfqpoint{8.892759in}{2.059357in}}{\pgfqpoint{8.900572in}{2.067170in}}%
\pgfpathcurveto{\pgfqpoint{8.908386in}{2.074984in}}{\pgfqpoint{8.912776in}{2.085583in}}{\pgfqpoint{8.912776in}{2.096633in}}%
\pgfpathcurveto{\pgfqpoint{8.912776in}{2.107683in}}{\pgfqpoint{8.908386in}{2.118282in}}{\pgfqpoint{8.900572in}{2.126096in}}%
\pgfpathcurveto{\pgfqpoint{8.892759in}{2.133910in}}{\pgfqpoint{8.882160in}{2.138300in}}{\pgfqpoint{8.871109in}{2.138300in}}%
\pgfpathcurveto{\pgfqpoint{8.860059in}{2.138300in}}{\pgfqpoint{8.849460in}{2.133910in}}{\pgfqpoint{8.841647in}{2.126096in}}%
\pgfpathcurveto{\pgfqpoint{8.833833in}{2.118282in}}{\pgfqpoint{8.829443in}{2.107683in}}{\pgfqpoint{8.829443in}{2.096633in}}%
\pgfpathcurveto{\pgfqpoint{8.829443in}{2.085583in}}{\pgfqpoint{8.833833in}{2.074984in}}{\pgfqpoint{8.841647in}{2.067170in}}%
\pgfpathcurveto{\pgfqpoint{8.849460in}{2.059357in}}{\pgfqpoint{8.860059in}{2.054966in}}{\pgfqpoint{8.871109in}{2.054966in}}%
\pgfpathclose%
\pgfusepath{stroke,fill}%
\end{pgfscope}%
\begin{pgfscope}%
\pgfpathrectangle{\pgfqpoint{7.640588in}{0.566125in}}{\pgfqpoint{5.699255in}{2.685432in}}%
\pgfusepath{clip}%
\pgfsetbuttcap%
\pgfsetroundjoin%
\definecolor{currentfill}{rgb}{0.000000,0.000000,0.000000}%
\pgfsetfillcolor{currentfill}%
\pgfsetlinewidth{1.003750pt}%
\definecolor{currentstroke}{rgb}{0.000000,0.000000,0.000000}%
\pgfsetstrokecolor{currentstroke}%
\pgfsetdash{}{0pt}%
\pgfpathmoveto{\pgfqpoint{8.871109in}{1.898473in}}%
\pgfpathcurveto{\pgfqpoint{8.882160in}{1.898473in}}{\pgfqpoint{8.892759in}{1.902863in}}{\pgfqpoint{8.900572in}{1.910677in}}%
\pgfpathcurveto{\pgfqpoint{8.908386in}{1.918490in}}{\pgfqpoint{8.912776in}{1.929089in}}{\pgfqpoint{8.912776in}{1.940139in}}%
\pgfpathcurveto{\pgfqpoint{8.912776in}{1.951190in}}{\pgfqpoint{8.908386in}{1.961789in}}{\pgfqpoint{8.900572in}{1.969602in}}%
\pgfpathcurveto{\pgfqpoint{8.892759in}{1.977416in}}{\pgfqpoint{8.882160in}{1.981806in}}{\pgfqpoint{8.871109in}{1.981806in}}%
\pgfpathcurveto{\pgfqpoint{8.860059in}{1.981806in}}{\pgfqpoint{8.849460in}{1.977416in}}{\pgfqpoint{8.841647in}{1.969602in}}%
\pgfpathcurveto{\pgfqpoint{8.833833in}{1.961789in}}{\pgfqpoint{8.829443in}{1.951190in}}{\pgfqpoint{8.829443in}{1.940139in}}%
\pgfpathcurveto{\pgfqpoint{8.829443in}{1.929089in}}{\pgfqpoint{8.833833in}{1.918490in}}{\pgfqpoint{8.841647in}{1.910677in}}%
\pgfpathcurveto{\pgfqpoint{8.849460in}{1.902863in}}{\pgfqpoint{8.860059in}{1.898473in}}{\pgfqpoint{8.871109in}{1.898473in}}%
\pgfpathclose%
\pgfusepath{stroke,fill}%
\end{pgfscope}%
\begin{pgfscope}%
\pgfpathrectangle{\pgfqpoint{7.640588in}{0.566125in}}{\pgfqpoint{5.699255in}{2.685432in}}%
\pgfusepath{clip}%
\pgfsetbuttcap%
\pgfsetroundjoin%
\definecolor{currentfill}{rgb}{0.000000,0.000000,0.000000}%
\pgfsetfillcolor{currentfill}%
\pgfsetlinewidth{1.003750pt}%
\definecolor{currentstroke}{rgb}{0.000000,0.000000,0.000000}%
\pgfsetstrokecolor{currentstroke}%
\pgfsetdash{}{0pt}%
\pgfpathmoveto{\pgfqpoint{9.227313in}{1.882823in}}%
\pgfpathcurveto{\pgfqpoint{9.238363in}{1.882823in}}{\pgfqpoint{9.248962in}{1.887214in}}{\pgfqpoint{9.256776in}{1.895027in}}%
\pgfpathcurveto{\pgfqpoint{9.264589in}{1.902841in}}{\pgfqpoint{9.268980in}{1.913440in}}{\pgfqpoint{9.268980in}{1.924490in}}%
\pgfpathcurveto{\pgfqpoint{9.268980in}{1.935540in}}{\pgfqpoint{9.264589in}{1.946139in}}{\pgfqpoint{9.256776in}{1.953953in}}%
\pgfpathcurveto{\pgfqpoint{9.248962in}{1.961766in}}{\pgfqpoint{9.238363in}{1.966157in}}{\pgfqpoint{9.227313in}{1.966157in}}%
\pgfpathcurveto{\pgfqpoint{9.216263in}{1.966157in}}{\pgfqpoint{9.205664in}{1.961766in}}{\pgfqpoint{9.197850in}{1.953953in}}%
\pgfpathcurveto{\pgfqpoint{9.190036in}{1.946139in}}{\pgfqpoint{9.185646in}{1.935540in}}{\pgfqpoint{9.185646in}{1.924490in}}%
\pgfpathcurveto{\pgfqpoint{9.185646in}{1.913440in}}{\pgfqpoint{9.190036in}{1.902841in}}{\pgfqpoint{9.197850in}{1.895027in}}%
\pgfpathcurveto{\pgfqpoint{9.205664in}{1.887214in}}{\pgfqpoint{9.216263in}{1.882823in}}{\pgfqpoint{9.227313in}{1.882823in}}%
\pgfpathclose%
\pgfusepath{stroke,fill}%
\end{pgfscope}%
\begin{pgfscope}%
\pgfpathrectangle{\pgfqpoint{7.640588in}{0.566125in}}{\pgfqpoint{5.699255in}{2.685432in}}%
\pgfusepath{clip}%
\pgfsetbuttcap%
\pgfsetroundjoin%
\definecolor{currentfill}{rgb}{0.000000,0.000000,0.000000}%
\pgfsetfillcolor{currentfill}%
\pgfsetlinewidth{1.003750pt}%
\definecolor{currentstroke}{rgb}{0.000000,0.000000,0.000000}%
\pgfsetstrokecolor{currentstroke}%
\pgfsetdash{}{0pt}%
\pgfpathmoveto{\pgfqpoint{9.324459in}{1.741979in}}%
\pgfpathcurveto{\pgfqpoint{9.335509in}{1.741979in}}{\pgfqpoint{9.346108in}{1.746369in}}{\pgfqpoint{9.353922in}{1.754183in}}%
\pgfpathcurveto{\pgfqpoint{9.361736in}{1.761997in}}{\pgfqpoint{9.366126in}{1.772596in}}{\pgfqpoint{9.366126in}{1.783646in}}%
\pgfpathcurveto{\pgfqpoint{9.366126in}{1.794696in}}{\pgfqpoint{9.361736in}{1.805295in}}{\pgfqpoint{9.353922in}{1.813108in}}%
\pgfpathcurveto{\pgfqpoint{9.346108in}{1.820922in}}{\pgfqpoint{9.335509in}{1.825312in}}{\pgfqpoint{9.324459in}{1.825312in}}%
\pgfpathcurveto{\pgfqpoint{9.313409in}{1.825312in}}{\pgfqpoint{9.302810in}{1.820922in}}{\pgfqpoint{9.294996in}{1.813108in}}%
\pgfpathcurveto{\pgfqpoint{9.287183in}{1.805295in}}{\pgfqpoint{9.282793in}{1.794696in}}{\pgfqpoint{9.282793in}{1.783646in}}%
\pgfpathcurveto{\pgfqpoint{9.282793in}{1.772596in}}{\pgfqpoint{9.287183in}{1.761997in}}{\pgfqpoint{9.294996in}{1.754183in}}%
\pgfpathcurveto{\pgfqpoint{9.302810in}{1.746369in}}{\pgfqpoint{9.313409in}{1.741979in}}{\pgfqpoint{9.324459in}{1.741979in}}%
\pgfpathclose%
\pgfusepath{stroke,fill}%
\end{pgfscope}%
\begin{pgfscope}%
\pgfpathrectangle{\pgfqpoint{7.640588in}{0.566125in}}{\pgfqpoint{5.699255in}{2.685432in}}%
\pgfusepath{clip}%
\pgfsetbuttcap%
\pgfsetroundjoin%
\definecolor{currentfill}{rgb}{0.000000,0.000000,0.000000}%
\pgfsetfillcolor{currentfill}%
\pgfsetlinewidth{1.003750pt}%
\definecolor{currentstroke}{rgb}{0.000000,0.000000,0.000000}%
\pgfsetstrokecolor{currentstroke}%
\pgfsetdash{}{0pt}%
\pgfpathmoveto{\pgfqpoint{8.158702in}{1.116004in}}%
\pgfpathcurveto{\pgfqpoint{8.169753in}{1.116004in}}{\pgfqpoint{8.180352in}{1.120394in}}{\pgfqpoint{8.188165in}{1.128208in}}%
\pgfpathcurveto{\pgfqpoint{8.195979in}{1.136022in}}{\pgfqpoint{8.200369in}{1.146621in}}{\pgfqpoint{8.200369in}{1.157671in}}%
\pgfpathcurveto{\pgfqpoint{8.200369in}{1.168721in}}{\pgfqpoint{8.195979in}{1.179320in}}{\pgfqpoint{8.188165in}{1.187134in}}%
\pgfpathcurveto{\pgfqpoint{8.180352in}{1.194947in}}{\pgfqpoint{8.169753in}{1.199338in}}{\pgfqpoint{8.158702in}{1.199338in}}%
\pgfpathcurveto{\pgfqpoint{8.147652in}{1.199338in}}{\pgfqpoint{8.137053in}{1.194947in}}{\pgfqpoint{8.129240in}{1.187134in}}%
\pgfpathcurveto{\pgfqpoint{8.121426in}{1.179320in}}{\pgfqpoint{8.117036in}{1.168721in}}{\pgfqpoint{8.117036in}{1.157671in}}%
\pgfpathcurveto{\pgfqpoint{8.117036in}{1.146621in}}{\pgfqpoint{8.121426in}{1.136022in}}{\pgfqpoint{8.129240in}{1.128208in}}%
\pgfpathcurveto{\pgfqpoint{8.137053in}{1.120394in}}{\pgfqpoint{8.147652in}{1.116004in}}{\pgfqpoint{8.158702in}{1.116004in}}%
\pgfpathclose%
\pgfusepath{stroke,fill}%
\end{pgfscope}%
\begin{pgfscope}%
\pgfpathrectangle{\pgfqpoint{7.640588in}{0.566125in}}{\pgfqpoint{5.699255in}{2.685432in}}%
\pgfusepath{clip}%
\pgfsetbuttcap%
\pgfsetroundjoin%
\definecolor{currentfill}{rgb}{0.000000,0.000000,0.000000}%
\pgfsetfillcolor{currentfill}%
\pgfsetlinewidth{1.003750pt}%
\definecolor{currentstroke}{rgb}{0.000000,0.000000,0.000000}%
\pgfsetstrokecolor{currentstroke}%
\pgfsetdash{}{0pt}%
\pgfpathmoveto{\pgfqpoint{7.899645in}{1.037757in}}%
\pgfpathcurveto{\pgfqpoint{7.910696in}{1.037757in}}{\pgfqpoint{7.921295in}{1.042148in}}{\pgfqpoint{7.929108in}{1.049961in}}%
\pgfpathcurveto{\pgfqpoint{7.936922in}{1.057775in}}{\pgfqpoint{7.941312in}{1.068374in}}{\pgfqpoint{7.941312in}{1.079424in}}%
\pgfpathcurveto{\pgfqpoint{7.941312in}{1.090474in}}{\pgfqpoint{7.936922in}{1.101073in}}{\pgfqpoint{7.929108in}{1.108887in}}%
\pgfpathcurveto{\pgfqpoint{7.921295in}{1.116700in}}{\pgfqpoint{7.910696in}{1.121091in}}{\pgfqpoint{7.899645in}{1.121091in}}%
\pgfpathcurveto{\pgfqpoint{7.888595in}{1.121091in}}{\pgfqpoint{7.877996in}{1.116700in}}{\pgfqpoint{7.870183in}{1.108887in}}%
\pgfpathcurveto{\pgfqpoint{7.862369in}{1.101073in}}{\pgfqpoint{7.857979in}{1.090474in}}{\pgfqpoint{7.857979in}{1.079424in}}%
\pgfpathcurveto{\pgfqpoint{7.857979in}{1.068374in}}{\pgfqpoint{7.862369in}{1.057775in}}{\pgfqpoint{7.870183in}{1.049961in}}%
\pgfpathcurveto{\pgfqpoint{7.877996in}{1.042148in}}{\pgfqpoint{7.888595in}{1.037757in}}{\pgfqpoint{7.899645in}{1.037757in}}%
\pgfpathclose%
\pgfusepath{stroke,fill}%
\end{pgfscope}%
\begin{pgfscope}%
\pgfpathrectangle{\pgfqpoint{7.640588in}{0.566125in}}{\pgfqpoint{5.699255in}{2.685432in}}%
\pgfusepath{clip}%
\pgfsetbuttcap%
\pgfsetroundjoin%
\definecolor{currentfill}{rgb}{0.000000,0.000000,0.000000}%
\pgfsetfillcolor{currentfill}%
\pgfsetlinewidth{1.003750pt}%
\definecolor{currentstroke}{rgb}{0.000000,0.000000,0.000000}%
\pgfsetstrokecolor{currentstroke}%
\pgfsetdash{}{0pt}%
\pgfpathmoveto{\pgfqpoint{8.126320in}{0.740419in}}%
\pgfpathcurveto{\pgfqpoint{8.137370in}{0.740419in}}{\pgfqpoint{8.147970in}{0.744810in}}{\pgfqpoint{8.155783in}{0.752623in}}%
\pgfpathcurveto{\pgfqpoint{8.163597in}{0.760437in}}{\pgfqpoint{8.167987in}{0.771036in}}{\pgfqpoint{8.167987in}{0.782086in}}%
\pgfpathcurveto{\pgfqpoint{8.167987in}{0.793136in}}{\pgfqpoint{8.163597in}{0.803735in}}{\pgfqpoint{8.155783in}{0.811549in}}%
\pgfpathcurveto{\pgfqpoint{8.147970in}{0.819362in}}{\pgfqpoint{8.137370in}{0.823753in}}{\pgfqpoint{8.126320in}{0.823753in}}%
\pgfpathcurveto{\pgfqpoint{8.115270in}{0.823753in}}{\pgfqpoint{8.104671in}{0.819362in}}{\pgfqpoint{8.096858in}{0.811549in}}%
\pgfpathcurveto{\pgfqpoint{8.089044in}{0.803735in}}{\pgfqpoint{8.084654in}{0.793136in}}{\pgfqpoint{8.084654in}{0.782086in}}%
\pgfpathcurveto{\pgfqpoint{8.084654in}{0.771036in}}{\pgfqpoint{8.089044in}{0.760437in}}{\pgfqpoint{8.096858in}{0.752623in}}%
\pgfpathcurveto{\pgfqpoint{8.104671in}{0.744810in}}{\pgfqpoint{8.115270in}{0.740419in}}{\pgfqpoint{8.126320in}{0.740419in}}%
\pgfpathclose%
\pgfusepath{stroke,fill}%
\end{pgfscope}%
\begin{pgfscope}%
\pgfpathrectangle{\pgfqpoint{7.640588in}{0.566125in}}{\pgfqpoint{5.699255in}{2.685432in}}%
\pgfusepath{clip}%
\pgfsetbuttcap%
\pgfsetroundjoin%
\definecolor{currentfill}{rgb}{0.000000,0.000000,0.000000}%
\pgfsetfillcolor{currentfill}%
\pgfsetlinewidth{1.003750pt}%
\definecolor{currentstroke}{rgb}{0.000000,0.000000,0.000000}%
\pgfsetstrokecolor{currentstroke}%
\pgfsetdash{}{0pt}%
\pgfpathmoveto{\pgfqpoint{8.320613in}{0.896913in}}%
\pgfpathcurveto{\pgfqpoint{8.331663in}{0.896913in}}{\pgfqpoint{8.342262in}{0.901303in}}{\pgfqpoint{8.350076in}{0.909117in}}%
\pgfpathcurveto{\pgfqpoint{8.357890in}{0.916930in}}{\pgfqpoint{8.362280in}{0.927530in}}{\pgfqpoint{8.362280in}{0.938580in}}%
\pgfpathcurveto{\pgfqpoint{8.362280in}{0.949630in}}{\pgfqpoint{8.357890in}{0.960229in}}{\pgfqpoint{8.350076in}{0.968042in}}%
\pgfpathcurveto{\pgfqpoint{8.342262in}{0.975856in}}{\pgfqpoint{8.331663in}{0.980246in}}{\pgfqpoint{8.320613in}{0.980246in}}%
\pgfpathcurveto{\pgfqpoint{8.309563in}{0.980246in}}{\pgfqpoint{8.298964in}{0.975856in}}{\pgfqpoint{8.291150in}{0.968042in}}%
\pgfpathcurveto{\pgfqpoint{8.283337in}{0.960229in}}{\pgfqpoint{8.278946in}{0.949630in}}{\pgfqpoint{8.278946in}{0.938580in}}%
\pgfpathcurveto{\pgfqpoint{8.278946in}{0.927530in}}{\pgfqpoint{8.283337in}{0.916930in}}{\pgfqpoint{8.291150in}{0.909117in}}%
\pgfpathcurveto{\pgfqpoint{8.298964in}{0.901303in}}{\pgfqpoint{8.309563in}{0.896913in}}{\pgfqpoint{8.320613in}{0.896913in}}%
\pgfpathclose%
\pgfusepath{stroke,fill}%
\end{pgfscope}%
\begin{pgfscope}%
\pgfpathrectangle{\pgfqpoint{7.640588in}{0.566125in}}{\pgfqpoint{5.699255in}{2.685432in}}%
\pgfusepath{clip}%
\pgfsetbuttcap%
\pgfsetroundjoin%
\definecolor{currentfill}{rgb}{0.000000,0.000000,0.000000}%
\pgfsetfillcolor{currentfill}%
\pgfsetlinewidth{1.003750pt}%
\definecolor{currentstroke}{rgb}{0.000000,0.000000,0.000000}%
\pgfsetstrokecolor{currentstroke}%
\pgfsetdash{}{0pt}%
\pgfpathmoveto{\pgfqpoint{8.417760in}{1.022108in}}%
\pgfpathcurveto{\pgfqpoint{8.428810in}{1.022108in}}{\pgfqpoint{8.439409in}{1.026498in}}{\pgfqpoint{8.447222in}{1.034312in}}%
\pgfpathcurveto{\pgfqpoint{8.455036in}{1.042125in}}{\pgfqpoint{8.459426in}{1.052724in}}{\pgfqpoint{8.459426in}{1.063775in}}%
\pgfpathcurveto{\pgfqpoint{8.459426in}{1.074825in}}{\pgfqpoint{8.455036in}{1.085424in}}{\pgfqpoint{8.447222in}{1.093237in}}%
\pgfpathcurveto{\pgfqpoint{8.439409in}{1.101051in}}{\pgfqpoint{8.428810in}{1.105441in}}{\pgfqpoint{8.417760in}{1.105441in}}%
\pgfpathcurveto{\pgfqpoint{8.406709in}{1.105441in}}{\pgfqpoint{8.396110in}{1.101051in}}{\pgfqpoint{8.388297in}{1.093237in}}%
\pgfpathcurveto{\pgfqpoint{8.380483in}{1.085424in}}{\pgfqpoint{8.376093in}{1.074825in}}{\pgfqpoint{8.376093in}{1.063775in}}%
\pgfpathcurveto{\pgfqpoint{8.376093in}{1.052724in}}{\pgfqpoint{8.380483in}{1.042125in}}{\pgfqpoint{8.388297in}{1.034312in}}%
\pgfpathcurveto{\pgfqpoint{8.396110in}{1.026498in}}{\pgfqpoint{8.406709in}{1.022108in}}{\pgfqpoint{8.417760in}{1.022108in}}%
\pgfpathclose%
\pgfusepath{stroke,fill}%
\end{pgfscope}%
\begin{pgfscope}%
\pgfpathrectangle{\pgfqpoint{7.640588in}{0.566125in}}{\pgfqpoint{5.699255in}{2.685432in}}%
\pgfusepath{clip}%
\pgfsetbuttcap%
\pgfsetroundjoin%
\definecolor{currentfill}{rgb}{0.000000,0.000000,0.000000}%
\pgfsetfillcolor{currentfill}%
\pgfsetlinewidth{1.003750pt}%
\definecolor{currentstroke}{rgb}{0.000000,0.000000,0.000000}%
\pgfsetstrokecolor{currentstroke}%
\pgfsetdash{}{0pt}%
\pgfpathmoveto{\pgfqpoint{8.547288in}{1.053407in}}%
\pgfpathcurveto{\pgfqpoint{8.558338in}{1.053407in}}{\pgfqpoint{8.568937in}{1.057797in}}{\pgfqpoint{8.576751in}{1.065611in}}%
\pgfpathcurveto{\pgfqpoint{8.584564in}{1.073424in}}{\pgfqpoint{8.588955in}{1.084023in}}{\pgfqpoint{8.588955in}{1.095073in}}%
\pgfpathcurveto{\pgfqpoint{8.588955in}{1.106123in}}{\pgfqpoint{8.584564in}{1.116723in}}{\pgfqpoint{8.576751in}{1.124536in}}%
\pgfpathcurveto{\pgfqpoint{8.568937in}{1.132350in}}{\pgfqpoint{8.558338in}{1.136740in}}{\pgfqpoint{8.547288in}{1.136740in}}%
\pgfpathcurveto{\pgfqpoint{8.536238in}{1.136740in}}{\pgfqpoint{8.525639in}{1.132350in}}{\pgfqpoint{8.517825in}{1.124536in}}%
\pgfpathcurveto{\pgfqpoint{8.510012in}{1.116723in}}{\pgfqpoint{8.505621in}{1.106123in}}{\pgfqpoint{8.505621in}{1.095073in}}%
\pgfpathcurveto{\pgfqpoint{8.505621in}{1.084023in}}{\pgfqpoint{8.510012in}{1.073424in}}{\pgfqpoint{8.517825in}{1.065611in}}%
\pgfpathcurveto{\pgfqpoint{8.525639in}{1.057797in}}{\pgfqpoint{8.536238in}{1.053407in}}{\pgfqpoint{8.547288in}{1.053407in}}%
\pgfpathclose%
\pgfusepath{stroke,fill}%
\end{pgfscope}%
\begin{pgfscope}%
\pgfpathrectangle{\pgfqpoint{7.640588in}{0.566125in}}{\pgfqpoint{5.699255in}{2.685432in}}%
\pgfusepath{clip}%
\pgfsetbuttcap%
\pgfsetroundjoin%
\definecolor{currentfill}{rgb}{0.000000,0.000000,0.000000}%
\pgfsetfillcolor{currentfill}%
\pgfsetlinewidth{1.003750pt}%
\definecolor{currentstroke}{rgb}{0.000000,0.000000,0.000000}%
\pgfsetstrokecolor{currentstroke}%
\pgfsetdash{}{0pt}%
\pgfpathmoveto{\pgfqpoint{8.806345in}{1.084705in}}%
\pgfpathcurveto{\pgfqpoint{8.817395in}{1.084705in}}{\pgfqpoint{8.827994in}{1.089096in}}{\pgfqpoint{8.835808in}{1.096909in}}%
\pgfpathcurveto{\pgfqpoint{8.843622in}{1.104723in}}{\pgfqpoint{8.848012in}{1.115322in}}{\pgfqpoint{8.848012in}{1.126372in}}%
\pgfpathcurveto{\pgfqpoint{8.848012in}{1.137422in}}{\pgfqpoint{8.843622in}{1.148021in}}{\pgfqpoint{8.835808in}{1.155835in}}%
\pgfpathcurveto{\pgfqpoint{8.827994in}{1.163649in}}{\pgfqpoint{8.817395in}{1.168039in}}{\pgfqpoint{8.806345in}{1.168039in}}%
\pgfpathcurveto{\pgfqpoint{8.795295in}{1.168039in}}{\pgfqpoint{8.784696in}{1.163649in}}{\pgfqpoint{8.776882in}{1.155835in}}%
\pgfpathcurveto{\pgfqpoint{8.769069in}{1.148021in}}{\pgfqpoint{8.764678in}{1.137422in}}{\pgfqpoint{8.764678in}{1.126372in}}%
\pgfpathcurveto{\pgfqpoint{8.764678in}{1.115322in}}{\pgfqpoint{8.769069in}{1.104723in}}{\pgfqpoint{8.776882in}{1.096909in}}%
\pgfpathcurveto{\pgfqpoint{8.784696in}{1.089096in}}{\pgfqpoint{8.795295in}{1.084705in}}{\pgfqpoint{8.806345in}{1.084705in}}%
\pgfpathclose%
\pgfusepath{stroke,fill}%
\end{pgfscope}%
\begin{pgfscope}%
\pgfpathrectangle{\pgfqpoint{7.640588in}{0.566125in}}{\pgfqpoint{5.699255in}{2.685432in}}%
\pgfusepath{clip}%
\pgfsetbuttcap%
\pgfsetroundjoin%
\definecolor{currentfill}{rgb}{0.000000,0.000000,0.000000}%
\pgfsetfillcolor{currentfill}%
\pgfsetlinewidth{1.003750pt}%
\definecolor{currentstroke}{rgb}{0.000000,0.000000,0.000000}%
\pgfsetstrokecolor{currentstroke}%
\pgfsetdash{}{0pt}%
\pgfpathmoveto{\pgfqpoint{8.676817in}{0.959510in}}%
\pgfpathcurveto{\pgfqpoint{8.687867in}{0.959510in}}{\pgfqpoint{8.698466in}{0.963901in}}{\pgfqpoint{8.706279in}{0.971714in}}%
\pgfpathcurveto{\pgfqpoint{8.714093in}{0.979528in}}{\pgfqpoint{8.718483in}{0.990127in}}{\pgfqpoint{8.718483in}{1.001177in}}%
\pgfpathcurveto{\pgfqpoint{8.718483in}{1.012227in}}{\pgfqpoint{8.714093in}{1.022826in}}{\pgfqpoint{8.706279in}{1.030640in}}%
\pgfpathcurveto{\pgfqpoint{8.698466in}{1.038454in}}{\pgfqpoint{8.687867in}{1.042844in}}{\pgfqpoint{8.676817in}{1.042844in}}%
\pgfpathcurveto{\pgfqpoint{8.665766in}{1.042844in}}{\pgfqpoint{8.655167in}{1.038454in}}{\pgfqpoint{8.647354in}{1.030640in}}%
\pgfpathcurveto{\pgfqpoint{8.639540in}{1.022826in}}{\pgfqpoint{8.635150in}{1.012227in}}{\pgfqpoint{8.635150in}{1.001177in}}%
\pgfpathcurveto{\pgfqpoint{8.635150in}{0.990127in}}{\pgfqpoint{8.639540in}{0.979528in}}{\pgfqpoint{8.647354in}{0.971714in}}%
\pgfpathcurveto{\pgfqpoint{8.655167in}{0.963901in}}{\pgfqpoint{8.665766in}{0.959510in}}{\pgfqpoint{8.676817in}{0.959510in}}%
\pgfpathclose%
\pgfusepath{stroke,fill}%
\end{pgfscope}%
\begin{pgfscope}%
\pgfpathrectangle{\pgfqpoint{7.640588in}{0.566125in}}{\pgfqpoint{5.699255in}{2.685432in}}%
\pgfusepath{clip}%
\pgfsetbuttcap%
\pgfsetroundjoin%
\definecolor{currentfill}{rgb}{0.000000,0.000000,0.000000}%
\pgfsetfillcolor{currentfill}%
\pgfsetlinewidth{1.003750pt}%
\definecolor{currentstroke}{rgb}{0.000000,0.000000,0.000000}%
\pgfsetstrokecolor{currentstroke}%
\pgfsetdash{}{0pt}%
\pgfpathmoveto{\pgfqpoint{8.903492in}{0.912562in}}%
\pgfpathcurveto{\pgfqpoint{8.914542in}{0.912562in}}{\pgfqpoint{8.925141in}{0.916953in}}{\pgfqpoint{8.932954in}{0.924766in}}%
\pgfpathcurveto{\pgfqpoint{8.940768in}{0.932580in}}{\pgfqpoint{8.945158in}{0.943179in}}{\pgfqpoint{8.945158in}{0.954229in}}%
\pgfpathcurveto{\pgfqpoint{8.945158in}{0.965279in}}{\pgfqpoint{8.940768in}{0.975878in}}{\pgfqpoint{8.932954in}{0.983692in}}%
\pgfpathcurveto{\pgfqpoint{8.925141in}{0.991505in}}{\pgfqpoint{8.914542in}{0.995896in}}{\pgfqpoint{8.903492in}{0.995896in}}%
\pgfpathcurveto{\pgfqpoint{8.892441in}{0.995896in}}{\pgfqpoint{8.881842in}{0.991505in}}{\pgfqpoint{8.874029in}{0.983692in}}%
\pgfpathcurveto{\pgfqpoint{8.866215in}{0.975878in}}{\pgfqpoint{8.861825in}{0.965279in}}{\pgfqpoint{8.861825in}{0.954229in}}%
\pgfpathcurveto{\pgfqpoint{8.861825in}{0.943179in}}{\pgfqpoint{8.866215in}{0.932580in}}{\pgfqpoint{8.874029in}{0.924766in}}%
\pgfpathcurveto{\pgfqpoint{8.881842in}{0.916953in}}{\pgfqpoint{8.892441in}{0.912562in}}{\pgfqpoint{8.903492in}{0.912562in}}%
\pgfpathclose%
\pgfusepath{stroke,fill}%
\end{pgfscope}%
\begin{pgfscope}%
\pgfpathrectangle{\pgfqpoint{7.640588in}{0.566125in}}{\pgfqpoint{5.699255in}{2.685432in}}%
\pgfusepath{clip}%
\pgfsetbuttcap%
\pgfsetroundjoin%
\definecolor{currentfill}{rgb}{0.000000,0.000000,0.000000}%
\pgfsetfillcolor{currentfill}%
\pgfsetlinewidth{1.003750pt}%
\definecolor{currentstroke}{rgb}{0.000000,0.000000,0.000000}%
\pgfsetstrokecolor{currentstroke}%
\pgfsetdash{}{0pt}%
\pgfpathmoveto{\pgfqpoint{9.033020in}{0.975160in}}%
\pgfpathcurveto{\pgfqpoint{9.044070in}{0.975160in}}{\pgfqpoint{9.054669in}{0.979550in}}{\pgfqpoint{9.062483in}{0.987364in}}%
\pgfpathcurveto{\pgfqpoint{9.070296in}{0.995177in}}{\pgfqpoint{9.074687in}{1.005776in}}{\pgfqpoint{9.074687in}{1.016827in}}%
\pgfpathcurveto{\pgfqpoint{9.074687in}{1.027877in}}{\pgfqpoint{9.070296in}{1.038476in}}{\pgfqpoint{9.062483in}{1.046289in}}%
\pgfpathcurveto{\pgfqpoint{9.054669in}{1.054103in}}{\pgfqpoint{9.044070in}{1.058493in}}{\pgfqpoint{9.033020in}{1.058493in}}%
\pgfpathcurveto{\pgfqpoint{9.021970in}{1.058493in}}{\pgfqpoint{9.011371in}{1.054103in}}{\pgfqpoint{9.003557in}{1.046289in}}%
\pgfpathcurveto{\pgfqpoint{8.995744in}{1.038476in}}{\pgfqpoint{8.991353in}{1.027877in}}{\pgfqpoint{8.991353in}{1.016827in}}%
\pgfpathcurveto{\pgfqpoint{8.991353in}{1.005776in}}{\pgfqpoint{8.995744in}{0.995177in}}{\pgfqpoint{9.003557in}{0.987364in}}%
\pgfpathcurveto{\pgfqpoint{9.011371in}{0.979550in}}{\pgfqpoint{9.021970in}{0.975160in}}{\pgfqpoint{9.033020in}{0.975160in}}%
\pgfpathclose%
\pgfusepath{stroke,fill}%
\end{pgfscope}%
\begin{pgfscope}%
\pgfpathrectangle{\pgfqpoint{7.640588in}{0.566125in}}{\pgfqpoint{5.699255in}{2.685432in}}%
\pgfusepath{clip}%
\pgfsetbuttcap%
\pgfsetroundjoin%
\definecolor{currentfill}{rgb}{0.000000,0.000000,0.000000}%
\pgfsetfillcolor{currentfill}%
\pgfsetlinewidth{1.003750pt}%
\definecolor{currentstroke}{rgb}{0.000000,0.000000,0.000000}%
\pgfsetstrokecolor{currentstroke}%
\pgfsetdash{}{0pt}%
\pgfpathmoveto{\pgfqpoint{9.194931in}{0.928212in}}%
\pgfpathcurveto{\pgfqpoint{9.205981in}{0.928212in}}{\pgfqpoint{9.216580in}{0.932602in}}{\pgfqpoint{9.224394in}{0.940416in}}%
\pgfpathcurveto{\pgfqpoint{9.232207in}{0.948229in}}{\pgfqpoint{9.236597in}{0.958828in}}{\pgfqpoint{9.236597in}{0.969878in}}%
\pgfpathcurveto{\pgfqpoint{9.236597in}{0.980929in}}{\pgfqpoint{9.232207in}{0.991528in}}{\pgfqpoint{9.224394in}{0.999341in}}%
\pgfpathcurveto{\pgfqpoint{9.216580in}{1.007155in}}{\pgfqpoint{9.205981in}{1.011545in}}{\pgfqpoint{9.194931in}{1.011545in}}%
\pgfpathcurveto{\pgfqpoint{9.183881in}{1.011545in}}{\pgfqpoint{9.173282in}{1.007155in}}{\pgfqpoint{9.165468in}{0.999341in}}%
\pgfpathcurveto{\pgfqpoint{9.157654in}{0.991528in}}{\pgfqpoint{9.153264in}{0.980929in}}{\pgfqpoint{9.153264in}{0.969878in}}%
\pgfpathcurveto{\pgfqpoint{9.153264in}{0.958828in}}{\pgfqpoint{9.157654in}{0.948229in}}{\pgfqpoint{9.165468in}{0.940416in}}%
\pgfpathcurveto{\pgfqpoint{9.173282in}{0.932602in}}{\pgfqpoint{9.183881in}{0.928212in}}{\pgfqpoint{9.194931in}{0.928212in}}%
\pgfpathclose%
\pgfusepath{stroke,fill}%
\end{pgfscope}%
\begin{pgfscope}%
\pgfpathrectangle{\pgfqpoint{7.640588in}{0.566125in}}{\pgfqpoint{5.699255in}{2.685432in}}%
\pgfusepath{clip}%
\pgfsetbuttcap%
\pgfsetroundjoin%
\definecolor{currentfill}{rgb}{0.000000,0.000000,0.000000}%
\pgfsetfillcolor{currentfill}%
\pgfsetlinewidth{1.003750pt}%
\definecolor{currentstroke}{rgb}{0.000000,0.000000,0.000000}%
\pgfsetstrokecolor{currentstroke}%
\pgfsetdash{}{0pt}%
\pgfpathmoveto{\pgfqpoint{9.356841in}{0.834315in}}%
\pgfpathcurveto{\pgfqpoint{9.367892in}{0.834315in}}{\pgfqpoint{9.378491in}{0.838706in}}{\pgfqpoint{9.386304in}{0.846519in}}%
\pgfpathcurveto{\pgfqpoint{9.394118in}{0.854333in}}{\pgfqpoint{9.398508in}{0.864932in}}{\pgfqpoint{9.398508in}{0.875982in}}%
\pgfpathcurveto{\pgfqpoint{9.398508in}{0.887032in}}{\pgfqpoint{9.394118in}{0.897631in}}{\pgfqpoint{9.386304in}{0.905445in}}%
\pgfpathcurveto{\pgfqpoint{9.378491in}{0.913259in}}{\pgfqpoint{9.367892in}{0.917649in}}{\pgfqpoint{9.356841in}{0.917649in}}%
\pgfpathcurveto{\pgfqpoint{9.345791in}{0.917649in}}{\pgfqpoint{9.335192in}{0.913259in}}{\pgfqpoint{9.327379in}{0.905445in}}%
\pgfpathcurveto{\pgfqpoint{9.319565in}{0.897631in}}{\pgfqpoint{9.315175in}{0.887032in}}{\pgfqpoint{9.315175in}{0.875982in}}%
\pgfpathcurveto{\pgfqpoint{9.315175in}{0.864932in}}{\pgfqpoint{9.319565in}{0.854333in}}{\pgfqpoint{9.327379in}{0.846519in}}%
\pgfpathcurveto{\pgfqpoint{9.335192in}{0.838706in}}{\pgfqpoint{9.345791in}{0.834315in}}{\pgfqpoint{9.356841in}{0.834315in}}%
\pgfpathclose%
\pgfusepath{stroke,fill}%
\end{pgfscope}%
\begin{pgfscope}%
\pgfpathrectangle{\pgfqpoint{7.640588in}{0.566125in}}{\pgfqpoint{5.699255in}{2.685432in}}%
\pgfusepath{clip}%
\pgfsetbuttcap%
\pgfsetroundjoin%
\definecolor{currentfill}{rgb}{0.000000,0.000000,0.000000}%
\pgfsetfillcolor{currentfill}%
\pgfsetlinewidth{1.003750pt}%
\definecolor{currentstroke}{rgb}{0.000000,0.000000,0.000000}%
\pgfsetstrokecolor{currentstroke}%
\pgfsetdash{}{0pt}%
\pgfpathmoveto{\pgfqpoint{9.615898in}{0.709121in}}%
\pgfpathcurveto{\pgfqpoint{9.626949in}{0.709121in}}{\pgfqpoint{9.637548in}{0.713511in}}{\pgfqpoint{9.645361in}{0.721324in}}%
\pgfpathcurveto{\pgfqpoint{9.653175in}{0.729138in}}{\pgfqpoint{9.657565in}{0.739737in}}{\pgfqpoint{9.657565in}{0.750787in}}%
\pgfpathcurveto{\pgfqpoint{9.657565in}{0.761837in}}{\pgfqpoint{9.653175in}{0.772436in}}{\pgfqpoint{9.645361in}{0.780250in}}%
\pgfpathcurveto{\pgfqpoint{9.637548in}{0.788064in}}{\pgfqpoint{9.626949in}{0.792454in}}{\pgfqpoint{9.615898in}{0.792454in}}%
\pgfpathcurveto{\pgfqpoint{9.604848in}{0.792454in}}{\pgfqpoint{9.594249in}{0.788064in}}{\pgfqpoint{9.586436in}{0.780250in}}%
\pgfpathcurveto{\pgfqpoint{9.578622in}{0.772436in}}{\pgfqpoint{9.574232in}{0.761837in}}{\pgfqpoint{9.574232in}{0.750787in}}%
\pgfpathcurveto{\pgfqpoint{9.574232in}{0.739737in}}{\pgfqpoint{9.578622in}{0.729138in}}{\pgfqpoint{9.586436in}{0.721324in}}%
\pgfpathcurveto{\pgfqpoint{9.594249in}{0.713511in}}{\pgfqpoint{9.604848in}{0.709121in}}{\pgfqpoint{9.615898in}{0.709121in}}%
\pgfpathclose%
\pgfusepath{stroke,fill}%
\end{pgfscope}%
\begin{pgfscope}%
\pgfpathrectangle{\pgfqpoint{7.640588in}{0.566125in}}{\pgfqpoint{5.699255in}{2.685432in}}%
\pgfusepath{clip}%
\pgfsetbuttcap%
\pgfsetroundjoin%
\definecolor{currentfill}{rgb}{0.000000,0.000000,0.000000}%
\pgfsetfillcolor{currentfill}%
\pgfsetlinewidth{1.003750pt}%
\definecolor{currentstroke}{rgb}{0.000000,0.000000,0.000000}%
\pgfsetstrokecolor{currentstroke}%
\pgfsetdash{}{0pt}%
\pgfpathmoveto{\pgfqpoint{9.810191in}{0.912562in}}%
\pgfpathcurveto{\pgfqpoint{9.821241in}{0.912562in}}{\pgfqpoint{9.831840in}{0.916953in}}{\pgfqpoint{9.839654in}{0.924766in}}%
\pgfpathcurveto{\pgfqpoint{9.847468in}{0.932580in}}{\pgfqpoint{9.851858in}{0.943179in}}{\pgfqpoint{9.851858in}{0.954229in}}%
\pgfpathcurveto{\pgfqpoint{9.851858in}{0.965279in}}{\pgfqpoint{9.847468in}{0.975878in}}{\pgfqpoint{9.839654in}{0.983692in}}%
\pgfpathcurveto{\pgfqpoint{9.831840in}{0.991505in}}{\pgfqpoint{9.821241in}{0.995896in}}{\pgfqpoint{9.810191in}{0.995896in}}%
\pgfpathcurveto{\pgfqpoint{9.799141in}{0.995896in}}{\pgfqpoint{9.788542in}{0.991505in}}{\pgfqpoint{9.780728in}{0.983692in}}%
\pgfpathcurveto{\pgfqpoint{9.772915in}{0.975878in}}{\pgfqpoint{9.768525in}{0.965279in}}{\pgfqpoint{9.768525in}{0.954229in}}%
\pgfpathcurveto{\pgfqpoint{9.768525in}{0.943179in}}{\pgfqpoint{9.772915in}{0.932580in}}{\pgfqpoint{9.780728in}{0.924766in}}%
\pgfpathcurveto{\pgfqpoint{9.788542in}{0.916953in}}{\pgfqpoint{9.799141in}{0.912562in}}{\pgfqpoint{9.810191in}{0.912562in}}%
\pgfpathclose%
\pgfusepath{stroke,fill}%
\end{pgfscope}%
\begin{pgfscope}%
\pgfpathrectangle{\pgfqpoint{7.640588in}{0.566125in}}{\pgfqpoint{5.699255in}{2.685432in}}%
\pgfusepath{clip}%
\pgfsetbuttcap%
\pgfsetroundjoin%
\definecolor{currentfill}{rgb}{0.000000,0.000000,0.000000}%
\pgfsetfillcolor{currentfill}%
\pgfsetlinewidth{1.003750pt}%
\definecolor{currentstroke}{rgb}{0.000000,0.000000,0.000000}%
\pgfsetstrokecolor{currentstroke}%
\pgfsetdash{}{0pt}%
\pgfpathmoveto{\pgfqpoint{9.907338in}{1.131654in}}%
\pgfpathcurveto{\pgfqpoint{9.918388in}{1.131654in}}{\pgfqpoint{9.928987in}{1.136044in}}{\pgfqpoint{9.936800in}{1.143857in}}%
\pgfpathcurveto{\pgfqpoint{9.944614in}{1.151671in}}{\pgfqpoint{9.949004in}{1.162270in}}{\pgfqpoint{9.949004in}{1.173320in}}%
\pgfpathcurveto{\pgfqpoint{9.949004in}{1.184370in}}{\pgfqpoint{9.944614in}{1.194969in}}{\pgfqpoint{9.936800in}{1.202783in}}%
\pgfpathcurveto{\pgfqpoint{9.928987in}{1.210597in}}{\pgfqpoint{9.918388in}{1.214987in}}{\pgfqpoint{9.907338in}{1.214987in}}%
\pgfpathcurveto{\pgfqpoint{9.896288in}{1.214987in}}{\pgfqpoint{9.885689in}{1.210597in}}{\pgfqpoint{9.877875in}{1.202783in}}%
\pgfpathcurveto{\pgfqpoint{9.870061in}{1.194969in}}{\pgfqpoint{9.865671in}{1.184370in}}{\pgfqpoint{9.865671in}{1.173320in}}%
\pgfpathcurveto{\pgfqpoint{9.865671in}{1.162270in}}{\pgfqpoint{9.870061in}{1.151671in}}{\pgfqpoint{9.877875in}{1.143857in}}%
\pgfpathcurveto{\pgfqpoint{9.885689in}{1.136044in}}{\pgfqpoint{9.896288in}{1.131654in}}{\pgfqpoint{9.907338in}{1.131654in}}%
\pgfpathclose%
\pgfusepath{stroke,fill}%
\end{pgfscope}%
\begin{pgfscope}%
\pgfpathrectangle{\pgfqpoint{7.640588in}{0.566125in}}{\pgfqpoint{5.699255in}{2.685432in}}%
\pgfusepath{clip}%
\pgfsetbuttcap%
\pgfsetroundjoin%
\definecolor{currentfill}{rgb}{0.000000,0.000000,0.000000}%
\pgfsetfillcolor{currentfill}%
\pgfsetlinewidth{1.003750pt}%
\definecolor{currentstroke}{rgb}{0.000000,0.000000,0.000000}%
\pgfsetstrokecolor{currentstroke}%
\pgfsetdash{}{0pt}%
\pgfpathmoveto{\pgfqpoint{10.263541in}{1.178602in}}%
\pgfpathcurveto{\pgfqpoint{10.274591in}{1.178602in}}{\pgfqpoint{10.285190in}{1.182992in}}{\pgfqpoint{10.293004in}{1.190806in}}%
\pgfpathcurveto{\pgfqpoint{10.300818in}{1.198619in}}{\pgfqpoint{10.305208in}{1.209218in}}{\pgfqpoint{10.305208in}{1.220268in}}%
\pgfpathcurveto{\pgfqpoint{10.305208in}{1.231318in}}{\pgfqpoint{10.300818in}{1.241917in}}{\pgfqpoint{10.293004in}{1.249731in}}%
\pgfpathcurveto{\pgfqpoint{10.285190in}{1.257545in}}{\pgfqpoint{10.274591in}{1.261935in}}{\pgfqpoint{10.263541in}{1.261935in}}%
\pgfpathcurveto{\pgfqpoint{10.252491in}{1.261935in}}{\pgfqpoint{10.241892in}{1.257545in}}{\pgfqpoint{10.234078in}{1.249731in}}%
\pgfpathcurveto{\pgfqpoint{10.226265in}{1.241917in}}{\pgfqpoint{10.221874in}{1.231318in}}{\pgfqpoint{10.221874in}{1.220268in}}%
\pgfpathcurveto{\pgfqpoint{10.221874in}{1.209218in}}{\pgfqpoint{10.226265in}{1.198619in}}{\pgfqpoint{10.234078in}{1.190806in}}%
\pgfpathcurveto{\pgfqpoint{10.241892in}{1.182992in}}{\pgfqpoint{10.252491in}{1.178602in}}{\pgfqpoint{10.263541in}{1.178602in}}%
\pgfpathclose%
\pgfusepath{stroke,fill}%
\end{pgfscope}%
\begin{pgfscope}%
\pgfpathrectangle{\pgfqpoint{7.640588in}{0.566125in}}{\pgfqpoint{5.699255in}{2.685432in}}%
\pgfusepath{clip}%
\pgfsetbuttcap%
\pgfsetroundjoin%
\definecolor{currentfill}{rgb}{0.000000,0.000000,0.000000}%
\pgfsetfillcolor{currentfill}%
\pgfsetlinewidth{1.003750pt}%
\definecolor{currentstroke}{rgb}{0.000000,0.000000,0.000000}%
\pgfsetstrokecolor{currentstroke}%
\pgfsetdash{}{0pt}%
\pgfpathmoveto{\pgfqpoint{10.166395in}{1.601135in}}%
\pgfpathcurveto{\pgfqpoint{10.177445in}{1.601135in}}{\pgfqpoint{10.188044in}{1.605525in}}{\pgfqpoint{10.195858in}{1.613339in}}%
\pgfpathcurveto{\pgfqpoint{10.203671in}{1.621152in}}{\pgfqpoint{10.208061in}{1.631751in}}{\pgfqpoint{10.208061in}{1.642801in}}%
\pgfpathcurveto{\pgfqpoint{10.208061in}{1.653851in}}{\pgfqpoint{10.203671in}{1.664451in}}{\pgfqpoint{10.195858in}{1.672264in}}%
\pgfpathcurveto{\pgfqpoint{10.188044in}{1.680078in}}{\pgfqpoint{10.177445in}{1.684468in}}{\pgfqpoint{10.166395in}{1.684468in}}%
\pgfpathcurveto{\pgfqpoint{10.155345in}{1.684468in}}{\pgfqpoint{10.144746in}{1.680078in}}{\pgfqpoint{10.136932in}{1.672264in}}%
\pgfpathcurveto{\pgfqpoint{10.129118in}{1.664451in}}{\pgfqpoint{10.124728in}{1.653851in}}{\pgfqpoint{10.124728in}{1.642801in}}%
\pgfpathcurveto{\pgfqpoint{10.124728in}{1.631751in}}{\pgfqpoint{10.129118in}{1.621152in}}{\pgfqpoint{10.136932in}{1.613339in}}%
\pgfpathcurveto{\pgfqpoint{10.144746in}{1.605525in}}{\pgfqpoint{10.155345in}{1.601135in}}{\pgfqpoint{10.166395in}{1.601135in}}%
\pgfpathclose%
\pgfusepath{stroke,fill}%
\end{pgfscope}%
\begin{pgfscope}%
\pgfpathrectangle{\pgfqpoint{7.640588in}{0.566125in}}{\pgfqpoint{5.699255in}{2.685432in}}%
\pgfusepath{clip}%
\pgfsetbuttcap%
\pgfsetroundjoin%
\definecolor{currentfill}{rgb}{0.000000,0.000000,0.000000}%
\pgfsetfillcolor{currentfill}%
\pgfsetlinewidth{1.003750pt}%
\definecolor{currentstroke}{rgb}{0.000000,0.000000,0.000000}%
\pgfsetstrokecolor{currentstroke}%
\pgfsetdash{}{0pt}%
\pgfpathmoveto{\pgfqpoint{10.198777in}{1.695031in}}%
\pgfpathcurveto{\pgfqpoint{10.209827in}{1.695031in}}{\pgfqpoint{10.220426in}{1.699421in}}{\pgfqpoint{10.228240in}{1.707235in}}%
\pgfpathcurveto{\pgfqpoint{10.236053in}{1.715048in}}{\pgfqpoint{10.240444in}{1.725647in}}{\pgfqpoint{10.240444in}{1.736698in}}%
\pgfpathcurveto{\pgfqpoint{10.240444in}{1.747748in}}{\pgfqpoint{10.236053in}{1.758347in}}{\pgfqpoint{10.228240in}{1.766160in}}%
\pgfpathcurveto{\pgfqpoint{10.220426in}{1.773974in}}{\pgfqpoint{10.209827in}{1.778364in}}{\pgfqpoint{10.198777in}{1.778364in}}%
\pgfpathcurveto{\pgfqpoint{10.187727in}{1.778364in}}{\pgfqpoint{10.177128in}{1.773974in}}{\pgfqpoint{10.169314in}{1.766160in}}%
\pgfpathcurveto{\pgfqpoint{10.161500in}{1.758347in}}{\pgfqpoint{10.157110in}{1.747748in}}{\pgfqpoint{10.157110in}{1.736698in}}%
\pgfpathcurveto{\pgfqpoint{10.157110in}{1.725647in}}{\pgfqpoint{10.161500in}{1.715048in}}{\pgfqpoint{10.169314in}{1.707235in}}%
\pgfpathcurveto{\pgfqpoint{10.177128in}{1.699421in}}{\pgfqpoint{10.187727in}{1.695031in}}{\pgfqpoint{10.198777in}{1.695031in}}%
\pgfpathclose%
\pgfusepath{stroke,fill}%
\end{pgfscope}%
\begin{pgfscope}%
\pgfpathrectangle{\pgfqpoint{7.640588in}{0.566125in}}{\pgfqpoint{5.699255in}{2.685432in}}%
\pgfusepath{clip}%
\pgfsetbuttcap%
\pgfsetroundjoin%
\definecolor{currentfill}{rgb}{0.000000,0.000000,0.000000}%
\pgfsetfillcolor{currentfill}%
\pgfsetlinewidth{1.003750pt}%
\definecolor{currentstroke}{rgb}{0.000000,0.000000,0.000000}%
\pgfsetstrokecolor{currentstroke}%
\pgfsetdash{}{0pt}%
\pgfpathmoveto{\pgfqpoint{9.615898in}{1.835875in}}%
\pgfpathcurveto{\pgfqpoint{9.626949in}{1.835875in}}{\pgfqpoint{9.637548in}{1.840266in}}{\pgfqpoint{9.645361in}{1.848079in}}%
\pgfpathcurveto{\pgfqpoint{9.653175in}{1.855893in}}{\pgfqpoint{9.657565in}{1.866492in}}{\pgfqpoint{9.657565in}{1.877542in}}%
\pgfpathcurveto{\pgfqpoint{9.657565in}{1.888592in}}{\pgfqpoint{9.653175in}{1.899191in}}{\pgfqpoint{9.645361in}{1.907005in}}%
\pgfpathcurveto{\pgfqpoint{9.637548in}{1.914818in}}{\pgfqpoint{9.626949in}{1.919209in}}{\pgfqpoint{9.615898in}{1.919209in}}%
\pgfpathcurveto{\pgfqpoint{9.604848in}{1.919209in}}{\pgfqpoint{9.594249in}{1.914818in}}{\pgfqpoint{9.586436in}{1.907005in}}%
\pgfpathcurveto{\pgfqpoint{9.578622in}{1.899191in}}{\pgfqpoint{9.574232in}{1.888592in}}{\pgfqpoint{9.574232in}{1.877542in}}%
\pgfpathcurveto{\pgfqpoint{9.574232in}{1.866492in}}{\pgfqpoint{9.578622in}{1.855893in}}{\pgfqpoint{9.586436in}{1.848079in}}%
\pgfpathcurveto{\pgfqpoint{9.594249in}{1.840266in}}{\pgfqpoint{9.604848in}{1.835875in}}{\pgfqpoint{9.615898in}{1.835875in}}%
\pgfpathclose%
\pgfusepath{stroke,fill}%
\end{pgfscope}%
\begin{pgfscope}%
\pgfpathrectangle{\pgfqpoint{7.640588in}{0.566125in}}{\pgfqpoint{5.699255in}{2.685432in}}%
\pgfusepath{clip}%
\pgfsetbuttcap%
\pgfsetroundjoin%
\definecolor{currentfill}{rgb}{0.000000,0.000000,0.000000}%
\pgfsetfillcolor{currentfill}%
\pgfsetlinewidth{1.003750pt}%
\definecolor{currentstroke}{rgb}{0.000000,0.000000,0.000000}%
\pgfsetstrokecolor{currentstroke}%
\pgfsetdash{}{0pt}%
\pgfpathmoveto{\pgfqpoint{9.810191in}{1.835875in}}%
\pgfpathcurveto{\pgfqpoint{9.821241in}{1.835875in}}{\pgfqpoint{9.831840in}{1.840266in}}{\pgfqpoint{9.839654in}{1.848079in}}%
\pgfpathcurveto{\pgfqpoint{9.847468in}{1.855893in}}{\pgfqpoint{9.851858in}{1.866492in}}{\pgfqpoint{9.851858in}{1.877542in}}%
\pgfpathcurveto{\pgfqpoint{9.851858in}{1.888592in}}{\pgfqpoint{9.847468in}{1.899191in}}{\pgfqpoint{9.839654in}{1.907005in}}%
\pgfpathcurveto{\pgfqpoint{9.831840in}{1.914818in}}{\pgfqpoint{9.821241in}{1.919209in}}{\pgfqpoint{9.810191in}{1.919209in}}%
\pgfpathcurveto{\pgfqpoint{9.799141in}{1.919209in}}{\pgfqpoint{9.788542in}{1.914818in}}{\pgfqpoint{9.780728in}{1.907005in}}%
\pgfpathcurveto{\pgfqpoint{9.772915in}{1.899191in}}{\pgfqpoint{9.768525in}{1.888592in}}{\pgfqpoint{9.768525in}{1.877542in}}%
\pgfpathcurveto{\pgfqpoint{9.768525in}{1.866492in}}{\pgfqpoint{9.772915in}{1.855893in}}{\pgfqpoint{9.780728in}{1.848079in}}%
\pgfpathcurveto{\pgfqpoint{9.788542in}{1.840266in}}{\pgfqpoint{9.799141in}{1.835875in}}{\pgfqpoint{9.810191in}{1.835875in}}%
\pgfpathclose%
\pgfusepath{stroke,fill}%
\end{pgfscope}%
\begin{pgfscope}%
\pgfpathrectangle{\pgfqpoint{7.640588in}{0.566125in}}{\pgfqpoint{5.699255in}{2.685432in}}%
\pgfusepath{clip}%
\pgfsetbuttcap%
\pgfsetroundjoin%
\definecolor{currentfill}{rgb}{0.000000,0.000000,0.000000}%
\pgfsetfillcolor{currentfill}%
\pgfsetlinewidth{1.003750pt}%
\definecolor{currentstroke}{rgb}{0.000000,0.000000,0.000000}%
\pgfsetstrokecolor{currentstroke}%
\pgfsetdash{}{0pt}%
\pgfpathmoveto{\pgfqpoint{10.134013in}{1.867174in}}%
\pgfpathcurveto{\pgfqpoint{10.145063in}{1.867174in}}{\pgfqpoint{10.155662in}{1.871564in}}{\pgfqpoint{10.163475in}{1.879378in}}%
\pgfpathcurveto{\pgfqpoint{10.171289in}{1.887192in}}{\pgfqpoint{10.175679in}{1.897791in}}{\pgfqpoint{10.175679in}{1.908841in}}%
\pgfpathcurveto{\pgfqpoint{10.175679in}{1.919891in}}{\pgfqpoint{10.171289in}{1.930490in}}{\pgfqpoint{10.163475in}{1.938303in}}%
\pgfpathcurveto{\pgfqpoint{10.155662in}{1.946117in}}{\pgfqpoint{10.145063in}{1.950507in}}{\pgfqpoint{10.134013in}{1.950507in}}%
\pgfpathcurveto{\pgfqpoint{10.122962in}{1.950507in}}{\pgfqpoint{10.112363in}{1.946117in}}{\pgfqpoint{10.104550in}{1.938303in}}%
\pgfpathcurveto{\pgfqpoint{10.096736in}{1.930490in}}{\pgfqpoint{10.092346in}{1.919891in}}{\pgfqpoint{10.092346in}{1.908841in}}%
\pgfpathcurveto{\pgfqpoint{10.092346in}{1.897791in}}{\pgfqpoint{10.096736in}{1.887192in}}{\pgfqpoint{10.104550in}{1.879378in}}%
\pgfpathcurveto{\pgfqpoint{10.112363in}{1.871564in}}{\pgfqpoint{10.122962in}{1.867174in}}{\pgfqpoint{10.134013in}{1.867174in}}%
\pgfpathclose%
\pgfusepath{stroke,fill}%
\end{pgfscope}%
\begin{pgfscope}%
\pgfpathrectangle{\pgfqpoint{7.640588in}{0.566125in}}{\pgfqpoint{5.699255in}{2.685432in}}%
\pgfusepath{clip}%
\pgfsetbuttcap%
\pgfsetroundjoin%
\definecolor{currentfill}{rgb}{0.000000,0.000000,0.000000}%
\pgfsetfillcolor{currentfill}%
\pgfsetlinewidth{1.003750pt}%
\definecolor{currentstroke}{rgb}{0.000000,0.000000,0.000000}%
\pgfsetstrokecolor{currentstroke}%
\pgfsetdash{}{0pt}%
\pgfpathmoveto{\pgfqpoint{10.101630in}{1.992369in}}%
\pgfpathcurveto{\pgfqpoint{10.112681in}{1.992369in}}{\pgfqpoint{10.123280in}{1.996759in}}{\pgfqpoint{10.131093in}{2.004573in}}%
\pgfpathcurveto{\pgfqpoint{10.138907in}{2.012386in}}{\pgfqpoint{10.143297in}{2.022986in}}{\pgfqpoint{10.143297in}{2.034036in}}%
\pgfpathcurveto{\pgfqpoint{10.143297in}{2.045086in}}{\pgfqpoint{10.138907in}{2.055685in}}{\pgfqpoint{10.131093in}{2.063498in}}%
\pgfpathcurveto{\pgfqpoint{10.123280in}{2.071312in}}{\pgfqpoint{10.112681in}{2.075702in}}{\pgfqpoint{10.101630in}{2.075702in}}%
\pgfpathcurveto{\pgfqpoint{10.090580in}{2.075702in}}{\pgfqpoint{10.079981in}{2.071312in}}{\pgfqpoint{10.072168in}{2.063498in}}%
\pgfpathcurveto{\pgfqpoint{10.064354in}{2.055685in}}{\pgfqpoint{10.059964in}{2.045086in}}{\pgfqpoint{10.059964in}{2.034036in}}%
\pgfpathcurveto{\pgfqpoint{10.059964in}{2.022986in}}{\pgfqpoint{10.064354in}{2.012386in}}{\pgfqpoint{10.072168in}{2.004573in}}%
\pgfpathcurveto{\pgfqpoint{10.079981in}{1.996759in}}{\pgfqpoint{10.090580in}{1.992369in}}{\pgfqpoint{10.101630in}{1.992369in}}%
\pgfpathclose%
\pgfusepath{stroke,fill}%
\end{pgfscope}%
\begin{pgfscope}%
\pgfpathrectangle{\pgfqpoint{7.640588in}{0.566125in}}{\pgfqpoint{5.699255in}{2.685432in}}%
\pgfusepath{clip}%
\pgfsetbuttcap%
\pgfsetroundjoin%
\definecolor{currentfill}{rgb}{0.000000,0.000000,0.000000}%
\pgfsetfillcolor{currentfill}%
\pgfsetlinewidth{1.003750pt}%
\definecolor{currentstroke}{rgb}{0.000000,0.000000,0.000000}%
\pgfsetstrokecolor{currentstroke}%
\pgfsetdash{}{0pt}%
\pgfpathmoveto{\pgfqpoint{10.101630in}{2.054966in}}%
\pgfpathcurveto{\pgfqpoint{10.112681in}{2.054966in}}{\pgfqpoint{10.123280in}{2.059357in}}{\pgfqpoint{10.131093in}{2.067170in}}%
\pgfpathcurveto{\pgfqpoint{10.138907in}{2.074984in}}{\pgfqpoint{10.143297in}{2.085583in}}{\pgfqpoint{10.143297in}{2.096633in}}%
\pgfpathcurveto{\pgfqpoint{10.143297in}{2.107683in}}{\pgfqpoint{10.138907in}{2.118282in}}{\pgfqpoint{10.131093in}{2.126096in}}%
\pgfpathcurveto{\pgfqpoint{10.123280in}{2.133910in}}{\pgfqpoint{10.112681in}{2.138300in}}{\pgfqpoint{10.101630in}{2.138300in}}%
\pgfpathcurveto{\pgfqpoint{10.090580in}{2.138300in}}{\pgfqpoint{10.079981in}{2.133910in}}{\pgfqpoint{10.072168in}{2.126096in}}%
\pgfpathcurveto{\pgfqpoint{10.064354in}{2.118282in}}{\pgfqpoint{10.059964in}{2.107683in}}{\pgfqpoint{10.059964in}{2.096633in}}%
\pgfpathcurveto{\pgfqpoint{10.059964in}{2.085583in}}{\pgfqpoint{10.064354in}{2.074984in}}{\pgfqpoint{10.072168in}{2.067170in}}%
\pgfpathcurveto{\pgfqpoint{10.079981in}{2.059357in}}{\pgfqpoint{10.090580in}{2.054966in}}{\pgfqpoint{10.101630in}{2.054966in}}%
\pgfpathclose%
\pgfusepath{stroke,fill}%
\end{pgfscope}%
\begin{pgfscope}%
\pgfpathrectangle{\pgfqpoint{7.640588in}{0.566125in}}{\pgfqpoint{5.699255in}{2.685432in}}%
\pgfusepath{clip}%
\pgfsetbuttcap%
\pgfsetroundjoin%
\definecolor{currentfill}{rgb}{0.000000,0.000000,0.000000}%
\pgfsetfillcolor{currentfill}%
\pgfsetlinewidth{1.003750pt}%
\definecolor{currentstroke}{rgb}{0.000000,0.000000,0.000000}%
\pgfsetstrokecolor{currentstroke}%
\pgfsetdash{}{0pt}%
\pgfpathmoveto{\pgfqpoint{10.004484in}{2.258408in}}%
\pgfpathcurveto{\pgfqpoint{10.015534in}{2.258408in}}{\pgfqpoint{10.026133in}{2.262799in}}{\pgfqpoint{10.033947in}{2.270612in}}%
\pgfpathcurveto{\pgfqpoint{10.041760in}{2.278426in}}{\pgfqpoint{10.046151in}{2.289025in}}{\pgfqpoint{10.046151in}{2.300075in}}%
\pgfpathcurveto{\pgfqpoint{10.046151in}{2.311125in}}{\pgfqpoint{10.041760in}{2.321724in}}{\pgfqpoint{10.033947in}{2.329538in}}%
\pgfpathcurveto{\pgfqpoint{10.026133in}{2.337351in}}{\pgfqpoint{10.015534in}{2.341742in}}{\pgfqpoint{10.004484in}{2.341742in}}%
\pgfpathcurveto{\pgfqpoint{9.993434in}{2.341742in}}{\pgfqpoint{9.982835in}{2.337351in}}{\pgfqpoint{9.975021in}{2.329538in}}%
\pgfpathcurveto{\pgfqpoint{9.967208in}{2.321724in}}{\pgfqpoint{9.962817in}{2.311125in}}{\pgfqpoint{9.962817in}{2.300075in}}%
\pgfpathcurveto{\pgfqpoint{9.962817in}{2.289025in}}{\pgfqpoint{9.967208in}{2.278426in}}{\pgfqpoint{9.975021in}{2.270612in}}%
\pgfpathcurveto{\pgfqpoint{9.982835in}{2.262799in}}{\pgfqpoint{9.993434in}{2.258408in}}{\pgfqpoint{10.004484in}{2.258408in}}%
\pgfpathclose%
\pgfusepath{stroke,fill}%
\end{pgfscope}%
\begin{pgfscope}%
\pgfpathrectangle{\pgfqpoint{7.640588in}{0.566125in}}{\pgfqpoint{5.699255in}{2.685432in}}%
\pgfusepath{clip}%
\pgfsetbuttcap%
\pgfsetroundjoin%
\definecolor{currentfill}{rgb}{0.000000,0.000000,0.000000}%
\pgfsetfillcolor{currentfill}%
\pgfsetlinewidth{1.003750pt}%
\definecolor{currentstroke}{rgb}{0.000000,0.000000,0.000000}%
\pgfsetstrokecolor{currentstroke}%
\pgfsetdash{}{0pt}%
\pgfpathmoveto{\pgfqpoint{8.935874in}{2.587045in}}%
\pgfpathcurveto{\pgfqpoint{8.946924in}{2.587045in}}{\pgfqpoint{8.957523in}{2.591435in}}{\pgfqpoint{8.965336in}{2.599249in}}%
\pgfpathcurveto{\pgfqpoint{8.973150in}{2.607063in}}{\pgfqpoint{8.977540in}{2.617662in}}{\pgfqpoint{8.977540in}{2.628712in}}%
\pgfpathcurveto{\pgfqpoint{8.977540in}{2.639762in}}{\pgfqpoint{8.973150in}{2.650361in}}{\pgfqpoint{8.965336in}{2.658175in}}%
\pgfpathcurveto{\pgfqpoint{8.957523in}{2.665988in}}{\pgfqpoint{8.946924in}{2.670378in}}{\pgfqpoint{8.935874in}{2.670378in}}%
\pgfpathcurveto{\pgfqpoint{8.924824in}{2.670378in}}{\pgfqpoint{8.914225in}{2.665988in}}{\pgfqpoint{8.906411in}{2.658175in}}%
\pgfpathcurveto{\pgfqpoint{8.898597in}{2.650361in}}{\pgfqpoint{8.894207in}{2.639762in}}{\pgfqpoint{8.894207in}{2.628712in}}%
\pgfpathcurveto{\pgfqpoint{8.894207in}{2.617662in}}{\pgfqpoint{8.898597in}{2.607063in}}{\pgfqpoint{8.906411in}{2.599249in}}%
\pgfpathcurveto{\pgfqpoint{8.914225in}{2.591435in}}{\pgfqpoint{8.924824in}{2.587045in}}{\pgfqpoint{8.935874in}{2.587045in}}%
\pgfpathclose%
\pgfusepath{stroke,fill}%
\end{pgfscope}%
\begin{pgfscope}%
\pgfpathrectangle{\pgfqpoint{7.640588in}{0.566125in}}{\pgfqpoint{5.699255in}{2.685432in}}%
\pgfusepath{clip}%
\pgfsetbuttcap%
\pgfsetroundjoin%
\definecolor{currentfill}{rgb}{0.000000,0.000000,0.000000}%
\pgfsetfillcolor{currentfill}%
\pgfsetlinewidth{1.003750pt}%
\definecolor{currentstroke}{rgb}{0.000000,0.000000,0.000000}%
\pgfsetstrokecolor{currentstroke}%
\pgfsetdash{}{0pt}%
\pgfpathmoveto{\pgfqpoint{8.871109in}{2.900032in}}%
\pgfpathcurveto{\pgfqpoint{8.882160in}{2.900032in}}{\pgfqpoint{8.892759in}{2.904423in}}{\pgfqpoint{8.900572in}{2.912236in}}%
\pgfpathcurveto{\pgfqpoint{8.908386in}{2.920050in}}{\pgfqpoint{8.912776in}{2.930649in}}{\pgfqpoint{8.912776in}{2.941699in}}%
\pgfpathcurveto{\pgfqpoint{8.912776in}{2.952749in}}{\pgfqpoint{8.908386in}{2.963348in}}{\pgfqpoint{8.900572in}{2.971162in}}%
\pgfpathcurveto{\pgfqpoint{8.892759in}{2.978976in}}{\pgfqpoint{8.882160in}{2.983366in}}{\pgfqpoint{8.871109in}{2.983366in}}%
\pgfpathcurveto{\pgfqpoint{8.860059in}{2.983366in}}{\pgfqpoint{8.849460in}{2.978976in}}{\pgfqpoint{8.841647in}{2.971162in}}%
\pgfpathcurveto{\pgfqpoint{8.833833in}{2.963348in}}{\pgfqpoint{8.829443in}{2.952749in}}{\pgfqpoint{8.829443in}{2.941699in}}%
\pgfpathcurveto{\pgfqpoint{8.829443in}{2.930649in}}{\pgfqpoint{8.833833in}{2.920050in}}{\pgfqpoint{8.841647in}{2.912236in}}%
\pgfpathcurveto{\pgfqpoint{8.849460in}{2.904423in}}{\pgfqpoint{8.860059in}{2.900032in}}{\pgfqpoint{8.871109in}{2.900032in}}%
\pgfpathclose%
\pgfusepath{stroke,fill}%
\end{pgfscope}%
\begin{pgfscope}%
\pgfpathrectangle{\pgfqpoint{7.640588in}{0.566125in}}{\pgfqpoint{5.699255in}{2.685432in}}%
\pgfusepath{clip}%
\pgfsetbuttcap%
\pgfsetroundjoin%
\definecolor{currentfill}{rgb}{0.000000,0.000000,0.000000}%
\pgfsetfillcolor{currentfill}%
\pgfsetlinewidth{1.003750pt}%
\definecolor{currentstroke}{rgb}{0.000000,0.000000,0.000000}%
\pgfsetstrokecolor{currentstroke}%
\pgfsetdash{}{0pt}%
\pgfpathmoveto{\pgfqpoint{8.871109in}{2.946981in}}%
\pgfpathcurveto{\pgfqpoint{8.882160in}{2.946981in}}{\pgfqpoint{8.892759in}{2.951371in}}{\pgfqpoint{8.900572in}{2.959184in}}%
\pgfpathcurveto{\pgfqpoint{8.908386in}{2.966998in}}{\pgfqpoint{8.912776in}{2.977597in}}{\pgfqpoint{8.912776in}{2.988647in}}%
\pgfpathcurveto{\pgfqpoint{8.912776in}{2.999697in}}{\pgfqpoint{8.908386in}{3.010296in}}{\pgfqpoint{8.900572in}{3.018110in}}%
\pgfpathcurveto{\pgfqpoint{8.892759in}{3.025924in}}{\pgfqpoint{8.882160in}{3.030314in}}{\pgfqpoint{8.871109in}{3.030314in}}%
\pgfpathcurveto{\pgfqpoint{8.860059in}{3.030314in}}{\pgfqpoint{8.849460in}{3.025924in}}{\pgfqpoint{8.841647in}{3.018110in}}%
\pgfpathcurveto{\pgfqpoint{8.833833in}{3.010296in}}{\pgfqpoint{8.829443in}{2.999697in}}{\pgfqpoint{8.829443in}{2.988647in}}%
\pgfpathcurveto{\pgfqpoint{8.829443in}{2.977597in}}{\pgfqpoint{8.833833in}{2.966998in}}{\pgfqpoint{8.841647in}{2.959184in}}%
\pgfpathcurveto{\pgfqpoint{8.849460in}{2.951371in}}{\pgfqpoint{8.860059in}{2.946981in}}{\pgfqpoint{8.871109in}{2.946981in}}%
\pgfpathclose%
\pgfusepath{stroke,fill}%
\end{pgfscope}%
\begin{pgfscope}%
\pgfpathrectangle{\pgfqpoint{7.640588in}{0.566125in}}{\pgfqpoint{5.699255in}{2.685432in}}%
\pgfusepath{clip}%
\pgfsetbuttcap%
\pgfsetroundjoin%
\definecolor{currentfill}{rgb}{0.000000,0.000000,0.000000}%
\pgfsetfillcolor{currentfill}%
\pgfsetlinewidth{1.003750pt}%
\definecolor{currentstroke}{rgb}{0.000000,0.000000,0.000000}%
\pgfsetstrokecolor{currentstroke}%
\pgfsetdash{}{0pt}%
\pgfpathmoveto{\pgfqpoint{8.773963in}{2.931331in}}%
\pgfpathcurveto{\pgfqpoint{8.785013in}{2.931331in}}{\pgfqpoint{8.795612in}{2.935722in}}{\pgfqpoint{8.803426in}{2.943535in}}%
\pgfpathcurveto{\pgfqpoint{8.811239in}{2.951349in}}{\pgfqpoint{8.815630in}{2.961948in}}{\pgfqpoint{8.815630in}{2.972998in}}%
\pgfpathcurveto{\pgfqpoint{8.815630in}{2.984048in}}{\pgfqpoint{8.811239in}{2.994647in}}{\pgfqpoint{8.803426in}{3.002461in}}%
\pgfpathcurveto{\pgfqpoint{8.795612in}{3.010274in}}{\pgfqpoint{8.785013in}{3.014665in}}{\pgfqpoint{8.773963in}{3.014665in}}%
\pgfpathcurveto{\pgfqpoint{8.762913in}{3.014665in}}{\pgfqpoint{8.752314in}{3.010274in}}{\pgfqpoint{8.744500in}{3.002461in}}%
\pgfpathcurveto{\pgfqpoint{8.736687in}{2.994647in}}{\pgfqpoint{8.732296in}{2.984048in}}{\pgfqpoint{8.732296in}{2.972998in}}%
\pgfpathcurveto{\pgfqpoint{8.732296in}{2.961948in}}{\pgfqpoint{8.736687in}{2.951349in}}{\pgfqpoint{8.744500in}{2.943535in}}%
\pgfpathcurveto{\pgfqpoint{8.752314in}{2.935722in}}{\pgfqpoint{8.762913in}{2.931331in}}{\pgfqpoint{8.773963in}{2.931331in}}%
\pgfpathclose%
\pgfusepath{stroke,fill}%
\end{pgfscope}%
\begin{pgfscope}%
\pgfpathrectangle{\pgfqpoint{7.640588in}{0.566125in}}{\pgfqpoint{5.699255in}{2.685432in}}%
\pgfusepath{clip}%
\pgfsetbuttcap%
\pgfsetroundjoin%
\definecolor{currentfill}{rgb}{0.000000,0.000000,0.000000}%
\pgfsetfillcolor{currentfill}%
\pgfsetlinewidth{1.003750pt}%
\definecolor{currentstroke}{rgb}{0.000000,0.000000,0.000000}%
\pgfsetstrokecolor{currentstroke}%
\pgfsetdash{}{0pt}%
\pgfpathmoveto{\pgfqpoint{8.773963in}{3.072176in}}%
\pgfpathcurveto{\pgfqpoint{8.785013in}{3.072176in}}{\pgfqpoint{8.795612in}{3.076566in}}{\pgfqpoint{8.803426in}{3.084379in}}%
\pgfpathcurveto{\pgfqpoint{8.811239in}{3.092193in}}{\pgfqpoint{8.815630in}{3.102792in}}{\pgfqpoint{8.815630in}{3.113842in}}%
\pgfpathcurveto{\pgfqpoint{8.815630in}{3.124892in}}{\pgfqpoint{8.811239in}{3.135491in}}{\pgfqpoint{8.803426in}{3.143305in}}%
\pgfpathcurveto{\pgfqpoint{8.795612in}{3.151119in}}{\pgfqpoint{8.785013in}{3.155509in}}{\pgfqpoint{8.773963in}{3.155509in}}%
\pgfpathcurveto{\pgfqpoint{8.762913in}{3.155509in}}{\pgfqpoint{8.752314in}{3.151119in}}{\pgfqpoint{8.744500in}{3.143305in}}%
\pgfpathcurveto{\pgfqpoint{8.736687in}{3.135491in}}{\pgfqpoint{8.732296in}{3.124892in}}{\pgfqpoint{8.732296in}{3.113842in}}%
\pgfpathcurveto{\pgfqpoint{8.732296in}{3.102792in}}{\pgfqpoint{8.736687in}{3.092193in}}{\pgfqpoint{8.744500in}{3.084379in}}%
\pgfpathcurveto{\pgfqpoint{8.752314in}{3.076566in}}{\pgfqpoint{8.762913in}{3.072176in}}{\pgfqpoint{8.773963in}{3.072176in}}%
\pgfpathclose%
\pgfusepath{stroke,fill}%
\end{pgfscope}%
\begin{pgfscope}%
\pgfpathrectangle{\pgfqpoint{7.640588in}{0.566125in}}{\pgfqpoint{5.699255in}{2.685432in}}%
\pgfusepath{clip}%
\pgfsetbuttcap%
\pgfsetroundjoin%
\definecolor{currentfill}{rgb}{0.000000,0.000000,0.000000}%
\pgfsetfillcolor{currentfill}%
\pgfsetlinewidth{1.003750pt}%
\definecolor{currentstroke}{rgb}{0.000000,0.000000,0.000000}%
\pgfsetstrokecolor{currentstroke}%
\pgfsetdash{}{0pt}%
\pgfpathmoveto{\pgfqpoint{9.162549in}{3.056526in}}%
\pgfpathcurveto{\pgfqpoint{9.173599in}{3.056526in}}{\pgfqpoint{9.184198in}{3.060916in}}{\pgfqpoint{9.192011in}{3.068730in}}%
\pgfpathcurveto{\pgfqpoint{9.199825in}{3.076544in}}{\pgfqpoint{9.204215in}{3.087143in}}{\pgfqpoint{9.204215in}{3.098193in}}%
\pgfpathcurveto{\pgfqpoint{9.204215in}{3.109243in}}{\pgfqpoint{9.199825in}{3.119842in}}{\pgfqpoint{9.192011in}{3.127656in}}%
\pgfpathcurveto{\pgfqpoint{9.184198in}{3.135469in}}{\pgfqpoint{9.173599in}{3.139860in}}{\pgfqpoint{9.162549in}{3.139860in}}%
\pgfpathcurveto{\pgfqpoint{9.151498in}{3.139860in}}{\pgfqpoint{9.140899in}{3.135469in}}{\pgfqpoint{9.133086in}{3.127656in}}%
\pgfpathcurveto{\pgfqpoint{9.125272in}{3.119842in}}{\pgfqpoint{9.120882in}{3.109243in}}{\pgfqpoint{9.120882in}{3.098193in}}%
\pgfpathcurveto{\pgfqpoint{9.120882in}{3.087143in}}{\pgfqpoint{9.125272in}{3.076544in}}{\pgfqpoint{9.133086in}{3.068730in}}%
\pgfpathcurveto{\pgfqpoint{9.140899in}{3.060916in}}{\pgfqpoint{9.151498in}{3.056526in}}{\pgfqpoint{9.162549in}{3.056526in}}%
\pgfpathclose%
\pgfusepath{stroke,fill}%
\end{pgfscope}%
\begin{pgfscope}%
\pgfpathrectangle{\pgfqpoint{7.640588in}{0.566125in}}{\pgfqpoint{5.699255in}{2.685432in}}%
\pgfusepath{clip}%
\pgfsetbuttcap%
\pgfsetroundjoin%
\definecolor{currentfill}{rgb}{0.000000,0.000000,0.000000}%
\pgfsetfillcolor{currentfill}%
\pgfsetlinewidth{1.003750pt}%
\definecolor{currentstroke}{rgb}{0.000000,0.000000,0.000000}%
\pgfsetstrokecolor{currentstroke}%
\pgfsetdash{}{0pt}%
\pgfpathmoveto{\pgfqpoint{10.295923in}{2.367954in}}%
\pgfpathcurveto{\pgfqpoint{10.306973in}{2.367954in}}{\pgfqpoint{10.317572in}{2.372344in}}{\pgfqpoint{10.325386in}{2.380158in}}%
\pgfpathcurveto{\pgfqpoint{10.333200in}{2.387971in}}{\pgfqpoint{10.337590in}{2.398570in}}{\pgfqpoint{10.337590in}{2.409621in}}%
\pgfpathcurveto{\pgfqpoint{10.337590in}{2.420671in}}{\pgfqpoint{10.333200in}{2.431270in}}{\pgfqpoint{10.325386in}{2.439083in}}%
\pgfpathcurveto{\pgfqpoint{10.317572in}{2.446897in}}{\pgfqpoint{10.306973in}{2.451287in}}{\pgfqpoint{10.295923in}{2.451287in}}%
\pgfpathcurveto{\pgfqpoint{10.284873in}{2.451287in}}{\pgfqpoint{10.274274in}{2.446897in}}{\pgfqpoint{10.266460in}{2.439083in}}%
\pgfpathcurveto{\pgfqpoint{10.258647in}{2.431270in}}{\pgfqpoint{10.254257in}{2.420671in}}{\pgfqpoint{10.254257in}{2.409621in}}%
\pgfpathcurveto{\pgfqpoint{10.254257in}{2.398570in}}{\pgfqpoint{10.258647in}{2.387971in}}{\pgfqpoint{10.266460in}{2.380158in}}%
\pgfpathcurveto{\pgfqpoint{10.274274in}{2.372344in}}{\pgfqpoint{10.284873in}{2.367954in}}{\pgfqpoint{10.295923in}{2.367954in}}%
\pgfpathclose%
\pgfusepath{stroke,fill}%
\end{pgfscope}%
\begin{pgfscope}%
\pgfpathrectangle{\pgfqpoint{7.640588in}{0.566125in}}{\pgfqpoint{5.699255in}{2.685432in}}%
\pgfusepath{clip}%
\pgfsetbuttcap%
\pgfsetroundjoin%
\definecolor{currentfill}{rgb}{0.000000,0.000000,0.000000}%
\pgfsetfillcolor{currentfill}%
\pgfsetlinewidth{1.003750pt}%
\definecolor{currentstroke}{rgb}{0.000000,0.000000,0.000000}%
\pgfsetstrokecolor{currentstroke}%
\pgfsetdash{}{0pt}%
\pgfpathmoveto{\pgfqpoint{10.814037in}{2.414902in}}%
\pgfpathcurveto{\pgfqpoint{10.825088in}{2.414902in}}{\pgfqpoint{10.835687in}{2.419292in}}{\pgfqpoint{10.843500in}{2.427106in}}%
\pgfpathcurveto{\pgfqpoint{10.851314in}{2.434919in}}{\pgfqpoint{10.855704in}{2.445519in}}{\pgfqpoint{10.855704in}{2.456569in}}%
\pgfpathcurveto{\pgfqpoint{10.855704in}{2.467619in}}{\pgfqpoint{10.851314in}{2.478218in}}{\pgfqpoint{10.843500in}{2.486031in}}%
\pgfpathcurveto{\pgfqpoint{10.835687in}{2.493845in}}{\pgfqpoint{10.825088in}{2.498235in}}{\pgfqpoint{10.814037in}{2.498235in}}%
\pgfpathcurveto{\pgfqpoint{10.802987in}{2.498235in}}{\pgfqpoint{10.792388in}{2.493845in}}{\pgfqpoint{10.784575in}{2.486031in}}%
\pgfpathcurveto{\pgfqpoint{10.776761in}{2.478218in}}{\pgfqpoint{10.772371in}{2.467619in}}{\pgfqpoint{10.772371in}{2.456569in}}%
\pgfpathcurveto{\pgfqpoint{10.772371in}{2.445519in}}{\pgfqpoint{10.776761in}{2.434919in}}{\pgfqpoint{10.784575in}{2.427106in}}%
\pgfpathcurveto{\pgfqpoint{10.792388in}{2.419292in}}{\pgfqpoint{10.802987in}{2.414902in}}{\pgfqpoint{10.814037in}{2.414902in}}%
\pgfpathclose%
\pgfusepath{stroke,fill}%
\end{pgfscope}%
\begin{pgfscope}%
\pgfpathrectangle{\pgfqpoint{7.640588in}{0.566125in}}{\pgfqpoint{5.699255in}{2.685432in}}%
\pgfusepath{clip}%
\pgfsetbuttcap%
\pgfsetroundjoin%
\definecolor{currentfill}{rgb}{0.000000,0.000000,0.000000}%
\pgfsetfillcolor{currentfill}%
\pgfsetlinewidth{1.003750pt}%
\definecolor{currentstroke}{rgb}{0.000000,0.000000,0.000000}%
\pgfsetstrokecolor{currentstroke}%
\pgfsetdash{}{0pt}%
\pgfpathmoveto{\pgfqpoint{11.105477in}{2.148863in}}%
\pgfpathcurveto{\pgfqpoint{11.116527in}{2.148863in}}{\pgfqpoint{11.127126in}{2.153253in}}{\pgfqpoint{11.134939in}{2.161067in}}%
\pgfpathcurveto{\pgfqpoint{11.142753in}{2.168880in}}{\pgfqpoint{11.147143in}{2.179479in}}{\pgfqpoint{11.147143in}{2.190529in}}%
\pgfpathcurveto{\pgfqpoint{11.147143in}{2.201579in}}{\pgfqpoint{11.142753in}{2.212179in}}{\pgfqpoint{11.134939in}{2.219992in}}%
\pgfpathcurveto{\pgfqpoint{11.127126in}{2.227806in}}{\pgfqpoint{11.116527in}{2.232196in}}{\pgfqpoint{11.105477in}{2.232196in}}%
\pgfpathcurveto{\pgfqpoint{11.094426in}{2.232196in}}{\pgfqpoint{11.083827in}{2.227806in}}{\pgfqpoint{11.076014in}{2.219992in}}%
\pgfpathcurveto{\pgfqpoint{11.068200in}{2.212179in}}{\pgfqpoint{11.063810in}{2.201579in}}{\pgfqpoint{11.063810in}{2.190529in}}%
\pgfpathcurveto{\pgfqpoint{11.063810in}{2.179479in}}{\pgfqpoint{11.068200in}{2.168880in}}{\pgfqpoint{11.076014in}{2.161067in}}%
\pgfpathcurveto{\pgfqpoint{11.083827in}{2.153253in}}{\pgfqpoint{11.094426in}{2.148863in}}{\pgfqpoint{11.105477in}{2.148863in}}%
\pgfpathclose%
\pgfusepath{stroke,fill}%
\end{pgfscope}%
\begin{pgfscope}%
\pgfpathrectangle{\pgfqpoint{7.640588in}{0.566125in}}{\pgfqpoint{5.699255in}{2.685432in}}%
\pgfusepath{clip}%
\pgfsetbuttcap%
\pgfsetroundjoin%
\definecolor{currentfill}{rgb}{0.000000,0.000000,0.000000}%
\pgfsetfillcolor{currentfill}%
\pgfsetlinewidth{1.003750pt}%
\definecolor{currentstroke}{rgb}{0.000000,0.000000,0.000000}%
\pgfsetstrokecolor{currentstroke}%
\pgfsetdash{}{0pt}%
\pgfpathmoveto{\pgfqpoint{11.170241in}{2.133213in}}%
\pgfpathcurveto{\pgfqpoint{11.181291in}{2.133213in}}{\pgfqpoint{11.191890in}{2.137604in}}{\pgfqpoint{11.199704in}{2.145417in}}%
\pgfpathcurveto{\pgfqpoint{11.207517in}{2.153231in}}{\pgfqpoint{11.211908in}{2.163830in}}{\pgfqpoint{11.211908in}{2.174880in}}%
\pgfpathcurveto{\pgfqpoint{11.211908in}{2.185930in}}{\pgfqpoint{11.207517in}{2.196529in}}{\pgfqpoint{11.199704in}{2.204343in}}%
\pgfpathcurveto{\pgfqpoint{11.191890in}{2.212156in}}{\pgfqpoint{11.181291in}{2.216547in}}{\pgfqpoint{11.170241in}{2.216547in}}%
\pgfpathcurveto{\pgfqpoint{11.159191in}{2.216547in}}{\pgfqpoint{11.148592in}{2.212156in}}{\pgfqpoint{11.140778in}{2.204343in}}%
\pgfpathcurveto{\pgfqpoint{11.132964in}{2.196529in}}{\pgfqpoint{11.128574in}{2.185930in}}{\pgfqpoint{11.128574in}{2.174880in}}%
\pgfpathcurveto{\pgfqpoint{11.128574in}{2.163830in}}{\pgfqpoint{11.132964in}{2.153231in}}{\pgfqpoint{11.140778in}{2.145417in}}%
\pgfpathcurveto{\pgfqpoint{11.148592in}{2.137604in}}{\pgfqpoint{11.159191in}{2.133213in}}{\pgfqpoint{11.170241in}{2.133213in}}%
\pgfpathclose%
\pgfusepath{stroke,fill}%
\end{pgfscope}%
\begin{pgfscope}%
\pgfpathrectangle{\pgfqpoint{7.640588in}{0.566125in}}{\pgfqpoint{5.699255in}{2.685432in}}%
\pgfusepath{clip}%
\pgfsetbuttcap%
\pgfsetroundjoin%
\definecolor{currentfill}{rgb}{0.000000,0.000000,0.000000}%
\pgfsetfillcolor{currentfill}%
\pgfsetlinewidth{1.003750pt}%
\definecolor{currentstroke}{rgb}{0.000000,0.000000,0.000000}%
\pgfsetstrokecolor{currentstroke}%
\pgfsetdash{}{0pt}%
\pgfpathmoveto{\pgfqpoint{11.429298in}{1.788927in}}%
\pgfpathcurveto{\pgfqpoint{11.440348in}{1.788927in}}{\pgfqpoint{11.450947in}{1.793317in}}{\pgfqpoint{11.458761in}{1.801131in}}%
\pgfpathcurveto{\pgfqpoint{11.466574in}{1.808945in}}{\pgfqpoint{11.470965in}{1.819544in}}{\pgfqpoint{11.470965in}{1.830594in}}%
\pgfpathcurveto{\pgfqpoint{11.470965in}{1.841644in}}{\pgfqpoint{11.466574in}{1.852243in}}{\pgfqpoint{11.458761in}{1.860057in}}%
\pgfpathcurveto{\pgfqpoint{11.450947in}{1.867870in}}{\pgfqpoint{11.440348in}{1.872260in}}{\pgfqpoint{11.429298in}{1.872260in}}%
\pgfpathcurveto{\pgfqpoint{11.418248in}{1.872260in}}{\pgfqpoint{11.407649in}{1.867870in}}{\pgfqpoint{11.399835in}{1.860057in}}%
\pgfpathcurveto{\pgfqpoint{11.392022in}{1.852243in}}{\pgfqpoint{11.387631in}{1.841644in}}{\pgfqpoint{11.387631in}{1.830594in}}%
\pgfpathcurveto{\pgfqpoint{11.387631in}{1.819544in}}{\pgfqpoint{11.392022in}{1.808945in}}{\pgfqpoint{11.399835in}{1.801131in}}%
\pgfpathcurveto{\pgfqpoint{11.407649in}{1.793317in}}{\pgfqpoint{11.418248in}{1.788927in}}{\pgfqpoint{11.429298in}{1.788927in}}%
\pgfpathclose%
\pgfusepath{stroke,fill}%
\end{pgfscope}%
\begin{pgfscope}%
\pgfpathrectangle{\pgfqpoint{7.640588in}{0.566125in}}{\pgfqpoint{5.699255in}{2.685432in}}%
\pgfusepath{clip}%
\pgfsetbuttcap%
\pgfsetroundjoin%
\definecolor{currentfill}{rgb}{0.000000,0.000000,0.000000}%
\pgfsetfillcolor{currentfill}%
\pgfsetlinewidth{1.003750pt}%
\definecolor{currentstroke}{rgb}{0.000000,0.000000,0.000000}%
\pgfsetstrokecolor{currentstroke}%
\pgfsetdash{}{0pt}%
\pgfpathmoveto{\pgfqpoint{10.878802in}{1.757628in}}%
\pgfpathcurveto{\pgfqpoint{10.889852in}{1.757628in}}{\pgfqpoint{10.900451in}{1.762019in}}{\pgfqpoint{10.908264in}{1.769832in}}%
\pgfpathcurveto{\pgfqpoint{10.916078in}{1.777646in}}{\pgfqpoint{10.920468in}{1.788245in}}{\pgfqpoint{10.920468in}{1.799295in}}%
\pgfpathcurveto{\pgfqpoint{10.920468in}{1.810345in}}{\pgfqpoint{10.916078in}{1.820944in}}{\pgfqpoint{10.908264in}{1.828758in}}%
\pgfpathcurveto{\pgfqpoint{10.900451in}{1.836571in}}{\pgfqpoint{10.889852in}{1.840962in}}{\pgfqpoint{10.878802in}{1.840962in}}%
\pgfpathcurveto{\pgfqpoint{10.867752in}{1.840962in}}{\pgfqpoint{10.857152in}{1.836571in}}{\pgfqpoint{10.849339in}{1.828758in}}%
\pgfpathcurveto{\pgfqpoint{10.841525in}{1.820944in}}{\pgfqpoint{10.837135in}{1.810345in}}{\pgfqpoint{10.837135in}{1.799295in}}%
\pgfpathcurveto{\pgfqpoint{10.837135in}{1.788245in}}{\pgfqpoint{10.841525in}{1.777646in}}{\pgfqpoint{10.849339in}{1.769832in}}%
\pgfpathcurveto{\pgfqpoint{10.857152in}{1.762019in}}{\pgfqpoint{10.867752in}{1.757628in}}{\pgfqpoint{10.878802in}{1.757628in}}%
\pgfpathclose%
\pgfusepath{stroke,fill}%
\end{pgfscope}%
\begin{pgfscope}%
\pgfpathrectangle{\pgfqpoint{7.640588in}{0.566125in}}{\pgfqpoint{5.699255in}{2.685432in}}%
\pgfusepath{clip}%
\pgfsetbuttcap%
\pgfsetroundjoin%
\definecolor{currentfill}{rgb}{0.000000,0.000000,0.000000}%
\pgfsetfillcolor{currentfill}%
\pgfsetlinewidth{1.003750pt}%
\definecolor{currentstroke}{rgb}{0.000000,0.000000,0.000000}%
\pgfsetstrokecolor{currentstroke}%
\pgfsetdash{}{0pt}%
\pgfpathmoveto{\pgfqpoint{10.716891in}{1.319446in}}%
\pgfpathcurveto{\pgfqpoint{10.727941in}{1.319446in}}{\pgfqpoint{10.738540in}{1.323836in}}{\pgfqpoint{10.746354in}{1.331650in}}%
\pgfpathcurveto{\pgfqpoint{10.754167in}{1.339464in}}{\pgfqpoint{10.758558in}{1.350063in}}{\pgfqpoint{10.758558in}{1.361113in}}%
\pgfpathcurveto{\pgfqpoint{10.758558in}{1.372163in}}{\pgfqpoint{10.754167in}{1.382762in}}{\pgfqpoint{10.746354in}{1.390575in}}%
\pgfpathcurveto{\pgfqpoint{10.738540in}{1.398389in}}{\pgfqpoint{10.727941in}{1.402779in}}{\pgfqpoint{10.716891in}{1.402779in}}%
\pgfpathcurveto{\pgfqpoint{10.705841in}{1.402779in}}{\pgfqpoint{10.695242in}{1.398389in}}{\pgfqpoint{10.687428in}{1.390575in}}%
\pgfpathcurveto{\pgfqpoint{10.679615in}{1.382762in}}{\pgfqpoint{10.675224in}{1.372163in}}{\pgfqpoint{10.675224in}{1.361113in}}%
\pgfpathcurveto{\pgfqpoint{10.675224in}{1.350063in}}{\pgfqpoint{10.679615in}{1.339464in}}{\pgfqpoint{10.687428in}{1.331650in}}%
\pgfpathcurveto{\pgfqpoint{10.695242in}{1.323836in}}{\pgfqpoint{10.705841in}{1.319446in}}{\pgfqpoint{10.716891in}{1.319446in}}%
\pgfpathclose%
\pgfusepath{stroke,fill}%
\end{pgfscope}%
\begin{pgfscope}%
\pgfpathrectangle{\pgfqpoint{7.640588in}{0.566125in}}{\pgfqpoint{5.699255in}{2.685432in}}%
\pgfusepath{clip}%
\pgfsetbuttcap%
\pgfsetroundjoin%
\definecolor{currentfill}{rgb}{0.000000,0.000000,0.000000}%
\pgfsetfillcolor{currentfill}%
\pgfsetlinewidth{1.003750pt}%
\definecolor{currentstroke}{rgb}{0.000000,0.000000,0.000000}%
\pgfsetstrokecolor{currentstroke}%
\pgfsetdash{}{0pt}%
\pgfpathmoveto{\pgfqpoint{11.429298in}{1.178602in}}%
\pgfpathcurveto{\pgfqpoint{11.440348in}{1.178602in}}{\pgfqpoint{11.450947in}{1.182992in}}{\pgfqpoint{11.458761in}{1.190806in}}%
\pgfpathcurveto{\pgfqpoint{11.466574in}{1.198619in}}{\pgfqpoint{11.470965in}{1.209218in}}{\pgfqpoint{11.470965in}{1.220268in}}%
\pgfpathcurveto{\pgfqpoint{11.470965in}{1.231318in}}{\pgfqpoint{11.466574in}{1.241917in}}{\pgfqpoint{11.458761in}{1.249731in}}%
\pgfpathcurveto{\pgfqpoint{11.450947in}{1.257545in}}{\pgfqpoint{11.440348in}{1.261935in}}{\pgfqpoint{11.429298in}{1.261935in}}%
\pgfpathcurveto{\pgfqpoint{11.418248in}{1.261935in}}{\pgfqpoint{11.407649in}{1.257545in}}{\pgfqpoint{11.399835in}{1.249731in}}%
\pgfpathcurveto{\pgfqpoint{11.392022in}{1.241917in}}{\pgfqpoint{11.387631in}{1.231318in}}{\pgfqpoint{11.387631in}{1.220268in}}%
\pgfpathcurveto{\pgfqpoint{11.387631in}{1.209218in}}{\pgfqpoint{11.392022in}{1.198619in}}{\pgfqpoint{11.399835in}{1.190806in}}%
\pgfpathcurveto{\pgfqpoint{11.407649in}{1.182992in}}{\pgfqpoint{11.418248in}{1.178602in}}{\pgfqpoint{11.429298in}{1.178602in}}%
\pgfpathclose%
\pgfusepath{stroke,fill}%
\end{pgfscope}%
\begin{pgfscope}%
\pgfpathrectangle{\pgfqpoint{7.640588in}{0.566125in}}{\pgfqpoint{5.699255in}{2.685432in}}%
\pgfusepath{clip}%
\pgfsetbuttcap%
\pgfsetroundjoin%
\definecolor{currentfill}{rgb}{0.000000,0.000000,0.000000}%
\pgfsetfillcolor{currentfill}%
\pgfsetlinewidth{1.003750pt}%
\definecolor{currentstroke}{rgb}{0.000000,0.000000,0.000000}%
\pgfsetstrokecolor{currentstroke}%
\pgfsetdash{}{0pt}%
\pgfpathmoveto{\pgfqpoint{11.558826in}{1.116004in}}%
\pgfpathcurveto{\pgfqpoint{11.569877in}{1.116004in}}{\pgfqpoint{11.580476in}{1.120394in}}{\pgfqpoint{11.588289in}{1.128208in}}%
\pgfpathcurveto{\pgfqpoint{11.596103in}{1.136022in}}{\pgfqpoint{11.600493in}{1.146621in}}{\pgfqpoint{11.600493in}{1.157671in}}%
\pgfpathcurveto{\pgfqpoint{11.600493in}{1.168721in}}{\pgfqpoint{11.596103in}{1.179320in}}{\pgfqpoint{11.588289in}{1.187134in}}%
\pgfpathcurveto{\pgfqpoint{11.580476in}{1.194947in}}{\pgfqpoint{11.569877in}{1.199338in}}{\pgfqpoint{11.558826in}{1.199338in}}%
\pgfpathcurveto{\pgfqpoint{11.547776in}{1.199338in}}{\pgfqpoint{11.537177in}{1.194947in}}{\pgfqpoint{11.529364in}{1.187134in}}%
\pgfpathcurveto{\pgfqpoint{11.521550in}{1.179320in}}{\pgfqpoint{11.517160in}{1.168721in}}{\pgfqpoint{11.517160in}{1.157671in}}%
\pgfpathcurveto{\pgfqpoint{11.517160in}{1.146621in}}{\pgfqpoint{11.521550in}{1.136022in}}{\pgfqpoint{11.529364in}{1.128208in}}%
\pgfpathcurveto{\pgfqpoint{11.537177in}{1.120394in}}{\pgfqpoint{11.547776in}{1.116004in}}{\pgfqpoint{11.558826in}{1.116004in}}%
\pgfpathclose%
\pgfusepath{stroke,fill}%
\end{pgfscope}%
\begin{pgfscope}%
\pgfpathrectangle{\pgfqpoint{7.640588in}{0.566125in}}{\pgfqpoint{5.699255in}{2.685432in}}%
\pgfusepath{clip}%
\pgfsetbuttcap%
\pgfsetroundjoin%
\definecolor{currentfill}{rgb}{0.000000,0.000000,0.000000}%
\pgfsetfillcolor{currentfill}%
\pgfsetlinewidth{1.003750pt}%
\definecolor{currentstroke}{rgb}{0.000000,0.000000,0.000000}%
\pgfsetstrokecolor{currentstroke}%
\pgfsetdash{}{0pt}%
\pgfpathmoveto{\pgfqpoint{11.235005in}{0.849965in}}%
\pgfpathcurveto{\pgfqpoint{11.246055in}{0.849965in}}{\pgfqpoint{11.256654in}{0.854355in}}{\pgfqpoint{11.264468in}{0.862169in}}%
\pgfpathcurveto{\pgfqpoint{11.272282in}{0.869982in}}{\pgfqpoint{11.276672in}{0.880581in}}{\pgfqpoint{11.276672in}{0.891632in}}%
\pgfpathcurveto{\pgfqpoint{11.276672in}{0.902682in}}{\pgfqpoint{11.272282in}{0.913281in}}{\pgfqpoint{11.264468in}{0.921094in}}%
\pgfpathcurveto{\pgfqpoint{11.256654in}{0.928908in}}{\pgfqpoint{11.246055in}{0.933298in}}{\pgfqpoint{11.235005in}{0.933298in}}%
\pgfpathcurveto{\pgfqpoint{11.223955in}{0.933298in}}{\pgfqpoint{11.213356in}{0.928908in}}{\pgfqpoint{11.205542in}{0.921094in}}%
\pgfpathcurveto{\pgfqpoint{11.197729in}{0.913281in}}{\pgfqpoint{11.193338in}{0.902682in}}{\pgfqpoint{11.193338in}{0.891632in}}%
\pgfpathcurveto{\pgfqpoint{11.193338in}{0.880581in}}{\pgfqpoint{11.197729in}{0.869982in}}{\pgfqpoint{11.205542in}{0.862169in}}%
\pgfpathcurveto{\pgfqpoint{11.213356in}{0.854355in}}{\pgfqpoint{11.223955in}{0.849965in}}{\pgfqpoint{11.235005in}{0.849965in}}%
\pgfpathclose%
\pgfusepath{stroke,fill}%
\end{pgfscope}%
\begin{pgfscope}%
\pgfpathrectangle{\pgfqpoint{7.640588in}{0.566125in}}{\pgfqpoint{5.699255in}{2.685432in}}%
\pgfusepath{clip}%
\pgfsetbuttcap%
\pgfsetroundjoin%
\definecolor{currentfill}{rgb}{0.000000,0.000000,0.000000}%
\pgfsetfillcolor{currentfill}%
\pgfsetlinewidth{1.003750pt}%
\definecolor{currentstroke}{rgb}{0.000000,0.000000,0.000000}%
\pgfsetstrokecolor{currentstroke}%
\pgfsetdash{}{0pt}%
\pgfpathmoveto{\pgfqpoint{10.652127in}{0.787367in}}%
\pgfpathcurveto{\pgfqpoint{10.663177in}{0.787367in}}{\pgfqpoint{10.673776in}{0.791758in}}{\pgfqpoint{10.681590in}{0.799571in}}%
\pgfpathcurveto{\pgfqpoint{10.689403in}{0.807385in}}{\pgfqpoint{10.693793in}{0.817984in}}{\pgfqpoint{10.693793in}{0.829034in}}%
\pgfpathcurveto{\pgfqpoint{10.693793in}{0.840084in}}{\pgfqpoint{10.689403in}{0.850683in}}{\pgfqpoint{10.681590in}{0.858497in}}%
\pgfpathcurveto{\pgfqpoint{10.673776in}{0.866310in}}{\pgfqpoint{10.663177in}{0.870701in}}{\pgfqpoint{10.652127in}{0.870701in}}%
\pgfpathcurveto{\pgfqpoint{10.641077in}{0.870701in}}{\pgfqpoint{10.630478in}{0.866310in}}{\pgfqpoint{10.622664in}{0.858497in}}%
\pgfpathcurveto{\pgfqpoint{10.614850in}{0.850683in}}{\pgfqpoint{10.610460in}{0.840084in}}{\pgfqpoint{10.610460in}{0.829034in}}%
\pgfpathcurveto{\pgfqpoint{10.610460in}{0.817984in}}{\pgfqpoint{10.614850in}{0.807385in}}{\pgfqpoint{10.622664in}{0.799571in}}%
\pgfpathcurveto{\pgfqpoint{10.630478in}{0.791758in}}{\pgfqpoint{10.641077in}{0.787367in}}{\pgfqpoint{10.652127in}{0.787367in}}%
\pgfpathclose%
\pgfusepath{stroke,fill}%
\end{pgfscope}%
\begin{pgfscope}%
\pgfpathrectangle{\pgfqpoint{7.640588in}{0.566125in}}{\pgfqpoint{5.699255in}{2.685432in}}%
\pgfusepath{clip}%
\pgfsetbuttcap%
\pgfsetroundjoin%
\definecolor{currentfill}{rgb}{0.000000,0.000000,0.000000}%
\pgfsetfillcolor{currentfill}%
\pgfsetlinewidth{1.003750pt}%
\definecolor{currentstroke}{rgb}{0.000000,0.000000,0.000000}%
\pgfsetstrokecolor{currentstroke}%
\pgfsetdash{}{0pt}%
\pgfpathmoveto{\pgfqpoint{10.619745in}{0.756069in}}%
\pgfpathcurveto{\pgfqpoint{10.630795in}{0.756069in}}{\pgfqpoint{10.641394in}{0.760459in}}{\pgfqpoint{10.649207in}{0.768273in}}%
\pgfpathcurveto{\pgfqpoint{10.657021in}{0.776086in}}{\pgfqpoint{10.661411in}{0.786685in}}{\pgfqpoint{10.661411in}{0.797735in}}%
\pgfpathcurveto{\pgfqpoint{10.661411in}{0.808785in}}{\pgfqpoint{10.657021in}{0.819384in}}{\pgfqpoint{10.649207in}{0.827198in}}%
\pgfpathcurveto{\pgfqpoint{10.641394in}{0.835012in}}{\pgfqpoint{10.630795in}{0.839402in}}{\pgfqpoint{10.619745in}{0.839402in}}%
\pgfpathcurveto{\pgfqpoint{10.608694in}{0.839402in}}{\pgfqpoint{10.598095in}{0.835012in}}{\pgfqpoint{10.590282in}{0.827198in}}%
\pgfpathcurveto{\pgfqpoint{10.582468in}{0.819384in}}{\pgfqpoint{10.578078in}{0.808785in}}{\pgfqpoint{10.578078in}{0.797735in}}%
\pgfpathcurveto{\pgfqpoint{10.578078in}{0.786685in}}{\pgfqpoint{10.582468in}{0.776086in}}{\pgfqpoint{10.590282in}{0.768273in}}%
\pgfpathcurveto{\pgfqpoint{10.598095in}{0.760459in}}{\pgfqpoint{10.608694in}{0.756069in}}{\pgfqpoint{10.619745in}{0.756069in}}%
\pgfpathclose%
\pgfusepath{stroke,fill}%
\end{pgfscope}%
\begin{pgfscope}%
\pgfpathrectangle{\pgfqpoint{7.640588in}{0.566125in}}{\pgfqpoint{5.699255in}{2.685432in}}%
\pgfusepath{clip}%
\pgfsetbuttcap%
\pgfsetroundjoin%
\definecolor{currentfill}{rgb}{0.000000,0.000000,0.000000}%
\pgfsetfillcolor{currentfill}%
\pgfsetlinewidth{1.003750pt}%
\definecolor{currentstroke}{rgb}{0.000000,0.000000,0.000000}%
\pgfsetstrokecolor{currentstroke}%
\pgfsetdash{}{0pt}%
\pgfpathmoveto{\pgfqpoint{11.461680in}{0.646523in}}%
\pgfpathcurveto{\pgfqpoint{11.472730in}{0.646523in}}{\pgfqpoint{11.483329in}{0.650913in}}{\pgfqpoint{11.491143in}{0.658727in}}%
\pgfpathcurveto{\pgfqpoint{11.498956in}{0.666541in}}{\pgfqpoint{11.503347in}{0.677140in}}{\pgfqpoint{11.503347in}{0.688190in}}%
\pgfpathcurveto{\pgfqpoint{11.503347in}{0.699240in}}{\pgfqpoint{11.498956in}{0.709839in}}{\pgfqpoint{11.491143in}{0.717652in}}%
\pgfpathcurveto{\pgfqpoint{11.483329in}{0.725466in}}{\pgfqpoint{11.472730in}{0.729856in}}{\pgfqpoint{11.461680in}{0.729856in}}%
\pgfpathcurveto{\pgfqpoint{11.450630in}{0.729856in}}{\pgfqpoint{11.440031in}{0.725466in}}{\pgfqpoint{11.432217in}{0.717652in}}%
\pgfpathcurveto{\pgfqpoint{11.424404in}{0.709839in}}{\pgfqpoint{11.420013in}{0.699240in}}{\pgfqpoint{11.420013in}{0.688190in}}%
\pgfpathcurveto{\pgfqpoint{11.420013in}{0.677140in}}{\pgfqpoint{11.424404in}{0.666541in}}{\pgfqpoint{11.432217in}{0.658727in}}%
\pgfpathcurveto{\pgfqpoint{11.440031in}{0.650913in}}{\pgfqpoint{11.450630in}{0.646523in}}{\pgfqpoint{11.461680in}{0.646523in}}%
\pgfpathclose%
\pgfusepath{stroke,fill}%
\end{pgfscope}%
\begin{pgfscope}%
\pgfpathrectangle{\pgfqpoint{7.640588in}{0.566125in}}{\pgfqpoint{5.699255in}{2.685432in}}%
\pgfusepath{clip}%
\pgfsetbuttcap%
\pgfsetroundjoin%
\definecolor{currentfill}{rgb}{0.000000,0.000000,0.000000}%
\pgfsetfillcolor{currentfill}%
\pgfsetlinewidth{1.003750pt}%
\definecolor{currentstroke}{rgb}{0.000000,0.000000,0.000000}%
\pgfsetstrokecolor{currentstroke}%
\pgfsetdash{}{0pt}%
\pgfpathmoveto{\pgfqpoint{11.979794in}{0.646523in}}%
\pgfpathcurveto{\pgfqpoint{11.990844in}{0.646523in}}{\pgfqpoint{12.001443in}{0.650913in}}{\pgfqpoint{12.009257in}{0.658727in}}%
\pgfpathcurveto{\pgfqpoint{12.017071in}{0.666541in}}{\pgfqpoint{12.021461in}{0.677140in}}{\pgfqpoint{12.021461in}{0.688190in}}%
\pgfpathcurveto{\pgfqpoint{12.021461in}{0.699240in}}{\pgfqpoint{12.017071in}{0.709839in}}{\pgfqpoint{12.009257in}{0.717652in}}%
\pgfpathcurveto{\pgfqpoint{12.001443in}{0.725466in}}{\pgfqpoint{11.990844in}{0.729856in}}{\pgfqpoint{11.979794in}{0.729856in}}%
\pgfpathcurveto{\pgfqpoint{11.968744in}{0.729856in}}{\pgfqpoint{11.958145in}{0.725466in}}{\pgfqpoint{11.950331in}{0.717652in}}%
\pgfpathcurveto{\pgfqpoint{11.942518in}{0.709839in}}{\pgfqpoint{11.938128in}{0.699240in}}{\pgfqpoint{11.938128in}{0.688190in}}%
\pgfpathcurveto{\pgfqpoint{11.938128in}{0.677140in}}{\pgfqpoint{11.942518in}{0.666541in}}{\pgfqpoint{11.950331in}{0.658727in}}%
\pgfpathcurveto{\pgfqpoint{11.958145in}{0.650913in}}{\pgfqpoint{11.968744in}{0.646523in}}{\pgfqpoint{11.979794in}{0.646523in}}%
\pgfpathclose%
\pgfusepath{stroke,fill}%
\end{pgfscope}%
\begin{pgfscope}%
\pgfpathrectangle{\pgfqpoint{7.640588in}{0.566125in}}{\pgfqpoint{5.699255in}{2.685432in}}%
\pgfusepath{clip}%
\pgfsetbuttcap%
\pgfsetroundjoin%
\definecolor{currentfill}{rgb}{0.000000,0.000000,0.000000}%
\pgfsetfillcolor{currentfill}%
\pgfsetlinewidth{1.003750pt}%
\definecolor{currentstroke}{rgb}{0.000000,0.000000,0.000000}%
\pgfsetstrokecolor{currentstroke}%
\pgfsetdash{}{0pt}%
\pgfpathmoveto{\pgfqpoint{12.141705in}{1.116004in}}%
\pgfpathcurveto{\pgfqpoint{12.152755in}{1.116004in}}{\pgfqpoint{12.163354in}{1.120394in}}{\pgfqpoint{12.171168in}{1.128208in}}%
\pgfpathcurveto{\pgfqpoint{12.178981in}{1.136022in}}{\pgfqpoint{12.183372in}{1.146621in}}{\pgfqpoint{12.183372in}{1.157671in}}%
\pgfpathcurveto{\pgfqpoint{12.183372in}{1.168721in}}{\pgfqpoint{12.178981in}{1.179320in}}{\pgfqpoint{12.171168in}{1.187134in}}%
\pgfpathcurveto{\pgfqpoint{12.163354in}{1.194947in}}{\pgfqpoint{12.152755in}{1.199338in}}{\pgfqpoint{12.141705in}{1.199338in}}%
\pgfpathcurveto{\pgfqpoint{12.130655in}{1.199338in}}{\pgfqpoint{12.120056in}{1.194947in}}{\pgfqpoint{12.112242in}{1.187134in}}%
\pgfpathcurveto{\pgfqpoint{12.104428in}{1.179320in}}{\pgfqpoint{12.100038in}{1.168721in}}{\pgfqpoint{12.100038in}{1.157671in}}%
\pgfpathcurveto{\pgfqpoint{12.100038in}{1.146621in}}{\pgfqpoint{12.104428in}{1.136022in}}{\pgfqpoint{12.112242in}{1.128208in}}%
\pgfpathcurveto{\pgfqpoint{12.120056in}{1.120394in}}{\pgfqpoint{12.130655in}{1.116004in}}{\pgfqpoint{12.141705in}{1.116004in}}%
\pgfpathclose%
\pgfusepath{stroke,fill}%
\end{pgfscope}%
\begin{pgfscope}%
\pgfpathrectangle{\pgfqpoint{7.640588in}{0.566125in}}{\pgfqpoint{5.699255in}{2.685432in}}%
\pgfusepath{clip}%
\pgfsetbuttcap%
\pgfsetroundjoin%
\definecolor{currentfill}{rgb}{0.000000,0.000000,0.000000}%
\pgfsetfillcolor{currentfill}%
\pgfsetlinewidth{1.003750pt}%
\definecolor{currentstroke}{rgb}{0.000000,0.000000,0.000000}%
\pgfsetstrokecolor{currentstroke}%
\pgfsetdash{}{0pt}%
\pgfpathmoveto{\pgfqpoint{11.947412in}{1.147303in}}%
\pgfpathcurveto{\pgfqpoint{11.958462in}{1.147303in}}{\pgfqpoint{11.969061in}{1.151693in}}{\pgfqpoint{11.976875in}{1.159507in}}%
\pgfpathcurveto{\pgfqpoint{11.984688in}{1.167320in}}{\pgfqpoint{11.989079in}{1.177919in}}{\pgfqpoint{11.989079in}{1.188970in}}%
\pgfpathcurveto{\pgfqpoint{11.989079in}{1.200020in}}{\pgfqpoint{11.984688in}{1.210619in}}{\pgfqpoint{11.976875in}{1.218432in}}%
\pgfpathcurveto{\pgfqpoint{11.969061in}{1.226246in}}{\pgfqpoint{11.958462in}{1.230636in}}{\pgfqpoint{11.947412in}{1.230636in}}%
\pgfpathcurveto{\pgfqpoint{11.936362in}{1.230636in}}{\pgfqpoint{11.925763in}{1.226246in}}{\pgfqpoint{11.917949in}{1.218432in}}%
\pgfpathcurveto{\pgfqpoint{11.910136in}{1.210619in}}{\pgfqpoint{11.905745in}{1.200020in}}{\pgfqpoint{11.905745in}{1.188970in}}%
\pgfpathcurveto{\pgfqpoint{11.905745in}{1.177919in}}{\pgfqpoint{11.910136in}{1.167320in}}{\pgfqpoint{11.917949in}{1.159507in}}%
\pgfpathcurveto{\pgfqpoint{11.925763in}{1.151693in}}{\pgfqpoint{11.936362in}{1.147303in}}{\pgfqpoint{11.947412in}{1.147303in}}%
\pgfpathclose%
\pgfusepath{stroke,fill}%
\end{pgfscope}%
\begin{pgfscope}%
\pgfpathrectangle{\pgfqpoint{7.640588in}{0.566125in}}{\pgfqpoint{5.699255in}{2.685432in}}%
\pgfusepath{clip}%
\pgfsetbuttcap%
\pgfsetroundjoin%
\definecolor{currentfill}{rgb}{0.000000,0.000000,0.000000}%
\pgfsetfillcolor{currentfill}%
\pgfsetlinewidth{1.003750pt}%
\definecolor{currentstroke}{rgb}{0.000000,0.000000,0.000000}%
\pgfsetstrokecolor{currentstroke}%
\pgfsetdash{}{0pt}%
\pgfpathmoveto{\pgfqpoint{11.947412in}{1.225550in}}%
\pgfpathcurveto{\pgfqpoint{11.958462in}{1.225550in}}{\pgfqpoint{11.969061in}{1.229940in}}{\pgfqpoint{11.976875in}{1.237754in}}%
\pgfpathcurveto{\pgfqpoint{11.984688in}{1.245567in}}{\pgfqpoint{11.989079in}{1.256166in}}{\pgfqpoint{11.989079in}{1.267216in}}%
\pgfpathcurveto{\pgfqpoint{11.989079in}{1.278267in}}{\pgfqpoint{11.984688in}{1.288866in}}{\pgfqpoint{11.976875in}{1.296679in}}%
\pgfpathcurveto{\pgfqpoint{11.969061in}{1.304493in}}{\pgfqpoint{11.958462in}{1.308883in}}{\pgfqpoint{11.947412in}{1.308883in}}%
\pgfpathcurveto{\pgfqpoint{11.936362in}{1.308883in}}{\pgfqpoint{11.925763in}{1.304493in}}{\pgfqpoint{11.917949in}{1.296679in}}%
\pgfpathcurveto{\pgfqpoint{11.910136in}{1.288866in}}{\pgfqpoint{11.905745in}{1.278267in}}{\pgfqpoint{11.905745in}{1.267216in}}%
\pgfpathcurveto{\pgfqpoint{11.905745in}{1.256166in}}{\pgfqpoint{11.910136in}{1.245567in}}{\pgfqpoint{11.917949in}{1.237754in}}%
\pgfpathcurveto{\pgfqpoint{11.925763in}{1.229940in}}{\pgfqpoint{11.936362in}{1.225550in}}{\pgfqpoint{11.947412in}{1.225550in}}%
\pgfpathclose%
\pgfusepath{stroke,fill}%
\end{pgfscope}%
\begin{pgfscope}%
\pgfpathrectangle{\pgfqpoint{7.640588in}{0.566125in}}{\pgfqpoint{5.699255in}{2.685432in}}%
\pgfusepath{clip}%
\pgfsetbuttcap%
\pgfsetroundjoin%
\definecolor{currentfill}{rgb}{0.000000,0.000000,0.000000}%
\pgfsetfillcolor{currentfill}%
\pgfsetlinewidth{1.003750pt}%
\definecolor{currentstroke}{rgb}{0.000000,0.000000,0.000000}%
\pgfsetstrokecolor{currentstroke}%
\pgfsetdash{}{0pt}%
\pgfpathmoveto{\pgfqpoint{12.044558in}{1.288147in}}%
\pgfpathcurveto{\pgfqpoint{12.055609in}{1.288147in}}{\pgfqpoint{12.066208in}{1.292538in}}{\pgfqpoint{12.074021in}{1.300351in}}%
\pgfpathcurveto{\pgfqpoint{12.081835in}{1.308165in}}{\pgfqpoint{12.086225in}{1.318764in}}{\pgfqpoint{12.086225in}{1.329814in}}%
\pgfpathcurveto{\pgfqpoint{12.086225in}{1.340864in}}{\pgfqpoint{12.081835in}{1.351463in}}{\pgfqpoint{12.074021in}{1.359277in}}%
\pgfpathcurveto{\pgfqpoint{12.066208in}{1.367090in}}{\pgfqpoint{12.055609in}{1.371481in}}{\pgfqpoint{12.044558in}{1.371481in}}%
\pgfpathcurveto{\pgfqpoint{12.033508in}{1.371481in}}{\pgfqpoint{12.022909in}{1.367090in}}{\pgfqpoint{12.015096in}{1.359277in}}%
\pgfpathcurveto{\pgfqpoint{12.007282in}{1.351463in}}{\pgfqpoint{12.002892in}{1.340864in}}{\pgfqpoint{12.002892in}{1.329814in}}%
\pgfpathcurveto{\pgfqpoint{12.002892in}{1.318764in}}{\pgfqpoint{12.007282in}{1.308165in}}{\pgfqpoint{12.015096in}{1.300351in}}%
\pgfpathcurveto{\pgfqpoint{12.022909in}{1.292538in}}{\pgfqpoint{12.033508in}{1.288147in}}{\pgfqpoint{12.044558in}{1.288147in}}%
\pgfpathclose%
\pgfusepath{stroke,fill}%
\end{pgfscope}%
\begin{pgfscope}%
\pgfpathrectangle{\pgfqpoint{7.640588in}{0.566125in}}{\pgfqpoint{5.699255in}{2.685432in}}%
\pgfusepath{clip}%
\pgfsetbuttcap%
\pgfsetroundjoin%
\definecolor{currentfill}{rgb}{0.000000,0.000000,0.000000}%
\pgfsetfillcolor{currentfill}%
\pgfsetlinewidth{1.003750pt}%
\definecolor{currentstroke}{rgb}{0.000000,0.000000,0.000000}%
\pgfsetstrokecolor{currentstroke}%
\pgfsetdash{}{0pt}%
\pgfpathmoveto{\pgfqpoint{12.854112in}{1.241199in}}%
\pgfpathcurveto{\pgfqpoint{12.865162in}{1.241199in}}{\pgfqpoint{12.875761in}{1.245589in}}{\pgfqpoint{12.883575in}{1.253403in}}%
\pgfpathcurveto{\pgfqpoint{12.891388in}{1.261217in}}{\pgfqpoint{12.895778in}{1.271816in}}{\pgfqpoint{12.895778in}{1.282866in}}%
\pgfpathcurveto{\pgfqpoint{12.895778in}{1.293916in}}{\pgfqpoint{12.891388in}{1.304515in}}{\pgfqpoint{12.883575in}{1.312329in}}%
\pgfpathcurveto{\pgfqpoint{12.875761in}{1.320142in}}{\pgfqpoint{12.865162in}{1.324532in}}{\pgfqpoint{12.854112in}{1.324532in}}%
\pgfpathcurveto{\pgfqpoint{12.843062in}{1.324532in}}{\pgfqpoint{12.832463in}{1.320142in}}{\pgfqpoint{12.824649in}{1.312329in}}%
\pgfpathcurveto{\pgfqpoint{12.816835in}{1.304515in}}{\pgfqpoint{12.812445in}{1.293916in}}{\pgfqpoint{12.812445in}{1.282866in}}%
\pgfpathcurveto{\pgfqpoint{12.812445in}{1.271816in}}{\pgfqpoint{12.816835in}{1.261217in}}{\pgfqpoint{12.824649in}{1.253403in}}%
\pgfpathcurveto{\pgfqpoint{12.832463in}{1.245589in}}{\pgfqpoint{12.843062in}{1.241199in}}{\pgfqpoint{12.854112in}{1.241199in}}%
\pgfpathclose%
\pgfusepath{stroke,fill}%
\end{pgfscope}%
\begin{pgfscope}%
\pgfpathrectangle{\pgfqpoint{7.640588in}{0.566125in}}{\pgfqpoint{5.699255in}{2.685432in}}%
\pgfusepath{clip}%
\pgfsetbuttcap%
\pgfsetroundjoin%
\definecolor{currentfill}{rgb}{0.000000,0.000000,0.000000}%
\pgfsetfillcolor{currentfill}%
\pgfsetlinewidth{1.003750pt}%
\definecolor{currentstroke}{rgb}{0.000000,0.000000,0.000000}%
\pgfsetstrokecolor{currentstroke}%
\pgfsetdash{}{0pt}%
\pgfpathmoveto{\pgfqpoint{12.465526in}{1.413342in}}%
\pgfpathcurveto{\pgfqpoint{12.476576in}{1.413342in}}{\pgfqpoint{12.487175in}{1.417732in}}{\pgfqpoint{12.494989in}{1.425546in}}%
\pgfpathcurveto{\pgfqpoint{12.502803in}{1.433360in}}{\pgfqpoint{12.507193in}{1.443959in}}{\pgfqpoint{12.507193in}{1.455009in}}%
\pgfpathcurveto{\pgfqpoint{12.507193in}{1.466059in}}{\pgfqpoint{12.502803in}{1.476658in}}{\pgfqpoint{12.494989in}{1.484472in}}%
\pgfpathcurveto{\pgfqpoint{12.487175in}{1.492285in}}{\pgfqpoint{12.476576in}{1.496676in}}{\pgfqpoint{12.465526in}{1.496676in}}%
\pgfpathcurveto{\pgfqpoint{12.454476in}{1.496676in}}{\pgfqpoint{12.443877in}{1.492285in}}{\pgfqpoint{12.436063in}{1.484472in}}%
\pgfpathcurveto{\pgfqpoint{12.428250in}{1.476658in}}{\pgfqpoint{12.423859in}{1.466059in}}{\pgfqpoint{12.423859in}{1.455009in}}%
\pgfpathcurveto{\pgfqpoint{12.423859in}{1.443959in}}{\pgfqpoint{12.428250in}{1.433360in}}{\pgfqpoint{12.436063in}{1.425546in}}%
\pgfpathcurveto{\pgfqpoint{12.443877in}{1.417732in}}{\pgfqpoint{12.454476in}{1.413342in}}{\pgfqpoint{12.465526in}{1.413342in}}%
\pgfpathclose%
\pgfusepath{stroke,fill}%
\end{pgfscope}%
\begin{pgfscope}%
\pgfpathrectangle{\pgfqpoint{7.640588in}{0.566125in}}{\pgfqpoint{5.699255in}{2.685432in}}%
\pgfusepath{clip}%
\pgfsetbuttcap%
\pgfsetroundjoin%
\definecolor{currentfill}{rgb}{0.000000,0.000000,0.000000}%
\pgfsetfillcolor{currentfill}%
\pgfsetlinewidth{1.003750pt}%
\definecolor{currentstroke}{rgb}{0.000000,0.000000,0.000000}%
\pgfsetstrokecolor{currentstroke}%
\pgfsetdash{}{0pt}%
\pgfpathmoveto{\pgfqpoint{12.012176in}{1.475940in}}%
\pgfpathcurveto{\pgfqpoint{12.023226in}{1.475940in}}{\pgfqpoint{12.033825in}{1.480330in}}{\pgfqpoint{12.041639in}{1.488144in}}%
\pgfpathcurveto{\pgfqpoint{12.049453in}{1.495957in}}{\pgfqpoint{12.053843in}{1.506556in}}{\pgfqpoint{12.053843in}{1.517606in}}%
\pgfpathcurveto{\pgfqpoint{12.053843in}{1.528657in}}{\pgfqpoint{12.049453in}{1.539256in}}{\pgfqpoint{12.041639in}{1.547069in}}%
\pgfpathcurveto{\pgfqpoint{12.033825in}{1.554883in}}{\pgfqpoint{12.023226in}{1.559273in}}{\pgfqpoint{12.012176in}{1.559273in}}%
\pgfpathcurveto{\pgfqpoint{12.001126in}{1.559273in}}{\pgfqpoint{11.990527in}{1.554883in}}{\pgfqpoint{11.982714in}{1.547069in}}%
\pgfpathcurveto{\pgfqpoint{11.974900in}{1.539256in}}{\pgfqpoint{11.970510in}{1.528657in}}{\pgfqpoint{11.970510in}{1.517606in}}%
\pgfpathcurveto{\pgfqpoint{11.970510in}{1.506556in}}{\pgfqpoint{11.974900in}{1.495957in}}{\pgfqpoint{11.982714in}{1.488144in}}%
\pgfpathcurveto{\pgfqpoint{11.990527in}{1.480330in}}{\pgfqpoint{12.001126in}{1.475940in}}{\pgfqpoint{12.012176in}{1.475940in}}%
\pgfpathclose%
\pgfusepath{stroke,fill}%
\end{pgfscope}%
\begin{pgfscope}%
\pgfpathrectangle{\pgfqpoint{7.640588in}{0.566125in}}{\pgfqpoint{5.699255in}{2.685432in}}%
\pgfusepath{clip}%
\pgfsetbuttcap%
\pgfsetroundjoin%
\definecolor{currentfill}{rgb}{0.000000,0.000000,0.000000}%
\pgfsetfillcolor{currentfill}%
\pgfsetlinewidth{1.003750pt}%
\definecolor{currentstroke}{rgb}{0.000000,0.000000,0.000000}%
\pgfsetstrokecolor{currentstroke}%
\pgfsetdash{}{0pt}%
\pgfpathmoveto{\pgfqpoint{11.817884in}{1.820226in}}%
\pgfpathcurveto{\pgfqpoint{11.828934in}{1.820226in}}{\pgfqpoint{11.839533in}{1.824616in}}{\pgfqpoint{11.847346in}{1.832430in}}%
\pgfpathcurveto{\pgfqpoint{11.855160in}{1.840243in}}{\pgfqpoint{11.859550in}{1.850842in}}{\pgfqpoint{11.859550in}{1.861893in}}%
\pgfpathcurveto{\pgfqpoint{11.859550in}{1.872943in}}{\pgfqpoint{11.855160in}{1.883542in}}{\pgfqpoint{11.847346in}{1.891355in}}%
\pgfpathcurveto{\pgfqpoint{11.839533in}{1.899169in}}{\pgfqpoint{11.828934in}{1.903559in}}{\pgfqpoint{11.817884in}{1.903559in}}%
\pgfpathcurveto{\pgfqpoint{11.806833in}{1.903559in}}{\pgfqpoint{11.796234in}{1.899169in}}{\pgfqpoint{11.788421in}{1.891355in}}%
\pgfpathcurveto{\pgfqpoint{11.780607in}{1.883542in}}{\pgfqpoint{11.776217in}{1.872943in}}{\pgfqpoint{11.776217in}{1.861893in}}%
\pgfpathcurveto{\pgfqpoint{11.776217in}{1.850842in}}{\pgfqpoint{11.780607in}{1.840243in}}{\pgfqpoint{11.788421in}{1.832430in}}%
\pgfpathcurveto{\pgfqpoint{11.796234in}{1.824616in}}{\pgfqpoint{11.806833in}{1.820226in}}{\pgfqpoint{11.817884in}{1.820226in}}%
\pgfpathclose%
\pgfusepath{stroke,fill}%
\end{pgfscope}%
\begin{pgfscope}%
\pgfpathrectangle{\pgfqpoint{7.640588in}{0.566125in}}{\pgfqpoint{5.699255in}{2.685432in}}%
\pgfusepath{clip}%
\pgfsetbuttcap%
\pgfsetroundjoin%
\definecolor{currentfill}{rgb}{0.000000,0.000000,1.000000}%
\pgfsetfillcolor{currentfill}%
\pgfsetlinewidth{1.003750pt}%
\definecolor{currentstroke}{rgb}{0.000000,0.000000,1.000000}%
\pgfsetstrokecolor{currentstroke}%
\pgfsetdash{}{0pt}%
\pgfpathmoveto{\pgfqpoint{12.692201in}{1.851525in}}%
\pgfpathcurveto{\pgfqpoint{12.703251in}{1.851525in}}{\pgfqpoint{12.713850in}{1.855915in}}{\pgfqpoint{12.721664in}{1.863729in}}%
\pgfpathcurveto{\pgfqpoint{12.729477in}{1.871542in}}{\pgfqpoint{12.733868in}{1.882141in}}{\pgfqpoint{12.733868in}{1.893191in}}%
\pgfpathcurveto{\pgfqpoint{12.733868in}{1.904241in}}{\pgfqpoint{12.729477in}{1.914840in}}{\pgfqpoint{12.721664in}{1.922654in}}%
\pgfpathcurveto{\pgfqpoint{12.713850in}{1.930468in}}{\pgfqpoint{12.703251in}{1.934858in}}{\pgfqpoint{12.692201in}{1.934858in}}%
\pgfpathcurveto{\pgfqpoint{12.681151in}{1.934858in}}{\pgfqpoint{12.670552in}{1.930468in}}{\pgfqpoint{12.662738in}{1.922654in}}%
\pgfpathcurveto{\pgfqpoint{12.654925in}{1.914840in}}{\pgfqpoint{12.650534in}{1.904241in}}{\pgfqpoint{12.650534in}{1.893191in}}%
\pgfpathcurveto{\pgfqpoint{12.650534in}{1.882141in}}{\pgfqpoint{12.654925in}{1.871542in}}{\pgfqpoint{12.662738in}{1.863729in}}%
\pgfpathcurveto{\pgfqpoint{12.670552in}{1.855915in}}{\pgfqpoint{12.681151in}{1.851525in}}{\pgfqpoint{12.692201in}{1.851525in}}%
\pgfpathclose%
\pgfusepath{stroke,fill}%
\end{pgfscope}%
\begin{pgfscope}%
\pgfsetbuttcap%
\pgfsetroundjoin%
\definecolor{currentfill}{rgb}{0.000000,0.000000,0.000000}%
\pgfsetfillcolor{currentfill}%
\pgfsetlinewidth{0.803000pt}%
\definecolor{currentstroke}{rgb}{0.000000,0.000000,0.000000}%
\pgfsetstrokecolor{currentstroke}%
\pgfsetdash{}{0pt}%
\pgfsys@defobject{currentmarker}{\pgfqpoint{0.000000in}{-0.048611in}}{\pgfqpoint{0.000000in}{0.000000in}}{%
\pgfpathmoveto{\pgfqpoint{0.000000in}{0.000000in}}%
\pgfpathlineto{\pgfqpoint{0.000000in}{-0.048611in}}%
\pgfusepath{stroke,fill}%
}%
\begin{pgfscope}%
\pgfsys@transformshift{7.802499in}{0.566125in}%
\pgfsys@useobject{currentmarker}{}%
\end{pgfscope}%
\end{pgfscope}%
\begin{pgfscope}%
\definecolor{textcolor}{rgb}{0.000000,0.000000,0.000000}%
\pgfsetstrokecolor{textcolor}%
\pgfsetfillcolor{textcolor}%
\pgftext[x=7.802499in,y=0.468902in,,top]{\color{textcolor}\rmfamily\fontsize{10.000000}{12.000000}\selectfont \(\displaystyle 0\)}%
\end{pgfscope}%
\begin{pgfscope}%
\pgfsetbuttcap%
\pgfsetroundjoin%
\definecolor{currentfill}{rgb}{0.000000,0.000000,0.000000}%
\pgfsetfillcolor{currentfill}%
\pgfsetlinewidth{0.803000pt}%
\definecolor{currentstroke}{rgb}{0.000000,0.000000,0.000000}%
\pgfsetstrokecolor{currentstroke}%
\pgfsetdash{}{0pt}%
\pgfsys@defobject{currentmarker}{\pgfqpoint{0.000000in}{-0.048611in}}{\pgfqpoint{0.000000in}{0.000000in}}{%
\pgfpathmoveto{\pgfqpoint{0.000000in}{0.000000in}}%
\pgfpathlineto{\pgfqpoint{0.000000in}{-0.048611in}}%
\pgfusepath{stroke,fill}%
}%
\begin{pgfscope}%
\pgfsys@transformshift{8.450142in}{0.566125in}%
\pgfsys@useobject{currentmarker}{}%
\end{pgfscope}%
\end{pgfscope}%
\begin{pgfscope}%
\definecolor{textcolor}{rgb}{0.000000,0.000000,0.000000}%
\pgfsetstrokecolor{textcolor}%
\pgfsetfillcolor{textcolor}%
\pgftext[x=8.450142in,y=0.468902in,,top]{\color{textcolor}\rmfamily\fontsize{10.000000}{12.000000}\selectfont \(\displaystyle 20\)}%
\end{pgfscope}%
\begin{pgfscope}%
\pgfsetbuttcap%
\pgfsetroundjoin%
\definecolor{currentfill}{rgb}{0.000000,0.000000,0.000000}%
\pgfsetfillcolor{currentfill}%
\pgfsetlinewidth{0.803000pt}%
\definecolor{currentstroke}{rgb}{0.000000,0.000000,0.000000}%
\pgfsetstrokecolor{currentstroke}%
\pgfsetdash{}{0pt}%
\pgfsys@defobject{currentmarker}{\pgfqpoint{0.000000in}{-0.048611in}}{\pgfqpoint{0.000000in}{0.000000in}}{%
\pgfpathmoveto{\pgfqpoint{0.000000in}{0.000000in}}%
\pgfpathlineto{\pgfqpoint{0.000000in}{-0.048611in}}%
\pgfusepath{stroke,fill}%
}%
\begin{pgfscope}%
\pgfsys@transformshift{9.097784in}{0.566125in}%
\pgfsys@useobject{currentmarker}{}%
\end{pgfscope}%
\end{pgfscope}%
\begin{pgfscope}%
\definecolor{textcolor}{rgb}{0.000000,0.000000,0.000000}%
\pgfsetstrokecolor{textcolor}%
\pgfsetfillcolor{textcolor}%
\pgftext[x=9.097784in,y=0.468902in,,top]{\color{textcolor}\rmfamily\fontsize{10.000000}{12.000000}\selectfont \(\displaystyle 40\)}%
\end{pgfscope}%
\begin{pgfscope}%
\pgfsetbuttcap%
\pgfsetroundjoin%
\definecolor{currentfill}{rgb}{0.000000,0.000000,0.000000}%
\pgfsetfillcolor{currentfill}%
\pgfsetlinewidth{0.803000pt}%
\definecolor{currentstroke}{rgb}{0.000000,0.000000,0.000000}%
\pgfsetstrokecolor{currentstroke}%
\pgfsetdash{}{0pt}%
\pgfsys@defobject{currentmarker}{\pgfqpoint{0.000000in}{-0.048611in}}{\pgfqpoint{0.000000in}{0.000000in}}{%
\pgfpathmoveto{\pgfqpoint{0.000000in}{0.000000in}}%
\pgfpathlineto{\pgfqpoint{0.000000in}{-0.048611in}}%
\pgfusepath{stroke,fill}%
}%
\begin{pgfscope}%
\pgfsys@transformshift{9.745427in}{0.566125in}%
\pgfsys@useobject{currentmarker}{}%
\end{pgfscope}%
\end{pgfscope}%
\begin{pgfscope}%
\definecolor{textcolor}{rgb}{0.000000,0.000000,0.000000}%
\pgfsetstrokecolor{textcolor}%
\pgfsetfillcolor{textcolor}%
\pgftext[x=9.745427in,y=0.468902in,,top]{\color{textcolor}\rmfamily\fontsize{10.000000}{12.000000}\selectfont \(\displaystyle 60\)}%
\end{pgfscope}%
\begin{pgfscope}%
\pgfsetbuttcap%
\pgfsetroundjoin%
\definecolor{currentfill}{rgb}{0.000000,0.000000,0.000000}%
\pgfsetfillcolor{currentfill}%
\pgfsetlinewidth{0.803000pt}%
\definecolor{currentstroke}{rgb}{0.000000,0.000000,0.000000}%
\pgfsetstrokecolor{currentstroke}%
\pgfsetdash{}{0pt}%
\pgfsys@defobject{currentmarker}{\pgfqpoint{0.000000in}{-0.048611in}}{\pgfqpoint{0.000000in}{0.000000in}}{%
\pgfpathmoveto{\pgfqpoint{0.000000in}{0.000000in}}%
\pgfpathlineto{\pgfqpoint{0.000000in}{-0.048611in}}%
\pgfusepath{stroke,fill}%
}%
\begin{pgfscope}%
\pgfsys@transformshift{10.393070in}{0.566125in}%
\pgfsys@useobject{currentmarker}{}%
\end{pgfscope}%
\end{pgfscope}%
\begin{pgfscope}%
\definecolor{textcolor}{rgb}{0.000000,0.000000,0.000000}%
\pgfsetstrokecolor{textcolor}%
\pgfsetfillcolor{textcolor}%
\pgftext[x=10.393070in,y=0.468902in,,top]{\color{textcolor}\rmfamily\fontsize{10.000000}{12.000000}\selectfont \(\displaystyle 80\)}%
\end{pgfscope}%
\begin{pgfscope}%
\pgfsetbuttcap%
\pgfsetroundjoin%
\definecolor{currentfill}{rgb}{0.000000,0.000000,0.000000}%
\pgfsetfillcolor{currentfill}%
\pgfsetlinewidth{0.803000pt}%
\definecolor{currentstroke}{rgb}{0.000000,0.000000,0.000000}%
\pgfsetstrokecolor{currentstroke}%
\pgfsetdash{}{0pt}%
\pgfsys@defobject{currentmarker}{\pgfqpoint{0.000000in}{-0.048611in}}{\pgfqpoint{0.000000in}{0.000000in}}{%
\pgfpathmoveto{\pgfqpoint{0.000000in}{0.000000in}}%
\pgfpathlineto{\pgfqpoint{0.000000in}{-0.048611in}}%
\pgfusepath{stroke,fill}%
}%
\begin{pgfscope}%
\pgfsys@transformshift{11.040712in}{0.566125in}%
\pgfsys@useobject{currentmarker}{}%
\end{pgfscope}%
\end{pgfscope}%
\begin{pgfscope}%
\definecolor{textcolor}{rgb}{0.000000,0.000000,0.000000}%
\pgfsetstrokecolor{textcolor}%
\pgfsetfillcolor{textcolor}%
\pgftext[x=11.040712in,y=0.468902in,,top]{\color{textcolor}\rmfamily\fontsize{10.000000}{12.000000}\selectfont \(\displaystyle 100\)}%
\end{pgfscope}%
\begin{pgfscope}%
\pgfsetbuttcap%
\pgfsetroundjoin%
\definecolor{currentfill}{rgb}{0.000000,0.000000,0.000000}%
\pgfsetfillcolor{currentfill}%
\pgfsetlinewidth{0.803000pt}%
\definecolor{currentstroke}{rgb}{0.000000,0.000000,0.000000}%
\pgfsetstrokecolor{currentstroke}%
\pgfsetdash{}{0pt}%
\pgfsys@defobject{currentmarker}{\pgfqpoint{0.000000in}{-0.048611in}}{\pgfqpoint{0.000000in}{0.000000in}}{%
\pgfpathmoveto{\pgfqpoint{0.000000in}{0.000000in}}%
\pgfpathlineto{\pgfqpoint{0.000000in}{-0.048611in}}%
\pgfusepath{stroke,fill}%
}%
\begin{pgfscope}%
\pgfsys@transformshift{11.688355in}{0.566125in}%
\pgfsys@useobject{currentmarker}{}%
\end{pgfscope}%
\end{pgfscope}%
\begin{pgfscope}%
\definecolor{textcolor}{rgb}{0.000000,0.000000,0.000000}%
\pgfsetstrokecolor{textcolor}%
\pgfsetfillcolor{textcolor}%
\pgftext[x=11.688355in,y=0.468902in,,top]{\color{textcolor}\rmfamily\fontsize{10.000000}{12.000000}\selectfont \(\displaystyle 120\)}%
\end{pgfscope}%
\begin{pgfscope}%
\pgfsetbuttcap%
\pgfsetroundjoin%
\definecolor{currentfill}{rgb}{0.000000,0.000000,0.000000}%
\pgfsetfillcolor{currentfill}%
\pgfsetlinewidth{0.803000pt}%
\definecolor{currentstroke}{rgb}{0.000000,0.000000,0.000000}%
\pgfsetstrokecolor{currentstroke}%
\pgfsetdash{}{0pt}%
\pgfsys@defobject{currentmarker}{\pgfqpoint{0.000000in}{-0.048611in}}{\pgfqpoint{0.000000in}{0.000000in}}{%
\pgfpathmoveto{\pgfqpoint{0.000000in}{0.000000in}}%
\pgfpathlineto{\pgfqpoint{0.000000in}{-0.048611in}}%
\pgfusepath{stroke,fill}%
}%
\begin{pgfscope}%
\pgfsys@transformshift{12.335998in}{0.566125in}%
\pgfsys@useobject{currentmarker}{}%
\end{pgfscope}%
\end{pgfscope}%
\begin{pgfscope}%
\definecolor{textcolor}{rgb}{0.000000,0.000000,0.000000}%
\pgfsetstrokecolor{textcolor}%
\pgfsetfillcolor{textcolor}%
\pgftext[x=12.335998in,y=0.468902in,,top]{\color{textcolor}\rmfamily\fontsize{10.000000}{12.000000}\selectfont \(\displaystyle 140\)}%
\end{pgfscope}%
\begin{pgfscope}%
\pgfsetbuttcap%
\pgfsetroundjoin%
\definecolor{currentfill}{rgb}{0.000000,0.000000,0.000000}%
\pgfsetfillcolor{currentfill}%
\pgfsetlinewidth{0.803000pt}%
\definecolor{currentstroke}{rgb}{0.000000,0.000000,0.000000}%
\pgfsetstrokecolor{currentstroke}%
\pgfsetdash{}{0pt}%
\pgfsys@defobject{currentmarker}{\pgfqpoint{0.000000in}{-0.048611in}}{\pgfqpoint{0.000000in}{0.000000in}}{%
\pgfpathmoveto{\pgfqpoint{0.000000in}{0.000000in}}%
\pgfpathlineto{\pgfqpoint{0.000000in}{-0.048611in}}%
\pgfusepath{stroke,fill}%
}%
\begin{pgfscope}%
\pgfsys@transformshift{12.983640in}{0.566125in}%
\pgfsys@useobject{currentmarker}{}%
\end{pgfscope}%
\end{pgfscope}%
\begin{pgfscope}%
\definecolor{textcolor}{rgb}{0.000000,0.000000,0.000000}%
\pgfsetstrokecolor{textcolor}%
\pgfsetfillcolor{textcolor}%
\pgftext[x=12.983640in,y=0.468902in,,top]{\color{textcolor}\rmfamily\fontsize{10.000000}{12.000000}\selectfont \(\displaystyle 160\)}%
\end{pgfscope}%
\begin{pgfscope}%
\pgfsetbuttcap%
\pgfsetroundjoin%
\definecolor{currentfill}{rgb}{0.000000,0.000000,0.000000}%
\pgfsetfillcolor{currentfill}%
\pgfsetlinewidth{0.803000pt}%
\definecolor{currentstroke}{rgb}{0.000000,0.000000,0.000000}%
\pgfsetstrokecolor{currentstroke}%
\pgfsetdash{}{0pt}%
\pgfsys@defobject{currentmarker}{\pgfqpoint{-0.048611in}{0.000000in}}{\pgfqpoint{0.000000in}{0.000000in}}{%
\pgfpathmoveto{\pgfqpoint{0.000000in}{0.000000in}}%
\pgfpathlineto{\pgfqpoint{-0.048611in}{0.000000in}}%
\pgfusepath{stroke,fill}%
}%
\begin{pgfscope}%
\pgfsys@transformshift{7.640588in}{0.688190in}%
\pgfsys@useobject{currentmarker}{}%
\end{pgfscope}%
\end{pgfscope}%
\begin{pgfscope}%
\definecolor{textcolor}{rgb}{0.000000,0.000000,0.000000}%
\pgfsetstrokecolor{textcolor}%
\pgfsetfillcolor{textcolor}%
\pgftext[x=7.473921in, y=0.635428in, left, base]{\color{textcolor}\rmfamily\fontsize{10.000000}{12.000000}\selectfont \(\displaystyle 0\)}%
\end{pgfscope}%
\begin{pgfscope}%
\pgfsetbuttcap%
\pgfsetroundjoin%
\definecolor{currentfill}{rgb}{0.000000,0.000000,0.000000}%
\pgfsetfillcolor{currentfill}%
\pgfsetlinewidth{0.803000pt}%
\definecolor{currentstroke}{rgb}{0.000000,0.000000,0.000000}%
\pgfsetstrokecolor{currentstroke}%
\pgfsetdash{}{0pt}%
\pgfsys@defobject{currentmarker}{\pgfqpoint{-0.048611in}{0.000000in}}{\pgfqpoint{0.000000in}{0.000000in}}{%
\pgfpathmoveto{\pgfqpoint{0.000000in}{0.000000in}}%
\pgfpathlineto{\pgfqpoint{-0.048611in}{0.000000in}}%
\pgfusepath{stroke,fill}%
}%
\begin{pgfscope}%
\pgfsys@transformshift{7.640588in}{1.001177in}%
\pgfsys@useobject{currentmarker}{}%
\end{pgfscope}%
\end{pgfscope}%
\begin{pgfscope}%
\definecolor{textcolor}{rgb}{0.000000,0.000000,0.000000}%
\pgfsetstrokecolor{textcolor}%
\pgfsetfillcolor{textcolor}%
\pgftext[x=7.404477in, y=0.948416in, left, base]{\color{textcolor}\rmfamily\fontsize{10.000000}{12.000000}\selectfont \(\displaystyle 20\)}%
\end{pgfscope}%
\begin{pgfscope}%
\pgfsetbuttcap%
\pgfsetroundjoin%
\definecolor{currentfill}{rgb}{0.000000,0.000000,0.000000}%
\pgfsetfillcolor{currentfill}%
\pgfsetlinewidth{0.803000pt}%
\definecolor{currentstroke}{rgb}{0.000000,0.000000,0.000000}%
\pgfsetstrokecolor{currentstroke}%
\pgfsetdash{}{0pt}%
\pgfsys@defobject{currentmarker}{\pgfqpoint{-0.048611in}{0.000000in}}{\pgfqpoint{0.000000in}{0.000000in}}{%
\pgfpathmoveto{\pgfqpoint{0.000000in}{0.000000in}}%
\pgfpathlineto{\pgfqpoint{-0.048611in}{0.000000in}}%
\pgfusepath{stroke,fill}%
}%
\begin{pgfscope}%
\pgfsys@transformshift{7.640588in}{1.314165in}%
\pgfsys@useobject{currentmarker}{}%
\end{pgfscope}%
\end{pgfscope}%
\begin{pgfscope}%
\definecolor{textcolor}{rgb}{0.000000,0.000000,0.000000}%
\pgfsetstrokecolor{textcolor}%
\pgfsetfillcolor{textcolor}%
\pgftext[x=7.404477in, y=1.261403in, left, base]{\color{textcolor}\rmfamily\fontsize{10.000000}{12.000000}\selectfont \(\displaystyle 40\)}%
\end{pgfscope}%
\begin{pgfscope}%
\pgfsetbuttcap%
\pgfsetroundjoin%
\definecolor{currentfill}{rgb}{0.000000,0.000000,0.000000}%
\pgfsetfillcolor{currentfill}%
\pgfsetlinewidth{0.803000pt}%
\definecolor{currentstroke}{rgb}{0.000000,0.000000,0.000000}%
\pgfsetstrokecolor{currentstroke}%
\pgfsetdash{}{0pt}%
\pgfsys@defobject{currentmarker}{\pgfqpoint{-0.048611in}{0.000000in}}{\pgfqpoint{0.000000in}{0.000000in}}{%
\pgfpathmoveto{\pgfqpoint{0.000000in}{0.000000in}}%
\pgfpathlineto{\pgfqpoint{-0.048611in}{0.000000in}}%
\pgfusepath{stroke,fill}%
}%
\begin{pgfscope}%
\pgfsys@transformshift{7.640588in}{1.627152in}%
\pgfsys@useobject{currentmarker}{}%
\end{pgfscope}%
\end{pgfscope}%
\begin{pgfscope}%
\definecolor{textcolor}{rgb}{0.000000,0.000000,0.000000}%
\pgfsetstrokecolor{textcolor}%
\pgfsetfillcolor{textcolor}%
\pgftext[x=7.404477in, y=1.574390in, left, base]{\color{textcolor}\rmfamily\fontsize{10.000000}{12.000000}\selectfont \(\displaystyle 60\)}%
\end{pgfscope}%
\begin{pgfscope}%
\pgfsetbuttcap%
\pgfsetroundjoin%
\definecolor{currentfill}{rgb}{0.000000,0.000000,0.000000}%
\pgfsetfillcolor{currentfill}%
\pgfsetlinewidth{0.803000pt}%
\definecolor{currentstroke}{rgb}{0.000000,0.000000,0.000000}%
\pgfsetstrokecolor{currentstroke}%
\pgfsetdash{}{0pt}%
\pgfsys@defobject{currentmarker}{\pgfqpoint{-0.048611in}{0.000000in}}{\pgfqpoint{0.000000in}{0.000000in}}{%
\pgfpathmoveto{\pgfqpoint{0.000000in}{0.000000in}}%
\pgfpathlineto{\pgfqpoint{-0.048611in}{0.000000in}}%
\pgfusepath{stroke,fill}%
}%
\begin{pgfscope}%
\pgfsys@transformshift{7.640588in}{1.940139in}%
\pgfsys@useobject{currentmarker}{}%
\end{pgfscope}%
\end{pgfscope}%
\begin{pgfscope}%
\definecolor{textcolor}{rgb}{0.000000,0.000000,0.000000}%
\pgfsetstrokecolor{textcolor}%
\pgfsetfillcolor{textcolor}%
\pgftext[x=7.404477in, y=1.887378in, left, base]{\color{textcolor}\rmfamily\fontsize{10.000000}{12.000000}\selectfont \(\displaystyle 80\)}%
\end{pgfscope}%
\begin{pgfscope}%
\pgfsetbuttcap%
\pgfsetroundjoin%
\definecolor{currentfill}{rgb}{0.000000,0.000000,0.000000}%
\pgfsetfillcolor{currentfill}%
\pgfsetlinewidth{0.803000pt}%
\definecolor{currentstroke}{rgb}{0.000000,0.000000,0.000000}%
\pgfsetstrokecolor{currentstroke}%
\pgfsetdash{}{0pt}%
\pgfsys@defobject{currentmarker}{\pgfqpoint{-0.048611in}{0.000000in}}{\pgfqpoint{0.000000in}{0.000000in}}{%
\pgfpathmoveto{\pgfqpoint{0.000000in}{0.000000in}}%
\pgfpathlineto{\pgfqpoint{-0.048611in}{0.000000in}}%
\pgfusepath{stroke,fill}%
}%
\begin{pgfscope}%
\pgfsys@transformshift{7.640588in}{2.253127in}%
\pgfsys@useobject{currentmarker}{}%
\end{pgfscope}%
\end{pgfscope}%
\begin{pgfscope}%
\definecolor{textcolor}{rgb}{0.000000,0.000000,0.000000}%
\pgfsetstrokecolor{textcolor}%
\pgfsetfillcolor{textcolor}%
\pgftext[x=7.335032in, y=2.200365in, left, base]{\color{textcolor}\rmfamily\fontsize{10.000000}{12.000000}\selectfont \(\displaystyle 100\)}%
\end{pgfscope}%
\begin{pgfscope}%
\pgfsetbuttcap%
\pgfsetroundjoin%
\definecolor{currentfill}{rgb}{0.000000,0.000000,0.000000}%
\pgfsetfillcolor{currentfill}%
\pgfsetlinewidth{0.803000pt}%
\definecolor{currentstroke}{rgb}{0.000000,0.000000,0.000000}%
\pgfsetstrokecolor{currentstroke}%
\pgfsetdash{}{0pt}%
\pgfsys@defobject{currentmarker}{\pgfqpoint{-0.048611in}{0.000000in}}{\pgfqpoint{0.000000in}{0.000000in}}{%
\pgfpathmoveto{\pgfqpoint{0.000000in}{0.000000in}}%
\pgfpathlineto{\pgfqpoint{-0.048611in}{0.000000in}}%
\pgfusepath{stroke,fill}%
}%
\begin{pgfscope}%
\pgfsys@transformshift{7.640588in}{2.566114in}%
\pgfsys@useobject{currentmarker}{}%
\end{pgfscope}%
\end{pgfscope}%
\begin{pgfscope}%
\definecolor{textcolor}{rgb}{0.000000,0.000000,0.000000}%
\pgfsetstrokecolor{textcolor}%
\pgfsetfillcolor{textcolor}%
\pgftext[x=7.335032in, y=2.513353in, left, base]{\color{textcolor}\rmfamily\fontsize{10.000000}{12.000000}\selectfont \(\displaystyle 120\)}%
\end{pgfscope}%
\begin{pgfscope}%
\pgfsetbuttcap%
\pgfsetroundjoin%
\definecolor{currentfill}{rgb}{0.000000,0.000000,0.000000}%
\pgfsetfillcolor{currentfill}%
\pgfsetlinewidth{0.803000pt}%
\definecolor{currentstroke}{rgb}{0.000000,0.000000,0.000000}%
\pgfsetstrokecolor{currentstroke}%
\pgfsetdash{}{0pt}%
\pgfsys@defobject{currentmarker}{\pgfqpoint{-0.048611in}{0.000000in}}{\pgfqpoint{0.000000in}{0.000000in}}{%
\pgfpathmoveto{\pgfqpoint{0.000000in}{0.000000in}}%
\pgfpathlineto{\pgfqpoint{-0.048611in}{0.000000in}}%
\pgfusepath{stroke,fill}%
}%
\begin{pgfscope}%
\pgfsys@transformshift{7.640588in}{2.879102in}%
\pgfsys@useobject{currentmarker}{}%
\end{pgfscope}%
\end{pgfscope}%
\begin{pgfscope}%
\definecolor{textcolor}{rgb}{0.000000,0.000000,0.000000}%
\pgfsetstrokecolor{textcolor}%
\pgfsetfillcolor{textcolor}%
\pgftext[x=7.335032in, y=2.826340in, left, base]{\color{textcolor}\rmfamily\fontsize{10.000000}{12.000000}\selectfont \(\displaystyle 140\)}%
\end{pgfscope}%
\begin{pgfscope}%
\pgfsetbuttcap%
\pgfsetroundjoin%
\definecolor{currentfill}{rgb}{0.000000,0.000000,0.000000}%
\pgfsetfillcolor{currentfill}%
\pgfsetlinewidth{0.803000pt}%
\definecolor{currentstroke}{rgb}{0.000000,0.000000,0.000000}%
\pgfsetstrokecolor{currentstroke}%
\pgfsetdash{}{0pt}%
\pgfsys@defobject{currentmarker}{\pgfqpoint{-0.048611in}{0.000000in}}{\pgfqpoint{0.000000in}{0.000000in}}{%
\pgfpathmoveto{\pgfqpoint{0.000000in}{0.000000in}}%
\pgfpathlineto{\pgfqpoint{-0.048611in}{0.000000in}}%
\pgfusepath{stroke,fill}%
}%
\begin{pgfscope}%
\pgfsys@transformshift{7.640588in}{3.192089in}%
\pgfsys@useobject{currentmarker}{}%
\end{pgfscope}%
\end{pgfscope}%
\begin{pgfscope}%
\definecolor{textcolor}{rgb}{0.000000,0.000000,0.000000}%
\pgfsetstrokecolor{textcolor}%
\pgfsetfillcolor{textcolor}%
\pgftext[x=7.335032in, y=3.139328in, left, base]{\color{textcolor}\rmfamily\fontsize{10.000000}{12.000000}\selectfont \(\displaystyle 160\)}%
\end{pgfscope}%
\begin{pgfscope}%
\pgfsetrectcap%
\pgfsetmiterjoin%
\pgfsetlinewidth{0.803000pt}%
\definecolor{currentstroke}{rgb}{0.000000,0.000000,0.000000}%
\pgfsetstrokecolor{currentstroke}%
\pgfsetdash{}{0pt}%
\pgfpathmoveto{\pgfqpoint{7.640588in}{0.566125in}}%
\pgfpathlineto{\pgfqpoint{7.640588in}{3.251557in}}%
\pgfusepath{stroke}%
\end{pgfscope}%
\begin{pgfscope}%
\pgfsetrectcap%
\pgfsetmiterjoin%
\pgfsetlinewidth{0.803000pt}%
\definecolor{currentstroke}{rgb}{0.000000,0.000000,0.000000}%
\pgfsetstrokecolor{currentstroke}%
\pgfsetdash{}{0pt}%
\pgfpathmoveto{\pgfqpoint{13.339844in}{0.566125in}}%
\pgfpathlineto{\pgfqpoint{13.339844in}{3.251557in}}%
\pgfusepath{stroke}%
\end{pgfscope}%
\begin{pgfscope}%
\pgfsetrectcap%
\pgfsetmiterjoin%
\pgfsetlinewidth{0.803000pt}%
\definecolor{currentstroke}{rgb}{0.000000,0.000000,0.000000}%
\pgfsetstrokecolor{currentstroke}%
\pgfsetdash{}{0pt}%
\pgfpathmoveto{\pgfqpoint{7.640588in}{0.566125in}}%
\pgfpathlineto{\pgfqpoint{13.339844in}{0.566125in}}%
\pgfusepath{stroke}%
\end{pgfscope}%
\begin{pgfscope}%
\pgfsetrectcap%
\pgfsetmiterjoin%
\pgfsetlinewidth{0.803000pt}%
\definecolor{currentstroke}{rgb}{0.000000,0.000000,0.000000}%
\pgfsetstrokecolor{currentstroke}%
\pgfsetdash{}{0pt}%
\pgfpathmoveto{\pgfqpoint{7.640588in}{3.251557in}}%
\pgfpathlineto{\pgfqpoint{13.339844in}{3.251557in}}%
\pgfusepath{stroke}%
\end{pgfscope}%
\begin{pgfscope}%
\definecolor{textcolor}{rgb}{0.000000,0.000000,0.000000}%
\pgfsetstrokecolor{textcolor}%
\pgfsetfillcolor{textcolor}%
\pgftext[x=10.490216in,y=3.334890in,,base]{\color{textcolor}\rmfamily\fontsize{12.000000}{14.400000}\selectfont custom 1299.9}%
\end{pgfscope}%
\end{pgfpicture}%
\makeatother%
\endgroup%
}
\end{figure}

\begin{figure}[h]
    \caption{Results from performing custom on 500 ER,MR,NMR graphs with different values for the lookahead parameter.}
    \centering
    \scalebox{0.5}{%% Creator: Matplotlib, PGF backend
%%
%% To include the figure in your LaTeX document, write
%%   \input{<filename>.pgf}
%%
%% Make sure the required packages are loaded in your preamble
%%   \usepackage{pgf}
%%
%% and, on pdftex
%%   \usepackage[utf8]{inputenc}\DeclareUnicodeCharacter{2212}{-}
%%
%% or, on luatex and xetex
%%   \usepackage{unicode-math}
%%
%% Figures using additional raster images can only be included by \input if
%% they are in the same directory as the main LaTeX file. For loading figures
%% from other directories you can use the `import` package
%%   \usepackage{import}
%%
%% and then include the figures with
%%   \import{<path to file>}{<filename>.pgf}
%%
%% Matplotlib used the following preamble
%%   \usepackage{fontspec}
%%   \setmainfont{DejaVuSerif.ttf}[Path=/home/maks/.local/share/virtualenvs/CW3-zMJxnm_q/lib/python3.7/site-packages/matplotlib/mpl-data/fonts/ttf/]
%%   \setsansfont{DejaVuSans.ttf}[Path=/home/maks/.local/share/virtualenvs/CW3-zMJxnm_q/lib/python3.7/site-packages/matplotlib/mpl-data/fonts/ttf/]
%%   \setmonofont{DejaVuSansMono.ttf}[Path=/home/maks/.local/share/virtualenvs/CW3-zMJxnm_q/lib/python3.7/site-packages/matplotlib/mpl-data/fonts/ttf/]
%%
\begingroup%
\makeatletter%
\begin{pgfpicture}%
\pgfpathrectangle{\pgfpointorigin}{\pgfqpoint{13.660000in}{7.340000in}}%
\pgfusepath{use as bounding box, clip}%
\begin{pgfscope}%
\pgfsetbuttcap%
\pgfsetmiterjoin%
\definecolor{currentfill}{rgb}{1.000000,1.000000,1.000000}%
\pgfsetfillcolor{currentfill}%
\pgfsetlinewidth{0.000000pt}%
\definecolor{currentstroke}{rgb}{1.000000,1.000000,1.000000}%
\pgfsetstrokecolor{currentstroke}%
\pgfsetdash{}{0pt}%
\pgfpathmoveto{\pgfqpoint{0.000000in}{0.000000in}}%
\pgfpathlineto{\pgfqpoint{13.660000in}{0.000000in}}%
\pgfpathlineto{\pgfqpoint{13.660000in}{7.340000in}}%
\pgfpathlineto{\pgfqpoint{0.000000in}{7.340000in}}%
\pgfpathclose%
\pgfusepath{fill}%
\end{pgfscope}%
\begin{pgfscope}%
\pgfsetbuttcap%
\pgfsetmiterjoin%
\definecolor{currentfill}{rgb}{1.000000,1.000000,1.000000}%
\pgfsetfillcolor{currentfill}%
\pgfsetlinewidth{0.000000pt}%
\definecolor{currentstroke}{rgb}{0.000000,0.000000,0.000000}%
\pgfsetstrokecolor{currentstroke}%
\pgfsetstrokeopacity{0.000000}%
\pgfsetdash{}{0pt}%
\pgfpathmoveto{\pgfqpoint{1.499429in}{0.566125in}}%
\pgfpathlineto{\pgfqpoint{13.339844in}{0.566125in}}%
\pgfpathlineto{\pgfqpoint{13.339844in}{6.806869in}}%
\pgfpathlineto{\pgfqpoint{1.499429in}{6.806869in}}%
\pgfpathclose%
\pgfusepath{fill}%
\end{pgfscope}%
\begin{pgfscope}%
\pgfpathrectangle{\pgfqpoint{1.499429in}{0.566125in}}{\pgfqpoint{11.840415in}{6.240745in}}%
\pgfusepath{clip}%
\pgfsetbuttcap%
\pgfsetmiterjoin%
\definecolor{currentfill}{rgb}{1.000000,0.650980,0.000000}%
\pgfsetfillcolor{currentfill}%
\pgfsetlinewidth{0.000000pt}%
\definecolor{currentstroke}{rgb}{0.000000,0.000000,0.000000}%
\pgfsetstrokecolor{currentstroke}%
\pgfsetstrokeopacity{0.000000}%
\pgfsetdash{}{0pt}%
\pgfpathmoveto{\pgfqpoint{2.037629in}{0.566125in}}%
\pgfpathlineto{\pgfqpoint{2.670807in}{0.566125in}}%
\pgfpathlineto{\pgfqpoint{2.670807in}{1.754059in}}%
\pgfpathlineto{\pgfqpoint{2.037629in}{1.754059in}}%
\pgfpathclose%
\pgfusepath{fill}%
\end{pgfscope}%
\begin{pgfscope}%
\pgfpathrectangle{\pgfqpoint{1.499429in}{0.566125in}}{\pgfqpoint{11.840415in}{6.240745in}}%
\pgfusepath{clip}%
\pgfsetbuttcap%
\pgfsetmiterjoin%
\definecolor{currentfill}{rgb}{1.000000,0.650980,0.000000}%
\pgfsetfillcolor{currentfill}%
\pgfsetlinewidth{0.000000pt}%
\definecolor{currentstroke}{rgb}{0.000000,0.000000,0.000000}%
\pgfsetstrokecolor{currentstroke}%
\pgfsetstrokeopacity{0.000000}%
\pgfsetdash{}{0pt}%
\pgfpathmoveto{\pgfqpoint{3.303984in}{0.566125in}}%
\pgfpathlineto{\pgfqpoint{3.937161in}{0.566125in}}%
\pgfpathlineto{\pgfqpoint{3.937161in}{1.754587in}}%
\pgfpathlineto{\pgfqpoint{3.303984in}{1.754587in}}%
\pgfpathclose%
\pgfusepath{fill}%
\end{pgfscope}%
\begin{pgfscope}%
\pgfpathrectangle{\pgfqpoint{1.499429in}{0.566125in}}{\pgfqpoint{11.840415in}{6.240745in}}%
\pgfusepath{clip}%
\pgfsetbuttcap%
\pgfsetmiterjoin%
\definecolor{currentfill}{rgb}{1.000000,0.650980,0.000000}%
\pgfsetfillcolor{currentfill}%
\pgfsetlinewidth{0.000000pt}%
\definecolor{currentstroke}{rgb}{0.000000,0.000000,0.000000}%
\pgfsetstrokecolor{currentstroke}%
\pgfsetstrokeopacity{0.000000}%
\pgfsetdash{}{0pt}%
\pgfpathmoveto{\pgfqpoint{4.570338in}{0.566125in}}%
\pgfpathlineto{\pgfqpoint{5.203516in}{0.566125in}}%
\pgfpathlineto{\pgfqpoint{5.203516in}{1.754011in}}%
\pgfpathlineto{\pgfqpoint{4.570338in}{1.754011in}}%
\pgfpathclose%
\pgfusepath{fill}%
\end{pgfscope}%
\begin{pgfscope}%
\pgfpathrectangle{\pgfqpoint{1.499429in}{0.566125in}}{\pgfqpoint{11.840415in}{6.240745in}}%
\pgfusepath{clip}%
\pgfsetbuttcap%
\pgfsetmiterjoin%
\definecolor{currentfill}{rgb}{1.000000,0.650980,0.000000}%
\pgfsetfillcolor{currentfill}%
\pgfsetlinewidth{0.000000pt}%
\definecolor{currentstroke}{rgb}{0.000000,0.000000,0.000000}%
\pgfsetstrokecolor{currentstroke}%
\pgfsetstrokeopacity{0.000000}%
\pgfsetdash{}{0pt}%
\pgfpathmoveto{\pgfqpoint{5.836693in}{0.566125in}}%
\pgfpathlineto{\pgfqpoint{6.469870in}{0.566125in}}%
\pgfpathlineto{\pgfqpoint{6.469870in}{6.504999in}}%
\pgfpathlineto{\pgfqpoint{5.836693in}{6.504999in}}%
\pgfpathclose%
\pgfusepath{fill}%
\end{pgfscope}%
\begin{pgfscope}%
\pgfpathrectangle{\pgfqpoint{1.499429in}{0.566125in}}{\pgfqpoint{11.840415in}{6.240745in}}%
\pgfusepath{clip}%
\pgfsetbuttcap%
\pgfsetmiterjoin%
\definecolor{currentfill}{rgb}{1.000000,0.650980,0.000000}%
\pgfsetfillcolor{currentfill}%
\pgfsetlinewidth{0.000000pt}%
\definecolor{currentstroke}{rgb}{0.000000,0.000000,0.000000}%
\pgfsetstrokecolor{currentstroke}%
\pgfsetstrokeopacity{0.000000}%
\pgfsetdash{}{0pt}%
\pgfpathmoveto{\pgfqpoint{7.103048in}{0.566125in}}%
\pgfpathlineto{\pgfqpoint{7.736225in}{0.566125in}}%
\pgfpathlineto{\pgfqpoint{7.736225in}{6.509691in}}%
\pgfpathlineto{\pgfqpoint{7.103048in}{6.509691in}}%
\pgfpathclose%
\pgfusepath{fill}%
\end{pgfscope}%
\begin{pgfscope}%
\pgfpathrectangle{\pgfqpoint{1.499429in}{0.566125in}}{\pgfqpoint{11.840415in}{6.240745in}}%
\pgfusepath{clip}%
\pgfsetbuttcap%
\pgfsetmiterjoin%
\definecolor{currentfill}{rgb}{1.000000,0.650980,0.000000}%
\pgfsetfillcolor{currentfill}%
\pgfsetlinewidth{0.000000pt}%
\definecolor{currentstroke}{rgb}{0.000000,0.000000,0.000000}%
\pgfsetstrokecolor{currentstroke}%
\pgfsetstrokeopacity{0.000000}%
\pgfsetdash{}{0pt}%
\pgfpathmoveto{\pgfqpoint{8.369402in}{0.566125in}}%
\pgfpathlineto{\pgfqpoint{9.002579in}{0.566125in}}%
\pgfpathlineto{\pgfqpoint{9.002579in}{6.509200in}}%
\pgfpathlineto{\pgfqpoint{8.369402in}{6.509200in}}%
\pgfpathclose%
\pgfusepath{fill}%
\end{pgfscope}%
\begin{pgfscope}%
\pgfpathrectangle{\pgfqpoint{1.499429in}{0.566125in}}{\pgfqpoint{11.840415in}{6.240745in}}%
\pgfusepath{clip}%
\pgfsetbuttcap%
\pgfsetmiterjoin%
\definecolor{currentfill}{rgb}{1.000000,0.650980,0.000000}%
\pgfsetfillcolor{currentfill}%
\pgfsetlinewidth{0.000000pt}%
\definecolor{currentstroke}{rgb}{0.000000,0.000000,0.000000}%
\pgfsetstrokecolor{currentstroke}%
\pgfsetstrokeopacity{0.000000}%
\pgfsetdash{}{0pt}%
\pgfpathmoveto{\pgfqpoint{9.635757in}{0.566125in}}%
\pgfpathlineto{\pgfqpoint{10.268934in}{0.566125in}}%
\pgfpathlineto{\pgfqpoint{10.268934in}{6.504999in}}%
\pgfpathlineto{\pgfqpoint{9.635757in}{6.504999in}}%
\pgfpathclose%
\pgfusepath{fill}%
\end{pgfscope}%
\begin{pgfscope}%
\pgfpathrectangle{\pgfqpoint{1.499429in}{0.566125in}}{\pgfqpoint{11.840415in}{6.240745in}}%
\pgfusepath{clip}%
\pgfsetbuttcap%
\pgfsetmiterjoin%
\definecolor{currentfill}{rgb}{1.000000,0.650980,0.000000}%
\pgfsetfillcolor{currentfill}%
\pgfsetlinewidth{0.000000pt}%
\definecolor{currentstroke}{rgb}{0.000000,0.000000,0.000000}%
\pgfsetstrokecolor{currentstroke}%
\pgfsetstrokeopacity{0.000000}%
\pgfsetdash{}{0pt}%
\pgfpathmoveto{\pgfqpoint{10.902111in}{0.566125in}}%
\pgfpathlineto{\pgfqpoint{11.535289in}{0.566125in}}%
\pgfpathlineto{\pgfqpoint{11.535289in}{6.509691in}}%
\pgfpathlineto{\pgfqpoint{10.902111in}{6.509691in}}%
\pgfpathclose%
\pgfusepath{fill}%
\end{pgfscope}%
\begin{pgfscope}%
\pgfpathrectangle{\pgfqpoint{1.499429in}{0.566125in}}{\pgfqpoint{11.840415in}{6.240745in}}%
\pgfusepath{clip}%
\pgfsetbuttcap%
\pgfsetmiterjoin%
\definecolor{currentfill}{rgb}{1.000000,0.650980,0.000000}%
\pgfsetfillcolor{currentfill}%
\pgfsetlinewidth{0.000000pt}%
\definecolor{currentstroke}{rgb}{0.000000,0.000000,0.000000}%
\pgfsetstrokecolor{currentstroke}%
\pgfsetstrokeopacity{0.000000}%
\pgfsetdash{}{0pt}%
\pgfpathmoveto{\pgfqpoint{12.168466in}{0.566125in}}%
\pgfpathlineto{\pgfqpoint{12.801643in}{0.566125in}}%
\pgfpathlineto{\pgfqpoint{12.801643in}{6.509200in}}%
\pgfpathlineto{\pgfqpoint{12.168466in}{6.509200in}}%
\pgfpathclose%
\pgfusepath{fill}%
\end{pgfscope}%
\begin{pgfscope}%
\pgfsetbuttcap%
\pgfsetroundjoin%
\definecolor{currentfill}{rgb}{0.000000,0.000000,0.000000}%
\pgfsetfillcolor{currentfill}%
\pgfsetlinewidth{0.803000pt}%
\definecolor{currentstroke}{rgb}{0.000000,0.000000,0.000000}%
\pgfsetstrokecolor{currentstroke}%
\pgfsetdash{}{0pt}%
\pgfsys@defobject{currentmarker}{\pgfqpoint{0.000000in}{-0.048611in}}{\pgfqpoint{0.000000in}{0.000000in}}{%
\pgfpathmoveto{\pgfqpoint{0.000000in}{0.000000in}}%
\pgfpathlineto{\pgfqpoint{0.000000in}{-0.048611in}}%
\pgfusepath{stroke,fill}%
}%
\begin{pgfscope}%
\pgfsys@transformshift{2.354218in}{0.566125in}%
\pgfsys@useobject{currentmarker}{}%
\end{pgfscope}%
\end{pgfscope}%
\begin{pgfscope}%
\definecolor{textcolor}{rgb}{0.000000,0.000000,0.000000}%
\pgfsetstrokecolor{textcolor}%
\pgfsetfillcolor{textcolor}%
\pgftext[x=2.354218in,y=0.468902in,,top]{\color{textcolor}\rmfamily\fontsize{10.000000}{12.000000}\selectfont ER:1}%
\end{pgfscope}%
\begin{pgfscope}%
\pgfsetbuttcap%
\pgfsetroundjoin%
\definecolor{currentfill}{rgb}{0.000000,0.000000,0.000000}%
\pgfsetfillcolor{currentfill}%
\pgfsetlinewidth{0.803000pt}%
\definecolor{currentstroke}{rgb}{0.000000,0.000000,0.000000}%
\pgfsetstrokecolor{currentstroke}%
\pgfsetdash{}{0pt}%
\pgfsys@defobject{currentmarker}{\pgfqpoint{0.000000in}{-0.048611in}}{\pgfqpoint{0.000000in}{0.000000in}}{%
\pgfpathmoveto{\pgfqpoint{0.000000in}{0.000000in}}%
\pgfpathlineto{\pgfqpoint{0.000000in}{-0.048611in}}%
\pgfusepath{stroke,fill}%
}%
\begin{pgfscope}%
\pgfsys@transformshift{3.620572in}{0.566125in}%
\pgfsys@useobject{currentmarker}{}%
\end{pgfscope}%
\end{pgfscope}%
\begin{pgfscope}%
\definecolor{textcolor}{rgb}{0.000000,0.000000,0.000000}%
\pgfsetstrokecolor{textcolor}%
\pgfsetfillcolor{textcolor}%
\pgftext[x=3.620572in,y=0.468902in,,top]{\color{textcolor}\rmfamily\fontsize{10.000000}{12.000000}\selectfont ER:2}%
\end{pgfscope}%
\begin{pgfscope}%
\pgfsetbuttcap%
\pgfsetroundjoin%
\definecolor{currentfill}{rgb}{0.000000,0.000000,0.000000}%
\pgfsetfillcolor{currentfill}%
\pgfsetlinewidth{0.803000pt}%
\definecolor{currentstroke}{rgb}{0.000000,0.000000,0.000000}%
\pgfsetstrokecolor{currentstroke}%
\pgfsetdash{}{0pt}%
\pgfsys@defobject{currentmarker}{\pgfqpoint{0.000000in}{-0.048611in}}{\pgfqpoint{0.000000in}{0.000000in}}{%
\pgfpathmoveto{\pgfqpoint{0.000000in}{0.000000in}}%
\pgfpathlineto{\pgfqpoint{0.000000in}{-0.048611in}}%
\pgfusepath{stroke,fill}%
}%
\begin{pgfscope}%
\pgfsys@transformshift{4.886927in}{0.566125in}%
\pgfsys@useobject{currentmarker}{}%
\end{pgfscope}%
\end{pgfscope}%
\begin{pgfscope}%
\definecolor{textcolor}{rgb}{0.000000,0.000000,0.000000}%
\pgfsetstrokecolor{textcolor}%
\pgfsetfillcolor{textcolor}%
\pgftext[x=4.886927in,y=0.468902in,,top]{\color{textcolor}\rmfamily\fontsize{10.000000}{12.000000}\selectfont ER:3}%
\end{pgfscope}%
\begin{pgfscope}%
\pgfsetbuttcap%
\pgfsetroundjoin%
\definecolor{currentfill}{rgb}{0.000000,0.000000,0.000000}%
\pgfsetfillcolor{currentfill}%
\pgfsetlinewidth{0.803000pt}%
\definecolor{currentstroke}{rgb}{0.000000,0.000000,0.000000}%
\pgfsetstrokecolor{currentstroke}%
\pgfsetdash{}{0pt}%
\pgfsys@defobject{currentmarker}{\pgfqpoint{0.000000in}{-0.048611in}}{\pgfqpoint{0.000000in}{0.000000in}}{%
\pgfpathmoveto{\pgfqpoint{0.000000in}{0.000000in}}%
\pgfpathlineto{\pgfqpoint{0.000000in}{-0.048611in}}%
\pgfusepath{stroke,fill}%
}%
\begin{pgfscope}%
\pgfsys@transformshift{6.153282in}{0.566125in}%
\pgfsys@useobject{currentmarker}{}%
\end{pgfscope}%
\end{pgfscope}%
\begin{pgfscope}%
\definecolor{textcolor}{rgb}{0.000000,0.000000,0.000000}%
\pgfsetstrokecolor{textcolor}%
\pgfsetfillcolor{textcolor}%
\pgftext[x=6.153282in,y=0.468902in,,top]{\color{textcolor}\rmfamily\fontsize{10.000000}{12.000000}\selectfont MR:1}%
\end{pgfscope}%
\begin{pgfscope}%
\pgfsetbuttcap%
\pgfsetroundjoin%
\definecolor{currentfill}{rgb}{0.000000,0.000000,0.000000}%
\pgfsetfillcolor{currentfill}%
\pgfsetlinewidth{0.803000pt}%
\definecolor{currentstroke}{rgb}{0.000000,0.000000,0.000000}%
\pgfsetstrokecolor{currentstroke}%
\pgfsetdash{}{0pt}%
\pgfsys@defobject{currentmarker}{\pgfqpoint{0.000000in}{-0.048611in}}{\pgfqpoint{0.000000in}{0.000000in}}{%
\pgfpathmoveto{\pgfqpoint{0.000000in}{0.000000in}}%
\pgfpathlineto{\pgfqpoint{0.000000in}{-0.048611in}}%
\pgfusepath{stroke,fill}%
}%
\begin{pgfscope}%
\pgfsys@transformshift{7.419636in}{0.566125in}%
\pgfsys@useobject{currentmarker}{}%
\end{pgfscope}%
\end{pgfscope}%
\begin{pgfscope}%
\definecolor{textcolor}{rgb}{0.000000,0.000000,0.000000}%
\pgfsetstrokecolor{textcolor}%
\pgfsetfillcolor{textcolor}%
\pgftext[x=7.419636in,y=0.468902in,,top]{\color{textcolor}\rmfamily\fontsize{10.000000}{12.000000}\selectfont MR:2}%
\end{pgfscope}%
\begin{pgfscope}%
\pgfsetbuttcap%
\pgfsetroundjoin%
\definecolor{currentfill}{rgb}{0.000000,0.000000,0.000000}%
\pgfsetfillcolor{currentfill}%
\pgfsetlinewidth{0.803000pt}%
\definecolor{currentstroke}{rgb}{0.000000,0.000000,0.000000}%
\pgfsetstrokecolor{currentstroke}%
\pgfsetdash{}{0pt}%
\pgfsys@defobject{currentmarker}{\pgfqpoint{0.000000in}{-0.048611in}}{\pgfqpoint{0.000000in}{0.000000in}}{%
\pgfpathmoveto{\pgfqpoint{0.000000in}{0.000000in}}%
\pgfpathlineto{\pgfqpoint{0.000000in}{-0.048611in}}%
\pgfusepath{stroke,fill}%
}%
\begin{pgfscope}%
\pgfsys@transformshift{8.685991in}{0.566125in}%
\pgfsys@useobject{currentmarker}{}%
\end{pgfscope}%
\end{pgfscope}%
\begin{pgfscope}%
\definecolor{textcolor}{rgb}{0.000000,0.000000,0.000000}%
\pgfsetstrokecolor{textcolor}%
\pgfsetfillcolor{textcolor}%
\pgftext[x=8.685991in,y=0.468902in,,top]{\color{textcolor}\rmfamily\fontsize{10.000000}{12.000000}\selectfont MR:3}%
\end{pgfscope}%
\begin{pgfscope}%
\pgfsetbuttcap%
\pgfsetroundjoin%
\definecolor{currentfill}{rgb}{0.000000,0.000000,0.000000}%
\pgfsetfillcolor{currentfill}%
\pgfsetlinewidth{0.803000pt}%
\definecolor{currentstroke}{rgb}{0.000000,0.000000,0.000000}%
\pgfsetstrokecolor{currentstroke}%
\pgfsetdash{}{0pt}%
\pgfsys@defobject{currentmarker}{\pgfqpoint{0.000000in}{-0.048611in}}{\pgfqpoint{0.000000in}{0.000000in}}{%
\pgfpathmoveto{\pgfqpoint{0.000000in}{0.000000in}}%
\pgfpathlineto{\pgfqpoint{0.000000in}{-0.048611in}}%
\pgfusepath{stroke,fill}%
}%
\begin{pgfscope}%
\pgfsys@transformshift{9.952345in}{0.566125in}%
\pgfsys@useobject{currentmarker}{}%
\end{pgfscope}%
\end{pgfscope}%
\begin{pgfscope}%
\definecolor{textcolor}{rgb}{0.000000,0.000000,0.000000}%
\pgfsetstrokecolor{textcolor}%
\pgfsetfillcolor{textcolor}%
\pgftext[x=9.952345in,y=0.468902in,,top]{\color{textcolor}\rmfamily\fontsize{10.000000}{12.000000}\selectfont NMR:1}%
\end{pgfscope}%
\begin{pgfscope}%
\pgfsetbuttcap%
\pgfsetroundjoin%
\definecolor{currentfill}{rgb}{0.000000,0.000000,0.000000}%
\pgfsetfillcolor{currentfill}%
\pgfsetlinewidth{0.803000pt}%
\definecolor{currentstroke}{rgb}{0.000000,0.000000,0.000000}%
\pgfsetstrokecolor{currentstroke}%
\pgfsetdash{}{0pt}%
\pgfsys@defobject{currentmarker}{\pgfqpoint{0.000000in}{-0.048611in}}{\pgfqpoint{0.000000in}{0.000000in}}{%
\pgfpathmoveto{\pgfqpoint{0.000000in}{0.000000in}}%
\pgfpathlineto{\pgfqpoint{0.000000in}{-0.048611in}}%
\pgfusepath{stroke,fill}%
}%
\begin{pgfscope}%
\pgfsys@transformshift{11.218700in}{0.566125in}%
\pgfsys@useobject{currentmarker}{}%
\end{pgfscope}%
\end{pgfscope}%
\begin{pgfscope}%
\definecolor{textcolor}{rgb}{0.000000,0.000000,0.000000}%
\pgfsetstrokecolor{textcolor}%
\pgfsetfillcolor{textcolor}%
\pgftext[x=11.218700in,y=0.468902in,,top]{\color{textcolor}\rmfamily\fontsize{10.000000}{12.000000}\selectfont NMR:2}%
\end{pgfscope}%
\begin{pgfscope}%
\pgfsetbuttcap%
\pgfsetroundjoin%
\definecolor{currentfill}{rgb}{0.000000,0.000000,0.000000}%
\pgfsetfillcolor{currentfill}%
\pgfsetlinewidth{0.803000pt}%
\definecolor{currentstroke}{rgb}{0.000000,0.000000,0.000000}%
\pgfsetstrokecolor{currentstroke}%
\pgfsetdash{}{0pt}%
\pgfsys@defobject{currentmarker}{\pgfqpoint{0.000000in}{-0.048611in}}{\pgfqpoint{0.000000in}{0.000000in}}{%
\pgfpathmoveto{\pgfqpoint{0.000000in}{0.000000in}}%
\pgfpathlineto{\pgfqpoint{0.000000in}{-0.048611in}}%
\pgfusepath{stroke,fill}%
}%
\begin{pgfscope}%
\pgfsys@transformshift{12.485054in}{0.566125in}%
\pgfsys@useobject{currentmarker}{}%
\end{pgfscope}%
\end{pgfscope}%
\begin{pgfscope}%
\definecolor{textcolor}{rgb}{0.000000,0.000000,0.000000}%
\pgfsetstrokecolor{textcolor}%
\pgfsetfillcolor{textcolor}%
\pgftext[x=12.485054in,y=0.468902in,,top]{\color{textcolor}\rmfamily\fontsize{10.000000}{12.000000}\selectfont NMR:3}%
\end{pgfscope}%
\begin{pgfscope}%
\pgfsetbuttcap%
\pgfsetroundjoin%
\definecolor{currentfill}{rgb}{0.000000,0.000000,0.000000}%
\pgfsetfillcolor{currentfill}%
\pgfsetlinewidth{0.803000pt}%
\definecolor{currentstroke}{rgb}{0.000000,0.000000,0.000000}%
\pgfsetstrokecolor{currentstroke}%
\pgfsetdash{}{0pt}%
\pgfsys@defobject{currentmarker}{\pgfqpoint{-0.048611in}{0.000000in}}{\pgfqpoint{0.000000in}{0.000000in}}{%
\pgfpathmoveto{\pgfqpoint{0.000000in}{0.000000in}}%
\pgfpathlineto{\pgfqpoint{-0.048611in}{0.000000in}}%
\pgfusepath{stroke,fill}%
}%
\begin{pgfscope}%
\pgfsys@transformshift{1.499429in}{0.566125in}%
\pgfsys@useobject{currentmarker}{}%
\end{pgfscope}%
\end{pgfscope}%
\begin{pgfscope}%
\definecolor{textcolor}{rgb}{0.000000,0.000000,0.000000}%
\pgfsetstrokecolor{textcolor}%
\pgfsetfillcolor{textcolor}%
\pgftext[x=1.332762in, y=0.513363in, left, base]{\color{textcolor}\rmfamily\fontsize{10.000000}{12.000000}\selectfont \(\displaystyle 0\)}%
\end{pgfscope}%
\begin{pgfscope}%
\pgfsetbuttcap%
\pgfsetroundjoin%
\definecolor{currentfill}{rgb}{0.000000,0.000000,0.000000}%
\pgfsetfillcolor{currentfill}%
\pgfsetlinewidth{0.803000pt}%
\definecolor{currentstroke}{rgb}{0.000000,0.000000,0.000000}%
\pgfsetstrokecolor{currentstroke}%
\pgfsetdash{}{0pt}%
\pgfsys@defobject{currentmarker}{\pgfqpoint{-0.048611in}{0.000000in}}{\pgfqpoint{0.000000in}{0.000000in}}{%
\pgfpathmoveto{\pgfqpoint{0.000000in}{0.000000in}}%
\pgfpathlineto{\pgfqpoint{-0.048611in}{0.000000in}}%
\pgfusepath{stroke,fill}%
}%
\begin{pgfscope}%
\pgfsys@transformshift{1.499429in}{1.406373in}%
\pgfsys@useobject{currentmarker}{}%
\end{pgfscope}%
\end{pgfscope}%
\begin{pgfscope}%
\definecolor{textcolor}{rgb}{0.000000,0.000000,0.000000}%
\pgfsetstrokecolor{textcolor}%
\pgfsetfillcolor{textcolor}%
\pgftext[x=1.193872in, y=1.353611in, left, base]{\color{textcolor}\rmfamily\fontsize{10.000000}{12.000000}\selectfont \(\displaystyle 500\)}%
\end{pgfscope}%
\begin{pgfscope}%
\pgfsetbuttcap%
\pgfsetroundjoin%
\definecolor{currentfill}{rgb}{0.000000,0.000000,0.000000}%
\pgfsetfillcolor{currentfill}%
\pgfsetlinewidth{0.803000pt}%
\definecolor{currentstroke}{rgb}{0.000000,0.000000,0.000000}%
\pgfsetstrokecolor{currentstroke}%
\pgfsetdash{}{0pt}%
\pgfsys@defobject{currentmarker}{\pgfqpoint{-0.048611in}{0.000000in}}{\pgfqpoint{0.000000in}{0.000000in}}{%
\pgfpathmoveto{\pgfqpoint{0.000000in}{0.000000in}}%
\pgfpathlineto{\pgfqpoint{-0.048611in}{0.000000in}}%
\pgfusepath{stroke,fill}%
}%
\begin{pgfscope}%
\pgfsys@transformshift{1.499429in}{2.246621in}%
\pgfsys@useobject{currentmarker}{}%
\end{pgfscope}%
\end{pgfscope}%
\begin{pgfscope}%
\definecolor{textcolor}{rgb}{0.000000,0.000000,0.000000}%
\pgfsetstrokecolor{textcolor}%
\pgfsetfillcolor{textcolor}%
\pgftext[x=1.124428in, y=2.193860in, left, base]{\color{textcolor}\rmfamily\fontsize{10.000000}{12.000000}\selectfont \(\displaystyle 1000\)}%
\end{pgfscope}%
\begin{pgfscope}%
\pgfsetbuttcap%
\pgfsetroundjoin%
\definecolor{currentfill}{rgb}{0.000000,0.000000,0.000000}%
\pgfsetfillcolor{currentfill}%
\pgfsetlinewidth{0.803000pt}%
\definecolor{currentstroke}{rgb}{0.000000,0.000000,0.000000}%
\pgfsetstrokecolor{currentstroke}%
\pgfsetdash{}{0pt}%
\pgfsys@defobject{currentmarker}{\pgfqpoint{-0.048611in}{0.000000in}}{\pgfqpoint{0.000000in}{0.000000in}}{%
\pgfpathmoveto{\pgfqpoint{0.000000in}{0.000000in}}%
\pgfpathlineto{\pgfqpoint{-0.048611in}{0.000000in}}%
\pgfusepath{stroke,fill}%
}%
\begin{pgfscope}%
\pgfsys@transformshift{1.499429in}{3.086869in}%
\pgfsys@useobject{currentmarker}{}%
\end{pgfscope}%
\end{pgfscope}%
\begin{pgfscope}%
\definecolor{textcolor}{rgb}{0.000000,0.000000,0.000000}%
\pgfsetstrokecolor{textcolor}%
\pgfsetfillcolor{textcolor}%
\pgftext[x=1.124428in, y=3.034108in, left, base]{\color{textcolor}\rmfamily\fontsize{10.000000}{12.000000}\selectfont \(\displaystyle 1500\)}%
\end{pgfscope}%
\begin{pgfscope}%
\pgfsetbuttcap%
\pgfsetroundjoin%
\definecolor{currentfill}{rgb}{0.000000,0.000000,0.000000}%
\pgfsetfillcolor{currentfill}%
\pgfsetlinewidth{0.803000pt}%
\definecolor{currentstroke}{rgb}{0.000000,0.000000,0.000000}%
\pgfsetstrokecolor{currentstroke}%
\pgfsetdash{}{0pt}%
\pgfsys@defobject{currentmarker}{\pgfqpoint{-0.048611in}{0.000000in}}{\pgfqpoint{0.000000in}{0.000000in}}{%
\pgfpathmoveto{\pgfqpoint{0.000000in}{0.000000in}}%
\pgfpathlineto{\pgfqpoint{-0.048611in}{0.000000in}}%
\pgfusepath{stroke,fill}%
}%
\begin{pgfscope}%
\pgfsys@transformshift{1.499429in}{3.927117in}%
\pgfsys@useobject{currentmarker}{}%
\end{pgfscope}%
\end{pgfscope}%
\begin{pgfscope}%
\definecolor{textcolor}{rgb}{0.000000,0.000000,0.000000}%
\pgfsetstrokecolor{textcolor}%
\pgfsetfillcolor{textcolor}%
\pgftext[x=1.124428in, y=3.874356in, left, base]{\color{textcolor}\rmfamily\fontsize{10.000000}{12.000000}\selectfont \(\displaystyle 2000\)}%
\end{pgfscope}%
\begin{pgfscope}%
\pgfsetbuttcap%
\pgfsetroundjoin%
\definecolor{currentfill}{rgb}{0.000000,0.000000,0.000000}%
\pgfsetfillcolor{currentfill}%
\pgfsetlinewidth{0.803000pt}%
\definecolor{currentstroke}{rgb}{0.000000,0.000000,0.000000}%
\pgfsetstrokecolor{currentstroke}%
\pgfsetdash{}{0pt}%
\pgfsys@defobject{currentmarker}{\pgfqpoint{-0.048611in}{0.000000in}}{\pgfqpoint{0.000000in}{0.000000in}}{%
\pgfpathmoveto{\pgfqpoint{0.000000in}{0.000000in}}%
\pgfpathlineto{\pgfqpoint{-0.048611in}{0.000000in}}%
\pgfusepath{stroke,fill}%
}%
\begin{pgfscope}%
\pgfsys@transformshift{1.499429in}{4.767366in}%
\pgfsys@useobject{currentmarker}{}%
\end{pgfscope}%
\end{pgfscope}%
\begin{pgfscope}%
\definecolor{textcolor}{rgb}{0.000000,0.000000,0.000000}%
\pgfsetstrokecolor{textcolor}%
\pgfsetfillcolor{textcolor}%
\pgftext[x=1.124428in, y=4.714604in, left, base]{\color{textcolor}\rmfamily\fontsize{10.000000}{12.000000}\selectfont \(\displaystyle 2500\)}%
\end{pgfscope}%
\begin{pgfscope}%
\pgfsetbuttcap%
\pgfsetroundjoin%
\definecolor{currentfill}{rgb}{0.000000,0.000000,0.000000}%
\pgfsetfillcolor{currentfill}%
\pgfsetlinewidth{0.803000pt}%
\definecolor{currentstroke}{rgb}{0.000000,0.000000,0.000000}%
\pgfsetstrokecolor{currentstroke}%
\pgfsetdash{}{0pt}%
\pgfsys@defobject{currentmarker}{\pgfqpoint{-0.048611in}{0.000000in}}{\pgfqpoint{0.000000in}{0.000000in}}{%
\pgfpathmoveto{\pgfqpoint{0.000000in}{0.000000in}}%
\pgfpathlineto{\pgfqpoint{-0.048611in}{0.000000in}}%
\pgfusepath{stroke,fill}%
}%
\begin{pgfscope}%
\pgfsys@transformshift{1.499429in}{5.607614in}%
\pgfsys@useobject{currentmarker}{}%
\end{pgfscope}%
\end{pgfscope}%
\begin{pgfscope}%
\definecolor{textcolor}{rgb}{0.000000,0.000000,0.000000}%
\pgfsetstrokecolor{textcolor}%
\pgfsetfillcolor{textcolor}%
\pgftext[x=1.124428in, y=5.554852in, left, base]{\color{textcolor}\rmfamily\fontsize{10.000000}{12.000000}\selectfont \(\displaystyle 3000\)}%
\end{pgfscope}%
\begin{pgfscope}%
\pgfsetbuttcap%
\pgfsetroundjoin%
\definecolor{currentfill}{rgb}{0.000000,0.000000,0.000000}%
\pgfsetfillcolor{currentfill}%
\pgfsetlinewidth{0.803000pt}%
\definecolor{currentstroke}{rgb}{0.000000,0.000000,0.000000}%
\pgfsetstrokecolor{currentstroke}%
\pgfsetdash{}{0pt}%
\pgfsys@defobject{currentmarker}{\pgfqpoint{-0.048611in}{0.000000in}}{\pgfqpoint{0.000000in}{0.000000in}}{%
\pgfpathmoveto{\pgfqpoint{0.000000in}{0.000000in}}%
\pgfpathlineto{\pgfqpoint{-0.048611in}{0.000000in}}%
\pgfusepath{stroke,fill}%
}%
\begin{pgfscope}%
\pgfsys@transformshift{1.499429in}{6.447862in}%
\pgfsys@useobject{currentmarker}{}%
\end{pgfscope}%
\end{pgfscope}%
\begin{pgfscope}%
\definecolor{textcolor}{rgb}{0.000000,0.000000,0.000000}%
\pgfsetstrokecolor{textcolor}%
\pgfsetfillcolor{textcolor}%
\pgftext[x=1.124428in, y=6.395101in, left, base]{\color{textcolor}\rmfamily\fontsize{10.000000}{12.000000}\selectfont \(\displaystyle 3500\)}%
\end{pgfscope}%
\begin{pgfscope}%
\definecolor{textcolor}{rgb}{0.000000,0.000000,0.000000}%
\pgfsetstrokecolor{textcolor}%
\pgfsetfillcolor{textcolor}%
\pgftext[x=1.068872in,y=3.686497in,,bottom,rotate=90.000000]{\color{textcolor}\rmfamily\fontsize{10.000000}{12.000000}\selectfont Average Tour Cost}%
\end{pgfscope}%
\begin{pgfscope}%
\pgfsetrectcap%
\pgfsetmiterjoin%
\pgfsetlinewidth{0.803000pt}%
\definecolor{currentstroke}{rgb}{0.000000,0.000000,0.000000}%
\pgfsetstrokecolor{currentstroke}%
\pgfsetdash{}{0pt}%
\pgfpathmoveto{\pgfqpoint{1.499429in}{0.566125in}}%
\pgfpathlineto{\pgfqpoint{1.499429in}{6.806869in}}%
\pgfusepath{stroke}%
\end{pgfscope}%
\begin{pgfscope}%
\pgfsetrectcap%
\pgfsetmiterjoin%
\pgfsetlinewidth{0.803000pt}%
\definecolor{currentstroke}{rgb}{0.000000,0.000000,0.000000}%
\pgfsetstrokecolor{currentstroke}%
\pgfsetdash{}{0pt}%
\pgfpathmoveto{\pgfqpoint{13.339844in}{0.566125in}}%
\pgfpathlineto{\pgfqpoint{13.339844in}{6.806869in}}%
\pgfusepath{stroke}%
\end{pgfscope}%
\begin{pgfscope}%
\pgfsetrectcap%
\pgfsetmiterjoin%
\pgfsetlinewidth{0.803000pt}%
\definecolor{currentstroke}{rgb}{0.000000,0.000000,0.000000}%
\pgfsetstrokecolor{currentstroke}%
\pgfsetdash{}{0pt}%
\pgfpathmoveto{\pgfqpoint{1.499429in}{0.566125in}}%
\pgfpathlineto{\pgfqpoint{13.339844in}{0.566125in}}%
\pgfusepath{stroke}%
\end{pgfscope}%
\begin{pgfscope}%
\pgfsetrectcap%
\pgfsetmiterjoin%
\pgfsetlinewidth{0.803000pt}%
\definecolor{currentstroke}{rgb}{0.000000,0.000000,0.000000}%
\pgfsetstrokecolor{currentstroke}%
\pgfsetdash{}{0pt}%
\pgfpathmoveto{\pgfqpoint{1.499429in}{6.806869in}}%
\pgfpathlineto{\pgfqpoint{13.339844in}{6.806869in}}%
\pgfusepath{stroke}%
\end{pgfscope}%
\begin{pgfscope}%
\definecolor{textcolor}{rgb}{0.000000,0.000000,0.000000}%
\pgfsetstrokecolor{textcolor}%
\pgfsetfillcolor{textcolor}%
\pgftext[x=2.354218in,y=1.754143in,,bottom]{\color{textcolor}\rmfamily\fontsize{10.000000}{12.000000}\selectfont 706.89}%
\end{pgfscope}%
\begin{pgfscope}%
\definecolor{textcolor}{rgb}{0.000000,0.000000,0.000000}%
\pgfsetstrokecolor{textcolor}%
\pgfsetfillcolor{textcolor}%
\pgftext[x=3.620572in,y=1.754671in,,bottom]{\color{textcolor}\rmfamily\fontsize{10.000000}{12.000000}\selectfont 707.21}%
\end{pgfscope}%
\begin{pgfscope}%
\definecolor{textcolor}{rgb}{0.000000,0.000000,0.000000}%
\pgfsetstrokecolor{textcolor}%
\pgfsetfillcolor{textcolor}%
\pgftext[x=4.886927in,y=1.754095in,,bottom]{\color{textcolor}\rmfamily\fontsize{10.000000}{12.000000}\selectfont 706.87}%
\end{pgfscope}%
\begin{pgfscope}%
\definecolor{textcolor}{rgb}{0.000000,0.000000,0.000000}%
\pgfsetstrokecolor{textcolor}%
\pgfsetfillcolor{textcolor}%
\pgftext[x=6.153282in,y=6.505083in,,bottom]{\color{textcolor}\rmfamily\fontsize{10.000000}{12.000000}\selectfont 3534.0}%
\end{pgfscope}%
\begin{pgfscope}%
\definecolor{textcolor}{rgb}{0.000000,0.000000,0.000000}%
\pgfsetstrokecolor{textcolor}%
\pgfsetfillcolor{textcolor}%
\pgftext[x=7.419636in,y=6.509775in,,bottom]{\color{textcolor}\rmfamily\fontsize{10.000000}{12.000000}\selectfont 3536.79}%
\end{pgfscope}%
\begin{pgfscope}%
\definecolor{textcolor}{rgb}{0.000000,0.000000,0.000000}%
\pgfsetstrokecolor{textcolor}%
\pgfsetfillcolor{textcolor}%
\pgftext[x=8.685991in,y=6.509284in,,bottom]{\color{textcolor}\rmfamily\fontsize{10.000000}{12.000000}\selectfont 3536.5}%
\end{pgfscope}%
\begin{pgfscope}%
\definecolor{textcolor}{rgb}{0.000000,0.000000,0.000000}%
\pgfsetstrokecolor{textcolor}%
\pgfsetfillcolor{textcolor}%
\pgftext[x=9.952345in,y=6.505083in,,bottom]{\color{textcolor}\rmfamily\fontsize{10.000000}{12.000000}\selectfont 3534.0}%
\end{pgfscope}%
\begin{pgfscope}%
\definecolor{textcolor}{rgb}{0.000000,0.000000,0.000000}%
\pgfsetstrokecolor{textcolor}%
\pgfsetfillcolor{textcolor}%
\pgftext[x=11.218700in,y=6.509775in,,bottom]{\color{textcolor}\rmfamily\fontsize{10.000000}{12.000000}\selectfont 3536.79}%
\end{pgfscope}%
\begin{pgfscope}%
\definecolor{textcolor}{rgb}{0.000000,0.000000,0.000000}%
\pgfsetstrokecolor{textcolor}%
\pgfsetfillcolor{textcolor}%
\pgftext[x=12.485054in,y=6.509284in,,bottom]{\color{textcolor}\rmfamily\fontsize{10.000000}{12.000000}\selectfont 3536.5}%
\end{pgfscope}%
\begin{pgfscope}%
\definecolor{textcolor}{rgb}{0.000000,0.000000,0.000000}%
\pgfsetstrokecolor{textcolor}%
\pgfsetfillcolor{textcolor}%
\pgftext[x=7.419636in,y=6.890203in,,base]{\color{textcolor}\rmfamily\fontsize{12.000000}{14.400000}\selectfont Effect of changing lookahead}%
\end{pgfscope}%
\begin{pgfscope}%
\pgfsetbuttcap%
\pgfsetmiterjoin%
\definecolor{currentfill}{rgb}{1.000000,1.000000,1.000000}%
\pgfsetfillcolor{currentfill}%
\pgfsetfillopacity{0.800000}%
\pgfsetlinewidth{1.003750pt}%
\definecolor{currentstroke}{rgb}{0.800000,0.800000,0.800000}%
\pgfsetstrokecolor{currentstroke}%
\pgfsetstrokeopacity{0.800000}%
\pgfsetdash{}{0pt}%
\pgfpathmoveto{\pgfqpoint{1.596651in}{6.491901in}}%
\pgfpathlineto{\pgfqpoint{2.550739in}{6.491901in}}%
\pgfpathquadraticcurveto{\pgfqpoint{2.578517in}{6.491901in}}{\pgfqpoint{2.578517in}{6.519679in}}%
\pgfpathlineto{\pgfqpoint{2.578517in}{6.709647in}}%
\pgfpathquadraticcurveto{\pgfqpoint{2.578517in}{6.737425in}}{\pgfqpoint{2.550739in}{6.737425in}}%
\pgfpathlineto{\pgfqpoint{1.596651in}{6.737425in}}%
\pgfpathquadraticcurveto{\pgfqpoint{1.568873in}{6.737425in}}{\pgfqpoint{1.568873in}{6.709647in}}%
\pgfpathlineto{\pgfqpoint{1.568873in}{6.519679in}}%
\pgfpathquadraticcurveto{\pgfqpoint{1.568873in}{6.491901in}}{\pgfqpoint{1.596651in}{6.491901in}}%
\pgfpathclose%
\pgfusepath{stroke,fill}%
\end{pgfscope}%
\begin{pgfscope}%
\pgfsetbuttcap%
\pgfsetmiterjoin%
\definecolor{currentfill}{rgb}{1.000000,0.650980,0.000000}%
\pgfsetfillcolor{currentfill}%
\pgfsetlinewidth{0.000000pt}%
\definecolor{currentstroke}{rgb}{0.000000,0.000000,0.000000}%
\pgfsetstrokecolor{currentstroke}%
\pgfsetstrokeopacity{0.000000}%
\pgfsetdash{}{0pt}%
\pgfpathmoveto{\pgfqpoint{1.624429in}{6.576346in}}%
\pgfpathlineto{\pgfqpoint{1.902206in}{6.576346in}}%
\pgfpathlineto{\pgfqpoint{1.902206in}{6.673568in}}%
\pgfpathlineto{\pgfqpoint{1.624429in}{6.673568in}}%
\pgfpathclose%
\pgfusepath{fill}%
\end{pgfscope}%
\begin{pgfscope}%
\definecolor{textcolor}{rgb}{0.000000,0.000000,0.000000}%
\pgfsetstrokecolor{textcolor}%
\pgfsetfillcolor{textcolor}%
\pgftext[x=2.013317in,y=6.576346in,left,base]{\color{textcolor}\rmfamily\fontsize{10.000000}{12.000000}\selectfont custom}%
\end{pgfscope}%
\end{pgfpicture}%
\makeatother%
\endgroup%
}
\end{figure}

\begin{figure}[h]
    \caption{Average runtimes of all heuristics after running them on 500 different ER,MR,NMR graphs with the same parameters}
    \centering
    \scalebox{0.5}{%% Creator: Matplotlib, PGF backend
%%
%% To include the figure in your LaTeX document, write
%%   \input{<filename>.pgf}
%%
%% Make sure the required packages are loaded in your preamble
%%   \usepackage{pgf}
%%
%% and, on pdftex
%%   \usepackage[utf8]{inputenc}\DeclareUnicodeCharacter{2212}{-}
%%
%% or, on luatex and xetex
%%   \usepackage{unicode-math}
%%
%% Figures using additional raster images can only be included by \input if
%% they are in the same directory as the main LaTeX file. For loading figures
%% from other directories you can use the `import` package
%%   \usepackage{import}
%%
%% and then include the figures with
%%   \import{<path to file>}{<filename>.pgf}
%%
%% Matplotlib used the following preamble
%%   \usepackage{fontspec}
%%   \setmainfont{DejaVuSerif.ttf}[Path=/home/maks/.local/share/virtualenvs/CW3-zMJxnm_q/lib/python3.7/site-packages/matplotlib/mpl-data/fonts/ttf/]
%%   \setsansfont{DejaVuSans.ttf}[Path=/home/maks/.local/share/virtualenvs/CW3-zMJxnm_q/lib/python3.7/site-packages/matplotlib/mpl-data/fonts/ttf/]
%%   \setmonofont{DejaVuSansMono.ttf}[Path=/home/maks/.local/share/virtualenvs/CW3-zMJxnm_q/lib/python3.7/site-packages/matplotlib/mpl-data/fonts/ttf/]
%%
\begingroup%
\makeatletter%
\begin{pgfpicture}%
\pgfpathrectangle{\pgfpointorigin}{\pgfqpoint{13.660000in}{7.340000in}}%
\pgfusepath{use as bounding box, clip}%
\begin{pgfscope}%
\pgfsetbuttcap%
\pgfsetmiterjoin%
\definecolor{currentfill}{rgb}{1.000000,1.000000,1.000000}%
\pgfsetfillcolor{currentfill}%
\pgfsetlinewidth{0.000000pt}%
\definecolor{currentstroke}{rgb}{1.000000,1.000000,1.000000}%
\pgfsetstrokecolor{currentstroke}%
\pgfsetdash{}{0pt}%
\pgfpathmoveto{\pgfqpoint{0.000000in}{0.000000in}}%
\pgfpathlineto{\pgfqpoint{13.660000in}{0.000000in}}%
\pgfpathlineto{\pgfqpoint{13.660000in}{7.340000in}}%
\pgfpathlineto{\pgfqpoint{0.000000in}{7.340000in}}%
\pgfpathclose%
\pgfusepath{fill}%
\end{pgfscope}%
\begin{pgfscope}%
\pgfsetbuttcap%
\pgfsetmiterjoin%
\definecolor{currentfill}{rgb}{1.000000,1.000000,1.000000}%
\pgfsetfillcolor{currentfill}%
\pgfsetlinewidth{0.000000pt}%
\definecolor{currentstroke}{rgb}{0.000000,0.000000,0.000000}%
\pgfsetstrokecolor{currentstroke}%
\pgfsetstrokeopacity{0.000000}%
\pgfsetdash{}{0pt}%
\pgfpathmoveto{\pgfqpoint{1.286132in}{0.839159in}}%
\pgfpathlineto{\pgfqpoint{13.339844in}{0.839159in}}%
\pgfpathlineto{\pgfqpoint{13.339844in}{6.806869in}}%
\pgfpathlineto{\pgfqpoint{1.286132in}{6.806869in}}%
\pgfpathclose%
\pgfusepath{fill}%
\end{pgfscope}%
\begin{pgfscope}%
\pgfsetbuttcap%
\pgfsetroundjoin%
\definecolor{currentfill}{rgb}{0.000000,0.000000,0.000000}%
\pgfsetfillcolor{currentfill}%
\pgfsetlinewidth{0.803000pt}%
\definecolor{currentstroke}{rgb}{0.000000,0.000000,0.000000}%
\pgfsetstrokecolor{currentstroke}%
\pgfsetdash{}{0pt}%
\pgfsys@defobject{currentmarker}{\pgfqpoint{0.000000in}{-0.048611in}}{\pgfqpoint{0.000000in}{0.000000in}}{%
\pgfpathmoveto{\pgfqpoint{0.000000in}{0.000000in}}%
\pgfpathlineto{\pgfqpoint{0.000000in}{-0.048611in}}%
\pgfusepath{stroke,fill}%
}%
\begin{pgfscope}%
\pgfsys@transformshift{1.834028in}{0.839159in}%
\pgfsys@useobject{currentmarker}{}%
\end{pgfscope}%
\end{pgfscope}%
\begin{pgfscope}%
\definecolor{textcolor}{rgb}{0.000000,0.000000,0.000000}%
\pgfsetstrokecolor{textcolor}%
\pgfsetfillcolor{textcolor}%
\pgftext[x=1.834028in,y=0.741937in,,top]{\color{textcolor}\rmfamily\fontsize{10.000000}{12.000000}\selectfont \(\displaystyle 0\)}%
\end{pgfscope}%
\begin{pgfscope}%
\pgfsetbuttcap%
\pgfsetroundjoin%
\definecolor{currentfill}{rgb}{0.000000,0.000000,0.000000}%
\pgfsetfillcolor{currentfill}%
\pgfsetlinewidth{0.803000pt}%
\definecolor{currentstroke}{rgb}{0.000000,0.000000,0.000000}%
\pgfsetstrokecolor{currentstroke}%
\pgfsetdash{}{0pt}%
\pgfsys@defobject{currentmarker}{\pgfqpoint{0.000000in}{-0.048611in}}{\pgfqpoint{0.000000in}{0.000000in}}{%
\pgfpathmoveto{\pgfqpoint{0.000000in}{0.000000in}}%
\pgfpathlineto{\pgfqpoint{0.000000in}{-0.048611in}}%
\pgfusepath{stroke,fill}%
}%
\begin{pgfscope}%
\pgfsys@transformshift{4.047749in}{0.839159in}%
\pgfsys@useobject{currentmarker}{}%
\end{pgfscope}%
\end{pgfscope}%
\begin{pgfscope}%
\definecolor{textcolor}{rgb}{0.000000,0.000000,0.000000}%
\pgfsetstrokecolor{textcolor}%
\pgfsetfillcolor{textcolor}%
\pgftext[x=4.047749in,y=0.741937in,,top]{\color{textcolor}\rmfamily\fontsize{10.000000}{12.000000}\selectfont \(\displaystyle 20\)}%
\end{pgfscope}%
\begin{pgfscope}%
\pgfsetbuttcap%
\pgfsetroundjoin%
\definecolor{currentfill}{rgb}{0.000000,0.000000,0.000000}%
\pgfsetfillcolor{currentfill}%
\pgfsetlinewidth{0.803000pt}%
\definecolor{currentstroke}{rgb}{0.000000,0.000000,0.000000}%
\pgfsetstrokecolor{currentstroke}%
\pgfsetdash{}{0pt}%
\pgfsys@defobject{currentmarker}{\pgfqpoint{0.000000in}{-0.048611in}}{\pgfqpoint{0.000000in}{0.000000in}}{%
\pgfpathmoveto{\pgfqpoint{0.000000in}{0.000000in}}%
\pgfpathlineto{\pgfqpoint{0.000000in}{-0.048611in}}%
\pgfusepath{stroke,fill}%
}%
\begin{pgfscope}%
\pgfsys@transformshift{6.261470in}{0.839159in}%
\pgfsys@useobject{currentmarker}{}%
\end{pgfscope}%
\end{pgfscope}%
\begin{pgfscope}%
\definecolor{textcolor}{rgb}{0.000000,0.000000,0.000000}%
\pgfsetstrokecolor{textcolor}%
\pgfsetfillcolor{textcolor}%
\pgftext[x=6.261470in,y=0.741937in,,top]{\color{textcolor}\rmfamily\fontsize{10.000000}{12.000000}\selectfont \(\displaystyle 40\)}%
\end{pgfscope}%
\begin{pgfscope}%
\pgfsetbuttcap%
\pgfsetroundjoin%
\definecolor{currentfill}{rgb}{0.000000,0.000000,0.000000}%
\pgfsetfillcolor{currentfill}%
\pgfsetlinewidth{0.803000pt}%
\definecolor{currentstroke}{rgb}{0.000000,0.000000,0.000000}%
\pgfsetstrokecolor{currentstroke}%
\pgfsetdash{}{0pt}%
\pgfsys@defobject{currentmarker}{\pgfqpoint{0.000000in}{-0.048611in}}{\pgfqpoint{0.000000in}{0.000000in}}{%
\pgfpathmoveto{\pgfqpoint{0.000000in}{0.000000in}}%
\pgfpathlineto{\pgfqpoint{0.000000in}{-0.048611in}}%
\pgfusepath{stroke,fill}%
}%
\begin{pgfscope}%
\pgfsys@transformshift{8.475191in}{0.839159in}%
\pgfsys@useobject{currentmarker}{}%
\end{pgfscope}%
\end{pgfscope}%
\begin{pgfscope}%
\definecolor{textcolor}{rgb}{0.000000,0.000000,0.000000}%
\pgfsetstrokecolor{textcolor}%
\pgfsetfillcolor{textcolor}%
\pgftext[x=8.475191in,y=0.741937in,,top]{\color{textcolor}\rmfamily\fontsize{10.000000}{12.000000}\selectfont \(\displaystyle 60\)}%
\end{pgfscope}%
\begin{pgfscope}%
\pgfsetbuttcap%
\pgfsetroundjoin%
\definecolor{currentfill}{rgb}{0.000000,0.000000,0.000000}%
\pgfsetfillcolor{currentfill}%
\pgfsetlinewidth{0.803000pt}%
\definecolor{currentstroke}{rgb}{0.000000,0.000000,0.000000}%
\pgfsetstrokecolor{currentstroke}%
\pgfsetdash{}{0pt}%
\pgfsys@defobject{currentmarker}{\pgfqpoint{0.000000in}{-0.048611in}}{\pgfqpoint{0.000000in}{0.000000in}}{%
\pgfpathmoveto{\pgfqpoint{0.000000in}{0.000000in}}%
\pgfpathlineto{\pgfqpoint{0.000000in}{-0.048611in}}%
\pgfusepath{stroke,fill}%
}%
\begin{pgfscope}%
\pgfsys@transformshift{10.688913in}{0.839159in}%
\pgfsys@useobject{currentmarker}{}%
\end{pgfscope}%
\end{pgfscope}%
\begin{pgfscope}%
\definecolor{textcolor}{rgb}{0.000000,0.000000,0.000000}%
\pgfsetstrokecolor{textcolor}%
\pgfsetfillcolor{textcolor}%
\pgftext[x=10.688913in,y=0.741937in,,top]{\color{textcolor}\rmfamily\fontsize{10.000000}{12.000000}\selectfont \(\displaystyle 80\)}%
\end{pgfscope}%
\begin{pgfscope}%
\pgfsetbuttcap%
\pgfsetroundjoin%
\definecolor{currentfill}{rgb}{0.000000,0.000000,0.000000}%
\pgfsetfillcolor{currentfill}%
\pgfsetlinewidth{0.803000pt}%
\definecolor{currentstroke}{rgb}{0.000000,0.000000,0.000000}%
\pgfsetstrokecolor{currentstroke}%
\pgfsetdash{}{0pt}%
\pgfsys@defobject{currentmarker}{\pgfqpoint{0.000000in}{-0.048611in}}{\pgfqpoint{0.000000in}{0.000000in}}{%
\pgfpathmoveto{\pgfqpoint{0.000000in}{0.000000in}}%
\pgfpathlineto{\pgfqpoint{0.000000in}{-0.048611in}}%
\pgfusepath{stroke,fill}%
}%
\begin{pgfscope}%
\pgfsys@transformshift{12.902634in}{0.839159in}%
\pgfsys@useobject{currentmarker}{}%
\end{pgfscope}%
\end{pgfscope}%
\begin{pgfscope}%
\definecolor{textcolor}{rgb}{0.000000,0.000000,0.000000}%
\pgfsetstrokecolor{textcolor}%
\pgfsetfillcolor{textcolor}%
\pgftext[x=12.902634in,y=0.741937in,,top]{\color{textcolor}\rmfamily\fontsize{10.000000}{12.000000}\selectfont \(\displaystyle 100\)}%
\end{pgfscope}%
\begin{pgfscope}%
\definecolor{textcolor}{rgb}{0.000000,0.000000,0.000000}%
\pgfsetstrokecolor{textcolor}%
\pgfsetfillcolor{textcolor}%
\pgftext[x=7.312988in,y=0.551968in,,top]{\color{textcolor}\rmfamily\fontsize{10.000000}{12.000000}\selectfont Number of nodes}%
\end{pgfscope}%
\begin{pgfscope}%
\pgfsetbuttcap%
\pgfsetroundjoin%
\definecolor{currentfill}{rgb}{0.000000,0.000000,0.000000}%
\pgfsetfillcolor{currentfill}%
\pgfsetlinewidth{0.803000pt}%
\definecolor{currentstroke}{rgb}{0.000000,0.000000,0.000000}%
\pgfsetstrokecolor{currentstroke}%
\pgfsetdash{}{0pt}%
\pgfsys@defobject{currentmarker}{\pgfqpoint{-0.048611in}{0.000000in}}{\pgfqpoint{0.000000in}{0.000000in}}{%
\pgfpathmoveto{\pgfqpoint{0.000000in}{0.000000in}}%
\pgfpathlineto{\pgfqpoint{-0.048611in}{0.000000in}}%
\pgfusepath{stroke,fill}%
}%
\begin{pgfscope}%
\pgfsys@transformshift{1.286132in}{1.119175in}%
\pgfsys@useobject{currentmarker}{}%
\end{pgfscope}%
\end{pgfscope}%
\begin{pgfscope}%
\definecolor{textcolor}{rgb}{0.000000,0.000000,0.000000}%
\pgfsetstrokecolor{textcolor}%
\pgfsetfillcolor{textcolor}%
\pgftext[x=1.011440in, y=1.066414in, left, base]{\color{textcolor}\rmfamily\fontsize{10.000000}{12.000000}\selectfont \(\displaystyle 0.0\)}%
\end{pgfscope}%
\begin{pgfscope}%
\pgfsetbuttcap%
\pgfsetroundjoin%
\definecolor{currentfill}{rgb}{0.000000,0.000000,0.000000}%
\pgfsetfillcolor{currentfill}%
\pgfsetlinewidth{0.803000pt}%
\definecolor{currentstroke}{rgb}{0.000000,0.000000,0.000000}%
\pgfsetstrokecolor{currentstroke}%
\pgfsetdash{}{0pt}%
\pgfsys@defobject{currentmarker}{\pgfqpoint{-0.048611in}{0.000000in}}{\pgfqpoint{0.000000in}{0.000000in}}{%
\pgfpathmoveto{\pgfqpoint{0.000000in}{0.000000in}}%
\pgfpathlineto{\pgfqpoint{-0.048611in}{0.000000in}}%
\pgfusepath{stroke,fill}%
}%
\begin{pgfscope}%
\pgfsys@transformshift{1.286132in}{2.055419in}%
\pgfsys@useobject{currentmarker}{}%
\end{pgfscope}%
\end{pgfscope}%
\begin{pgfscope}%
\definecolor{textcolor}{rgb}{0.000000,0.000000,0.000000}%
\pgfsetstrokecolor{textcolor}%
\pgfsetfillcolor{textcolor}%
\pgftext[x=1.011440in, y=2.002657in, left, base]{\color{textcolor}\rmfamily\fontsize{10.000000}{12.000000}\selectfont \(\displaystyle 0.1\)}%
\end{pgfscope}%
\begin{pgfscope}%
\pgfsetbuttcap%
\pgfsetroundjoin%
\definecolor{currentfill}{rgb}{0.000000,0.000000,0.000000}%
\pgfsetfillcolor{currentfill}%
\pgfsetlinewidth{0.803000pt}%
\definecolor{currentstroke}{rgb}{0.000000,0.000000,0.000000}%
\pgfsetstrokecolor{currentstroke}%
\pgfsetdash{}{0pt}%
\pgfsys@defobject{currentmarker}{\pgfqpoint{-0.048611in}{0.000000in}}{\pgfqpoint{0.000000in}{0.000000in}}{%
\pgfpathmoveto{\pgfqpoint{0.000000in}{0.000000in}}%
\pgfpathlineto{\pgfqpoint{-0.048611in}{0.000000in}}%
\pgfusepath{stroke,fill}%
}%
\begin{pgfscope}%
\pgfsys@transformshift{1.286132in}{2.991662in}%
\pgfsys@useobject{currentmarker}{}%
\end{pgfscope}%
\end{pgfscope}%
\begin{pgfscope}%
\definecolor{textcolor}{rgb}{0.000000,0.000000,0.000000}%
\pgfsetstrokecolor{textcolor}%
\pgfsetfillcolor{textcolor}%
\pgftext[x=1.011440in, y=2.938901in, left, base]{\color{textcolor}\rmfamily\fontsize{10.000000}{12.000000}\selectfont \(\displaystyle 0.2\)}%
\end{pgfscope}%
\begin{pgfscope}%
\pgfsetbuttcap%
\pgfsetroundjoin%
\definecolor{currentfill}{rgb}{0.000000,0.000000,0.000000}%
\pgfsetfillcolor{currentfill}%
\pgfsetlinewidth{0.803000pt}%
\definecolor{currentstroke}{rgb}{0.000000,0.000000,0.000000}%
\pgfsetstrokecolor{currentstroke}%
\pgfsetdash{}{0pt}%
\pgfsys@defobject{currentmarker}{\pgfqpoint{-0.048611in}{0.000000in}}{\pgfqpoint{0.000000in}{0.000000in}}{%
\pgfpathmoveto{\pgfqpoint{0.000000in}{0.000000in}}%
\pgfpathlineto{\pgfqpoint{-0.048611in}{0.000000in}}%
\pgfusepath{stroke,fill}%
}%
\begin{pgfscope}%
\pgfsys@transformshift{1.286132in}{3.927906in}%
\pgfsys@useobject{currentmarker}{}%
\end{pgfscope}%
\end{pgfscope}%
\begin{pgfscope}%
\definecolor{textcolor}{rgb}{0.000000,0.000000,0.000000}%
\pgfsetstrokecolor{textcolor}%
\pgfsetfillcolor{textcolor}%
\pgftext[x=1.011440in, y=3.875144in, left, base]{\color{textcolor}\rmfamily\fontsize{10.000000}{12.000000}\selectfont \(\displaystyle 0.3\)}%
\end{pgfscope}%
\begin{pgfscope}%
\pgfsetbuttcap%
\pgfsetroundjoin%
\definecolor{currentfill}{rgb}{0.000000,0.000000,0.000000}%
\pgfsetfillcolor{currentfill}%
\pgfsetlinewidth{0.803000pt}%
\definecolor{currentstroke}{rgb}{0.000000,0.000000,0.000000}%
\pgfsetstrokecolor{currentstroke}%
\pgfsetdash{}{0pt}%
\pgfsys@defobject{currentmarker}{\pgfqpoint{-0.048611in}{0.000000in}}{\pgfqpoint{0.000000in}{0.000000in}}{%
\pgfpathmoveto{\pgfqpoint{0.000000in}{0.000000in}}%
\pgfpathlineto{\pgfqpoint{-0.048611in}{0.000000in}}%
\pgfusepath{stroke,fill}%
}%
\begin{pgfscope}%
\pgfsys@transformshift{1.286132in}{4.864149in}%
\pgfsys@useobject{currentmarker}{}%
\end{pgfscope}%
\end{pgfscope}%
\begin{pgfscope}%
\definecolor{textcolor}{rgb}{0.000000,0.000000,0.000000}%
\pgfsetstrokecolor{textcolor}%
\pgfsetfillcolor{textcolor}%
\pgftext[x=1.011440in, y=4.811388in, left, base]{\color{textcolor}\rmfamily\fontsize{10.000000}{12.000000}\selectfont \(\displaystyle 0.4\)}%
\end{pgfscope}%
\begin{pgfscope}%
\pgfsetbuttcap%
\pgfsetroundjoin%
\definecolor{currentfill}{rgb}{0.000000,0.000000,0.000000}%
\pgfsetfillcolor{currentfill}%
\pgfsetlinewidth{0.803000pt}%
\definecolor{currentstroke}{rgb}{0.000000,0.000000,0.000000}%
\pgfsetstrokecolor{currentstroke}%
\pgfsetdash{}{0pt}%
\pgfsys@defobject{currentmarker}{\pgfqpoint{-0.048611in}{0.000000in}}{\pgfqpoint{0.000000in}{0.000000in}}{%
\pgfpathmoveto{\pgfqpoint{0.000000in}{0.000000in}}%
\pgfpathlineto{\pgfqpoint{-0.048611in}{0.000000in}}%
\pgfusepath{stroke,fill}%
}%
\begin{pgfscope}%
\pgfsys@transformshift{1.286132in}{5.800393in}%
\pgfsys@useobject{currentmarker}{}%
\end{pgfscope}%
\end{pgfscope}%
\begin{pgfscope}%
\definecolor{textcolor}{rgb}{0.000000,0.000000,0.000000}%
\pgfsetstrokecolor{textcolor}%
\pgfsetfillcolor{textcolor}%
\pgftext[x=1.011440in, y=5.747631in, left, base]{\color{textcolor}\rmfamily\fontsize{10.000000}{12.000000}\selectfont \(\displaystyle 0.5\)}%
\end{pgfscope}%
\begin{pgfscope}%
\pgfsetbuttcap%
\pgfsetroundjoin%
\definecolor{currentfill}{rgb}{0.000000,0.000000,0.000000}%
\pgfsetfillcolor{currentfill}%
\pgfsetlinewidth{0.803000pt}%
\definecolor{currentstroke}{rgb}{0.000000,0.000000,0.000000}%
\pgfsetstrokecolor{currentstroke}%
\pgfsetdash{}{0pt}%
\pgfsys@defobject{currentmarker}{\pgfqpoint{-0.048611in}{0.000000in}}{\pgfqpoint{0.000000in}{0.000000in}}{%
\pgfpathmoveto{\pgfqpoint{0.000000in}{0.000000in}}%
\pgfpathlineto{\pgfqpoint{-0.048611in}{0.000000in}}%
\pgfusepath{stroke,fill}%
}%
\begin{pgfscope}%
\pgfsys@transformshift{1.286132in}{6.736636in}%
\pgfsys@useobject{currentmarker}{}%
\end{pgfscope}%
\end{pgfscope}%
\begin{pgfscope}%
\definecolor{textcolor}{rgb}{0.000000,0.000000,0.000000}%
\pgfsetstrokecolor{textcolor}%
\pgfsetfillcolor{textcolor}%
\pgftext[x=1.011440in, y=6.683875in, left, base]{\color{textcolor}\rmfamily\fontsize{10.000000}{12.000000}\selectfont \(\displaystyle 0.6\)}%
\end{pgfscope}%
\begin{pgfscope}%
\definecolor{textcolor}{rgb}{0.000000,0.000000,0.000000}%
\pgfsetstrokecolor{textcolor}%
\pgfsetfillcolor{textcolor}%
\pgftext[x=0.955884in,y=3.823014in,,bottom,rotate=90.000000]{\color{textcolor}\rmfamily\fontsize{10.000000}{12.000000}\selectfont Runtime in seconds}%
\end{pgfscope}%
\begin{pgfscope}%
\pgfpathrectangle{\pgfqpoint{1.286132in}{0.839159in}}{\pgfqpoint{12.053712in}{5.967710in}}%
\pgfusepath{clip}%
\pgfsetbuttcap%
\pgfsetroundjoin%
\pgfsetlinewidth{1.505625pt}%
\definecolor{currentstroke}{rgb}{0.121569,0.466667,0.705882}%
\pgfsetstrokecolor{currentstroke}%
\pgfsetdash{}{0pt}%
\pgfpathmoveto{\pgfqpoint{1.834028in}{1.119275in}}%
\pgfpathlineto{\pgfqpoint{1.834028in}{1.119294in}}%
\pgfusepath{stroke}%
\end{pgfscope}%
\begin{pgfscope}%
\pgfpathrectangle{\pgfqpoint{1.286132in}{0.839159in}}{\pgfqpoint{12.053712in}{5.967710in}}%
\pgfusepath{clip}%
\pgfsetbuttcap%
\pgfsetroundjoin%
\pgfsetlinewidth{1.505625pt}%
\definecolor{currentstroke}{rgb}{0.121569,0.466667,0.705882}%
\pgfsetstrokecolor{currentstroke}%
\pgfsetdash{}{0pt}%
\pgfpathmoveto{\pgfqpoint{1.944714in}{1.119256in}}%
\pgfpathlineto{\pgfqpoint{1.944714in}{1.119259in}}%
\pgfusepath{stroke}%
\end{pgfscope}%
\begin{pgfscope}%
\pgfpathrectangle{\pgfqpoint{1.286132in}{0.839159in}}{\pgfqpoint{12.053712in}{5.967710in}}%
\pgfusepath{clip}%
\pgfsetbuttcap%
\pgfsetroundjoin%
\pgfsetlinewidth{1.505625pt}%
\definecolor{currentstroke}{rgb}{0.121569,0.466667,0.705882}%
\pgfsetstrokecolor{currentstroke}%
\pgfsetdash{}{0pt}%
\pgfpathmoveto{\pgfqpoint{2.055400in}{1.119324in}}%
\pgfpathlineto{\pgfqpoint{2.055400in}{1.119329in}}%
\pgfusepath{stroke}%
\end{pgfscope}%
\begin{pgfscope}%
\pgfpathrectangle{\pgfqpoint{1.286132in}{0.839159in}}{\pgfqpoint{12.053712in}{5.967710in}}%
\pgfusepath{clip}%
\pgfsetbuttcap%
\pgfsetroundjoin%
\pgfsetlinewidth{1.505625pt}%
\definecolor{currentstroke}{rgb}{0.121569,0.466667,0.705882}%
\pgfsetstrokecolor{currentstroke}%
\pgfsetdash{}{0pt}%
\pgfpathmoveto{\pgfqpoint{2.166086in}{1.119391in}}%
\pgfpathlineto{\pgfqpoint{2.166086in}{1.119480in}}%
\pgfusepath{stroke}%
\end{pgfscope}%
\begin{pgfscope}%
\pgfpathrectangle{\pgfqpoint{1.286132in}{0.839159in}}{\pgfqpoint{12.053712in}{5.967710in}}%
\pgfusepath{clip}%
\pgfsetbuttcap%
\pgfsetroundjoin%
\pgfsetlinewidth{1.505625pt}%
\definecolor{currentstroke}{rgb}{0.121569,0.466667,0.705882}%
\pgfsetstrokecolor{currentstroke}%
\pgfsetdash{}{0pt}%
\pgfpathmoveto{\pgfqpoint{2.276772in}{1.119422in}}%
\pgfpathlineto{\pgfqpoint{2.276772in}{1.119591in}}%
\pgfusepath{stroke}%
\end{pgfscope}%
\begin{pgfscope}%
\pgfpathrectangle{\pgfqpoint{1.286132in}{0.839159in}}{\pgfqpoint{12.053712in}{5.967710in}}%
\pgfusepath{clip}%
\pgfsetbuttcap%
\pgfsetroundjoin%
\pgfsetlinewidth{1.505625pt}%
\definecolor{currentstroke}{rgb}{0.121569,0.466667,0.705882}%
\pgfsetstrokecolor{currentstroke}%
\pgfsetdash{}{0pt}%
\pgfpathmoveto{\pgfqpoint{2.387458in}{1.119518in}}%
\pgfpathlineto{\pgfqpoint{2.387458in}{1.119765in}}%
\pgfusepath{stroke}%
\end{pgfscope}%
\begin{pgfscope}%
\pgfpathrectangle{\pgfqpoint{1.286132in}{0.839159in}}{\pgfqpoint{12.053712in}{5.967710in}}%
\pgfusepath{clip}%
\pgfsetbuttcap%
\pgfsetroundjoin%
\pgfsetlinewidth{1.505625pt}%
\definecolor{currentstroke}{rgb}{0.121569,0.466667,0.705882}%
\pgfsetstrokecolor{currentstroke}%
\pgfsetdash{}{0pt}%
\pgfpathmoveto{\pgfqpoint{2.498144in}{1.119609in}}%
\pgfpathlineto{\pgfqpoint{2.498144in}{1.119776in}}%
\pgfusepath{stroke}%
\end{pgfscope}%
\begin{pgfscope}%
\pgfpathrectangle{\pgfqpoint{1.286132in}{0.839159in}}{\pgfqpoint{12.053712in}{5.967710in}}%
\pgfusepath{clip}%
\pgfsetbuttcap%
\pgfsetroundjoin%
\pgfsetlinewidth{1.505625pt}%
\definecolor{currentstroke}{rgb}{0.121569,0.466667,0.705882}%
\pgfsetstrokecolor{currentstroke}%
\pgfsetdash{}{0pt}%
\pgfpathmoveto{\pgfqpoint{2.608830in}{1.119702in}}%
\pgfpathlineto{\pgfqpoint{2.608830in}{1.120332in}}%
\pgfusepath{stroke}%
\end{pgfscope}%
\begin{pgfscope}%
\pgfpathrectangle{\pgfqpoint{1.286132in}{0.839159in}}{\pgfqpoint{12.053712in}{5.967710in}}%
\pgfusepath{clip}%
\pgfsetbuttcap%
\pgfsetroundjoin%
\pgfsetlinewidth{1.505625pt}%
\definecolor{currentstroke}{rgb}{0.121569,0.466667,0.705882}%
\pgfsetstrokecolor{currentstroke}%
\pgfsetdash{}{0pt}%
\pgfpathmoveto{\pgfqpoint{2.719516in}{1.119631in}}%
\pgfpathlineto{\pgfqpoint{2.719516in}{1.120220in}}%
\pgfusepath{stroke}%
\end{pgfscope}%
\begin{pgfscope}%
\pgfpathrectangle{\pgfqpoint{1.286132in}{0.839159in}}{\pgfqpoint{12.053712in}{5.967710in}}%
\pgfusepath{clip}%
\pgfsetbuttcap%
\pgfsetroundjoin%
\pgfsetlinewidth{1.505625pt}%
\definecolor{currentstroke}{rgb}{0.121569,0.466667,0.705882}%
\pgfsetstrokecolor{currentstroke}%
\pgfsetdash{}{0pt}%
\pgfpathmoveto{\pgfqpoint{2.830202in}{1.119750in}}%
\pgfpathlineto{\pgfqpoint{2.830202in}{1.120054in}}%
\pgfusepath{stroke}%
\end{pgfscope}%
\begin{pgfscope}%
\pgfpathrectangle{\pgfqpoint{1.286132in}{0.839159in}}{\pgfqpoint{12.053712in}{5.967710in}}%
\pgfusepath{clip}%
\pgfsetbuttcap%
\pgfsetroundjoin%
\pgfsetlinewidth{1.505625pt}%
\definecolor{currentstroke}{rgb}{0.121569,0.466667,0.705882}%
\pgfsetstrokecolor{currentstroke}%
\pgfsetdash{}{0pt}%
\pgfpathmoveto{\pgfqpoint{2.940888in}{1.119803in}}%
\pgfpathlineto{\pgfqpoint{2.940888in}{1.120034in}}%
\pgfusepath{stroke}%
\end{pgfscope}%
\begin{pgfscope}%
\pgfpathrectangle{\pgfqpoint{1.286132in}{0.839159in}}{\pgfqpoint{12.053712in}{5.967710in}}%
\pgfusepath{clip}%
\pgfsetbuttcap%
\pgfsetroundjoin%
\pgfsetlinewidth{1.505625pt}%
\definecolor{currentstroke}{rgb}{0.121569,0.466667,0.705882}%
\pgfsetstrokecolor{currentstroke}%
\pgfsetdash{}{0pt}%
\pgfpathmoveto{\pgfqpoint{3.051574in}{1.119984in}}%
\pgfpathlineto{\pgfqpoint{3.051574in}{1.120609in}}%
\pgfusepath{stroke}%
\end{pgfscope}%
\begin{pgfscope}%
\pgfpathrectangle{\pgfqpoint{1.286132in}{0.839159in}}{\pgfqpoint{12.053712in}{5.967710in}}%
\pgfusepath{clip}%
\pgfsetbuttcap%
\pgfsetroundjoin%
\pgfsetlinewidth{1.505625pt}%
\definecolor{currentstroke}{rgb}{0.121569,0.466667,0.705882}%
\pgfsetstrokecolor{currentstroke}%
\pgfsetdash{}{0pt}%
\pgfpathmoveto{\pgfqpoint{3.162260in}{1.120009in}}%
\pgfpathlineto{\pgfqpoint{3.162260in}{1.120587in}}%
\pgfusepath{stroke}%
\end{pgfscope}%
\begin{pgfscope}%
\pgfpathrectangle{\pgfqpoint{1.286132in}{0.839159in}}{\pgfqpoint{12.053712in}{5.967710in}}%
\pgfusepath{clip}%
\pgfsetbuttcap%
\pgfsetroundjoin%
\pgfsetlinewidth{1.505625pt}%
\definecolor{currentstroke}{rgb}{0.121569,0.466667,0.705882}%
\pgfsetstrokecolor{currentstroke}%
\pgfsetdash{}{0pt}%
\pgfpathmoveto{\pgfqpoint{3.272946in}{1.120210in}}%
\pgfpathlineto{\pgfqpoint{3.272946in}{1.120245in}}%
\pgfusepath{stroke}%
\end{pgfscope}%
\begin{pgfscope}%
\pgfpathrectangle{\pgfqpoint{1.286132in}{0.839159in}}{\pgfqpoint{12.053712in}{5.967710in}}%
\pgfusepath{clip}%
\pgfsetbuttcap%
\pgfsetroundjoin%
\pgfsetlinewidth{1.505625pt}%
\definecolor{currentstroke}{rgb}{0.121569,0.466667,0.705882}%
\pgfsetstrokecolor{currentstroke}%
\pgfsetdash{}{0pt}%
\pgfpathmoveto{\pgfqpoint{3.383632in}{1.120440in}}%
\pgfpathlineto{\pgfqpoint{3.383632in}{1.122444in}}%
\pgfusepath{stroke}%
\end{pgfscope}%
\begin{pgfscope}%
\pgfpathrectangle{\pgfqpoint{1.286132in}{0.839159in}}{\pgfqpoint{12.053712in}{5.967710in}}%
\pgfusepath{clip}%
\pgfsetbuttcap%
\pgfsetroundjoin%
\pgfsetlinewidth{1.505625pt}%
\definecolor{currentstroke}{rgb}{0.121569,0.466667,0.705882}%
\pgfsetstrokecolor{currentstroke}%
\pgfsetdash{}{0pt}%
\pgfpathmoveto{\pgfqpoint{3.494319in}{1.120114in}}%
\pgfpathlineto{\pgfqpoint{3.494319in}{1.120579in}}%
\pgfusepath{stroke}%
\end{pgfscope}%
\begin{pgfscope}%
\pgfpathrectangle{\pgfqpoint{1.286132in}{0.839159in}}{\pgfqpoint{12.053712in}{5.967710in}}%
\pgfusepath{clip}%
\pgfsetbuttcap%
\pgfsetroundjoin%
\pgfsetlinewidth{1.505625pt}%
\definecolor{currentstroke}{rgb}{0.121569,0.466667,0.705882}%
\pgfsetstrokecolor{currentstroke}%
\pgfsetdash{}{0pt}%
\pgfpathmoveto{\pgfqpoint{3.605005in}{1.118762in}}%
\pgfpathlineto{\pgfqpoint{3.605005in}{1.125393in}}%
\pgfusepath{stroke}%
\end{pgfscope}%
\begin{pgfscope}%
\pgfpathrectangle{\pgfqpoint{1.286132in}{0.839159in}}{\pgfqpoint{12.053712in}{5.967710in}}%
\pgfusepath{clip}%
\pgfsetbuttcap%
\pgfsetroundjoin%
\pgfsetlinewidth{1.505625pt}%
\definecolor{currentstroke}{rgb}{0.121569,0.466667,0.705882}%
\pgfsetstrokecolor{currentstroke}%
\pgfsetdash{}{0pt}%
\pgfpathmoveto{\pgfqpoint{3.715691in}{1.120463in}}%
\pgfpathlineto{\pgfqpoint{3.715691in}{1.120837in}}%
\pgfusepath{stroke}%
\end{pgfscope}%
\begin{pgfscope}%
\pgfpathrectangle{\pgfqpoint{1.286132in}{0.839159in}}{\pgfqpoint{12.053712in}{5.967710in}}%
\pgfusepath{clip}%
\pgfsetbuttcap%
\pgfsetroundjoin%
\pgfsetlinewidth{1.505625pt}%
\definecolor{currentstroke}{rgb}{0.121569,0.466667,0.705882}%
\pgfsetstrokecolor{currentstroke}%
\pgfsetdash{}{0pt}%
\pgfpathmoveto{\pgfqpoint{3.826377in}{1.120528in}}%
\pgfpathlineto{\pgfqpoint{3.826377in}{1.120901in}}%
\pgfusepath{stroke}%
\end{pgfscope}%
\begin{pgfscope}%
\pgfpathrectangle{\pgfqpoint{1.286132in}{0.839159in}}{\pgfqpoint{12.053712in}{5.967710in}}%
\pgfusepath{clip}%
\pgfsetbuttcap%
\pgfsetroundjoin%
\pgfsetlinewidth{1.505625pt}%
\definecolor{currentstroke}{rgb}{0.121569,0.466667,0.705882}%
\pgfsetstrokecolor{currentstroke}%
\pgfsetdash{}{0pt}%
\pgfpathmoveto{\pgfqpoint{3.937063in}{1.120348in}}%
\pgfpathlineto{\pgfqpoint{3.937063in}{1.120873in}}%
\pgfusepath{stroke}%
\end{pgfscope}%
\begin{pgfscope}%
\pgfpathrectangle{\pgfqpoint{1.286132in}{0.839159in}}{\pgfqpoint{12.053712in}{5.967710in}}%
\pgfusepath{clip}%
\pgfsetbuttcap%
\pgfsetroundjoin%
\pgfsetlinewidth{1.505625pt}%
\definecolor{currentstroke}{rgb}{0.121569,0.466667,0.705882}%
\pgfsetstrokecolor{currentstroke}%
\pgfsetdash{}{0pt}%
\pgfpathmoveto{\pgfqpoint{4.047749in}{1.120530in}}%
\pgfpathlineto{\pgfqpoint{4.047749in}{1.121256in}}%
\pgfusepath{stroke}%
\end{pgfscope}%
\begin{pgfscope}%
\pgfpathrectangle{\pgfqpoint{1.286132in}{0.839159in}}{\pgfqpoint{12.053712in}{5.967710in}}%
\pgfusepath{clip}%
\pgfsetbuttcap%
\pgfsetroundjoin%
\pgfsetlinewidth{1.505625pt}%
\definecolor{currentstroke}{rgb}{0.121569,0.466667,0.705882}%
\pgfsetstrokecolor{currentstroke}%
\pgfsetdash{}{0pt}%
\pgfpathmoveto{\pgfqpoint{4.158435in}{1.120449in}}%
\pgfpathlineto{\pgfqpoint{4.158435in}{1.121220in}}%
\pgfusepath{stroke}%
\end{pgfscope}%
\begin{pgfscope}%
\pgfpathrectangle{\pgfqpoint{1.286132in}{0.839159in}}{\pgfqpoint{12.053712in}{5.967710in}}%
\pgfusepath{clip}%
\pgfsetbuttcap%
\pgfsetroundjoin%
\pgfsetlinewidth{1.505625pt}%
\definecolor{currentstroke}{rgb}{0.121569,0.466667,0.705882}%
\pgfsetstrokecolor{currentstroke}%
\pgfsetdash{}{0pt}%
\pgfpathmoveto{\pgfqpoint{4.269121in}{1.121026in}}%
\pgfpathlineto{\pgfqpoint{4.269121in}{1.122129in}}%
\pgfusepath{stroke}%
\end{pgfscope}%
\begin{pgfscope}%
\pgfpathrectangle{\pgfqpoint{1.286132in}{0.839159in}}{\pgfqpoint{12.053712in}{5.967710in}}%
\pgfusepath{clip}%
\pgfsetbuttcap%
\pgfsetroundjoin%
\pgfsetlinewidth{1.505625pt}%
\definecolor{currentstroke}{rgb}{0.121569,0.466667,0.705882}%
\pgfsetstrokecolor{currentstroke}%
\pgfsetdash{}{0pt}%
\pgfpathmoveto{\pgfqpoint{4.379807in}{1.120634in}}%
\pgfpathlineto{\pgfqpoint{4.379807in}{1.121405in}}%
\pgfusepath{stroke}%
\end{pgfscope}%
\begin{pgfscope}%
\pgfpathrectangle{\pgfqpoint{1.286132in}{0.839159in}}{\pgfqpoint{12.053712in}{5.967710in}}%
\pgfusepath{clip}%
\pgfsetbuttcap%
\pgfsetroundjoin%
\pgfsetlinewidth{1.505625pt}%
\definecolor{currentstroke}{rgb}{0.121569,0.466667,0.705882}%
\pgfsetstrokecolor{currentstroke}%
\pgfsetdash{}{0pt}%
\pgfpathmoveto{\pgfqpoint{4.490493in}{1.120863in}}%
\pgfpathlineto{\pgfqpoint{4.490493in}{1.121837in}}%
\pgfusepath{stroke}%
\end{pgfscope}%
\begin{pgfscope}%
\pgfpathrectangle{\pgfqpoint{1.286132in}{0.839159in}}{\pgfqpoint{12.053712in}{5.967710in}}%
\pgfusepath{clip}%
\pgfsetbuttcap%
\pgfsetroundjoin%
\pgfsetlinewidth{1.505625pt}%
\definecolor{currentstroke}{rgb}{0.121569,0.466667,0.705882}%
\pgfsetstrokecolor{currentstroke}%
\pgfsetdash{}{0pt}%
\pgfpathmoveto{\pgfqpoint{4.601179in}{1.121676in}}%
\pgfpathlineto{\pgfqpoint{4.601179in}{1.122443in}}%
\pgfusepath{stroke}%
\end{pgfscope}%
\begin{pgfscope}%
\pgfpathrectangle{\pgfqpoint{1.286132in}{0.839159in}}{\pgfqpoint{12.053712in}{5.967710in}}%
\pgfusepath{clip}%
\pgfsetbuttcap%
\pgfsetroundjoin%
\pgfsetlinewidth{1.505625pt}%
\definecolor{currentstroke}{rgb}{0.121569,0.466667,0.705882}%
\pgfsetstrokecolor{currentstroke}%
\pgfsetdash{}{0pt}%
\pgfpathmoveto{\pgfqpoint{4.711865in}{1.121574in}}%
\pgfpathlineto{\pgfqpoint{4.711865in}{1.122522in}}%
\pgfusepath{stroke}%
\end{pgfscope}%
\begin{pgfscope}%
\pgfpathrectangle{\pgfqpoint{1.286132in}{0.839159in}}{\pgfqpoint{12.053712in}{5.967710in}}%
\pgfusepath{clip}%
\pgfsetbuttcap%
\pgfsetroundjoin%
\pgfsetlinewidth{1.505625pt}%
\definecolor{currentstroke}{rgb}{0.121569,0.466667,0.705882}%
\pgfsetstrokecolor{currentstroke}%
\pgfsetdash{}{0pt}%
\pgfpathmoveto{\pgfqpoint{4.822551in}{1.121743in}}%
\pgfpathlineto{\pgfqpoint{4.822551in}{1.122285in}}%
\pgfusepath{stroke}%
\end{pgfscope}%
\begin{pgfscope}%
\pgfpathrectangle{\pgfqpoint{1.286132in}{0.839159in}}{\pgfqpoint{12.053712in}{5.967710in}}%
\pgfusepath{clip}%
\pgfsetbuttcap%
\pgfsetroundjoin%
\pgfsetlinewidth{1.505625pt}%
\definecolor{currentstroke}{rgb}{0.121569,0.466667,0.705882}%
\pgfsetstrokecolor{currentstroke}%
\pgfsetdash{}{0pt}%
\pgfpathmoveto{\pgfqpoint{4.933237in}{1.121477in}}%
\pgfpathlineto{\pgfqpoint{4.933237in}{1.122256in}}%
\pgfusepath{stroke}%
\end{pgfscope}%
\begin{pgfscope}%
\pgfpathrectangle{\pgfqpoint{1.286132in}{0.839159in}}{\pgfqpoint{12.053712in}{5.967710in}}%
\pgfusepath{clip}%
\pgfsetbuttcap%
\pgfsetroundjoin%
\pgfsetlinewidth{1.505625pt}%
\definecolor{currentstroke}{rgb}{0.121569,0.466667,0.705882}%
\pgfsetstrokecolor{currentstroke}%
\pgfsetdash{}{0pt}%
\pgfpathmoveto{\pgfqpoint{5.043923in}{1.121569in}}%
\pgfpathlineto{\pgfqpoint{5.043923in}{1.122598in}}%
\pgfusepath{stroke}%
\end{pgfscope}%
\begin{pgfscope}%
\pgfpathrectangle{\pgfqpoint{1.286132in}{0.839159in}}{\pgfqpoint{12.053712in}{5.967710in}}%
\pgfusepath{clip}%
\pgfsetbuttcap%
\pgfsetroundjoin%
\pgfsetlinewidth{1.505625pt}%
\definecolor{currentstroke}{rgb}{0.121569,0.466667,0.705882}%
\pgfsetstrokecolor{currentstroke}%
\pgfsetdash{}{0pt}%
\pgfpathmoveto{\pgfqpoint{5.154609in}{1.121622in}}%
\pgfpathlineto{\pgfqpoint{5.154609in}{1.122439in}}%
\pgfusepath{stroke}%
\end{pgfscope}%
\begin{pgfscope}%
\pgfpathrectangle{\pgfqpoint{1.286132in}{0.839159in}}{\pgfqpoint{12.053712in}{5.967710in}}%
\pgfusepath{clip}%
\pgfsetbuttcap%
\pgfsetroundjoin%
\pgfsetlinewidth{1.505625pt}%
\definecolor{currentstroke}{rgb}{0.121569,0.466667,0.705882}%
\pgfsetstrokecolor{currentstroke}%
\pgfsetdash{}{0pt}%
\pgfpathmoveto{\pgfqpoint{5.265296in}{1.121906in}}%
\pgfpathlineto{\pgfqpoint{5.265296in}{1.124206in}}%
\pgfusepath{stroke}%
\end{pgfscope}%
\begin{pgfscope}%
\pgfpathrectangle{\pgfqpoint{1.286132in}{0.839159in}}{\pgfqpoint{12.053712in}{5.967710in}}%
\pgfusepath{clip}%
\pgfsetbuttcap%
\pgfsetroundjoin%
\pgfsetlinewidth{1.505625pt}%
\definecolor{currentstroke}{rgb}{0.121569,0.466667,0.705882}%
\pgfsetstrokecolor{currentstroke}%
\pgfsetdash{}{0pt}%
\pgfpathmoveto{\pgfqpoint{5.375982in}{1.122513in}}%
\pgfpathlineto{\pgfqpoint{5.375982in}{1.123196in}}%
\pgfusepath{stroke}%
\end{pgfscope}%
\begin{pgfscope}%
\pgfpathrectangle{\pgfqpoint{1.286132in}{0.839159in}}{\pgfqpoint{12.053712in}{5.967710in}}%
\pgfusepath{clip}%
\pgfsetbuttcap%
\pgfsetroundjoin%
\pgfsetlinewidth{1.505625pt}%
\definecolor{currentstroke}{rgb}{0.121569,0.466667,0.705882}%
\pgfsetstrokecolor{currentstroke}%
\pgfsetdash{}{0pt}%
\pgfpathmoveto{\pgfqpoint{5.486668in}{1.121995in}}%
\pgfpathlineto{\pgfqpoint{5.486668in}{1.123662in}}%
\pgfusepath{stroke}%
\end{pgfscope}%
\begin{pgfscope}%
\pgfpathrectangle{\pgfqpoint{1.286132in}{0.839159in}}{\pgfqpoint{12.053712in}{5.967710in}}%
\pgfusepath{clip}%
\pgfsetbuttcap%
\pgfsetroundjoin%
\pgfsetlinewidth{1.505625pt}%
\definecolor{currentstroke}{rgb}{0.121569,0.466667,0.705882}%
\pgfsetstrokecolor{currentstroke}%
\pgfsetdash{}{0pt}%
\pgfpathmoveto{\pgfqpoint{5.597354in}{1.122247in}}%
\pgfpathlineto{\pgfqpoint{5.597354in}{1.122613in}}%
\pgfusepath{stroke}%
\end{pgfscope}%
\begin{pgfscope}%
\pgfpathrectangle{\pgfqpoint{1.286132in}{0.839159in}}{\pgfqpoint{12.053712in}{5.967710in}}%
\pgfusepath{clip}%
\pgfsetbuttcap%
\pgfsetroundjoin%
\pgfsetlinewidth{1.505625pt}%
\definecolor{currentstroke}{rgb}{0.121569,0.466667,0.705882}%
\pgfsetstrokecolor{currentstroke}%
\pgfsetdash{}{0pt}%
\pgfpathmoveto{\pgfqpoint{5.708040in}{1.117312in}}%
\pgfpathlineto{\pgfqpoint{5.708040in}{1.140123in}}%
\pgfusepath{stroke}%
\end{pgfscope}%
\begin{pgfscope}%
\pgfpathrectangle{\pgfqpoint{1.286132in}{0.839159in}}{\pgfqpoint{12.053712in}{5.967710in}}%
\pgfusepath{clip}%
\pgfsetbuttcap%
\pgfsetroundjoin%
\pgfsetlinewidth{1.505625pt}%
\definecolor{currentstroke}{rgb}{0.121569,0.466667,0.705882}%
\pgfsetstrokecolor{currentstroke}%
\pgfsetdash{}{0pt}%
\pgfpathmoveto{\pgfqpoint{5.818726in}{1.121812in}}%
\pgfpathlineto{\pgfqpoint{5.818726in}{1.122591in}}%
\pgfusepath{stroke}%
\end{pgfscope}%
\begin{pgfscope}%
\pgfpathrectangle{\pgfqpoint{1.286132in}{0.839159in}}{\pgfqpoint{12.053712in}{5.967710in}}%
\pgfusepath{clip}%
\pgfsetbuttcap%
\pgfsetroundjoin%
\pgfsetlinewidth{1.505625pt}%
\definecolor{currentstroke}{rgb}{0.121569,0.466667,0.705882}%
\pgfsetstrokecolor{currentstroke}%
\pgfsetdash{}{0pt}%
\pgfpathmoveto{\pgfqpoint{5.929412in}{1.122700in}}%
\pgfpathlineto{\pgfqpoint{5.929412in}{1.124031in}}%
\pgfusepath{stroke}%
\end{pgfscope}%
\begin{pgfscope}%
\pgfpathrectangle{\pgfqpoint{1.286132in}{0.839159in}}{\pgfqpoint{12.053712in}{5.967710in}}%
\pgfusepath{clip}%
\pgfsetbuttcap%
\pgfsetroundjoin%
\pgfsetlinewidth{1.505625pt}%
\definecolor{currentstroke}{rgb}{0.121569,0.466667,0.705882}%
\pgfsetstrokecolor{currentstroke}%
\pgfsetdash{}{0pt}%
\pgfpathmoveto{\pgfqpoint{6.040098in}{1.122586in}}%
\pgfpathlineto{\pgfqpoint{6.040098in}{1.123794in}}%
\pgfusepath{stroke}%
\end{pgfscope}%
\begin{pgfscope}%
\pgfpathrectangle{\pgfqpoint{1.286132in}{0.839159in}}{\pgfqpoint{12.053712in}{5.967710in}}%
\pgfusepath{clip}%
\pgfsetbuttcap%
\pgfsetroundjoin%
\pgfsetlinewidth{1.505625pt}%
\definecolor{currentstroke}{rgb}{0.121569,0.466667,0.705882}%
\pgfsetstrokecolor{currentstroke}%
\pgfsetdash{}{0pt}%
\pgfpathmoveto{\pgfqpoint{6.150784in}{1.122443in}}%
\pgfpathlineto{\pgfqpoint{6.150784in}{1.125129in}}%
\pgfusepath{stroke}%
\end{pgfscope}%
\begin{pgfscope}%
\pgfpathrectangle{\pgfqpoint{1.286132in}{0.839159in}}{\pgfqpoint{12.053712in}{5.967710in}}%
\pgfusepath{clip}%
\pgfsetbuttcap%
\pgfsetroundjoin%
\pgfsetlinewidth{1.505625pt}%
\definecolor{currentstroke}{rgb}{0.121569,0.466667,0.705882}%
\pgfsetstrokecolor{currentstroke}%
\pgfsetdash{}{0pt}%
\pgfpathmoveto{\pgfqpoint{6.261470in}{1.122096in}}%
\pgfpathlineto{\pgfqpoint{6.261470in}{1.123369in}}%
\pgfusepath{stroke}%
\end{pgfscope}%
\begin{pgfscope}%
\pgfpathrectangle{\pgfqpoint{1.286132in}{0.839159in}}{\pgfqpoint{12.053712in}{5.967710in}}%
\pgfusepath{clip}%
\pgfsetbuttcap%
\pgfsetroundjoin%
\pgfsetlinewidth{1.505625pt}%
\definecolor{currentstroke}{rgb}{0.121569,0.466667,0.705882}%
\pgfsetstrokecolor{currentstroke}%
\pgfsetdash{}{0pt}%
\pgfpathmoveto{\pgfqpoint{6.372156in}{1.122583in}}%
\pgfpathlineto{\pgfqpoint{6.372156in}{1.125487in}}%
\pgfusepath{stroke}%
\end{pgfscope}%
\begin{pgfscope}%
\pgfpathrectangle{\pgfqpoint{1.286132in}{0.839159in}}{\pgfqpoint{12.053712in}{5.967710in}}%
\pgfusepath{clip}%
\pgfsetbuttcap%
\pgfsetroundjoin%
\pgfsetlinewidth{1.505625pt}%
\definecolor{currentstroke}{rgb}{0.121569,0.466667,0.705882}%
\pgfsetstrokecolor{currentstroke}%
\pgfsetdash{}{0pt}%
\pgfpathmoveto{\pgfqpoint{6.482842in}{1.122531in}}%
\pgfpathlineto{\pgfqpoint{6.482842in}{1.124314in}}%
\pgfusepath{stroke}%
\end{pgfscope}%
\begin{pgfscope}%
\pgfpathrectangle{\pgfqpoint{1.286132in}{0.839159in}}{\pgfqpoint{12.053712in}{5.967710in}}%
\pgfusepath{clip}%
\pgfsetbuttcap%
\pgfsetroundjoin%
\pgfsetlinewidth{1.505625pt}%
\definecolor{currentstroke}{rgb}{0.121569,0.466667,0.705882}%
\pgfsetstrokecolor{currentstroke}%
\pgfsetdash{}{0pt}%
\pgfpathmoveto{\pgfqpoint{6.593528in}{1.122244in}}%
\pgfpathlineto{\pgfqpoint{6.593528in}{1.125256in}}%
\pgfusepath{stroke}%
\end{pgfscope}%
\begin{pgfscope}%
\pgfpathrectangle{\pgfqpoint{1.286132in}{0.839159in}}{\pgfqpoint{12.053712in}{5.967710in}}%
\pgfusepath{clip}%
\pgfsetbuttcap%
\pgfsetroundjoin%
\pgfsetlinewidth{1.505625pt}%
\definecolor{currentstroke}{rgb}{0.121569,0.466667,0.705882}%
\pgfsetstrokecolor{currentstroke}%
\pgfsetdash{}{0pt}%
\pgfpathmoveto{\pgfqpoint{6.704214in}{1.123642in}}%
\pgfpathlineto{\pgfqpoint{6.704214in}{1.124327in}}%
\pgfusepath{stroke}%
\end{pgfscope}%
\begin{pgfscope}%
\pgfpathrectangle{\pgfqpoint{1.286132in}{0.839159in}}{\pgfqpoint{12.053712in}{5.967710in}}%
\pgfusepath{clip}%
\pgfsetbuttcap%
\pgfsetroundjoin%
\pgfsetlinewidth{1.505625pt}%
\definecolor{currentstroke}{rgb}{0.121569,0.466667,0.705882}%
\pgfsetstrokecolor{currentstroke}%
\pgfsetdash{}{0pt}%
\pgfpathmoveto{\pgfqpoint{6.814900in}{1.123231in}}%
\pgfpathlineto{\pgfqpoint{6.814900in}{1.124132in}}%
\pgfusepath{stroke}%
\end{pgfscope}%
\begin{pgfscope}%
\pgfpathrectangle{\pgfqpoint{1.286132in}{0.839159in}}{\pgfqpoint{12.053712in}{5.967710in}}%
\pgfusepath{clip}%
\pgfsetbuttcap%
\pgfsetroundjoin%
\pgfsetlinewidth{1.505625pt}%
\definecolor{currentstroke}{rgb}{0.121569,0.466667,0.705882}%
\pgfsetstrokecolor{currentstroke}%
\pgfsetdash{}{0pt}%
\pgfpathmoveto{\pgfqpoint{6.925586in}{1.122776in}}%
\pgfpathlineto{\pgfqpoint{6.925586in}{1.124065in}}%
\pgfusepath{stroke}%
\end{pgfscope}%
\begin{pgfscope}%
\pgfpathrectangle{\pgfqpoint{1.286132in}{0.839159in}}{\pgfqpoint{12.053712in}{5.967710in}}%
\pgfusepath{clip}%
\pgfsetbuttcap%
\pgfsetroundjoin%
\pgfsetlinewidth{1.505625pt}%
\definecolor{currentstroke}{rgb}{0.121569,0.466667,0.705882}%
\pgfsetstrokecolor{currentstroke}%
\pgfsetdash{}{0pt}%
\pgfpathmoveto{\pgfqpoint{7.036272in}{1.122853in}}%
\pgfpathlineto{\pgfqpoint{7.036272in}{1.124856in}}%
\pgfusepath{stroke}%
\end{pgfscope}%
\begin{pgfscope}%
\pgfpathrectangle{\pgfqpoint{1.286132in}{0.839159in}}{\pgfqpoint{12.053712in}{5.967710in}}%
\pgfusepath{clip}%
\pgfsetbuttcap%
\pgfsetroundjoin%
\pgfsetlinewidth{1.505625pt}%
\definecolor{currentstroke}{rgb}{0.121569,0.466667,0.705882}%
\pgfsetstrokecolor{currentstroke}%
\pgfsetdash{}{0pt}%
\pgfpathmoveto{\pgfqpoint{7.146959in}{1.123001in}}%
\pgfpathlineto{\pgfqpoint{7.146959in}{1.124802in}}%
\pgfusepath{stroke}%
\end{pgfscope}%
\begin{pgfscope}%
\pgfpathrectangle{\pgfqpoint{1.286132in}{0.839159in}}{\pgfqpoint{12.053712in}{5.967710in}}%
\pgfusepath{clip}%
\pgfsetbuttcap%
\pgfsetroundjoin%
\pgfsetlinewidth{1.505625pt}%
\definecolor{currentstroke}{rgb}{0.121569,0.466667,0.705882}%
\pgfsetstrokecolor{currentstroke}%
\pgfsetdash{}{0pt}%
\pgfpathmoveto{\pgfqpoint{7.257645in}{1.123144in}}%
\pgfpathlineto{\pgfqpoint{7.257645in}{1.126811in}}%
\pgfusepath{stroke}%
\end{pgfscope}%
\begin{pgfscope}%
\pgfpathrectangle{\pgfqpoint{1.286132in}{0.839159in}}{\pgfqpoint{12.053712in}{5.967710in}}%
\pgfusepath{clip}%
\pgfsetbuttcap%
\pgfsetroundjoin%
\pgfsetlinewidth{1.505625pt}%
\definecolor{currentstroke}{rgb}{0.121569,0.466667,0.705882}%
\pgfsetstrokecolor{currentstroke}%
\pgfsetdash{}{0pt}%
\pgfpathmoveto{\pgfqpoint{7.368331in}{1.124006in}}%
\pgfpathlineto{\pgfqpoint{7.368331in}{1.128656in}}%
\pgfusepath{stroke}%
\end{pgfscope}%
\begin{pgfscope}%
\pgfpathrectangle{\pgfqpoint{1.286132in}{0.839159in}}{\pgfqpoint{12.053712in}{5.967710in}}%
\pgfusepath{clip}%
\pgfsetbuttcap%
\pgfsetroundjoin%
\pgfsetlinewidth{1.505625pt}%
\definecolor{currentstroke}{rgb}{0.121569,0.466667,0.705882}%
\pgfsetstrokecolor{currentstroke}%
\pgfsetdash{}{0pt}%
\pgfpathmoveto{\pgfqpoint{7.479017in}{1.123753in}}%
\pgfpathlineto{\pgfqpoint{7.479017in}{1.126985in}}%
\pgfusepath{stroke}%
\end{pgfscope}%
\begin{pgfscope}%
\pgfpathrectangle{\pgfqpoint{1.286132in}{0.839159in}}{\pgfqpoint{12.053712in}{5.967710in}}%
\pgfusepath{clip}%
\pgfsetbuttcap%
\pgfsetroundjoin%
\pgfsetlinewidth{1.505625pt}%
\definecolor{currentstroke}{rgb}{0.121569,0.466667,0.705882}%
\pgfsetstrokecolor{currentstroke}%
\pgfsetdash{}{0pt}%
\pgfpathmoveto{\pgfqpoint{7.589703in}{1.123794in}}%
\pgfpathlineto{\pgfqpoint{7.589703in}{1.127720in}}%
\pgfusepath{stroke}%
\end{pgfscope}%
\begin{pgfscope}%
\pgfpathrectangle{\pgfqpoint{1.286132in}{0.839159in}}{\pgfqpoint{12.053712in}{5.967710in}}%
\pgfusepath{clip}%
\pgfsetbuttcap%
\pgfsetroundjoin%
\pgfsetlinewidth{1.505625pt}%
\definecolor{currentstroke}{rgb}{0.121569,0.466667,0.705882}%
\pgfsetstrokecolor{currentstroke}%
\pgfsetdash{}{0pt}%
\pgfpathmoveto{\pgfqpoint{7.700389in}{1.124160in}}%
\pgfpathlineto{\pgfqpoint{7.700389in}{1.125877in}}%
\pgfusepath{stroke}%
\end{pgfscope}%
\begin{pgfscope}%
\pgfpathrectangle{\pgfqpoint{1.286132in}{0.839159in}}{\pgfqpoint{12.053712in}{5.967710in}}%
\pgfusepath{clip}%
\pgfsetbuttcap%
\pgfsetroundjoin%
\pgfsetlinewidth{1.505625pt}%
\definecolor{currentstroke}{rgb}{0.121569,0.466667,0.705882}%
\pgfsetstrokecolor{currentstroke}%
\pgfsetdash{}{0pt}%
\pgfpathmoveto{\pgfqpoint{7.811075in}{1.124029in}}%
\pgfpathlineto{\pgfqpoint{7.811075in}{1.126573in}}%
\pgfusepath{stroke}%
\end{pgfscope}%
\begin{pgfscope}%
\pgfpathrectangle{\pgfqpoint{1.286132in}{0.839159in}}{\pgfqpoint{12.053712in}{5.967710in}}%
\pgfusepath{clip}%
\pgfsetbuttcap%
\pgfsetroundjoin%
\pgfsetlinewidth{1.505625pt}%
\definecolor{currentstroke}{rgb}{0.121569,0.466667,0.705882}%
\pgfsetstrokecolor{currentstroke}%
\pgfsetdash{}{0pt}%
\pgfpathmoveto{\pgfqpoint{7.921761in}{1.123770in}}%
\pgfpathlineto{\pgfqpoint{7.921761in}{1.126071in}}%
\pgfusepath{stroke}%
\end{pgfscope}%
\begin{pgfscope}%
\pgfpathrectangle{\pgfqpoint{1.286132in}{0.839159in}}{\pgfqpoint{12.053712in}{5.967710in}}%
\pgfusepath{clip}%
\pgfsetbuttcap%
\pgfsetroundjoin%
\pgfsetlinewidth{1.505625pt}%
\definecolor{currentstroke}{rgb}{0.121569,0.466667,0.705882}%
\pgfsetstrokecolor{currentstroke}%
\pgfsetdash{}{0pt}%
\pgfpathmoveto{\pgfqpoint{8.032447in}{1.124140in}}%
\pgfpathlineto{\pgfqpoint{8.032447in}{1.125023in}}%
\pgfusepath{stroke}%
\end{pgfscope}%
\begin{pgfscope}%
\pgfpathrectangle{\pgfqpoint{1.286132in}{0.839159in}}{\pgfqpoint{12.053712in}{5.967710in}}%
\pgfusepath{clip}%
\pgfsetbuttcap%
\pgfsetroundjoin%
\pgfsetlinewidth{1.505625pt}%
\definecolor{currentstroke}{rgb}{0.121569,0.466667,0.705882}%
\pgfsetstrokecolor{currentstroke}%
\pgfsetdash{}{0pt}%
\pgfpathmoveto{\pgfqpoint{8.143133in}{1.123676in}}%
\pgfpathlineto{\pgfqpoint{8.143133in}{1.126144in}}%
\pgfusepath{stroke}%
\end{pgfscope}%
\begin{pgfscope}%
\pgfpathrectangle{\pgfqpoint{1.286132in}{0.839159in}}{\pgfqpoint{12.053712in}{5.967710in}}%
\pgfusepath{clip}%
\pgfsetbuttcap%
\pgfsetroundjoin%
\pgfsetlinewidth{1.505625pt}%
\definecolor{currentstroke}{rgb}{0.121569,0.466667,0.705882}%
\pgfsetstrokecolor{currentstroke}%
\pgfsetdash{}{0pt}%
\pgfpathmoveto{\pgfqpoint{8.253819in}{1.123797in}}%
\pgfpathlineto{\pgfqpoint{8.253819in}{1.125767in}}%
\pgfusepath{stroke}%
\end{pgfscope}%
\begin{pgfscope}%
\pgfpathrectangle{\pgfqpoint{1.286132in}{0.839159in}}{\pgfqpoint{12.053712in}{5.967710in}}%
\pgfusepath{clip}%
\pgfsetbuttcap%
\pgfsetroundjoin%
\pgfsetlinewidth{1.505625pt}%
\definecolor{currentstroke}{rgb}{0.121569,0.466667,0.705882}%
\pgfsetstrokecolor{currentstroke}%
\pgfsetdash{}{0pt}%
\pgfpathmoveto{\pgfqpoint{8.364505in}{1.124615in}}%
\pgfpathlineto{\pgfqpoint{8.364505in}{1.126477in}}%
\pgfusepath{stroke}%
\end{pgfscope}%
\begin{pgfscope}%
\pgfpathrectangle{\pgfqpoint{1.286132in}{0.839159in}}{\pgfqpoint{12.053712in}{5.967710in}}%
\pgfusepath{clip}%
\pgfsetbuttcap%
\pgfsetroundjoin%
\pgfsetlinewidth{1.505625pt}%
\definecolor{currentstroke}{rgb}{0.121569,0.466667,0.705882}%
\pgfsetstrokecolor{currentstroke}%
\pgfsetdash{}{0pt}%
\pgfpathmoveto{\pgfqpoint{8.475191in}{1.124743in}}%
\pgfpathlineto{\pgfqpoint{8.475191in}{1.127445in}}%
\pgfusepath{stroke}%
\end{pgfscope}%
\begin{pgfscope}%
\pgfpathrectangle{\pgfqpoint{1.286132in}{0.839159in}}{\pgfqpoint{12.053712in}{5.967710in}}%
\pgfusepath{clip}%
\pgfsetbuttcap%
\pgfsetroundjoin%
\pgfsetlinewidth{1.505625pt}%
\definecolor{currentstroke}{rgb}{0.121569,0.466667,0.705882}%
\pgfsetstrokecolor{currentstroke}%
\pgfsetdash{}{0pt}%
\pgfpathmoveto{\pgfqpoint{8.585877in}{1.125271in}}%
\pgfpathlineto{\pgfqpoint{8.585877in}{1.126577in}}%
\pgfusepath{stroke}%
\end{pgfscope}%
\begin{pgfscope}%
\pgfpathrectangle{\pgfqpoint{1.286132in}{0.839159in}}{\pgfqpoint{12.053712in}{5.967710in}}%
\pgfusepath{clip}%
\pgfsetbuttcap%
\pgfsetroundjoin%
\pgfsetlinewidth{1.505625pt}%
\definecolor{currentstroke}{rgb}{0.121569,0.466667,0.705882}%
\pgfsetstrokecolor{currentstroke}%
\pgfsetdash{}{0pt}%
\pgfpathmoveto{\pgfqpoint{8.696563in}{1.124053in}}%
\pgfpathlineto{\pgfqpoint{8.696563in}{1.126046in}}%
\pgfusepath{stroke}%
\end{pgfscope}%
\begin{pgfscope}%
\pgfpathrectangle{\pgfqpoint{1.286132in}{0.839159in}}{\pgfqpoint{12.053712in}{5.967710in}}%
\pgfusepath{clip}%
\pgfsetbuttcap%
\pgfsetroundjoin%
\pgfsetlinewidth{1.505625pt}%
\definecolor{currentstroke}{rgb}{0.121569,0.466667,0.705882}%
\pgfsetstrokecolor{currentstroke}%
\pgfsetdash{}{0pt}%
\pgfpathmoveto{\pgfqpoint{8.807249in}{1.126372in}}%
\pgfpathlineto{\pgfqpoint{8.807249in}{1.128305in}}%
\pgfusepath{stroke}%
\end{pgfscope}%
\begin{pgfscope}%
\pgfpathrectangle{\pgfqpoint{1.286132in}{0.839159in}}{\pgfqpoint{12.053712in}{5.967710in}}%
\pgfusepath{clip}%
\pgfsetbuttcap%
\pgfsetroundjoin%
\pgfsetlinewidth{1.505625pt}%
\definecolor{currentstroke}{rgb}{0.121569,0.466667,0.705882}%
\pgfsetstrokecolor{currentstroke}%
\pgfsetdash{}{0pt}%
\pgfpathmoveto{\pgfqpoint{8.917936in}{1.125559in}}%
\pgfpathlineto{\pgfqpoint{8.917936in}{1.126587in}}%
\pgfusepath{stroke}%
\end{pgfscope}%
\begin{pgfscope}%
\pgfpathrectangle{\pgfqpoint{1.286132in}{0.839159in}}{\pgfqpoint{12.053712in}{5.967710in}}%
\pgfusepath{clip}%
\pgfsetbuttcap%
\pgfsetroundjoin%
\pgfsetlinewidth{1.505625pt}%
\definecolor{currentstroke}{rgb}{0.121569,0.466667,0.705882}%
\pgfsetstrokecolor{currentstroke}%
\pgfsetdash{}{0pt}%
\pgfpathmoveto{\pgfqpoint{9.028622in}{1.125189in}}%
\pgfpathlineto{\pgfqpoint{9.028622in}{1.127516in}}%
\pgfusepath{stroke}%
\end{pgfscope}%
\begin{pgfscope}%
\pgfpathrectangle{\pgfqpoint{1.286132in}{0.839159in}}{\pgfqpoint{12.053712in}{5.967710in}}%
\pgfusepath{clip}%
\pgfsetbuttcap%
\pgfsetroundjoin%
\pgfsetlinewidth{1.505625pt}%
\definecolor{currentstroke}{rgb}{0.121569,0.466667,0.705882}%
\pgfsetstrokecolor{currentstroke}%
\pgfsetdash{}{0pt}%
\pgfpathmoveto{\pgfqpoint{9.139308in}{1.125037in}}%
\pgfpathlineto{\pgfqpoint{9.139308in}{1.126647in}}%
\pgfusepath{stroke}%
\end{pgfscope}%
\begin{pgfscope}%
\pgfpathrectangle{\pgfqpoint{1.286132in}{0.839159in}}{\pgfqpoint{12.053712in}{5.967710in}}%
\pgfusepath{clip}%
\pgfsetbuttcap%
\pgfsetroundjoin%
\pgfsetlinewidth{1.505625pt}%
\definecolor{currentstroke}{rgb}{0.121569,0.466667,0.705882}%
\pgfsetstrokecolor{currentstroke}%
\pgfsetdash{}{0pt}%
\pgfpathmoveto{\pgfqpoint{9.249994in}{1.125900in}}%
\pgfpathlineto{\pgfqpoint{9.249994in}{1.127878in}}%
\pgfusepath{stroke}%
\end{pgfscope}%
\begin{pgfscope}%
\pgfpathrectangle{\pgfqpoint{1.286132in}{0.839159in}}{\pgfqpoint{12.053712in}{5.967710in}}%
\pgfusepath{clip}%
\pgfsetbuttcap%
\pgfsetroundjoin%
\pgfsetlinewidth{1.505625pt}%
\definecolor{currentstroke}{rgb}{0.121569,0.466667,0.705882}%
\pgfsetstrokecolor{currentstroke}%
\pgfsetdash{}{0pt}%
\pgfpathmoveto{\pgfqpoint{9.360680in}{1.124531in}}%
\pgfpathlineto{\pgfqpoint{9.360680in}{1.128010in}}%
\pgfusepath{stroke}%
\end{pgfscope}%
\begin{pgfscope}%
\pgfpathrectangle{\pgfqpoint{1.286132in}{0.839159in}}{\pgfqpoint{12.053712in}{5.967710in}}%
\pgfusepath{clip}%
\pgfsetbuttcap%
\pgfsetroundjoin%
\pgfsetlinewidth{1.505625pt}%
\definecolor{currentstroke}{rgb}{0.121569,0.466667,0.705882}%
\pgfsetstrokecolor{currentstroke}%
\pgfsetdash{}{0pt}%
\pgfpathmoveto{\pgfqpoint{9.471366in}{1.126347in}}%
\pgfpathlineto{\pgfqpoint{9.471366in}{1.127863in}}%
\pgfusepath{stroke}%
\end{pgfscope}%
\begin{pgfscope}%
\pgfpathrectangle{\pgfqpoint{1.286132in}{0.839159in}}{\pgfqpoint{12.053712in}{5.967710in}}%
\pgfusepath{clip}%
\pgfsetbuttcap%
\pgfsetroundjoin%
\pgfsetlinewidth{1.505625pt}%
\definecolor{currentstroke}{rgb}{0.121569,0.466667,0.705882}%
\pgfsetstrokecolor{currentstroke}%
\pgfsetdash{}{0pt}%
\pgfpathmoveto{\pgfqpoint{9.582052in}{1.126451in}}%
\pgfpathlineto{\pgfqpoint{9.582052in}{1.129856in}}%
\pgfusepath{stroke}%
\end{pgfscope}%
\begin{pgfscope}%
\pgfpathrectangle{\pgfqpoint{1.286132in}{0.839159in}}{\pgfqpoint{12.053712in}{5.967710in}}%
\pgfusepath{clip}%
\pgfsetbuttcap%
\pgfsetroundjoin%
\pgfsetlinewidth{1.505625pt}%
\definecolor{currentstroke}{rgb}{0.121569,0.466667,0.705882}%
\pgfsetstrokecolor{currentstroke}%
\pgfsetdash{}{0pt}%
\pgfpathmoveto{\pgfqpoint{9.692738in}{1.126839in}}%
\pgfpathlineto{\pgfqpoint{9.692738in}{1.128049in}}%
\pgfusepath{stroke}%
\end{pgfscope}%
\begin{pgfscope}%
\pgfpathrectangle{\pgfqpoint{1.286132in}{0.839159in}}{\pgfqpoint{12.053712in}{5.967710in}}%
\pgfusepath{clip}%
\pgfsetbuttcap%
\pgfsetroundjoin%
\pgfsetlinewidth{1.505625pt}%
\definecolor{currentstroke}{rgb}{0.121569,0.466667,0.705882}%
\pgfsetstrokecolor{currentstroke}%
\pgfsetdash{}{0pt}%
\pgfpathmoveto{\pgfqpoint{9.803424in}{1.126252in}}%
\pgfpathlineto{\pgfqpoint{9.803424in}{1.127511in}}%
\pgfusepath{stroke}%
\end{pgfscope}%
\begin{pgfscope}%
\pgfpathrectangle{\pgfqpoint{1.286132in}{0.839159in}}{\pgfqpoint{12.053712in}{5.967710in}}%
\pgfusepath{clip}%
\pgfsetbuttcap%
\pgfsetroundjoin%
\pgfsetlinewidth{1.505625pt}%
\definecolor{currentstroke}{rgb}{0.121569,0.466667,0.705882}%
\pgfsetstrokecolor{currentstroke}%
\pgfsetdash{}{0pt}%
\pgfpathmoveto{\pgfqpoint{9.914110in}{1.126411in}}%
\pgfpathlineto{\pgfqpoint{9.914110in}{1.129403in}}%
\pgfusepath{stroke}%
\end{pgfscope}%
\begin{pgfscope}%
\pgfpathrectangle{\pgfqpoint{1.286132in}{0.839159in}}{\pgfqpoint{12.053712in}{5.967710in}}%
\pgfusepath{clip}%
\pgfsetbuttcap%
\pgfsetroundjoin%
\pgfsetlinewidth{1.505625pt}%
\definecolor{currentstroke}{rgb}{0.121569,0.466667,0.705882}%
\pgfsetstrokecolor{currentstroke}%
\pgfsetdash{}{0pt}%
\pgfpathmoveto{\pgfqpoint{10.024796in}{1.124989in}}%
\pgfpathlineto{\pgfqpoint{10.024796in}{1.129799in}}%
\pgfusepath{stroke}%
\end{pgfscope}%
\begin{pgfscope}%
\pgfpathrectangle{\pgfqpoint{1.286132in}{0.839159in}}{\pgfqpoint{12.053712in}{5.967710in}}%
\pgfusepath{clip}%
\pgfsetbuttcap%
\pgfsetroundjoin%
\pgfsetlinewidth{1.505625pt}%
\definecolor{currentstroke}{rgb}{0.121569,0.466667,0.705882}%
\pgfsetstrokecolor{currentstroke}%
\pgfsetdash{}{0pt}%
\pgfpathmoveto{\pgfqpoint{10.135482in}{1.126173in}}%
\pgfpathlineto{\pgfqpoint{10.135482in}{1.128400in}}%
\pgfusepath{stroke}%
\end{pgfscope}%
\begin{pgfscope}%
\pgfpathrectangle{\pgfqpoint{1.286132in}{0.839159in}}{\pgfqpoint{12.053712in}{5.967710in}}%
\pgfusepath{clip}%
\pgfsetbuttcap%
\pgfsetroundjoin%
\pgfsetlinewidth{1.505625pt}%
\definecolor{currentstroke}{rgb}{0.121569,0.466667,0.705882}%
\pgfsetstrokecolor{currentstroke}%
\pgfsetdash{}{0pt}%
\pgfpathmoveto{\pgfqpoint{10.246168in}{1.127876in}}%
\pgfpathlineto{\pgfqpoint{10.246168in}{1.129397in}}%
\pgfusepath{stroke}%
\end{pgfscope}%
\begin{pgfscope}%
\pgfpathrectangle{\pgfqpoint{1.286132in}{0.839159in}}{\pgfqpoint{12.053712in}{5.967710in}}%
\pgfusepath{clip}%
\pgfsetbuttcap%
\pgfsetroundjoin%
\pgfsetlinewidth{1.505625pt}%
\definecolor{currentstroke}{rgb}{0.121569,0.466667,0.705882}%
\pgfsetstrokecolor{currentstroke}%
\pgfsetdash{}{0pt}%
\pgfpathmoveto{\pgfqpoint{10.356854in}{1.125745in}}%
\pgfpathlineto{\pgfqpoint{10.356854in}{1.128816in}}%
\pgfusepath{stroke}%
\end{pgfscope}%
\begin{pgfscope}%
\pgfpathrectangle{\pgfqpoint{1.286132in}{0.839159in}}{\pgfqpoint{12.053712in}{5.967710in}}%
\pgfusepath{clip}%
\pgfsetbuttcap%
\pgfsetroundjoin%
\pgfsetlinewidth{1.505625pt}%
\definecolor{currentstroke}{rgb}{0.121569,0.466667,0.705882}%
\pgfsetstrokecolor{currentstroke}%
\pgfsetdash{}{0pt}%
\pgfpathmoveto{\pgfqpoint{10.467540in}{1.125722in}}%
\pgfpathlineto{\pgfqpoint{10.467540in}{1.128468in}}%
\pgfusepath{stroke}%
\end{pgfscope}%
\begin{pgfscope}%
\pgfpathrectangle{\pgfqpoint{1.286132in}{0.839159in}}{\pgfqpoint{12.053712in}{5.967710in}}%
\pgfusepath{clip}%
\pgfsetbuttcap%
\pgfsetroundjoin%
\pgfsetlinewidth{1.505625pt}%
\definecolor{currentstroke}{rgb}{0.121569,0.466667,0.705882}%
\pgfsetstrokecolor{currentstroke}%
\pgfsetdash{}{0pt}%
\pgfpathmoveto{\pgfqpoint{10.578226in}{1.128066in}}%
\pgfpathlineto{\pgfqpoint{10.578226in}{1.130060in}}%
\pgfusepath{stroke}%
\end{pgfscope}%
\begin{pgfscope}%
\pgfpathrectangle{\pgfqpoint{1.286132in}{0.839159in}}{\pgfqpoint{12.053712in}{5.967710in}}%
\pgfusepath{clip}%
\pgfsetbuttcap%
\pgfsetroundjoin%
\pgfsetlinewidth{1.505625pt}%
\definecolor{currentstroke}{rgb}{0.121569,0.466667,0.705882}%
\pgfsetstrokecolor{currentstroke}%
\pgfsetdash{}{0pt}%
\pgfpathmoveto{\pgfqpoint{10.688913in}{1.126582in}}%
\pgfpathlineto{\pgfqpoint{10.688913in}{1.129771in}}%
\pgfusepath{stroke}%
\end{pgfscope}%
\begin{pgfscope}%
\pgfpathrectangle{\pgfqpoint{1.286132in}{0.839159in}}{\pgfqpoint{12.053712in}{5.967710in}}%
\pgfusepath{clip}%
\pgfsetbuttcap%
\pgfsetroundjoin%
\pgfsetlinewidth{1.505625pt}%
\definecolor{currentstroke}{rgb}{0.121569,0.466667,0.705882}%
\pgfsetstrokecolor{currentstroke}%
\pgfsetdash{}{0pt}%
\pgfpathmoveto{\pgfqpoint{10.799599in}{1.127788in}}%
\pgfpathlineto{\pgfqpoint{10.799599in}{1.131570in}}%
\pgfusepath{stroke}%
\end{pgfscope}%
\begin{pgfscope}%
\pgfpathrectangle{\pgfqpoint{1.286132in}{0.839159in}}{\pgfqpoint{12.053712in}{5.967710in}}%
\pgfusepath{clip}%
\pgfsetbuttcap%
\pgfsetroundjoin%
\pgfsetlinewidth{1.505625pt}%
\definecolor{currentstroke}{rgb}{0.121569,0.466667,0.705882}%
\pgfsetstrokecolor{currentstroke}%
\pgfsetdash{}{0pt}%
\pgfpathmoveto{\pgfqpoint{10.910285in}{1.127250in}}%
\pgfpathlineto{\pgfqpoint{10.910285in}{1.129779in}}%
\pgfusepath{stroke}%
\end{pgfscope}%
\begin{pgfscope}%
\pgfpathrectangle{\pgfqpoint{1.286132in}{0.839159in}}{\pgfqpoint{12.053712in}{5.967710in}}%
\pgfusepath{clip}%
\pgfsetbuttcap%
\pgfsetroundjoin%
\pgfsetlinewidth{1.505625pt}%
\definecolor{currentstroke}{rgb}{0.121569,0.466667,0.705882}%
\pgfsetstrokecolor{currentstroke}%
\pgfsetdash{}{0pt}%
\pgfpathmoveto{\pgfqpoint{11.020971in}{1.127238in}}%
\pgfpathlineto{\pgfqpoint{11.020971in}{1.134857in}}%
\pgfusepath{stroke}%
\end{pgfscope}%
\begin{pgfscope}%
\pgfpathrectangle{\pgfqpoint{1.286132in}{0.839159in}}{\pgfqpoint{12.053712in}{5.967710in}}%
\pgfusepath{clip}%
\pgfsetbuttcap%
\pgfsetroundjoin%
\pgfsetlinewidth{1.505625pt}%
\definecolor{currentstroke}{rgb}{0.121569,0.466667,0.705882}%
\pgfsetstrokecolor{currentstroke}%
\pgfsetdash{}{0pt}%
\pgfpathmoveto{\pgfqpoint{11.131657in}{1.126581in}}%
\pgfpathlineto{\pgfqpoint{11.131657in}{1.129520in}}%
\pgfusepath{stroke}%
\end{pgfscope}%
\begin{pgfscope}%
\pgfpathrectangle{\pgfqpoint{1.286132in}{0.839159in}}{\pgfqpoint{12.053712in}{5.967710in}}%
\pgfusepath{clip}%
\pgfsetbuttcap%
\pgfsetroundjoin%
\pgfsetlinewidth{1.505625pt}%
\definecolor{currentstroke}{rgb}{0.121569,0.466667,0.705882}%
\pgfsetstrokecolor{currentstroke}%
\pgfsetdash{}{0pt}%
\pgfpathmoveto{\pgfqpoint{11.242343in}{1.128252in}}%
\pgfpathlineto{\pgfqpoint{11.242343in}{1.130921in}}%
\pgfusepath{stroke}%
\end{pgfscope}%
\begin{pgfscope}%
\pgfpathrectangle{\pgfqpoint{1.286132in}{0.839159in}}{\pgfqpoint{12.053712in}{5.967710in}}%
\pgfusepath{clip}%
\pgfsetbuttcap%
\pgfsetroundjoin%
\pgfsetlinewidth{1.505625pt}%
\definecolor{currentstroke}{rgb}{0.121569,0.466667,0.705882}%
\pgfsetstrokecolor{currentstroke}%
\pgfsetdash{}{0pt}%
\pgfpathmoveto{\pgfqpoint{11.353029in}{1.126546in}}%
\pgfpathlineto{\pgfqpoint{11.353029in}{1.129970in}}%
\pgfusepath{stroke}%
\end{pgfscope}%
\begin{pgfscope}%
\pgfpathrectangle{\pgfqpoint{1.286132in}{0.839159in}}{\pgfqpoint{12.053712in}{5.967710in}}%
\pgfusepath{clip}%
\pgfsetbuttcap%
\pgfsetroundjoin%
\pgfsetlinewidth{1.505625pt}%
\definecolor{currentstroke}{rgb}{0.121569,0.466667,0.705882}%
\pgfsetstrokecolor{currentstroke}%
\pgfsetdash{}{0pt}%
\pgfpathmoveto{\pgfqpoint{11.463715in}{1.126567in}}%
\pgfpathlineto{\pgfqpoint{11.463715in}{1.131124in}}%
\pgfusepath{stroke}%
\end{pgfscope}%
\begin{pgfscope}%
\pgfpathrectangle{\pgfqpoint{1.286132in}{0.839159in}}{\pgfqpoint{12.053712in}{5.967710in}}%
\pgfusepath{clip}%
\pgfsetbuttcap%
\pgfsetroundjoin%
\pgfsetlinewidth{1.505625pt}%
\definecolor{currentstroke}{rgb}{0.121569,0.466667,0.705882}%
\pgfsetstrokecolor{currentstroke}%
\pgfsetdash{}{0pt}%
\pgfpathmoveto{\pgfqpoint{11.574401in}{1.127334in}}%
\pgfpathlineto{\pgfqpoint{11.574401in}{1.132445in}}%
\pgfusepath{stroke}%
\end{pgfscope}%
\begin{pgfscope}%
\pgfpathrectangle{\pgfqpoint{1.286132in}{0.839159in}}{\pgfqpoint{12.053712in}{5.967710in}}%
\pgfusepath{clip}%
\pgfsetbuttcap%
\pgfsetroundjoin%
\pgfsetlinewidth{1.505625pt}%
\definecolor{currentstroke}{rgb}{0.121569,0.466667,0.705882}%
\pgfsetstrokecolor{currentstroke}%
\pgfsetdash{}{0pt}%
\pgfpathmoveto{\pgfqpoint{11.685087in}{1.129045in}}%
\pgfpathlineto{\pgfqpoint{11.685087in}{1.131854in}}%
\pgfusepath{stroke}%
\end{pgfscope}%
\begin{pgfscope}%
\pgfpathrectangle{\pgfqpoint{1.286132in}{0.839159in}}{\pgfqpoint{12.053712in}{5.967710in}}%
\pgfusepath{clip}%
\pgfsetbuttcap%
\pgfsetroundjoin%
\pgfsetlinewidth{1.505625pt}%
\definecolor{currentstroke}{rgb}{0.121569,0.466667,0.705882}%
\pgfsetstrokecolor{currentstroke}%
\pgfsetdash{}{0pt}%
\pgfpathmoveto{\pgfqpoint{11.795773in}{1.127689in}}%
\pgfpathlineto{\pgfqpoint{11.795773in}{1.132669in}}%
\pgfusepath{stroke}%
\end{pgfscope}%
\begin{pgfscope}%
\pgfpathrectangle{\pgfqpoint{1.286132in}{0.839159in}}{\pgfqpoint{12.053712in}{5.967710in}}%
\pgfusepath{clip}%
\pgfsetbuttcap%
\pgfsetroundjoin%
\pgfsetlinewidth{1.505625pt}%
\definecolor{currentstroke}{rgb}{0.121569,0.466667,0.705882}%
\pgfsetstrokecolor{currentstroke}%
\pgfsetdash{}{0pt}%
\pgfpathmoveto{\pgfqpoint{11.906459in}{1.129137in}}%
\pgfpathlineto{\pgfqpoint{11.906459in}{1.137175in}}%
\pgfusepath{stroke}%
\end{pgfscope}%
\begin{pgfscope}%
\pgfpathrectangle{\pgfqpoint{1.286132in}{0.839159in}}{\pgfqpoint{12.053712in}{5.967710in}}%
\pgfusepath{clip}%
\pgfsetbuttcap%
\pgfsetroundjoin%
\pgfsetlinewidth{1.505625pt}%
\definecolor{currentstroke}{rgb}{0.121569,0.466667,0.705882}%
\pgfsetstrokecolor{currentstroke}%
\pgfsetdash{}{0pt}%
\pgfpathmoveto{\pgfqpoint{12.017145in}{1.115708in}}%
\pgfpathlineto{\pgfqpoint{12.017145in}{1.175289in}}%
\pgfusepath{stroke}%
\end{pgfscope}%
\begin{pgfscope}%
\pgfpathrectangle{\pgfqpoint{1.286132in}{0.839159in}}{\pgfqpoint{12.053712in}{5.967710in}}%
\pgfusepath{clip}%
\pgfsetbuttcap%
\pgfsetroundjoin%
\pgfsetlinewidth{1.505625pt}%
\definecolor{currentstroke}{rgb}{0.121569,0.466667,0.705882}%
\pgfsetstrokecolor{currentstroke}%
\pgfsetdash{}{0pt}%
\pgfpathmoveto{\pgfqpoint{12.127831in}{1.129341in}}%
\pgfpathlineto{\pgfqpoint{12.127831in}{1.131543in}}%
\pgfusepath{stroke}%
\end{pgfscope}%
\begin{pgfscope}%
\pgfpathrectangle{\pgfqpoint{1.286132in}{0.839159in}}{\pgfqpoint{12.053712in}{5.967710in}}%
\pgfusepath{clip}%
\pgfsetbuttcap%
\pgfsetroundjoin%
\pgfsetlinewidth{1.505625pt}%
\definecolor{currentstroke}{rgb}{0.121569,0.466667,0.705882}%
\pgfsetstrokecolor{currentstroke}%
\pgfsetdash{}{0pt}%
\pgfpathmoveto{\pgfqpoint{12.238517in}{1.128598in}}%
\pgfpathlineto{\pgfqpoint{12.238517in}{1.130174in}}%
\pgfusepath{stroke}%
\end{pgfscope}%
\begin{pgfscope}%
\pgfpathrectangle{\pgfqpoint{1.286132in}{0.839159in}}{\pgfqpoint{12.053712in}{5.967710in}}%
\pgfusepath{clip}%
\pgfsetbuttcap%
\pgfsetroundjoin%
\pgfsetlinewidth{1.505625pt}%
\definecolor{currentstroke}{rgb}{0.121569,0.466667,0.705882}%
\pgfsetstrokecolor{currentstroke}%
\pgfsetdash{}{0pt}%
\pgfpathmoveto{\pgfqpoint{12.349203in}{1.127962in}}%
\pgfpathlineto{\pgfqpoint{12.349203in}{1.131265in}}%
\pgfusepath{stroke}%
\end{pgfscope}%
\begin{pgfscope}%
\pgfpathrectangle{\pgfqpoint{1.286132in}{0.839159in}}{\pgfqpoint{12.053712in}{5.967710in}}%
\pgfusepath{clip}%
\pgfsetbuttcap%
\pgfsetroundjoin%
\pgfsetlinewidth{1.505625pt}%
\definecolor{currentstroke}{rgb}{0.121569,0.466667,0.705882}%
\pgfsetstrokecolor{currentstroke}%
\pgfsetdash{}{0pt}%
\pgfpathmoveto{\pgfqpoint{12.459890in}{1.129995in}}%
\pgfpathlineto{\pgfqpoint{12.459890in}{1.133960in}}%
\pgfusepath{stroke}%
\end{pgfscope}%
\begin{pgfscope}%
\pgfpathrectangle{\pgfqpoint{1.286132in}{0.839159in}}{\pgfqpoint{12.053712in}{5.967710in}}%
\pgfusepath{clip}%
\pgfsetbuttcap%
\pgfsetroundjoin%
\pgfsetlinewidth{1.505625pt}%
\definecolor{currentstroke}{rgb}{0.121569,0.466667,0.705882}%
\pgfsetstrokecolor{currentstroke}%
\pgfsetdash{}{0pt}%
\pgfpathmoveto{\pgfqpoint{12.570576in}{1.129397in}}%
\pgfpathlineto{\pgfqpoint{12.570576in}{1.133079in}}%
\pgfusepath{stroke}%
\end{pgfscope}%
\begin{pgfscope}%
\pgfpathrectangle{\pgfqpoint{1.286132in}{0.839159in}}{\pgfqpoint{12.053712in}{5.967710in}}%
\pgfusepath{clip}%
\pgfsetbuttcap%
\pgfsetroundjoin%
\pgfsetlinewidth{1.505625pt}%
\definecolor{currentstroke}{rgb}{0.121569,0.466667,0.705882}%
\pgfsetstrokecolor{currentstroke}%
\pgfsetdash{}{0pt}%
\pgfpathmoveto{\pgfqpoint{12.681262in}{1.129174in}}%
\pgfpathlineto{\pgfqpoint{12.681262in}{1.131509in}}%
\pgfusepath{stroke}%
\end{pgfscope}%
\begin{pgfscope}%
\pgfpathrectangle{\pgfqpoint{1.286132in}{0.839159in}}{\pgfqpoint{12.053712in}{5.967710in}}%
\pgfusepath{clip}%
\pgfsetbuttcap%
\pgfsetroundjoin%
\pgfsetlinewidth{1.505625pt}%
\definecolor{currentstroke}{rgb}{0.121569,0.466667,0.705882}%
\pgfsetstrokecolor{currentstroke}%
\pgfsetdash{}{0pt}%
\pgfpathmoveto{\pgfqpoint{12.791948in}{1.128061in}}%
\pgfpathlineto{\pgfqpoint{12.791948in}{1.130213in}}%
\pgfusepath{stroke}%
\end{pgfscope}%
\begin{pgfscope}%
\pgfpathrectangle{\pgfqpoint{1.286132in}{0.839159in}}{\pgfqpoint{12.053712in}{5.967710in}}%
\pgfusepath{clip}%
\pgfsetbuttcap%
\pgfsetroundjoin%
\pgfsetlinewidth{1.505625pt}%
\definecolor{currentstroke}{rgb}{1.000000,0.498039,0.054902}%
\pgfsetstrokecolor{currentstroke}%
\pgfsetdash{}{0pt}%
\pgfpathmoveto{\pgfqpoint{1.834028in}{1.119210in}}%
\pgfpathlineto{\pgfqpoint{1.834028in}{1.119217in}}%
\pgfusepath{stroke}%
\end{pgfscope}%
\begin{pgfscope}%
\pgfpathrectangle{\pgfqpoint{1.286132in}{0.839159in}}{\pgfqpoint{12.053712in}{5.967710in}}%
\pgfusepath{clip}%
\pgfsetbuttcap%
\pgfsetroundjoin%
\pgfsetlinewidth{1.505625pt}%
\definecolor{currentstroke}{rgb}{1.000000,0.498039,0.054902}%
\pgfsetstrokecolor{currentstroke}%
\pgfsetdash{}{0pt}%
\pgfpathmoveto{\pgfqpoint{1.944714in}{1.119220in}}%
\pgfpathlineto{\pgfqpoint{1.944714in}{1.119223in}}%
\pgfusepath{stroke}%
\end{pgfscope}%
\begin{pgfscope}%
\pgfpathrectangle{\pgfqpoint{1.286132in}{0.839159in}}{\pgfqpoint{12.053712in}{5.967710in}}%
\pgfusepath{clip}%
\pgfsetbuttcap%
\pgfsetroundjoin%
\pgfsetlinewidth{1.505625pt}%
\definecolor{currentstroke}{rgb}{1.000000,0.498039,0.054902}%
\pgfsetstrokecolor{currentstroke}%
\pgfsetdash{}{0pt}%
\pgfpathmoveto{\pgfqpoint{2.055400in}{1.119256in}}%
\pgfpathlineto{\pgfqpoint{2.055400in}{1.119263in}}%
\pgfusepath{stroke}%
\end{pgfscope}%
\begin{pgfscope}%
\pgfpathrectangle{\pgfqpoint{1.286132in}{0.839159in}}{\pgfqpoint{12.053712in}{5.967710in}}%
\pgfusepath{clip}%
\pgfsetbuttcap%
\pgfsetroundjoin%
\pgfsetlinewidth{1.505625pt}%
\definecolor{currentstroke}{rgb}{1.000000,0.498039,0.054902}%
\pgfsetstrokecolor{currentstroke}%
\pgfsetdash{}{0pt}%
\pgfpathmoveto{\pgfqpoint{2.166086in}{1.119360in}}%
\pgfpathlineto{\pgfqpoint{2.166086in}{1.119475in}}%
\pgfusepath{stroke}%
\end{pgfscope}%
\begin{pgfscope}%
\pgfpathrectangle{\pgfqpoint{1.286132in}{0.839159in}}{\pgfqpoint{12.053712in}{5.967710in}}%
\pgfusepath{clip}%
\pgfsetbuttcap%
\pgfsetroundjoin%
\pgfsetlinewidth{1.505625pt}%
\definecolor{currentstroke}{rgb}{1.000000,0.498039,0.054902}%
\pgfsetstrokecolor{currentstroke}%
\pgfsetdash{}{0pt}%
\pgfpathmoveto{\pgfqpoint{2.276772in}{1.119464in}}%
\pgfpathlineto{\pgfqpoint{2.276772in}{1.119699in}}%
\pgfusepath{stroke}%
\end{pgfscope}%
\begin{pgfscope}%
\pgfpathrectangle{\pgfqpoint{1.286132in}{0.839159in}}{\pgfqpoint{12.053712in}{5.967710in}}%
\pgfusepath{clip}%
\pgfsetbuttcap%
\pgfsetroundjoin%
\pgfsetlinewidth{1.505625pt}%
\definecolor{currentstroke}{rgb}{1.000000,0.498039,0.054902}%
\pgfsetstrokecolor{currentstroke}%
\pgfsetdash{}{0pt}%
\pgfpathmoveto{\pgfqpoint{2.387458in}{1.119708in}}%
\pgfpathlineto{\pgfqpoint{2.387458in}{1.119922in}}%
\pgfusepath{stroke}%
\end{pgfscope}%
\begin{pgfscope}%
\pgfpathrectangle{\pgfqpoint{1.286132in}{0.839159in}}{\pgfqpoint{12.053712in}{5.967710in}}%
\pgfusepath{clip}%
\pgfsetbuttcap%
\pgfsetroundjoin%
\pgfsetlinewidth{1.505625pt}%
\definecolor{currentstroke}{rgb}{1.000000,0.498039,0.054902}%
\pgfsetstrokecolor{currentstroke}%
\pgfsetdash{}{0pt}%
\pgfpathmoveto{\pgfqpoint{2.498144in}{1.120002in}}%
\pgfpathlineto{\pgfqpoint{2.498144in}{1.120376in}}%
\pgfusepath{stroke}%
\end{pgfscope}%
\begin{pgfscope}%
\pgfpathrectangle{\pgfqpoint{1.286132in}{0.839159in}}{\pgfqpoint{12.053712in}{5.967710in}}%
\pgfusepath{clip}%
\pgfsetbuttcap%
\pgfsetroundjoin%
\pgfsetlinewidth{1.505625pt}%
\definecolor{currentstroke}{rgb}{1.000000,0.498039,0.054902}%
\pgfsetstrokecolor{currentstroke}%
\pgfsetdash{}{0pt}%
\pgfpathmoveto{\pgfqpoint{2.608830in}{1.120178in}}%
\pgfpathlineto{\pgfqpoint{2.608830in}{1.120874in}}%
\pgfusepath{stroke}%
\end{pgfscope}%
\begin{pgfscope}%
\pgfpathrectangle{\pgfqpoint{1.286132in}{0.839159in}}{\pgfqpoint{12.053712in}{5.967710in}}%
\pgfusepath{clip}%
\pgfsetbuttcap%
\pgfsetroundjoin%
\pgfsetlinewidth{1.505625pt}%
\definecolor{currentstroke}{rgb}{1.000000,0.498039,0.054902}%
\pgfsetstrokecolor{currentstroke}%
\pgfsetdash{}{0pt}%
\pgfpathmoveto{\pgfqpoint{2.719516in}{1.120813in}}%
\pgfpathlineto{\pgfqpoint{2.719516in}{1.121481in}}%
\pgfusepath{stroke}%
\end{pgfscope}%
\begin{pgfscope}%
\pgfpathrectangle{\pgfqpoint{1.286132in}{0.839159in}}{\pgfqpoint{12.053712in}{5.967710in}}%
\pgfusepath{clip}%
\pgfsetbuttcap%
\pgfsetroundjoin%
\pgfsetlinewidth{1.505625pt}%
\definecolor{currentstroke}{rgb}{1.000000,0.498039,0.054902}%
\pgfsetstrokecolor{currentstroke}%
\pgfsetdash{}{0pt}%
\pgfpathmoveto{\pgfqpoint{2.830202in}{1.120884in}}%
\pgfpathlineto{\pgfqpoint{2.830202in}{1.122029in}}%
\pgfusepath{stroke}%
\end{pgfscope}%
\begin{pgfscope}%
\pgfpathrectangle{\pgfqpoint{1.286132in}{0.839159in}}{\pgfqpoint{12.053712in}{5.967710in}}%
\pgfusepath{clip}%
\pgfsetbuttcap%
\pgfsetroundjoin%
\pgfsetlinewidth{1.505625pt}%
\definecolor{currentstroke}{rgb}{1.000000,0.498039,0.054902}%
\pgfsetstrokecolor{currentstroke}%
\pgfsetdash{}{0pt}%
\pgfpathmoveto{\pgfqpoint{2.940888in}{1.116957in}}%
\pgfpathlineto{\pgfqpoint{2.940888in}{1.138885in}}%
\pgfusepath{stroke}%
\end{pgfscope}%
\begin{pgfscope}%
\pgfpathrectangle{\pgfqpoint{1.286132in}{0.839159in}}{\pgfqpoint{12.053712in}{5.967710in}}%
\pgfusepath{clip}%
\pgfsetbuttcap%
\pgfsetroundjoin%
\pgfsetlinewidth{1.505625pt}%
\definecolor{currentstroke}{rgb}{1.000000,0.498039,0.054902}%
\pgfsetstrokecolor{currentstroke}%
\pgfsetdash{}{0pt}%
\pgfpathmoveto{\pgfqpoint{3.051574in}{1.122345in}}%
\pgfpathlineto{\pgfqpoint{3.051574in}{1.124240in}}%
\pgfusepath{stroke}%
\end{pgfscope}%
\begin{pgfscope}%
\pgfpathrectangle{\pgfqpoint{1.286132in}{0.839159in}}{\pgfqpoint{12.053712in}{5.967710in}}%
\pgfusepath{clip}%
\pgfsetbuttcap%
\pgfsetroundjoin%
\pgfsetlinewidth{1.505625pt}%
\definecolor{currentstroke}{rgb}{1.000000,0.498039,0.054902}%
\pgfsetstrokecolor{currentstroke}%
\pgfsetdash{}{0pt}%
\pgfpathmoveto{\pgfqpoint{3.162260in}{1.123223in}}%
\pgfpathlineto{\pgfqpoint{3.162260in}{1.124572in}}%
\pgfusepath{stroke}%
\end{pgfscope}%
\begin{pgfscope}%
\pgfpathrectangle{\pgfqpoint{1.286132in}{0.839159in}}{\pgfqpoint{12.053712in}{5.967710in}}%
\pgfusepath{clip}%
\pgfsetbuttcap%
\pgfsetroundjoin%
\pgfsetlinewidth{1.505625pt}%
\definecolor{currentstroke}{rgb}{1.000000,0.498039,0.054902}%
\pgfsetstrokecolor{currentstroke}%
\pgfsetdash{}{0pt}%
\pgfpathmoveto{\pgfqpoint{3.272946in}{1.118581in}}%
\pgfpathlineto{\pgfqpoint{3.272946in}{1.138929in}}%
\pgfusepath{stroke}%
\end{pgfscope}%
\begin{pgfscope}%
\pgfpathrectangle{\pgfqpoint{1.286132in}{0.839159in}}{\pgfqpoint{12.053712in}{5.967710in}}%
\pgfusepath{clip}%
\pgfsetbuttcap%
\pgfsetroundjoin%
\pgfsetlinewidth{1.505625pt}%
\definecolor{currentstroke}{rgb}{1.000000,0.498039,0.054902}%
\pgfsetstrokecolor{currentstroke}%
\pgfsetdash{}{0pt}%
\pgfpathmoveto{\pgfqpoint{3.383632in}{1.123998in}}%
\pgfpathlineto{\pgfqpoint{3.383632in}{1.127970in}}%
\pgfusepath{stroke}%
\end{pgfscope}%
\begin{pgfscope}%
\pgfpathrectangle{\pgfqpoint{1.286132in}{0.839159in}}{\pgfqpoint{12.053712in}{5.967710in}}%
\pgfusepath{clip}%
\pgfsetbuttcap%
\pgfsetroundjoin%
\pgfsetlinewidth{1.505625pt}%
\definecolor{currentstroke}{rgb}{1.000000,0.498039,0.054902}%
\pgfsetstrokecolor{currentstroke}%
\pgfsetdash{}{0pt}%
\pgfpathmoveto{\pgfqpoint{3.494319in}{1.125900in}}%
\pgfpathlineto{\pgfqpoint{3.494319in}{1.129478in}}%
\pgfusepath{stroke}%
\end{pgfscope}%
\begin{pgfscope}%
\pgfpathrectangle{\pgfqpoint{1.286132in}{0.839159in}}{\pgfqpoint{12.053712in}{5.967710in}}%
\pgfusepath{clip}%
\pgfsetbuttcap%
\pgfsetroundjoin%
\pgfsetlinewidth{1.505625pt}%
\definecolor{currentstroke}{rgb}{1.000000,0.498039,0.054902}%
\pgfsetstrokecolor{currentstroke}%
\pgfsetdash{}{0pt}%
\pgfpathmoveto{\pgfqpoint{3.605005in}{1.126269in}}%
\pgfpathlineto{\pgfqpoint{3.605005in}{1.128833in}}%
\pgfusepath{stroke}%
\end{pgfscope}%
\begin{pgfscope}%
\pgfpathrectangle{\pgfqpoint{1.286132in}{0.839159in}}{\pgfqpoint{12.053712in}{5.967710in}}%
\pgfusepath{clip}%
\pgfsetbuttcap%
\pgfsetroundjoin%
\pgfsetlinewidth{1.505625pt}%
\definecolor{currentstroke}{rgb}{1.000000,0.498039,0.054902}%
\pgfsetstrokecolor{currentstroke}%
\pgfsetdash{}{0pt}%
\pgfpathmoveto{\pgfqpoint{3.715691in}{1.126948in}}%
\pgfpathlineto{\pgfqpoint{3.715691in}{1.131472in}}%
\pgfusepath{stroke}%
\end{pgfscope}%
\begin{pgfscope}%
\pgfpathrectangle{\pgfqpoint{1.286132in}{0.839159in}}{\pgfqpoint{12.053712in}{5.967710in}}%
\pgfusepath{clip}%
\pgfsetbuttcap%
\pgfsetroundjoin%
\pgfsetlinewidth{1.505625pt}%
\definecolor{currentstroke}{rgb}{1.000000,0.498039,0.054902}%
\pgfsetstrokecolor{currentstroke}%
\pgfsetdash{}{0pt}%
\pgfpathmoveto{\pgfqpoint{3.826377in}{1.128195in}}%
\pgfpathlineto{\pgfqpoint{3.826377in}{1.134564in}}%
\pgfusepath{stroke}%
\end{pgfscope}%
\begin{pgfscope}%
\pgfpathrectangle{\pgfqpoint{1.286132in}{0.839159in}}{\pgfqpoint{12.053712in}{5.967710in}}%
\pgfusepath{clip}%
\pgfsetbuttcap%
\pgfsetroundjoin%
\pgfsetlinewidth{1.505625pt}%
\definecolor{currentstroke}{rgb}{1.000000,0.498039,0.054902}%
\pgfsetstrokecolor{currentstroke}%
\pgfsetdash{}{0pt}%
\pgfpathmoveto{\pgfqpoint{3.937063in}{1.127548in}}%
\pgfpathlineto{\pgfqpoint{3.937063in}{1.135827in}}%
\pgfusepath{stroke}%
\end{pgfscope}%
\begin{pgfscope}%
\pgfpathrectangle{\pgfqpoint{1.286132in}{0.839159in}}{\pgfqpoint{12.053712in}{5.967710in}}%
\pgfusepath{clip}%
\pgfsetbuttcap%
\pgfsetroundjoin%
\pgfsetlinewidth{1.505625pt}%
\definecolor{currentstroke}{rgb}{1.000000,0.498039,0.054902}%
\pgfsetstrokecolor{currentstroke}%
\pgfsetdash{}{0pt}%
\pgfpathmoveto{\pgfqpoint{4.047749in}{1.130415in}}%
\pgfpathlineto{\pgfqpoint{4.047749in}{1.130896in}}%
\pgfusepath{stroke}%
\end{pgfscope}%
\begin{pgfscope}%
\pgfpathrectangle{\pgfqpoint{1.286132in}{0.839159in}}{\pgfqpoint{12.053712in}{5.967710in}}%
\pgfusepath{clip}%
\pgfsetbuttcap%
\pgfsetroundjoin%
\pgfsetlinewidth{1.505625pt}%
\definecolor{currentstroke}{rgb}{1.000000,0.498039,0.054902}%
\pgfsetstrokecolor{currentstroke}%
\pgfsetdash{}{0pt}%
\pgfpathmoveto{\pgfqpoint{4.158435in}{1.129521in}}%
\pgfpathlineto{\pgfqpoint{4.158435in}{1.135462in}}%
\pgfusepath{stroke}%
\end{pgfscope}%
\begin{pgfscope}%
\pgfpathrectangle{\pgfqpoint{1.286132in}{0.839159in}}{\pgfqpoint{12.053712in}{5.967710in}}%
\pgfusepath{clip}%
\pgfsetbuttcap%
\pgfsetroundjoin%
\pgfsetlinewidth{1.505625pt}%
\definecolor{currentstroke}{rgb}{1.000000,0.498039,0.054902}%
\pgfsetstrokecolor{currentstroke}%
\pgfsetdash{}{0pt}%
\pgfpathmoveto{\pgfqpoint{4.269121in}{1.132599in}}%
\pgfpathlineto{\pgfqpoint{4.269121in}{1.139516in}}%
\pgfusepath{stroke}%
\end{pgfscope}%
\begin{pgfscope}%
\pgfpathrectangle{\pgfqpoint{1.286132in}{0.839159in}}{\pgfqpoint{12.053712in}{5.967710in}}%
\pgfusepath{clip}%
\pgfsetbuttcap%
\pgfsetroundjoin%
\pgfsetlinewidth{1.505625pt}%
\definecolor{currentstroke}{rgb}{1.000000,0.498039,0.054902}%
\pgfsetstrokecolor{currentstroke}%
\pgfsetdash{}{0pt}%
\pgfpathmoveto{\pgfqpoint{4.379807in}{1.135242in}}%
\pgfpathlineto{\pgfqpoint{4.379807in}{1.142856in}}%
\pgfusepath{stroke}%
\end{pgfscope}%
\begin{pgfscope}%
\pgfpathrectangle{\pgfqpoint{1.286132in}{0.839159in}}{\pgfqpoint{12.053712in}{5.967710in}}%
\pgfusepath{clip}%
\pgfsetbuttcap%
\pgfsetroundjoin%
\pgfsetlinewidth{1.505625pt}%
\definecolor{currentstroke}{rgb}{1.000000,0.498039,0.054902}%
\pgfsetstrokecolor{currentstroke}%
\pgfsetdash{}{0pt}%
\pgfpathmoveto{\pgfqpoint{4.490493in}{1.140387in}}%
\pgfpathlineto{\pgfqpoint{4.490493in}{1.144113in}}%
\pgfusepath{stroke}%
\end{pgfscope}%
\begin{pgfscope}%
\pgfpathrectangle{\pgfqpoint{1.286132in}{0.839159in}}{\pgfqpoint{12.053712in}{5.967710in}}%
\pgfusepath{clip}%
\pgfsetbuttcap%
\pgfsetroundjoin%
\pgfsetlinewidth{1.505625pt}%
\definecolor{currentstroke}{rgb}{1.000000,0.498039,0.054902}%
\pgfsetstrokecolor{currentstroke}%
\pgfsetdash{}{0pt}%
\pgfpathmoveto{\pgfqpoint{4.601179in}{1.139632in}}%
\pgfpathlineto{\pgfqpoint{4.601179in}{1.150840in}}%
\pgfusepath{stroke}%
\end{pgfscope}%
\begin{pgfscope}%
\pgfpathrectangle{\pgfqpoint{1.286132in}{0.839159in}}{\pgfqpoint{12.053712in}{5.967710in}}%
\pgfusepath{clip}%
\pgfsetbuttcap%
\pgfsetroundjoin%
\pgfsetlinewidth{1.505625pt}%
\definecolor{currentstroke}{rgb}{1.000000,0.498039,0.054902}%
\pgfsetstrokecolor{currentstroke}%
\pgfsetdash{}{0pt}%
\pgfpathmoveto{\pgfqpoint{4.711865in}{1.144740in}}%
\pgfpathlineto{\pgfqpoint{4.711865in}{1.145389in}}%
\pgfusepath{stroke}%
\end{pgfscope}%
\begin{pgfscope}%
\pgfpathrectangle{\pgfqpoint{1.286132in}{0.839159in}}{\pgfqpoint{12.053712in}{5.967710in}}%
\pgfusepath{clip}%
\pgfsetbuttcap%
\pgfsetroundjoin%
\pgfsetlinewidth{1.505625pt}%
\definecolor{currentstroke}{rgb}{1.000000,0.498039,0.054902}%
\pgfsetstrokecolor{currentstroke}%
\pgfsetdash{}{0pt}%
\pgfpathmoveto{\pgfqpoint{4.822551in}{1.142381in}}%
\pgfpathlineto{\pgfqpoint{4.822551in}{1.148113in}}%
\pgfusepath{stroke}%
\end{pgfscope}%
\begin{pgfscope}%
\pgfpathrectangle{\pgfqpoint{1.286132in}{0.839159in}}{\pgfqpoint{12.053712in}{5.967710in}}%
\pgfusepath{clip}%
\pgfsetbuttcap%
\pgfsetroundjoin%
\pgfsetlinewidth{1.505625pt}%
\definecolor{currentstroke}{rgb}{1.000000,0.498039,0.054902}%
\pgfsetstrokecolor{currentstroke}%
\pgfsetdash{}{0pt}%
\pgfpathmoveto{\pgfqpoint{4.933237in}{1.145576in}}%
\pgfpathlineto{\pgfqpoint{4.933237in}{1.155429in}}%
\pgfusepath{stroke}%
\end{pgfscope}%
\begin{pgfscope}%
\pgfpathrectangle{\pgfqpoint{1.286132in}{0.839159in}}{\pgfqpoint{12.053712in}{5.967710in}}%
\pgfusepath{clip}%
\pgfsetbuttcap%
\pgfsetroundjoin%
\pgfsetlinewidth{1.505625pt}%
\definecolor{currentstroke}{rgb}{1.000000,0.498039,0.054902}%
\pgfsetstrokecolor{currentstroke}%
\pgfsetdash{}{0pt}%
\pgfpathmoveto{\pgfqpoint{5.043923in}{1.150259in}}%
\pgfpathlineto{\pgfqpoint{5.043923in}{1.163235in}}%
\pgfusepath{stroke}%
\end{pgfscope}%
\begin{pgfscope}%
\pgfpathrectangle{\pgfqpoint{1.286132in}{0.839159in}}{\pgfqpoint{12.053712in}{5.967710in}}%
\pgfusepath{clip}%
\pgfsetbuttcap%
\pgfsetroundjoin%
\pgfsetlinewidth{1.505625pt}%
\definecolor{currentstroke}{rgb}{1.000000,0.498039,0.054902}%
\pgfsetstrokecolor{currentstroke}%
\pgfsetdash{}{0pt}%
\pgfpathmoveto{\pgfqpoint{5.154609in}{1.150433in}}%
\pgfpathlineto{\pgfqpoint{5.154609in}{1.166589in}}%
\pgfusepath{stroke}%
\end{pgfscope}%
\begin{pgfscope}%
\pgfpathrectangle{\pgfqpoint{1.286132in}{0.839159in}}{\pgfqpoint{12.053712in}{5.967710in}}%
\pgfusepath{clip}%
\pgfsetbuttcap%
\pgfsetroundjoin%
\pgfsetlinewidth{1.505625pt}%
\definecolor{currentstroke}{rgb}{1.000000,0.498039,0.054902}%
\pgfsetstrokecolor{currentstroke}%
\pgfsetdash{}{0pt}%
\pgfpathmoveto{\pgfqpoint{5.265296in}{1.148072in}}%
\pgfpathlineto{\pgfqpoint{5.265296in}{1.161428in}}%
\pgfusepath{stroke}%
\end{pgfscope}%
\begin{pgfscope}%
\pgfpathrectangle{\pgfqpoint{1.286132in}{0.839159in}}{\pgfqpoint{12.053712in}{5.967710in}}%
\pgfusepath{clip}%
\pgfsetbuttcap%
\pgfsetroundjoin%
\pgfsetlinewidth{1.505625pt}%
\definecolor{currentstroke}{rgb}{1.000000,0.498039,0.054902}%
\pgfsetstrokecolor{currentstroke}%
\pgfsetdash{}{0pt}%
\pgfpathmoveto{\pgfqpoint{5.375982in}{1.149594in}}%
\pgfpathlineto{\pgfqpoint{5.375982in}{1.163443in}}%
\pgfusepath{stroke}%
\end{pgfscope}%
\begin{pgfscope}%
\pgfpathrectangle{\pgfqpoint{1.286132in}{0.839159in}}{\pgfqpoint{12.053712in}{5.967710in}}%
\pgfusepath{clip}%
\pgfsetbuttcap%
\pgfsetroundjoin%
\pgfsetlinewidth{1.505625pt}%
\definecolor{currentstroke}{rgb}{1.000000,0.498039,0.054902}%
\pgfsetstrokecolor{currentstroke}%
\pgfsetdash{}{0pt}%
\pgfpathmoveto{\pgfqpoint{5.486668in}{1.150882in}}%
\pgfpathlineto{\pgfqpoint{5.486668in}{1.208982in}}%
\pgfusepath{stroke}%
\end{pgfscope}%
\begin{pgfscope}%
\pgfpathrectangle{\pgfqpoint{1.286132in}{0.839159in}}{\pgfqpoint{12.053712in}{5.967710in}}%
\pgfusepath{clip}%
\pgfsetbuttcap%
\pgfsetroundjoin%
\pgfsetlinewidth{1.505625pt}%
\definecolor{currentstroke}{rgb}{1.000000,0.498039,0.054902}%
\pgfsetstrokecolor{currentstroke}%
\pgfsetdash{}{0pt}%
\pgfpathmoveto{\pgfqpoint{5.597354in}{1.168340in}}%
\pgfpathlineto{\pgfqpoint{5.597354in}{1.200224in}}%
\pgfusepath{stroke}%
\end{pgfscope}%
\begin{pgfscope}%
\pgfpathrectangle{\pgfqpoint{1.286132in}{0.839159in}}{\pgfqpoint{12.053712in}{5.967710in}}%
\pgfusepath{clip}%
\pgfsetbuttcap%
\pgfsetroundjoin%
\pgfsetlinewidth{1.505625pt}%
\definecolor{currentstroke}{rgb}{1.000000,0.498039,0.054902}%
\pgfsetstrokecolor{currentstroke}%
\pgfsetdash{}{0pt}%
\pgfpathmoveto{\pgfqpoint{5.708040in}{1.198816in}}%
\pgfpathlineto{\pgfqpoint{5.708040in}{1.289654in}}%
\pgfusepath{stroke}%
\end{pgfscope}%
\begin{pgfscope}%
\pgfpathrectangle{\pgfqpoint{1.286132in}{0.839159in}}{\pgfqpoint{12.053712in}{5.967710in}}%
\pgfusepath{clip}%
\pgfsetbuttcap%
\pgfsetroundjoin%
\pgfsetlinewidth{1.505625pt}%
\definecolor{currentstroke}{rgb}{1.000000,0.498039,0.054902}%
\pgfsetstrokecolor{currentstroke}%
\pgfsetdash{}{0pt}%
\pgfpathmoveto{\pgfqpoint{5.818726in}{1.176899in}}%
\pgfpathlineto{\pgfqpoint{5.818726in}{1.214287in}}%
\pgfusepath{stroke}%
\end{pgfscope}%
\begin{pgfscope}%
\pgfpathrectangle{\pgfqpoint{1.286132in}{0.839159in}}{\pgfqpoint{12.053712in}{5.967710in}}%
\pgfusepath{clip}%
\pgfsetbuttcap%
\pgfsetroundjoin%
\pgfsetlinewidth{1.505625pt}%
\definecolor{currentstroke}{rgb}{1.000000,0.498039,0.054902}%
\pgfsetstrokecolor{currentstroke}%
\pgfsetdash{}{0pt}%
\pgfpathmoveto{\pgfqpoint{5.929412in}{1.168474in}}%
\pgfpathlineto{\pgfqpoint{5.929412in}{1.189792in}}%
\pgfusepath{stroke}%
\end{pgfscope}%
\begin{pgfscope}%
\pgfpathrectangle{\pgfqpoint{1.286132in}{0.839159in}}{\pgfqpoint{12.053712in}{5.967710in}}%
\pgfusepath{clip}%
\pgfsetbuttcap%
\pgfsetroundjoin%
\pgfsetlinewidth{1.505625pt}%
\definecolor{currentstroke}{rgb}{1.000000,0.498039,0.054902}%
\pgfsetstrokecolor{currentstroke}%
\pgfsetdash{}{0pt}%
\pgfpathmoveto{\pgfqpoint{6.040098in}{1.174478in}}%
\pgfpathlineto{\pgfqpoint{6.040098in}{1.188828in}}%
\pgfusepath{stroke}%
\end{pgfscope}%
\begin{pgfscope}%
\pgfpathrectangle{\pgfqpoint{1.286132in}{0.839159in}}{\pgfqpoint{12.053712in}{5.967710in}}%
\pgfusepath{clip}%
\pgfsetbuttcap%
\pgfsetroundjoin%
\pgfsetlinewidth{1.505625pt}%
\definecolor{currentstroke}{rgb}{1.000000,0.498039,0.054902}%
\pgfsetstrokecolor{currentstroke}%
\pgfsetdash{}{0pt}%
\pgfpathmoveto{\pgfqpoint{6.150784in}{1.175345in}}%
\pgfpathlineto{\pgfqpoint{6.150784in}{1.189312in}}%
\pgfusepath{stroke}%
\end{pgfscope}%
\begin{pgfscope}%
\pgfpathrectangle{\pgfqpoint{1.286132in}{0.839159in}}{\pgfqpoint{12.053712in}{5.967710in}}%
\pgfusepath{clip}%
\pgfsetbuttcap%
\pgfsetroundjoin%
\pgfsetlinewidth{1.505625pt}%
\definecolor{currentstroke}{rgb}{1.000000,0.498039,0.054902}%
\pgfsetstrokecolor{currentstroke}%
\pgfsetdash{}{0pt}%
\pgfpathmoveto{\pgfqpoint{6.261470in}{1.176933in}}%
\pgfpathlineto{\pgfqpoint{6.261470in}{1.201277in}}%
\pgfusepath{stroke}%
\end{pgfscope}%
\begin{pgfscope}%
\pgfpathrectangle{\pgfqpoint{1.286132in}{0.839159in}}{\pgfqpoint{12.053712in}{5.967710in}}%
\pgfusepath{clip}%
\pgfsetbuttcap%
\pgfsetroundjoin%
\pgfsetlinewidth{1.505625pt}%
\definecolor{currentstroke}{rgb}{1.000000,0.498039,0.054902}%
\pgfsetstrokecolor{currentstroke}%
\pgfsetdash{}{0pt}%
\pgfpathmoveto{\pgfqpoint{6.372156in}{1.182933in}}%
\pgfpathlineto{\pgfqpoint{6.372156in}{1.207184in}}%
\pgfusepath{stroke}%
\end{pgfscope}%
\begin{pgfscope}%
\pgfpathrectangle{\pgfqpoint{1.286132in}{0.839159in}}{\pgfqpoint{12.053712in}{5.967710in}}%
\pgfusepath{clip}%
\pgfsetbuttcap%
\pgfsetroundjoin%
\pgfsetlinewidth{1.505625pt}%
\definecolor{currentstroke}{rgb}{1.000000,0.498039,0.054902}%
\pgfsetstrokecolor{currentstroke}%
\pgfsetdash{}{0pt}%
\pgfpathmoveto{\pgfqpoint{6.482842in}{1.194464in}}%
\pgfpathlineto{\pgfqpoint{6.482842in}{1.216971in}}%
\pgfusepath{stroke}%
\end{pgfscope}%
\begin{pgfscope}%
\pgfpathrectangle{\pgfqpoint{1.286132in}{0.839159in}}{\pgfqpoint{12.053712in}{5.967710in}}%
\pgfusepath{clip}%
\pgfsetbuttcap%
\pgfsetroundjoin%
\pgfsetlinewidth{1.505625pt}%
\definecolor{currentstroke}{rgb}{1.000000,0.498039,0.054902}%
\pgfsetstrokecolor{currentstroke}%
\pgfsetdash{}{0pt}%
\pgfpathmoveto{\pgfqpoint{6.593528in}{1.177811in}}%
\pgfpathlineto{\pgfqpoint{6.593528in}{1.207195in}}%
\pgfusepath{stroke}%
\end{pgfscope}%
\begin{pgfscope}%
\pgfpathrectangle{\pgfqpoint{1.286132in}{0.839159in}}{\pgfqpoint{12.053712in}{5.967710in}}%
\pgfusepath{clip}%
\pgfsetbuttcap%
\pgfsetroundjoin%
\pgfsetlinewidth{1.505625pt}%
\definecolor{currentstroke}{rgb}{1.000000,0.498039,0.054902}%
\pgfsetstrokecolor{currentstroke}%
\pgfsetdash{}{0pt}%
\pgfpathmoveto{\pgfqpoint{6.704214in}{1.188954in}}%
\pgfpathlineto{\pgfqpoint{6.704214in}{1.207406in}}%
\pgfusepath{stroke}%
\end{pgfscope}%
\begin{pgfscope}%
\pgfpathrectangle{\pgfqpoint{1.286132in}{0.839159in}}{\pgfqpoint{12.053712in}{5.967710in}}%
\pgfusepath{clip}%
\pgfsetbuttcap%
\pgfsetroundjoin%
\pgfsetlinewidth{1.505625pt}%
\definecolor{currentstroke}{rgb}{1.000000,0.498039,0.054902}%
\pgfsetstrokecolor{currentstroke}%
\pgfsetdash{}{0pt}%
\pgfpathmoveto{\pgfqpoint{6.814900in}{1.187552in}}%
\pgfpathlineto{\pgfqpoint{6.814900in}{1.209930in}}%
\pgfusepath{stroke}%
\end{pgfscope}%
\begin{pgfscope}%
\pgfpathrectangle{\pgfqpoint{1.286132in}{0.839159in}}{\pgfqpoint{12.053712in}{5.967710in}}%
\pgfusepath{clip}%
\pgfsetbuttcap%
\pgfsetroundjoin%
\pgfsetlinewidth{1.505625pt}%
\definecolor{currentstroke}{rgb}{1.000000,0.498039,0.054902}%
\pgfsetstrokecolor{currentstroke}%
\pgfsetdash{}{0pt}%
\pgfpathmoveto{\pgfqpoint{6.925586in}{1.197888in}}%
\pgfpathlineto{\pgfqpoint{6.925586in}{1.215719in}}%
\pgfusepath{stroke}%
\end{pgfscope}%
\begin{pgfscope}%
\pgfpathrectangle{\pgfqpoint{1.286132in}{0.839159in}}{\pgfqpoint{12.053712in}{5.967710in}}%
\pgfusepath{clip}%
\pgfsetbuttcap%
\pgfsetroundjoin%
\pgfsetlinewidth{1.505625pt}%
\definecolor{currentstroke}{rgb}{1.000000,0.498039,0.054902}%
\pgfsetstrokecolor{currentstroke}%
\pgfsetdash{}{0pt}%
\pgfpathmoveto{\pgfqpoint{7.036272in}{1.208724in}}%
\pgfpathlineto{\pgfqpoint{7.036272in}{1.227001in}}%
\pgfusepath{stroke}%
\end{pgfscope}%
\begin{pgfscope}%
\pgfpathrectangle{\pgfqpoint{1.286132in}{0.839159in}}{\pgfqpoint{12.053712in}{5.967710in}}%
\pgfusepath{clip}%
\pgfsetbuttcap%
\pgfsetroundjoin%
\pgfsetlinewidth{1.505625pt}%
\definecolor{currentstroke}{rgb}{1.000000,0.498039,0.054902}%
\pgfsetstrokecolor{currentstroke}%
\pgfsetdash{}{0pt}%
\pgfpathmoveto{\pgfqpoint{7.146959in}{1.202692in}}%
\pgfpathlineto{\pgfqpoint{7.146959in}{1.227342in}}%
\pgfusepath{stroke}%
\end{pgfscope}%
\begin{pgfscope}%
\pgfpathrectangle{\pgfqpoint{1.286132in}{0.839159in}}{\pgfqpoint{12.053712in}{5.967710in}}%
\pgfusepath{clip}%
\pgfsetbuttcap%
\pgfsetroundjoin%
\pgfsetlinewidth{1.505625pt}%
\definecolor{currentstroke}{rgb}{1.000000,0.498039,0.054902}%
\pgfsetstrokecolor{currentstroke}%
\pgfsetdash{}{0pt}%
\pgfpathmoveto{\pgfqpoint{7.257645in}{1.218328in}}%
\pgfpathlineto{\pgfqpoint{7.257645in}{1.245541in}}%
\pgfusepath{stroke}%
\end{pgfscope}%
\begin{pgfscope}%
\pgfpathrectangle{\pgfqpoint{1.286132in}{0.839159in}}{\pgfqpoint{12.053712in}{5.967710in}}%
\pgfusepath{clip}%
\pgfsetbuttcap%
\pgfsetroundjoin%
\pgfsetlinewidth{1.505625pt}%
\definecolor{currentstroke}{rgb}{1.000000,0.498039,0.054902}%
\pgfsetstrokecolor{currentstroke}%
\pgfsetdash{}{0pt}%
\pgfpathmoveto{\pgfqpoint{7.368331in}{1.215919in}}%
\pgfpathlineto{\pgfqpoint{7.368331in}{1.242107in}}%
\pgfusepath{stroke}%
\end{pgfscope}%
\begin{pgfscope}%
\pgfpathrectangle{\pgfqpoint{1.286132in}{0.839159in}}{\pgfqpoint{12.053712in}{5.967710in}}%
\pgfusepath{clip}%
\pgfsetbuttcap%
\pgfsetroundjoin%
\pgfsetlinewidth{1.505625pt}%
\definecolor{currentstroke}{rgb}{1.000000,0.498039,0.054902}%
\pgfsetstrokecolor{currentstroke}%
\pgfsetdash{}{0pt}%
\pgfpathmoveto{\pgfqpoint{7.479017in}{1.212243in}}%
\pgfpathlineto{\pgfqpoint{7.479017in}{1.252479in}}%
\pgfusepath{stroke}%
\end{pgfscope}%
\begin{pgfscope}%
\pgfpathrectangle{\pgfqpoint{1.286132in}{0.839159in}}{\pgfqpoint{12.053712in}{5.967710in}}%
\pgfusepath{clip}%
\pgfsetbuttcap%
\pgfsetroundjoin%
\pgfsetlinewidth{1.505625pt}%
\definecolor{currentstroke}{rgb}{1.000000,0.498039,0.054902}%
\pgfsetstrokecolor{currentstroke}%
\pgfsetdash{}{0pt}%
\pgfpathmoveto{\pgfqpoint{7.589703in}{1.232491in}}%
\pgfpathlineto{\pgfqpoint{7.589703in}{1.324141in}}%
\pgfusepath{stroke}%
\end{pgfscope}%
\begin{pgfscope}%
\pgfpathrectangle{\pgfqpoint{1.286132in}{0.839159in}}{\pgfqpoint{12.053712in}{5.967710in}}%
\pgfusepath{clip}%
\pgfsetbuttcap%
\pgfsetroundjoin%
\pgfsetlinewidth{1.505625pt}%
\definecolor{currentstroke}{rgb}{1.000000,0.498039,0.054902}%
\pgfsetstrokecolor{currentstroke}%
\pgfsetdash{}{0pt}%
\pgfpathmoveto{\pgfqpoint{7.700389in}{1.209679in}}%
\pgfpathlineto{\pgfqpoint{7.700389in}{1.255727in}}%
\pgfusepath{stroke}%
\end{pgfscope}%
\begin{pgfscope}%
\pgfpathrectangle{\pgfqpoint{1.286132in}{0.839159in}}{\pgfqpoint{12.053712in}{5.967710in}}%
\pgfusepath{clip}%
\pgfsetbuttcap%
\pgfsetroundjoin%
\pgfsetlinewidth{1.505625pt}%
\definecolor{currentstroke}{rgb}{1.000000,0.498039,0.054902}%
\pgfsetstrokecolor{currentstroke}%
\pgfsetdash{}{0pt}%
\pgfpathmoveto{\pgfqpoint{7.811075in}{1.231728in}}%
\pgfpathlineto{\pgfqpoint{7.811075in}{1.258781in}}%
\pgfusepath{stroke}%
\end{pgfscope}%
\begin{pgfscope}%
\pgfpathrectangle{\pgfqpoint{1.286132in}{0.839159in}}{\pgfqpoint{12.053712in}{5.967710in}}%
\pgfusepath{clip}%
\pgfsetbuttcap%
\pgfsetroundjoin%
\pgfsetlinewidth{1.505625pt}%
\definecolor{currentstroke}{rgb}{1.000000,0.498039,0.054902}%
\pgfsetstrokecolor{currentstroke}%
\pgfsetdash{}{0pt}%
\pgfpathmoveto{\pgfqpoint{7.921761in}{1.244062in}}%
\pgfpathlineto{\pgfqpoint{7.921761in}{1.280090in}}%
\pgfusepath{stroke}%
\end{pgfscope}%
\begin{pgfscope}%
\pgfpathrectangle{\pgfqpoint{1.286132in}{0.839159in}}{\pgfqpoint{12.053712in}{5.967710in}}%
\pgfusepath{clip}%
\pgfsetbuttcap%
\pgfsetroundjoin%
\pgfsetlinewidth{1.505625pt}%
\definecolor{currentstroke}{rgb}{1.000000,0.498039,0.054902}%
\pgfsetstrokecolor{currentstroke}%
\pgfsetdash{}{0pt}%
\pgfpathmoveto{\pgfqpoint{8.032447in}{1.232082in}}%
\pgfpathlineto{\pgfqpoint{8.032447in}{1.277523in}}%
\pgfusepath{stroke}%
\end{pgfscope}%
\begin{pgfscope}%
\pgfpathrectangle{\pgfqpoint{1.286132in}{0.839159in}}{\pgfqpoint{12.053712in}{5.967710in}}%
\pgfusepath{clip}%
\pgfsetbuttcap%
\pgfsetroundjoin%
\pgfsetlinewidth{1.505625pt}%
\definecolor{currentstroke}{rgb}{1.000000,0.498039,0.054902}%
\pgfsetstrokecolor{currentstroke}%
\pgfsetdash{}{0pt}%
\pgfpathmoveto{\pgfqpoint{8.143133in}{1.239673in}}%
\pgfpathlineto{\pgfqpoint{8.143133in}{1.268975in}}%
\pgfusepath{stroke}%
\end{pgfscope}%
\begin{pgfscope}%
\pgfpathrectangle{\pgfqpoint{1.286132in}{0.839159in}}{\pgfqpoint{12.053712in}{5.967710in}}%
\pgfusepath{clip}%
\pgfsetbuttcap%
\pgfsetroundjoin%
\pgfsetlinewidth{1.505625pt}%
\definecolor{currentstroke}{rgb}{1.000000,0.498039,0.054902}%
\pgfsetstrokecolor{currentstroke}%
\pgfsetdash{}{0pt}%
\pgfpathmoveto{\pgfqpoint{8.253819in}{1.254447in}}%
\pgfpathlineto{\pgfqpoint{8.253819in}{1.297602in}}%
\pgfusepath{stroke}%
\end{pgfscope}%
\begin{pgfscope}%
\pgfpathrectangle{\pgfqpoint{1.286132in}{0.839159in}}{\pgfqpoint{12.053712in}{5.967710in}}%
\pgfusepath{clip}%
\pgfsetbuttcap%
\pgfsetroundjoin%
\pgfsetlinewidth{1.505625pt}%
\definecolor{currentstroke}{rgb}{1.000000,0.498039,0.054902}%
\pgfsetstrokecolor{currentstroke}%
\pgfsetdash{}{0pt}%
\pgfpathmoveto{\pgfqpoint{8.364505in}{1.254171in}}%
\pgfpathlineto{\pgfqpoint{8.364505in}{1.303236in}}%
\pgfusepath{stroke}%
\end{pgfscope}%
\begin{pgfscope}%
\pgfpathrectangle{\pgfqpoint{1.286132in}{0.839159in}}{\pgfqpoint{12.053712in}{5.967710in}}%
\pgfusepath{clip}%
\pgfsetbuttcap%
\pgfsetroundjoin%
\pgfsetlinewidth{1.505625pt}%
\definecolor{currentstroke}{rgb}{1.000000,0.498039,0.054902}%
\pgfsetstrokecolor{currentstroke}%
\pgfsetdash{}{0pt}%
\pgfpathmoveto{\pgfqpoint{8.475191in}{1.270663in}}%
\pgfpathlineto{\pgfqpoint{8.475191in}{1.296963in}}%
\pgfusepath{stroke}%
\end{pgfscope}%
\begin{pgfscope}%
\pgfpathrectangle{\pgfqpoint{1.286132in}{0.839159in}}{\pgfqpoint{12.053712in}{5.967710in}}%
\pgfusepath{clip}%
\pgfsetbuttcap%
\pgfsetroundjoin%
\pgfsetlinewidth{1.505625pt}%
\definecolor{currentstroke}{rgb}{1.000000,0.498039,0.054902}%
\pgfsetstrokecolor{currentstroke}%
\pgfsetdash{}{0pt}%
\pgfpathmoveto{\pgfqpoint{8.585877in}{1.258658in}}%
\pgfpathlineto{\pgfqpoint{8.585877in}{1.292380in}}%
\pgfusepath{stroke}%
\end{pgfscope}%
\begin{pgfscope}%
\pgfpathrectangle{\pgfqpoint{1.286132in}{0.839159in}}{\pgfqpoint{12.053712in}{5.967710in}}%
\pgfusepath{clip}%
\pgfsetbuttcap%
\pgfsetroundjoin%
\pgfsetlinewidth{1.505625pt}%
\definecolor{currentstroke}{rgb}{1.000000,0.498039,0.054902}%
\pgfsetstrokecolor{currentstroke}%
\pgfsetdash{}{0pt}%
\pgfpathmoveto{\pgfqpoint{8.696563in}{1.259109in}}%
\pgfpathlineto{\pgfqpoint{8.696563in}{1.332071in}}%
\pgfusepath{stroke}%
\end{pgfscope}%
\begin{pgfscope}%
\pgfpathrectangle{\pgfqpoint{1.286132in}{0.839159in}}{\pgfqpoint{12.053712in}{5.967710in}}%
\pgfusepath{clip}%
\pgfsetbuttcap%
\pgfsetroundjoin%
\pgfsetlinewidth{1.505625pt}%
\definecolor{currentstroke}{rgb}{1.000000,0.498039,0.054902}%
\pgfsetstrokecolor{currentstroke}%
\pgfsetdash{}{0pt}%
\pgfpathmoveto{\pgfqpoint{8.807249in}{1.276367in}}%
\pgfpathlineto{\pgfqpoint{8.807249in}{1.313553in}}%
\pgfusepath{stroke}%
\end{pgfscope}%
\begin{pgfscope}%
\pgfpathrectangle{\pgfqpoint{1.286132in}{0.839159in}}{\pgfqpoint{12.053712in}{5.967710in}}%
\pgfusepath{clip}%
\pgfsetbuttcap%
\pgfsetroundjoin%
\pgfsetlinewidth{1.505625pt}%
\definecolor{currentstroke}{rgb}{1.000000,0.498039,0.054902}%
\pgfsetstrokecolor{currentstroke}%
\pgfsetdash{}{0pt}%
\pgfpathmoveto{\pgfqpoint{8.917936in}{1.290518in}}%
\pgfpathlineto{\pgfqpoint{8.917936in}{1.366608in}}%
\pgfusepath{stroke}%
\end{pgfscope}%
\begin{pgfscope}%
\pgfpathrectangle{\pgfqpoint{1.286132in}{0.839159in}}{\pgfqpoint{12.053712in}{5.967710in}}%
\pgfusepath{clip}%
\pgfsetbuttcap%
\pgfsetroundjoin%
\pgfsetlinewidth{1.505625pt}%
\definecolor{currentstroke}{rgb}{1.000000,0.498039,0.054902}%
\pgfsetstrokecolor{currentstroke}%
\pgfsetdash{}{0pt}%
\pgfpathmoveto{\pgfqpoint{9.028622in}{1.286967in}}%
\pgfpathlineto{\pgfqpoint{9.028622in}{1.390377in}}%
\pgfusepath{stroke}%
\end{pgfscope}%
\begin{pgfscope}%
\pgfpathrectangle{\pgfqpoint{1.286132in}{0.839159in}}{\pgfqpoint{12.053712in}{5.967710in}}%
\pgfusepath{clip}%
\pgfsetbuttcap%
\pgfsetroundjoin%
\pgfsetlinewidth{1.505625pt}%
\definecolor{currentstroke}{rgb}{1.000000,0.498039,0.054902}%
\pgfsetstrokecolor{currentstroke}%
\pgfsetdash{}{0pt}%
\pgfpathmoveto{\pgfqpoint{9.139308in}{1.311463in}}%
\pgfpathlineto{\pgfqpoint{9.139308in}{1.404522in}}%
\pgfusepath{stroke}%
\end{pgfscope}%
\begin{pgfscope}%
\pgfpathrectangle{\pgfqpoint{1.286132in}{0.839159in}}{\pgfqpoint{12.053712in}{5.967710in}}%
\pgfusepath{clip}%
\pgfsetbuttcap%
\pgfsetroundjoin%
\pgfsetlinewidth{1.505625pt}%
\definecolor{currentstroke}{rgb}{1.000000,0.498039,0.054902}%
\pgfsetstrokecolor{currentstroke}%
\pgfsetdash{}{0pt}%
\pgfpathmoveto{\pgfqpoint{9.249994in}{1.306421in}}%
\pgfpathlineto{\pgfqpoint{9.249994in}{1.340420in}}%
\pgfusepath{stroke}%
\end{pgfscope}%
\begin{pgfscope}%
\pgfpathrectangle{\pgfqpoint{1.286132in}{0.839159in}}{\pgfqpoint{12.053712in}{5.967710in}}%
\pgfusepath{clip}%
\pgfsetbuttcap%
\pgfsetroundjoin%
\pgfsetlinewidth{1.505625pt}%
\definecolor{currentstroke}{rgb}{1.000000,0.498039,0.054902}%
\pgfsetstrokecolor{currentstroke}%
\pgfsetdash{}{0pt}%
\pgfpathmoveto{\pgfqpoint{9.360680in}{1.301257in}}%
\pgfpathlineto{\pgfqpoint{9.360680in}{1.343804in}}%
\pgfusepath{stroke}%
\end{pgfscope}%
\begin{pgfscope}%
\pgfpathrectangle{\pgfqpoint{1.286132in}{0.839159in}}{\pgfqpoint{12.053712in}{5.967710in}}%
\pgfusepath{clip}%
\pgfsetbuttcap%
\pgfsetroundjoin%
\pgfsetlinewidth{1.505625pt}%
\definecolor{currentstroke}{rgb}{1.000000,0.498039,0.054902}%
\pgfsetstrokecolor{currentstroke}%
\pgfsetdash{}{0pt}%
\pgfpathmoveto{\pgfqpoint{9.471366in}{1.302854in}}%
\pgfpathlineto{\pgfqpoint{9.471366in}{1.421241in}}%
\pgfusepath{stroke}%
\end{pgfscope}%
\begin{pgfscope}%
\pgfpathrectangle{\pgfqpoint{1.286132in}{0.839159in}}{\pgfqpoint{12.053712in}{5.967710in}}%
\pgfusepath{clip}%
\pgfsetbuttcap%
\pgfsetroundjoin%
\pgfsetlinewidth{1.505625pt}%
\definecolor{currentstroke}{rgb}{1.000000,0.498039,0.054902}%
\pgfsetstrokecolor{currentstroke}%
\pgfsetdash{}{0pt}%
\pgfpathmoveto{\pgfqpoint{9.582052in}{1.311282in}}%
\pgfpathlineto{\pgfqpoint{9.582052in}{1.357371in}}%
\pgfusepath{stroke}%
\end{pgfscope}%
\begin{pgfscope}%
\pgfpathrectangle{\pgfqpoint{1.286132in}{0.839159in}}{\pgfqpoint{12.053712in}{5.967710in}}%
\pgfusepath{clip}%
\pgfsetbuttcap%
\pgfsetroundjoin%
\pgfsetlinewidth{1.505625pt}%
\definecolor{currentstroke}{rgb}{1.000000,0.498039,0.054902}%
\pgfsetstrokecolor{currentstroke}%
\pgfsetdash{}{0pt}%
\pgfpathmoveto{\pgfqpoint{9.692738in}{1.339946in}}%
\pgfpathlineto{\pgfqpoint{9.692738in}{1.433678in}}%
\pgfusepath{stroke}%
\end{pgfscope}%
\begin{pgfscope}%
\pgfpathrectangle{\pgfqpoint{1.286132in}{0.839159in}}{\pgfqpoint{12.053712in}{5.967710in}}%
\pgfusepath{clip}%
\pgfsetbuttcap%
\pgfsetroundjoin%
\pgfsetlinewidth{1.505625pt}%
\definecolor{currentstroke}{rgb}{1.000000,0.498039,0.054902}%
\pgfsetstrokecolor{currentstroke}%
\pgfsetdash{}{0pt}%
\pgfpathmoveto{\pgfqpoint{9.803424in}{1.337822in}}%
\pgfpathlineto{\pgfqpoint{9.803424in}{1.379791in}}%
\pgfusepath{stroke}%
\end{pgfscope}%
\begin{pgfscope}%
\pgfpathrectangle{\pgfqpoint{1.286132in}{0.839159in}}{\pgfqpoint{12.053712in}{5.967710in}}%
\pgfusepath{clip}%
\pgfsetbuttcap%
\pgfsetroundjoin%
\pgfsetlinewidth{1.505625pt}%
\definecolor{currentstroke}{rgb}{1.000000,0.498039,0.054902}%
\pgfsetstrokecolor{currentstroke}%
\pgfsetdash{}{0pt}%
\pgfpathmoveto{\pgfqpoint{9.914110in}{1.373293in}}%
\pgfpathlineto{\pgfqpoint{9.914110in}{1.452337in}}%
\pgfusepath{stroke}%
\end{pgfscope}%
\begin{pgfscope}%
\pgfpathrectangle{\pgfqpoint{1.286132in}{0.839159in}}{\pgfqpoint{12.053712in}{5.967710in}}%
\pgfusepath{clip}%
\pgfsetbuttcap%
\pgfsetroundjoin%
\pgfsetlinewidth{1.505625pt}%
\definecolor{currentstroke}{rgb}{1.000000,0.498039,0.054902}%
\pgfsetstrokecolor{currentstroke}%
\pgfsetdash{}{0pt}%
\pgfpathmoveto{\pgfqpoint{10.024796in}{1.353768in}}%
\pgfpathlineto{\pgfqpoint{10.024796in}{1.429094in}}%
\pgfusepath{stroke}%
\end{pgfscope}%
\begin{pgfscope}%
\pgfpathrectangle{\pgfqpoint{1.286132in}{0.839159in}}{\pgfqpoint{12.053712in}{5.967710in}}%
\pgfusepath{clip}%
\pgfsetbuttcap%
\pgfsetroundjoin%
\pgfsetlinewidth{1.505625pt}%
\definecolor{currentstroke}{rgb}{1.000000,0.498039,0.054902}%
\pgfsetstrokecolor{currentstroke}%
\pgfsetdash{}{0pt}%
\pgfpathmoveto{\pgfqpoint{10.135482in}{1.345001in}}%
\pgfpathlineto{\pgfqpoint{10.135482in}{1.449029in}}%
\pgfusepath{stroke}%
\end{pgfscope}%
\begin{pgfscope}%
\pgfpathrectangle{\pgfqpoint{1.286132in}{0.839159in}}{\pgfqpoint{12.053712in}{5.967710in}}%
\pgfusepath{clip}%
\pgfsetbuttcap%
\pgfsetroundjoin%
\pgfsetlinewidth{1.505625pt}%
\definecolor{currentstroke}{rgb}{1.000000,0.498039,0.054902}%
\pgfsetstrokecolor{currentstroke}%
\pgfsetdash{}{0pt}%
\pgfpathmoveto{\pgfqpoint{10.246168in}{1.334910in}}%
\pgfpathlineto{\pgfqpoint{10.246168in}{1.389131in}}%
\pgfusepath{stroke}%
\end{pgfscope}%
\begin{pgfscope}%
\pgfpathrectangle{\pgfqpoint{1.286132in}{0.839159in}}{\pgfqpoint{12.053712in}{5.967710in}}%
\pgfusepath{clip}%
\pgfsetbuttcap%
\pgfsetroundjoin%
\pgfsetlinewidth{1.505625pt}%
\definecolor{currentstroke}{rgb}{1.000000,0.498039,0.054902}%
\pgfsetstrokecolor{currentstroke}%
\pgfsetdash{}{0pt}%
\pgfpathmoveto{\pgfqpoint{10.356854in}{1.397091in}}%
\pgfpathlineto{\pgfqpoint{10.356854in}{1.452181in}}%
\pgfusepath{stroke}%
\end{pgfscope}%
\begin{pgfscope}%
\pgfpathrectangle{\pgfqpoint{1.286132in}{0.839159in}}{\pgfqpoint{12.053712in}{5.967710in}}%
\pgfusepath{clip}%
\pgfsetbuttcap%
\pgfsetroundjoin%
\pgfsetlinewidth{1.505625pt}%
\definecolor{currentstroke}{rgb}{1.000000,0.498039,0.054902}%
\pgfsetstrokecolor{currentstroke}%
\pgfsetdash{}{0pt}%
\pgfpathmoveto{\pgfqpoint{10.467540in}{1.375960in}}%
\pgfpathlineto{\pgfqpoint{10.467540in}{1.490807in}}%
\pgfusepath{stroke}%
\end{pgfscope}%
\begin{pgfscope}%
\pgfpathrectangle{\pgfqpoint{1.286132in}{0.839159in}}{\pgfqpoint{12.053712in}{5.967710in}}%
\pgfusepath{clip}%
\pgfsetbuttcap%
\pgfsetroundjoin%
\pgfsetlinewidth{1.505625pt}%
\definecolor{currentstroke}{rgb}{1.000000,0.498039,0.054902}%
\pgfsetstrokecolor{currentstroke}%
\pgfsetdash{}{0pt}%
\pgfpathmoveto{\pgfqpoint{10.578226in}{1.430863in}}%
\pgfpathlineto{\pgfqpoint{10.578226in}{1.526156in}}%
\pgfusepath{stroke}%
\end{pgfscope}%
\begin{pgfscope}%
\pgfpathrectangle{\pgfqpoint{1.286132in}{0.839159in}}{\pgfqpoint{12.053712in}{5.967710in}}%
\pgfusepath{clip}%
\pgfsetbuttcap%
\pgfsetroundjoin%
\pgfsetlinewidth{1.505625pt}%
\definecolor{currentstroke}{rgb}{1.000000,0.498039,0.054902}%
\pgfsetstrokecolor{currentstroke}%
\pgfsetdash{}{0pt}%
\pgfpathmoveto{\pgfqpoint{10.688913in}{1.388153in}}%
\pgfpathlineto{\pgfqpoint{10.688913in}{1.460994in}}%
\pgfusepath{stroke}%
\end{pgfscope}%
\begin{pgfscope}%
\pgfpathrectangle{\pgfqpoint{1.286132in}{0.839159in}}{\pgfqpoint{12.053712in}{5.967710in}}%
\pgfusepath{clip}%
\pgfsetbuttcap%
\pgfsetroundjoin%
\pgfsetlinewidth{1.505625pt}%
\definecolor{currentstroke}{rgb}{1.000000,0.498039,0.054902}%
\pgfsetstrokecolor{currentstroke}%
\pgfsetdash{}{0pt}%
\pgfpathmoveto{\pgfqpoint{10.799599in}{1.420129in}}%
\pgfpathlineto{\pgfqpoint{10.799599in}{1.468596in}}%
\pgfusepath{stroke}%
\end{pgfscope}%
\begin{pgfscope}%
\pgfpathrectangle{\pgfqpoint{1.286132in}{0.839159in}}{\pgfqpoint{12.053712in}{5.967710in}}%
\pgfusepath{clip}%
\pgfsetbuttcap%
\pgfsetroundjoin%
\pgfsetlinewidth{1.505625pt}%
\definecolor{currentstroke}{rgb}{1.000000,0.498039,0.054902}%
\pgfsetstrokecolor{currentstroke}%
\pgfsetdash{}{0pt}%
\pgfpathmoveto{\pgfqpoint{10.910285in}{1.441476in}}%
\pgfpathlineto{\pgfqpoint{10.910285in}{1.547635in}}%
\pgfusepath{stroke}%
\end{pgfscope}%
\begin{pgfscope}%
\pgfpathrectangle{\pgfqpoint{1.286132in}{0.839159in}}{\pgfqpoint{12.053712in}{5.967710in}}%
\pgfusepath{clip}%
\pgfsetbuttcap%
\pgfsetroundjoin%
\pgfsetlinewidth{1.505625pt}%
\definecolor{currentstroke}{rgb}{1.000000,0.498039,0.054902}%
\pgfsetstrokecolor{currentstroke}%
\pgfsetdash{}{0pt}%
\pgfpathmoveto{\pgfqpoint{11.020971in}{1.452180in}}%
\pgfpathlineto{\pgfqpoint{11.020971in}{1.646140in}}%
\pgfusepath{stroke}%
\end{pgfscope}%
\begin{pgfscope}%
\pgfpathrectangle{\pgfqpoint{1.286132in}{0.839159in}}{\pgfqpoint{12.053712in}{5.967710in}}%
\pgfusepath{clip}%
\pgfsetbuttcap%
\pgfsetroundjoin%
\pgfsetlinewidth{1.505625pt}%
\definecolor{currentstroke}{rgb}{1.000000,0.498039,0.054902}%
\pgfsetstrokecolor{currentstroke}%
\pgfsetdash{}{0pt}%
\pgfpathmoveto{\pgfqpoint{11.131657in}{1.406737in}}%
\pgfpathlineto{\pgfqpoint{11.131657in}{1.494801in}}%
\pgfusepath{stroke}%
\end{pgfscope}%
\begin{pgfscope}%
\pgfpathrectangle{\pgfqpoint{1.286132in}{0.839159in}}{\pgfqpoint{12.053712in}{5.967710in}}%
\pgfusepath{clip}%
\pgfsetbuttcap%
\pgfsetroundjoin%
\pgfsetlinewidth{1.505625pt}%
\definecolor{currentstroke}{rgb}{1.000000,0.498039,0.054902}%
\pgfsetstrokecolor{currentstroke}%
\pgfsetdash{}{0pt}%
\pgfpathmoveto{\pgfqpoint{11.242343in}{1.420951in}}%
\pgfpathlineto{\pgfqpoint{11.242343in}{1.504554in}}%
\pgfusepath{stroke}%
\end{pgfscope}%
\begin{pgfscope}%
\pgfpathrectangle{\pgfqpoint{1.286132in}{0.839159in}}{\pgfqpoint{12.053712in}{5.967710in}}%
\pgfusepath{clip}%
\pgfsetbuttcap%
\pgfsetroundjoin%
\pgfsetlinewidth{1.505625pt}%
\definecolor{currentstroke}{rgb}{1.000000,0.498039,0.054902}%
\pgfsetstrokecolor{currentstroke}%
\pgfsetdash{}{0pt}%
\pgfpathmoveto{\pgfqpoint{11.353029in}{1.511536in}}%
\pgfpathlineto{\pgfqpoint{11.353029in}{1.621154in}}%
\pgfusepath{stroke}%
\end{pgfscope}%
\begin{pgfscope}%
\pgfpathrectangle{\pgfqpoint{1.286132in}{0.839159in}}{\pgfqpoint{12.053712in}{5.967710in}}%
\pgfusepath{clip}%
\pgfsetbuttcap%
\pgfsetroundjoin%
\pgfsetlinewidth{1.505625pt}%
\definecolor{currentstroke}{rgb}{1.000000,0.498039,0.054902}%
\pgfsetstrokecolor{currentstroke}%
\pgfsetdash{}{0pt}%
\pgfpathmoveto{\pgfqpoint{11.463715in}{1.493052in}}%
\pgfpathlineto{\pgfqpoint{11.463715in}{1.636443in}}%
\pgfusepath{stroke}%
\end{pgfscope}%
\begin{pgfscope}%
\pgfpathrectangle{\pgfqpoint{1.286132in}{0.839159in}}{\pgfqpoint{12.053712in}{5.967710in}}%
\pgfusepath{clip}%
\pgfsetbuttcap%
\pgfsetroundjoin%
\pgfsetlinewidth{1.505625pt}%
\definecolor{currentstroke}{rgb}{1.000000,0.498039,0.054902}%
\pgfsetstrokecolor{currentstroke}%
\pgfsetdash{}{0pt}%
\pgfpathmoveto{\pgfqpoint{11.574401in}{1.483518in}}%
\pgfpathlineto{\pgfqpoint{11.574401in}{1.490474in}}%
\pgfusepath{stroke}%
\end{pgfscope}%
\begin{pgfscope}%
\pgfpathrectangle{\pgfqpoint{1.286132in}{0.839159in}}{\pgfqpoint{12.053712in}{5.967710in}}%
\pgfusepath{clip}%
\pgfsetbuttcap%
\pgfsetroundjoin%
\pgfsetlinewidth{1.505625pt}%
\definecolor{currentstroke}{rgb}{1.000000,0.498039,0.054902}%
\pgfsetstrokecolor{currentstroke}%
\pgfsetdash{}{0pt}%
\pgfpathmoveto{\pgfqpoint{11.685087in}{1.495731in}}%
\pgfpathlineto{\pgfqpoint{11.685087in}{1.564877in}}%
\pgfusepath{stroke}%
\end{pgfscope}%
\begin{pgfscope}%
\pgfpathrectangle{\pgfqpoint{1.286132in}{0.839159in}}{\pgfqpoint{12.053712in}{5.967710in}}%
\pgfusepath{clip}%
\pgfsetbuttcap%
\pgfsetroundjoin%
\pgfsetlinewidth{1.505625pt}%
\definecolor{currentstroke}{rgb}{1.000000,0.498039,0.054902}%
\pgfsetstrokecolor{currentstroke}%
\pgfsetdash{}{0pt}%
\pgfpathmoveto{\pgfqpoint{11.795773in}{1.483428in}}%
\pgfpathlineto{\pgfqpoint{11.795773in}{1.670993in}}%
\pgfusepath{stroke}%
\end{pgfscope}%
\begin{pgfscope}%
\pgfpathrectangle{\pgfqpoint{1.286132in}{0.839159in}}{\pgfqpoint{12.053712in}{5.967710in}}%
\pgfusepath{clip}%
\pgfsetbuttcap%
\pgfsetroundjoin%
\pgfsetlinewidth{1.505625pt}%
\definecolor{currentstroke}{rgb}{1.000000,0.498039,0.054902}%
\pgfsetstrokecolor{currentstroke}%
\pgfsetdash{}{0pt}%
\pgfpathmoveto{\pgfqpoint{11.906459in}{1.520291in}}%
\pgfpathlineto{\pgfqpoint{11.906459in}{1.632325in}}%
\pgfusepath{stroke}%
\end{pgfscope}%
\begin{pgfscope}%
\pgfpathrectangle{\pgfqpoint{1.286132in}{0.839159in}}{\pgfqpoint{12.053712in}{5.967710in}}%
\pgfusepath{clip}%
\pgfsetbuttcap%
\pgfsetroundjoin%
\pgfsetlinewidth{1.505625pt}%
\definecolor{currentstroke}{rgb}{1.000000,0.498039,0.054902}%
\pgfsetstrokecolor{currentstroke}%
\pgfsetdash{}{0pt}%
\pgfpathmoveto{\pgfqpoint{12.017145in}{1.513808in}}%
\pgfpathlineto{\pgfqpoint{12.017145in}{1.663277in}}%
\pgfusepath{stroke}%
\end{pgfscope}%
\begin{pgfscope}%
\pgfpathrectangle{\pgfqpoint{1.286132in}{0.839159in}}{\pgfqpoint{12.053712in}{5.967710in}}%
\pgfusepath{clip}%
\pgfsetbuttcap%
\pgfsetroundjoin%
\pgfsetlinewidth{1.505625pt}%
\definecolor{currentstroke}{rgb}{1.000000,0.498039,0.054902}%
\pgfsetstrokecolor{currentstroke}%
\pgfsetdash{}{0pt}%
\pgfpathmoveto{\pgfqpoint{12.127831in}{1.454272in}}%
\pgfpathlineto{\pgfqpoint{12.127831in}{1.575190in}}%
\pgfusepath{stroke}%
\end{pgfscope}%
\begin{pgfscope}%
\pgfpathrectangle{\pgfqpoint{1.286132in}{0.839159in}}{\pgfqpoint{12.053712in}{5.967710in}}%
\pgfusepath{clip}%
\pgfsetbuttcap%
\pgfsetroundjoin%
\pgfsetlinewidth{1.505625pt}%
\definecolor{currentstroke}{rgb}{1.000000,0.498039,0.054902}%
\pgfsetstrokecolor{currentstroke}%
\pgfsetdash{}{0pt}%
\pgfpathmoveto{\pgfqpoint{12.238517in}{1.541341in}}%
\pgfpathlineto{\pgfqpoint{12.238517in}{1.618772in}}%
\pgfusepath{stroke}%
\end{pgfscope}%
\begin{pgfscope}%
\pgfpathrectangle{\pgfqpoint{1.286132in}{0.839159in}}{\pgfqpoint{12.053712in}{5.967710in}}%
\pgfusepath{clip}%
\pgfsetbuttcap%
\pgfsetroundjoin%
\pgfsetlinewidth{1.505625pt}%
\definecolor{currentstroke}{rgb}{1.000000,0.498039,0.054902}%
\pgfsetstrokecolor{currentstroke}%
\pgfsetdash{}{0pt}%
\pgfpathmoveto{\pgfqpoint{12.349203in}{1.557980in}}%
\pgfpathlineto{\pgfqpoint{12.349203in}{1.643054in}}%
\pgfusepath{stroke}%
\end{pgfscope}%
\begin{pgfscope}%
\pgfpathrectangle{\pgfqpoint{1.286132in}{0.839159in}}{\pgfqpoint{12.053712in}{5.967710in}}%
\pgfusepath{clip}%
\pgfsetbuttcap%
\pgfsetroundjoin%
\pgfsetlinewidth{1.505625pt}%
\definecolor{currentstroke}{rgb}{1.000000,0.498039,0.054902}%
\pgfsetstrokecolor{currentstroke}%
\pgfsetdash{}{0pt}%
\pgfpathmoveto{\pgfqpoint{12.459890in}{1.544684in}}%
\pgfpathlineto{\pgfqpoint{12.459890in}{1.680966in}}%
\pgfusepath{stroke}%
\end{pgfscope}%
\begin{pgfscope}%
\pgfpathrectangle{\pgfqpoint{1.286132in}{0.839159in}}{\pgfqpoint{12.053712in}{5.967710in}}%
\pgfusepath{clip}%
\pgfsetbuttcap%
\pgfsetroundjoin%
\pgfsetlinewidth{1.505625pt}%
\definecolor{currentstroke}{rgb}{1.000000,0.498039,0.054902}%
\pgfsetstrokecolor{currentstroke}%
\pgfsetdash{}{0pt}%
\pgfpathmoveto{\pgfqpoint{12.570576in}{1.545672in}}%
\pgfpathlineto{\pgfqpoint{12.570576in}{1.766677in}}%
\pgfusepath{stroke}%
\end{pgfscope}%
\begin{pgfscope}%
\pgfpathrectangle{\pgfqpoint{1.286132in}{0.839159in}}{\pgfqpoint{12.053712in}{5.967710in}}%
\pgfusepath{clip}%
\pgfsetbuttcap%
\pgfsetroundjoin%
\pgfsetlinewidth{1.505625pt}%
\definecolor{currentstroke}{rgb}{1.000000,0.498039,0.054902}%
\pgfsetstrokecolor{currentstroke}%
\pgfsetdash{}{0pt}%
\pgfpathmoveto{\pgfqpoint{12.681262in}{1.514005in}}%
\pgfpathlineto{\pgfqpoint{12.681262in}{1.852746in}}%
\pgfusepath{stroke}%
\end{pgfscope}%
\begin{pgfscope}%
\pgfpathrectangle{\pgfqpoint{1.286132in}{0.839159in}}{\pgfqpoint{12.053712in}{5.967710in}}%
\pgfusepath{clip}%
\pgfsetbuttcap%
\pgfsetroundjoin%
\pgfsetlinewidth{1.505625pt}%
\definecolor{currentstroke}{rgb}{1.000000,0.498039,0.054902}%
\pgfsetstrokecolor{currentstroke}%
\pgfsetdash{}{0pt}%
\pgfpathmoveto{\pgfqpoint{12.791948in}{1.584040in}}%
\pgfpathlineto{\pgfqpoint{12.791948in}{1.868242in}}%
\pgfusepath{stroke}%
\end{pgfscope}%
\begin{pgfscope}%
\pgfpathrectangle{\pgfqpoint{1.286132in}{0.839159in}}{\pgfqpoint{12.053712in}{5.967710in}}%
\pgfusepath{clip}%
\pgfsetbuttcap%
\pgfsetroundjoin%
\pgfsetlinewidth{1.505625pt}%
\definecolor{currentstroke}{rgb}{0.172549,0.627451,0.172549}%
\pgfsetstrokecolor{currentstroke}%
\pgfsetdash{}{0pt}%
\pgfpathmoveto{\pgfqpoint{1.834028in}{1.119234in}}%
\pgfpathlineto{\pgfqpoint{1.834028in}{1.119246in}}%
\pgfusepath{stroke}%
\end{pgfscope}%
\begin{pgfscope}%
\pgfpathrectangle{\pgfqpoint{1.286132in}{0.839159in}}{\pgfqpoint{12.053712in}{5.967710in}}%
\pgfusepath{clip}%
\pgfsetbuttcap%
\pgfsetroundjoin%
\pgfsetlinewidth{1.505625pt}%
\definecolor{currentstroke}{rgb}{0.172549,0.627451,0.172549}%
\pgfsetstrokecolor{currentstroke}%
\pgfsetdash{}{0pt}%
\pgfpathmoveto{\pgfqpoint{1.944714in}{1.119340in}}%
\pgfpathlineto{\pgfqpoint{1.944714in}{1.119342in}}%
\pgfusepath{stroke}%
\end{pgfscope}%
\begin{pgfscope}%
\pgfpathrectangle{\pgfqpoint{1.286132in}{0.839159in}}{\pgfqpoint{12.053712in}{5.967710in}}%
\pgfusepath{clip}%
\pgfsetbuttcap%
\pgfsetroundjoin%
\pgfsetlinewidth{1.505625pt}%
\definecolor{currentstroke}{rgb}{0.172549,0.627451,0.172549}%
\pgfsetstrokecolor{currentstroke}%
\pgfsetdash{}{0pt}%
\pgfpathmoveto{\pgfqpoint{2.055400in}{1.119397in}}%
\pgfpathlineto{\pgfqpoint{2.055400in}{1.119404in}}%
\pgfusepath{stroke}%
\end{pgfscope}%
\begin{pgfscope}%
\pgfpathrectangle{\pgfqpoint{1.286132in}{0.839159in}}{\pgfqpoint{12.053712in}{5.967710in}}%
\pgfusepath{clip}%
\pgfsetbuttcap%
\pgfsetroundjoin%
\pgfsetlinewidth{1.505625pt}%
\definecolor{currentstroke}{rgb}{0.172549,0.627451,0.172549}%
\pgfsetstrokecolor{currentstroke}%
\pgfsetdash{}{0pt}%
\pgfpathmoveto{\pgfqpoint{2.166086in}{1.119446in}}%
\pgfpathlineto{\pgfqpoint{2.166086in}{1.119451in}}%
\pgfusepath{stroke}%
\end{pgfscope}%
\begin{pgfscope}%
\pgfpathrectangle{\pgfqpoint{1.286132in}{0.839159in}}{\pgfqpoint{12.053712in}{5.967710in}}%
\pgfusepath{clip}%
\pgfsetbuttcap%
\pgfsetroundjoin%
\pgfsetlinewidth{1.505625pt}%
\definecolor{currentstroke}{rgb}{0.172549,0.627451,0.172549}%
\pgfsetstrokecolor{currentstroke}%
\pgfsetdash{}{0pt}%
\pgfpathmoveto{\pgfqpoint{2.276772in}{1.119512in}}%
\pgfpathlineto{\pgfqpoint{2.276772in}{1.119521in}}%
\pgfusepath{stroke}%
\end{pgfscope}%
\begin{pgfscope}%
\pgfpathrectangle{\pgfqpoint{1.286132in}{0.839159in}}{\pgfqpoint{12.053712in}{5.967710in}}%
\pgfusepath{clip}%
\pgfsetbuttcap%
\pgfsetroundjoin%
\pgfsetlinewidth{1.505625pt}%
\definecolor{currentstroke}{rgb}{0.172549,0.627451,0.172549}%
\pgfsetstrokecolor{currentstroke}%
\pgfsetdash{}{0pt}%
\pgfpathmoveto{\pgfqpoint{2.387458in}{1.119578in}}%
\pgfpathlineto{\pgfqpoint{2.387458in}{1.119589in}}%
\pgfusepath{stroke}%
\end{pgfscope}%
\begin{pgfscope}%
\pgfpathrectangle{\pgfqpoint{1.286132in}{0.839159in}}{\pgfqpoint{12.053712in}{5.967710in}}%
\pgfusepath{clip}%
\pgfsetbuttcap%
\pgfsetroundjoin%
\pgfsetlinewidth{1.505625pt}%
\definecolor{currentstroke}{rgb}{0.172549,0.627451,0.172549}%
\pgfsetstrokecolor{currentstroke}%
\pgfsetdash{}{0pt}%
\pgfpathmoveto{\pgfqpoint{2.498144in}{1.119647in}}%
\pgfpathlineto{\pgfqpoint{2.498144in}{1.119665in}}%
\pgfusepath{stroke}%
\end{pgfscope}%
\begin{pgfscope}%
\pgfpathrectangle{\pgfqpoint{1.286132in}{0.839159in}}{\pgfqpoint{12.053712in}{5.967710in}}%
\pgfusepath{clip}%
\pgfsetbuttcap%
\pgfsetroundjoin%
\pgfsetlinewidth{1.505625pt}%
\definecolor{currentstroke}{rgb}{0.172549,0.627451,0.172549}%
\pgfsetstrokecolor{currentstroke}%
\pgfsetdash{}{0pt}%
\pgfpathmoveto{\pgfqpoint{2.608830in}{1.119702in}}%
\pgfpathlineto{\pgfqpoint{2.608830in}{1.119730in}}%
\pgfusepath{stroke}%
\end{pgfscope}%
\begin{pgfscope}%
\pgfpathrectangle{\pgfqpoint{1.286132in}{0.839159in}}{\pgfqpoint{12.053712in}{5.967710in}}%
\pgfusepath{clip}%
\pgfsetbuttcap%
\pgfsetroundjoin%
\pgfsetlinewidth{1.505625pt}%
\definecolor{currentstroke}{rgb}{0.172549,0.627451,0.172549}%
\pgfsetstrokecolor{currentstroke}%
\pgfsetdash{}{0pt}%
\pgfpathmoveto{\pgfqpoint{2.719516in}{1.119786in}}%
\pgfpathlineto{\pgfqpoint{2.719516in}{1.119816in}}%
\pgfusepath{stroke}%
\end{pgfscope}%
\begin{pgfscope}%
\pgfpathrectangle{\pgfqpoint{1.286132in}{0.839159in}}{\pgfqpoint{12.053712in}{5.967710in}}%
\pgfusepath{clip}%
\pgfsetbuttcap%
\pgfsetroundjoin%
\pgfsetlinewidth{1.505625pt}%
\definecolor{currentstroke}{rgb}{0.172549,0.627451,0.172549}%
\pgfsetstrokecolor{currentstroke}%
\pgfsetdash{}{0pt}%
\pgfpathmoveto{\pgfqpoint{2.830202in}{1.119857in}}%
\pgfpathlineto{\pgfqpoint{2.830202in}{1.119914in}}%
\pgfusepath{stroke}%
\end{pgfscope}%
\begin{pgfscope}%
\pgfpathrectangle{\pgfqpoint{1.286132in}{0.839159in}}{\pgfqpoint{12.053712in}{5.967710in}}%
\pgfusepath{clip}%
\pgfsetbuttcap%
\pgfsetroundjoin%
\pgfsetlinewidth{1.505625pt}%
\definecolor{currentstroke}{rgb}{0.172549,0.627451,0.172549}%
\pgfsetstrokecolor{currentstroke}%
\pgfsetdash{}{0pt}%
\pgfpathmoveto{\pgfqpoint{2.940888in}{1.119985in}}%
\pgfpathlineto{\pgfqpoint{2.940888in}{1.120080in}}%
\pgfusepath{stroke}%
\end{pgfscope}%
\begin{pgfscope}%
\pgfpathrectangle{\pgfqpoint{1.286132in}{0.839159in}}{\pgfqpoint{12.053712in}{5.967710in}}%
\pgfusepath{clip}%
\pgfsetbuttcap%
\pgfsetroundjoin%
\pgfsetlinewidth{1.505625pt}%
\definecolor{currentstroke}{rgb}{0.172549,0.627451,0.172549}%
\pgfsetstrokecolor{currentstroke}%
\pgfsetdash{}{0pt}%
\pgfpathmoveto{\pgfqpoint{3.051574in}{1.120051in}}%
\pgfpathlineto{\pgfqpoint{3.051574in}{1.120110in}}%
\pgfusepath{stroke}%
\end{pgfscope}%
\begin{pgfscope}%
\pgfpathrectangle{\pgfqpoint{1.286132in}{0.839159in}}{\pgfqpoint{12.053712in}{5.967710in}}%
\pgfusepath{clip}%
\pgfsetbuttcap%
\pgfsetroundjoin%
\pgfsetlinewidth{1.505625pt}%
\definecolor{currentstroke}{rgb}{0.172549,0.627451,0.172549}%
\pgfsetstrokecolor{currentstroke}%
\pgfsetdash{}{0pt}%
\pgfpathmoveto{\pgfqpoint{3.162260in}{1.120134in}}%
\pgfpathlineto{\pgfqpoint{3.162260in}{1.120336in}}%
\pgfusepath{stroke}%
\end{pgfscope}%
\begin{pgfscope}%
\pgfpathrectangle{\pgfqpoint{1.286132in}{0.839159in}}{\pgfqpoint{12.053712in}{5.967710in}}%
\pgfusepath{clip}%
\pgfsetbuttcap%
\pgfsetroundjoin%
\pgfsetlinewidth{1.505625pt}%
\definecolor{currentstroke}{rgb}{0.172549,0.627451,0.172549}%
\pgfsetstrokecolor{currentstroke}%
\pgfsetdash{}{0pt}%
\pgfpathmoveto{\pgfqpoint{3.272946in}{1.120250in}}%
\pgfpathlineto{\pgfqpoint{3.272946in}{1.120393in}}%
\pgfusepath{stroke}%
\end{pgfscope}%
\begin{pgfscope}%
\pgfpathrectangle{\pgfqpoint{1.286132in}{0.839159in}}{\pgfqpoint{12.053712in}{5.967710in}}%
\pgfusepath{clip}%
\pgfsetbuttcap%
\pgfsetroundjoin%
\pgfsetlinewidth{1.505625pt}%
\definecolor{currentstroke}{rgb}{0.172549,0.627451,0.172549}%
\pgfsetstrokecolor{currentstroke}%
\pgfsetdash{}{0pt}%
\pgfpathmoveto{\pgfqpoint{3.383632in}{1.120334in}}%
\pgfpathlineto{\pgfqpoint{3.383632in}{1.120525in}}%
\pgfusepath{stroke}%
\end{pgfscope}%
\begin{pgfscope}%
\pgfpathrectangle{\pgfqpoint{1.286132in}{0.839159in}}{\pgfqpoint{12.053712in}{5.967710in}}%
\pgfusepath{clip}%
\pgfsetbuttcap%
\pgfsetroundjoin%
\pgfsetlinewidth{1.505625pt}%
\definecolor{currentstroke}{rgb}{0.172549,0.627451,0.172549}%
\pgfsetstrokecolor{currentstroke}%
\pgfsetdash{}{0pt}%
\pgfpathmoveto{\pgfqpoint{3.494319in}{1.120432in}}%
\pgfpathlineto{\pgfqpoint{3.494319in}{1.120667in}}%
\pgfusepath{stroke}%
\end{pgfscope}%
\begin{pgfscope}%
\pgfpathrectangle{\pgfqpoint{1.286132in}{0.839159in}}{\pgfqpoint{12.053712in}{5.967710in}}%
\pgfusepath{clip}%
\pgfsetbuttcap%
\pgfsetroundjoin%
\pgfsetlinewidth{1.505625pt}%
\definecolor{currentstroke}{rgb}{0.172549,0.627451,0.172549}%
\pgfsetstrokecolor{currentstroke}%
\pgfsetdash{}{0pt}%
\pgfpathmoveto{\pgfqpoint{3.605005in}{1.120640in}}%
\pgfpathlineto{\pgfqpoint{3.605005in}{1.120988in}}%
\pgfusepath{stroke}%
\end{pgfscope}%
\begin{pgfscope}%
\pgfpathrectangle{\pgfqpoint{1.286132in}{0.839159in}}{\pgfqpoint{12.053712in}{5.967710in}}%
\pgfusepath{clip}%
\pgfsetbuttcap%
\pgfsetroundjoin%
\pgfsetlinewidth{1.505625pt}%
\definecolor{currentstroke}{rgb}{0.172549,0.627451,0.172549}%
\pgfsetstrokecolor{currentstroke}%
\pgfsetdash{}{0pt}%
\pgfpathmoveto{\pgfqpoint{3.715691in}{1.120648in}}%
\pgfpathlineto{\pgfqpoint{3.715691in}{1.121053in}}%
\pgfusepath{stroke}%
\end{pgfscope}%
\begin{pgfscope}%
\pgfpathrectangle{\pgfqpoint{1.286132in}{0.839159in}}{\pgfqpoint{12.053712in}{5.967710in}}%
\pgfusepath{clip}%
\pgfsetbuttcap%
\pgfsetroundjoin%
\pgfsetlinewidth{1.505625pt}%
\definecolor{currentstroke}{rgb}{0.172549,0.627451,0.172549}%
\pgfsetstrokecolor{currentstroke}%
\pgfsetdash{}{0pt}%
\pgfpathmoveto{\pgfqpoint{3.826377in}{1.120840in}}%
\pgfpathlineto{\pgfqpoint{3.826377in}{1.121175in}}%
\pgfusepath{stroke}%
\end{pgfscope}%
\begin{pgfscope}%
\pgfpathrectangle{\pgfqpoint{1.286132in}{0.839159in}}{\pgfqpoint{12.053712in}{5.967710in}}%
\pgfusepath{clip}%
\pgfsetbuttcap%
\pgfsetroundjoin%
\pgfsetlinewidth{1.505625pt}%
\definecolor{currentstroke}{rgb}{0.172549,0.627451,0.172549}%
\pgfsetstrokecolor{currentstroke}%
\pgfsetdash{}{0pt}%
\pgfpathmoveto{\pgfqpoint{3.937063in}{1.120917in}}%
\pgfpathlineto{\pgfqpoint{3.937063in}{1.121264in}}%
\pgfusepath{stroke}%
\end{pgfscope}%
\begin{pgfscope}%
\pgfpathrectangle{\pgfqpoint{1.286132in}{0.839159in}}{\pgfqpoint{12.053712in}{5.967710in}}%
\pgfusepath{clip}%
\pgfsetbuttcap%
\pgfsetroundjoin%
\pgfsetlinewidth{1.505625pt}%
\definecolor{currentstroke}{rgb}{0.172549,0.627451,0.172549}%
\pgfsetstrokecolor{currentstroke}%
\pgfsetdash{}{0pt}%
\pgfpathmoveto{\pgfqpoint{4.047749in}{1.121095in}}%
\pgfpathlineto{\pgfqpoint{4.047749in}{1.121298in}}%
\pgfusepath{stroke}%
\end{pgfscope}%
\begin{pgfscope}%
\pgfpathrectangle{\pgfqpoint{1.286132in}{0.839159in}}{\pgfqpoint{12.053712in}{5.967710in}}%
\pgfusepath{clip}%
\pgfsetbuttcap%
\pgfsetroundjoin%
\pgfsetlinewidth{1.505625pt}%
\definecolor{currentstroke}{rgb}{0.172549,0.627451,0.172549}%
\pgfsetstrokecolor{currentstroke}%
\pgfsetdash{}{0pt}%
\pgfpathmoveto{\pgfqpoint{4.158435in}{1.121310in}}%
\pgfpathlineto{\pgfqpoint{4.158435in}{1.121671in}}%
\pgfusepath{stroke}%
\end{pgfscope}%
\begin{pgfscope}%
\pgfpathrectangle{\pgfqpoint{1.286132in}{0.839159in}}{\pgfqpoint{12.053712in}{5.967710in}}%
\pgfusepath{clip}%
\pgfsetbuttcap%
\pgfsetroundjoin%
\pgfsetlinewidth{1.505625pt}%
\definecolor{currentstroke}{rgb}{0.172549,0.627451,0.172549}%
\pgfsetstrokecolor{currentstroke}%
\pgfsetdash{}{0pt}%
\pgfpathmoveto{\pgfqpoint{4.269121in}{1.121301in}}%
\pgfpathlineto{\pgfqpoint{4.269121in}{1.121660in}}%
\pgfusepath{stroke}%
\end{pgfscope}%
\begin{pgfscope}%
\pgfpathrectangle{\pgfqpoint{1.286132in}{0.839159in}}{\pgfqpoint{12.053712in}{5.967710in}}%
\pgfusepath{clip}%
\pgfsetbuttcap%
\pgfsetroundjoin%
\pgfsetlinewidth{1.505625pt}%
\definecolor{currentstroke}{rgb}{0.172549,0.627451,0.172549}%
\pgfsetstrokecolor{currentstroke}%
\pgfsetdash{}{0pt}%
\pgfpathmoveto{\pgfqpoint{4.379807in}{1.121547in}}%
\pgfpathlineto{\pgfqpoint{4.379807in}{1.122067in}}%
\pgfusepath{stroke}%
\end{pgfscope}%
\begin{pgfscope}%
\pgfpathrectangle{\pgfqpoint{1.286132in}{0.839159in}}{\pgfqpoint{12.053712in}{5.967710in}}%
\pgfusepath{clip}%
\pgfsetbuttcap%
\pgfsetroundjoin%
\pgfsetlinewidth{1.505625pt}%
\definecolor{currentstroke}{rgb}{0.172549,0.627451,0.172549}%
\pgfsetstrokecolor{currentstroke}%
\pgfsetdash{}{0pt}%
\pgfpathmoveto{\pgfqpoint{4.490493in}{1.121671in}}%
\pgfpathlineto{\pgfqpoint{4.490493in}{1.121908in}}%
\pgfusepath{stroke}%
\end{pgfscope}%
\begin{pgfscope}%
\pgfpathrectangle{\pgfqpoint{1.286132in}{0.839159in}}{\pgfqpoint{12.053712in}{5.967710in}}%
\pgfusepath{clip}%
\pgfsetbuttcap%
\pgfsetroundjoin%
\pgfsetlinewidth{1.505625pt}%
\definecolor{currentstroke}{rgb}{0.172549,0.627451,0.172549}%
\pgfsetstrokecolor{currentstroke}%
\pgfsetdash{}{0pt}%
\pgfpathmoveto{\pgfqpoint{4.601179in}{1.121853in}}%
\pgfpathlineto{\pgfqpoint{4.601179in}{1.122699in}}%
\pgfusepath{stroke}%
\end{pgfscope}%
\begin{pgfscope}%
\pgfpathrectangle{\pgfqpoint{1.286132in}{0.839159in}}{\pgfqpoint{12.053712in}{5.967710in}}%
\pgfusepath{clip}%
\pgfsetbuttcap%
\pgfsetroundjoin%
\pgfsetlinewidth{1.505625pt}%
\definecolor{currentstroke}{rgb}{0.172549,0.627451,0.172549}%
\pgfsetstrokecolor{currentstroke}%
\pgfsetdash{}{0pt}%
\pgfpathmoveto{\pgfqpoint{4.711865in}{1.121947in}}%
\pgfpathlineto{\pgfqpoint{4.711865in}{1.122716in}}%
\pgfusepath{stroke}%
\end{pgfscope}%
\begin{pgfscope}%
\pgfpathrectangle{\pgfqpoint{1.286132in}{0.839159in}}{\pgfqpoint{12.053712in}{5.967710in}}%
\pgfusepath{clip}%
\pgfsetbuttcap%
\pgfsetroundjoin%
\pgfsetlinewidth{1.505625pt}%
\definecolor{currentstroke}{rgb}{0.172549,0.627451,0.172549}%
\pgfsetstrokecolor{currentstroke}%
\pgfsetdash{}{0pt}%
\pgfpathmoveto{\pgfqpoint{4.822551in}{1.122247in}}%
\pgfpathlineto{\pgfqpoint{4.822551in}{1.122482in}}%
\pgfusepath{stroke}%
\end{pgfscope}%
\begin{pgfscope}%
\pgfpathrectangle{\pgfqpoint{1.286132in}{0.839159in}}{\pgfqpoint{12.053712in}{5.967710in}}%
\pgfusepath{clip}%
\pgfsetbuttcap%
\pgfsetroundjoin%
\pgfsetlinewidth{1.505625pt}%
\definecolor{currentstroke}{rgb}{0.172549,0.627451,0.172549}%
\pgfsetstrokecolor{currentstroke}%
\pgfsetdash{}{0pt}%
\pgfpathmoveto{\pgfqpoint{4.933237in}{1.122125in}}%
\pgfpathlineto{\pgfqpoint{4.933237in}{1.123469in}}%
\pgfusepath{stroke}%
\end{pgfscope}%
\begin{pgfscope}%
\pgfpathrectangle{\pgfqpoint{1.286132in}{0.839159in}}{\pgfqpoint{12.053712in}{5.967710in}}%
\pgfusepath{clip}%
\pgfsetbuttcap%
\pgfsetroundjoin%
\pgfsetlinewidth{1.505625pt}%
\definecolor{currentstroke}{rgb}{0.172549,0.627451,0.172549}%
\pgfsetstrokecolor{currentstroke}%
\pgfsetdash{}{0pt}%
\pgfpathmoveto{\pgfqpoint{5.043923in}{1.122545in}}%
\pgfpathlineto{\pgfqpoint{5.043923in}{1.122595in}}%
\pgfusepath{stroke}%
\end{pgfscope}%
\begin{pgfscope}%
\pgfpathrectangle{\pgfqpoint{1.286132in}{0.839159in}}{\pgfqpoint{12.053712in}{5.967710in}}%
\pgfusepath{clip}%
\pgfsetbuttcap%
\pgfsetroundjoin%
\pgfsetlinewidth{1.505625pt}%
\definecolor{currentstroke}{rgb}{0.172549,0.627451,0.172549}%
\pgfsetstrokecolor{currentstroke}%
\pgfsetdash{}{0pt}%
\pgfpathmoveto{\pgfqpoint{5.154609in}{1.122837in}}%
\pgfpathlineto{\pgfqpoint{5.154609in}{1.123059in}}%
\pgfusepath{stroke}%
\end{pgfscope}%
\begin{pgfscope}%
\pgfpathrectangle{\pgfqpoint{1.286132in}{0.839159in}}{\pgfqpoint{12.053712in}{5.967710in}}%
\pgfusepath{clip}%
\pgfsetbuttcap%
\pgfsetroundjoin%
\pgfsetlinewidth{1.505625pt}%
\definecolor{currentstroke}{rgb}{0.172549,0.627451,0.172549}%
\pgfsetstrokecolor{currentstroke}%
\pgfsetdash{}{0pt}%
\pgfpathmoveto{\pgfqpoint{5.265296in}{1.122875in}}%
\pgfpathlineto{\pgfqpoint{5.265296in}{1.123005in}}%
\pgfusepath{stroke}%
\end{pgfscope}%
\begin{pgfscope}%
\pgfpathrectangle{\pgfqpoint{1.286132in}{0.839159in}}{\pgfqpoint{12.053712in}{5.967710in}}%
\pgfusepath{clip}%
\pgfsetbuttcap%
\pgfsetroundjoin%
\pgfsetlinewidth{1.505625pt}%
\definecolor{currentstroke}{rgb}{0.172549,0.627451,0.172549}%
\pgfsetstrokecolor{currentstroke}%
\pgfsetdash{}{0pt}%
\pgfpathmoveto{\pgfqpoint{5.375982in}{1.116357in}}%
\pgfpathlineto{\pgfqpoint{5.375982in}{1.144249in}}%
\pgfusepath{stroke}%
\end{pgfscope}%
\begin{pgfscope}%
\pgfpathrectangle{\pgfqpoint{1.286132in}{0.839159in}}{\pgfqpoint{12.053712in}{5.967710in}}%
\pgfusepath{clip}%
\pgfsetbuttcap%
\pgfsetroundjoin%
\pgfsetlinewidth{1.505625pt}%
\definecolor{currentstroke}{rgb}{0.172549,0.627451,0.172549}%
\pgfsetstrokecolor{currentstroke}%
\pgfsetdash{}{0pt}%
\pgfpathmoveto{\pgfqpoint{5.486668in}{1.123305in}}%
\pgfpathlineto{\pgfqpoint{5.486668in}{1.124071in}}%
\pgfusepath{stroke}%
\end{pgfscope}%
\begin{pgfscope}%
\pgfpathrectangle{\pgfqpoint{1.286132in}{0.839159in}}{\pgfqpoint{12.053712in}{5.967710in}}%
\pgfusepath{clip}%
\pgfsetbuttcap%
\pgfsetroundjoin%
\pgfsetlinewidth{1.505625pt}%
\definecolor{currentstroke}{rgb}{0.172549,0.627451,0.172549}%
\pgfsetstrokecolor{currentstroke}%
\pgfsetdash{}{0pt}%
\pgfpathmoveto{\pgfqpoint{5.597354in}{1.121096in}}%
\pgfpathlineto{\pgfqpoint{5.597354in}{1.131469in}}%
\pgfusepath{stroke}%
\end{pgfscope}%
\begin{pgfscope}%
\pgfpathrectangle{\pgfqpoint{1.286132in}{0.839159in}}{\pgfqpoint{12.053712in}{5.967710in}}%
\pgfusepath{clip}%
\pgfsetbuttcap%
\pgfsetroundjoin%
\pgfsetlinewidth{1.505625pt}%
\definecolor{currentstroke}{rgb}{0.172549,0.627451,0.172549}%
\pgfsetstrokecolor{currentstroke}%
\pgfsetdash{}{0pt}%
\pgfpathmoveto{\pgfqpoint{5.708040in}{1.121357in}}%
\pgfpathlineto{\pgfqpoint{5.708040in}{1.131240in}}%
\pgfusepath{stroke}%
\end{pgfscope}%
\begin{pgfscope}%
\pgfpathrectangle{\pgfqpoint{1.286132in}{0.839159in}}{\pgfqpoint{12.053712in}{5.967710in}}%
\pgfusepath{clip}%
\pgfsetbuttcap%
\pgfsetroundjoin%
\pgfsetlinewidth{1.505625pt}%
\definecolor{currentstroke}{rgb}{0.172549,0.627451,0.172549}%
\pgfsetstrokecolor{currentstroke}%
\pgfsetdash{}{0pt}%
\pgfpathmoveto{\pgfqpoint{5.818726in}{1.123880in}}%
\pgfpathlineto{\pgfqpoint{5.818726in}{1.124409in}}%
\pgfusepath{stroke}%
\end{pgfscope}%
\begin{pgfscope}%
\pgfpathrectangle{\pgfqpoint{1.286132in}{0.839159in}}{\pgfqpoint{12.053712in}{5.967710in}}%
\pgfusepath{clip}%
\pgfsetbuttcap%
\pgfsetroundjoin%
\pgfsetlinewidth{1.505625pt}%
\definecolor{currentstroke}{rgb}{0.172549,0.627451,0.172549}%
\pgfsetstrokecolor{currentstroke}%
\pgfsetdash{}{0pt}%
\pgfpathmoveto{\pgfqpoint{5.929412in}{1.124074in}}%
\pgfpathlineto{\pgfqpoint{5.929412in}{1.124707in}}%
\pgfusepath{stroke}%
\end{pgfscope}%
\begin{pgfscope}%
\pgfpathrectangle{\pgfqpoint{1.286132in}{0.839159in}}{\pgfqpoint{12.053712in}{5.967710in}}%
\pgfusepath{clip}%
\pgfsetbuttcap%
\pgfsetroundjoin%
\pgfsetlinewidth{1.505625pt}%
\definecolor{currentstroke}{rgb}{0.172549,0.627451,0.172549}%
\pgfsetstrokecolor{currentstroke}%
\pgfsetdash{}{0pt}%
\pgfpathmoveto{\pgfqpoint{6.040098in}{1.124424in}}%
\pgfpathlineto{\pgfqpoint{6.040098in}{1.125506in}}%
\pgfusepath{stroke}%
\end{pgfscope}%
\begin{pgfscope}%
\pgfpathrectangle{\pgfqpoint{1.286132in}{0.839159in}}{\pgfqpoint{12.053712in}{5.967710in}}%
\pgfusepath{clip}%
\pgfsetbuttcap%
\pgfsetroundjoin%
\pgfsetlinewidth{1.505625pt}%
\definecolor{currentstroke}{rgb}{0.172549,0.627451,0.172549}%
\pgfsetstrokecolor{currentstroke}%
\pgfsetdash{}{0pt}%
\pgfpathmoveto{\pgfqpoint{6.150784in}{1.124280in}}%
\pgfpathlineto{\pgfqpoint{6.150784in}{1.125742in}}%
\pgfusepath{stroke}%
\end{pgfscope}%
\begin{pgfscope}%
\pgfpathrectangle{\pgfqpoint{1.286132in}{0.839159in}}{\pgfqpoint{12.053712in}{5.967710in}}%
\pgfusepath{clip}%
\pgfsetbuttcap%
\pgfsetroundjoin%
\pgfsetlinewidth{1.505625pt}%
\definecolor{currentstroke}{rgb}{0.172549,0.627451,0.172549}%
\pgfsetstrokecolor{currentstroke}%
\pgfsetdash{}{0pt}%
\pgfpathmoveto{\pgfqpoint{6.261470in}{1.123893in}}%
\pgfpathlineto{\pgfqpoint{6.261470in}{1.127593in}}%
\pgfusepath{stroke}%
\end{pgfscope}%
\begin{pgfscope}%
\pgfpathrectangle{\pgfqpoint{1.286132in}{0.839159in}}{\pgfqpoint{12.053712in}{5.967710in}}%
\pgfusepath{clip}%
\pgfsetbuttcap%
\pgfsetroundjoin%
\pgfsetlinewidth{1.505625pt}%
\definecolor{currentstroke}{rgb}{0.172549,0.627451,0.172549}%
\pgfsetstrokecolor{currentstroke}%
\pgfsetdash{}{0pt}%
\pgfpathmoveto{\pgfqpoint{6.372156in}{1.124774in}}%
\pgfpathlineto{\pgfqpoint{6.372156in}{1.125028in}}%
\pgfusepath{stroke}%
\end{pgfscope}%
\begin{pgfscope}%
\pgfpathrectangle{\pgfqpoint{1.286132in}{0.839159in}}{\pgfqpoint{12.053712in}{5.967710in}}%
\pgfusepath{clip}%
\pgfsetbuttcap%
\pgfsetroundjoin%
\pgfsetlinewidth{1.505625pt}%
\definecolor{currentstroke}{rgb}{0.172549,0.627451,0.172549}%
\pgfsetstrokecolor{currentstroke}%
\pgfsetdash{}{0pt}%
\pgfpathmoveto{\pgfqpoint{6.482842in}{1.125144in}}%
\pgfpathlineto{\pgfqpoint{6.482842in}{1.125230in}}%
\pgfusepath{stroke}%
\end{pgfscope}%
\begin{pgfscope}%
\pgfpathrectangle{\pgfqpoint{1.286132in}{0.839159in}}{\pgfqpoint{12.053712in}{5.967710in}}%
\pgfusepath{clip}%
\pgfsetbuttcap%
\pgfsetroundjoin%
\pgfsetlinewidth{1.505625pt}%
\definecolor{currentstroke}{rgb}{0.172549,0.627451,0.172549}%
\pgfsetstrokecolor{currentstroke}%
\pgfsetdash{}{0pt}%
\pgfpathmoveto{\pgfqpoint{6.593528in}{1.125172in}}%
\pgfpathlineto{\pgfqpoint{6.593528in}{1.125993in}}%
\pgfusepath{stroke}%
\end{pgfscope}%
\begin{pgfscope}%
\pgfpathrectangle{\pgfqpoint{1.286132in}{0.839159in}}{\pgfqpoint{12.053712in}{5.967710in}}%
\pgfusepath{clip}%
\pgfsetbuttcap%
\pgfsetroundjoin%
\pgfsetlinewidth{1.505625pt}%
\definecolor{currentstroke}{rgb}{0.172549,0.627451,0.172549}%
\pgfsetstrokecolor{currentstroke}%
\pgfsetdash{}{0pt}%
\pgfpathmoveto{\pgfqpoint{6.704214in}{1.125404in}}%
\pgfpathlineto{\pgfqpoint{6.704214in}{1.125728in}}%
\pgfusepath{stroke}%
\end{pgfscope}%
\begin{pgfscope}%
\pgfpathrectangle{\pgfqpoint{1.286132in}{0.839159in}}{\pgfqpoint{12.053712in}{5.967710in}}%
\pgfusepath{clip}%
\pgfsetbuttcap%
\pgfsetroundjoin%
\pgfsetlinewidth{1.505625pt}%
\definecolor{currentstroke}{rgb}{0.172549,0.627451,0.172549}%
\pgfsetstrokecolor{currentstroke}%
\pgfsetdash{}{0pt}%
\pgfpathmoveto{\pgfqpoint{6.814900in}{1.125839in}}%
\pgfpathlineto{\pgfqpoint{6.814900in}{1.126336in}}%
\pgfusepath{stroke}%
\end{pgfscope}%
\begin{pgfscope}%
\pgfpathrectangle{\pgfqpoint{1.286132in}{0.839159in}}{\pgfqpoint{12.053712in}{5.967710in}}%
\pgfusepath{clip}%
\pgfsetbuttcap%
\pgfsetroundjoin%
\pgfsetlinewidth{1.505625pt}%
\definecolor{currentstroke}{rgb}{0.172549,0.627451,0.172549}%
\pgfsetstrokecolor{currentstroke}%
\pgfsetdash{}{0pt}%
\pgfpathmoveto{\pgfqpoint{6.925586in}{1.126034in}}%
\pgfpathlineto{\pgfqpoint{6.925586in}{1.126888in}}%
\pgfusepath{stroke}%
\end{pgfscope}%
\begin{pgfscope}%
\pgfpathrectangle{\pgfqpoint{1.286132in}{0.839159in}}{\pgfqpoint{12.053712in}{5.967710in}}%
\pgfusepath{clip}%
\pgfsetbuttcap%
\pgfsetroundjoin%
\pgfsetlinewidth{1.505625pt}%
\definecolor{currentstroke}{rgb}{0.172549,0.627451,0.172549}%
\pgfsetstrokecolor{currentstroke}%
\pgfsetdash{}{0pt}%
\pgfpathmoveto{\pgfqpoint{7.036272in}{1.126247in}}%
\pgfpathlineto{\pgfqpoint{7.036272in}{1.126928in}}%
\pgfusepath{stroke}%
\end{pgfscope}%
\begin{pgfscope}%
\pgfpathrectangle{\pgfqpoint{1.286132in}{0.839159in}}{\pgfqpoint{12.053712in}{5.967710in}}%
\pgfusepath{clip}%
\pgfsetbuttcap%
\pgfsetroundjoin%
\pgfsetlinewidth{1.505625pt}%
\definecolor{currentstroke}{rgb}{0.172549,0.627451,0.172549}%
\pgfsetstrokecolor{currentstroke}%
\pgfsetdash{}{0pt}%
\pgfpathmoveto{\pgfqpoint{7.146959in}{1.126481in}}%
\pgfpathlineto{\pgfqpoint{7.146959in}{1.127138in}}%
\pgfusepath{stroke}%
\end{pgfscope}%
\begin{pgfscope}%
\pgfpathrectangle{\pgfqpoint{1.286132in}{0.839159in}}{\pgfqpoint{12.053712in}{5.967710in}}%
\pgfusepath{clip}%
\pgfsetbuttcap%
\pgfsetroundjoin%
\pgfsetlinewidth{1.505625pt}%
\definecolor{currentstroke}{rgb}{0.172549,0.627451,0.172549}%
\pgfsetstrokecolor{currentstroke}%
\pgfsetdash{}{0pt}%
\pgfpathmoveto{\pgfqpoint{7.257645in}{1.127100in}}%
\pgfpathlineto{\pgfqpoint{7.257645in}{1.127672in}}%
\pgfusepath{stroke}%
\end{pgfscope}%
\begin{pgfscope}%
\pgfpathrectangle{\pgfqpoint{1.286132in}{0.839159in}}{\pgfqpoint{12.053712in}{5.967710in}}%
\pgfusepath{clip}%
\pgfsetbuttcap%
\pgfsetroundjoin%
\pgfsetlinewidth{1.505625pt}%
\definecolor{currentstroke}{rgb}{0.172549,0.627451,0.172549}%
\pgfsetstrokecolor{currentstroke}%
\pgfsetdash{}{0pt}%
\pgfpathmoveto{\pgfqpoint{7.368331in}{1.127374in}}%
\pgfpathlineto{\pgfqpoint{7.368331in}{1.127789in}}%
\pgfusepath{stroke}%
\end{pgfscope}%
\begin{pgfscope}%
\pgfpathrectangle{\pgfqpoint{1.286132in}{0.839159in}}{\pgfqpoint{12.053712in}{5.967710in}}%
\pgfusepath{clip}%
\pgfsetbuttcap%
\pgfsetroundjoin%
\pgfsetlinewidth{1.505625pt}%
\definecolor{currentstroke}{rgb}{0.172549,0.627451,0.172549}%
\pgfsetstrokecolor{currentstroke}%
\pgfsetdash{}{0pt}%
\pgfpathmoveto{\pgfqpoint{7.479017in}{1.127784in}}%
\pgfpathlineto{\pgfqpoint{7.479017in}{1.128158in}}%
\pgfusepath{stroke}%
\end{pgfscope}%
\begin{pgfscope}%
\pgfpathrectangle{\pgfqpoint{1.286132in}{0.839159in}}{\pgfqpoint{12.053712in}{5.967710in}}%
\pgfusepath{clip}%
\pgfsetbuttcap%
\pgfsetroundjoin%
\pgfsetlinewidth{1.505625pt}%
\definecolor{currentstroke}{rgb}{0.172549,0.627451,0.172549}%
\pgfsetstrokecolor{currentstroke}%
\pgfsetdash{}{0pt}%
\pgfpathmoveto{\pgfqpoint{7.589703in}{1.115104in}}%
\pgfpathlineto{\pgfqpoint{7.589703in}{1.170441in}}%
\pgfusepath{stroke}%
\end{pgfscope}%
\begin{pgfscope}%
\pgfpathrectangle{\pgfqpoint{1.286132in}{0.839159in}}{\pgfqpoint{12.053712in}{5.967710in}}%
\pgfusepath{clip}%
\pgfsetbuttcap%
\pgfsetroundjoin%
\pgfsetlinewidth{1.505625pt}%
\definecolor{currentstroke}{rgb}{0.172549,0.627451,0.172549}%
\pgfsetstrokecolor{currentstroke}%
\pgfsetdash{}{0pt}%
\pgfpathmoveto{\pgfqpoint{7.700389in}{1.127547in}}%
\pgfpathlineto{\pgfqpoint{7.700389in}{1.130887in}}%
\pgfusepath{stroke}%
\end{pgfscope}%
\begin{pgfscope}%
\pgfpathrectangle{\pgfqpoint{1.286132in}{0.839159in}}{\pgfqpoint{12.053712in}{5.967710in}}%
\pgfusepath{clip}%
\pgfsetbuttcap%
\pgfsetroundjoin%
\pgfsetlinewidth{1.505625pt}%
\definecolor{currentstroke}{rgb}{0.172549,0.627451,0.172549}%
\pgfsetstrokecolor{currentstroke}%
\pgfsetdash{}{0pt}%
\pgfpathmoveto{\pgfqpoint{7.811075in}{1.128691in}}%
\pgfpathlineto{\pgfqpoint{7.811075in}{1.128904in}}%
\pgfusepath{stroke}%
\end{pgfscope}%
\begin{pgfscope}%
\pgfpathrectangle{\pgfqpoint{1.286132in}{0.839159in}}{\pgfqpoint{12.053712in}{5.967710in}}%
\pgfusepath{clip}%
\pgfsetbuttcap%
\pgfsetroundjoin%
\pgfsetlinewidth{1.505625pt}%
\definecolor{currentstroke}{rgb}{0.172549,0.627451,0.172549}%
\pgfsetstrokecolor{currentstroke}%
\pgfsetdash{}{0pt}%
\pgfpathmoveto{\pgfqpoint{7.921761in}{1.129032in}}%
\pgfpathlineto{\pgfqpoint{7.921761in}{1.129271in}}%
\pgfusepath{stroke}%
\end{pgfscope}%
\begin{pgfscope}%
\pgfpathrectangle{\pgfqpoint{1.286132in}{0.839159in}}{\pgfqpoint{12.053712in}{5.967710in}}%
\pgfusepath{clip}%
\pgfsetbuttcap%
\pgfsetroundjoin%
\pgfsetlinewidth{1.505625pt}%
\definecolor{currentstroke}{rgb}{0.172549,0.627451,0.172549}%
\pgfsetstrokecolor{currentstroke}%
\pgfsetdash{}{0pt}%
\pgfpathmoveto{\pgfqpoint{8.032447in}{1.129358in}}%
\pgfpathlineto{\pgfqpoint{8.032447in}{1.129703in}}%
\pgfusepath{stroke}%
\end{pgfscope}%
\begin{pgfscope}%
\pgfpathrectangle{\pgfqpoint{1.286132in}{0.839159in}}{\pgfqpoint{12.053712in}{5.967710in}}%
\pgfusepath{clip}%
\pgfsetbuttcap%
\pgfsetroundjoin%
\pgfsetlinewidth{1.505625pt}%
\definecolor{currentstroke}{rgb}{0.172549,0.627451,0.172549}%
\pgfsetstrokecolor{currentstroke}%
\pgfsetdash{}{0pt}%
\pgfpathmoveto{\pgfqpoint{8.143133in}{1.129632in}}%
\pgfpathlineto{\pgfqpoint{8.143133in}{1.129941in}}%
\pgfusepath{stroke}%
\end{pgfscope}%
\begin{pgfscope}%
\pgfpathrectangle{\pgfqpoint{1.286132in}{0.839159in}}{\pgfqpoint{12.053712in}{5.967710in}}%
\pgfusepath{clip}%
\pgfsetbuttcap%
\pgfsetroundjoin%
\pgfsetlinewidth{1.505625pt}%
\definecolor{currentstroke}{rgb}{0.172549,0.627451,0.172549}%
\pgfsetstrokecolor{currentstroke}%
\pgfsetdash{}{0pt}%
\pgfpathmoveto{\pgfqpoint{8.253819in}{1.129899in}}%
\pgfpathlineto{\pgfqpoint{8.253819in}{1.130539in}}%
\pgfusepath{stroke}%
\end{pgfscope}%
\begin{pgfscope}%
\pgfpathrectangle{\pgfqpoint{1.286132in}{0.839159in}}{\pgfqpoint{12.053712in}{5.967710in}}%
\pgfusepath{clip}%
\pgfsetbuttcap%
\pgfsetroundjoin%
\pgfsetlinewidth{1.505625pt}%
\definecolor{currentstroke}{rgb}{0.172549,0.627451,0.172549}%
\pgfsetstrokecolor{currentstroke}%
\pgfsetdash{}{0pt}%
\pgfpathmoveto{\pgfqpoint{8.364505in}{1.130138in}}%
\pgfpathlineto{\pgfqpoint{8.364505in}{1.131332in}}%
\pgfusepath{stroke}%
\end{pgfscope}%
\begin{pgfscope}%
\pgfpathrectangle{\pgfqpoint{1.286132in}{0.839159in}}{\pgfqpoint{12.053712in}{5.967710in}}%
\pgfusepath{clip}%
\pgfsetbuttcap%
\pgfsetroundjoin%
\pgfsetlinewidth{1.505625pt}%
\definecolor{currentstroke}{rgb}{0.172549,0.627451,0.172549}%
\pgfsetstrokecolor{currentstroke}%
\pgfsetdash{}{0pt}%
\pgfpathmoveto{\pgfqpoint{8.475191in}{1.130506in}}%
\pgfpathlineto{\pgfqpoint{8.475191in}{1.131469in}}%
\pgfusepath{stroke}%
\end{pgfscope}%
\begin{pgfscope}%
\pgfpathrectangle{\pgfqpoint{1.286132in}{0.839159in}}{\pgfqpoint{12.053712in}{5.967710in}}%
\pgfusepath{clip}%
\pgfsetbuttcap%
\pgfsetroundjoin%
\pgfsetlinewidth{1.505625pt}%
\definecolor{currentstroke}{rgb}{0.172549,0.627451,0.172549}%
\pgfsetstrokecolor{currentstroke}%
\pgfsetdash{}{0pt}%
\pgfpathmoveto{\pgfqpoint{8.585877in}{1.130950in}}%
\pgfpathlineto{\pgfqpoint{8.585877in}{1.131364in}}%
\pgfusepath{stroke}%
\end{pgfscope}%
\begin{pgfscope}%
\pgfpathrectangle{\pgfqpoint{1.286132in}{0.839159in}}{\pgfqpoint{12.053712in}{5.967710in}}%
\pgfusepath{clip}%
\pgfsetbuttcap%
\pgfsetroundjoin%
\pgfsetlinewidth{1.505625pt}%
\definecolor{currentstroke}{rgb}{0.172549,0.627451,0.172549}%
\pgfsetstrokecolor{currentstroke}%
\pgfsetdash{}{0pt}%
\pgfpathmoveto{\pgfqpoint{8.696563in}{1.131137in}}%
\pgfpathlineto{\pgfqpoint{8.696563in}{1.131747in}}%
\pgfusepath{stroke}%
\end{pgfscope}%
\begin{pgfscope}%
\pgfpathrectangle{\pgfqpoint{1.286132in}{0.839159in}}{\pgfqpoint{12.053712in}{5.967710in}}%
\pgfusepath{clip}%
\pgfsetbuttcap%
\pgfsetroundjoin%
\pgfsetlinewidth{1.505625pt}%
\definecolor{currentstroke}{rgb}{0.172549,0.627451,0.172549}%
\pgfsetstrokecolor{currentstroke}%
\pgfsetdash{}{0pt}%
\pgfpathmoveto{\pgfqpoint{8.807249in}{1.131437in}}%
\pgfpathlineto{\pgfqpoint{8.807249in}{1.132322in}}%
\pgfusepath{stroke}%
\end{pgfscope}%
\begin{pgfscope}%
\pgfpathrectangle{\pgfqpoint{1.286132in}{0.839159in}}{\pgfqpoint{12.053712in}{5.967710in}}%
\pgfusepath{clip}%
\pgfsetbuttcap%
\pgfsetroundjoin%
\pgfsetlinewidth{1.505625pt}%
\definecolor{currentstroke}{rgb}{0.172549,0.627451,0.172549}%
\pgfsetstrokecolor{currentstroke}%
\pgfsetdash{}{0pt}%
\pgfpathmoveto{\pgfqpoint{8.917936in}{1.131524in}}%
\pgfpathlineto{\pgfqpoint{8.917936in}{1.132559in}}%
\pgfusepath{stroke}%
\end{pgfscope}%
\begin{pgfscope}%
\pgfpathrectangle{\pgfqpoint{1.286132in}{0.839159in}}{\pgfqpoint{12.053712in}{5.967710in}}%
\pgfusepath{clip}%
\pgfsetbuttcap%
\pgfsetroundjoin%
\pgfsetlinewidth{1.505625pt}%
\definecolor{currentstroke}{rgb}{0.172549,0.627451,0.172549}%
\pgfsetstrokecolor{currentstroke}%
\pgfsetdash{}{0pt}%
\pgfpathmoveto{\pgfqpoint{9.028622in}{1.131564in}}%
\pgfpathlineto{\pgfqpoint{9.028622in}{1.133443in}}%
\pgfusepath{stroke}%
\end{pgfscope}%
\begin{pgfscope}%
\pgfpathrectangle{\pgfqpoint{1.286132in}{0.839159in}}{\pgfqpoint{12.053712in}{5.967710in}}%
\pgfusepath{clip}%
\pgfsetbuttcap%
\pgfsetroundjoin%
\pgfsetlinewidth{1.505625pt}%
\definecolor{currentstroke}{rgb}{0.172549,0.627451,0.172549}%
\pgfsetstrokecolor{currentstroke}%
\pgfsetdash{}{0pt}%
\pgfpathmoveto{\pgfqpoint{9.139308in}{1.131910in}}%
\pgfpathlineto{\pgfqpoint{9.139308in}{1.132995in}}%
\pgfusepath{stroke}%
\end{pgfscope}%
\begin{pgfscope}%
\pgfpathrectangle{\pgfqpoint{1.286132in}{0.839159in}}{\pgfqpoint{12.053712in}{5.967710in}}%
\pgfusepath{clip}%
\pgfsetbuttcap%
\pgfsetroundjoin%
\pgfsetlinewidth{1.505625pt}%
\definecolor{currentstroke}{rgb}{0.172549,0.627451,0.172549}%
\pgfsetstrokecolor{currentstroke}%
\pgfsetdash{}{0pt}%
\pgfpathmoveto{\pgfqpoint{9.249994in}{1.132622in}}%
\pgfpathlineto{\pgfqpoint{9.249994in}{1.133685in}}%
\pgfusepath{stroke}%
\end{pgfscope}%
\begin{pgfscope}%
\pgfpathrectangle{\pgfqpoint{1.286132in}{0.839159in}}{\pgfqpoint{12.053712in}{5.967710in}}%
\pgfusepath{clip}%
\pgfsetbuttcap%
\pgfsetroundjoin%
\pgfsetlinewidth{1.505625pt}%
\definecolor{currentstroke}{rgb}{0.172549,0.627451,0.172549}%
\pgfsetstrokecolor{currentstroke}%
\pgfsetdash{}{0pt}%
\pgfpathmoveto{\pgfqpoint{9.360680in}{1.132827in}}%
\pgfpathlineto{\pgfqpoint{9.360680in}{1.133747in}}%
\pgfusepath{stroke}%
\end{pgfscope}%
\begin{pgfscope}%
\pgfpathrectangle{\pgfqpoint{1.286132in}{0.839159in}}{\pgfqpoint{12.053712in}{5.967710in}}%
\pgfusepath{clip}%
\pgfsetbuttcap%
\pgfsetroundjoin%
\pgfsetlinewidth{1.505625pt}%
\definecolor{currentstroke}{rgb}{0.172549,0.627451,0.172549}%
\pgfsetstrokecolor{currentstroke}%
\pgfsetdash{}{0pt}%
\pgfpathmoveto{\pgfqpoint{9.471366in}{1.132962in}}%
\pgfpathlineto{\pgfqpoint{9.471366in}{1.134436in}}%
\pgfusepath{stroke}%
\end{pgfscope}%
\begin{pgfscope}%
\pgfpathrectangle{\pgfqpoint{1.286132in}{0.839159in}}{\pgfqpoint{12.053712in}{5.967710in}}%
\pgfusepath{clip}%
\pgfsetbuttcap%
\pgfsetroundjoin%
\pgfsetlinewidth{1.505625pt}%
\definecolor{currentstroke}{rgb}{0.172549,0.627451,0.172549}%
\pgfsetstrokecolor{currentstroke}%
\pgfsetdash{}{0pt}%
\pgfpathmoveto{\pgfqpoint{9.582052in}{1.133526in}}%
\pgfpathlineto{\pgfqpoint{9.582052in}{1.134479in}}%
\pgfusepath{stroke}%
\end{pgfscope}%
\begin{pgfscope}%
\pgfpathrectangle{\pgfqpoint{1.286132in}{0.839159in}}{\pgfqpoint{12.053712in}{5.967710in}}%
\pgfusepath{clip}%
\pgfsetbuttcap%
\pgfsetroundjoin%
\pgfsetlinewidth{1.505625pt}%
\definecolor{currentstroke}{rgb}{0.172549,0.627451,0.172549}%
\pgfsetstrokecolor{currentstroke}%
\pgfsetdash{}{0pt}%
\pgfpathmoveto{\pgfqpoint{9.692738in}{1.134107in}}%
\pgfpathlineto{\pgfqpoint{9.692738in}{1.135417in}}%
\pgfusepath{stroke}%
\end{pgfscope}%
\begin{pgfscope}%
\pgfpathrectangle{\pgfqpoint{1.286132in}{0.839159in}}{\pgfqpoint{12.053712in}{5.967710in}}%
\pgfusepath{clip}%
\pgfsetbuttcap%
\pgfsetroundjoin%
\pgfsetlinewidth{1.505625pt}%
\definecolor{currentstroke}{rgb}{0.172549,0.627451,0.172549}%
\pgfsetstrokecolor{currentstroke}%
\pgfsetdash{}{0pt}%
\pgfpathmoveto{\pgfqpoint{9.803424in}{1.134342in}}%
\pgfpathlineto{\pgfqpoint{9.803424in}{1.134794in}}%
\pgfusepath{stroke}%
\end{pgfscope}%
\begin{pgfscope}%
\pgfpathrectangle{\pgfqpoint{1.286132in}{0.839159in}}{\pgfqpoint{12.053712in}{5.967710in}}%
\pgfusepath{clip}%
\pgfsetbuttcap%
\pgfsetroundjoin%
\pgfsetlinewidth{1.505625pt}%
\definecolor{currentstroke}{rgb}{0.172549,0.627451,0.172549}%
\pgfsetstrokecolor{currentstroke}%
\pgfsetdash{}{0pt}%
\pgfpathmoveto{\pgfqpoint{9.914110in}{1.134517in}}%
\pgfpathlineto{\pgfqpoint{9.914110in}{1.136134in}}%
\pgfusepath{stroke}%
\end{pgfscope}%
\begin{pgfscope}%
\pgfpathrectangle{\pgfqpoint{1.286132in}{0.839159in}}{\pgfqpoint{12.053712in}{5.967710in}}%
\pgfusepath{clip}%
\pgfsetbuttcap%
\pgfsetroundjoin%
\pgfsetlinewidth{1.505625pt}%
\definecolor{currentstroke}{rgb}{0.172549,0.627451,0.172549}%
\pgfsetstrokecolor{currentstroke}%
\pgfsetdash{}{0pt}%
\pgfpathmoveto{\pgfqpoint{10.024796in}{1.135000in}}%
\pgfpathlineto{\pgfqpoint{10.024796in}{1.136379in}}%
\pgfusepath{stroke}%
\end{pgfscope}%
\begin{pgfscope}%
\pgfpathrectangle{\pgfqpoint{1.286132in}{0.839159in}}{\pgfqpoint{12.053712in}{5.967710in}}%
\pgfusepath{clip}%
\pgfsetbuttcap%
\pgfsetroundjoin%
\pgfsetlinewidth{1.505625pt}%
\definecolor{currentstroke}{rgb}{0.172549,0.627451,0.172549}%
\pgfsetstrokecolor{currentstroke}%
\pgfsetdash{}{0pt}%
\pgfpathmoveto{\pgfqpoint{10.135482in}{1.134742in}}%
\pgfpathlineto{\pgfqpoint{10.135482in}{1.139259in}}%
\pgfusepath{stroke}%
\end{pgfscope}%
\begin{pgfscope}%
\pgfpathrectangle{\pgfqpoint{1.286132in}{0.839159in}}{\pgfqpoint{12.053712in}{5.967710in}}%
\pgfusepath{clip}%
\pgfsetbuttcap%
\pgfsetroundjoin%
\pgfsetlinewidth{1.505625pt}%
\definecolor{currentstroke}{rgb}{0.172549,0.627451,0.172549}%
\pgfsetstrokecolor{currentstroke}%
\pgfsetdash{}{0pt}%
\pgfpathmoveto{\pgfqpoint{10.246168in}{1.135865in}}%
\pgfpathlineto{\pgfqpoint{10.246168in}{1.136863in}}%
\pgfusepath{stroke}%
\end{pgfscope}%
\begin{pgfscope}%
\pgfpathrectangle{\pgfqpoint{1.286132in}{0.839159in}}{\pgfqpoint{12.053712in}{5.967710in}}%
\pgfusepath{clip}%
\pgfsetbuttcap%
\pgfsetroundjoin%
\pgfsetlinewidth{1.505625pt}%
\definecolor{currentstroke}{rgb}{0.172549,0.627451,0.172549}%
\pgfsetstrokecolor{currentstroke}%
\pgfsetdash{}{0pt}%
\pgfpathmoveto{\pgfqpoint{10.356854in}{1.136444in}}%
\pgfpathlineto{\pgfqpoint{10.356854in}{1.136893in}}%
\pgfusepath{stroke}%
\end{pgfscope}%
\begin{pgfscope}%
\pgfpathrectangle{\pgfqpoint{1.286132in}{0.839159in}}{\pgfqpoint{12.053712in}{5.967710in}}%
\pgfusepath{clip}%
\pgfsetbuttcap%
\pgfsetroundjoin%
\pgfsetlinewidth{1.505625pt}%
\definecolor{currentstroke}{rgb}{0.172549,0.627451,0.172549}%
\pgfsetstrokecolor{currentstroke}%
\pgfsetdash{}{0pt}%
\pgfpathmoveto{\pgfqpoint{10.467540in}{1.137093in}}%
\pgfpathlineto{\pgfqpoint{10.467540in}{1.138407in}}%
\pgfusepath{stroke}%
\end{pgfscope}%
\begin{pgfscope}%
\pgfpathrectangle{\pgfqpoint{1.286132in}{0.839159in}}{\pgfqpoint{12.053712in}{5.967710in}}%
\pgfusepath{clip}%
\pgfsetbuttcap%
\pgfsetroundjoin%
\pgfsetlinewidth{1.505625pt}%
\definecolor{currentstroke}{rgb}{0.172549,0.627451,0.172549}%
\pgfsetstrokecolor{currentstroke}%
\pgfsetdash{}{0pt}%
\pgfpathmoveto{\pgfqpoint{10.578226in}{1.137415in}}%
\pgfpathlineto{\pgfqpoint{10.578226in}{1.138591in}}%
\pgfusepath{stroke}%
\end{pgfscope}%
\begin{pgfscope}%
\pgfpathrectangle{\pgfqpoint{1.286132in}{0.839159in}}{\pgfqpoint{12.053712in}{5.967710in}}%
\pgfusepath{clip}%
\pgfsetbuttcap%
\pgfsetroundjoin%
\pgfsetlinewidth{1.505625pt}%
\definecolor{currentstroke}{rgb}{0.172549,0.627451,0.172549}%
\pgfsetstrokecolor{currentstroke}%
\pgfsetdash{}{0pt}%
\pgfpathmoveto{\pgfqpoint{10.688913in}{1.137810in}}%
\pgfpathlineto{\pgfqpoint{10.688913in}{1.139657in}}%
\pgfusepath{stroke}%
\end{pgfscope}%
\begin{pgfscope}%
\pgfpathrectangle{\pgfqpoint{1.286132in}{0.839159in}}{\pgfqpoint{12.053712in}{5.967710in}}%
\pgfusepath{clip}%
\pgfsetbuttcap%
\pgfsetroundjoin%
\pgfsetlinewidth{1.505625pt}%
\definecolor{currentstroke}{rgb}{0.172549,0.627451,0.172549}%
\pgfsetstrokecolor{currentstroke}%
\pgfsetdash{}{0pt}%
\pgfpathmoveto{\pgfqpoint{10.799599in}{1.138203in}}%
\pgfpathlineto{\pgfqpoint{10.799599in}{1.139233in}}%
\pgfusepath{stroke}%
\end{pgfscope}%
\begin{pgfscope}%
\pgfpathrectangle{\pgfqpoint{1.286132in}{0.839159in}}{\pgfqpoint{12.053712in}{5.967710in}}%
\pgfusepath{clip}%
\pgfsetbuttcap%
\pgfsetroundjoin%
\pgfsetlinewidth{1.505625pt}%
\definecolor{currentstroke}{rgb}{0.172549,0.627451,0.172549}%
\pgfsetstrokecolor{currentstroke}%
\pgfsetdash{}{0pt}%
\pgfpathmoveto{\pgfqpoint{10.910285in}{1.138516in}}%
\pgfpathlineto{\pgfqpoint{10.910285in}{1.140312in}}%
\pgfusepath{stroke}%
\end{pgfscope}%
\begin{pgfscope}%
\pgfpathrectangle{\pgfqpoint{1.286132in}{0.839159in}}{\pgfqpoint{12.053712in}{5.967710in}}%
\pgfusepath{clip}%
\pgfsetbuttcap%
\pgfsetroundjoin%
\pgfsetlinewidth{1.505625pt}%
\definecolor{currentstroke}{rgb}{0.172549,0.627451,0.172549}%
\pgfsetstrokecolor{currentstroke}%
\pgfsetdash{}{0pt}%
\pgfpathmoveto{\pgfqpoint{11.020971in}{1.139019in}}%
\pgfpathlineto{\pgfqpoint{11.020971in}{1.141637in}}%
\pgfusepath{stroke}%
\end{pgfscope}%
\begin{pgfscope}%
\pgfpathrectangle{\pgfqpoint{1.286132in}{0.839159in}}{\pgfqpoint{12.053712in}{5.967710in}}%
\pgfusepath{clip}%
\pgfsetbuttcap%
\pgfsetroundjoin%
\pgfsetlinewidth{1.505625pt}%
\definecolor{currentstroke}{rgb}{0.172549,0.627451,0.172549}%
\pgfsetstrokecolor{currentstroke}%
\pgfsetdash{}{0pt}%
\pgfpathmoveto{\pgfqpoint{11.131657in}{1.139729in}}%
\pgfpathlineto{\pgfqpoint{11.131657in}{1.140249in}}%
\pgfusepath{stroke}%
\end{pgfscope}%
\begin{pgfscope}%
\pgfpathrectangle{\pgfqpoint{1.286132in}{0.839159in}}{\pgfqpoint{12.053712in}{5.967710in}}%
\pgfusepath{clip}%
\pgfsetbuttcap%
\pgfsetroundjoin%
\pgfsetlinewidth{1.505625pt}%
\definecolor{currentstroke}{rgb}{0.172549,0.627451,0.172549}%
\pgfsetstrokecolor{currentstroke}%
\pgfsetdash{}{0pt}%
\pgfpathmoveto{\pgfqpoint{11.242343in}{1.140289in}}%
\pgfpathlineto{\pgfqpoint{11.242343in}{1.141864in}}%
\pgfusepath{stroke}%
\end{pgfscope}%
\begin{pgfscope}%
\pgfpathrectangle{\pgfqpoint{1.286132in}{0.839159in}}{\pgfqpoint{12.053712in}{5.967710in}}%
\pgfusepath{clip}%
\pgfsetbuttcap%
\pgfsetroundjoin%
\pgfsetlinewidth{1.505625pt}%
\definecolor{currentstroke}{rgb}{0.172549,0.627451,0.172549}%
\pgfsetstrokecolor{currentstroke}%
\pgfsetdash{}{0pt}%
\pgfpathmoveto{\pgfqpoint{11.353029in}{1.131364in}}%
\pgfpathlineto{\pgfqpoint{11.353029in}{1.172495in}}%
\pgfusepath{stroke}%
\end{pgfscope}%
\begin{pgfscope}%
\pgfpathrectangle{\pgfqpoint{1.286132in}{0.839159in}}{\pgfqpoint{12.053712in}{5.967710in}}%
\pgfusepath{clip}%
\pgfsetbuttcap%
\pgfsetroundjoin%
\pgfsetlinewidth{1.505625pt}%
\definecolor{currentstroke}{rgb}{0.172549,0.627451,0.172549}%
\pgfsetstrokecolor{currentstroke}%
\pgfsetdash{}{0pt}%
\pgfpathmoveto{\pgfqpoint{11.463715in}{1.140933in}}%
\pgfpathlineto{\pgfqpoint{11.463715in}{1.141173in}}%
\pgfusepath{stroke}%
\end{pgfscope}%
\begin{pgfscope}%
\pgfpathrectangle{\pgfqpoint{1.286132in}{0.839159in}}{\pgfqpoint{12.053712in}{5.967710in}}%
\pgfusepath{clip}%
\pgfsetbuttcap%
\pgfsetroundjoin%
\pgfsetlinewidth{1.505625pt}%
\definecolor{currentstroke}{rgb}{0.172549,0.627451,0.172549}%
\pgfsetstrokecolor{currentstroke}%
\pgfsetdash{}{0pt}%
\pgfpathmoveto{\pgfqpoint{11.574401in}{1.141509in}}%
\pgfpathlineto{\pgfqpoint{11.574401in}{1.142123in}}%
\pgfusepath{stroke}%
\end{pgfscope}%
\begin{pgfscope}%
\pgfpathrectangle{\pgfqpoint{1.286132in}{0.839159in}}{\pgfqpoint{12.053712in}{5.967710in}}%
\pgfusepath{clip}%
\pgfsetbuttcap%
\pgfsetroundjoin%
\pgfsetlinewidth{1.505625pt}%
\definecolor{currentstroke}{rgb}{0.172549,0.627451,0.172549}%
\pgfsetstrokecolor{currentstroke}%
\pgfsetdash{}{0pt}%
\pgfpathmoveto{\pgfqpoint{11.685087in}{1.139203in}}%
\pgfpathlineto{\pgfqpoint{11.685087in}{1.151661in}}%
\pgfusepath{stroke}%
\end{pgfscope}%
\begin{pgfscope}%
\pgfpathrectangle{\pgfqpoint{1.286132in}{0.839159in}}{\pgfqpoint{12.053712in}{5.967710in}}%
\pgfusepath{clip}%
\pgfsetbuttcap%
\pgfsetroundjoin%
\pgfsetlinewidth{1.505625pt}%
\definecolor{currentstroke}{rgb}{0.172549,0.627451,0.172549}%
\pgfsetstrokecolor{currentstroke}%
\pgfsetdash{}{0pt}%
\pgfpathmoveto{\pgfqpoint{11.795773in}{1.142762in}}%
\pgfpathlineto{\pgfqpoint{11.795773in}{1.143618in}}%
\pgfusepath{stroke}%
\end{pgfscope}%
\begin{pgfscope}%
\pgfpathrectangle{\pgfqpoint{1.286132in}{0.839159in}}{\pgfqpoint{12.053712in}{5.967710in}}%
\pgfusepath{clip}%
\pgfsetbuttcap%
\pgfsetroundjoin%
\pgfsetlinewidth{1.505625pt}%
\definecolor{currentstroke}{rgb}{0.172549,0.627451,0.172549}%
\pgfsetstrokecolor{currentstroke}%
\pgfsetdash{}{0pt}%
\pgfpathmoveto{\pgfqpoint{11.906459in}{1.143104in}}%
\pgfpathlineto{\pgfqpoint{11.906459in}{1.143888in}}%
\pgfusepath{stroke}%
\end{pgfscope}%
\begin{pgfscope}%
\pgfpathrectangle{\pgfqpoint{1.286132in}{0.839159in}}{\pgfqpoint{12.053712in}{5.967710in}}%
\pgfusepath{clip}%
\pgfsetbuttcap%
\pgfsetroundjoin%
\pgfsetlinewidth{1.505625pt}%
\definecolor{currentstroke}{rgb}{0.172549,0.627451,0.172549}%
\pgfsetstrokecolor{currentstroke}%
\pgfsetdash{}{0pt}%
\pgfpathmoveto{\pgfqpoint{12.017145in}{1.143838in}}%
\pgfpathlineto{\pgfqpoint{12.017145in}{1.147708in}}%
\pgfusepath{stroke}%
\end{pgfscope}%
\begin{pgfscope}%
\pgfpathrectangle{\pgfqpoint{1.286132in}{0.839159in}}{\pgfqpoint{12.053712in}{5.967710in}}%
\pgfusepath{clip}%
\pgfsetbuttcap%
\pgfsetroundjoin%
\pgfsetlinewidth{1.505625pt}%
\definecolor{currentstroke}{rgb}{0.172549,0.627451,0.172549}%
\pgfsetstrokecolor{currentstroke}%
\pgfsetdash{}{0pt}%
\pgfpathmoveto{\pgfqpoint{12.127831in}{1.144682in}}%
\pgfpathlineto{\pgfqpoint{12.127831in}{1.145951in}}%
\pgfusepath{stroke}%
\end{pgfscope}%
\begin{pgfscope}%
\pgfpathrectangle{\pgfqpoint{1.286132in}{0.839159in}}{\pgfqpoint{12.053712in}{5.967710in}}%
\pgfusepath{clip}%
\pgfsetbuttcap%
\pgfsetroundjoin%
\pgfsetlinewidth{1.505625pt}%
\definecolor{currentstroke}{rgb}{0.172549,0.627451,0.172549}%
\pgfsetstrokecolor{currentstroke}%
\pgfsetdash{}{0pt}%
\pgfpathmoveto{\pgfqpoint{12.238517in}{1.144121in}}%
\pgfpathlineto{\pgfqpoint{12.238517in}{1.147080in}}%
\pgfusepath{stroke}%
\end{pgfscope}%
\begin{pgfscope}%
\pgfpathrectangle{\pgfqpoint{1.286132in}{0.839159in}}{\pgfqpoint{12.053712in}{5.967710in}}%
\pgfusepath{clip}%
\pgfsetbuttcap%
\pgfsetroundjoin%
\pgfsetlinewidth{1.505625pt}%
\definecolor{currentstroke}{rgb}{0.172549,0.627451,0.172549}%
\pgfsetstrokecolor{currentstroke}%
\pgfsetdash{}{0pt}%
\pgfpathmoveto{\pgfqpoint{12.349203in}{1.145011in}}%
\pgfpathlineto{\pgfqpoint{12.349203in}{1.145557in}}%
\pgfusepath{stroke}%
\end{pgfscope}%
\begin{pgfscope}%
\pgfpathrectangle{\pgfqpoint{1.286132in}{0.839159in}}{\pgfqpoint{12.053712in}{5.967710in}}%
\pgfusepath{clip}%
\pgfsetbuttcap%
\pgfsetroundjoin%
\pgfsetlinewidth{1.505625pt}%
\definecolor{currentstroke}{rgb}{0.172549,0.627451,0.172549}%
\pgfsetstrokecolor{currentstroke}%
\pgfsetdash{}{0pt}%
\pgfpathmoveto{\pgfqpoint{12.459890in}{1.145698in}}%
\pgfpathlineto{\pgfqpoint{12.459890in}{1.147192in}}%
\pgfusepath{stroke}%
\end{pgfscope}%
\begin{pgfscope}%
\pgfpathrectangle{\pgfqpoint{1.286132in}{0.839159in}}{\pgfqpoint{12.053712in}{5.967710in}}%
\pgfusepath{clip}%
\pgfsetbuttcap%
\pgfsetroundjoin%
\pgfsetlinewidth{1.505625pt}%
\definecolor{currentstroke}{rgb}{0.172549,0.627451,0.172549}%
\pgfsetstrokecolor{currentstroke}%
\pgfsetdash{}{0pt}%
\pgfpathmoveto{\pgfqpoint{12.570576in}{1.145958in}}%
\pgfpathlineto{\pgfqpoint{12.570576in}{1.147698in}}%
\pgfusepath{stroke}%
\end{pgfscope}%
\begin{pgfscope}%
\pgfpathrectangle{\pgfqpoint{1.286132in}{0.839159in}}{\pgfqpoint{12.053712in}{5.967710in}}%
\pgfusepath{clip}%
\pgfsetbuttcap%
\pgfsetroundjoin%
\pgfsetlinewidth{1.505625pt}%
\definecolor{currentstroke}{rgb}{0.172549,0.627451,0.172549}%
\pgfsetstrokecolor{currentstroke}%
\pgfsetdash{}{0pt}%
\pgfpathmoveto{\pgfqpoint{12.681262in}{1.146225in}}%
\pgfpathlineto{\pgfqpoint{12.681262in}{1.150359in}}%
\pgfusepath{stroke}%
\end{pgfscope}%
\begin{pgfscope}%
\pgfpathrectangle{\pgfqpoint{1.286132in}{0.839159in}}{\pgfqpoint{12.053712in}{5.967710in}}%
\pgfusepath{clip}%
\pgfsetbuttcap%
\pgfsetroundjoin%
\pgfsetlinewidth{1.505625pt}%
\definecolor{currentstroke}{rgb}{0.172549,0.627451,0.172549}%
\pgfsetstrokecolor{currentstroke}%
\pgfsetdash{}{0pt}%
\pgfpathmoveto{\pgfqpoint{12.791948in}{1.147225in}}%
\pgfpathlineto{\pgfqpoint{12.791948in}{1.151101in}}%
\pgfusepath{stroke}%
\end{pgfscope}%
\begin{pgfscope}%
\pgfpathrectangle{\pgfqpoint{1.286132in}{0.839159in}}{\pgfqpoint{12.053712in}{5.967710in}}%
\pgfusepath{clip}%
\pgfsetbuttcap%
\pgfsetroundjoin%
\pgfsetlinewidth{1.505625pt}%
\definecolor{currentstroke}{rgb}{0.839216,0.152941,0.156863}%
\pgfsetstrokecolor{currentstroke}%
\pgfsetdash{}{0pt}%
\pgfpathmoveto{\pgfqpoint{1.834028in}{1.119256in}}%
\pgfpathlineto{\pgfqpoint{1.834028in}{1.119266in}}%
\pgfusepath{stroke}%
\end{pgfscope}%
\begin{pgfscope}%
\pgfpathrectangle{\pgfqpoint{1.286132in}{0.839159in}}{\pgfqpoint{12.053712in}{5.967710in}}%
\pgfusepath{clip}%
\pgfsetbuttcap%
\pgfsetroundjoin%
\pgfsetlinewidth{1.505625pt}%
\definecolor{currentstroke}{rgb}{0.839216,0.152941,0.156863}%
\pgfsetstrokecolor{currentstroke}%
\pgfsetdash{}{0pt}%
\pgfpathmoveto{\pgfqpoint{1.944714in}{1.119585in}}%
\pgfpathlineto{\pgfqpoint{1.944714in}{1.119639in}}%
\pgfusepath{stroke}%
\end{pgfscope}%
\begin{pgfscope}%
\pgfpathrectangle{\pgfqpoint{1.286132in}{0.839159in}}{\pgfqpoint{12.053712in}{5.967710in}}%
\pgfusepath{clip}%
\pgfsetbuttcap%
\pgfsetroundjoin%
\pgfsetlinewidth{1.505625pt}%
\definecolor{currentstroke}{rgb}{0.839216,0.152941,0.156863}%
\pgfsetstrokecolor{currentstroke}%
\pgfsetdash{}{0pt}%
\pgfpathmoveto{\pgfqpoint{2.055400in}{1.119913in}}%
\pgfpathlineto{\pgfqpoint{2.055400in}{1.119949in}}%
\pgfusepath{stroke}%
\end{pgfscope}%
\begin{pgfscope}%
\pgfpathrectangle{\pgfqpoint{1.286132in}{0.839159in}}{\pgfqpoint{12.053712in}{5.967710in}}%
\pgfusepath{clip}%
\pgfsetbuttcap%
\pgfsetroundjoin%
\pgfsetlinewidth{1.505625pt}%
\definecolor{currentstroke}{rgb}{0.839216,0.152941,0.156863}%
\pgfsetstrokecolor{currentstroke}%
\pgfsetdash{}{0pt}%
\pgfpathmoveto{\pgfqpoint{2.166086in}{1.120316in}}%
\pgfpathlineto{\pgfqpoint{2.166086in}{1.120631in}}%
\pgfusepath{stroke}%
\end{pgfscope}%
\begin{pgfscope}%
\pgfpathrectangle{\pgfqpoint{1.286132in}{0.839159in}}{\pgfqpoint{12.053712in}{5.967710in}}%
\pgfusepath{clip}%
\pgfsetbuttcap%
\pgfsetroundjoin%
\pgfsetlinewidth{1.505625pt}%
\definecolor{currentstroke}{rgb}{0.839216,0.152941,0.156863}%
\pgfsetstrokecolor{currentstroke}%
\pgfsetdash{}{0pt}%
\pgfpathmoveto{\pgfqpoint{2.276772in}{1.120839in}}%
\pgfpathlineto{\pgfqpoint{2.276772in}{1.120944in}}%
\pgfusepath{stroke}%
\end{pgfscope}%
\begin{pgfscope}%
\pgfpathrectangle{\pgfqpoint{1.286132in}{0.839159in}}{\pgfqpoint{12.053712in}{5.967710in}}%
\pgfusepath{clip}%
\pgfsetbuttcap%
\pgfsetroundjoin%
\pgfsetlinewidth{1.505625pt}%
\definecolor{currentstroke}{rgb}{0.839216,0.152941,0.156863}%
\pgfsetstrokecolor{currentstroke}%
\pgfsetdash{}{0pt}%
\pgfpathmoveto{\pgfqpoint{2.387458in}{1.121510in}}%
\pgfpathlineto{\pgfqpoint{2.387458in}{1.121696in}}%
\pgfusepath{stroke}%
\end{pgfscope}%
\begin{pgfscope}%
\pgfpathrectangle{\pgfqpoint{1.286132in}{0.839159in}}{\pgfqpoint{12.053712in}{5.967710in}}%
\pgfusepath{clip}%
\pgfsetbuttcap%
\pgfsetroundjoin%
\pgfsetlinewidth{1.505625pt}%
\definecolor{currentstroke}{rgb}{0.839216,0.152941,0.156863}%
\pgfsetstrokecolor{currentstroke}%
\pgfsetdash{}{0pt}%
\pgfpathmoveto{\pgfqpoint{2.498144in}{1.122404in}}%
\pgfpathlineto{\pgfqpoint{2.498144in}{1.122672in}}%
\pgfusepath{stroke}%
\end{pgfscope}%
\begin{pgfscope}%
\pgfpathrectangle{\pgfqpoint{1.286132in}{0.839159in}}{\pgfqpoint{12.053712in}{5.967710in}}%
\pgfusepath{clip}%
\pgfsetbuttcap%
\pgfsetroundjoin%
\pgfsetlinewidth{1.505625pt}%
\definecolor{currentstroke}{rgb}{0.839216,0.152941,0.156863}%
\pgfsetstrokecolor{currentstroke}%
\pgfsetdash{}{0pt}%
\pgfpathmoveto{\pgfqpoint{2.608830in}{1.123292in}}%
\pgfpathlineto{\pgfqpoint{2.608830in}{1.123689in}}%
\pgfusepath{stroke}%
\end{pgfscope}%
\begin{pgfscope}%
\pgfpathrectangle{\pgfqpoint{1.286132in}{0.839159in}}{\pgfqpoint{12.053712in}{5.967710in}}%
\pgfusepath{clip}%
\pgfsetbuttcap%
\pgfsetroundjoin%
\pgfsetlinewidth{1.505625pt}%
\definecolor{currentstroke}{rgb}{0.839216,0.152941,0.156863}%
\pgfsetstrokecolor{currentstroke}%
\pgfsetdash{}{0pt}%
\pgfpathmoveto{\pgfqpoint{2.719516in}{1.124450in}}%
\pgfpathlineto{\pgfqpoint{2.719516in}{1.124790in}}%
\pgfusepath{stroke}%
\end{pgfscope}%
\begin{pgfscope}%
\pgfpathrectangle{\pgfqpoint{1.286132in}{0.839159in}}{\pgfqpoint{12.053712in}{5.967710in}}%
\pgfusepath{clip}%
\pgfsetbuttcap%
\pgfsetroundjoin%
\pgfsetlinewidth{1.505625pt}%
\definecolor{currentstroke}{rgb}{0.839216,0.152941,0.156863}%
\pgfsetstrokecolor{currentstroke}%
\pgfsetdash{}{0pt}%
\pgfpathmoveto{\pgfqpoint{2.830202in}{1.126157in}}%
\pgfpathlineto{\pgfqpoint{2.830202in}{1.126690in}}%
\pgfusepath{stroke}%
\end{pgfscope}%
\begin{pgfscope}%
\pgfpathrectangle{\pgfqpoint{1.286132in}{0.839159in}}{\pgfqpoint{12.053712in}{5.967710in}}%
\pgfusepath{clip}%
\pgfsetbuttcap%
\pgfsetroundjoin%
\pgfsetlinewidth{1.505625pt}%
\definecolor{currentstroke}{rgb}{0.839216,0.152941,0.156863}%
\pgfsetstrokecolor{currentstroke}%
\pgfsetdash{}{0pt}%
\pgfpathmoveto{\pgfqpoint{2.940888in}{1.118908in}}%
\pgfpathlineto{\pgfqpoint{2.940888in}{1.153888in}}%
\pgfusepath{stroke}%
\end{pgfscope}%
\begin{pgfscope}%
\pgfpathrectangle{\pgfqpoint{1.286132in}{0.839159in}}{\pgfqpoint{12.053712in}{5.967710in}}%
\pgfusepath{clip}%
\pgfsetbuttcap%
\pgfsetroundjoin%
\pgfsetlinewidth{1.505625pt}%
\definecolor{currentstroke}{rgb}{0.839216,0.152941,0.156863}%
\pgfsetstrokecolor{currentstroke}%
\pgfsetdash{}{0pt}%
\pgfpathmoveto{\pgfqpoint{3.051574in}{1.129430in}}%
\pgfpathlineto{\pgfqpoint{3.051574in}{1.129828in}}%
\pgfusepath{stroke}%
\end{pgfscope}%
\begin{pgfscope}%
\pgfpathrectangle{\pgfqpoint{1.286132in}{0.839159in}}{\pgfqpoint{12.053712in}{5.967710in}}%
\pgfusepath{clip}%
\pgfsetbuttcap%
\pgfsetroundjoin%
\pgfsetlinewidth{1.505625pt}%
\definecolor{currentstroke}{rgb}{0.839216,0.152941,0.156863}%
\pgfsetstrokecolor{currentstroke}%
\pgfsetdash{}{0pt}%
\pgfpathmoveto{\pgfqpoint{3.162260in}{1.124033in}}%
\pgfpathlineto{\pgfqpoint{3.162260in}{1.154822in}}%
\pgfusepath{stroke}%
\end{pgfscope}%
\begin{pgfscope}%
\pgfpathrectangle{\pgfqpoint{1.286132in}{0.839159in}}{\pgfqpoint{12.053712in}{5.967710in}}%
\pgfusepath{clip}%
\pgfsetbuttcap%
\pgfsetroundjoin%
\pgfsetlinewidth{1.505625pt}%
\definecolor{currentstroke}{rgb}{0.839216,0.152941,0.156863}%
\pgfsetstrokecolor{currentstroke}%
\pgfsetdash{}{0pt}%
\pgfpathmoveto{\pgfqpoint{3.272946in}{1.134031in}}%
\pgfpathlineto{\pgfqpoint{3.272946in}{1.134667in}}%
\pgfusepath{stroke}%
\end{pgfscope}%
\begin{pgfscope}%
\pgfpathrectangle{\pgfqpoint{1.286132in}{0.839159in}}{\pgfqpoint{12.053712in}{5.967710in}}%
\pgfusepath{clip}%
\pgfsetbuttcap%
\pgfsetroundjoin%
\pgfsetlinewidth{1.505625pt}%
\definecolor{currentstroke}{rgb}{0.839216,0.152941,0.156863}%
\pgfsetstrokecolor{currentstroke}%
\pgfsetdash{}{0pt}%
\pgfpathmoveto{\pgfqpoint{3.383632in}{1.136312in}}%
\pgfpathlineto{\pgfqpoint{3.383632in}{1.143895in}}%
\pgfusepath{stroke}%
\end{pgfscope}%
\begin{pgfscope}%
\pgfpathrectangle{\pgfqpoint{1.286132in}{0.839159in}}{\pgfqpoint{12.053712in}{5.967710in}}%
\pgfusepath{clip}%
\pgfsetbuttcap%
\pgfsetroundjoin%
\pgfsetlinewidth{1.505625pt}%
\definecolor{currentstroke}{rgb}{0.839216,0.152941,0.156863}%
\pgfsetstrokecolor{currentstroke}%
\pgfsetdash{}{0pt}%
\pgfpathmoveto{\pgfqpoint{3.494319in}{1.139661in}}%
\pgfpathlineto{\pgfqpoint{3.494319in}{1.140530in}}%
\pgfusepath{stroke}%
\end{pgfscope}%
\begin{pgfscope}%
\pgfpathrectangle{\pgfqpoint{1.286132in}{0.839159in}}{\pgfqpoint{12.053712in}{5.967710in}}%
\pgfusepath{clip}%
\pgfsetbuttcap%
\pgfsetroundjoin%
\pgfsetlinewidth{1.505625pt}%
\definecolor{currentstroke}{rgb}{0.839216,0.152941,0.156863}%
\pgfsetstrokecolor{currentstroke}%
\pgfsetdash{}{0pt}%
\pgfpathmoveto{\pgfqpoint{3.605005in}{1.144066in}}%
\pgfpathlineto{\pgfqpoint{3.605005in}{1.148018in}}%
\pgfusepath{stroke}%
\end{pgfscope}%
\begin{pgfscope}%
\pgfpathrectangle{\pgfqpoint{1.286132in}{0.839159in}}{\pgfqpoint{12.053712in}{5.967710in}}%
\pgfusepath{clip}%
\pgfsetbuttcap%
\pgfsetroundjoin%
\pgfsetlinewidth{1.505625pt}%
\definecolor{currentstroke}{rgb}{0.839216,0.152941,0.156863}%
\pgfsetstrokecolor{currentstroke}%
\pgfsetdash{}{0pt}%
\pgfpathmoveto{\pgfqpoint{3.715691in}{1.146940in}}%
\pgfpathlineto{\pgfqpoint{3.715691in}{1.150972in}}%
\pgfusepath{stroke}%
\end{pgfscope}%
\begin{pgfscope}%
\pgfpathrectangle{\pgfqpoint{1.286132in}{0.839159in}}{\pgfqpoint{12.053712in}{5.967710in}}%
\pgfusepath{clip}%
\pgfsetbuttcap%
\pgfsetroundjoin%
\pgfsetlinewidth{1.505625pt}%
\definecolor{currentstroke}{rgb}{0.839216,0.152941,0.156863}%
\pgfsetstrokecolor{currentstroke}%
\pgfsetdash{}{0pt}%
\pgfpathmoveto{\pgfqpoint{3.826377in}{1.150640in}}%
\pgfpathlineto{\pgfqpoint{3.826377in}{1.158588in}}%
\pgfusepath{stroke}%
\end{pgfscope}%
\begin{pgfscope}%
\pgfpathrectangle{\pgfqpoint{1.286132in}{0.839159in}}{\pgfqpoint{12.053712in}{5.967710in}}%
\pgfusepath{clip}%
\pgfsetbuttcap%
\pgfsetroundjoin%
\pgfsetlinewidth{1.505625pt}%
\definecolor{currentstroke}{rgb}{0.839216,0.152941,0.156863}%
\pgfsetstrokecolor{currentstroke}%
\pgfsetdash{}{0pt}%
\pgfpathmoveto{\pgfqpoint{3.937063in}{1.156854in}}%
\pgfpathlineto{\pgfqpoint{3.937063in}{1.158723in}}%
\pgfusepath{stroke}%
\end{pgfscope}%
\begin{pgfscope}%
\pgfpathrectangle{\pgfqpoint{1.286132in}{0.839159in}}{\pgfqpoint{12.053712in}{5.967710in}}%
\pgfusepath{clip}%
\pgfsetbuttcap%
\pgfsetroundjoin%
\pgfsetlinewidth{1.505625pt}%
\definecolor{currentstroke}{rgb}{0.839216,0.152941,0.156863}%
\pgfsetstrokecolor{currentstroke}%
\pgfsetdash{}{0pt}%
\pgfpathmoveto{\pgfqpoint{4.047749in}{1.159799in}}%
\pgfpathlineto{\pgfqpoint{4.047749in}{1.175422in}}%
\pgfusepath{stroke}%
\end{pgfscope}%
\begin{pgfscope}%
\pgfpathrectangle{\pgfqpoint{1.286132in}{0.839159in}}{\pgfqpoint{12.053712in}{5.967710in}}%
\pgfusepath{clip}%
\pgfsetbuttcap%
\pgfsetroundjoin%
\pgfsetlinewidth{1.505625pt}%
\definecolor{currentstroke}{rgb}{0.839216,0.152941,0.156863}%
\pgfsetstrokecolor{currentstroke}%
\pgfsetdash{}{0pt}%
\pgfpathmoveto{\pgfqpoint{4.158435in}{1.164452in}}%
\pgfpathlineto{\pgfqpoint{4.158435in}{1.178078in}}%
\pgfusepath{stroke}%
\end{pgfscope}%
\begin{pgfscope}%
\pgfpathrectangle{\pgfqpoint{1.286132in}{0.839159in}}{\pgfqpoint{12.053712in}{5.967710in}}%
\pgfusepath{clip}%
\pgfsetbuttcap%
\pgfsetroundjoin%
\pgfsetlinewidth{1.505625pt}%
\definecolor{currentstroke}{rgb}{0.839216,0.152941,0.156863}%
\pgfsetstrokecolor{currentstroke}%
\pgfsetdash{}{0pt}%
\pgfpathmoveto{\pgfqpoint{4.269121in}{1.173247in}}%
\pgfpathlineto{\pgfqpoint{4.269121in}{1.176111in}}%
\pgfusepath{stroke}%
\end{pgfscope}%
\begin{pgfscope}%
\pgfpathrectangle{\pgfqpoint{1.286132in}{0.839159in}}{\pgfqpoint{12.053712in}{5.967710in}}%
\pgfusepath{clip}%
\pgfsetbuttcap%
\pgfsetroundjoin%
\pgfsetlinewidth{1.505625pt}%
\definecolor{currentstroke}{rgb}{0.839216,0.152941,0.156863}%
\pgfsetstrokecolor{currentstroke}%
\pgfsetdash{}{0pt}%
\pgfpathmoveto{\pgfqpoint{4.379807in}{1.182646in}}%
\pgfpathlineto{\pgfqpoint{4.379807in}{1.203335in}}%
\pgfusepath{stroke}%
\end{pgfscope}%
\begin{pgfscope}%
\pgfpathrectangle{\pgfqpoint{1.286132in}{0.839159in}}{\pgfqpoint{12.053712in}{5.967710in}}%
\pgfusepath{clip}%
\pgfsetbuttcap%
\pgfsetroundjoin%
\pgfsetlinewidth{1.505625pt}%
\definecolor{currentstroke}{rgb}{0.839216,0.152941,0.156863}%
\pgfsetstrokecolor{currentstroke}%
\pgfsetdash{}{0pt}%
\pgfpathmoveto{\pgfqpoint{4.490493in}{1.191801in}}%
\pgfpathlineto{\pgfqpoint{4.490493in}{1.201465in}}%
\pgfusepath{stroke}%
\end{pgfscope}%
\begin{pgfscope}%
\pgfpathrectangle{\pgfqpoint{1.286132in}{0.839159in}}{\pgfqpoint{12.053712in}{5.967710in}}%
\pgfusepath{clip}%
\pgfsetbuttcap%
\pgfsetroundjoin%
\pgfsetlinewidth{1.505625pt}%
\definecolor{currentstroke}{rgb}{0.839216,0.152941,0.156863}%
\pgfsetstrokecolor{currentstroke}%
\pgfsetdash{}{0pt}%
\pgfpathmoveto{\pgfqpoint{4.601179in}{1.197526in}}%
\pgfpathlineto{\pgfqpoint{4.601179in}{1.206768in}}%
\pgfusepath{stroke}%
\end{pgfscope}%
\begin{pgfscope}%
\pgfpathrectangle{\pgfqpoint{1.286132in}{0.839159in}}{\pgfqpoint{12.053712in}{5.967710in}}%
\pgfusepath{clip}%
\pgfsetbuttcap%
\pgfsetroundjoin%
\pgfsetlinewidth{1.505625pt}%
\definecolor{currentstroke}{rgb}{0.839216,0.152941,0.156863}%
\pgfsetstrokecolor{currentstroke}%
\pgfsetdash{}{0pt}%
\pgfpathmoveto{\pgfqpoint{4.711865in}{1.207489in}}%
\pgfpathlineto{\pgfqpoint{4.711865in}{1.216761in}}%
\pgfusepath{stroke}%
\end{pgfscope}%
\begin{pgfscope}%
\pgfpathrectangle{\pgfqpoint{1.286132in}{0.839159in}}{\pgfqpoint{12.053712in}{5.967710in}}%
\pgfusepath{clip}%
\pgfsetbuttcap%
\pgfsetroundjoin%
\pgfsetlinewidth{1.505625pt}%
\definecolor{currentstroke}{rgb}{0.839216,0.152941,0.156863}%
\pgfsetstrokecolor{currentstroke}%
\pgfsetdash{}{0pt}%
\pgfpathmoveto{\pgfqpoint{4.822551in}{1.217707in}}%
\pgfpathlineto{\pgfqpoint{4.822551in}{1.223530in}}%
\pgfusepath{stroke}%
\end{pgfscope}%
\begin{pgfscope}%
\pgfpathrectangle{\pgfqpoint{1.286132in}{0.839159in}}{\pgfqpoint{12.053712in}{5.967710in}}%
\pgfusepath{clip}%
\pgfsetbuttcap%
\pgfsetroundjoin%
\pgfsetlinewidth{1.505625pt}%
\definecolor{currentstroke}{rgb}{0.839216,0.152941,0.156863}%
\pgfsetstrokecolor{currentstroke}%
\pgfsetdash{}{0pt}%
\pgfpathmoveto{\pgfqpoint{4.933237in}{1.227658in}}%
\pgfpathlineto{\pgfqpoint{4.933237in}{1.238272in}}%
\pgfusepath{stroke}%
\end{pgfscope}%
\begin{pgfscope}%
\pgfpathrectangle{\pgfqpoint{1.286132in}{0.839159in}}{\pgfqpoint{12.053712in}{5.967710in}}%
\pgfusepath{clip}%
\pgfsetbuttcap%
\pgfsetroundjoin%
\pgfsetlinewidth{1.505625pt}%
\definecolor{currentstroke}{rgb}{0.839216,0.152941,0.156863}%
\pgfsetstrokecolor{currentstroke}%
\pgfsetdash{}{0pt}%
\pgfpathmoveto{\pgfqpoint{5.043923in}{1.235351in}}%
\pgfpathlineto{\pgfqpoint{5.043923in}{1.243712in}}%
\pgfusepath{stroke}%
\end{pgfscope}%
\begin{pgfscope}%
\pgfpathrectangle{\pgfqpoint{1.286132in}{0.839159in}}{\pgfqpoint{12.053712in}{5.967710in}}%
\pgfusepath{clip}%
\pgfsetbuttcap%
\pgfsetroundjoin%
\pgfsetlinewidth{1.505625pt}%
\definecolor{currentstroke}{rgb}{0.839216,0.152941,0.156863}%
\pgfsetstrokecolor{currentstroke}%
\pgfsetdash{}{0pt}%
\pgfpathmoveto{\pgfqpoint{5.154609in}{1.248194in}}%
\pgfpathlineto{\pgfqpoint{5.154609in}{1.254318in}}%
\pgfusepath{stroke}%
\end{pgfscope}%
\begin{pgfscope}%
\pgfpathrectangle{\pgfqpoint{1.286132in}{0.839159in}}{\pgfqpoint{12.053712in}{5.967710in}}%
\pgfusepath{clip}%
\pgfsetbuttcap%
\pgfsetroundjoin%
\pgfsetlinewidth{1.505625pt}%
\definecolor{currentstroke}{rgb}{0.839216,0.152941,0.156863}%
\pgfsetstrokecolor{currentstroke}%
\pgfsetdash{}{0pt}%
\pgfpathmoveto{\pgfqpoint{5.265296in}{1.258496in}}%
\pgfpathlineto{\pgfqpoint{5.265296in}{1.286137in}}%
\pgfusepath{stroke}%
\end{pgfscope}%
\begin{pgfscope}%
\pgfpathrectangle{\pgfqpoint{1.286132in}{0.839159in}}{\pgfqpoint{12.053712in}{5.967710in}}%
\pgfusepath{clip}%
\pgfsetbuttcap%
\pgfsetroundjoin%
\pgfsetlinewidth{1.505625pt}%
\definecolor{currentstroke}{rgb}{0.839216,0.152941,0.156863}%
\pgfsetstrokecolor{currentstroke}%
\pgfsetdash{}{0pt}%
\pgfpathmoveto{\pgfqpoint{5.375982in}{1.274539in}}%
\pgfpathlineto{\pgfqpoint{5.375982in}{1.291168in}}%
\pgfusepath{stroke}%
\end{pgfscope}%
\begin{pgfscope}%
\pgfpathrectangle{\pgfqpoint{1.286132in}{0.839159in}}{\pgfqpoint{12.053712in}{5.967710in}}%
\pgfusepath{clip}%
\pgfsetbuttcap%
\pgfsetroundjoin%
\pgfsetlinewidth{1.505625pt}%
\definecolor{currentstroke}{rgb}{0.839216,0.152941,0.156863}%
\pgfsetstrokecolor{currentstroke}%
\pgfsetdash{}{0pt}%
\pgfpathmoveto{\pgfqpoint{5.486668in}{1.336576in}}%
\pgfpathlineto{\pgfqpoint{5.486668in}{1.418080in}}%
\pgfusepath{stroke}%
\end{pgfscope}%
\begin{pgfscope}%
\pgfpathrectangle{\pgfqpoint{1.286132in}{0.839159in}}{\pgfqpoint{12.053712in}{5.967710in}}%
\pgfusepath{clip}%
\pgfsetbuttcap%
\pgfsetroundjoin%
\pgfsetlinewidth{1.505625pt}%
\definecolor{currentstroke}{rgb}{0.839216,0.152941,0.156863}%
\pgfsetstrokecolor{currentstroke}%
\pgfsetdash{}{0pt}%
\pgfpathmoveto{\pgfqpoint{5.597354in}{1.321181in}}%
\pgfpathlineto{\pgfqpoint{5.597354in}{1.416869in}}%
\pgfusepath{stroke}%
\end{pgfscope}%
\begin{pgfscope}%
\pgfpathrectangle{\pgfqpoint{1.286132in}{0.839159in}}{\pgfqpoint{12.053712in}{5.967710in}}%
\pgfusepath{clip}%
\pgfsetbuttcap%
\pgfsetroundjoin%
\pgfsetlinewidth{1.505625pt}%
\definecolor{currentstroke}{rgb}{0.839216,0.152941,0.156863}%
\pgfsetstrokecolor{currentstroke}%
\pgfsetdash{}{0pt}%
\pgfpathmoveto{\pgfqpoint{5.708040in}{1.402435in}}%
\pgfpathlineto{\pgfqpoint{5.708040in}{1.696156in}}%
\pgfusepath{stroke}%
\end{pgfscope}%
\begin{pgfscope}%
\pgfpathrectangle{\pgfqpoint{1.286132in}{0.839159in}}{\pgfqpoint{12.053712in}{5.967710in}}%
\pgfusepath{clip}%
\pgfsetbuttcap%
\pgfsetroundjoin%
\pgfsetlinewidth{1.505625pt}%
\definecolor{currentstroke}{rgb}{0.839216,0.152941,0.156863}%
\pgfsetstrokecolor{currentstroke}%
\pgfsetdash{}{0pt}%
\pgfpathmoveto{\pgfqpoint{5.818726in}{1.360118in}}%
\pgfpathlineto{\pgfqpoint{5.818726in}{1.409444in}}%
\pgfusepath{stroke}%
\end{pgfscope}%
\begin{pgfscope}%
\pgfpathrectangle{\pgfqpoint{1.286132in}{0.839159in}}{\pgfqpoint{12.053712in}{5.967710in}}%
\pgfusepath{clip}%
\pgfsetbuttcap%
\pgfsetroundjoin%
\pgfsetlinewidth{1.505625pt}%
\definecolor{currentstroke}{rgb}{0.839216,0.152941,0.156863}%
\pgfsetstrokecolor{currentstroke}%
\pgfsetdash{}{0pt}%
\pgfpathmoveto{\pgfqpoint{5.929412in}{1.345382in}}%
\pgfpathlineto{\pgfqpoint{5.929412in}{1.402659in}}%
\pgfusepath{stroke}%
\end{pgfscope}%
\begin{pgfscope}%
\pgfpathrectangle{\pgfqpoint{1.286132in}{0.839159in}}{\pgfqpoint{12.053712in}{5.967710in}}%
\pgfusepath{clip}%
\pgfsetbuttcap%
\pgfsetroundjoin%
\pgfsetlinewidth{1.505625pt}%
\definecolor{currentstroke}{rgb}{0.839216,0.152941,0.156863}%
\pgfsetstrokecolor{currentstroke}%
\pgfsetdash{}{0pt}%
\pgfpathmoveto{\pgfqpoint{6.040098in}{1.373420in}}%
\pgfpathlineto{\pgfqpoint{6.040098in}{1.380971in}}%
\pgfusepath{stroke}%
\end{pgfscope}%
\begin{pgfscope}%
\pgfpathrectangle{\pgfqpoint{1.286132in}{0.839159in}}{\pgfqpoint{12.053712in}{5.967710in}}%
\pgfusepath{clip}%
\pgfsetbuttcap%
\pgfsetroundjoin%
\pgfsetlinewidth{1.505625pt}%
\definecolor{currentstroke}{rgb}{0.839216,0.152941,0.156863}%
\pgfsetstrokecolor{currentstroke}%
\pgfsetdash{}{0pt}%
\pgfpathmoveto{\pgfqpoint{6.150784in}{1.383963in}}%
\pgfpathlineto{\pgfqpoint{6.150784in}{1.398321in}}%
\pgfusepath{stroke}%
\end{pgfscope}%
\begin{pgfscope}%
\pgfpathrectangle{\pgfqpoint{1.286132in}{0.839159in}}{\pgfqpoint{12.053712in}{5.967710in}}%
\pgfusepath{clip}%
\pgfsetbuttcap%
\pgfsetroundjoin%
\pgfsetlinewidth{1.505625pt}%
\definecolor{currentstroke}{rgb}{0.839216,0.152941,0.156863}%
\pgfsetstrokecolor{currentstroke}%
\pgfsetdash{}{0pt}%
\pgfpathmoveto{\pgfqpoint{6.261470in}{1.385277in}}%
\pgfpathlineto{\pgfqpoint{6.261470in}{1.475212in}}%
\pgfusepath{stroke}%
\end{pgfscope}%
\begin{pgfscope}%
\pgfpathrectangle{\pgfqpoint{1.286132in}{0.839159in}}{\pgfqpoint{12.053712in}{5.967710in}}%
\pgfusepath{clip}%
\pgfsetbuttcap%
\pgfsetroundjoin%
\pgfsetlinewidth{1.505625pt}%
\definecolor{currentstroke}{rgb}{0.839216,0.152941,0.156863}%
\pgfsetstrokecolor{currentstroke}%
\pgfsetdash{}{0pt}%
\pgfpathmoveto{\pgfqpoint{6.372156in}{1.415129in}}%
\pgfpathlineto{\pgfqpoint{6.372156in}{1.429681in}}%
\pgfusepath{stroke}%
\end{pgfscope}%
\begin{pgfscope}%
\pgfpathrectangle{\pgfqpoint{1.286132in}{0.839159in}}{\pgfqpoint{12.053712in}{5.967710in}}%
\pgfusepath{clip}%
\pgfsetbuttcap%
\pgfsetroundjoin%
\pgfsetlinewidth{1.505625pt}%
\definecolor{currentstroke}{rgb}{0.839216,0.152941,0.156863}%
\pgfsetstrokecolor{currentstroke}%
\pgfsetdash{}{0pt}%
\pgfpathmoveto{\pgfqpoint{6.482842in}{1.437645in}}%
\pgfpathlineto{\pgfqpoint{6.482842in}{1.450411in}}%
\pgfusepath{stroke}%
\end{pgfscope}%
\begin{pgfscope}%
\pgfpathrectangle{\pgfqpoint{1.286132in}{0.839159in}}{\pgfqpoint{12.053712in}{5.967710in}}%
\pgfusepath{clip}%
\pgfsetbuttcap%
\pgfsetroundjoin%
\pgfsetlinewidth{1.505625pt}%
\definecolor{currentstroke}{rgb}{0.839216,0.152941,0.156863}%
\pgfsetstrokecolor{currentstroke}%
\pgfsetdash{}{0pt}%
\pgfpathmoveto{\pgfqpoint{6.593528in}{1.452296in}}%
\pgfpathlineto{\pgfqpoint{6.593528in}{1.472850in}}%
\pgfusepath{stroke}%
\end{pgfscope}%
\begin{pgfscope}%
\pgfpathrectangle{\pgfqpoint{1.286132in}{0.839159in}}{\pgfqpoint{12.053712in}{5.967710in}}%
\pgfusepath{clip}%
\pgfsetbuttcap%
\pgfsetroundjoin%
\pgfsetlinewidth{1.505625pt}%
\definecolor{currentstroke}{rgb}{0.839216,0.152941,0.156863}%
\pgfsetstrokecolor{currentstroke}%
\pgfsetdash{}{0pt}%
\pgfpathmoveto{\pgfqpoint{6.704214in}{1.480227in}}%
\pgfpathlineto{\pgfqpoint{6.704214in}{1.503186in}}%
\pgfusepath{stroke}%
\end{pgfscope}%
\begin{pgfscope}%
\pgfpathrectangle{\pgfqpoint{1.286132in}{0.839159in}}{\pgfqpoint{12.053712in}{5.967710in}}%
\pgfusepath{clip}%
\pgfsetbuttcap%
\pgfsetroundjoin%
\pgfsetlinewidth{1.505625pt}%
\definecolor{currentstroke}{rgb}{0.839216,0.152941,0.156863}%
\pgfsetstrokecolor{currentstroke}%
\pgfsetdash{}{0pt}%
\pgfpathmoveto{\pgfqpoint{6.814900in}{1.496970in}}%
\pgfpathlineto{\pgfqpoint{6.814900in}{1.509403in}}%
\pgfusepath{stroke}%
\end{pgfscope}%
\begin{pgfscope}%
\pgfpathrectangle{\pgfqpoint{1.286132in}{0.839159in}}{\pgfqpoint{12.053712in}{5.967710in}}%
\pgfusepath{clip}%
\pgfsetbuttcap%
\pgfsetroundjoin%
\pgfsetlinewidth{1.505625pt}%
\definecolor{currentstroke}{rgb}{0.839216,0.152941,0.156863}%
\pgfsetstrokecolor{currentstroke}%
\pgfsetdash{}{0pt}%
\pgfpathmoveto{\pgfqpoint{6.925586in}{1.521509in}}%
\pgfpathlineto{\pgfqpoint{6.925586in}{1.538148in}}%
\pgfusepath{stroke}%
\end{pgfscope}%
\begin{pgfscope}%
\pgfpathrectangle{\pgfqpoint{1.286132in}{0.839159in}}{\pgfqpoint{12.053712in}{5.967710in}}%
\pgfusepath{clip}%
\pgfsetbuttcap%
\pgfsetroundjoin%
\pgfsetlinewidth{1.505625pt}%
\definecolor{currentstroke}{rgb}{0.839216,0.152941,0.156863}%
\pgfsetstrokecolor{currentstroke}%
\pgfsetdash{}{0pt}%
\pgfpathmoveto{\pgfqpoint{7.036272in}{1.554254in}}%
\pgfpathlineto{\pgfqpoint{7.036272in}{1.560924in}}%
\pgfusepath{stroke}%
\end{pgfscope}%
\begin{pgfscope}%
\pgfpathrectangle{\pgfqpoint{1.286132in}{0.839159in}}{\pgfqpoint{12.053712in}{5.967710in}}%
\pgfusepath{clip}%
\pgfsetbuttcap%
\pgfsetroundjoin%
\pgfsetlinewidth{1.505625pt}%
\definecolor{currentstroke}{rgb}{0.839216,0.152941,0.156863}%
\pgfsetstrokecolor{currentstroke}%
\pgfsetdash{}{0pt}%
\pgfpathmoveto{\pgfqpoint{7.146959in}{1.580224in}}%
\pgfpathlineto{\pgfqpoint{7.146959in}{1.590414in}}%
\pgfusepath{stroke}%
\end{pgfscope}%
\begin{pgfscope}%
\pgfpathrectangle{\pgfqpoint{1.286132in}{0.839159in}}{\pgfqpoint{12.053712in}{5.967710in}}%
\pgfusepath{clip}%
\pgfsetbuttcap%
\pgfsetroundjoin%
\pgfsetlinewidth{1.505625pt}%
\definecolor{currentstroke}{rgb}{0.839216,0.152941,0.156863}%
\pgfsetstrokecolor{currentstroke}%
\pgfsetdash{}{0pt}%
\pgfpathmoveto{\pgfqpoint{7.257645in}{1.595010in}}%
\pgfpathlineto{\pgfqpoint{7.257645in}{1.628702in}}%
\pgfusepath{stroke}%
\end{pgfscope}%
\begin{pgfscope}%
\pgfpathrectangle{\pgfqpoint{1.286132in}{0.839159in}}{\pgfqpoint{12.053712in}{5.967710in}}%
\pgfusepath{clip}%
\pgfsetbuttcap%
\pgfsetroundjoin%
\pgfsetlinewidth{1.505625pt}%
\definecolor{currentstroke}{rgb}{0.839216,0.152941,0.156863}%
\pgfsetstrokecolor{currentstroke}%
\pgfsetdash{}{0pt}%
\pgfpathmoveto{\pgfqpoint{7.368331in}{1.625004in}}%
\pgfpathlineto{\pgfqpoint{7.368331in}{1.658314in}}%
\pgfusepath{stroke}%
\end{pgfscope}%
\begin{pgfscope}%
\pgfpathrectangle{\pgfqpoint{1.286132in}{0.839159in}}{\pgfqpoint{12.053712in}{5.967710in}}%
\pgfusepath{clip}%
\pgfsetbuttcap%
\pgfsetroundjoin%
\pgfsetlinewidth{1.505625pt}%
\definecolor{currentstroke}{rgb}{0.839216,0.152941,0.156863}%
\pgfsetstrokecolor{currentstroke}%
\pgfsetdash{}{0pt}%
\pgfpathmoveto{\pgfqpoint{7.479017in}{1.647175in}}%
\pgfpathlineto{\pgfqpoint{7.479017in}{1.670618in}}%
\pgfusepath{stroke}%
\end{pgfscope}%
\begin{pgfscope}%
\pgfpathrectangle{\pgfqpoint{1.286132in}{0.839159in}}{\pgfqpoint{12.053712in}{5.967710in}}%
\pgfusepath{clip}%
\pgfsetbuttcap%
\pgfsetroundjoin%
\pgfsetlinewidth{1.505625pt}%
\definecolor{currentstroke}{rgb}{0.839216,0.152941,0.156863}%
\pgfsetstrokecolor{currentstroke}%
\pgfsetdash{}{0pt}%
\pgfpathmoveto{\pgfqpoint{7.589703in}{1.701888in}}%
\pgfpathlineto{\pgfqpoint{7.589703in}{1.829252in}}%
\pgfusepath{stroke}%
\end{pgfscope}%
\begin{pgfscope}%
\pgfpathrectangle{\pgfqpoint{1.286132in}{0.839159in}}{\pgfqpoint{12.053712in}{5.967710in}}%
\pgfusepath{clip}%
\pgfsetbuttcap%
\pgfsetroundjoin%
\pgfsetlinewidth{1.505625pt}%
\definecolor{currentstroke}{rgb}{0.839216,0.152941,0.156863}%
\pgfsetstrokecolor{currentstroke}%
\pgfsetdash{}{0pt}%
\pgfpathmoveto{\pgfqpoint{7.700389in}{1.695673in}}%
\pgfpathlineto{\pgfqpoint{7.700389in}{1.808746in}}%
\pgfusepath{stroke}%
\end{pgfscope}%
\begin{pgfscope}%
\pgfpathrectangle{\pgfqpoint{1.286132in}{0.839159in}}{\pgfqpoint{12.053712in}{5.967710in}}%
\pgfusepath{clip}%
\pgfsetbuttcap%
\pgfsetroundjoin%
\pgfsetlinewidth{1.505625pt}%
\definecolor{currentstroke}{rgb}{0.839216,0.152941,0.156863}%
\pgfsetstrokecolor{currentstroke}%
\pgfsetdash{}{0pt}%
\pgfpathmoveto{\pgfqpoint{7.811075in}{1.742679in}}%
\pgfpathlineto{\pgfqpoint{7.811075in}{1.772150in}}%
\pgfusepath{stroke}%
\end{pgfscope}%
\begin{pgfscope}%
\pgfpathrectangle{\pgfqpoint{1.286132in}{0.839159in}}{\pgfqpoint{12.053712in}{5.967710in}}%
\pgfusepath{clip}%
\pgfsetbuttcap%
\pgfsetroundjoin%
\pgfsetlinewidth{1.505625pt}%
\definecolor{currentstroke}{rgb}{0.839216,0.152941,0.156863}%
\pgfsetstrokecolor{currentstroke}%
\pgfsetdash{}{0pt}%
\pgfpathmoveto{\pgfqpoint{7.921761in}{1.768578in}}%
\pgfpathlineto{\pgfqpoint{7.921761in}{1.815563in}}%
\pgfusepath{stroke}%
\end{pgfscope}%
\begin{pgfscope}%
\pgfpathrectangle{\pgfqpoint{1.286132in}{0.839159in}}{\pgfqpoint{12.053712in}{5.967710in}}%
\pgfusepath{clip}%
\pgfsetbuttcap%
\pgfsetroundjoin%
\pgfsetlinewidth{1.505625pt}%
\definecolor{currentstroke}{rgb}{0.839216,0.152941,0.156863}%
\pgfsetstrokecolor{currentstroke}%
\pgfsetdash{}{0pt}%
\pgfpathmoveto{\pgfqpoint{8.032447in}{1.817115in}}%
\pgfpathlineto{\pgfqpoint{8.032447in}{1.852057in}}%
\pgfusepath{stroke}%
\end{pgfscope}%
\begin{pgfscope}%
\pgfpathrectangle{\pgfqpoint{1.286132in}{0.839159in}}{\pgfqpoint{12.053712in}{5.967710in}}%
\pgfusepath{clip}%
\pgfsetbuttcap%
\pgfsetroundjoin%
\pgfsetlinewidth{1.505625pt}%
\definecolor{currentstroke}{rgb}{0.839216,0.152941,0.156863}%
\pgfsetstrokecolor{currentstroke}%
\pgfsetdash{}{0pt}%
\pgfpathmoveto{\pgfqpoint{8.143133in}{1.845118in}}%
\pgfpathlineto{\pgfqpoint{8.143133in}{1.897379in}}%
\pgfusepath{stroke}%
\end{pgfscope}%
\begin{pgfscope}%
\pgfpathrectangle{\pgfqpoint{1.286132in}{0.839159in}}{\pgfqpoint{12.053712in}{5.967710in}}%
\pgfusepath{clip}%
\pgfsetbuttcap%
\pgfsetroundjoin%
\pgfsetlinewidth{1.505625pt}%
\definecolor{currentstroke}{rgb}{0.839216,0.152941,0.156863}%
\pgfsetstrokecolor{currentstroke}%
\pgfsetdash{}{0pt}%
\pgfpathmoveto{\pgfqpoint{8.253819in}{1.890360in}}%
\pgfpathlineto{\pgfqpoint{8.253819in}{1.932486in}}%
\pgfusepath{stroke}%
\end{pgfscope}%
\begin{pgfscope}%
\pgfpathrectangle{\pgfqpoint{1.286132in}{0.839159in}}{\pgfqpoint{12.053712in}{5.967710in}}%
\pgfusepath{clip}%
\pgfsetbuttcap%
\pgfsetroundjoin%
\pgfsetlinewidth{1.505625pt}%
\definecolor{currentstroke}{rgb}{0.839216,0.152941,0.156863}%
\pgfsetstrokecolor{currentstroke}%
\pgfsetdash{}{0pt}%
\pgfpathmoveto{\pgfqpoint{8.364505in}{1.911748in}}%
\pgfpathlineto{\pgfqpoint{8.364505in}{1.959758in}}%
\pgfusepath{stroke}%
\end{pgfscope}%
\begin{pgfscope}%
\pgfpathrectangle{\pgfqpoint{1.286132in}{0.839159in}}{\pgfqpoint{12.053712in}{5.967710in}}%
\pgfusepath{clip}%
\pgfsetbuttcap%
\pgfsetroundjoin%
\pgfsetlinewidth{1.505625pt}%
\definecolor{currentstroke}{rgb}{0.839216,0.152941,0.156863}%
\pgfsetstrokecolor{currentstroke}%
\pgfsetdash{}{0pt}%
\pgfpathmoveto{\pgfqpoint{8.475191in}{1.956251in}}%
\pgfpathlineto{\pgfqpoint{8.475191in}{1.981050in}}%
\pgfusepath{stroke}%
\end{pgfscope}%
\begin{pgfscope}%
\pgfpathrectangle{\pgfqpoint{1.286132in}{0.839159in}}{\pgfqpoint{12.053712in}{5.967710in}}%
\pgfusepath{clip}%
\pgfsetbuttcap%
\pgfsetroundjoin%
\pgfsetlinewidth{1.505625pt}%
\definecolor{currentstroke}{rgb}{0.839216,0.152941,0.156863}%
\pgfsetstrokecolor{currentstroke}%
\pgfsetdash{}{0pt}%
\pgfpathmoveto{\pgfqpoint{8.585877in}{2.001324in}}%
\pgfpathlineto{\pgfqpoint{8.585877in}{2.034466in}}%
\pgfusepath{stroke}%
\end{pgfscope}%
\begin{pgfscope}%
\pgfpathrectangle{\pgfqpoint{1.286132in}{0.839159in}}{\pgfqpoint{12.053712in}{5.967710in}}%
\pgfusepath{clip}%
\pgfsetbuttcap%
\pgfsetroundjoin%
\pgfsetlinewidth{1.505625pt}%
\definecolor{currentstroke}{rgb}{0.839216,0.152941,0.156863}%
\pgfsetstrokecolor{currentstroke}%
\pgfsetdash{}{0pt}%
\pgfpathmoveto{\pgfqpoint{8.696563in}{2.030127in}}%
\pgfpathlineto{\pgfqpoint{8.696563in}{2.210386in}}%
\pgfusepath{stroke}%
\end{pgfscope}%
\begin{pgfscope}%
\pgfpathrectangle{\pgfqpoint{1.286132in}{0.839159in}}{\pgfqpoint{12.053712in}{5.967710in}}%
\pgfusepath{clip}%
\pgfsetbuttcap%
\pgfsetroundjoin%
\pgfsetlinewidth{1.505625pt}%
\definecolor{currentstroke}{rgb}{0.839216,0.152941,0.156863}%
\pgfsetstrokecolor{currentstroke}%
\pgfsetdash{}{0pt}%
\pgfpathmoveto{\pgfqpoint{8.807249in}{2.078620in}}%
\pgfpathlineto{\pgfqpoint{8.807249in}{2.124164in}}%
\pgfusepath{stroke}%
\end{pgfscope}%
\begin{pgfscope}%
\pgfpathrectangle{\pgfqpoint{1.286132in}{0.839159in}}{\pgfqpoint{12.053712in}{5.967710in}}%
\pgfusepath{clip}%
\pgfsetbuttcap%
\pgfsetroundjoin%
\pgfsetlinewidth{1.505625pt}%
\definecolor{currentstroke}{rgb}{0.839216,0.152941,0.156863}%
\pgfsetstrokecolor{currentstroke}%
\pgfsetdash{}{0pt}%
\pgfpathmoveto{\pgfqpoint{8.917936in}{2.135236in}}%
\pgfpathlineto{\pgfqpoint{8.917936in}{2.181117in}}%
\pgfusepath{stroke}%
\end{pgfscope}%
\begin{pgfscope}%
\pgfpathrectangle{\pgfqpoint{1.286132in}{0.839159in}}{\pgfqpoint{12.053712in}{5.967710in}}%
\pgfusepath{clip}%
\pgfsetbuttcap%
\pgfsetroundjoin%
\pgfsetlinewidth{1.505625pt}%
\definecolor{currentstroke}{rgb}{0.839216,0.152941,0.156863}%
\pgfsetstrokecolor{currentstroke}%
\pgfsetdash{}{0pt}%
\pgfpathmoveto{\pgfqpoint{9.028622in}{2.154366in}}%
\pgfpathlineto{\pgfqpoint{9.028622in}{2.246568in}}%
\pgfusepath{stroke}%
\end{pgfscope}%
\begin{pgfscope}%
\pgfpathrectangle{\pgfqpoint{1.286132in}{0.839159in}}{\pgfqpoint{12.053712in}{5.967710in}}%
\pgfusepath{clip}%
\pgfsetbuttcap%
\pgfsetroundjoin%
\pgfsetlinewidth{1.505625pt}%
\definecolor{currentstroke}{rgb}{0.839216,0.152941,0.156863}%
\pgfsetstrokecolor{currentstroke}%
\pgfsetdash{}{0pt}%
\pgfpathmoveto{\pgfqpoint{9.139308in}{2.221667in}}%
\pgfpathlineto{\pgfqpoint{9.139308in}{2.252513in}}%
\pgfusepath{stroke}%
\end{pgfscope}%
\begin{pgfscope}%
\pgfpathrectangle{\pgfqpoint{1.286132in}{0.839159in}}{\pgfqpoint{12.053712in}{5.967710in}}%
\pgfusepath{clip}%
\pgfsetbuttcap%
\pgfsetroundjoin%
\pgfsetlinewidth{1.505625pt}%
\definecolor{currentstroke}{rgb}{0.839216,0.152941,0.156863}%
\pgfsetstrokecolor{currentstroke}%
\pgfsetdash{}{0pt}%
\pgfpathmoveto{\pgfqpoint{9.249994in}{2.267448in}}%
\pgfpathlineto{\pgfqpoint{9.249994in}{2.314242in}}%
\pgfusepath{stroke}%
\end{pgfscope}%
\begin{pgfscope}%
\pgfpathrectangle{\pgfqpoint{1.286132in}{0.839159in}}{\pgfqpoint{12.053712in}{5.967710in}}%
\pgfusepath{clip}%
\pgfsetbuttcap%
\pgfsetroundjoin%
\pgfsetlinewidth{1.505625pt}%
\definecolor{currentstroke}{rgb}{0.839216,0.152941,0.156863}%
\pgfsetstrokecolor{currentstroke}%
\pgfsetdash{}{0pt}%
\pgfpathmoveto{\pgfqpoint{9.360680in}{2.313718in}}%
\pgfpathlineto{\pgfqpoint{9.360680in}{2.335316in}}%
\pgfusepath{stroke}%
\end{pgfscope}%
\begin{pgfscope}%
\pgfpathrectangle{\pgfqpoint{1.286132in}{0.839159in}}{\pgfqpoint{12.053712in}{5.967710in}}%
\pgfusepath{clip}%
\pgfsetbuttcap%
\pgfsetroundjoin%
\pgfsetlinewidth{1.505625pt}%
\definecolor{currentstroke}{rgb}{0.839216,0.152941,0.156863}%
\pgfsetstrokecolor{currentstroke}%
\pgfsetdash{}{0pt}%
\pgfpathmoveto{\pgfqpoint{9.471366in}{2.272395in}}%
\pgfpathlineto{\pgfqpoint{9.471366in}{2.767838in}}%
\pgfusepath{stroke}%
\end{pgfscope}%
\begin{pgfscope}%
\pgfpathrectangle{\pgfqpoint{1.286132in}{0.839159in}}{\pgfqpoint{12.053712in}{5.967710in}}%
\pgfusepath{clip}%
\pgfsetbuttcap%
\pgfsetroundjoin%
\pgfsetlinewidth{1.505625pt}%
\definecolor{currentstroke}{rgb}{0.839216,0.152941,0.156863}%
\pgfsetstrokecolor{currentstroke}%
\pgfsetdash{}{0pt}%
\pgfpathmoveto{\pgfqpoint{9.582052in}{2.419436in}}%
\pgfpathlineto{\pgfqpoint{9.582052in}{2.509065in}}%
\pgfusepath{stroke}%
\end{pgfscope}%
\begin{pgfscope}%
\pgfpathrectangle{\pgfqpoint{1.286132in}{0.839159in}}{\pgfqpoint{12.053712in}{5.967710in}}%
\pgfusepath{clip}%
\pgfsetbuttcap%
\pgfsetroundjoin%
\pgfsetlinewidth{1.505625pt}%
\definecolor{currentstroke}{rgb}{0.839216,0.152941,0.156863}%
\pgfsetstrokecolor{currentstroke}%
\pgfsetdash{}{0pt}%
\pgfpathmoveto{\pgfqpoint{9.692738in}{2.475772in}}%
\pgfpathlineto{\pgfqpoint{9.692738in}{2.492016in}}%
\pgfusepath{stroke}%
\end{pgfscope}%
\begin{pgfscope}%
\pgfpathrectangle{\pgfqpoint{1.286132in}{0.839159in}}{\pgfqpoint{12.053712in}{5.967710in}}%
\pgfusepath{clip}%
\pgfsetbuttcap%
\pgfsetroundjoin%
\pgfsetlinewidth{1.505625pt}%
\definecolor{currentstroke}{rgb}{0.839216,0.152941,0.156863}%
\pgfsetstrokecolor{currentstroke}%
\pgfsetdash{}{0pt}%
\pgfpathmoveto{\pgfqpoint{9.803424in}{2.540212in}}%
\pgfpathlineto{\pgfqpoint{9.803424in}{2.570627in}}%
\pgfusepath{stroke}%
\end{pgfscope}%
\begin{pgfscope}%
\pgfpathrectangle{\pgfqpoint{1.286132in}{0.839159in}}{\pgfqpoint{12.053712in}{5.967710in}}%
\pgfusepath{clip}%
\pgfsetbuttcap%
\pgfsetroundjoin%
\pgfsetlinewidth{1.505625pt}%
\definecolor{currentstroke}{rgb}{0.839216,0.152941,0.156863}%
\pgfsetstrokecolor{currentstroke}%
\pgfsetdash{}{0pt}%
\pgfpathmoveto{\pgfqpoint{9.914110in}{2.586749in}}%
\pgfpathlineto{\pgfqpoint{9.914110in}{2.673469in}}%
\pgfusepath{stroke}%
\end{pgfscope}%
\begin{pgfscope}%
\pgfpathrectangle{\pgfqpoint{1.286132in}{0.839159in}}{\pgfqpoint{12.053712in}{5.967710in}}%
\pgfusepath{clip}%
\pgfsetbuttcap%
\pgfsetroundjoin%
\pgfsetlinewidth{1.505625pt}%
\definecolor{currentstroke}{rgb}{0.839216,0.152941,0.156863}%
\pgfsetstrokecolor{currentstroke}%
\pgfsetdash{}{0pt}%
\pgfpathmoveto{\pgfqpoint{10.024796in}{2.633433in}}%
\pgfpathlineto{\pgfqpoint{10.024796in}{2.975468in}}%
\pgfusepath{stroke}%
\end{pgfscope}%
\begin{pgfscope}%
\pgfpathrectangle{\pgfqpoint{1.286132in}{0.839159in}}{\pgfqpoint{12.053712in}{5.967710in}}%
\pgfusepath{clip}%
\pgfsetbuttcap%
\pgfsetroundjoin%
\pgfsetlinewidth{1.505625pt}%
\definecolor{currentstroke}{rgb}{0.839216,0.152941,0.156863}%
\pgfsetstrokecolor{currentstroke}%
\pgfsetdash{}{0pt}%
\pgfpathmoveto{\pgfqpoint{10.135482in}{2.713035in}}%
\pgfpathlineto{\pgfqpoint{10.135482in}{2.785755in}}%
\pgfusepath{stroke}%
\end{pgfscope}%
\begin{pgfscope}%
\pgfpathrectangle{\pgfqpoint{1.286132in}{0.839159in}}{\pgfqpoint{12.053712in}{5.967710in}}%
\pgfusepath{clip}%
\pgfsetbuttcap%
\pgfsetroundjoin%
\pgfsetlinewidth{1.505625pt}%
\definecolor{currentstroke}{rgb}{0.839216,0.152941,0.156863}%
\pgfsetstrokecolor{currentstroke}%
\pgfsetdash{}{0pt}%
\pgfpathmoveto{\pgfqpoint{10.246168in}{2.773316in}}%
\pgfpathlineto{\pgfqpoint{10.246168in}{2.784215in}}%
\pgfusepath{stroke}%
\end{pgfscope}%
\begin{pgfscope}%
\pgfpathrectangle{\pgfqpoint{1.286132in}{0.839159in}}{\pgfqpoint{12.053712in}{5.967710in}}%
\pgfusepath{clip}%
\pgfsetbuttcap%
\pgfsetroundjoin%
\pgfsetlinewidth{1.505625pt}%
\definecolor{currentstroke}{rgb}{0.839216,0.152941,0.156863}%
\pgfsetstrokecolor{currentstroke}%
\pgfsetdash{}{0pt}%
\pgfpathmoveto{\pgfqpoint{10.356854in}{2.855658in}}%
\pgfpathlineto{\pgfqpoint{10.356854in}{2.919681in}}%
\pgfusepath{stroke}%
\end{pgfscope}%
\begin{pgfscope}%
\pgfpathrectangle{\pgfqpoint{1.286132in}{0.839159in}}{\pgfqpoint{12.053712in}{5.967710in}}%
\pgfusepath{clip}%
\pgfsetbuttcap%
\pgfsetroundjoin%
\pgfsetlinewidth{1.505625pt}%
\definecolor{currentstroke}{rgb}{0.839216,0.152941,0.156863}%
\pgfsetstrokecolor{currentstroke}%
\pgfsetdash{}{0pt}%
\pgfpathmoveto{\pgfqpoint{10.467540in}{2.873156in}}%
\pgfpathlineto{\pgfqpoint{10.467540in}{3.083564in}}%
\pgfusepath{stroke}%
\end{pgfscope}%
\begin{pgfscope}%
\pgfpathrectangle{\pgfqpoint{1.286132in}{0.839159in}}{\pgfqpoint{12.053712in}{5.967710in}}%
\pgfusepath{clip}%
\pgfsetbuttcap%
\pgfsetroundjoin%
\pgfsetlinewidth{1.505625pt}%
\definecolor{currentstroke}{rgb}{0.839216,0.152941,0.156863}%
\pgfsetstrokecolor{currentstroke}%
\pgfsetdash{}{0pt}%
\pgfpathmoveto{\pgfqpoint{10.578226in}{2.961547in}}%
\pgfpathlineto{\pgfqpoint{10.578226in}{3.046913in}}%
\pgfusepath{stroke}%
\end{pgfscope}%
\begin{pgfscope}%
\pgfpathrectangle{\pgfqpoint{1.286132in}{0.839159in}}{\pgfqpoint{12.053712in}{5.967710in}}%
\pgfusepath{clip}%
\pgfsetbuttcap%
\pgfsetroundjoin%
\pgfsetlinewidth{1.505625pt}%
\definecolor{currentstroke}{rgb}{0.839216,0.152941,0.156863}%
\pgfsetstrokecolor{currentstroke}%
\pgfsetdash{}{0pt}%
\pgfpathmoveto{\pgfqpoint{10.688913in}{3.046757in}}%
\pgfpathlineto{\pgfqpoint{10.688913in}{3.111364in}}%
\pgfusepath{stroke}%
\end{pgfscope}%
\begin{pgfscope}%
\pgfpathrectangle{\pgfqpoint{1.286132in}{0.839159in}}{\pgfqpoint{12.053712in}{5.967710in}}%
\pgfusepath{clip}%
\pgfsetbuttcap%
\pgfsetroundjoin%
\pgfsetlinewidth{1.505625pt}%
\definecolor{currentstroke}{rgb}{0.839216,0.152941,0.156863}%
\pgfsetstrokecolor{currentstroke}%
\pgfsetdash{}{0pt}%
\pgfpathmoveto{\pgfqpoint{10.799599in}{3.104412in}}%
\pgfpathlineto{\pgfqpoint{10.799599in}{3.170338in}}%
\pgfusepath{stroke}%
\end{pgfscope}%
\begin{pgfscope}%
\pgfpathrectangle{\pgfqpoint{1.286132in}{0.839159in}}{\pgfqpoint{12.053712in}{5.967710in}}%
\pgfusepath{clip}%
\pgfsetbuttcap%
\pgfsetroundjoin%
\pgfsetlinewidth{1.505625pt}%
\definecolor{currentstroke}{rgb}{0.839216,0.152941,0.156863}%
\pgfsetstrokecolor{currentstroke}%
\pgfsetdash{}{0pt}%
\pgfpathmoveto{\pgfqpoint{10.910285in}{3.180987in}}%
\pgfpathlineto{\pgfqpoint{10.910285in}{3.392283in}}%
\pgfusepath{stroke}%
\end{pgfscope}%
\begin{pgfscope}%
\pgfpathrectangle{\pgfqpoint{1.286132in}{0.839159in}}{\pgfqpoint{12.053712in}{5.967710in}}%
\pgfusepath{clip}%
\pgfsetbuttcap%
\pgfsetroundjoin%
\pgfsetlinewidth{1.505625pt}%
\definecolor{currentstroke}{rgb}{0.839216,0.152941,0.156863}%
\pgfsetstrokecolor{currentstroke}%
\pgfsetdash{}{0pt}%
\pgfpathmoveto{\pgfqpoint{11.020971in}{3.276532in}}%
\pgfpathlineto{\pgfqpoint{11.020971in}{3.351077in}}%
\pgfusepath{stroke}%
\end{pgfscope}%
\begin{pgfscope}%
\pgfpathrectangle{\pgfqpoint{1.286132in}{0.839159in}}{\pgfqpoint{12.053712in}{5.967710in}}%
\pgfusepath{clip}%
\pgfsetbuttcap%
\pgfsetroundjoin%
\pgfsetlinewidth{1.505625pt}%
\definecolor{currentstroke}{rgb}{0.839216,0.152941,0.156863}%
\pgfsetstrokecolor{currentstroke}%
\pgfsetdash{}{0pt}%
\pgfpathmoveto{\pgfqpoint{11.131657in}{3.312430in}}%
\pgfpathlineto{\pgfqpoint{11.131657in}{3.414438in}}%
\pgfusepath{stroke}%
\end{pgfscope}%
\begin{pgfscope}%
\pgfpathrectangle{\pgfqpoint{1.286132in}{0.839159in}}{\pgfqpoint{12.053712in}{5.967710in}}%
\pgfusepath{clip}%
\pgfsetbuttcap%
\pgfsetroundjoin%
\pgfsetlinewidth{1.505625pt}%
\definecolor{currentstroke}{rgb}{0.839216,0.152941,0.156863}%
\pgfsetstrokecolor{currentstroke}%
\pgfsetdash{}{0pt}%
\pgfpathmoveto{\pgfqpoint{11.242343in}{3.414080in}}%
\pgfpathlineto{\pgfqpoint{11.242343in}{3.517814in}}%
\pgfusepath{stroke}%
\end{pgfscope}%
\begin{pgfscope}%
\pgfpathrectangle{\pgfqpoint{1.286132in}{0.839159in}}{\pgfqpoint{12.053712in}{5.967710in}}%
\pgfusepath{clip}%
\pgfsetbuttcap%
\pgfsetroundjoin%
\pgfsetlinewidth{1.505625pt}%
\definecolor{currentstroke}{rgb}{0.839216,0.152941,0.156863}%
\pgfsetstrokecolor{currentstroke}%
\pgfsetdash{}{0pt}%
\pgfpathmoveto{\pgfqpoint{11.353029in}{3.526390in}}%
\pgfpathlineto{\pgfqpoint{11.353029in}{3.644561in}}%
\pgfusepath{stroke}%
\end{pgfscope}%
\begin{pgfscope}%
\pgfpathrectangle{\pgfqpoint{1.286132in}{0.839159in}}{\pgfqpoint{12.053712in}{5.967710in}}%
\pgfusepath{clip}%
\pgfsetbuttcap%
\pgfsetroundjoin%
\pgfsetlinewidth{1.505625pt}%
\definecolor{currentstroke}{rgb}{0.839216,0.152941,0.156863}%
\pgfsetstrokecolor{currentstroke}%
\pgfsetdash{}{0pt}%
\pgfpathmoveto{\pgfqpoint{11.463715in}{3.566105in}}%
\pgfpathlineto{\pgfqpoint{11.463715in}{3.611593in}}%
\pgfusepath{stroke}%
\end{pgfscope}%
\begin{pgfscope}%
\pgfpathrectangle{\pgfqpoint{1.286132in}{0.839159in}}{\pgfqpoint{12.053712in}{5.967710in}}%
\pgfusepath{clip}%
\pgfsetbuttcap%
\pgfsetroundjoin%
\pgfsetlinewidth{1.505625pt}%
\definecolor{currentstroke}{rgb}{0.839216,0.152941,0.156863}%
\pgfsetstrokecolor{currentstroke}%
\pgfsetdash{}{0pt}%
\pgfpathmoveto{\pgfqpoint{11.574401in}{3.684942in}}%
\pgfpathlineto{\pgfqpoint{11.574401in}{3.814557in}}%
\pgfusepath{stroke}%
\end{pgfscope}%
\begin{pgfscope}%
\pgfpathrectangle{\pgfqpoint{1.286132in}{0.839159in}}{\pgfqpoint{12.053712in}{5.967710in}}%
\pgfusepath{clip}%
\pgfsetbuttcap%
\pgfsetroundjoin%
\pgfsetlinewidth{1.505625pt}%
\definecolor{currentstroke}{rgb}{0.839216,0.152941,0.156863}%
\pgfsetstrokecolor{currentstroke}%
\pgfsetdash{}{0pt}%
\pgfpathmoveto{\pgfqpoint{11.685087in}{3.688803in}}%
\pgfpathlineto{\pgfqpoint{11.685087in}{4.068778in}}%
\pgfusepath{stroke}%
\end{pgfscope}%
\begin{pgfscope}%
\pgfpathrectangle{\pgfqpoint{1.286132in}{0.839159in}}{\pgfqpoint{12.053712in}{5.967710in}}%
\pgfusepath{clip}%
\pgfsetbuttcap%
\pgfsetroundjoin%
\pgfsetlinewidth{1.505625pt}%
\definecolor{currentstroke}{rgb}{0.839216,0.152941,0.156863}%
\pgfsetstrokecolor{currentstroke}%
\pgfsetdash{}{0pt}%
\pgfpathmoveto{\pgfqpoint{11.795773in}{3.851340in}}%
\pgfpathlineto{\pgfqpoint{11.795773in}{4.085526in}}%
\pgfusepath{stroke}%
\end{pgfscope}%
\begin{pgfscope}%
\pgfpathrectangle{\pgfqpoint{1.286132in}{0.839159in}}{\pgfqpoint{12.053712in}{5.967710in}}%
\pgfusepath{clip}%
\pgfsetbuttcap%
\pgfsetroundjoin%
\pgfsetlinewidth{1.505625pt}%
\definecolor{currentstroke}{rgb}{0.839216,0.152941,0.156863}%
\pgfsetstrokecolor{currentstroke}%
\pgfsetdash{}{0pt}%
\pgfpathmoveto{\pgfqpoint{11.906459in}{3.929013in}}%
\pgfpathlineto{\pgfqpoint{11.906459in}{4.035065in}}%
\pgfusepath{stroke}%
\end{pgfscope}%
\begin{pgfscope}%
\pgfpathrectangle{\pgfqpoint{1.286132in}{0.839159in}}{\pgfqpoint{12.053712in}{5.967710in}}%
\pgfusepath{clip}%
\pgfsetbuttcap%
\pgfsetroundjoin%
\pgfsetlinewidth{1.505625pt}%
\definecolor{currentstroke}{rgb}{0.839216,0.152941,0.156863}%
\pgfsetstrokecolor{currentstroke}%
\pgfsetdash{}{0pt}%
\pgfpathmoveto{\pgfqpoint{12.017145in}{4.053357in}}%
\pgfpathlineto{\pgfqpoint{12.017145in}{4.269708in}}%
\pgfusepath{stroke}%
\end{pgfscope}%
\begin{pgfscope}%
\pgfpathrectangle{\pgfqpoint{1.286132in}{0.839159in}}{\pgfqpoint{12.053712in}{5.967710in}}%
\pgfusepath{clip}%
\pgfsetbuttcap%
\pgfsetroundjoin%
\pgfsetlinewidth{1.505625pt}%
\definecolor{currentstroke}{rgb}{0.839216,0.152941,0.156863}%
\pgfsetstrokecolor{currentstroke}%
\pgfsetdash{}{0pt}%
\pgfpathmoveto{\pgfqpoint{12.127831in}{4.110327in}}%
\pgfpathlineto{\pgfqpoint{12.127831in}{4.248848in}}%
\pgfusepath{stroke}%
\end{pgfscope}%
\begin{pgfscope}%
\pgfpathrectangle{\pgfqpoint{1.286132in}{0.839159in}}{\pgfqpoint{12.053712in}{5.967710in}}%
\pgfusepath{clip}%
\pgfsetbuttcap%
\pgfsetroundjoin%
\pgfsetlinewidth{1.505625pt}%
\definecolor{currentstroke}{rgb}{0.839216,0.152941,0.156863}%
\pgfsetstrokecolor{currentstroke}%
\pgfsetdash{}{0pt}%
\pgfpathmoveto{\pgfqpoint{12.238517in}{4.230087in}}%
\pgfpathlineto{\pgfqpoint{12.238517in}{4.436998in}}%
\pgfusepath{stroke}%
\end{pgfscope}%
\begin{pgfscope}%
\pgfpathrectangle{\pgfqpoint{1.286132in}{0.839159in}}{\pgfqpoint{12.053712in}{5.967710in}}%
\pgfusepath{clip}%
\pgfsetbuttcap%
\pgfsetroundjoin%
\pgfsetlinewidth{1.505625pt}%
\definecolor{currentstroke}{rgb}{0.839216,0.152941,0.156863}%
\pgfsetstrokecolor{currentstroke}%
\pgfsetdash{}{0pt}%
\pgfpathmoveto{\pgfqpoint{12.349203in}{4.300488in}}%
\pgfpathlineto{\pgfqpoint{12.349203in}{4.611727in}}%
\pgfusepath{stroke}%
\end{pgfscope}%
\begin{pgfscope}%
\pgfpathrectangle{\pgfqpoint{1.286132in}{0.839159in}}{\pgfqpoint{12.053712in}{5.967710in}}%
\pgfusepath{clip}%
\pgfsetbuttcap%
\pgfsetroundjoin%
\pgfsetlinewidth{1.505625pt}%
\definecolor{currentstroke}{rgb}{0.839216,0.152941,0.156863}%
\pgfsetstrokecolor{currentstroke}%
\pgfsetdash{}{0pt}%
\pgfpathmoveto{\pgfqpoint{12.459890in}{4.400985in}}%
\pgfpathlineto{\pgfqpoint{12.459890in}{4.684413in}}%
\pgfusepath{stroke}%
\end{pgfscope}%
\begin{pgfscope}%
\pgfpathrectangle{\pgfqpoint{1.286132in}{0.839159in}}{\pgfqpoint{12.053712in}{5.967710in}}%
\pgfusepath{clip}%
\pgfsetbuttcap%
\pgfsetroundjoin%
\pgfsetlinewidth{1.505625pt}%
\definecolor{currentstroke}{rgb}{0.839216,0.152941,0.156863}%
\pgfsetstrokecolor{currentstroke}%
\pgfsetdash{}{0pt}%
\pgfpathmoveto{\pgfqpoint{12.570576in}{4.542318in}}%
\pgfpathlineto{\pgfqpoint{12.570576in}{4.660663in}}%
\pgfusepath{stroke}%
\end{pgfscope}%
\begin{pgfscope}%
\pgfpathrectangle{\pgfqpoint{1.286132in}{0.839159in}}{\pgfqpoint{12.053712in}{5.967710in}}%
\pgfusepath{clip}%
\pgfsetbuttcap%
\pgfsetroundjoin%
\pgfsetlinewidth{1.505625pt}%
\definecolor{currentstroke}{rgb}{0.839216,0.152941,0.156863}%
\pgfsetstrokecolor{currentstroke}%
\pgfsetdash{}{0pt}%
\pgfpathmoveto{\pgfqpoint{12.681262in}{4.611754in}}%
\pgfpathlineto{\pgfqpoint{12.681262in}{4.677434in}}%
\pgfusepath{stroke}%
\end{pgfscope}%
\begin{pgfscope}%
\pgfpathrectangle{\pgfqpoint{1.286132in}{0.839159in}}{\pgfqpoint{12.053712in}{5.967710in}}%
\pgfusepath{clip}%
\pgfsetbuttcap%
\pgfsetroundjoin%
\pgfsetlinewidth{1.505625pt}%
\definecolor{currentstroke}{rgb}{0.839216,0.152941,0.156863}%
\pgfsetstrokecolor{currentstroke}%
\pgfsetdash{}{0pt}%
\pgfpathmoveto{\pgfqpoint{12.791948in}{4.756826in}}%
\pgfpathlineto{\pgfqpoint{12.791948in}{4.988912in}}%
\pgfusepath{stroke}%
\end{pgfscope}%
\begin{pgfscope}%
\pgfpathrectangle{\pgfqpoint{1.286132in}{0.839159in}}{\pgfqpoint{12.053712in}{5.967710in}}%
\pgfusepath{clip}%
\pgfsetbuttcap%
\pgfsetroundjoin%
\pgfsetlinewidth{1.505625pt}%
\definecolor{currentstroke}{rgb}{0.580392,0.403922,0.741176}%
\pgfsetstrokecolor{currentstroke}%
\pgfsetdash{}{0pt}%
\pgfpathmoveto{\pgfqpoint{1.834028in}{1.119277in}}%
\pgfpathlineto{\pgfqpoint{1.834028in}{1.119299in}}%
\pgfusepath{stroke}%
\end{pgfscope}%
\begin{pgfscope}%
\pgfpathrectangle{\pgfqpoint{1.286132in}{0.839159in}}{\pgfqpoint{12.053712in}{5.967710in}}%
\pgfusepath{clip}%
\pgfsetbuttcap%
\pgfsetroundjoin%
\pgfsetlinewidth{1.505625pt}%
\definecolor{currentstroke}{rgb}{0.580392,0.403922,0.741176}%
\pgfsetstrokecolor{currentstroke}%
\pgfsetdash{}{0pt}%
\pgfpathmoveto{\pgfqpoint{1.944714in}{1.119249in}}%
\pgfpathlineto{\pgfqpoint{1.944714in}{1.119264in}}%
\pgfusepath{stroke}%
\end{pgfscope}%
\begin{pgfscope}%
\pgfpathrectangle{\pgfqpoint{1.286132in}{0.839159in}}{\pgfqpoint{12.053712in}{5.967710in}}%
\pgfusepath{clip}%
\pgfsetbuttcap%
\pgfsetroundjoin%
\pgfsetlinewidth{1.505625pt}%
\definecolor{currentstroke}{rgb}{0.580392,0.403922,0.741176}%
\pgfsetstrokecolor{currentstroke}%
\pgfsetdash{}{0pt}%
\pgfpathmoveto{\pgfqpoint{2.055400in}{1.119323in}}%
\pgfpathlineto{\pgfqpoint{2.055400in}{1.119330in}}%
\pgfusepath{stroke}%
\end{pgfscope}%
\begin{pgfscope}%
\pgfpathrectangle{\pgfqpoint{1.286132in}{0.839159in}}{\pgfqpoint{12.053712in}{5.967710in}}%
\pgfusepath{clip}%
\pgfsetbuttcap%
\pgfsetroundjoin%
\pgfsetlinewidth{1.505625pt}%
\definecolor{currentstroke}{rgb}{0.580392,0.403922,0.741176}%
\pgfsetstrokecolor{currentstroke}%
\pgfsetdash{}{0pt}%
\pgfpathmoveto{\pgfqpoint{2.166086in}{1.119333in}}%
\pgfpathlineto{\pgfqpoint{2.166086in}{1.119442in}}%
\pgfusepath{stroke}%
\end{pgfscope}%
\begin{pgfscope}%
\pgfpathrectangle{\pgfqpoint{1.286132in}{0.839159in}}{\pgfqpoint{12.053712in}{5.967710in}}%
\pgfusepath{clip}%
\pgfsetbuttcap%
\pgfsetroundjoin%
\pgfsetlinewidth{1.505625pt}%
\definecolor{currentstroke}{rgb}{0.580392,0.403922,0.741176}%
\pgfsetstrokecolor{currentstroke}%
\pgfsetdash{}{0pt}%
\pgfpathmoveto{\pgfqpoint{2.276772in}{1.119510in}}%
\pgfpathlineto{\pgfqpoint{2.276772in}{1.119630in}}%
\pgfusepath{stroke}%
\end{pgfscope}%
\begin{pgfscope}%
\pgfpathrectangle{\pgfqpoint{1.286132in}{0.839159in}}{\pgfqpoint{12.053712in}{5.967710in}}%
\pgfusepath{clip}%
\pgfsetbuttcap%
\pgfsetroundjoin%
\pgfsetlinewidth{1.505625pt}%
\definecolor{currentstroke}{rgb}{0.580392,0.403922,0.741176}%
\pgfsetstrokecolor{currentstroke}%
\pgfsetdash{}{0pt}%
\pgfpathmoveto{\pgfqpoint{2.387458in}{1.119510in}}%
\pgfpathlineto{\pgfqpoint{2.387458in}{1.119624in}}%
\pgfusepath{stroke}%
\end{pgfscope}%
\begin{pgfscope}%
\pgfpathrectangle{\pgfqpoint{1.286132in}{0.839159in}}{\pgfqpoint{12.053712in}{5.967710in}}%
\pgfusepath{clip}%
\pgfsetbuttcap%
\pgfsetroundjoin%
\pgfsetlinewidth{1.505625pt}%
\definecolor{currentstroke}{rgb}{0.580392,0.403922,0.741176}%
\pgfsetstrokecolor{currentstroke}%
\pgfsetdash{}{0pt}%
\pgfpathmoveto{\pgfqpoint{2.498144in}{1.119542in}}%
\pgfpathlineto{\pgfqpoint{2.498144in}{1.119712in}}%
\pgfusepath{stroke}%
\end{pgfscope}%
\begin{pgfscope}%
\pgfpathrectangle{\pgfqpoint{1.286132in}{0.839159in}}{\pgfqpoint{12.053712in}{5.967710in}}%
\pgfusepath{clip}%
\pgfsetbuttcap%
\pgfsetroundjoin%
\pgfsetlinewidth{1.505625pt}%
\definecolor{currentstroke}{rgb}{0.580392,0.403922,0.741176}%
\pgfsetstrokecolor{currentstroke}%
\pgfsetdash{}{0pt}%
\pgfpathmoveto{\pgfqpoint{2.608830in}{1.119643in}}%
\pgfpathlineto{\pgfqpoint{2.608830in}{1.119826in}}%
\pgfusepath{stroke}%
\end{pgfscope}%
\begin{pgfscope}%
\pgfpathrectangle{\pgfqpoint{1.286132in}{0.839159in}}{\pgfqpoint{12.053712in}{5.967710in}}%
\pgfusepath{clip}%
\pgfsetbuttcap%
\pgfsetroundjoin%
\pgfsetlinewidth{1.505625pt}%
\definecolor{currentstroke}{rgb}{0.580392,0.403922,0.741176}%
\pgfsetstrokecolor{currentstroke}%
\pgfsetdash{}{0pt}%
\pgfpathmoveto{\pgfqpoint{2.719516in}{1.119588in}}%
\pgfpathlineto{\pgfqpoint{2.719516in}{1.119969in}}%
\pgfusepath{stroke}%
\end{pgfscope}%
\begin{pgfscope}%
\pgfpathrectangle{\pgfqpoint{1.286132in}{0.839159in}}{\pgfqpoint{12.053712in}{5.967710in}}%
\pgfusepath{clip}%
\pgfsetbuttcap%
\pgfsetroundjoin%
\pgfsetlinewidth{1.505625pt}%
\definecolor{currentstroke}{rgb}{0.580392,0.403922,0.741176}%
\pgfsetstrokecolor{currentstroke}%
\pgfsetdash{}{0pt}%
\pgfpathmoveto{\pgfqpoint{2.830202in}{1.119687in}}%
\pgfpathlineto{\pgfqpoint{2.830202in}{1.120184in}}%
\pgfusepath{stroke}%
\end{pgfscope}%
\begin{pgfscope}%
\pgfpathrectangle{\pgfqpoint{1.286132in}{0.839159in}}{\pgfqpoint{12.053712in}{5.967710in}}%
\pgfusepath{clip}%
\pgfsetbuttcap%
\pgfsetroundjoin%
\pgfsetlinewidth{1.505625pt}%
\definecolor{currentstroke}{rgb}{0.580392,0.403922,0.741176}%
\pgfsetstrokecolor{currentstroke}%
\pgfsetdash{}{0pt}%
\pgfpathmoveto{\pgfqpoint{2.940888in}{1.119788in}}%
\pgfpathlineto{\pgfqpoint{2.940888in}{1.120022in}}%
\pgfusepath{stroke}%
\end{pgfscope}%
\begin{pgfscope}%
\pgfpathrectangle{\pgfqpoint{1.286132in}{0.839159in}}{\pgfqpoint{12.053712in}{5.967710in}}%
\pgfusepath{clip}%
\pgfsetbuttcap%
\pgfsetroundjoin%
\pgfsetlinewidth{1.505625pt}%
\definecolor{currentstroke}{rgb}{0.580392,0.403922,0.741176}%
\pgfsetstrokecolor{currentstroke}%
\pgfsetdash{}{0pt}%
\pgfpathmoveto{\pgfqpoint{3.051574in}{1.120077in}}%
\pgfpathlineto{\pgfqpoint{3.051574in}{1.120741in}}%
\pgfusepath{stroke}%
\end{pgfscope}%
\begin{pgfscope}%
\pgfpathrectangle{\pgfqpoint{1.286132in}{0.839159in}}{\pgfqpoint{12.053712in}{5.967710in}}%
\pgfusepath{clip}%
\pgfsetbuttcap%
\pgfsetroundjoin%
\pgfsetlinewidth{1.505625pt}%
\definecolor{currentstroke}{rgb}{0.580392,0.403922,0.741176}%
\pgfsetstrokecolor{currentstroke}%
\pgfsetdash{}{0pt}%
\pgfpathmoveto{\pgfqpoint{3.162260in}{1.119892in}}%
\pgfpathlineto{\pgfqpoint{3.162260in}{1.120296in}}%
\pgfusepath{stroke}%
\end{pgfscope}%
\begin{pgfscope}%
\pgfpathrectangle{\pgfqpoint{1.286132in}{0.839159in}}{\pgfqpoint{12.053712in}{5.967710in}}%
\pgfusepath{clip}%
\pgfsetbuttcap%
\pgfsetroundjoin%
\pgfsetlinewidth{1.505625pt}%
\definecolor{currentstroke}{rgb}{0.580392,0.403922,0.741176}%
\pgfsetstrokecolor{currentstroke}%
\pgfsetdash{}{0pt}%
\pgfpathmoveto{\pgfqpoint{3.272946in}{1.120004in}}%
\pgfpathlineto{\pgfqpoint{3.272946in}{1.120612in}}%
\pgfusepath{stroke}%
\end{pgfscope}%
\begin{pgfscope}%
\pgfpathrectangle{\pgfqpoint{1.286132in}{0.839159in}}{\pgfqpoint{12.053712in}{5.967710in}}%
\pgfusepath{clip}%
\pgfsetbuttcap%
\pgfsetroundjoin%
\pgfsetlinewidth{1.505625pt}%
\definecolor{currentstroke}{rgb}{0.580392,0.403922,0.741176}%
\pgfsetstrokecolor{currentstroke}%
\pgfsetdash{}{0pt}%
\pgfpathmoveto{\pgfqpoint{3.383632in}{1.120182in}}%
\pgfpathlineto{\pgfqpoint{3.383632in}{1.120729in}}%
\pgfusepath{stroke}%
\end{pgfscope}%
\begin{pgfscope}%
\pgfpathrectangle{\pgfqpoint{1.286132in}{0.839159in}}{\pgfqpoint{12.053712in}{5.967710in}}%
\pgfusepath{clip}%
\pgfsetbuttcap%
\pgfsetroundjoin%
\pgfsetlinewidth{1.505625pt}%
\definecolor{currentstroke}{rgb}{0.580392,0.403922,0.741176}%
\pgfsetstrokecolor{currentstroke}%
\pgfsetdash{}{0pt}%
\pgfpathmoveto{\pgfqpoint{3.494319in}{1.120082in}}%
\pgfpathlineto{\pgfqpoint{3.494319in}{1.120536in}}%
\pgfusepath{stroke}%
\end{pgfscope}%
\begin{pgfscope}%
\pgfpathrectangle{\pgfqpoint{1.286132in}{0.839159in}}{\pgfqpoint{12.053712in}{5.967710in}}%
\pgfusepath{clip}%
\pgfsetbuttcap%
\pgfsetroundjoin%
\pgfsetlinewidth{1.505625pt}%
\definecolor{currentstroke}{rgb}{0.580392,0.403922,0.741176}%
\pgfsetstrokecolor{currentstroke}%
\pgfsetdash{}{0pt}%
\pgfpathmoveto{\pgfqpoint{3.605005in}{1.120050in}}%
\pgfpathlineto{\pgfqpoint{3.605005in}{1.120586in}}%
\pgfusepath{stroke}%
\end{pgfscope}%
\begin{pgfscope}%
\pgfpathrectangle{\pgfqpoint{1.286132in}{0.839159in}}{\pgfqpoint{12.053712in}{5.967710in}}%
\pgfusepath{clip}%
\pgfsetbuttcap%
\pgfsetroundjoin%
\pgfsetlinewidth{1.505625pt}%
\definecolor{currentstroke}{rgb}{0.580392,0.403922,0.741176}%
\pgfsetstrokecolor{currentstroke}%
\pgfsetdash{}{0pt}%
\pgfpathmoveto{\pgfqpoint{3.715691in}{1.120471in}}%
\pgfpathlineto{\pgfqpoint{3.715691in}{1.120814in}}%
\pgfusepath{stroke}%
\end{pgfscope}%
\begin{pgfscope}%
\pgfpathrectangle{\pgfqpoint{1.286132in}{0.839159in}}{\pgfqpoint{12.053712in}{5.967710in}}%
\pgfusepath{clip}%
\pgfsetbuttcap%
\pgfsetroundjoin%
\pgfsetlinewidth{1.505625pt}%
\definecolor{currentstroke}{rgb}{0.580392,0.403922,0.741176}%
\pgfsetstrokecolor{currentstroke}%
\pgfsetdash{}{0pt}%
\pgfpathmoveto{\pgfqpoint{3.826377in}{1.120675in}}%
\pgfpathlineto{\pgfqpoint{3.826377in}{1.121186in}}%
\pgfusepath{stroke}%
\end{pgfscope}%
\begin{pgfscope}%
\pgfpathrectangle{\pgfqpoint{1.286132in}{0.839159in}}{\pgfqpoint{12.053712in}{5.967710in}}%
\pgfusepath{clip}%
\pgfsetbuttcap%
\pgfsetroundjoin%
\pgfsetlinewidth{1.505625pt}%
\definecolor{currentstroke}{rgb}{0.580392,0.403922,0.741176}%
\pgfsetstrokecolor{currentstroke}%
\pgfsetdash{}{0pt}%
\pgfpathmoveto{\pgfqpoint{3.937063in}{1.120351in}}%
\pgfpathlineto{\pgfqpoint{3.937063in}{1.120972in}}%
\pgfusepath{stroke}%
\end{pgfscope}%
\begin{pgfscope}%
\pgfpathrectangle{\pgfqpoint{1.286132in}{0.839159in}}{\pgfqpoint{12.053712in}{5.967710in}}%
\pgfusepath{clip}%
\pgfsetbuttcap%
\pgfsetroundjoin%
\pgfsetlinewidth{1.505625pt}%
\definecolor{currentstroke}{rgb}{0.580392,0.403922,0.741176}%
\pgfsetstrokecolor{currentstroke}%
\pgfsetdash{}{0pt}%
\pgfpathmoveto{\pgfqpoint{4.047749in}{1.111875in}}%
\pgfpathlineto{\pgfqpoint{4.047749in}{1.146873in}}%
\pgfusepath{stroke}%
\end{pgfscope}%
\begin{pgfscope}%
\pgfpathrectangle{\pgfqpoint{1.286132in}{0.839159in}}{\pgfqpoint{12.053712in}{5.967710in}}%
\pgfusepath{clip}%
\pgfsetbuttcap%
\pgfsetroundjoin%
\pgfsetlinewidth{1.505625pt}%
\definecolor{currentstroke}{rgb}{0.580392,0.403922,0.741176}%
\pgfsetstrokecolor{currentstroke}%
\pgfsetdash{}{0pt}%
\pgfpathmoveto{\pgfqpoint{4.158435in}{1.120773in}}%
\pgfpathlineto{\pgfqpoint{4.158435in}{1.121627in}}%
\pgfusepath{stroke}%
\end{pgfscope}%
\begin{pgfscope}%
\pgfpathrectangle{\pgfqpoint{1.286132in}{0.839159in}}{\pgfqpoint{12.053712in}{5.967710in}}%
\pgfusepath{clip}%
\pgfsetbuttcap%
\pgfsetroundjoin%
\pgfsetlinewidth{1.505625pt}%
\definecolor{currentstroke}{rgb}{0.580392,0.403922,0.741176}%
\pgfsetstrokecolor{currentstroke}%
\pgfsetdash{}{0pt}%
\pgfpathmoveto{\pgfqpoint{4.269121in}{1.120646in}}%
\pgfpathlineto{\pgfqpoint{4.269121in}{1.122022in}}%
\pgfusepath{stroke}%
\end{pgfscope}%
\begin{pgfscope}%
\pgfpathrectangle{\pgfqpoint{1.286132in}{0.839159in}}{\pgfqpoint{12.053712in}{5.967710in}}%
\pgfusepath{clip}%
\pgfsetbuttcap%
\pgfsetroundjoin%
\pgfsetlinewidth{1.505625pt}%
\definecolor{currentstroke}{rgb}{0.580392,0.403922,0.741176}%
\pgfsetstrokecolor{currentstroke}%
\pgfsetdash{}{0pt}%
\pgfpathmoveto{\pgfqpoint{4.379807in}{1.120847in}}%
\pgfpathlineto{\pgfqpoint{4.379807in}{1.121593in}}%
\pgfusepath{stroke}%
\end{pgfscope}%
\begin{pgfscope}%
\pgfpathrectangle{\pgfqpoint{1.286132in}{0.839159in}}{\pgfqpoint{12.053712in}{5.967710in}}%
\pgfusepath{clip}%
\pgfsetbuttcap%
\pgfsetroundjoin%
\pgfsetlinewidth{1.505625pt}%
\definecolor{currentstroke}{rgb}{0.580392,0.403922,0.741176}%
\pgfsetstrokecolor{currentstroke}%
\pgfsetdash{}{0pt}%
\pgfpathmoveto{\pgfqpoint{4.490493in}{1.120901in}}%
\pgfpathlineto{\pgfqpoint{4.490493in}{1.121767in}}%
\pgfusepath{stroke}%
\end{pgfscope}%
\begin{pgfscope}%
\pgfpathrectangle{\pgfqpoint{1.286132in}{0.839159in}}{\pgfqpoint{12.053712in}{5.967710in}}%
\pgfusepath{clip}%
\pgfsetbuttcap%
\pgfsetroundjoin%
\pgfsetlinewidth{1.505625pt}%
\definecolor{currentstroke}{rgb}{0.580392,0.403922,0.741176}%
\pgfsetstrokecolor{currentstroke}%
\pgfsetdash{}{0pt}%
\pgfpathmoveto{\pgfqpoint{4.601179in}{1.120867in}}%
\pgfpathlineto{\pgfqpoint{4.601179in}{1.121436in}}%
\pgfusepath{stroke}%
\end{pgfscope}%
\begin{pgfscope}%
\pgfpathrectangle{\pgfqpoint{1.286132in}{0.839159in}}{\pgfqpoint{12.053712in}{5.967710in}}%
\pgfusepath{clip}%
\pgfsetbuttcap%
\pgfsetroundjoin%
\pgfsetlinewidth{1.505625pt}%
\definecolor{currentstroke}{rgb}{0.580392,0.403922,0.741176}%
\pgfsetstrokecolor{currentstroke}%
\pgfsetdash{}{0pt}%
\pgfpathmoveto{\pgfqpoint{4.711865in}{1.120758in}}%
\pgfpathlineto{\pgfqpoint{4.711865in}{1.121977in}}%
\pgfusepath{stroke}%
\end{pgfscope}%
\begin{pgfscope}%
\pgfpathrectangle{\pgfqpoint{1.286132in}{0.839159in}}{\pgfqpoint{12.053712in}{5.967710in}}%
\pgfusepath{clip}%
\pgfsetbuttcap%
\pgfsetroundjoin%
\pgfsetlinewidth{1.505625pt}%
\definecolor{currentstroke}{rgb}{0.580392,0.403922,0.741176}%
\pgfsetstrokecolor{currentstroke}%
\pgfsetdash{}{0pt}%
\pgfpathmoveto{\pgfqpoint{4.822551in}{1.121129in}}%
\pgfpathlineto{\pgfqpoint{4.822551in}{1.122225in}}%
\pgfusepath{stroke}%
\end{pgfscope}%
\begin{pgfscope}%
\pgfpathrectangle{\pgfqpoint{1.286132in}{0.839159in}}{\pgfqpoint{12.053712in}{5.967710in}}%
\pgfusepath{clip}%
\pgfsetbuttcap%
\pgfsetroundjoin%
\pgfsetlinewidth{1.505625pt}%
\definecolor{currentstroke}{rgb}{0.580392,0.403922,0.741176}%
\pgfsetstrokecolor{currentstroke}%
\pgfsetdash{}{0pt}%
\pgfpathmoveto{\pgfqpoint{4.933237in}{1.121699in}}%
\pgfpathlineto{\pgfqpoint{4.933237in}{1.122328in}}%
\pgfusepath{stroke}%
\end{pgfscope}%
\begin{pgfscope}%
\pgfpathrectangle{\pgfqpoint{1.286132in}{0.839159in}}{\pgfqpoint{12.053712in}{5.967710in}}%
\pgfusepath{clip}%
\pgfsetbuttcap%
\pgfsetroundjoin%
\pgfsetlinewidth{1.505625pt}%
\definecolor{currentstroke}{rgb}{0.580392,0.403922,0.741176}%
\pgfsetstrokecolor{currentstroke}%
\pgfsetdash{}{0pt}%
\pgfpathmoveto{\pgfqpoint{5.043923in}{1.121447in}}%
\pgfpathlineto{\pgfqpoint{5.043923in}{1.122215in}}%
\pgfusepath{stroke}%
\end{pgfscope}%
\begin{pgfscope}%
\pgfpathrectangle{\pgfqpoint{1.286132in}{0.839159in}}{\pgfqpoint{12.053712in}{5.967710in}}%
\pgfusepath{clip}%
\pgfsetbuttcap%
\pgfsetroundjoin%
\pgfsetlinewidth{1.505625pt}%
\definecolor{currentstroke}{rgb}{0.580392,0.403922,0.741176}%
\pgfsetstrokecolor{currentstroke}%
\pgfsetdash{}{0pt}%
\pgfpathmoveto{\pgfqpoint{5.154609in}{1.121291in}}%
\pgfpathlineto{\pgfqpoint{5.154609in}{1.122595in}}%
\pgfusepath{stroke}%
\end{pgfscope}%
\begin{pgfscope}%
\pgfpathrectangle{\pgfqpoint{1.286132in}{0.839159in}}{\pgfqpoint{12.053712in}{5.967710in}}%
\pgfusepath{clip}%
\pgfsetbuttcap%
\pgfsetroundjoin%
\pgfsetlinewidth{1.505625pt}%
\definecolor{currentstroke}{rgb}{0.580392,0.403922,0.741176}%
\pgfsetstrokecolor{currentstroke}%
\pgfsetdash{}{0pt}%
\pgfpathmoveto{\pgfqpoint{5.265296in}{1.120466in}}%
\pgfpathlineto{\pgfqpoint{5.265296in}{1.123202in}}%
\pgfusepath{stroke}%
\end{pgfscope}%
\begin{pgfscope}%
\pgfpathrectangle{\pgfqpoint{1.286132in}{0.839159in}}{\pgfqpoint{12.053712in}{5.967710in}}%
\pgfusepath{clip}%
\pgfsetbuttcap%
\pgfsetroundjoin%
\pgfsetlinewidth{1.505625pt}%
\definecolor{currentstroke}{rgb}{0.580392,0.403922,0.741176}%
\pgfsetstrokecolor{currentstroke}%
\pgfsetdash{}{0pt}%
\pgfpathmoveto{\pgfqpoint{5.375982in}{1.122083in}}%
\pgfpathlineto{\pgfqpoint{5.375982in}{1.123146in}}%
\pgfusepath{stroke}%
\end{pgfscope}%
\begin{pgfscope}%
\pgfpathrectangle{\pgfqpoint{1.286132in}{0.839159in}}{\pgfqpoint{12.053712in}{5.967710in}}%
\pgfusepath{clip}%
\pgfsetbuttcap%
\pgfsetroundjoin%
\pgfsetlinewidth{1.505625pt}%
\definecolor{currentstroke}{rgb}{0.580392,0.403922,0.741176}%
\pgfsetstrokecolor{currentstroke}%
\pgfsetdash{}{0pt}%
\pgfpathmoveto{\pgfqpoint{5.486668in}{1.121943in}}%
\pgfpathlineto{\pgfqpoint{5.486668in}{1.122427in}}%
\pgfusepath{stroke}%
\end{pgfscope}%
\begin{pgfscope}%
\pgfpathrectangle{\pgfqpoint{1.286132in}{0.839159in}}{\pgfqpoint{12.053712in}{5.967710in}}%
\pgfusepath{clip}%
\pgfsetbuttcap%
\pgfsetroundjoin%
\pgfsetlinewidth{1.505625pt}%
\definecolor{currentstroke}{rgb}{0.580392,0.403922,0.741176}%
\pgfsetstrokecolor{currentstroke}%
\pgfsetdash{}{0pt}%
\pgfpathmoveto{\pgfqpoint{5.597354in}{1.121829in}}%
\pgfpathlineto{\pgfqpoint{5.597354in}{1.122948in}}%
\pgfusepath{stroke}%
\end{pgfscope}%
\begin{pgfscope}%
\pgfpathrectangle{\pgfqpoint{1.286132in}{0.839159in}}{\pgfqpoint{12.053712in}{5.967710in}}%
\pgfusepath{clip}%
\pgfsetbuttcap%
\pgfsetroundjoin%
\pgfsetlinewidth{1.505625pt}%
\definecolor{currentstroke}{rgb}{0.580392,0.403922,0.741176}%
\pgfsetstrokecolor{currentstroke}%
\pgfsetdash{}{0pt}%
\pgfpathmoveto{\pgfqpoint{5.708040in}{1.122248in}}%
\pgfpathlineto{\pgfqpoint{5.708040in}{1.123041in}}%
\pgfusepath{stroke}%
\end{pgfscope}%
\begin{pgfscope}%
\pgfpathrectangle{\pgfqpoint{1.286132in}{0.839159in}}{\pgfqpoint{12.053712in}{5.967710in}}%
\pgfusepath{clip}%
\pgfsetbuttcap%
\pgfsetroundjoin%
\pgfsetlinewidth{1.505625pt}%
\definecolor{currentstroke}{rgb}{0.580392,0.403922,0.741176}%
\pgfsetstrokecolor{currentstroke}%
\pgfsetdash{}{0pt}%
\pgfpathmoveto{\pgfqpoint{5.818726in}{1.121453in}}%
\pgfpathlineto{\pgfqpoint{5.818726in}{1.122742in}}%
\pgfusepath{stroke}%
\end{pgfscope}%
\begin{pgfscope}%
\pgfpathrectangle{\pgfqpoint{1.286132in}{0.839159in}}{\pgfqpoint{12.053712in}{5.967710in}}%
\pgfusepath{clip}%
\pgfsetbuttcap%
\pgfsetroundjoin%
\pgfsetlinewidth{1.505625pt}%
\definecolor{currentstroke}{rgb}{0.580392,0.403922,0.741176}%
\pgfsetstrokecolor{currentstroke}%
\pgfsetdash{}{0pt}%
\pgfpathmoveto{\pgfqpoint{5.929412in}{1.121294in}}%
\pgfpathlineto{\pgfqpoint{5.929412in}{1.126407in}}%
\pgfusepath{stroke}%
\end{pgfscope}%
\begin{pgfscope}%
\pgfpathrectangle{\pgfqpoint{1.286132in}{0.839159in}}{\pgfqpoint{12.053712in}{5.967710in}}%
\pgfusepath{clip}%
\pgfsetbuttcap%
\pgfsetroundjoin%
\pgfsetlinewidth{1.505625pt}%
\definecolor{currentstroke}{rgb}{0.580392,0.403922,0.741176}%
\pgfsetstrokecolor{currentstroke}%
\pgfsetdash{}{0pt}%
\pgfpathmoveto{\pgfqpoint{6.040098in}{1.122164in}}%
\pgfpathlineto{\pgfqpoint{6.040098in}{1.122837in}}%
\pgfusepath{stroke}%
\end{pgfscope}%
\begin{pgfscope}%
\pgfpathrectangle{\pgfqpoint{1.286132in}{0.839159in}}{\pgfqpoint{12.053712in}{5.967710in}}%
\pgfusepath{clip}%
\pgfsetbuttcap%
\pgfsetroundjoin%
\pgfsetlinewidth{1.505625pt}%
\definecolor{currentstroke}{rgb}{0.580392,0.403922,0.741176}%
\pgfsetstrokecolor{currentstroke}%
\pgfsetdash{}{0pt}%
\pgfpathmoveto{\pgfqpoint{6.150784in}{1.121804in}}%
\pgfpathlineto{\pgfqpoint{6.150784in}{1.122475in}}%
\pgfusepath{stroke}%
\end{pgfscope}%
\begin{pgfscope}%
\pgfpathrectangle{\pgfqpoint{1.286132in}{0.839159in}}{\pgfqpoint{12.053712in}{5.967710in}}%
\pgfusepath{clip}%
\pgfsetbuttcap%
\pgfsetroundjoin%
\pgfsetlinewidth{1.505625pt}%
\definecolor{currentstroke}{rgb}{0.580392,0.403922,0.741176}%
\pgfsetstrokecolor{currentstroke}%
\pgfsetdash{}{0pt}%
\pgfpathmoveto{\pgfqpoint{6.261470in}{1.121733in}}%
\pgfpathlineto{\pgfqpoint{6.261470in}{1.122451in}}%
\pgfusepath{stroke}%
\end{pgfscope}%
\begin{pgfscope}%
\pgfpathrectangle{\pgfqpoint{1.286132in}{0.839159in}}{\pgfqpoint{12.053712in}{5.967710in}}%
\pgfusepath{clip}%
\pgfsetbuttcap%
\pgfsetroundjoin%
\pgfsetlinewidth{1.505625pt}%
\definecolor{currentstroke}{rgb}{0.580392,0.403922,0.741176}%
\pgfsetstrokecolor{currentstroke}%
\pgfsetdash{}{0pt}%
\pgfpathmoveto{\pgfqpoint{6.372156in}{1.122903in}}%
\pgfpathlineto{\pgfqpoint{6.372156in}{1.124311in}}%
\pgfusepath{stroke}%
\end{pgfscope}%
\begin{pgfscope}%
\pgfpathrectangle{\pgfqpoint{1.286132in}{0.839159in}}{\pgfqpoint{12.053712in}{5.967710in}}%
\pgfusepath{clip}%
\pgfsetbuttcap%
\pgfsetroundjoin%
\pgfsetlinewidth{1.505625pt}%
\definecolor{currentstroke}{rgb}{0.580392,0.403922,0.741176}%
\pgfsetstrokecolor{currentstroke}%
\pgfsetdash{}{0pt}%
\pgfpathmoveto{\pgfqpoint{6.482842in}{1.122643in}}%
\pgfpathlineto{\pgfqpoint{6.482842in}{1.124414in}}%
\pgfusepath{stroke}%
\end{pgfscope}%
\begin{pgfscope}%
\pgfpathrectangle{\pgfqpoint{1.286132in}{0.839159in}}{\pgfqpoint{12.053712in}{5.967710in}}%
\pgfusepath{clip}%
\pgfsetbuttcap%
\pgfsetroundjoin%
\pgfsetlinewidth{1.505625pt}%
\definecolor{currentstroke}{rgb}{0.580392,0.403922,0.741176}%
\pgfsetstrokecolor{currentstroke}%
\pgfsetdash{}{0pt}%
\pgfpathmoveto{\pgfqpoint{6.593528in}{1.122424in}}%
\pgfpathlineto{\pgfqpoint{6.593528in}{1.123558in}}%
\pgfusepath{stroke}%
\end{pgfscope}%
\begin{pgfscope}%
\pgfpathrectangle{\pgfqpoint{1.286132in}{0.839159in}}{\pgfqpoint{12.053712in}{5.967710in}}%
\pgfusepath{clip}%
\pgfsetbuttcap%
\pgfsetroundjoin%
\pgfsetlinewidth{1.505625pt}%
\definecolor{currentstroke}{rgb}{0.580392,0.403922,0.741176}%
\pgfsetstrokecolor{currentstroke}%
\pgfsetdash{}{0pt}%
\pgfpathmoveto{\pgfqpoint{6.704214in}{1.122062in}}%
\pgfpathlineto{\pgfqpoint{6.704214in}{1.123184in}}%
\pgfusepath{stroke}%
\end{pgfscope}%
\begin{pgfscope}%
\pgfpathrectangle{\pgfqpoint{1.286132in}{0.839159in}}{\pgfqpoint{12.053712in}{5.967710in}}%
\pgfusepath{clip}%
\pgfsetbuttcap%
\pgfsetroundjoin%
\pgfsetlinewidth{1.505625pt}%
\definecolor{currentstroke}{rgb}{0.580392,0.403922,0.741176}%
\pgfsetstrokecolor{currentstroke}%
\pgfsetdash{}{0pt}%
\pgfpathmoveto{\pgfqpoint{6.814900in}{1.122810in}}%
\pgfpathlineto{\pgfqpoint{6.814900in}{1.124378in}}%
\pgfusepath{stroke}%
\end{pgfscope}%
\begin{pgfscope}%
\pgfpathrectangle{\pgfqpoint{1.286132in}{0.839159in}}{\pgfqpoint{12.053712in}{5.967710in}}%
\pgfusepath{clip}%
\pgfsetbuttcap%
\pgfsetroundjoin%
\pgfsetlinewidth{1.505625pt}%
\definecolor{currentstroke}{rgb}{0.580392,0.403922,0.741176}%
\pgfsetstrokecolor{currentstroke}%
\pgfsetdash{}{0pt}%
\pgfpathmoveto{\pgfqpoint{6.925586in}{1.122783in}}%
\pgfpathlineto{\pgfqpoint{6.925586in}{1.125991in}}%
\pgfusepath{stroke}%
\end{pgfscope}%
\begin{pgfscope}%
\pgfpathrectangle{\pgfqpoint{1.286132in}{0.839159in}}{\pgfqpoint{12.053712in}{5.967710in}}%
\pgfusepath{clip}%
\pgfsetbuttcap%
\pgfsetroundjoin%
\pgfsetlinewidth{1.505625pt}%
\definecolor{currentstroke}{rgb}{0.580392,0.403922,0.741176}%
\pgfsetstrokecolor{currentstroke}%
\pgfsetdash{}{0pt}%
\pgfpathmoveto{\pgfqpoint{7.036272in}{1.122076in}}%
\pgfpathlineto{\pgfqpoint{7.036272in}{1.123986in}}%
\pgfusepath{stroke}%
\end{pgfscope}%
\begin{pgfscope}%
\pgfpathrectangle{\pgfqpoint{1.286132in}{0.839159in}}{\pgfqpoint{12.053712in}{5.967710in}}%
\pgfusepath{clip}%
\pgfsetbuttcap%
\pgfsetroundjoin%
\pgfsetlinewidth{1.505625pt}%
\definecolor{currentstroke}{rgb}{0.580392,0.403922,0.741176}%
\pgfsetstrokecolor{currentstroke}%
\pgfsetdash{}{0pt}%
\pgfpathmoveto{\pgfqpoint{7.146959in}{1.123489in}}%
\pgfpathlineto{\pgfqpoint{7.146959in}{1.124495in}}%
\pgfusepath{stroke}%
\end{pgfscope}%
\begin{pgfscope}%
\pgfpathrectangle{\pgfqpoint{1.286132in}{0.839159in}}{\pgfqpoint{12.053712in}{5.967710in}}%
\pgfusepath{clip}%
\pgfsetbuttcap%
\pgfsetroundjoin%
\pgfsetlinewidth{1.505625pt}%
\definecolor{currentstroke}{rgb}{0.580392,0.403922,0.741176}%
\pgfsetstrokecolor{currentstroke}%
\pgfsetdash{}{0pt}%
\pgfpathmoveto{\pgfqpoint{7.257645in}{1.123565in}}%
\pgfpathlineto{\pgfqpoint{7.257645in}{1.124658in}}%
\pgfusepath{stroke}%
\end{pgfscope}%
\begin{pgfscope}%
\pgfpathrectangle{\pgfqpoint{1.286132in}{0.839159in}}{\pgfqpoint{12.053712in}{5.967710in}}%
\pgfusepath{clip}%
\pgfsetbuttcap%
\pgfsetroundjoin%
\pgfsetlinewidth{1.505625pt}%
\definecolor{currentstroke}{rgb}{0.580392,0.403922,0.741176}%
\pgfsetstrokecolor{currentstroke}%
\pgfsetdash{}{0pt}%
\pgfpathmoveto{\pgfqpoint{7.368331in}{1.123699in}}%
\pgfpathlineto{\pgfqpoint{7.368331in}{1.126597in}}%
\pgfusepath{stroke}%
\end{pgfscope}%
\begin{pgfscope}%
\pgfpathrectangle{\pgfqpoint{1.286132in}{0.839159in}}{\pgfqpoint{12.053712in}{5.967710in}}%
\pgfusepath{clip}%
\pgfsetbuttcap%
\pgfsetroundjoin%
\pgfsetlinewidth{1.505625pt}%
\definecolor{currentstroke}{rgb}{0.580392,0.403922,0.741176}%
\pgfsetstrokecolor{currentstroke}%
\pgfsetdash{}{0pt}%
\pgfpathmoveto{\pgfqpoint{7.479017in}{1.122011in}}%
\pgfpathlineto{\pgfqpoint{7.479017in}{1.126432in}}%
\pgfusepath{stroke}%
\end{pgfscope}%
\begin{pgfscope}%
\pgfpathrectangle{\pgfqpoint{1.286132in}{0.839159in}}{\pgfqpoint{12.053712in}{5.967710in}}%
\pgfusepath{clip}%
\pgfsetbuttcap%
\pgfsetroundjoin%
\pgfsetlinewidth{1.505625pt}%
\definecolor{currentstroke}{rgb}{0.580392,0.403922,0.741176}%
\pgfsetstrokecolor{currentstroke}%
\pgfsetdash{}{0pt}%
\pgfpathmoveto{\pgfqpoint{7.589703in}{1.123637in}}%
\pgfpathlineto{\pgfqpoint{7.589703in}{1.124493in}}%
\pgfusepath{stroke}%
\end{pgfscope}%
\begin{pgfscope}%
\pgfpathrectangle{\pgfqpoint{1.286132in}{0.839159in}}{\pgfqpoint{12.053712in}{5.967710in}}%
\pgfusepath{clip}%
\pgfsetbuttcap%
\pgfsetroundjoin%
\pgfsetlinewidth{1.505625pt}%
\definecolor{currentstroke}{rgb}{0.580392,0.403922,0.741176}%
\pgfsetstrokecolor{currentstroke}%
\pgfsetdash{}{0pt}%
\pgfpathmoveto{\pgfqpoint{7.700389in}{1.123235in}}%
\pgfpathlineto{\pgfqpoint{7.700389in}{1.124917in}}%
\pgfusepath{stroke}%
\end{pgfscope}%
\begin{pgfscope}%
\pgfpathrectangle{\pgfqpoint{1.286132in}{0.839159in}}{\pgfqpoint{12.053712in}{5.967710in}}%
\pgfusepath{clip}%
\pgfsetbuttcap%
\pgfsetroundjoin%
\pgfsetlinewidth{1.505625pt}%
\definecolor{currentstroke}{rgb}{0.580392,0.403922,0.741176}%
\pgfsetstrokecolor{currentstroke}%
\pgfsetdash{}{0pt}%
\pgfpathmoveto{\pgfqpoint{7.811075in}{1.123591in}}%
\pgfpathlineto{\pgfqpoint{7.811075in}{1.125729in}}%
\pgfusepath{stroke}%
\end{pgfscope}%
\begin{pgfscope}%
\pgfpathrectangle{\pgfqpoint{1.286132in}{0.839159in}}{\pgfqpoint{12.053712in}{5.967710in}}%
\pgfusepath{clip}%
\pgfsetbuttcap%
\pgfsetroundjoin%
\pgfsetlinewidth{1.505625pt}%
\definecolor{currentstroke}{rgb}{0.580392,0.403922,0.741176}%
\pgfsetstrokecolor{currentstroke}%
\pgfsetdash{}{0pt}%
\pgfpathmoveto{\pgfqpoint{7.921761in}{1.123556in}}%
\pgfpathlineto{\pgfqpoint{7.921761in}{1.125557in}}%
\pgfusepath{stroke}%
\end{pgfscope}%
\begin{pgfscope}%
\pgfpathrectangle{\pgfqpoint{1.286132in}{0.839159in}}{\pgfqpoint{12.053712in}{5.967710in}}%
\pgfusepath{clip}%
\pgfsetbuttcap%
\pgfsetroundjoin%
\pgfsetlinewidth{1.505625pt}%
\definecolor{currentstroke}{rgb}{0.580392,0.403922,0.741176}%
\pgfsetstrokecolor{currentstroke}%
\pgfsetdash{}{0pt}%
\pgfpathmoveto{\pgfqpoint{8.032447in}{1.124008in}}%
\pgfpathlineto{\pgfqpoint{8.032447in}{1.125135in}}%
\pgfusepath{stroke}%
\end{pgfscope}%
\begin{pgfscope}%
\pgfpathrectangle{\pgfqpoint{1.286132in}{0.839159in}}{\pgfqpoint{12.053712in}{5.967710in}}%
\pgfusepath{clip}%
\pgfsetbuttcap%
\pgfsetroundjoin%
\pgfsetlinewidth{1.505625pt}%
\definecolor{currentstroke}{rgb}{0.580392,0.403922,0.741176}%
\pgfsetstrokecolor{currentstroke}%
\pgfsetdash{}{0pt}%
\pgfpathmoveto{\pgfqpoint{8.143133in}{1.124082in}}%
\pgfpathlineto{\pgfqpoint{8.143133in}{1.125843in}}%
\pgfusepath{stroke}%
\end{pgfscope}%
\begin{pgfscope}%
\pgfpathrectangle{\pgfqpoint{1.286132in}{0.839159in}}{\pgfqpoint{12.053712in}{5.967710in}}%
\pgfusepath{clip}%
\pgfsetbuttcap%
\pgfsetroundjoin%
\pgfsetlinewidth{1.505625pt}%
\definecolor{currentstroke}{rgb}{0.580392,0.403922,0.741176}%
\pgfsetstrokecolor{currentstroke}%
\pgfsetdash{}{0pt}%
\pgfpathmoveto{\pgfqpoint{8.253819in}{1.123551in}}%
\pgfpathlineto{\pgfqpoint{8.253819in}{1.125325in}}%
\pgfusepath{stroke}%
\end{pgfscope}%
\begin{pgfscope}%
\pgfpathrectangle{\pgfqpoint{1.286132in}{0.839159in}}{\pgfqpoint{12.053712in}{5.967710in}}%
\pgfusepath{clip}%
\pgfsetbuttcap%
\pgfsetroundjoin%
\pgfsetlinewidth{1.505625pt}%
\definecolor{currentstroke}{rgb}{0.580392,0.403922,0.741176}%
\pgfsetstrokecolor{currentstroke}%
\pgfsetdash{}{0pt}%
\pgfpathmoveto{\pgfqpoint{8.364505in}{1.124786in}}%
\pgfpathlineto{\pgfqpoint{8.364505in}{1.127502in}}%
\pgfusepath{stroke}%
\end{pgfscope}%
\begin{pgfscope}%
\pgfpathrectangle{\pgfqpoint{1.286132in}{0.839159in}}{\pgfqpoint{12.053712in}{5.967710in}}%
\pgfusepath{clip}%
\pgfsetbuttcap%
\pgfsetroundjoin%
\pgfsetlinewidth{1.505625pt}%
\definecolor{currentstroke}{rgb}{0.580392,0.403922,0.741176}%
\pgfsetstrokecolor{currentstroke}%
\pgfsetdash{}{0pt}%
\pgfpathmoveto{\pgfqpoint{8.475191in}{1.124538in}}%
\pgfpathlineto{\pgfqpoint{8.475191in}{1.126941in}}%
\pgfusepath{stroke}%
\end{pgfscope}%
\begin{pgfscope}%
\pgfpathrectangle{\pgfqpoint{1.286132in}{0.839159in}}{\pgfqpoint{12.053712in}{5.967710in}}%
\pgfusepath{clip}%
\pgfsetbuttcap%
\pgfsetroundjoin%
\pgfsetlinewidth{1.505625pt}%
\definecolor{currentstroke}{rgb}{0.580392,0.403922,0.741176}%
\pgfsetstrokecolor{currentstroke}%
\pgfsetdash{}{0pt}%
\pgfpathmoveto{\pgfqpoint{8.585877in}{1.124046in}}%
\pgfpathlineto{\pgfqpoint{8.585877in}{1.127406in}}%
\pgfusepath{stroke}%
\end{pgfscope}%
\begin{pgfscope}%
\pgfpathrectangle{\pgfqpoint{1.286132in}{0.839159in}}{\pgfqpoint{12.053712in}{5.967710in}}%
\pgfusepath{clip}%
\pgfsetbuttcap%
\pgfsetroundjoin%
\pgfsetlinewidth{1.505625pt}%
\definecolor{currentstroke}{rgb}{0.580392,0.403922,0.741176}%
\pgfsetstrokecolor{currentstroke}%
\pgfsetdash{}{0pt}%
\pgfpathmoveto{\pgfqpoint{8.696563in}{1.124164in}}%
\pgfpathlineto{\pgfqpoint{8.696563in}{1.130109in}}%
\pgfusepath{stroke}%
\end{pgfscope}%
\begin{pgfscope}%
\pgfpathrectangle{\pgfqpoint{1.286132in}{0.839159in}}{\pgfqpoint{12.053712in}{5.967710in}}%
\pgfusepath{clip}%
\pgfsetbuttcap%
\pgfsetroundjoin%
\pgfsetlinewidth{1.505625pt}%
\definecolor{currentstroke}{rgb}{0.580392,0.403922,0.741176}%
\pgfsetstrokecolor{currentstroke}%
\pgfsetdash{}{0pt}%
\pgfpathmoveto{\pgfqpoint{8.807249in}{1.123187in}}%
\pgfpathlineto{\pgfqpoint{8.807249in}{1.128015in}}%
\pgfusepath{stroke}%
\end{pgfscope}%
\begin{pgfscope}%
\pgfpathrectangle{\pgfqpoint{1.286132in}{0.839159in}}{\pgfqpoint{12.053712in}{5.967710in}}%
\pgfusepath{clip}%
\pgfsetbuttcap%
\pgfsetroundjoin%
\pgfsetlinewidth{1.505625pt}%
\definecolor{currentstroke}{rgb}{0.580392,0.403922,0.741176}%
\pgfsetstrokecolor{currentstroke}%
\pgfsetdash{}{0pt}%
\pgfpathmoveto{\pgfqpoint{8.917936in}{1.124690in}}%
\pgfpathlineto{\pgfqpoint{8.917936in}{1.128647in}}%
\pgfusepath{stroke}%
\end{pgfscope}%
\begin{pgfscope}%
\pgfpathrectangle{\pgfqpoint{1.286132in}{0.839159in}}{\pgfqpoint{12.053712in}{5.967710in}}%
\pgfusepath{clip}%
\pgfsetbuttcap%
\pgfsetroundjoin%
\pgfsetlinewidth{1.505625pt}%
\definecolor{currentstroke}{rgb}{0.580392,0.403922,0.741176}%
\pgfsetstrokecolor{currentstroke}%
\pgfsetdash{}{0pt}%
\pgfpathmoveto{\pgfqpoint{9.028622in}{1.126104in}}%
\pgfpathlineto{\pgfqpoint{9.028622in}{1.127664in}}%
\pgfusepath{stroke}%
\end{pgfscope}%
\begin{pgfscope}%
\pgfpathrectangle{\pgfqpoint{1.286132in}{0.839159in}}{\pgfqpoint{12.053712in}{5.967710in}}%
\pgfusepath{clip}%
\pgfsetbuttcap%
\pgfsetroundjoin%
\pgfsetlinewidth{1.505625pt}%
\definecolor{currentstroke}{rgb}{0.580392,0.403922,0.741176}%
\pgfsetstrokecolor{currentstroke}%
\pgfsetdash{}{0pt}%
\pgfpathmoveto{\pgfqpoint{9.139308in}{1.124162in}}%
\pgfpathlineto{\pgfqpoint{9.139308in}{1.125373in}}%
\pgfusepath{stroke}%
\end{pgfscope}%
\begin{pgfscope}%
\pgfpathrectangle{\pgfqpoint{1.286132in}{0.839159in}}{\pgfqpoint{12.053712in}{5.967710in}}%
\pgfusepath{clip}%
\pgfsetbuttcap%
\pgfsetroundjoin%
\pgfsetlinewidth{1.505625pt}%
\definecolor{currentstroke}{rgb}{0.580392,0.403922,0.741176}%
\pgfsetstrokecolor{currentstroke}%
\pgfsetdash{}{0pt}%
\pgfpathmoveto{\pgfqpoint{9.249994in}{1.125853in}}%
\pgfpathlineto{\pgfqpoint{9.249994in}{1.126630in}}%
\pgfusepath{stroke}%
\end{pgfscope}%
\begin{pgfscope}%
\pgfpathrectangle{\pgfqpoint{1.286132in}{0.839159in}}{\pgfqpoint{12.053712in}{5.967710in}}%
\pgfusepath{clip}%
\pgfsetbuttcap%
\pgfsetroundjoin%
\pgfsetlinewidth{1.505625pt}%
\definecolor{currentstroke}{rgb}{0.580392,0.403922,0.741176}%
\pgfsetstrokecolor{currentstroke}%
\pgfsetdash{}{0pt}%
\pgfpathmoveto{\pgfqpoint{9.360680in}{1.122968in}}%
\pgfpathlineto{\pgfqpoint{9.360680in}{1.134254in}}%
\pgfusepath{stroke}%
\end{pgfscope}%
\begin{pgfscope}%
\pgfpathrectangle{\pgfqpoint{1.286132in}{0.839159in}}{\pgfqpoint{12.053712in}{5.967710in}}%
\pgfusepath{clip}%
\pgfsetbuttcap%
\pgfsetroundjoin%
\pgfsetlinewidth{1.505625pt}%
\definecolor{currentstroke}{rgb}{0.580392,0.403922,0.741176}%
\pgfsetstrokecolor{currentstroke}%
\pgfsetdash{}{0pt}%
\pgfpathmoveto{\pgfqpoint{9.471366in}{1.123478in}}%
\pgfpathlineto{\pgfqpoint{9.471366in}{1.129536in}}%
\pgfusepath{stroke}%
\end{pgfscope}%
\begin{pgfscope}%
\pgfpathrectangle{\pgfqpoint{1.286132in}{0.839159in}}{\pgfqpoint{12.053712in}{5.967710in}}%
\pgfusepath{clip}%
\pgfsetbuttcap%
\pgfsetroundjoin%
\pgfsetlinewidth{1.505625pt}%
\definecolor{currentstroke}{rgb}{0.580392,0.403922,0.741176}%
\pgfsetstrokecolor{currentstroke}%
\pgfsetdash{}{0pt}%
\pgfpathmoveto{\pgfqpoint{9.582052in}{1.125253in}}%
\pgfpathlineto{\pgfqpoint{9.582052in}{1.125596in}}%
\pgfusepath{stroke}%
\end{pgfscope}%
\begin{pgfscope}%
\pgfpathrectangle{\pgfqpoint{1.286132in}{0.839159in}}{\pgfqpoint{12.053712in}{5.967710in}}%
\pgfusepath{clip}%
\pgfsetbuttcap%
\pgfsetroundjoin%
\pgfsetlinewidth{1.505625pt}%
\definecolor{currentstroke}{rgb}{0.580392,0.403922,0.741176}%
\pgfsetstrokecolor{currentstroke}%
\pgfsetdash{}{0pt}%
\pgfpathmoveto{\pgfqpoint{9.692738in}{1.125583in}}%
\pgfpathlineto{\pgfqpoint{9.692738in}{1.127831in}}%
\pgfusepath{stroke}%
\end{pgfscope}%
\begin{pgfscope}%
\pgfpathrectangle{\pgfqpoint{1.286132in}{0.839159in}}{\pgfqpoint{12.053712in}{5.967710in}}%
\pgfusepath{clip}%
\pgfsetbuttcap%
\pgfsetroundjoin%
\pgfsetlinewidth{1.505625pt}%
\definecolor{currentstroke}{rgb}{0.580392,0.403922,0.741176}%
\pgfsetstrokecolor{currentstroke}%
\pgfsetdash{}{0pt}%
\pgfpathmoveto{\pgfqpoint{9.803424in}{1.126682in}}%
\pgfpathlineto{\pgfqpoint{9.803424in}{1.131756in}}%
\pgfusepath{stroke}%
\end{pgfscope}%
\begin{pgfscope}%
\pgfpathrectangle{\pgfqpoint{1.286132in}{0.839159in}}{\pgfqpoint{12.053712in}{5.967710in}}%
\pgfusepath{clip}%
\pgfsetbuttcap%
\pgfsetroundjoin%
\pgfsetlinewidth{1.505625pt}%
\definecolor{currentstroke}{rgb}{0.580392,0.403922,0.741176}%
\pgfsetstrokecolor{currentstroke}%
\pgfsetdash{}{0pt}%
\pgfpathmoveto{\pgfqpoint{9.914110in}{1.124694in}}%
\pgfpathlineto{\pgfqpoint{9.914110in}{1.127589in}}%
\pgfusepath{stroke}%
\end{pgfscope}%
\begin{pgfscope}%
\pgfpathrectangle{\pgfqpoint{1.286132in}{0.839159in}}{\pgfqpoint{12.053712in}{5.967710in}}%
\pgfusepath{clip}%
\pgfsetbuttcap%
\pgfsetroundjoin%
\pgfsetlinewidth{1.505625pt}%
\definecolor{currentstroke}{rgb}{0.580392,0.403922,0.741176}%
\pgfsetstrokecolor{currentstroke}%
\pgfsetdash{}{0pt}%
\pgfpathmoveto{\pgfqpoint{10.024796in}{1.126020in}}%
\pgfpathlineto{\pgfqpoint{10.024796in}{1.128738in}}%
\pgfusepath{stroke}%
\end{pgfscope}%
\begin{pgfscope}%
\pgfpathrectangle{\pgfqpoint{1.286132in}{0.839159in}}{\pgfqpoint{12.053712in}{5.967710in}}%
\pgfusepath{clip}%
\pgfsetbuttcap%
\pgfsetroundjoin%
\pgfsetlinewidth{1.505625pt}%
\definecolor{currentstroke}{rgb}{0.580392,0.403922,0.741176}%
\pgfsetstrokecolor{currentstroke}%
\pgfsetdash{}{0pt}%
\pgfpathmoveto{\pgfqpoint{10.135482in}{1.125964in}}%
\pgfpathlineto{\pgfqpoint{10.135482in}{1.128963in}}%
\pgfusepath{stroke}%
\end{pgfscope}%
\begin{pgfscope}%
\pgfpathrectangle{\pgfqpoint{1.286132in}{0.839159in}}{\pgfqpoint{12.053712in}{5.967710in}}%
\pgfusepath{clip}%
\pgfsetbuttcap%
\pgfsetroundjoin%
\pgfsetlinewidth{1.505625pt}%
\definecolor{currentstroke}{rgb}{0.580392,0.403922,0.741176}%
\pgfsetstrokecolor{currentstroke}%
\pgfsetdash{}{0pt}%
\pgfpathmoveto{\pgfqpoint{10.246168in}{1.125730in}}%
\pgfpathlineto{\pgfqpoint{10.246168in}{1.127499in}}%
\pgfusepath{stroke}%
\end{pgfscope}%
\begin{pgfscope}%
\pgfpathrectangle{\pgfqpoint{1.286132in}{0.839159in}}{\pgfqpoint{12.053712in}{5.967710in}}%
\pgfusepath{clip}%
\pgfsetbuttcap%
\pgfsetroundjoin%
\pgfsetlinewidth{1.505625pt}%
\definecolor{currentstroke}{rgb}{0.580392,0.403922,0.741176}%
\pgfsetstrokecolor{currentstroke}%
\pgfsetdash{}{0pt}%
\pgfpathmoveto{\pgfqpoint{10.356854in}{1.126099in}}%
\pgfpathlineto{\pgfqpoint{10.356854in}{1.128514in}}%
\pgfusepath{stroke}%
\end{pgfscope}%
\begin{pgfscope}%
\pgfpathrectangle{\pgfqpoint{1.286132in}{0.839159in}}{\pgfqpoint{12.053712in}{5.967710in}}%
\pgfusepath{clip}%
\pgfsetbuttcap%
\pgfsetroundjoin%
\pgfsetlinewidth{1.505625pt}%
\definecolor{currentstroke}{rgb}{0.580392,0.403922,0.741176}%
\pgfsetstrokecolor{currentstroke}%
\pgfsetdash{}{0pt}%
\pgfpathmoveto{\pgfqpoint{10.467540in}{1.125794in}}%
\pgfpathlineto{\pgfqpoint{10.467540in}{1.127816in}}%
\pgfusepath{stroke}%
\end{pgfscope}%
\begin{pgfscope}%
\pgfpathrectangle{\pgfqpoint{1.286132in}{0.839159in}}{\pgfqpoint{12.053712in}{5.967710in}}%
\pgfusepath{clip}%
\pgfsetbuttcap%
\pgfsetroundjoin%
\pgfsetlinewidth{1.505625pt}%
\definecolor{currentstroke}{rgb}{0.580392,0.403922,0.741176}%
\pgfsetstrokecolor{currentstroke}%
\pgfsetdash{}{0pt}%
\pgfpathmoveto{\pgfqpoint{10.578226in}{1.125546in}}%
\pgfpathlineto{\pgfqpoint{10.578226in}{1.130673in}}%
\pgfusepath{stroke}%
\end{pgfscope}%
\begin{pgfscope}%
\pgfpathrectangle{\pgfqpoint{1.286132in}{0.839159in}}{\pgfqpoint{12.053712in}{5.967710in}}%
\pgfusepath{clip}%
\pgfsetbuttcap%
\pgfsetroundjoin%
\pgfsetlinewidth{1.505625pt}%
\definecolor{currentstroke}{rgb}{0.580392,0.403922,0.741176}%
\pgfsetstrokecolor{currentstroke}%
\pgfsetdash{}{0pt}%
\pgfpathmoveto{\pgfqpoint{10.688913in}{1.127020in}}%
\pgfpathlineto{\pgfqpoint{10.688913in}{1.130107in}}%
\pgfusepath{stroke}%
\end{pgfscope}%
\begin{pgfscope}%
\pgfpathrectangle{\pgfqpoint{1.286132in}{0.839159in}}{\pgfqpoint{12.053712in}{5.967710in}}%
\pgfusepath{clip}%
\pgfsetbuttcap%
\pgfsetroundjoin%
\pgfsetlinewidth{1.505625pt}%
\definecolor{currentstroke}{rgb}{0.580392,0.403922,0.741176}%
\pgfsetstrokecolor{currentstroke}%
\pgfsetdash{}{0pt}%
\pgfpathmoveto{\pgfqpoint{10.799599in}{1.126673in}}%
\pgfpathlineto{\pgfqpoint{10.799599in}{1.129050in}}%
\pgfusepath{stroke}%
\end{pgfscope}%
\begin{pgfscope}%
\pgfpathrectangle{\pgfqpoint{1.286132in}{0.839159in}}{\pgfqpoint{12.053712in}{5.967710in}}%
\pgfusepath{clip}%
\pgfsetbuttcap%
\pgfsetroundjoin%
\pgfsetlinewidth{1.505625pt}%
\definecolor{currentstroke}{rgb}{0.580392,0.403922,0.741176}%
\pgfsetstrokecolor{currentstroke}%
\pgfsetdash{}{0pt}%
\pgfpathmoveto{\pgfqpoint{10.910285in}{1.126745in}}%
\pgfpathlineto{\pgfqpoint{10.910285in}{1.130990in}}%
\pgfusepath{stroke}%
\end{pgfscope}%
\begin{pgfscope}%
\pgfpathrectangle{\pgfqpoint{1.286132in}{0.839159in}}{\pgfqpoint{12.053712in}{5.967710in}}%
\pgfusepath{clip}%
\pgfsetbuttcap%
\pgfsetroundjoin%
\pgfsetlinewidth{1.505625pt}%
\definecolor{currentstroke}{rgb}{0.580392,0.403922,0.741176}%
\pgfsetstrokecolor{currentstroke}%
\pgfsetdash{}{0pt}%
\pgfpathmoveto{\pgfqpoint{11.020971in}{1.126930in}}%
\pgfpathlineto{\pgfqpoint{11.020971in}{1.130099in}}%
\pgfusepath{stroke}%
\end{pgfscope}%
\begin{pgfscope}%
\pgfpathrectangle{\pgfqpoint{1.286132in}{0.839159in}}{\pgfqpoint{12.053712in}{5.967710in}}%
\pgfusepath{clip}%
\pgfsetbuttcap%
\pgfsetroundjoin%
\pgfsetlinewidth{1.505625pt}%
\definecolor{currentstroke}{rgb}{0.580392,0.403922,0.741176}%
\pgfsetstrokecolor{currentstroke}%
\pgfsetdash{}{0pt}%
\pgfpathmoveto{\pgfqpoint{11.131657in}{1.126770in}}%
\pgfpathlineto{\pgfqpoint{11.131657in}{1.129926in}}%
\pgfusepath{stroke}%
\end{pgfscope}%
\begin{pgfscope}%
\pgfpathrectangle{\pgfqpoint{1.286132in}{0.839159in}}{\pgfqpoint{12.053712in}{5.967710in}}%
\pgfusepath{clip}%
\pgfsetbuttcap%
\pgfsetroundjoin%
\pgfsetlinewidth{1.505625pt}%
\definecolor{currentstroke}{rgb}{0.580392,0.403922,0.741176}%
\pgfsetstrokecolor{currentstroke}%
\pgfsetdash{}{0pt}%
\pgfpathmoveto{\pgfqpoint{11.242343in}{1.130231in}}%
\pgfpathlineto{\pgfqpoint{11.242343in}{1.135171in}}%
\pgfusepath{stroke}%
\end{pgfscope}%
\begin{pgfscope}%
\pgfpathrectangle{\pgfqpoint{1.286132in}{0.839159in}}{\pgfqpoint{12.053712in}{5.967710in}}%
\pgfusepath{clip}%
\pgfsetbuttcap%
\pgfsetroundjoin%
\pgfsetlinewidth{1.505625pt}%
\definecolor{currentstroke}{rgb}{0.580392,0.403922,0.741176}%
\pgfsetstrokecolor{currentstroke}%
\pgfsetdash{}{0pt}%
\pgfpathmoveto{\pgfqpoint{11.353029in}{1.127344in}}%
\pgfpathlineto{\pgfqpoint{11.353029in}{1.129924in}}%
\pgfusepath{stroke}%
\end{pgfscope}%
\begin{pgfscope}%
\pgfpathrectangle{\pgfqpoint{1.286132in}{0.839159in}}{\pgfqpoint{12.053712in}{5.967710in}}%
\pgfusepath{clip}%
\pgfsetbuttcap%
\pgfsetroundjoin%
\pgfsetlinewidth{1.505625pt}%
\definecolor{currentstroke}{rgb}{0.580392,0.403922,0.741176}%
\pgfsetstrokecolor{currentstroke}%
\pgfsetdash{}{0pt}%
\pgfpathmoveto{\pgfqpoint{11.463715in}{1.126692in}}%
\pgfpathlineto{\pgfqpoint{11.463715in}{1.128506in}}%
\pgfusepath{stroke}%
\end{pgfscope}%
\begin{pgfscope}%
\pgfpathrectangle{\pgfqpoint{1.286132in}{0.839159in}}{\pgfqpoint{12.053712in}{5.967710in}}%
\pgfusepath{clip}%
\pgfsetbuttcap%
\pgfsetroundjoin%
\pgfsetlinewidth{1.505625pt}%
\definecolor{currentstroke}{rgb}{0.580392,0.403922,0.741176}%
\pgfsetstrokecolor{currentstroke}%
\pgfsetdash{}{0pt}%
\pgfpathmoveto{\pgfqpoint{11.574401in}{1.128216in}}%
\pgfpathlineto{\pgfqpoint{11.574401in}{1.130846in}}%
\pgfusepath{stroke}%
\end{pgfscope}%
\begin{pgfscope}%
\pgfpathrectangle{\pgfqpoint{1.286132in}{0.839159in}}{\pgfqpoint{12.053712in}{5.967710in}}%
\pgfusepath{clip}%
\pgfsetbuttcap%
\pgfsetroundjoin%
\pgfsetlinewidth{1.505625pt}%
\definecolor{currentstroke}{rgb}{0.580392,0.403922,0.741176}%
\pgfsetstrokecolor{currentstroke}%
\pgfsetdash{}{0pt}%
\pgfpathmoveto{\pgfqpoint{11.685087in}{1.127760in}}%
\pgfpathlineto{\pgfqpoint{11.685087in}{1.131735in}}%
\pgfusepath{stroke}%
\end{pgfscope}%
\begin{pgfscope}%
\pgfpathrectangle{\pgfqpoint{1.286132in}{0.839159in}}{\pgfqpoint{12.053712in}{5.967710in}}%
\pgfusepath{clip}%
\pgfsetbuttcap%
\pgfsetroundjoin%
\pgfsetlinewidth{1.505625pt}%
\definecolor{currentstroke}{rgb}{0.580392,0.403922,0.741176}%
\pgfsetstrokecolor{currentstroke}%
\pgfsetdash{}{0pt}%
\pgfpathmoveto{\pgfqpoint{11.795773in}{1.127595in}}%
\pgfpathlineto{\pgfqpoint{11.795773in}{1.135597in}}%
\pgfusepath{stroke}%
\end{pgfscope}%
\begin{pgfscope}%
\pgfpathrectangle{\pgfqpoint{1.286132in}{0.839159in}}{\pgfqpoint{12.053712in}{5.967710in}}%
\pgfusepath{clip}%
\pgfsetbuttcap%
\pgfsetroundjoin%
\pgfsetlinewidth{1.505625pt}%
\definecolor{currentstroke}{rgb}{0.580392,0.403922,0.741176}%
\pgfsetstrokecolor{currentstroke}%
\pgfsetdash{}{0pt}%
\pgfpathmoveto{\pgfqpoint{11.906459in}{1.127291in}}%
\pgfpathlineto{\pgfqpoint{11.906459in}{1.133593in}}%
\pgfusepath{stroke}%
\end{pgfscope}%
\begin{pgfscope}%
\pgfpathrectangle{\pgfqpoint{1.286132in}{0.839159in}}{\pgfqpoint{12.053712in}{5.967710in}}%
\pgfusepath{clip}%
\pgfsetbuttcap%
\pgfsetroundjoin%
\pgfsetlinewidth{1.505625pt}%
\definecolor{currentstroke}{rgb}{0.580392,0.403922,0.741176}%
\pgfsetstrokecolor{currentstroke}%
\pgfsetdash{}{0pt}%
\pgfpathmoveto{\pgfqpoint{12.017145in}{1.128584in}}%
\pgfpathlineto{\pgfqpoint{12.017145in}{1.131542in}}%
\pgfusepath{stroke}%
\end{pgfscope}%
\begin{pgfscope}%
\pgfpathrectangle{\pgfqpoint{1.286132in}{0.839159in}}{\pgfqpoint{12.053712in}{5.967710in}}%
\pgfusepath{clip}%
\pgfsetbuttcap%
\pgfsetroundjoin%
\pgfsetlinewidth{1.505625pt}%
\definecolor{currentstroke}{rgb}{0.580392,0.403922,0.741176}%
\pgfsetstrokecolor{currentstroke}%
\pgfsetdash{}{0pt}%
\pgfpathmoveto{\pgfqpoint{12.127831in}{1.116028in}}%
\pgfpathlineto{\pgfqpoint{12.127831in}{1.175676in}}%
\pgfusepath{stroke}%
\end{pgfscope}%
\begin{pgfscope}%
\pgfpathrectangle{\pgfqpoint{1.286132in}{0.839159in}}{\pgfqpoint{12.053712in}{5.967710in}}%
\pgfusepath{clip}%
\pgfsetbuttcap%
\pgfsetroundjoin%
\pgfsetlinewidth{1.505625pt}%
\definecolor{currentstroke}{rgb}{0.580392,0.403922,0.741176}%
\pgfsetstrokecolor{currentstroke}%
\pgfsetdash{}{0pt}%
\pgfpathmoveto{\pgfqpoint{12.238517in}{1.127385in}}%
\pgfpathlineto{\pgfqpoint{12.238517in}{1.131337in}}%
\pgfusepath{stroke}%
\end{pgfscope}%
\begin{pgfscope}%
\pgfpathrectangle{\pgfqpoint{1.286132in}{0.839159in}}{\pgfqpoint{12.053712in}{5.967710in}}%
\pgfusepath{clip}%
\pgfsetbuttcap%
\pgfsetroundjoin%
\pgfsetlinewidth{1.505625pt}%
\definecolor{currentstroke}{rgb}{0.580392,0.403922,0.741176}%
\pgfsetstrokecolor{currentstroke}%
\pgfsetdash{}{0pt}%
\pgfpathmoveto{\pgfqpoint{12.349203in}{1.129006in}}%
\pgfpathlineto{\pgfqpoint{12.349203in}{1.133006in}}%
\pgfusepath{stroke}%
\end{pgfscope}%
\begin{pgfscope}%
\pgfpathrectangle{\pgfqpoint{1.286132in}{0.839159in}}{\pgfqpoint{12.053712in}{5.967710in}}%
\pgfusepath{clip}%
\pgfsetbuttcap%
\pgfsetroundjoin%
\pgfsetlinewidth{1.505625pt}%
\definecolor{currentstroke}{rgb}{0.580392,0.403922,0.741176}%
\pgfsetstrokecolor{currentstroke}%
\pgfsetdash{}{0pt}%
\pgfpathmoveto{\pgfqpoint{12.459890in}{1.129288in}}%
\pgfpathlineto{\pgfqpoint{12.459890in}{1.137423in}}%
\pgfusepath{stroke}%
\end{pgfscope}%
\begin{pgfscope}%
\pgfpathrectangle{\pgfqpoint{1.286132in}{0.839159in}}{\pgfqpoint{12.053712in}{5.967710in}}%
\pgfusepath{clip}%
\pgfsetbuttcap%
\pgfsetroundjoin%
\pgfsetlinewidth{1.505625pt}%
\definecolor{currentstroke}{rgb}{0.580392,0.403922,0.741176}%
\pgfsetstrokecolor{currentstroke}%
\pgfsetdash{}{0pt}%
\pgfpathmoveto{\pgfqpoint{12.570576in}{1.127907in}}%
\pgfpathlineto{\pgfqpoint{12.570576in}{1.131453in}}%
\pgfusepath{stroke}%
\end{pgfscope}%
\begin{pgfscope}%
\pgfpathrectangle{\pgfqpoint{1.286132in}{0.839159in}}{\pgfqpoint{12.053712in}{5.967710in}}%
\pgfusepath{clip}%
\pgfsetbuttcap%
\pgfsetroundjoin%
\pgfsetlinewidth{1.505625pt}%
\definecolor{currentstroke}{rgb}{0.580392,0.403922,0.741176}%
\pgfsetstrokecolor{currentstroke}%
\pgfsetdash{}{0pt}%
\pgfpathmoveto{\pgfqpoint{12.681262in}{1.128274in}}%
\pgfpathlineto{\pgfqpoint{12.681262in}{1.130993in}}%
\pgfusepath{stroke}%
\end{pgfscope}%
\begin{pgfscope}%
\pgfpathrectangle{\pgfqpoint{1.286132in}{0.839159in}}{\pgfqpoint{12.053712in}{5.967710in}}%
\pgfusepath{clip}%
\pgfsetbuttcap%
\pgfsetroundjoin%
\pgfsetlinewidth{1.505625pt}%
\definecolor{currentstroke}{rgb}{0.580392,0.403922,0.741176}%
\pgfsetstrokecolor{currentstroke}%
\pgfsetdash{}{0pt}%
\pgfpathmoveto{\pgfqpoint{12.791948in}{1.128337in}}%
\pgfpathlineto{\pgfqpoint{12.791948in}{1.132297in}}%
\pgfusepath{stroke}%
\end{pgfscope}%
\begin{pgfscope}%
\pgfpathrectangle{\pgfqpoint{1.286132in}{0.839159in}}{\pgfqpoint{12.053712in}{5.967710in}}%
\pgfusepath{clip}%
\pgfsetbuttcap%
\pgfsetroundjoin%
\pgfsetlinewidth{1.505625pt}%
\definecolor{currentstroke}{rgb}{0.549020,0.337255,0.294118}%
\pgfsetstrokecolor{currentstroke}%
\pgfsetdash{}{0pt}%
\pgfpathmoveto{\pgfqpoint{1.834028in}{1.119213in}}%
\pgfpathlineto{\pgfqpoint{1.834028in}{1.119218in}}%
\pgfusepath{stroke}%
\end{pgfscope}%
\begin{pgfscope}%
\pgfpathrectangle{\pgfqpoint{1.286132in}{0.839159in}}{\pgfqpoint{12.053712in}{5.967710in}}%
\pgfusepath{clip}%
\pgfsetbuttcap%
\pgfsetroundjoin%
\pgfsetlinewidth{1.505625pt}%
\definecolor{currentstroke}{rgb}{0.549020,0.337255,0.294118}%
\pgfsetstrokecolor{currentstroke}%
\pgfsetdash{}{0pt}%
\pgfpathmoveto{\pgfqpoint{1.944714in}{1.119223in}}%
\pgfpathlineto{\pgfqpoint{1.944714in}{1.119227in}}%
\pgfusepath{stroke}%
\end{pgfscope}%
\begin{pgfscope}%
\pgfpathrectangle{\pgfqpoint{1.286132in}{0.839159in}}{\pgfqpoint{12.053712in}{5.967710in}}%
\pgfusepath{clip}%
\pgfsetbuttcap%
\pgfsetroundjoin%
\pgfsetlinewidth{1.505625pt}%
\definecolor{currentstroke}{rgb}{0.549020,0.337255,0.294118}%
\pgfsetstrokecolor{currentstroke}%
\pgfsetdash{}{0pt}%
\pgfpathmoveto{\pgfqpoint{2.055400in}{1.119259in}}%
\pgfpathlineto{\pgfqpoint{2.055400in}{1.119267in}}%
\pgfusepath{stroke}%
\end{pgfscope}%
\begin{pgfscope}%
\pgfpathrectangle{\pgfqpoint{1.286132in}{0.839159in}}{\pgfqpoint{12.053712in}{5.967710in}}%
\pgfusepath{clip}%
\pgfsetbuttcap%
\pgfsetroundjoin%
\pgfsetlinewidth{1.505625pt}%
\definecolor{currentstroke}{rgb}{0.549020,0.337255,0.294118}%
\pgfsetstrokecolor{currentstroke}%
\pgfsetdash{}{0pt}%
\pgfpathmoveto{\pgfqpoint{2.166086in}{1.119293in}}%
\pgfpathlineto{\pgfqpoint{2.166086in}{1.119423in}}%
\pgfusepath{stroke}%
\end{pgfscope}%
\begin{pgfscope}%
\pgfpathrectangle{\pgfqpoint{1.286132in}{0.839159in}}{\pgfqpoint{12.053712in}{5.967710in}}%
\pgfusepath{clip}%
\pgfsetbuttcap%
\pgfsetroundjoin%
\pgfsetlinewidth{1.505625pt}%
\definecolor{currentstroke}{rgb}{0.549020,0.337255,0.294118}%
\pgfsetstrokecolor{currentstroke}%
\pgfsetdash{}{0pt}%
\pgfpathmoveto{\pgfqpoint{2.276772in}{1.119531in}}%
\pgfpathlineto{\pgfqpoint{2.276772in}{1.119674in}}%
\pgfusepath{stroke}%
\end{pgfscope}%
\begin{pgfscope}%
\pgfpathrectangle{\pgfqpoint{1.286132in}{0.839159in}}{\pgfqpoint{12.053712in}{5.967710in}}%
\pgfusepath{clip}%
\pgfsetbuttcap%
\pgfsetroundjoin%
\pgfsetlinewidth{1.505625pt}%
\definecolor{currentstroke}{rgb}{0.549020,0.337255,0.294118}%
\pgfsetstrokecolor{currentstroke}%
\pgfsetdash{}{0pt}%
\pgfpathmoveto{\pgfqpoint{2.387458in}{1.119713in}}%
\pgfpathlineto{\pgfqpoint{2.387458in}{1.119757in}}%
\pgfusepath{stroke}%
\end{pgfscope}%
\begin{pgfscope}%
\pgfpathrectangle{\pgfqpoint{1.286132in}{0.839159in}}{\pgfqpoint{12.053712in}{5.967710in}}%
\pgfusepath{clip}%
\pgfsetbuttcap%
\pgfsetroundjoin%
\pgfsetlinewidth{1.505625pt}%
\definecolor{currentstroke}{rgb}{0.549020,0.337255,0.294118}%
\pgfsetstrokecolor{currentstroke}%
\pgfsetdash{}{0pt}%
\pgfpathmoveto{\pgfqpoint{2.498144in}{1.120027in}}%
\pgfpathlineto{\pgfqpoint{2.498144in}{1.120640in}}%
\pgfusepath{stroke}%
\end{pgfscope}%
\begin{pgfscope}%
\pgfpathrectangle{\pgfqpoint{1.286132in}{0.839159in}}{\pgfqpoint{12.053712in}{5.967710in}}%
\pgfusepath{clip}%
\pgfsetbuttcap%
\pgfsetroundjoin%
\pgfsetlinewidth{1.505625pt}%
\definecolor{currentstroke}{rgb}{0.549020,0.337255,0.294118}%
\pgfsetstrokecolor{currentstroke}%
\pgfsetdash{}{0pt}%
\pgfpathmoveto{\pgfqpoint{2.608830in}{1.118149in}}%
\pgfpathlineto{\pgfqpoint{2.608830in}{1.126786in}}%
\pgfusepath{stroke}%
\end{pgfscope}%
\begin{pgfscope}%
\pgfpathrectangle{\pgfqpoint{1.286132in}{0.839159in}}{\pgfqpoint{12.053712in}{5.967710in}}%
\pgfusepath{clip}%
\pgfsetbuttcap%
\pgfsetroundjoin%
\pgfsetlinewidth{1.505625pt}%
\definecolor{currentstroke}{rgb}{0.549020,0.337255,0.294118}%
\pgfsetstrokecolor{currentstroke}%
\pgfsetdash{}{0pt}%
\pgfpathmoveto{\pgfqpoint{2.719516in}{1.120748in}}%
\pgfpathlineto{\pgfqpoint{2.719516in}{1.121199in}}%
\pgfusepath{stroke}%
\end{pgfscope}%
\begin{pgfscope}%
\pgfpathrectangle{\pgfqpoint{1.286132in}{0.839159in}}{\pgfqpoint{12.053712in}{5.967710in}}%
\pgfusepath{clip}%
\pgfsetbuttcap%
\pgfsetroundjoin%
\pgfsetlinewidth{1.505625pt}%
\definecolor{currentstroke}{rgb}{0.549020,0.337255,0.294118}%
\pgfsetstrokecolor{currentstroke}%
\pgfsetdash{}{0pt}%
\pgfpathmoveto{\pgfqpoint{2.830202in}{1.120151in}}%
\pgfpathlineto{\pgfqpoint{2.830202in}{1.124094in}}%
\pgfusepath{stroke}%
\end{pgfscope}%
\begin{pgfscope}%
\pgfpathrectangle{\pgfqpoint{1.286132in}{0.839159in}}{\pgfqpoint{12.053712in}{5.967710in}}%
\pgfusepath{clip}%
\pgfsetbuttcap%
\pgfsetroundjoin%
\pgfsetlinewidth{1.505625pt}%
\definecolor{currentstroke}{rgb}{0.549020,0.337255,0.294118}%
\pgfsetstrokecolor{currentstroke}%
\pgfsetdash{}{0pt}%
\pgfpathmoveto{\pgfqpoint{2.940888in}{1.121995in}}%
\pgfpathlineto{\pgfqpoint{2.940888in}{1.122855in}}%
\pgfusepath{stroke}%
\end{pgfscope}%
\begin{pgfscope}%
\pgfpathrectangle{\pgfqpoint{1.286132in}{0.839159in}}{\pgfqpoint{12.053712in}{5.967710in}}%
\pgfusepath{clip}%
\pgfsetbuttcap%
\pgfsetroundjoin%
\pgfsetlinewidth{1.505625pt}%
\definecolor{currentstroke}{rgb}{0.549020,0.337255,0.294118}%
\pgfsetstrokecolor{currentstroke}%
\pgfsetdash{}{0pt}%
\pgfpathmoveto{\pgfqpoint{3.051574in}{1.121813in}}%
\pgfpathlineto{\pgfqpoint{3.051574in}{1.123950in}}%
\pgfusepath{stroke}%
\end{pgfscope}%
\begin{pgfscope}%
\pgfpathrectangle{\pgfqpoint{1.286132in}{0.839159in}}{\pgfqpoint{12.053712in}{5.967710in}}%
\pgfusepath{clip}%
\pgfsetbuttcap%
\pgfsetroundjoin%
\pgfsetlinewidth{1.505625pt}%
\definecolor{currentstroke}{rgb}{0.549020,0.337255,0.294118}%
\pgfsetstrokecolor{currentstroke}%
\pgfsetdash{}{0pt}%
\pgfpathmoveto{\pgfqpoint{3.162260in}{1.122570in}}%
\pgfpathlineto{\pgfqpoint{3.162260in}{1.125005in}}%
\pgfusepath{stroke}%
\end{pgfscope}%
\begin{pgfscope}%
\pgfpathrectangle{\pgfqpoint{1.286132in}{0.839159in}}{\pgfqpoint{12.053712in}{5.967710in}}%
\pgfusepath{clip}%
\pgfsetbuttcap%
\pgfsetroundjoin%
\pgfsetlinewidth{1.505625pt}%
\definecolor{currentstroke}{rgb}{0.549020,0.337255,0.294118}%
\pgfsetstrokecolor{currentstroke}%
\pgfsetdash{}{0pt}%
\pgfpathmoveto{\pgfqpoint{3.272946in}{1.123689in}}%
\pgfpathlineto{\pgfqpoint{3.272946in}{1.124855in}}%
\pgfusepath{stroke}%
\end{pgfscope}%
\begin{pgfscope}%
\pgfpathrectangle{\pgfqpoint{1.286132in}{0.839159in}}{\pgfqpoint{12.053712in}{5.967710in}}%
\pgfusepath{clip}%
\pgfsetbuttcap%
\pgfsetroundjoin%
\pgfsetlinewidth{1.505625pt}%
\definecolor{currentstroke}{rgb}{0.549020,0.337255,0.294118}%
\pgfsetstrokecolor{currentstroke}%
\pgfsetdash{}{0pt}%
\pgfpathmoveto{\pgfqpoint{3.383632in}{1.125406in}}%
\pgfpathlineto{\pgfqpoint{3.383632in}{1.127766in}}%
\pgfusepath{stroke}%
\end{pgfscope}%
\begin{pgfscope}%
\pgfpathrectangle{\pgfqpoint{1.286132in}{0.839159in}}{\pgfqpoint{12.053712in}{5.967710in}}%
\pgfusepath{clip}%
\pgfsetbuttcap%
\pgfsetroundjoin%
\pgfsetlinewidth{1.505625pt}%
\definecolor{currentstroke}{rgb}{0.549020,0.337255,0.294118}%
\pgfsetstrokecolor{currentstroke}%
\pgfsetdash{}{0pt}%
\pgfpathmoveto{\pgfqpoint{3.494319in}{1.125740in}}%
\pgfpathlineto{\pgfqpoint{3.494319in}{1.127649in}}%
\pgfusepath{stroke}%
\end{pgfscope}%
\begin{pgfscope}%
\pgfpathrectangle{\pgfqpoint{1.286132in}{0.839159in}}{\pgfqpoint{12.053712in}{5.967710in}}%
\pgfusepath{clip}%
\pgfsetbuttcap%
\pgfsetroundjoin%
\pgfsetlinewidth{1.505625pt}%
\definecolor{currentstroke}{rgb}{0.549020,0.337255,0.294118}%
\pgfsetstrokecolor{currentstroke}%
\pgfsetdash{}{0pt}%
\pgfpathmoveto{\pgfqpoint{3.605005in}{1.127622in}}%
\pgfpathlineto{\pgfqpoint{3.605005in}{1.131496in}}%
\pgfusepath{stroke}%
\end{pgfscope}%
\begin{pgfscope}%
\pgfpathrectangle{\pgfqpoint{1.286132in}{0.839159in}}{\pgfqpoint{12.053712in}{5.967710in}}%
\pgfusepath{clip}%
\pgfsetbuttcap%
\pgfsetroundjoin%
\pgfsetlinewidth{1.505625pt}%
\definecolor{currentstroke}{rgb}{0.549020,0.337255,0.294118}%
\pgfsetstrokecolor{currentstroke}%
\pgfsetdash{}{0pt}%
\pgfpathmoveto{\pgfqpoint{3.715691in}{1.127688in}}%
\pgfpathlineto{\pgfqpoint{3.715691in}{1.130120in}}%
\pgfusepath{stroke}%
\end{pgfscope}%
\begin{pgfscope}%
\pgfpathrectangle{\pgfqpoint{1.286132in}{0.839159in}}{\pgfqpoint{12.053712in}{5.967710in}}%
\pgfusepath{clip}%
\pgfsetbuttcap%
\pgfsetroundjoin%
\pgfsetlinewidth{1.505625pt}%
\definecolor{currentstroke}{rgb}{0.549020,0.337255,0.294118}%
\pgfsetstrokecolor{currentstroke}%
\pgfsetdash{}{0pt}%
\pgfpathmoveto{\pgfqpoint{3.826377in}{1.130747in}}%
\pgfpathlineto{\pgfqpoint{3.826377in}{1.134486in}}%
\pgfusepath{stroke}%
\end{pgfscope}%
\begin{pgfscope}%
\pgfpathrectangle{\pgfqpoint{1.286132in}{0.839159in}}{\pgfqpoint{12.053712in}{5.967710in}}%
\pgfusepath{clip}%
\pgfsetbuttcap%
\pgfsetroundjoin%
\pgfsetlinewidth{1.505625pt}%
\definecolor{currentstroke}{rgb}{0.549020,0.337255,0.294118}%
\pgfsetstrokecolor{currentstroke}%
\pgfsetdash{}{0pt}%
\pgfpathmoveto{\pgfqpoint{3.937063in}{1.130097in}}%
\pgfpathlineto{\pgfqpoint{3.937063in}{1.136644in}}%
\pgfusepath{stroke}%
\end{pgfscope}%
\begin{pgfscope}%
\pgfpathrectangle{\pgfqpoint{1.286132in}{0.839159in}}{\pgfqpoint{12.053712in}{5.967710in}}%
\pgfusepath{clip}%
\pgfsetbuttcap%
\pgfsetroundjoin%
\pgfsetlinewidth{1.505625pt}%
\definecolor{currentstroke}{rgb}{0.549020,0.337255,0.294118}%
\pgfsetstrokecolor{currentstroke}%
\pgfsetdash{}{0pt}%
\pgfpathmoveto{\pgfqpoint{4.047749in}{1.133732in}}%
\pgfpathlineto{\pgfqpoint{4.047749in}{1.139074in}}%
\pgfusepath{stroke}%
\end{pgfscope}%
\begin{pgfscope}%
\pgfpathrectangle{\pgfqpoint{1.286132in}{0.839159in}}{\pgfqpoint{12.053712in}{5.967710in}}%
\pgfusepath{clip}%
\pgfsetbuttcap%
\pgfsetroundjoin%
\pgfsetlinewidth{1.505625pt}%
\definecolor{currentstroke}{rgb}{0.549020,0.337255,0.294118}%
\pgfsetstrokecolor{currentstroke}%
\pgfsetdash{}{0pt}%
\pgfpathmoveto{\pgfqpoint{4.158435in}{1.131281in}}%
\pgfpathlineto{\pgfqpoint{4.158435in}{1.138050in}}%
\pgfusepath{stroke}%
\end{pgfscope}%
\begin{pgfscope}%
\pgfpathrectangle{\pgfqpoint{1.286132in}{0.839159in}}{\pgfqpoint{12.053712in}{5.967710in}}%
\pgfusepath{clip}%
\pgfsetbuttcap%
\pgfsetroundjoin%
\pgfsetlinewidth{1.505625pt}%
\definecolor{currentstroke}{rgb}{0.549020,0.337255,0.294118}%
\pgfsetstrokecolor{currentstroke}%
\pgfsetdash{}{0pt}%
\pgfpathmoveto{\pgfqpoint{4.269121in}{1.134226in}}%
\pgfpathlineto{\pgfqpoint{4.269121in}{1.143453in}}%
\pgfusepath{stroke}%
\end{pgfscope}%
\begin{pgfscope}%
\pgfpathrectangle{\pgfqpoint{1.286132in}{0.839159in}}{\pgfqpoint{12.053712in}{5.967710in}}%
\pgfusepath{clip}%
\pgfsetbuttcap%
\pgfsetroundjoin%
\pgfsetlinewidth{1.505625pt}%
\definecolor{currentstroke}{rgb}{0.549020,0.337255,0.294118}%
\pgfsetstrokecolor{currentstroke}%
\pgfsetdash{}{0pt}%
\pgfpathmoveto{\pgfqpoint{4.379807in}{1.134459in}}%
\pgfpathlineto{\pgfqpoint{4.379807in}{1.145466in}}%
\pgfusepath{stroke}%
\end{pgfscope}%
\begin{pgfscope}%
\pgfpathrectangle{\pgfqpoint{1.286132in}{0.839159in}}{\pgfqpoint{12.053712in}{5.967710in}}%
\pgfusepath{clip}%
\pgfsetbuttcap%
\pgfsetroundjoin%
\pgfsetlinewidth{1.505625pt}%
\definecolor{currentstroke}{rgb}{0.549020,0.337255,0.294118}%
\pgfsetstrokecolor{currentstroke}%
\pgfsetdash{}{0pt}%
\pgfpathmoveto{\pgfqpoint{4.490493in}{1.138732in}}%
\pgfpathlineto{\pgfqpoint{4.490493in}{1.144771in}}%
\pgfusepath{stroke}%
\end{pgfscope}%
\begin{pgfscope}%
\pgfpathrectangle{\pgfqpoint{1.286132in}{0.839159in}}{\pgfqpoint{12.053712in}{5.967710in}}%
\pgfusepath{clip}%
\pgfsetbuttcap%
\pgfsetroundjoin%
\pgfsetlinewidth{1.505625pt}%
\definecolor{currentstroke}{rgb}{0.549020,0.337255,0.294118}%
\pgfsetstrokecolor{currentstroke}%
\pgfsetdash{}{0pt}%
\pgfpathmoveto{\pgfqpoint{4.601179in}{1.138632in}}%
\pgfpathlineto{\pgfqpoint{4.601179in}{1.151455in}}%
\pgfusepath{stroke}%
\end{pgfscope}%
\begin{pgfscope}%
\pgfpathrectangle{\pgfqpoint{1.286132in}{0.839159in}}{\pgfqpoint{12.053712in}{5.967710in}}%
\pgfusepath{clip}%
\pgfsetbuttcap%
\pgfsetroundjoin%
\pgfsetlinewidth{1.505625pt}%
\definecolor{currentstroke}{rgb}{0.549020,0.337255,0.294118}%
\pgfsetstrokecolor{currentstroke}%
\pgfsetdash{}{0pt}%
\pgfpathmoveto{\pgfqpoint{4.711865in}{1.143486in}}%
\pgfpathlineto{\pgfqpoint{4.711865in}{1.147505in}}%
\pgfusepath{stroke}%
\end{pgfscope}%
\begin{pgfscope}%
\pgfpathrectangle{\pgfqpoint{1.286132in}{0.839159in}}{\pgfqpoint{12.053712in}{5.967710in}}%
\pgfusepath{clip}%
\pgfsetbuttcap%
\pgfsetroundjoin%
\pgfsetlinewidth{1.505625pt}%
\definecolor{currentstroke}{rgb}{0.549020,0.337255,0.294118}%
\pgfsetstrokecolor{currentstroke}%
\pgfsetdash{}{0pt}%
\pgfpathmoveto{\pgfqpoint{4.822551in}{1.144616in}}%
\pgfpathlineto{\pgfqpoint{4.822551in}{1.161199in}}%
\pgfusepath{stroke}%
\end{pgfscope}%
\begin{pgfscope}%
\pgfpathrectangle{\pgfqpoint{1.286132in}{0.839159in}}{\pgfqpoint{12.053712in}{5.967710in}}%
\pgfusepath{clip}%
\pgfsetbuttcap%
\pgfsetroundjoin%
\pgfsetlinewidth{1.505625pt}%
\definecolor{currentstroke}{rgb}{0.549020,0.337255,0.294118}%
\pgfsetstrokecolor{currentstroke}%
\pgfsetdash{}{0pt}%
\pgfpathmoveto{\pgfqpoint{4.933237in}{1.147143in}}%
\pgfpathlineto{\pgfqpoint{4.933237in}{1.154508in}}%
\pgfusepath{stroke}%
\end{pgfscope}%
\begin{pgfscope}%
\pgfpathrectangle{\pgfqpoint{1.286132in}{0.839159in}}{\pgfqpoint{12.053712in}{5.967710in}}%
\pgfusepath{clip}%
\pgfsetbuttcap%
\pgfsetroundjoin%
\pgfsetlinewidth{1.505625pt}%
\definecolor{currentstroke}{rgb}{0.549020,0.337255,0.294118}%
\pgfsetstrokecolor{currentstroke}%
\pgfsetdash{}{0pt}%
\pgfpathmoveto{\pgfqpoint{5.043923in}{1.142053in}}%
\pgfpathlineto{\pgfqpoint{5.043923in}{1.157654in}}%
\pgfusepath{stroke}%
\end{pgfscope}%
\begin{pgfscope}%
\pgfpathrectangle{\pgfqpoint{1.286132in}{0.839159in}}{\pgfqpoint{12.053712in}{5.967710in}}%
\pgfusepath{clip}%
\pgfsetbuttcap%
\pgfsetroundjoin%
\pgfsetlinewidth{1.505625pt}%
\definecolor{currentstroke}{rgb}{0.549020,0.337255,0.294118}%
\pgfsetstrokecolor{currentstroke}%
\pgfsetdash{}{0pt}%
\pgfpathmoveto{\pgfqpoint{5.154609in}{1.150671in}}%
\pgfpathlineto{\pgfqpoint{5.154609in}{1.159454in}}%
\pgfusepath{stroke}%
\end{pgfscope}%
\begin{pgfscope}%
\pgfpathrectangle{\pgfqpoint{1.286132in}{0.839159in}}{\pgfqpoint{12.053712in}{5.967710in}}%
\pgfusepath{clip}%
\pgfsetbuttcap%
\pgfsetroundjoin%
\pgfsetlinewidth{1.505625pt}%
\definecolor{currentstroke}{rgb}{0.549020,0.337255,0.294118}%
\pgfsetstrokecolor{currentstroke}%
\pgfsetdash{}{0pt}%
\pgfpathmoveto{\pgfqpoint{5.265296in}{1.154540in}}%
\pgfpathlineto{\pgfqpoint{5.265296in}{1.178401in}}%
\pgfusepath{stroke}%
\end{pgfscope}%
\begin{pgfscope}%
\pgfpathrectangle{\pgfqpoint{1.286132in}{0.839159in}}{\pgfqpoint{12.053712in}{5.967710in}}%
\pgfusepath{clip}%
\pgfsetbuttcap%
\pgfsetroundjoin%
\pgfsetlinewidth{1.505625pt}%
\definecolor{currentstroke}{rgb}{0.549020,0.337255,0.294118}%
\pgfsetstrokecolor{currentstroke}%
\pgfsetdash{}{0pt}%
\pgfpathmoveto{\pgfqpoint{5.375982in}{1.158637in}}%
\pgfpathlineto{\pgfqpoint{5.375982in}{1.170680in}}%
\pgfusepath{stroke}%
\end{pgfscope}%
\begin{pgfscope}%
\pgfpathrectangle{\pgfqpoint{1.286132in}{0.839159in}}{\pgfqpoint{12.053712in}{5.967710in}}%
\pgfusepath{clip}%
\pgfsetbuttcap%
\pgfsetroundjoin%
\pgfsetlinewidth{1.505625pt}%
\definecolor{currentstroke}{rgb}{0.549020,0.337255,0.294118}%
\pgfsetstrokecolor{currentstroke}%
\pgfsetdash{}{0pt}%
\pgfpathmoveto{\pgfqpoint{5.486668in}{1.163540in}}%
\pgfpathlineto{\pgfqpoint{5.486668in}{1.183968in}}%
\pgfusepath{stroke}%
\end{pgfscope}%
\begin{pgfscope}%
\pgfpathrectangle{\pgfqpoint{1.286132in}{0.839159in}}{\pgfqpoint{12.053712in}{5.967710in}}%
\pgfusepath{clip}%
\pgfsetbuttcap%
\pgfsetroundjoin%
\pgfsetlinewidth{1.505625pt}%
\definecolor{currentstroke}{rgb}{0.549020,0.337255,0.294118}%
\pgfsetstrokecolor{currentstroke}%
\pgfsetdash{}{0pt}%
\pgfpathmoveto{\pgfqpoint{5.597354in}{1.155719in}}%
\pgfpathlineto{\pgfqpoint{5.597354in}{1.164782in}}%
\pgfusepath{stroke}%
\end{pgfscope}%
\begin{pgfscope}%
\pgfpathrectangle{\pgfqpoint{1.286132in}{0.839159in}}{\pgfqpoint{12.053712in}{5.967710in}}%
\pgfusepath{clip}%
\pgfsetbuttcap%
\pgfsetroundjoin%
\pgfsetlinewidth{1.505625pt}%
\definecolor{currentstroke}{rgb}{0.549020,0.337255,0.294118}%
\pgfsetstrokecolor{currentstroke}%
\pgfsetdash{}{0pt}%
\pgfpathmoveto{\pgfqpoint{5.708040in}{1.177142in}}%
\pgfpathlineto{\pgfqpoint{5.708040in}{1.219511in}}%
\pgfusepath{stroke}%
\end{pgfscope}%
\begin{pgfscope}%
\pgfpathrectangle{\pgfqpoint{1.286132in}{0.839159in}}{\pgfqpoint{12.053712in}{5.967710in}}%
\pgfusepath{clip}%
\pgfsetbuttcap%
\pgfsetroundjoin%
\pgfsetlinewidth{1.505625pt}%
\definecolor{currentstroke}{rgb}{0.549020,0.337255,0.294118}%
\pgfsetstrokecolor{currentstroke}%
\pgfsetdash{}{0pt}%
\pgfpathmoveto{\pgfqpoint{5.818726in}{1.184858in}}%
\pgfpathlineto{\pgfqpoint{5.818726in}{1.242744in}}%
\pgfusepath{stroke}%
\end{pgfscope}%
\begin{pgfscope}%
\pgfpathrectangle{\pgfqpoint{1.286132in}{0.839159in}}{\pgfqpoint{12.053712in}{5.967710in}}%
\pgfusepath{clip}%
\pgfsetbuttcap%
\pgfsetroundjoin%
\pgfsetlinewidth{1.505625pt}%
\definecolor{currentstroke}{rgb}{0.549020,0.337255,0.294118}%
\pgfsetstrokecolor{currentstroke}%
\pgfsetdash{}{0pt}%
\pgfpathmoveto{\pgfqpoint{5.929412in}{1.182784in}}%
\pgfpathlineto{\pgfqpoint{5.929412in}{1.298884in}}%
\pgfusepath{stroke}%
\end{pgfscope}%
\begin{pgfscope}%
\pgfpathrectangle{\pgfqpoint{1.286132in}{0.839159in}}{\pgfqpoint{12.053712in}{5.967710in}}%
\pgfusepath{clip}%
\pgfsetbuttcap%
\pgfsetroundjoin%
\pgfsetlinewidth{1.505625pt}%
\definecolor{currentstroke}{rgb}{0.549020,0.337255,0.294118}%
\pgfsetstrokecolor{currentstroke}%
\pgfsetdash{}{0pt}%
\pgfpathmoveto{\pgfqpoint{6.040098in}{1.176891in}}%
\pgfpathlineto{\pgfqpoint{6.040098in}{1.211059in}}%
\pgfusepath{stroke}%
\end{pgfscope}%
\begin{pgfscope}%
\pgfpathrectangle{\pgfqpoint{1.286132in}{0.839159in}}{\pgfqpoint{12.053712in}{5.967710in}}%
\pgfusepath{clip}%
\pgfsetbuttcap%
\pgfsetroundjoin%
\pgfsetlinewidth{1.505625pt}%
\definecolor{currentstroke}{rgb}{0.549020,0.337255,0.294118}%
\pgfsetstrokecolor{currentstroke}%
\pgfsetdash{}{0pt}%
\pgfpathmoveto{\pgfqpoint{6.150784in}{1.179407in}}%
\pgfpathlineto{\pgfqpoint{6.150784in}{1.216300in}}%
\pgfusepath{stroke}%
\end{pgfscope}%
\begin{pgfscope}%
\pgfpathrectangle{\pgfqpoint{1.286132in}{0.839159in}}{\pgfqpoint{12.053712in}{5.967710in}}%
\pgfusepath{clip}%
\pgfsetbuttcap%
\pgfsetroundjoin%
\pgfsetlinewidth{1.505625pt}%
\definecolor{currentstroke}{rgb}{0.549020,0.337255,0.294118}%
\pgfsetstrokecolor{currentstroke}%
\pgfsetdash{}{0pt}%
\pgfpathmoveto{\pgfqpoint{6.261470in}{1.176857in}}%
\pgfpathlineto{\pgfqpoint{6.261470in}{1.222353in}}%
\pgfusepath{stroke}%
\end{pgfscope}%
\begin{pgfscope}%
\pgfpathrectangle{\pgfqpoint{1.286132in}{0.839159in}}{\pgfqpoint{12.053712in}{5.967710in}}%
\pgfusepath{clip}%
\pgfsetbuttcap%
\pgfsetroundjoin%
\pgfsetlinewidth{1.505625pt}%
\definecolor{currentstroke}{rgb}{0.549020,0.337255,0.294118}%
\pgfsetstrokecolor{currentstroke}%
\pgfsetdash{}{0pt}%
\pgfpathmoveto{\pgfqpoint{6.372156in}{1.195107in}}%
\pgfpathlineto{\pgfqpoint{6.372156in}{1.238770in}}%
\pgfusepath{stroke}%
\end{pgfscope}%
\begin{pgfscope}%
\pgfpathrectangle{\pgfqpoint{1.286132in}{0.839159in}}{\pgfqpoint{12.053712in}{5.967710in}}%
\pgfusepath{clip}%
\pgfsetbuttcap%
\pgfsetroundjoin%
\pgfsetlinewidth{1.505625pt}%
\definecolor{currentstroke}{rgb}{0.549020,0.337255,0.294118}%
\pgfsetstrokecolor{currentstroke}%
\pgfsetdash{}{0pt}%
\pgfpathmoveto{\pgfqpoint{6.482842in}{1.199624in}}%
\pgfpathlineto{\pgfqpoint{6.482842in}{1.223338in}}%
\pgfusepath{stroke}%
\end{pgfscope}%
\begin{pgfscope}%
\pgfpathrectangle{\pgfqpoint{1.286132in}{0.839159in}}{\pgfqpoint{12.053712in}{5.967710in}}%
\pgfusepath{clip}%
\pgfsetbuttcap%
\pgfsetroundjoin%
\pgfsetlinewidth{1.505625pt}%
\definecolor{currentstroke}{rgb}{0.549020,0.337255,0.294118}%
\pgfsetstrokecolor{currentstroke}%
\pgfsetdash{}{0pt}%
\pgfpathmoveto{\pgfqpoint{6.593528in}{1.194960in}}%
\pgfpathlineto{\pgfqpoint{6.593528in}{1.261270in}}%
\pgfusepath{stroke}%
\end{pgfscope}%
\begin{pgfscope}%
\pgfpathrectangle{\pgfqpoint{1.286132in}{0.839159in}}{\pgfqpoint{12.053712in}{5.967710in}}%
\pgfusepath{clip}%
\pgfsetbuttcap%
\pgfsetroundjoin%
\pgfsetlinewidth{1.505625pt}%
\definecolor{currentstroke}{rgb}{0.549020,0.337255,0.294118}%
\pgfsetstrokecolor{currentstroke}%
\pgfsetdash{}{0pt}%
\pgfpathmoveto{\pgfqpoint{6.704214in}{1.207603in}}%
\pgfpathlineto{\pgfqpoint{6.704214in}{1.220929in}}%
\pgfusepath{stroke}%
\end{pgfscope}%
\begin{pgfscope}%
\pgfpathrectangle{\pgfqpoint{1.286132in}{0.839159in}}{\pgfqpoint{12.053712in}{5.967710in}}%
\pgfusepath{clip}%
\pgfsetbuttcap%
\pgfsetroundjoin%
\pgfsetlinewidth{1.505625pt}%
\definecolor{currentstroke}{rgb}{0.549020,0.337255,0.294118}%
\pgfsetstrokecolor{currentstroke}%
\pgfsetdash{}{0pt}%
\pgfpathmoveto{\pgfqpoint{6.814900in}{1.202805in}}%
\pgfpathlineto{\pgfqpoint{6.814900in}{1.248557in}}%
\pgfusepath{stroke}%
\end{pgfscope}%
\begin{pgfscope}%
\pgfpathrectangle{\pgfqpoint{1.286132in}{0.839159in}}{\pgfqpoint{12.053712in}{5.967710in}}%
\pgfusepath{clip}%
\pgfsetbuttcap%
\pgfsetroundjoin%
\pgfsetlinewidth{1.505625pt}%
\definecolor{currentstroke}{rgb}{0.549020,0.337255,0.294118}%
\pgfsetstrokecolor{currentstroke}%
\pgfsetdash{}{0pt}%
\pgfpathmoveto{\pgfqpoint{6.925586in}{1.215984in}}%
\pgfpathlineto{\pgfqpoint{6.925586in}{1.240741in}}%
\pgfusepath{stroke}%
\end{pgfscope}%
\begin{pgfscope}%
\pgfpathrectangle{\pgfqpoint{1.286132in}{0.839159in}}{\pgfqpoint{12.053712in}{5.967710in}}%
\pgfusepath{clip}%
\pgfsetbuttcap%
\pgfsetroundjoin%
\pgfsetlinewidth{1.505625pt}%
\definecolor{currentstroke}{rgb}{0.549020,0.337255,0.294118}%
\pgfsetstrokecolor{currentstroke}%
\pgfsetdash{}{0pt}%
\pgfpathmoveto{\pgfqpoint{7.036272in}{1.217550in}}%
\pgfpathlineto{\pgfqpoint{7.036272in}{1.263542in}}%
\pgfusepath{stroke}%
\end{pgfscope}%
\begin{pgfscope}%
\pgfpathrectangle{\pgfqpoint{1.286132in}{0.839159in}}{\pgfqpoint{12.053712in}{5.967710in}}%
\pgfusepath{clip}%
\pgfsetbuttcap%
\pgfsetroundjoin%
\pgfsetlinewidth{1.505625pt}%
\definecolor{currentstroke}{rgb}{0.549020,0.337255,0.294118}%
\pgfsetstrokecolor{currentstroke}%
\pgfsetdash{}{0pt}%
\pgfpathmoveto{\pgfqpoint{7.146959in}{1.226181in}}%
\pgfpathlineto{\pgfqpoint{7.146959in}{1.275129in}}%
\pgfusepath{stroke}%
\end{pgfscope}%
\begin{pgfscope}%
\pgfpathrectangle{\pgfqpoint{1.286132in}{0.839159in}}{\pgfqpoint{12.053712in}{5.967710in}}%
\pgfusepath{clip}%
\pgfsetbuttcap%
\pgfsetroundjoin%
\pgfsetlinewidth{1.505625pt}%
\definecolor{currentstroke}{rgb}{0.549020,0.337255,0.294118}%
\pgfsetstrokecolor{currentstroke}%
\pgfsetdash{}{0pt}%
\pgfpathmoveto{\pgfqpoint{7.257645in}{1.238851in}}%
\pgfpathlineto{\pgfqpoint{7.257645in}{1.270253in}}%
\pgfusepath{stroke}%
\end{pgfscope}%
\begin{pgfscope}%
\pgfpathrectangle{\pgfqpoint{1.286132in}{0.839159in}}{\pgfqpoint{12.053712in}{5.967710in}}%
\pgfusepath{clip}%
\pgfsetbuttcap%
\pgfsetroundjoin%
\pgfsetlinewidth{1.505625pt}%
\definecolor{currentstroke}{rgb}{0.549020,0.337255,0.294118}%
\pgfsetstrokecolor{currentstroke}%
\pgfsetdash{}{0pt}%
\pgfpathmoveto{\pgfqpoint{7.368331in}{1.220704in}}%
\pgfpathlineto{\pgfqpoint{7.368331in}{1.272014in}}%
\pgfusepath{stroke}%
\end{pgfscope}%
\begin{pgfscope}%
\pgfpathrectangle{\pgfqpoint{1.286132in}{0.839159in}}{\pgfqpoint{12.053712in}{5.967710in}}%
\pgfusepath{clip}%
\pgfsetbuttcap%
\pgfsetroundjoin%
\pgfsetlinewidth{1.505625pt}%
\definecolor{currentstroke}{rgb}{0.549020,0.337255,0.294118}%
\pgfsetstrokecolor{currentstroke}%
\pgfsetdash{}{0pt}%
\pgfpathmoveto{\pgfqpoint{7.479017in}{1.233668in}}%
\pgfpathlineto{\pgfqpoint{7.479017in}{1.277163in}}%
\pgfusepath{stroke}%
\end{pgfscope}%
\begin{pgfscope}%
\pgfpathrectangle{\pgfqpoint{1.286132in}{0.839159in}}{\pgfqpoint{12.053712in}{5.967710in}}%
\pgfusepath{clip}%
\pgfsetbuttcap%
\pgfsetroundjoin%
\pgfsetlinewidth{1.505625pt}%
\definecolor{currentstroke}{rgb}{0.549020,0.337255,0.294118}%
\pgfsetstrokecolor{currentstroke}%
\pgfsetdash{}{0pt}%
\pgfpathmoveto{\pgfqpoint{7.589703in}{1.261363in}}%
\pgfpathlineto{\pgfqpoint{7.589703in}{1.375158in}}%
\pgfusepath{stroke}%
\end{pgfscope}%
\begin{pgfscope}%
\pgfpathrectangle{\pgfqpoint{1.286132in}{0.839159in}}{\pgfqpoint{12.053712in}{5.967710in}}%
\pgfusepath{clip}%
\pgfsetbuttcap%
\pgfsetroundjoin%
\pgfsetlinewidth{1.505625pt}%
\definecolor{currentstroke}{rgb}{0.549020,0.337255,0.294118}%
\pgfsetstrokecolor{currentstroke}%
\pgfsetdash{}{0pt}%
\pgfpathmoveto{\pgfqpoint{7.700389in}{1.243467in}}%
\pgfpathlineto{\pgfqpoint{7.700389in}{1.281510in}}%
\pgfusepath{stroke}%
\end{pgfscope}%
\begin{pgfscope}%
\pgfpathrectangle{\pgfqpoint{1.286132in}{0.839159in}}{\pgfqpoint{12.053712in}{5.967710in}}%
\pgfusepath{clip}%
\pgfsetbuttcap%
\pgfsetroundjoin%
\pgfsetlinewidth{1.505625pt}%
\definecolor{currentstroke}{rgb}{0.549020,0.337255,0.294118}%
\pgfsetstrokecolor{currentstroke}%
\pgfsetdash{}{0pt}%
\pgfpathmoveto{\pgfqpoint{7.811075in}{1.254243in}}%
\pgfpathlineto{\pgfqpoint{7.811075in}{1.314828in}}%
\pgfusepath{stroke}%
\end{pgfscope}%
\begin{pgfscope}%
\pgfpathrectangle{\pgfqpoint{1.286132in}{0.839159in}}{\pgfqpoint{12.053712in}{5.967710in}}%
\pgfusepath{clip}%
\pgfsetbuttcap%
\pgfsetroundjoin%
\pgfsetlinewidth{1.505625pt}%
\definecolor{currentstroke}{rgb}{0.549020,0.337255,0.294118}%
\pgfsetstrokecolor{currentstroke}%
\pgfsetdash{}{0pt}%
\pgfpathmoveto{\pgfqpoint{7.921761in}{1.254489in}}%
\pgfpathlineto{\pgfqpoint{7.921761in}{1.313492in}}%
\pgfusepath{stroke}%
\end{pgfscope}%
\begin{pgfscope}%
\pgfpathrectangle{\pgfqpoint{1.286132in}{0.839159in}}{\pgfqpoint{12.053712in}{5.967710in}}%
\pgfusepath{clip}%
\pgfsetbuttcap%
\pgfsetroundjoin%
\pgfsetlinewidth{1.505625pt}%
\definecolor{currentstroke}{rgb}{0.549020,0.337255,0.294118}%
\pgfsetstrokecolor{currentstroke}%
\pgfsetdash{}{0pt}%
\pgfpathmoveto{\pgfqpoint{8.032447in}{1.265791in}}%
\pgfpathlineto{\pgfqpoint{8.032447in}{1.292220in}}%
\pgfusepath{stroke}%
\end{pgfscope}%
\begin{pgfscope}%
\pgfpathrectangle{\pgfqpoint{1.286132in}{0.839159in}}{\pgfqpoint{12.053712in}{5.967710in}}%
\pgfusepath{clip}%
\pgfsetbuttcap%
\pgfsetroundjoin%
\pgfsetlinewidth{1.505625pt}%
\definecolor{currentstroke}{rgb}{0.549020,0.337255,0.294118}%
\pgfsetstrokecolor{currentstroke}%
\pgfsetdash{}{0pt}%
\pgfpathmoveto{\pgfqpoint{8.143133in}{1.278571in}}%
\pgfpathlineto{\pgfqpoint{8.143133in}{1.315844in}}%
\pgfusepath{stroke}%
\end{pgfscope}%
\begin{pgfscope}%
\pgfpathrectangle{\pgfqpoint{1.286132in}{0.839159in}}{\pgfqpoint{12.053712in}{5.967710in}}%
\pgfusepath{clip}%
\pgfsetbuttcap%
\pgfsetroundjoin%
\pgfsetlinewidth{1.505625pt}%
\definecolor{currentstroke}{rgb}{0.549020,0.337255,0.294118}%
\pgfsetstrokecolor{currentstroke}%
\pgfsetdash{}{0pt}%
\pgfpathmoveto{\pgfqpoint{8.253819in}{1.263553in}}%
\pgfpathlineto{\pgfqpoint{8.253819in}{1.309190in}}%
\pgfusepath{stroke}%
\end{pgfscope}%
\begin{pgfscope}%
\pgfpathrectangle{\pgfqpoint{1.286132in}{0.839159in}}{\pgfqpoint{12.053712in}{5.967710in}}%
\pgfusepath{clip}%
\pgfsetbuttcap%
\pgfsetroundjoin%
\pgfsetlinewidth{1.505625pt}%
\definecolor{currentstroke}{rgb}{0.549020,0.337255,0.294118}%
\pgfsetstrokecolor{currentstroke}%
\pgfsetdash{}{0pt}%
\pgfpathmoveto{\pgfqpoint{8.364505in}{1.280392in}}%
\pgfpathlineto{\pgfqpoint{8.364505in}{1.356124in}}%
\pgfusepath{stroke}%
\end{pgfscope}%
\begin{pgfscope}%
\pgfpathrectangle{\pgfqpoint{1.286132in}{0.839159in}}{\pgfqpoint{12.053712in}{5.967710in}}%
\pgfusepath{clip}%
\pgfsetbuttcap%
\pgfsetroundjoin%
\pgfsetlinewidth{1.505625pt}%
\definecolor{currentstroke}{rgb}{0.549020,0.337255,0.294118}%
\pgfsetstrokecolor{currentstroke}%
\pgfsetdash{}{0pt}%
\pgfpathmoveto{\pgfqpoint{8.475191in}{1.291739in}}%
\pgfpathlineto{\pgfqpoint{8.475191in}{1.343694in}}%
\pgfusepath{stroke}%
\end{pgfscope}%
\begin{pgfscope}%
\pgfpathrectangle{\pgfqpoint{1.286132in}{0.839159in}}{\pgfqpoint{12.053712in}{5.967710in}}%
\pgfusepath{clip}%
\pgfsetbuttcap%
\pgfsetroundjoin%
\pgfsetlinewidth{1.505625pt}%
\definecolor{currentstroke}{rgb}{0.549020,0.337255,0.294118}%
\pgfsetstrokecolor{currentstroke}%
\pgfsetdash{}{0pt}%
\pgfpathmoveto{\pgfqpoint{8.585877in}{1.334589in}}%
\pgfpathlineto{\pgfqpoint{8.585877in}{1.373150in}}%
\pgfusepath{stroke}%
\end{pgfscope}%
\begin{pgfscope}%
\pgfpathrectangle{\pgfqpoint{1.286132in}{0.839159in}}{\pgfqpoint{12.053712in}{5.967710in}}%
\pgfusepath{clip}%
\pgfsetbuttcap%
\pgfsetroundjoin%
\pgfsetlinewidth{1.505625pt}%
\definecolor{currentstroke}{rgb}{0.549020,0.337255,0.294118}%
\pgfsetstrokecolor{currentstroke}%
\pgfsetdash{}{0pt}%
\pgfpathmoveto{\pgfqpoint{8.696563in}{1.369698in}}%
\pgfpathlineto{\pgfqpoint{8.696563in}{1.403681in}}%
\pgfusepath{stroke}%
\end{pgfscope}%
\begin{pgfscope}%
\pgfpathrectangle{\pgfqpoint{1.286132in}{0.839159in}}{\pgfqpoint{12.053712in}{5.967710in}}%
\pgfusepath{clip}%
\pgfsetbuttcap%
\pgfsetroundjoin%
\pgfsetlinewidth{1.505625pt}%
\definecolor{currentstroke}{rgb}{0.549020,0.337255,0.294118}%
\pgfsetstrokecolor{currentstroke}%
\pgfsetdash{}{0pt}%
\pgfpathmoveto{\pgfqpoint{8.807249in}{1.340643in}}%
\pgfpathlineto{\pgfqpoint{8.807249in}{1.436961in}}%
\pgfusepath{stroke}%
\end{pgfscope}%
\begin{pgfscope}%
\pgfpathrectangle{\pgfqpoint{1.286132in}{0.839159in}}{\pgfqpoint{12.053712in}{5.967710in}}%
\pgfusepath{clip}%
\pgfsetbuttcap%
\pgfsetroundjoin%
\pgfsetlinewidth{1.505625pt}%
\definecolor{currentstroke}{rgb}{0.549020,0.337255,0.294118}%
\pgfsetstrokecolor{currentstroke}%
\pgfsetdash{}{0pt}%
\pgfpathmoveto{\pgfqpoint{8.917936in}{1.290549in}}%
\pgfpathlineto{\pgfqpoint{8.917936in}{1.402382in}}%
\pgfusepath{stroke}%
\end{pgfscope}%
\begin{pgfscope}%
\pgfpathrectangle{\pgfqpoint{1.286132in}{0.839159in}}{\pgfqpoint{12.053712in}{5.967710in}}%
\pgfusepath{clip}%
\pgfsetbuttcap%
\pgfsetroundjoin%
\pgfsetlinewidth{1.505625pt}%
\definecolor{currentstroke}{rgb}{0.549020,0.337255,0.294118}%
\pgfsetstrokecolor{currentstroke}%
\pgfsetdash{}{0pt}%
\pgfpathmoveto{\pgfqpoint{9.028622in}{1.310458in}}%
\pgfpathlineto{\pgfqpoint{9.028622in}{1.472091in}}%
\pgfusepath{stroke}%
\end{pgfscope}%
\begin{pgfscope}%
\pgfpathrectangle{\pgfqpoint{1.286132in}{0.839159in}}{\pgfqpoint{12.053712in}{5.967710in}}%
\pgfusepath{clip}%
\pgfsetbuttcap%
\pgfsetroundjoin%
\pgfsetlinewidth{1.505625pt}%
\definecolor{currentstroke}{rgb}{0.549020,0.337255,0.294118}%
\pgfsetstrokecolor{currentstroke}%
\pgfsetdash{}{0pt}%
\pgfpathmoveto{\pgfqpoint{9.139308in}{1.344541in}}%
\pgfpathlineto{\pgfqpoint{9.139308in}{1.413140in}}%
\pgfusepath{stroke}%
\end{pgfscope}%
\begin{pgfscope}%
\pgfpathrectangle{\pgfqpoint{1.286132in}{0.839159in}}{\pgfqpoint{12.053712in}{5.967710in}}%
\pgfusepath{clip}%
\pgfsetbuttcap%
\pgfsetroundjoin%
\pgfsetlinewidth{1.505625pt}%
\definecolor{currentstroke}{rgb}{0.549020,0.337255,0.294118}%
\pgfsetstrokecolor{currentstroke}%
\pgfsetdash{}{0pt}%
\pgfpathmoveto{\pgfqpoint{9.249994in}{1.345257in}}%
\pgfpathlineto{\pgfqpoint{9.249994in}{1.407375in}}%
\pgfusepath{stroke}%
\end{pgfscope}%
\begin{pgfscope}%
\pgfpathrectangle{\pgfqpoint{1.286132in}{0.839159in}}{\pgfqpoint{12.053712in}{5.967710in}}%
\pgfusepath{clip}%
\pgfsetbuttcap%
\pgfsetroundjoin%
\pgfsetlinewidth{1.505625pt}%
\definecolor{currentstroke}{rgb}{0.549020,0.337255,0.294118}%
\pgfsetstrokecolor{currentstroke}%
\pgfsetdash{}{0pt}%
\pgfpathmoveto{\pgfqpoint{9.360680in}{1.371709in}}%
\pgfpathlineto{\pgfqpoint{9.360680in}{1.474824in}}%
\pgfusepath{stroke}%
\end{pgfscope}%
\begin{pgfscope}%
\pgfpathrectangle{\pgfqpoint{1.286132in}{0.839159in}}{\pgfqpoint{12.053712in}{5.967710in}}%
\pgfusepath{clip}%
\pgfsetbuttcap%
\pgfsetroundjoin%
\pgfsetlinewidth{1.505625pt}%
\definecolor{currentstroke}{rgb}{0.549020,0.337255,0.294118}%
\pgfsetstrokecolor{currentstroke}%
\pgfsetdash{}{0pt}%
\pgfpathmoveto{\pgfqpoint{9.471366in}{1.403023in}}%
\pgfpathlineto{\pgfqpoint{9.471366in}{1.464869in}}%
\pgfusepath{stroke}%
\end{pgfscope}%
\begin{pgfscope}%
\pgfpathrectangle{\pgfqpoint{1.286132in}{0.839159in}}{\pgfqpoint{12.053712in}{5.967710in}}%
\pgfusepath{clip}%
\pgfsetbuttcap%
\pgfsetroundjoin%
\pgfsetlinewidth{1.505625pt}%
\definecolor{currentstroke}{rgb}{0.549020,0.337255,0.294118}%
\pgfsetstrokecolor{currentstroke}%
\pgfsetdash{}{0pt}%
\pgfpathmoveto{\pgfqpoint{9.582052in}{1.322700in}}%
\pgfpathlineto{\pgfqpoint{9.582052in}{1.491529in}}%
\pgfusepath{stroke}%
\end{pgfscope}%
\begin{pgfscope}%
\pgfpathrectangle{\pgfqpoint{1.286132in}{0.839159in}}{\pgfqpoint{12.053712in}{5.967710in}}%
\pgfusepath{clip}%
\pgfsetbuttcap%
\pgfsetroundjoin%
\pgfsetlinewidth{1.505625pt}%
\definecolor{currentstroke}{rgb}{0.549020,0.337255,0.294118}%
\pgfsetstrokecolor{currentstroke}%
\pgfsetdash{}{0pt}%
\pgfpathmoveto{\pgfqpoint{9.692738in}{1.340529in}}%
\pgfpathlineto{\pgfqpoint{9.692738in}{1.441029in}}%
\pgfusepath{stroke}%
\end{pgfscope}%
\begin{pgfscope}%
\pgfpathrectangle{\pgfqpoint{1.286132in}{0.839159in}}{\pgfqpoint{12.053712in}{5.967710in}}%
\pgfusepath{clip}%
\pgfsetbuttcap%
\pgfsetroundjoin%
\pgfsetlinewidth{1.505625pt}%
\definecolor{currentstroke}{rgb}{0.549020,0.337255,0.294118}%
\pgfsetstrokecolor{currentstroke}%
\pgfsetdash{}{0pt}%
\pgfpathmoveto{\pgfqpoint{9.803424in}{1.376645in}}%
\pgfpathlineto{\pgfqpoint{9.803424in}{1.534463in}}%
\pgfusepath{stroke}%
\end{pgfscope}%
\begin{pgfscope}%
\pgfpathrectangle{\pgfqpoint{1.286132in}{0.839159in}}{\pgfqpoint{12.053712in}{5.967710in}}%
\pgfusepath{clip}%
\pgfsetbuttcap%
\pgfsetroundjoin%
\pgfsetlinewidth{1.505625pt}%
\definecolor{currentstroke}{rgb}{0.549020,0.337255,0.294118}%
\pgfsetstrokecolor{currentstroke}%
\pgfsetdash{}{0pt}%
\pgfpathmoveto{\pgfqpoint{9.914110in}{1.451846in}}%
\pgfpathlineto{\pgfqpoint{9.914110in}{1.603849in}}%
\pgfusepath{stroke}%
\end{pgfscope}%
\begin{pgfscope}%
\pgfpathrectangle{\pgfqpoint{1.286132in}{0.839159in}}{\pgfqpoint{12.053712in}{5.967710in}}%
\pgfusepath{clip}%
\pgfsetbuttcap%
\pgfsetroundjoin%
\pgfsetlinewidth{1.505625pt}%
\definecolor{currentstroke}{rgb}{0.549020,0.337255,0.294118}%
\pgfsetstrokecolor{currentstroke}%
\pgfsetdash{}{0pt}%
\pgfpathmoveto{\pgfqpoint{10.024796in}{1.408445in}}%
\pgfpathlineto{\pgfqpoint{10.024796in}{1.561440in}}%
\pgfusepath{stroke}%
\end{pgfscope}%
\begin{pgfscope}%
\pgfpathrectangle{\pgfqpoint{1.286132in}{0.839159in}}{\pgfqpoint{12.053712in}{5.967710in}}%
\pgfusepath{clip}%
\pgfsetbuttcap%
\pgfsetroundjoin%
\pgfsetlinewidth{1.505625pt}%
\definecolor{currentstroke}{rgb}{0.549020,0.337255,0.294118}%
\pgfsetstrokecolor{currentstroke}%
\pgfsetdash{}{0pt}%
\pgfpathmoveto{\pgfqpoint{10.135482in}{1.398249in}}%
\pgfpathlineto{\pgfqpoint{10.135482in}{1.552886in}}%
\pgfusepath{stroke}%
\end{pgfscope}%
\begin{pgfscope}%
\pgfpathrectangle{\pgfqpoint{1.286132in}{0.839159in}}{\pgfqpoint{12.053712in}{5.967710in}}%
\pgfusepath{clip}%
\pgfsetbuttcap%
\pgfsetroundjoin%
\pgfsetlinewidth{1.505625pt}%
\definecolor{currentstroke}{rgb}{0.549020,0.337255,0.294118}%
\pgfsetstrokecolor{currentstroke}%
\pgfsetdash{}{0pt}%
\pgfpathmoveto{\pgfqpoint{10.246168in}{1.459567in}}%
\pgfpathlineto{\pgfqpoint{10.246168in}{1.576792in}}%
\pgfusepath{stroke}%
\end{pgfscope}%
\begin{pgfscope}%
\pgfpathrectangle{\pgfqpoint{1.286132in}{0.839159in}}{\pgfqpoint{12.053712in}{5.967710in}}%
\pgfusepath{clip}%
\pgfsetbuttcap%
\pgfsetroundjoin%
\pgfsetlinewidth{1.505625pt}%
\definecolor{currentstroke}{rgb}{0.549020,0.337255,0.294118}%
\pgfsetstrokecolor{currentstroke}%
\pgfsetdash{}{0pt}%
\pgfpathmoveto{\pgfqpoint{10.356854in}{1.396403in}}%
\pgfpathlineto{\pgfqpoint{10.356854in}{1.571587in}}%
\pgfusepath{stroke}%
\end{pgfscope}%
\begin{pgfscope}%
\pgfpathrectangle{\pgfqpoint{1.286132in}{0.839159in}}{\pgfqpoint{12.053712in}{5.967710in}}%
\pgfusepath{clip}%
\pgfsetbuttcap%
\pgfsetroundjoin%
\pgfsetlinewidth{1.505625pt}%
\definecolor{currentstroke}{rgb}{0.549020,0.337255,0.294118}%
\pgfsetstrokecolor{currentstroke}%
\pgfsetdash{}{0pt}%
\pgfpathmoveto{\pgfqpoint{10.467540in}{1.407638in}}%
\pgfpathlineto{\pgfqpoint{10.467540in}{1.619367in}}%
\pgfusepath{stroke}%
\end{pgfscope}%
\begin{pgfscope}%
\pgfpathrectangle{\pgfqpoint{1.286132in}{0.839159in}}{\pgfqpoint{12.053712in}{5.967710in}}%
\pgfusepath{clip}%
\pgfsetbuttcap%
\pgfsetroundjoin%
\pgfsetlinewidth{1.505625pt}%
\definecolor{currentstroke}{rgb}{0.549020,0.337255,0.294118}%
\pgfsetstrokecolor{currentstroke}%
\pgfsetdash{}{0pt}%
\pgfpathmoveto{\pgfqpoint{10.578226in}{1.467394in}}%
\pgfpathlineto{\pgfqpoint{10.578226in}{1.600747in}}%
\pgfusepath{stroke}%
\end{pgfscope}%
\begin{pgfscope}%
\pgfpathrectangle{\pgfqpoint{1.286132in}{0.839159in}}{\pgfqpoint{12.053712in}{5.967710in}}%
\pgfusepath{clip}%
\pgfsetbuttcap%
\pgfsetroundjoin%
\pgfsetlinewidth{1.505625pt}%
\definecolor{currentstroke}{rgb}{0.549020,0.337255,0.294118}%
\pgfsetstrokecolor{currentstroke}%
\pgfsetdash{}{0pt}%
\pgfpathmoveto{\pgfqpoint{10.688913in}{1.428601in}}%
\pgfpathlineto{\pgfqpoint{10.688913in}{1.535549in}}%
\pgfusepath{stroke}%
\end{pgfscope}%
\begin{pgfscope}%
\pgfpathrectangle{\pgfqpoint{1.286132in}{0.839159in}}{\pgfqpoint{12.053712in}{5.967710in}}%
\pgfusepath{clip}%
\pgfsetbuttcap%
\pgfsetroundjoin%
\pgfsetlinewidth{1.505625pt}%
\definecolor{currentstroke}{rgb}{0.549020,0.337255,0.294118}%
\pgfsetstrokecolor{currentstroke}%
\pgfsetdash{}{0pt}%
\pgfpathmoveto{\pgfqpoint{10.799599in}{1.424695in}}%
\pgfpathlineto{\pgfqpoint{10.799599in}{1.609691in}}%
\pgfusepath{stroke}%
\end{pgfscope}%
\begin{pgfscope}%
\pgfpathrectangle{\pgfqpoint{1.286132in}{0.839159in}}{\pgfqpoint{12.053712in}{5.967710in}}%
\pgfusepath{clip}%
\pgfsetbuttcap%
\pgfsetroundjoin%
\pgfsetlinewidth{1.505625pt}%
\definecolor{currentstroke}{rgb}{0.549020,0.337255,0.294118}%
\pgfsetstrokecolor{currentstroke}%
\pgfsetdash{}{0pt}%
\pgfpathmoveto{\pgfqpoint{10.910285in}{1.553249in}}%
\pgfpathlineto{\pgfqpoint{10.910285in}{1.700611in}}%
\pgfusepath{stroke}%
\end{pgfscope}%
\begin{pgfscope}%
\pgfpathrectangle{\pgfqpoint{1.286132in}{0.839159in}}{\pgfqpoint{12.053712in}{5.967710in}}%
\pgfusepath{clip}%
\pgfsetbuttcap%
\pgfsetroundjoin%
\pgfsetlinewidth{1.505625pt}%
\definecolor{currentstroke}{rgb}{0.549020,0.337255,0.294118}%
\pgfsetstrokecolor{currentstroke}%
\pgfsetdash{}{0pt}%
\pgfpathmoveto{\pgfqpoint{11.020971in}{1.473400in}}%
\pgfpathlineto{\pgfqpoint{11.020971in}{1.559582in}}%
\pgfusepath{stroke}%
\end{pgfscope}%
\begin{pgfscope}%
\pgfpathrectangle{\pgfqpoint{1.286132in}{0.839159in}}{\pgfqpoint{12.053712in}{5.967710in}}%
\pgfusepath{clip}%
\pgfsetbuttcap%
\pgfsetroundjoin%
\pgfsetlinewidth{1.505625pt}%
\definecolor{currentstroke}{rgb}{0.549020,0.337255,0.294118}%
\pgfsetstrokecolor{currentstroke}%
\pgfsetdash{}{0pt}%
\pgfpathmoveto{\pgfqpoint{11.131657in}{1.488131in}}%
\pgfpathlineto{\pgfqpoint{11.131657in}{1.678711in}}%
\pgfusepath{stroke}%
\end{pgfscope}%
\begin{pgfscope}%
\pgfpathrectangle{\pgfqpoint{1.286132in}{0.839159in}}{\pgfqpoint{12.053712in}{5.967710in}}%
\pgfusepath{clip}%
\pgfsetbuttcap%
\pgfsetroundjoin%
\pgfsetlinewidth{1.505625pt}%
\definecolor{currentstroke}{rgb}{0.549020,0.337255,0.294118}%
\pgfsetstrokecolor{currentstroke}%
\pgfsetdash{}{0pt}%
\pgfpathmoveto{\pgfqpoint{11.242343in}{1.540650in}}%
\pgfpathlineto{\pgfqpoint{11.242343in}{1.612358in}}%
\pgfusepath{stroke}%
\end{pgfscope}%
\begin{pgfscope}%
\pgfpathrectangle{\pgfqpoint{1.286132in}{0.839159in}}{\pgfqpoint{12.053712in}{5.967710in}}%
\pgfusepath{clip}%
\pgfsetbuttcap%
\pgfsetroundjoin%
\pgfsetlinewidth{1.505625pt}%
\definecolor{currentstroke}{rgb}{0.549020,0.337255,0.294118}%
\pgfsetstrokecolor{currentstroke}%
\pgfsetdash{}{0pt}%
\pgfpathmoveto{\pgfqpoint{11.353029in}{1.543553in}}%
\pgfpathlineto{\pgfqpoint{11.353029in}{1.796808in}}%
\pgfusepath{stroke}%
\end{pgfscope}%
\begin{pgfscope}%
\pgfpathrectangle{\pgfqpoint{1.286132in}{0.839159in}}{\pgfqpoint{12.053712in}{5.967710in}}%
\pgfusepath{clip}%
\pgfsetbuttcap%
\pgfsetroundjoin%
\pgfsetlinewidth{1.505625pt}%
\definecolor{currentstroke}{rgb}{0.549020,0.337255,0.294118}%
\pgfsetstrokecolor{currentstroke}%
\pgfsetdash{}{0pt}%
\pgfpathmoveto{\pgfqpoint{11.463715in}{1.515663in}}%
\pgfpathlineto{\pgfqpoint{11.463715in}{1.664469in}}%
\pgfusepath{stroke}%
\end{pgfscope}%
\begin{pgfscope}%
\pgfpathrectangle{\pgfqpoint{1.286132in}{0.839159in}}{\pgfqpoint{12.053712in}{5.967710in}}%
\pgfusepath{clip}%
\pgfsetbuttcap%
\pgfsetroundjoin%
\pgfsetlinewidth{1.505625pt}%
\definecolor{currentstroke}{rgb}{0.549020,0.337255,0.294118}%
\pgfsetstrokecolor{currentstroke}%
\pgfsetdash{}{0pt}%
\pgfpathmoveto{\pgfqpoint{11.574401in}{1.425748in}}%
\pgfpathlineto{\pgfqpoint{11.574401in}{1.874818in}}%
\pgfusepath{stroke}%
\end{pgfscope}%
\begin{pgfscope}%
\pgfpathrectangle{\pgfqpoint{1.286132in}{0.839159in}}{\pgfqpoint{12.053712in}{5.967710in}}%
\pgfusepath{clip}%
\pgfsetbuttcap%
\pgfsetroundjoin%
\pgfsetlinewidth{1.505625pt}%
\definecolor{currentstroke}{rgb}{0.549020,0.337255,0.294118}%
\pgfsetstrokecolor{currentstroke}%
\pgfsetdash{}{0pt}%
\pgfpathmoveto{\pgfqpoint{11.685087in}{1.558576in}}%
\pgfpathlineto{\pgfqpoint{11.685087in}{1.734864in}}%
\pgfusepath{stroke}%
\end{pgfscope}%
\begin{pgfscope}%
\pgfpathrectangle{\pgfqpoint{1.286132in}{0.839159in}}{\pgfqpoint{12.053712in}{5.967710in}}%
\pgfusepath{clip}%
\pgfsetbuttcap%
\pgfsetroundjoin%
\pgfsetlinewidth{1.505625pt}%
\definecolor{currentstroke}{rgb}{0.549020,0.337255,0.294118}%
\pgfsetstrokecolor{currentstroke}%
\pgfsetdash{}{0pt}%
\pgfpathmoveto{\pgfqpoint{11.795773in}{1.545086in}}%
\pgfpathlineto{\pgfqpoint{11.795773in}{1.789938in}}%
\pgfusepath{stroke}%
\end{pgfscope}%
\begin{pgfscope}%
\pgfpathrectangle{\pgfqpoint{1.286132in}{0.839159in}}{\pgfqpoint{12.053712in}{5.967710in}}%
\pgfusepath{clip}%
\pgfsetbuttcap%
\pgfsetroundjoin%
\pgfsetlinewidth{1.505625pt}%
\definecolor{currentstroke}{rgb}{0.549020,0.337255,0.294118}%
\pgfsetstrokecolor{currentstroke}%
\pgfsetdash{}{0pt}%
\pgfpathmoveto{\pgfqpoint{11.906459in}{1.577630in}}%
\pgfpathlineto{\pgfqpoint{11.906459in}{1.820321in}}%
\pgfusepath{stroke}%
\end{pgfscope}%
\begin{pgfscope}%
\pgfpathrectangle{\pgfqpoint{1.286132in}{0.839159in}}{\pgfqpoint{12.053712in}{5.967710in}}%
\pgfusepath{clip}%
\pgfsetbuttcap%
\pgfsetroundjoin%
\pgfsetlinewidth{1.505625pt}%
\definecolor{currentstroke}{rgb}{0.549020,0.337255,0.294118}%
\pgfsetstrokecolor{currentstroke}%
\pgfsetdash{}{0pt}%
\pgfpathmoveto{\pgfqpoint{12.017145in}{1.582225in}}%
\pgfpathlineto{\pgfqpoint{12.017145in}{1.767947in}}%
\pgfusepath{stroke}%
\end{pgfscope}%
\begin{pgfscope}%
\pgfpathrectangle{\pgfqpoint{1.286132in}{0.839159in}}{\pgfqpoint{12.053712in}{5.967710in}}%
\pgfusepath{clip}%
\pgfsetbuttcap%
\pgfsetroundjoin%
\pgfsetlinewidth{1.505625pt}%
\definecolor{currentstroke}{rgb}{0.549020,0.337255,0.294118}%
\pgfsetstrokecolor{currentstroke}%
\pgfsetdash{}{0pt}%
\pgfpathmoveto{\pgfqpoint{12.127831in}{1.592643in}}%
\pgfpathlineto{\pgfqpoint{12.127831in}{2.010473in}}%
\pgfusepath{stroke}%
\end{pgfscope}%
\begin{pgfscope}%
\pgfpathrectangle{\pgfqpoint{1.286132in}{0.839159in}}{\pgfqpoint{12.053712in}{5.967710in}}%
\pgfusepath{clip}%
\pgfsetbuttcap%
\pgfsetroundjoin%
\pgfsetlinewidth{1.505625pt}%
\definecolor{currentstroke}{rgb}{0.549020,0.337255,0.294118}%
\pgfsetstrokecolor{currentstroke}%
\pgfsetdash{}{0pt}%
\pgfpathmoveto{\pgfqpoint{12.238517in}{1.673178in}}%
\pgfpathlineto{\pgfqpoint{12.238517in}{1.851916in}}%
\pgfusepath{stroke}%
\end{pgfscope}%
\begin{pgfscope}%
\pgfpathrectangle{\pgfqpoint{1.286132in}{0.839159in}}{\pgfqpoint{12.053712in}{5.967710in}}%
\pgfusepath{clip}%
\pgfsetbuttcap%
\pgfsetroundjoin%
\pgfsetlinewidth{1.505625pt}%
\definecolor{currentstroke}{rgb}{0.549020,0.337255,0.294118}%
\pgfsetstrokecolor{currentstroke}%
\pgfsetdash{}{0pt}%
\pgfpathmoveto{\pgfqpoint{12.349203in}{1.753743in}}%
\pgfpathlineto{\pgfqpoint{12.349203in}{1.958214in}}%
\pgfusepath{stroke}%
\end{pgfscope}%
\begin{pgfscope}%
\pgfpathrectangle{\pgfqpoint{1.286132in}{0.839159in}}{\pgfqpoint{12.053712in}{5.967710in}}%
\pgfusepath{clip}%
\pgfsetbuttcap%
\pgfsetroundjoin%
\pgfsetlinewidth{1.505625pt}%
\definecolor{currentstroke}{rgb}{0.549020,0.337255,0.294118}%
\pgfsetstrokecolor{currentstroke}%
\pgfsetdash{}{0pt}%
\pgfpathmoveto{\pgfqpoint{12.459890in}{1.675792in}}%
\pgfpathlineto{\pgfqpoint{12.459890in}{2.043943in}}%
\pgfusepath{stroke}%
\end{pgfscope}%
\begin{pgfscope}%
\pgfpathrectangle{\pgfqpoint{1.286132in}{0.839159in}}{\pgfqpoint{12.053712in}{5.967710in}}%
\pgfusepath{clip}%
\pgfsetbuttcap%
\pgfsetroundjoin%
\pgfsetlinewidth{1.505625pt}%
\definecolor{currentstroke}{rgb}{0.549020,0.337255,0.294118}%
\pgfsetstrokecolor{currentstroke}%
\pgfsetdash{}{0pt}%
\pgfpathmoveto{\pgfqpoint{12.570576in}{1.660390in}}%
\pgfpathlineto{\pgfqpoint{12.570576in}{2.339741in}}%
\pgfusepath{stroke}%
\end{pgfscope}%
\begin{pgfscope}%
\pgfpathrectangle{\pgfqpoint{1.286132in}{0.839159in}}{\pgfqpoint{12.053712in}{5.967710in}}%
\pgfusepath{clip}%
\pgfsetbuttcap%
\pgfsetroundjoin%
\pgfsetlinewidth{1.505625pt}%
\definecolor{currentstroke}{rgb}{0.549020,0.337255,0.294118}%
\pgfsetstrokecolor{currentstroke}%
\pgfsetdash{}{0pt}%
\pgfpathmoveto{\pgfqpoint{12.681262in}{1.763296in}}%
\pgfpathlineto{\pgfqpoint{12.681262in}{2.036876in}}%
\pgfusepath{stroke}%
\end{pgfscope}%
\begin{pgfscope}%
\pgfpathrectangle{\pgfqpoint{1.286132in}{0.839159in}}{\pgfqpoint{12.053712in}{5.967710in}}%
\pgfusepath{clip}%
\pgfsetbuttcap%
\pgfsetroundjoin%
\pgfsetlinewidth{1.505625pt}%
\definecolor{currentstroke}{rgb}{0.549020,0.337255,0.294118}%
\pgfsetstrokecolor{currentstroke}%
\pgfsetdash{}{0pt}%
\pgfpathmoveto{\pgfqpoint{12.791948in}{1.701511in}}%
\pgfpathlineto{\pgfqpoint{12.791948in}{1.951964in}}%
\pgfusepath{stroke}%
\end{pgfscope}%
\begin{pgfscope}%
\pgfpathrectangle{\pgfqpoint{1.286132in}{0.839159in}}{\pgfqpoint{12.053712in}{5.967710in}}%
\pgfusepath{clip}%
\pgfsetbuttcap%
\pgfsetroundjoin%
\pgfsetlinewidth{1.505625pt}%
\definecolor{currentstroke}{rgb}{0.890196,0.466667,0.760784}%
\pgfsetstrokecolor{currentstroke}%
\pgfsetdash{}{0pt}%
\pgfpathmoveto{\pgfqpoint{1.834028in}{1.119238in}}%
\pgfpathlineto{\pgfqpoint{1.834028in}{1.119252in}}%
\pgfusepath{stroke}%
\end{pgfscope}%
\begin{pgfscope}%
\pgfpathrectangle{\pgfqpoint{1.286132in}{0.839159in}}{\pgfqpoint{12.053712in}{5.967710in}}%
\pgfusepath{clip}%
\pgfsetbuttcap%
\pgfsetroundjoin%
\pgfsetlinewidth{1.505625pt}%
\definecolor{currentstroke}{rgb}{0.890196,0.466667,0.760784}%
\pgfsetstrokecolor{currentstroke}%
\pgfsetdash{}{0pt}%
\pgfpathmoveto{\pgfqpoint{1.944714in}{1.119346in}}%
\pgfpathlineto{\pgfqpoint{1.944714in}{1.119351in}}%
\pgfusepath{stroke}%
\end{pgfscope}%
\begin{pgfscope}%
\pgfpathrectangle{\pgfqpoint{1.286132in}{0.839159in}}{\pgfqpoint{12.053712in}{5.967710in}}%
\pgfusepath{clip}%
\pgfsetbuttcap%
\pgfsetroundjoin%
\pgfsetlinewidth{1.505625pt}%
\definecolor{currentstroke}{rgb}{0.890196,0.466667,0.760784}%
\pgfsetstrokecolor{currentstroke}%
\pgfsetdash{}{0pt}%
\pgfpathmoveto{\pgfqpoint{2.055400in}{1.119399in}}%
\pgfpathlineto{\pgfqpoint{2.055400in}{1.119409in}}%
\pgfusepath{stroke}%
\end{pgfscope}%
\begin{pgfscope}%
\pgfpathrectangle{\pgfqpoint{1.286132in}{0.839159in}}{\pgfqpoint{12.053712in}{5.967710in}}%
\pgfusepath{clip}%
\pgfsetbuttcap%
\pgfsetroundjoin%
\pgfsetlinewidth{1.505625pt}%
\definecolor{currentstroke}{rgb}{0.890196,0.466667,0.760784}%
\pgfsetstrokecolor{currentstroke}%
\pgfsetdash{}{0pt}%
\pgfpathmoveto{\pgfqpoint{2.166086in}{1.119446in}}%
\pgfpathlineto{\pgfqpoint{2.166086in}{1.119456in}}%
\pgfusepath{stroke}%
\end{pgfscope}%
\begin{pgfscope}%
\pgfpathrectangle{\pgfqpoint{1.286132in}{0.839159in}}{\pgfqpoint{12.053712in}{5.967710in}}%
\pgfusepath{clip}%
\pgfsetbuttcap%
\pgfsetroundjoin%
\pgfsetlinewidth{1.505625pt}%
\definecolor{currentstroke}{rgb}{0.890196,0.466667,0.760784}%
\pgfsetstrokecolor{currentstroke}%
\pgfsetdash{}{0pt}%
\pgfpathmoveto{\pgfqpoint{2.276772in}{1.119522in}}%
\pgfpathlineto{\pgfqpoint{2.276772in}{1.119532in}}%
\pgfusepath{stroke}%
\end{pgfscope}%
\begin{pgfscope}%
\pgfpathrectangle{\pgfqpoint{1.286132in}{0.839159in}}{\pgfqpoint{12.053712in}{5.967710in}}%
\pgfusepath{clip}%
\pgfsetbuttcap%
\pgfsetroundjoin%
\pgfsetlinewidth{1.505625pt}%
\definecolor{currentstroke}{rgb}{0.890196,0.466667,0.760784}%
\pgfsetstrokecolor{currentstroke}%
\pgfsetdash{}{0pt}%
\pgfpathmoveto{\pgfqpoint{2.387458in}{1.119583in}}%
\pgfpathlineto{\pgfqpoint{2.387458in}{1.119593in}}%
\pgfusepath{stroke}%
\end{pgfscope}%
\begin{pgfscope}%
\pgfpathrectangle{\pgfqpoint{1.286132in}{0.839159in}}{\pgfqpoint{12.053712in}{5.967710in}}%
\pgfusepath{clip}%
\pgfsetbuttcap%
\pgfsetroundjoin%
\pgfsetlinewidth{1.505625pt}%
\definecolor{currentstroke}{rgb}{0.890196,0.466667,0.760784}%
\pgfsetstrokecolor{currentstroke}%
\pgfsetdash{}{0pt}%
\pgfpathmoveto{\pgfqpoint{2.498144in}{1.119616in}}%
\pgfpathlineto{\pgfqpoint{2.498144in}{1.119857in}}%
\pgfusepath{stroke}%
\end{pgfscope}%
\begin{pgfscope}%
\pgfpathrectangle{\pgfqpoint{1.286132in}{0.839159in}}{\pgfqpoint{12.053712in}{5.967710in}}%
\pgfusepath{clip}%
\pgfsetbuttcap%
\pgfsetroundjoin%
\pgfsetlinewidth{1.505625pt}%
\definecolor{currentstroke}{rgb}{0.890196,0.466667,0.760784}%
\pgfsetstrokecolor{currentstroke}%
\pgfsetdash{}{0pt}%
\pgfpathmoveto{\pgfqpoint{2.608830in}{1.119729in}}%
\pgfpathlineto{\pgfqpoint{2.608830in}{1.119757in}}%
\pgfusepath{stroke}%
\end{pgfscope}%
\begin{pgfscope}%
\pgfpathrectangle{\pgfqpoint{1.286132in}{0.839159in}}{\pgfqpoint{12.053712in}{5.967710in}}%
\pgfusepath{clip}%
\pgfsetbuttcap%
\pgfsetroundjoin%
\pgfsetlinewidth{1.505625pt}%
\definecolor{currentstroke}{rgb}{0.890196,0.466667,0.760784}%
\pgfsetstrokecolor{currentstroke}%
\pgfsetdash{}{0pt}%
\pgfpathmoveto{\pgfqpoint{2.719516in}{1.119813in}}%
\pgfpathlineto{\pgfqpoint{2.719516in}{1.119879in}}%
\pgfusepath{stroke}%
\end{pgfscope}%
\begin{pgfscope}%
\pgfpathrectangle{\pgfqpoint{1.286132in}{0.839159in}}{\pgfqpoint{12.053712in}{5.967710in}}%
\pgfusepath{clip}%
\pgfsetbuttcap%
\pgfsetroundjoin%
\pgfsetlinewidth{1.505625pt}%
\definecolor{currentstroke}{rgb}{0.890196,0.466667,0.760784}%
\pgfsetstrokecolor{currentstroke}%
\pgfsetdash{}{0pt}%
\pgfpathmoveto{\pgfqpoint{2.830202in}{1.119812in}}%
\pgfpathlineto{\pgfqpoint{2.830202in}{1.120189in}}%
\pgfusepath{stroke}%
\end{pgfscope}%
\begin{pgfscope}%
\pgfpathrectangle{\pgfqpoint{1.286132in}{0.839159in}}{\pgfqpoint{12.053712in}{5.967710in}}%
\pgfusepath{clip}%
\pgfsetbuttcap%
\pgfsetroundjoin%
\pgfsetlinewidth{1.505625pt}%
\definecolor{currentstroke}{rgb}{0.890196,0.466667,0.760784}%
\pgfsetstrokecolor{currentstroke}%
\pgfsetdash{}{0pt}%
\pgfpathmoveto{\pgfqpoint{2.940888in}{1.119966in}}%
\pgfpathlineto{\pgfqpoint{2.940888in}{1.119995in}}%
\pgfusepath{stroke}%
\end{pgfscope}%
\begin{pgfscope}%
\pgfpathrectangle{\pgfqpoint{1.286132in}{0.839159in}}{\pgfqpoint{12.053712in}{5.967710in}}%
\pgfusepath{clip}%
\pgfsetbuttcap%
\pgfsetroundjoin%
\pgfsetlinewidth{1.505625pt}%
\definecolor{currentstroke}{rgb}{0.890196,0.466667,0.760784}%
\pgfsetstrokecolor{currentstroke}%
\pgfsetdash{}{0pt}%
\pgfpathmoveto{\pgfqpoint{3.051574in}{1.120094in}}%
\pgfpathlineto{\pgfqpoint{3.051574in}{1.120166in}}%
\pgfusepath{stroke}%
\end{pgfscope}%
\begin{pgfscope}%
\pgfpathrectangle{\pgfqpoint{1.286132in}{0.839159in}}{\pgfqpoint{12.053712in}{5.967710in}}%
\pgfusepath{clip}%
\pgfsetbuttcap%
\pgfsetroundjoin%
\pgfsetlinewidth{1.505625pt}%
\definecolor{currentstroke}{rgb}{0.890196,0.466667,0.760784}%
\pgfsetstrokecolor{currentstroke}%
\pgfsetdash{}{0pt}%
\pgfpathmoveto{\pgfqpoint{3.162260in}{1.120118in}}%
\pgfpathlineto{\pgfqpoint{3.162260in}{1.120549in}}%
\pgfusepath{stroke}%
\end{pgfscope}%
\begin{pgfscope}%
\pgfpathrectangle{\pgfqpoint{1.286132in}{0.839159in}}{\pgfqpoint{12.053712in}{5.967710in}}%
\pgfusepath{clip}%
\pgfsetbuttcap%
\pgfsetroundjoin%
\pgfsetlinewidth{1.505625pt}%
\definecolor{currentstroke}{rgb}{0.890196,0.466667,0.760784}%
\pgfsetstrokecolor{currentstroke}%
\pgfsetdash{}{0pt}%
\pgfpathmoveto{\pgfqpoint{3.272946in}{1.120249in}}%
\pgfpathlineto{\pgfqpoint{3.272946in}{1.120763in}}%
\pgfusepath{stroke}%
\end{pgfscope}%
\begin{pgfscope}%
\pgfpathrectangle{\pgfqpoint{1.286132in}{0.839159in}}{\pgfqpoint{12.053712in}{5.967710in}}%
\pgfusepath{clip}%
\pgfsetbuttcap%
\pgfsetroundjoin%
\pgfsetlinewidth{1.505625pt}%
\definecolor{currentstroke}{rgb}{0.890196,0.466667,0.760784}%
\pgfsetstrokecolor{currentstroke}%
\pgfsetdash{}{0pt}%
\pgfpathmoveto{\pgfqpoint{3.383632in}{1.120323in}}%
\pgfpathlineto{\pgfqpoint{3.383632in}{1.120514in}}%
\pgfusepath{stroke}%
\end{pgfscope}%
\begin{pgfscope}%
\pgfpathrectangle{\pgfqpoint{1.286132in}{0.839159in}}{\pgfqpoint{12.053712in}{5.967710in}}%
\pgfusepath{clip}%
\pgfsetbuttcap%
\pgfsetroundjoin%
\pgfsetlinewidth{1.505625pt}%
\definecolor{currentstroke}{rgb}{0.890196,0.466667,0.760784}%
\pgfsetstrokecolor{currentstroke}%
\pgfsetdash{}{0pt}%
\pgfpathmoveto{\pgfqpoint{3.494319in}{1.120462in}}%
\pgfpathlineto{\pgfqpoint{3.494319in}{1.120694in}}%
\pgfusepath{stroke}%
\end{pgfscope}%
\begin{pgfscope}%
\pgfpathrectangle{\pgfqpoint{1.286132in}{0.839159in}}{\pgfqpoint{12.053712in}{5.967710in}}%
\pgfusepath{clip}%
\pgfsetbuttcap%
\pgfsetroundjoin%
\pgfsetlinewidth{1.505625pt}%
\definecolor{currentstroke}{rgb}{0.890196,0.466667,0.760784}%
\pgfsetstrokecolor{currentstroke}%
\pgfsetdash{}{0pt}%
\pgfpathmoveto{\pgfqpoint{3.605005in}{1.120717in}}%
\pgfpathlineto{\pgfqpoint{3.605005in}{1.121141in}}%
\pgfusepath{stroke}%
\end{pgfscope}%
\begin{pgfscope}%
\pgfpathrectangle{\pgfqpoint{1.286132in}{0.839159in}}{\pgfqpoint{12.053712in}{5.967710in}}%
\pgfusepath{clip}%
\pgfsetbuttcap%
\pgfsetroundjoin%
\pgfsetlinewidth{1.505625pt}%
\definecolor{currentstroke}{rgb}{0.890196,0.466667,0.760784}%
\pgfsetstrokecolor{currentstroke}%
\pgfsetdash{}{0pt}%
\pgfpathmoveto{\pgfqpoint{3.715691in}{1.120652in}}%
\pgfpathlineto{\pgfqpoint{3.715691in}{1.120906in}}%
\pgfusepath{stroke}%
\end{pgfscope}%
\begin{pgfscope}%
\pgfpathrectangle{\pgfqpoint{1.286132in}{0.839159in}}{\pgfqpoint{12.053712in}{5.967710in}}%
\pgfusepath{clip}%
\pgfsetbuttcap%
\pgfsetroundjoin%
\pgfsetlinewidth{1.505625pt}%
\definecolor{currentstroke}{rgb}{0.890196,0.466667,0.760784}%
\pgfsetstrokecolor{currentstroke}%
\pgfsetdash{}{0pt}%
\pgfpathmoveto{\pgfqpoint{3.826377in}{1.120995in}}%
\pgfpathlineto{\pgfqpoint{3.826377in}{1.121516in}}%
\pgfusepath{stroke}%
\end{pgfscope}%
\begin{pgfscope}%
\pgfpathrectangle{\pgfqpoint{1.286132in}{0.839159in}}{\pgfqpoint{12.053712in}{5.967710in}}%
\pgfusepath{clip}%
\pgfsetbuttcap%
\pgfsetroundjoin%
\pgfsetlinewidth{1.505625pt}%
\definecolor{currentstroke}{rgb}{0.890196,0.466667,0.760784}%
\pgfsetstrokecolor{currentstroke}%
\pgfsetdash{}{0pt}%
\pgfpathmoveto{\pgfqpoint{3.937063in}{1.121013in}}%
\pgfpathlineto{\pgfqpoint{3.937063in}{1.121360in}}%
\pgfusepath{stroke}%
\end{pgfscope}%
\begin{pgfscope}%
\pgfpathrectangle{\pgfqpoint{1.286132in}{0.839159in}}{\pgfqpoint{12.053712in}{5.967710in}}%
\pgfusepath{clip}%
\pgfsetbuttcap%
\pgfsetroundjoin%
\pgfsetlinewidth{1.505625pt}%
\definecolor{currentstroke}{rgb}{0.890196,0.466667,0.760784}%
\pgfsetstrokecolor{currentstroke}%
\pgfsetdash{}{0pt}%
\pgfpathmoveto{\pgfqpoint{4.047749in}{1.121050in}}%
\pgfpathlineto{\pgfqpoint{4.047749in}{1.122303in}}%
\pgfusepath{stroke}%
\end{pgfscope}%
\begin{pgfscope}%
\pgfpathrectangle{\pgfqpoint{1.286132in}{0.839159in}}{\pgfqpoint{12.053712in}{5.967710in}}%
\pgfusepath{clip}%
\pgfsetbuttcap%
\pgfsetroundjoin%
\pgfsetlinewidth{1.505625pt}%
\definecolor{currentstroke}{rgb}{0.890196,0.466667,0.760784}%
\pgfsetstrokecolor{currentstroke}%
\pgfsetdash{}{0pt}%
\pgfpathmoveto{\pgfqpoint{4.158435in}{1.121355in}}%
\pgfpathlineto{\pgfqpoint{4.158435in}{1.121646in}}%
\pgfusepath{stroke}%
\end{pgfscope}%
\begin{pgfscope}%
\pgfpathrectangle{\pgfqpoint{1.286132in}{0.839159in}}{\pgfqpoint{12.053712in}{5.967710in}}%
\pgfusepath{clip}%
\pgfsetbuttcap%
\pgfsetroundjoin%
\pgfsetlinewidth{1.505625pt}%
\definecolor{currentstroke}{rgb}{0.890196,0.466667,0.760784}%
\pgfsetstrokecolor{currentstroke}%
\pgfsetdash{}{0pt}%
\pgfpathmoveto{\pgfqpoint{4.269121in}{1.121377in}}%
\pgfpathlineto{\pgfqpoint{4.269121in}{1.121572in}}%
\pgfusepath{stroke}%
\end{pgfscope}%
\begin{pgfscope}%
\pgfpathrectangle{\pgfqpoint{1.286132in}{0.839159in}}{\pgfqpoint{12.053712in}{5.967710in}}%
\pgfusepath{clip}%
\pgfsetbuttcap%
\pgfsetroundjoin%
\pgfsetlinewidth{1.505625pt}%
\definecolor{currentstroke}{rgb}{0.890196,0.466667,0.760784}%
\pgfsetstrokecolor{currentstroke}%
\pgfsetdash{}{0pt}%
\pgfpathmoveto{\pgfqpoint{4.379807in}{1.121349in}}%
\pgfpathlineto{\pgfqpoint{4.379807in}{1.122500in}}%
\pgfusepath{stroke}%
\end{pgfscope}%
\begin{pgfscope}%
\pgfpathrectangle{\pgfqpoint{1.286132in}{0.839159in}}{\pgfqpoint{12.053712in}{5.967710in}}%
\pgfusepath{clip}%
\pgfsetbuttcap%
\pgfsetroundjoin%
\pgfsetlinewidth{1.505625pt}%
\definecolor{currentstroke}{rgb}{0.890196,0.466667,0.760784}%
\pgfsetstrokecolor{currentstroke}%
\pgfsetdash{}{0pt}%
\pgfpathmoveto{\pgfqpoint{4.490493in}{1.121678in}}%
\pgfpathlineto{\pgfqpoint{4.490493in}{1.122233in}}%
\pgfusepath{stroke}%
\end{pgfscope}%
\begin{pgfscope}%
\pgfpathrectangle{\pgfqpoint{1.286132in}{0.839159in}}{\pgfqpoint{12.053712in}{5.967710in}}%
\pgfusepath{clip}%
\pgfsetbuttcap%
\pgfsetroundjoin%
\pgfsetlinewidth{1.505625pt}%
\definecolor{currentstroke}{rgb}{0.890196,0.466667,0.760784}%
\pgfsetstrokecolor{currentstroke}%
\pgfsetdash{}{0pt}%
\pgfpathmoveto{\pgfqpoint{4.601179in}{1.121931in}}%
\pgfpathlineto{\pgfqpoint{4.601179in}{1.122160in}}%
\pgfusepath{stroke}%
\end{pgfscope}%
\begin{pgfscope}%
\pgfpathrectangle{\pgfqpoint{1.286132in}{0.839159in}}{\pgfqpoint{12.053712in}{5.967710in}}%
\pgfusepath{clip}%
\pgfsetbuttcap%
\pgfsetroundjoin%
\pgfsetlinewidth{1.505625pt}%
\definecolor{currentstroke}{rgb}{0.890196,0.466667,0.760784}%
\pgfsetstrokecolor{currentstroke}%
\pgfsetdash{}{0pt}%
\pgfpathmoveto{\pgfqpoint{4.711865in}{1.122109in}}%
\pgfpathlineto{\pgfqpoint{4.711865in}{1.123306in}}%
\pgfusepath{stroke}%
\end{pgfscope}%
\begin{pgfscope}%
\pgfpathrectangle{\pgfqpoint{1.286132in}{0.839159in}}{\pgfqpoint{12.053712in}{5.967710in}}%
\pgfusepath{clip}%
\pgfsetbuttcap%
\pgfsetroundjoin%
\pgfsetlinewidth{1.505625pt}%
\definecolor{currentstroke}{rgb}{0.890196,0.466667,0.760784}%
\pgfsetstrokecolor{currentstroke}%
\pgfsetdash{}{0pt}%
\pgfpathmoveto{\pgfqpoint{4.822551in}{1.121904in}}%
\pgfpathlineto{\pgfqpoint{4.822551in}{1.123318in}}%
\pgfusepath{stroke}%
\end{pgfscope}%
\begin{pgfscope}%
\pgfpathrectangle{\pgfqpoint{1.286132in}{0.839159in}}{\pgfqpoint{12.053712in}{5.967710in}}%
\pgfusepath{clip}%
\pgfsetbuttcap%
\pgfsetroundjoin%
\pgfsetlinewidth{1.505625pt}%
\definecolor{currentstroke}{rgb}{0.890196,0.466667,0.760784}%
\pgfsetstrokecolor{currentstroke}%
\pgfsetdash{}{0pt}%
\pgfpathmoveto{\pgfqpoint{4.933237in}{1.122403in}}%
\pgfpathlineto{\pgfqpoint{4.933237in}{1.122633in}}%
\pgfusepath{stroke}%
\end{pgfscope}%
\begin{pgfscope}%
\pgfpathrectangle{\pgfqpoint{1.286132in}{0.839159in}}{\pgfqpoint{12.053712in}{5.967710in}}%
\pgfusepath{clip}%
\pgfsetbuttcap%
\pgfsetroundjoin%
\pgfsetlinewidth{1.505625pt}%
\definecolor{currentstroke}{rgb}{0.890196,0.466667,0.760784}%
\pgfsetstrokecolor{currentstroke}%
\pgfsetdash{}{0pt}%
\pgfpathmoveto{\pgfqpoint{5.043923in}{1.122454in}}%
\pgfpathlineto{\pgfqpoint{5.043923in}{1.122606in}}%
\pgfusepath{stroke}%
\end{pgfscope}%
\begin{pgfscope}%
\pgfpathrectangle{\pgfqpoint{1.286132in}{0.839159in}}{\pgfqpoint{12.053712in}{5.967710in}}%
\pgfusepath{clip}%
\pgfsetbuttcap%
\pgfsetroundjoin%
\pgfsetlinewidth{1.505625pt}%
\definecolor{currentstroke}{rgb}{0.890196,0.466667,0.760784}%
\pgfsetstrokecolor{currentstroke}%
\pgfsetdash{}{0pt}%
\pgfpathmoveto{\pgfqpoint{5.154609in}{1.122626in}}%
\pgfpathlineto{\pgfqpoint{5.154609in}{1.124524in}}%
\pgfusepath{stroke}%
\end{pgfscope}%
\begin{pgfscope}%
\pgfpathrectangle{\pgfqpoint{1.286132in}{0.839159in}}{\pgfqpoint{12.053712in}{5.967710in}}%
\pgfusepath{clip}%
\pgfsetbuttcap%
\pgfsetroundjoin%
\pgfsetlinewidth{1.505625pt}%
\definecolor{currentstroke}{rgb}{0.890196,0.466667,0.760784}%
\pgfsetstrokecolor{currentstroke}%
\pgfsetdash{}{0pt}%
\pgfpathmoveto{\pgfqpoint{5.265296in}{1.122863in}}%
\pgfpathlineto{\pgfqpoint{5.265296in}{1.122989in}}%
\pgfusepath{stroke}%
\end{pgfscope}%
\begin{pgfscope}%
\pgfpathrectangle{\pgfqpoint{1.286132in}{0.839159in}}{\pgfqpoint{12.053712in}{5.967710in}}%
\pgfusepath{clip}%
\pgfsetbuttcap%
\pgfsetroundjoin%
\pgfsetlinewidth{1.505625pt}%
\definecolor{currentstroke}{rgb}{0.890196,0.466667,0.760784}%
\pgfsetstrokecolor{currentstroke}%
\pgfsetdash{}{0pt}%
\pgfpathmoveto{\pgfqpoint{5.375982in}{1.123024in}}%
\pgfpathlineto{\pgfqpoint{5.375982in}{1.123281in}}%
\pgfusepath{stroke}%
\end{pgfscope}%
\begin{pgfscope}%
\pgfpathrectangle{\pgfqpoint{1.286132in}{0.839159in}}{\pgfqpoint{12.053712in}{5.967710in}}%
\pgfusepath{clip}%
\pgfsetbuttcap%
\pgfsetroundjoin%
\pgfsetlinewidth{1.505625pt}%
\definecolor{currentstroke}{rgb}{0.890196,0.466667,0.760784}%
\pgfsetstrokecolor{currentstroke}%
\pgfsetdash{}{0pt}%
\pgfpathmoveto{\pgfqpoint{5.486668in}{1.123204in}}%
\pgfpathlineto{\pgfqpoint{5.486668in}{1.123642in}}%
\pgfusepath{stroke}%
\end{pgfscope}%
\begin{pgfscope}%
\pgfpathrectangle{\pgfqpoint{1.286132in}{0.839159in}}{\pgfqpoint{12.053712in}{5.967710in}}%
\pgfusepath{clip}%
\pgfsetbuttcap%
\pgfsetroundjoin%
\pgfsetlinewidth{1.505625pt}%
\definecolor{currentstroke}{rgb}{0.890196,0.466667,0.760784}%
\pgfsetstrokecolor{currentstroke}%
\pgfsetdash{}{0pt}%
\pgfpathmoveto{\pgfqpoint{5.597354in}{1.122960in}}%
\pgfpathlineto{\pgfqpoint{5.597354in}{1.125004in}}%
\pgfusepath{stroke}%
\end{pgfscope}%
\begin{pgfscope}%
\pgfpathrectangle{\pgfqpoint{1.286132in}{0.839159in}}{\pgfqpoint{12.053712in}{5.967710in}}%
\pgfusepath{clip}%
\pgfsetbuttcap%
\pgfsetroundjoin%
\pgfsetlinewidth{1.505625pt}%
\definecolor{currentstroke}{rgb}{0.890196,0.466667,0.760784}%
\pgfsetstrokecolor{currentstroke}%
\pgfsetdash{}{0pt}%
\pgfpathmoveto{\pgfqpoint{5.708040in}{1.114483in}}%
\pgfpathlineto{\pgfqpoint{5.708040in}{1.151475in}}%
\pgfusepath{stroke}%
\end{pgfscope}%
\begin{pgfscope}%
\pgfpathrectangle{\pgfqpoint{1.286132in}{0.839159in}}{\pgfqpoint{12.053712in}{5.967710in}}%
\pgfusepath{clip}%
\pgfsetbuttcap%
\pgfsetroundjoin%
\pgfsetlinewidth{1.505625pt}%
\definecolor{currentstroke}{rgb}{0.890196,0.466667,0.760784}%
\pgfsetstrokecolor{currentstroke}%
\pgfsetdash{}{0pt}%
\pgfpathmoveto{\pgfqpoint{5.818726in}{1.123837in}}%
\pgfpathlineto{\pgfqpoint{5.818726in}{1.124091in}}%
\pgfusepath{stroke}%
\end{pgfscope}%
\begin{pgfscope}%
\pgfpathrectangle{\pgfqpoint{1.286132in}{0.839159in}}{\pgfqpoint{12.053712in}{5.967710in}}%
\pgfusepath{clip}%
\pgfsetbuttcap%
\pgfsetroundjoin%
\pgfsetlinewidth{1.505625pt}%
\definecolor{currentstroke}{rgb}{0.890196,0.466667,0.760784}%
\pgfsetstrokecolor{currentstroke}%
\pgfsetdash{}{0pt}%
\pgfpathmoveto{\pgfqpoint{5.929412in}{1.123612in}}%
\pgfpathlineto{\pgfqpoint{5.929412in}{1.125877in}}%
\pgfusepath{stroke}%
\end{pgfscope}%
\begin{pgfscope}%
\pgfpathrectangle{\pgfqpoint{1.286132in}{0.839159in}}{\pgfqpoint{12.053712in}{5.967710in}}%
\pgfusepath{clip}%
\pgfsetbuttcap%
\pgfsetroundjoin%
\pgfsetlinewidth{1.505625pt}%
\definecolor{currentstroke}{rgb}{0.890196,0.466667,0.760784}%
\pgfsetstrokecolor{currentstroke}%
\pgfsetdash{}{0pt}%
\pgfpathmoveto{\pgfqpoint{6.040098in}{1.124384in}}%
\pgfpathlineto{\pgfqpoint{6.040098in}{1.124818in}}%
\pgfusepath{stroke}%
\end{pgfscope}%
\begin{pgfscope}%
\pgfpathrectangle{\pgfqpoint{1.286132in}{0.839159in}}{\pgfqpoint{12.053712in}{5.967710in}}%
\pgfusepath{clip}%
\pgfsetbuttcap%
\pgfsetroundjoin%
\pgfsetlinewidth{1.505625pt}%
\definecolor{currentstroke}{rgb}{0.890196,0.466667,0.760784}%
\pgfsetstrokecolor{currentstroke}%
\pgfsetdash{}{0pt}%
\pgfpathmoveto{\pgfqpoint{6.150784in}{1.124621in}}%
\pgfpathlineto{\pgfqpoint{6.150784in}{1.125436in}}%
\pgfusepath{stroke}%
\end{pgfscope}%
\begin{pgfscope}%
\pgfpathrectangle{\pgfqpoint{1.286132in}{0.839159in}}{\pgfqpoint{12.053712in}{5.967710in}}%
\pgfusepath{clip}%
\pgfsetbuttcap%
\pgfsetroundjoin%
\pgfsetlinewidth{1.505625pt}%
\definecolor{currentstroke}{rgb}{0.890196,0.466667,0.760784}%
\pgfsetstrokecolor{currentstroke}%
\pgfsetdash{}{0pt}%
\pgfpathmoveto{\pgfqpoint{6.261470in}{1.124888in}}%
\pgfpathlineto{\pgfqpoint{6.261470in}{1.125404in}}%
\pgfusepath{stroke}%
\end{pgfscope}%
\begin{pgfscope}%
\pgfpathrectangle{\pgfqpoint{1.286132in}{0.839159in}}{\pgfqpoint{12.053712in}{5.967710in}}%
\pgfusepath{clip}%
\pgfsetbuttcap%
\pgfsetroundjoin%
\pgfsetlinewidth{1.505625pt}%
\definecolor{currentstroke}{rgb}{0.890196,0.466667,0.760784}%
\pgfsetstrokecolor{currentstroke}%
\pgfsetdash{}{0pt}%
\pgfpathmoveto{\pgfqpoint{6.372156in}{1.124748in}}%
\pgfpathlineto{\pgfqpoint{6.372156in}{1.126704in}}%
\pgfusepath{stroke}%
\end{pgfscope}%
\begin{pgfscope}%
\pgfpathrectangle{\pgfqpoint{1.286132in}{0.839159in}}{\pgfqpoint{12.053712in}{5.967710in}}%
\pgfusepath{clip}%
\pgfsetbuttcap%
\pgfsetroundjoin%
\pgfsetlinewidth{1.505625pt}%
\definecolor{currentstroke}{rgb}{0.890196,0.466667,0.760784}%
\pgfsetstrokecolor{currentstroke}%
\pgfsetdash{}{0pt}%
\pgfpathmoveto{\pgfqpoint{6.482842in}{1.125158in}}%
\pgfpathlineto{\pgfqpoint{6.482842in}{1.125412in}}%
\pgfusepath{stroke}%
\end{pgfscope}%
\begin{pgfscope}%
\pgfpathrectangle{\pgfqpoint{1.286132in}{0.839159in}}{\pgfqpoint{12.053712in}{5.967710in}}%
\pgfusepath{clip}%
\pgfsetbuttcap%
\pgfsetroundjoin%
\pgfsetlinewidth{1.505625pt}%
\definecolor{currentstroke}{rgb}{0.890196,0.466667,0.760784}%
\pgfsetstrokecolor{currentstroke}%
\pgfsetdash{}{0pt}%
\pgfpathmoveto{\pgfqpoint{6.593528in}{1.125450in}}%
\pgfpathlineto{\pgfqpoint{6.593528in}{1.125549in}}%
\pgfusepath{stroke}%
\end{pgfscope}%
\begin{pgfscope}%
\pgfpathrectangle{\pgfqpoint{1.286132in}{0.839159in}}{\pgfqpoint{12.053712in}{5.967710in}}%
\pgfusepath{clip}%
\pgfsetbuttcap%
\pgfsetroundjoin%
\pgfsetlinewidth{1.505625pt}%
\definecolor{currentstroke}{rgb}{0.890196,0.466667,0.760784}%
\pgfsetstrokecolor{currentstroke}%
\pgfsetdash{}{0pt}%
\pgfpathmoveto{\pgfqpoint{6.704214in}{1.125670in}}%
\pgfpathlineto{\pgfqpoint{6.704214in}{1.126026in}}%
\pgfusepath{stroke}%
\end{pgfscope}%
\begin{pgfscope}%
\pgfpathrectangle{\pgfqpoint{1.286132in}{0.839159in}}{\pgfqpoint{12.053712in}{5.967710in}}%
\pgfusepath{clip}%
\pgfsetbuttcap%
\pgfsetroundjoin%
\pgfsetlinewidth{1.505625pt}%
\definecolor{currentstroke}{rgb}{0.890196,0.466667,0.760784}%
\pgfsetstrokecolor{currentstroke}%
\pgfsetdash{}{0pt}%
\pgfpathmoveto{\pgfqpoint{6.814900in}{1.125556in}}%
\pgfpathlineto{\pgfqpoint{6.814900in}{1.127720in}}%
\pgfusepath{stroke}%
\end{pgfscope}%
\begin{pgfscope}%
\pgfpathrectangle{\pgfqpoint{1.286132in}{0.839159in}}{\pgfqpoint{12.053712in}{5.967710in}}%
\pgfusepath{clip}%
\pgfsetbuttcap%
\pgfsetroundjoin%
\pgfsetlinewidth{1.505625pt}%
\definecolor{currentstroke}{rgb}{0.890196,0.466667,0.760784}%
\pgfsetstrokecolor{currentstroke}%
\pgfsetdash{}{0pt}%
\pgfpathmoveto{\pgfqpoint{6.925586in}{1.125887in}}%
\pgfpathlineto{\pgfqpoint{6.925586in}{1.127402in}}%
\pgfusepath{stroke}%
\end{pgfscope}%
\begin{pgfscope}%
\pgfpathrectangle{\pgfqpoint{1.286132in}{0.839159in}}{\pgfqpoint{12.053712in}{5.967710in}}%
\pgfusepath{clip}%
\pgfsetbuttcap%
\pgfsetroundjoin%
\pgfsetlinewidth{1.505625pt}%
\definecolor{currentstroke}{rgb}{0.890196,0.466667,0.760784}%
\pgfsetstrokecolor{currentstroke}%
\pgfsetdash{}{0pt}%
\pgfpathmoveto{\pgfqpoint{7.036272in}{1.126032in}}%
\pgfpathlineto{\pgfqpoint{7.036272in}{1.128186in}}%
\pgfusepath{stroke}%
\end{pgfscope}%
\begin{pgfscope}%
\pgfpathrectangle{\pgfqpoint{1.286132in}{0.839159in}}{\pgfqpoint{12.053712in}{5.967710in}}%
\pgfusepath{clip}%
\pgfsetbuttcap%
\pgfsetroundjoin%
\pgfsetlinewidth{1.505625pt}%
\definecolor{currentstroke}{rgb}{0.890196,0.466667,0.760784}%
\pgfsetstrokecolor{currentstroke}%
\pgfsetdash{}{0pt}%
\pgfpathmoveto{\pgfqpoint{7.146959in}{1.126720in}}%
\pgfpathlineto{\pgfqpoint{7.146959in}{1.127002in}}%
\pgfusepath{stroke}%
\end{pgfscope}%
\begin{pgfscope}%
\pgfpathrectangle{\pgfqpoint{1.286132in}{0.839159in}}{\pgfqpoint{12.053712in}{5.967710in}}%
\pgfusepath{clip}%
\pgfsetbuttcap%
\pgfsetroundjoin%
\pgfsetlinewidth{1.505625pt}%
\definecolor{currentstroke}{rgb}{0.890196,0.466667,0.760784}%
\pgfsetstrokecolor{currentstroke}%
\pgfsetdash{}{0pt}%
\pgfpathmoveto{\pgfqpoint{7.257645in}{1.126967in}}%
\pgfpathlineto{\pgfqpoint{7.257645in}{1.127273in}}%
\pgfusepath{stroke}%
\end{pgfscope}%
\begin{pgfscope}%
\pgfpathrectangle{\pgfqpoint{1.286132in}{0.839159in}}{\pgfqpoint{12.053712in}{5.967710in}}%
\pgfusepath{clip}%
\pgfsetbuttcap%
\pgfsetroundjoin%
\pgfsetlinewidth{1.505625pt}%
\definecolor{currentstroke}{rgb}{0.890196,0.466667,0.760784}%
\pgfsetstrokecolor{currentstroke}%
\pgfsetdash{}{0pt}%
\pgfpathmoveto{\pgfqpoint{7.368331in}{1.126084in}}%
\pgfpathlineto{\pgfqpoint{7.368331in}{1.131642in}}%
\pgfusepath{stroke}%
\end{pgfscope}%
\begin{pgfscope}%
\pgfpathrectangle{\pgfqpoint{1.286132in}{0.839159in}}{\pgfqpoint{12.053712in}{5.967710in}}%
\pgfusepath{clip}%
\pgfsetbuttcap%
\pgfsetroundjoin%
\pgfsetlinewidth{1.505625pt}%
\definecolor{currentstroke}{rgb}{0.890196,0.466667,0.760784}%
\pgfsetstrokecolor{currentstroke}%
\pgfsetdash{}{0pt}%
\pgfpathmoveto{\pgfqpoint{7.479017in}{1.127181in}}%
\pgfpathlineto{\pgfqpoint{7.479017in}{1.129538in}}%
\pgfusepath{stroke}%
\end{pgfscope}%
\begin{pgfscope}%
\pgfpathrectangle{\pgfqpoint{1.286132in}{0.839159in}}{\pgfqpoint{12.053712in}{5.967710in}}%
\pgfusepath{clip}%
\pgfsetbuttcap%
\pgfsetroundjoin%
\pgfsetlinewidth{1.505625pt}%
\definecolor{currentstroke}{rgb}{0.890196,0.466667,0.760784}%
\pgfsetstrokecolor{currentstroke}%
\pgfsetdash{}{0pt}%
\pgfpathmoveto{\pgfqpoint{7.589703in}{1.118617in}}%
\pgfpathlineto{\pgfqpoint{7.589703in}{1.156164in}}%
\pgfusepath{stroke}%
\end{pgfscope}%
\begin{pgfscope}%
\pgfpathrectangle{\pgfqpoint{1.286132in}{0.839159in}}{\pgfqpoint{12.053712in}{5.967710in}}%
\pgfusepath{clip}%
\pgfsetbuttcap%
\pgfsetroundjoin%
\pgfsetlinewidth{1.505625pt}%
\definecolor{currentstroke}{rgb}{0.890196,0.466667,0.760784}%
\pgfsetstrokecolor{currentstroke}%
\pgfsetdash{}{0pt}%
\pgfpathmoveto{\pgfqpoint{7.700389in}{1.128158in}}%
\pgfpathlineto{\pgfqpoint{7.700389in}{1.129985in}}%
\pgfusepath{stroke}%
\end{pgfscope}%
\begin{pgfscope}%
\pgfpathrectangle{\pgfqpoint{1.286132in}{0.839159in}}{\pgfqpoint{12.053712in}{5.967710in}}%
\pgfusepath{clip}%
\pgfsetbuttcap%
\pgfsetroundjoin%
\pgfsetlinewidth{1.505625pt}%
\definecolor{currentstroke}{rgb}{0.890196,0.466667,0.760784}%
\pgfsetstrokecolor{currentstroke}%
\pgfsetdash{}{0pt}%
\pgfpathmoveto{\pgfqpoint{7.811075in}{1.128483in}}%
\pgfpathlineto{\pgfqpoint{7.811075in}{1.129052in}}%
\pgfusepath{stroke}%
\end{pgfscope}%
\begin{pgfscope}%
\pgfpathrectangle{\pgfqpoint{1.286132in}{0.839159in}}{\pgfqpoint{12.053712in}{5.967710in}}%
\pgfusepath{clip}%
\pgfsetbuttcap%
\pgfsetroundjoin%
\pgfsetlinewidth{1.505625pt}%
\definecolor{currentstroke}{rgb}{0.890196,0.466667,0.760784}%
\pgfsetstrokecolor{currentstroke}%
\pgfsetdash{}{0pt}%
\pgfpathmoveto{\pgfqpoint{7.921761in}{1.128445in}}%
\pgfpathlineto{\pgfqpoint{7.921761in}{1.130162in}}%
\pgfusepath{stroke}%
\end{pgfscope}%
\begin{pgfscope}%
\pgfpathrectangle{\pgfqpoint{1.286132in}{0.839159in}}{\pgfqpoint{12.053712in}{5.967710in}}%
\pgfusepath{clip}%
\pgfsetbuttcap%
\pgfsetroundjoin%
\pgfsetlinewidth{1.505625pt}%
\definecolor{currentstroke}{rgb}{0.890196,0.466667,0.760784}%
\pgfsetstrokecolor{currentstroke}%
\pgfsetdash{}{0pt}%
\pgfpathmoveto{\pgfqpoint{8.032447in}{1.128955in}}%
\pgfpathlineto{\pgfqpoint{8.032447in}{1.129244in}}%
\pgfusepath{stroke}%
\end{pgfscope}%
\begin{pgfscope}%
\pgfpathrectangle{\pgfqpoint{1.286132in}{0.839159in}}{\pgfqpoint{12.053712in}{5.967710in}}%
\pgfusepath{clip}%
\pgfsetbuttcap%
\pgfsetroundjoin%
\pgfsetlinewidth{1.505625pt}%
\definecolor{currentstroke}{rgb}{0.890196,0.466667,0.760784}%
\pgfsetstrokecolor{currentstroke}%
\pgfsetdash{}{0pt}%
\pgfpathmoveto{\pgfqpoint{8.143133in}{1.129068in}}%
\pgfpathlineto{\pgfqpoint{8.143133in}{1.129524in}}%
\pgfusepath{stroke}%
\end{pgfscope}%
\begin{pgfscope}%
\pgfpathrectangle{\pgfqpoint{1.286132in}{0.839159in}}{\pgfqpoint{12.053712in}{5.967710in}}%
\pgfusepath{clip}%
\pgfsetbuttcap%
\pgfsetroundjoin%
\pgfsetlinewidth{1.505625pt}%
\definecolor{currentstroke}{rgb}{0.890196,0.466667,0.760784}%
\pgfsetstrokecolor{currentstroke}%
\pgfsetdash{}{0pt}%
\pgfpathmoveto{\pgfqpoint{8.253819in}{1.129381in}}%
\pgfpathlineto{\pgfqpoint{8.253819in}{1.129442in}}%
\pgfusepath{stroke}%
\end{pgfscope}%
\begin{pgfscope}%
\pgfpathrectangle{\pgfqpoint{1.286132in}{0.839159in}}{\pgfqpoint{12.053712in}{5.967710in}}%
\pgfusepath{clip}%
\pgfsetbuttcap%
\pgfsetroundjoin%
\pgfsetlinewidth{1.505625pt}%
\definecolor{currentstroke}{rgb}{0.890196,0.466667,0.760784}%
\pgfsetstrokecolor{currentstroke}%
\pgfsetdash{}{0pt}%
\pgfpathmoveto{\pgfqpoint{8.364505in}{1.129682in}}%
\pgfpathlineto{\pgfqpoint{8.364505in}{1.130556in}}%
\pgfusepath{stroke}%
\end{pgfscope}%
\begin{pgfscope}%
\pgfpathrectangle{\pgfqpoint{1.286132in}{0.839159in}}{\pgfqpoint{12.053712in}{5.967710in}}%
\pgfusepath{clip}%
\pgfsetbuttcap%
\pgfsetroundjoin%
\pgfsetlinewidth{1.505625pt}%
\definecolor{currentstroke}{rgb}{0.890196,0.466667,0.760784}%
\pgfsetstrokecolor{currentstroke}%
\pgfsetdash{}{0pt}%
\pgfpathmoveto{\pgfqpoint{8.475191in}{1.129956in}}%
\pgfpathlineto{\pgfqpoint{8.475191in}{1.130223in}}%
\pgfusepath{stroke}%
\end{pgfscope}%
\begin{pgfscope}%
\pgfpathrectangle{\pgfqpoint{1.286132in}{0.839159in}}{\pgfqpoint{12.053712in}{5.967710in}}%
\pgfusepath{clip}%
\pgfsetbuttcap%
\pgfsetroundjoin%
\pgfsetlinewidth{1.505625pt}%
\definecolor{currentstroke}{rgb}{0.890196,0.466667,0.760784}%
\pgfsetstrokecolor{currentstroke}%
\pgfsetdash{}{0pt}%
\pgfpathmoveto{\pgfqpoint{8.585877in}{1.130497in}}%
\pgfpathlineto{\pgfqpoint{8.585877in}{1.131018in}}%
\pgfusepath{stroke}%
\end{pgfscope}%
\begin{pgfscope}%
\pgfpathrectangle{\pgfqpoint{1.286132in}{0.839159in}}{\pgfqpoint{12.053712in}{5.967710in}}%
\pgfusepath{clip}%
\pgfsetbuttcap%
\pgfsetroundjoin%
\pgfsetlinewidth{1.505625pt}%
\definecolor{currentstroke}{rgb}{0.890196,0.466667,0.760784}%
\pgfsetstrokecolor{currentstroke}%
\pgfsetdash{}{0pt}%
\pgfpathmoveto{\pgfqpoint{8.696563in}{1.130435in}}%
\pgfpathlineto{\pgfqpoint{8.696563in}{1.131993in}}%
\pgfusepath{stroke}%
\end{pgfscope}%
\begin{pgfscope}%
\pgfpathrectangle{\pgfqpoint{1.286132in}{0.839159in}}{\pgfqpoint{12.053712in}{5.967710in}}%
\pgfusepath{clip}%
\pgfsetbuttcap%
\pgfsetroundjoin%
\pgfsetlinewidth{1.505625pt}%
\definecolor{currentstroke}{rgb}{0.890196,0.466667,0.760784}%
\pgfsetstrokecolor{currentstroke}%
\pgfsetdash{}{0pt}%
\pgfpathmoveto{\pgfqpoint{8.807249in}{1.131088in}}%
\pgfpathlineto{\pgfqpoint{8.807249in}{1.131878in}}%
\pgfusepath{stroke}%
\end{pgfscope}%
\begin{pgfscope}%
\pgfpathrectangle{\pgfqpoint{1.286132in}{0.839159in}}{\pgfqpoint{12.053712in}{5.967710in}}%
\pgfusepath{clip}%
\pgfsetbuttcap%
\pgfsetroundjoin%
\pgfsetlinewidth{1.505625pt}%
\definecolor{currentstroke}{rgb}{0.890196,0.466667,0.760784}%
\pgfsetstrokecolor{currentstroke}%
\pgfsetdash{}{0pt}%
\pgfpathmoveto{\pgfqpoint{8.917936in}{1.131463in}}%
\pgfpathlineto{\pgfqpoint{8.917936in}{1.131861in}}%
\pgfusepath{stroke}%
\end{pgfscope}%
\begin{pgfscope}%
\pgfpathrectangle{\pgfqpoint{1.286132in}{0.839159in}}{\pgfqpoint{12.053712in}{5.967710in}}%
\pgfusepath{clip}%
\pgfsetbuttcap%
\pgfsetroundjoin%
\pgfsetlinewidth{1.505625pt}%
\definecolor{currentstroke}{rgb}{0.890196,0.466667,0.760784}%
\pgfsetstrokecolor{currentstroke}%
\pgfsetdash{}{0pt}%
\pgfpathmoveto{\pgfqpoint{9.028622in}{1.131479in}}%
\pgfpathlineto{\pgfqpoint{9.028622in}{1.134352in}}%
\pgfusepath{stroke}%
\end{pgfscope}%
\begin{pgfscope}%
\pgfpathrectangle{\pgfqpoint{1.286132in}{0.839159in}}{\pgfqpoint{12.053712in}{5.967710in}}%
\pgfusepath{clip}%
\pgfsetbuttcap%
\pgfsetroundjoin%
\pgfsetlinewidth{1.505625pt}%
\definecolor{currentstroke}{rgb}{0.890196,0.466667,0.760784}%
\pgfsetstrokecolor{currentstroke}%
\pgfsetdash{}{0pt}%
\pgfpathmoveto{\pgfqpoint{9.139308in}{1.131624in}}%
\pgfpathlineto{\pgfqpoint{9.139308in}{1.134149in}}%
\pgfusepath{stroke}%
\end{pgfscope}%
\begin{pgfscope}%
\pgfpathrectangle{\pgfqpoint{1.286132in}{0.839159in}}{\pgfqpoint{12.053712in}{5.967710in}}%
\pgfusepath{clip}%
\pgfsetbuttcap%
\pgfsetroundjoin%
\pgfsetlinewidth{1.505625pt}%
\definecolor{currentstroke}{rgb}{0.890196,0.466667,0.760784}%
\pgfsetstrokecolor{currentstroke}%
\pgfsetdash{}{0pt}%
\pgfpathmoveto{\pgfqpoint{9.249994in}{1.132228in}}%
\pgfpathlineto{\pgfqpoint{9.249994in}{1.133307in}}%
\pgfusepath{stroke}%
\end{pgfscope}%
\begin{pgfscope}%
\pgfpathrectangle{\pgfqpoint{1.286132in}{0.839159in}}{\pgfqpoint{12.053712in}{5.967710in}}%
\pgfusepath{clip}%
\pgfsetbuttcap%
\pgfsetroundjoin%
\pgfsetlinewidth{1.505625pt}%
\definecolor{currentstroke}{rgb}{0.890196,0.466667,0.760784}%
\pgfsetstrokecolor{currentstroke}%
\pgfsetdash{}{0pt}%
\pgfpathmoveto{\pgfqpoint{9.360680in}{1.132493in}}%
\pgfpathlineto{\pgfqpoint{9.360680in}{1.133883in}}%
\pgfusepath{stroke}%
\end{pgfscope}%
\begin{pgfscope}%
\pgfpathrectangle{\pgfqpoint{1.286132in}{0.839159in}}{\pgfqpoint{12.053712in}{5.967710in}}%
\pgfusepath{clip}%
\pgfsetbuttcap%
\pgfsetroundjoin%
\pgfsetlinewidth{1.505625pt}%
\definecolor{currentstroke}{rgb}{0.890196,0.466667,0.760784}%
\pgfsetstrokecolor{currentstroke}%
\pgfsetdash{}{0pt}%
\pgfpathmoveto{\pgfqpoint{9.471366in}{1.133255in}}%
\pgfpathlineto{\pgfqpoint{9.471366in}{1.133888in}}%
\pgfusepath{stroke}%
\end{pgfscope}%
\begin{pgfscope}%
\pgfpathrectangle{\pgfqpoint{1.286132in}{0.839159in}}{\pgfqpoint{12.053712in}{5.967710in}}%
\pgfusepath{clip}%
\pgfsetbuttcap%
\pgfsetroundjoin%
\pgfsetlinewidth{1.505625pt}%
\definecolor{currentstroke}{rgb}{0.890196,0.466667,0.760784}%
\pgfsetstrokecolor{currentstroke}%
\pgfsetdash{}{0pt}%
\pgfpathmoveto{\pgfqpoint{9.582052in}{1.133176in}}%
\pgfpathlineto{\pgfqpoint{9.582052in}{1.136083in}}%
\pgfusepath{stroke}%
\end{pgfscope}%
\begin{pgfscope}%
\pgfpathrectangle{\pgfqpoint{1.286132in}{0.839159in}}{\pgfqpoint{12.053712in}{5.967710in}}%
\pgfusepath{clip}%
\pgfsetbuttcap%
\pgfsetroundjoin%
\pgfsetlinewidth{1.505625pt}%
\definecolor{currentstroke}{rgb}{0.890196,0.466667,0.760784}%
\pgfsetstrokecolor{currentstroke}%
\pgfsetdash{}{0pt}%
\pgfpathmoveto{\pgfqpoint{9.692738in}{1.133494in}}%
\pgfpathlineto{\pgfqpoint{9.692738in}{1.135919in}}%
\pgfusepath{stroke}%
\end{pgfscope}%
\begin{pgfscope}%
\pgfpathrectangle{\pgfqpoint{1.286132in}{0.839159in}}{\pgfqpoint{12.053712in}{5.967710in}}%
\pgfusepath{clip}%
\pgfsetbuttcap%
\pgfsetroundjoin%
\pgfsetlinewidth{1.505625pt}%
\definecolor{currentstroke}{rgb}{0.890196,0.466667,0.760784}%
\pgfsetstrokecolor{currentstroke}%
\pgfsetdash{}{0pt}%
\pgfpathmoveto{\pgfqpoint{9.803424in}{1.134187in}}%
\pgfpathlineto{\pgfqpoint{9.803424in}{1.134758in}}%
\pgfusepath{stroke}%
\end{pgfscope}%
\begin{pgfscope}%
\pgfpathrectangle{\pgfqpoint{1.286132in}{0.839159in}}{\pgfqpoint{12.053712in}{5.967710in}}%
\pgfusepath{clip}%
\pgfsetbuttcap%
\pgfsetroundjoin%
\pgfsetlinewidth{1.505625pt}%
\definecolor{currentstroke}{rgb}{0.890196,0.466667,0.760784}%
\pgfsetstrokecolor{currentstroke}%
\pgfsetdash{}{0pt}%
\pgfpathmoveto{\pgfqpoint{9.914110in}{1.123317in}}%
\pgfpathlineto{\pgfqpoint{9.914110in}{1.172873in}}%
\pgfusepath{stroke}%
\end{pgfscope}%
\begin{pgfscope}%
\pgfpathrectangle{\pgfqpoint{1.286132in}{0.839159in}}{\pgfqpoint{12.053712in}{5.967710in}}%
\pgfusepath{clip}%
\pgfsetbuttcap%
\pgfsetroundjoin%
\pgfsetlinewidth{1.505625pt}%
\definecolor{currentstroke}{rgb}{0.890196,0.466667,0.760784}%
\pgfsetstrokecolor{currentstroke}%
\pgfsetdash{}{0pt}%
\pgfpathmoveto{\pgfqpoint{10.024796in}{1.135320in}}%
\pgfpathlineto{\pgfqpoint{10.024796in}{1.136433in}}%
\pgfusepath{stroke}%
\end{pgfscope}%
\begin{pgfscope}%
\pgfpathrectangle{\pgfqpoint{1.286132in}{0.839159in}}{\pgfqpoint{12.053712in}{5.967710in}}%
\pgfusepath{clip}%
\pgfsetbuttcap%
\pgfsetroundjoin%
\pgfsetlinewidth{1.505625pt}%
\definecolor{currentstroke}{rgb}{0.890196,0.466667,0.760784}%
\pgfsetstrokecolor{currentstroke}%
\pgfsetdash{}{0pt}%
\pgfpathmoveto{\pgfqpoint{10.135482in}{1.135190in}}%
\pgfpathlineto{\pgfqpoint{10.135482in}{1.137997in}}%
\pgfusepath{stroke}%
\end{pgfscope}%
\begin{pgfscope}%
\pgfpathrectangle{\pgfqpoint{1.286132in}{0.839159in}}{\pgfqpoint{12.053712in}{5.967710in}}%
\pgfusepath{clip}%
\pgfsetbuttcap%
\pgfsetroundjoin%
\pgfsetlinewidth{1.505625pt}%
\definecolor{currentstroke}{rgb}{0.890196,0.466667,0.760784}%
\pgfsetstrokecolor{currentstroke}%
\pgfsetdash{}{0pt}%
\pgfpathmoveto{\pgfqpoint{10.246168in}{1.135937in}}%
\pgfpathlineto{\pgfqpoint{10.246168in}{1.137427in}}%
\pgfusepath{stroke}%
\end{pgfscope}%
\begin{pgfscope}%
\pgfpathrectangle{\pgfqpoint{1.286132in}{0.839159in}}{\pgfqpoint{12.053712in}{5.967710in}}%
\pgfusepath{clip}%
\pgfsetbuttcap%
\pgfsetroundjoin%
\pgfsetlinewidth{1.505625pt}%
\definecolor{currentstroke}{rgb}{0.890196,0.466667,0.760784}%
\pgfsetstrokecolor{currentstroke}%
\pgfsetdash{}{0pt}%
\pgfpathmoveto{\pgfqpoint{10.356854in}{1.135402in}}%
\pgfpathlineto{\pgfqpoint{10.356854in}{1.138995in}}%
\pgfusepath{stroke}%
\end{pgfscope}%
\begin{pgfscope}%
\pgfpathrectangle{\pgfqpoint{1.286132in}{0.839159in}}{\pgfqpoint{12.053712in}{5.967710in}}%
\pgfusepath{clip}%
\pgfsetbuttcap%
\pgfsetroundjoin%
\pgfsetlinewidth{1.505625pt}%
\definecolor{currentstroke}{rgb}{0.890196,0.466667,0.760784}%
\pgfsetstrokecolor{currentstroke}%
\pgfsetdash{}{0pt}%
\pgfpathmoveto{\pgfqpoint{10.467540in}{1.131431in}}%
\pgfpathlineto{\pgfqpoint{10.467540in}{1.156384in}}%
\pgfusepath{stroke}%
\end{pgfscope}%
\begin{pgfscope}%
\pgfpathrectangle{\pgfqpoint{1.286132in}{0.839159in}}{\pgfqpoint{12.053712in}{5.967710in}}%
\pgfusepath{clip}%
\pgfsetbuttcap%
\pgfsetroundjoin%
\pgfsetlinewidth{1.505625pt}%
\definecolor{currentstroke}{rgb}{0.890196,0.466667,0.760784}%
\pgfsetstrokecolor{currentstroke}%
\pgfsetdash{}{0pt}%
\pgfpathmoveto{\pgfqpoint{10.578226in}{1.137364in}}%
\pgfpathlineto{\pgfqpoint{10.578226in}{1.138733in}}%
\pgfusepath{stroke}%
\end{pgfscope}%
\begin{pgfscope}%
\pgfpathrectangle{\pgfqpoint{1.286132in}{0.839159in}}{\pgfqpoint{12.053712in}{5.967710in}}%
\pgfusepath{clip}%
\pgfsetbuttcap%
\pgfsetroundjoin%
\pgfsetlinewidth{1.505625pt}%
\definecolor{currentstroke}{rgb}{0.890196,0.466667,0.760784}%
\pgfsetstrokecolor{currentstroke}%
\pgfsetdash{}{0pt}%
\pgfpathmoveto{\pgfqpoint{10.688913in}{1.137561in}}%
\pgfpathlineto{\pgfqpoint{10.688913in}{1.138212in}}%
\pgfusepath{stroke}%
\end{pgfscope}%
\begin{pgfscope}%
\pgfpathrectangle{\pgfqpoint{1.286132in}{0.839159in}}{\pgfqpoint{12.053712in}{5.967710in}}%
\pgfusepath{clip}%
\pgfsetbuttcap%
\pgfsetroundjoin%
\pgfsetlinewidth{1.505625pt}%
\definecolor{currentstroke}{rgb}{0.890196,0.466667,0.760784}%
\pgfsetstrokecolor{currentstroke}%
\pgfsetdash{}{0pt}%
\pgfpathmoveto{\pgfqpoint{10.799599in}{1.138216in}}%
\pgfpathlineto{\pgfqpoint{10.799599in}{1.139195in}}%
\pgfusepath{stroke}%
\end{pgfscope}%
\begin{pgfscope}%
\pgfpathrectangle{\pgfqpoint{1.286132in}{0.839159in}}{\pgfqpoint{12.053712in}{5.967710in}}%
\pgfusepath{clip}%
\pgfsetbuttcap%
\pgfsetroundjoin%
\pgfsetlinewidth{1.505625pt}%
\definecolor{currentstroke}{rgb}{0.890196,0.466667,0.760784}%
\pgfsetstrokecolor{currentstroke}%
\pgfsetdash{}{0pt}%
\pgfpathmoveto{\pgfqpoint{10.910285in}{1.138087in}}%
\pgfpathlineto{\pgfqpoint{10.910285in}{1.140771in}}%
\pgfusepath{stroke}%
\end{pgfscope}%
\begin{pgfscope}%
\pgfpathrectangle{\pgfqpoint{1.286132in}{0.839159in}}{\pgfqpoint{12.053712in}{5.967710in}}%
\pgfusepath{clip}%
\pgfsetbuttcap%
\pgfsetroundjoin%
\pgfsetlinewidth{1.505625pt}%
\definecolor{currentstroke}{rgb}{0.890196,0.466667,0.760784}%
\pgfsetstrokecolor{currentstroke}%
\pgfsetdash{}{0pt}%
\pgfpathmoveto{\pgfqpoint{11.020971in}{1.139103in}}%
\pgfpathlineto{\pgfqpoint{11.020971in}{1.142730in}}%
\pgfusepath{stroke}%
\end{pgfscope}%
\begin{pgfscope}%
\pgfpathrectangle{\pgfqpoint{1.286132in}{0.839159in}}{\pgfqpoint{12.053712in}{5.967710in}}%
\pgfusepath{clip}%
\pgfsetbuttcap%
\pgfsetroundjoin%
\pgfsetlinewidth{1.505625pt}%
\definecolor{currentstroke}{rgb}{0.890196,0.466667,0.760784}%
\pgfsetstrokecolor{currentstroke}%
\pgfsetdash{}{0pt}%
\pgfpathmoveto{\pgfqpoint{11.131657in}{1.139305in}}%
\pgfpathlineto{\pgfqpoint{11.131657in}{1.139439in}}%
\pgfusepath{stroke}%
\end{pgfscope}%
\begin{pgfscope}%
\pgfpathrectangle{\pgfqpoint{1.286132in}{0.839159in}}{\pgfqpoint{12.053712in}{5.967710in}}%
\pgfusepath{clip}%
\pgfsetbuttcap%
\pgfsetroundjoin%
\pgfsetlinewidth{1.505625pt}%
\definecolor{currentstroke}{rgb}{0.890196,0.466667,0.760784}%
\pgfsetstrokecolor{currentstroke}%
\pgfsetdash{}{0pt}%
\pgfpathmoveto{\pgfqpoint{11.242343in}{1.139701in}}%
\pgfpathlineto{\pgfqpoint{11.242343in}{1.140237in}}%
\pgfusepath{stroke}%
\end{pgfscope}%
\begin{pgfscope}%
\pgfpathrectangle{\pgfqpoint{1.286132in}{0.839159in}}{\pgfqpoint{12.053712in}{5.967710in}}%
\pgfusepath{clip}%
\pgfsetbuttcap%
\pgfsetroundjoin%
\pgfsetlinewidth{1.505625pt}%
\definecolor{currentstroke}{rgb}{0.890196,0.466667,0.760784}%
\pgfsetstrokecolor{currentstroke}%
\pgfsetdash{}{0pt}%
\pgfpathmoveto{\pgfqpoint{11.353029in}{1.139715in}}%
\pgfpathlineto{\pgfqpoint{11.353029in}{1.141952in}}%
\pgfusepath{stroke}%
\end{pgfscope}%
\begin{pgfscope}%
\pgfpathrectangle{\pgfqpoint{1.286132in}{0.839159in}}{\pgfqpoint{12.053712in}{5.967710in}}%
\pgfusepath{clip}%
\pgfsetbuttcap%
\pgfsetroundjoin%
\pgfsetlinewidth{1.505625pt}%
\definecolor{currentstroke}{rgb}{0.890196,0.466667,0.760784}%
\pgfsetstrokecolor{currentstroke}%
\pgfsetdash{}{0pt}%
\pgfpathmoveto{\pgfqpoint{11.463715in}{1.140596in}}%
\pgfpathlineto{\pgfqpoint{11.463715in}{1.140900in}}%
\pgfusepath{stroke}%
\end{pgfscope}%
\begin{pgfscope}%
\pgfpathrectangle{\pgfqpoint{1.286132in}{0.839159in}}{\pgfqpoint{12.053712in}{5.967710in}}%
\pgfusepath{clip}%
\pgfsetbuttcap%
\pgfsetroundjoin%
\pgfsetlinewidth{1.505625pt}%
\definecolor{currentstroke}{rgb}{0.890196,0.466667,0.760784}%
\pgfsetstrokecolor{currentstroke}%
\pgfsetdash{}{0pt}%
\pgfpathmoveto{\pgfqpoint{11.574401in}{1.141015in}}%
\pgfpathlineto{\pgfqpoint{11.574401in}{1.141681in}}%
\pgfusepath{stroke}%
\end{pgfscope}%
\begin{pgfscope}%
\pgfpathrectangle{\pgfqpoint{1.286132in}{0.839159in}}{\pgfqpoint{12.053712in}{5.967710in}}%
\pgfusepath{clip}%
\pgfsetbuttcap%
\pgfsetroundjoin%
\pgfsetlinewidth{1.505625pt}%
\definecolor{currentstroke}{rgb}{0.890196,0.466667,0.760784}%
\pgfsetstrokecolor{currentstroke}%
\pgfsetdash{}{0pt}%
\pgfpathmoveto{\pgfqpoint{11.685087in}{1.141349in}}%
\pgfpathlineto{\pgfqpoint{11.685087in}{1.142156in}}%
\pgfusepath{stroke}%
\end{pgfscope}%
\begin{pgfscope}%
\pgfpathrectangle{\pgfqpoint{1.286132in}{0.839159in}}{\pgfqpoint{12.053712in}{5.967710in}}%
\pgfusepath{clip}%
\pgfsetbuttcap%
\pgfsetroundjoin%
\pgfsetlinewidth{1.505625pt}%
\definecolor{currentstroke}{rgb}{0.890196,0.466667,0.760784}%
\pgfsetstrokecolor{currentstroke}%
\pgfsetdash{}{0pt}%
\pgfpathmoveto{\pgfqpoint{11.795773in}{1.141991in}}%
\pgfpathlineto{\pgfqpoint{11.795773in}{1.142263in}}%
\pgfusepath{stroke}%
\end{pgfscope}%
\begin{pgfscope}%
\pgfpathrectangle{\pgfqpoint{1.286132in}{0.839159in}}{\pgfqpoint{12.053712in}{5.967710in}}%
\pgfusepath{clip}%
\pgfsetbuttcap%
\pgfsetroundjoin%
\pgfsetlinewidth{1.505625pt}%
\definecolor{currentstroke}{rgb}{0.890196,0.466667,0.760784}%
\pgfsetstrokecolor{currentstroke}%
\pgfsetdash{}{0pt}%
\pgfpathmoveto{\pgfqpoint{11.906459in}{1.141543in}}%
\pgfpathlineto{\pgfqpoint{11.906459in}{1.145618in}}%
\pgfusepath{stroke}%
\end{pgfscope}%
\begin{pgfscope}%
\pgfpathrectangle{\pgfqpoint{1.286132in}{0.839159in}}{\pgfqpoint{12.053712in}{5.967710in}}%
\pgfusepath{clip}%
\pgfsetbuttcap%
\pgfsetroundjoin%
\pgfsetlinewidth{1.505625pt}%
\definecolor{currentstroke}{rgb}{0.890196,0.466667,0.760784}%
\pgfsetstrokecolor{currentstroke}%
\pgfsetdash{}{0pt}%
\pgfpathmoveto{\pgfqpoint{12.017145in}{1.143364in}}%
\pgfpathlineto{\pgfqpoint{12.017145in}{1.146607in}}%
\pgfusepath{stroke}%
\end{pgfscope}%
\begin{pgfscope}%
\pgfpathrectangle{\pgfqpoint{1.286132in}{0.839159in}}{\pgfqpoint{12.053712in}{5.967710in}}%
\pgfusepath{clip}%
\pgfsetbuttcap%
\pgfsetroundjoin%
\pgfsetlinewidth{1.505625pt}%
\definecolor{currentstroke}{rgb}{0.890196,0.466667,0.760784}%
\pgfsetstrokecolor{currentstroke}%
\pgfsetdash{}{0pt}%
\pgfpathmoveto{\pgfqpoint{12.127831in}{1.140409in}}%
\pgfpathlineto{\pgfqpoint{12.127831in}{1.161345in}}%
\pgfusepath{stroke}%
\end{pgfscope}%
\begin{pgfscope}%
\pgfpathrectangle{\pgfqpoint{1.286132in}{0.839159in}}{\pgfqpoint{12.053712in}{5.967710in}}%
\pgfusepath{clip}%
\pgfsetbuttcap%
\pgfsetroundjoin%
\pgfsetlinewidth{1.505625pt}%
\definecolor{currentstroke}{rgb}{0.890196,0.466667,0.760784}%
\pgfsetstrokecolor{currentstroke}%
\pgfsetdash{}{0pt}%
\pgfpathmoveto{\pgfqpoint{12.238517in}{1.143413in}}%
\pgfpathlineto{\pgfqpoint{12.238517in}{1.148990in}}%
\pgfusepath{stroke}%
\end{pgfscope}%
\begin{pgfscope}%
\pgfpathrectangle{\pgfqpoint{1.286132in}{0.839159in}}{\pgfqpoint{12.053712in}{5.967710in}}%
\pgfusepath{clip}%
\pgfsetbuttcap%
\pgfsetroundjoin%
\pgfsetlinewidth{1.505625pt}%
\definecolor{currentstroke}{rgb}{0.890196,0.466667,0.760784}%
\pgfsetstrokecolor{currentstroke}%
\pgfsetdash{}{0pt}%
\pgfpathmoveto{\pgfqpoint{12.349203in}{1.144965in}}%
\pgfpathlineto{\pgfqpoint{12.349203in}{1.148395in}}%
\pgfusepath{stroke}%
\end{pgfscope}%
\begin{pgfscope}%
\pgfpathrectangle{\pgfqpoint{1.286132in}{0.839159in}}{\pgfqpoint{12.053712in}{5.967710in}}%
\pgfusepath{clip}%
\pgfsetbuttcap%
\pgfsetroundjoin%
\pgfsetlinewidth{1.505625pt}%
\definecolor{currentstroke}{rgb}{0.890196,0.466667,0.760784}%
\pgfsetstrokecolor{currentstroke}%
\pgfsetdash{}{0pt}%
\pgfpathmoveto{\pgfqpoint{12.459890in}{1.144882in}}%
\pgfpathlineto{\pgfqpoint{12.459890in}{1.151752in}}%
\pgfusepath{stroke}%
\end{pgfscope}%
\begin{pgfscope}%
\pgfpathrectangle{\pgfqpoint{1.286132in}{0.839159in}}{\pgfqpoint{12.053712in}{5.967710in}}%
\pgfusepath{clip}%
\pgfsetbuttcap%
\pgfsetroundjoin%
\pgfsetlinewidth{1.505625pt}%
\definecolor{currentstroke}{rgb}{0.890196,0.466667,0.760784}%
\pgfsetstrokecolor{currentstroke}%
\pgfsetdash{}{0pt}%
\pgfpathmoveto{\pgfqpoint{12.570576in}{1.149998in}}%
\pgfpathlineto{\pgfqpoint{12.570576in}{1.206149in}}%
\pgfusepath{stroke}%
\end{pgfscope}%
\begin{pgfscope}%
\pgfpathrectangle{\pgfqpoint{1.286132in}{0.839159in}}{\pgfqpoint{12.053712in}{5.967710in}}%
\pgfusepath{clip}%
\pgfsetbuttcap%
\pgfsetroundjoin%
\pgfsetlinewidth{1.505625pt}%
\definecolor{currentstroke}{rgb}{0.890196,0.466667,0.760784}%
\pgfsetstrokecolor{currentstroke}%
\pgfsetdash{}{0pt}%
\pgfpathmoveto{\pgfqpoint{12.681262in}{1.146548in}}%
\pgfpathlineto{\pgfqpoint{12.681262in}{1.147302in}}%
\pgfusepath{stroke}%
\end{pgfscope}%
\begin{pgfscope}%
\pgfpathrectangle{\pgfqpoint{1.286132in}{0.839159in}}{\pgfqpoint{12.053712in}{5.967710in}}%
\pgfusepath{clip}%
\pgfsetbuttcap%
\pgfsetroundjoin%
\pgfsetlinewidth{1.505625pt}%
\definecolor{currentstroke}{rgb}{0.890196,0.466667,0.760784}%
\pgfsetstrokecolor{currentstroke}%
\pgfsetdash{}{0pt}%
\pgfpathmoveto{\pgfqpoint{12.791948in}{1.146935in}}%
\pgfpathlineto{\pgfqpoint{12.791948in}{1.147864in}}%
\pgfusepath{stroke}%
\end{pgfscope}%
\begin{pgfscope}%
\pgfpathrectangle{\pgfqpoint{1.286132in}{0.839159in}}{\pgfqpoint{12.053712in}{5.967710in}}%
\pgfusepath{clip}%
\pgfsetbuttcap%
\pgfsetroundjoin%
\pgfsetlinewidth{1.505625pt}%
\definecolor{currentstroke}{rgb}{0.498039,0.498039,0.498039}%
\pgfsetstrokecolor{currentstroke}%
\pgfsetdash{}{0pt}%
\pgfpathmoveto{\pgfqpoint{1.834028in}{1.119256in}}%
\pgfpathlineto{\pgfqpoint{1.834028in}{1.119274in}}%
\pgfusepath{stroke}%
\end{pgfscope}%
\begin{pgfscope}%
\pgfpathrectangle{\pgfqpoint{1.286132in}{0.839159in}}{\pgfqpoint{12.053712in}{5.967710in}}%
\pgfusepath{clip}%
\pgfsetbuttcap%
\pgfsetroundjoin%
\pgfsetlinewidth{1.505625pt}%
\definecolor{currentstroke}{rgb}{0.498039,0.498039,0.498039}%
\pgfsetstrokecolor{currentstroke}%
\pgfsetdash{}{0pt}%
\pgfpathmoveto{\pgfqpoint{1.944714in}{1.119592in}}%
\pgfpathlineto{\pgfqpoint{1.944714in}{1.119670in}}%
\pgfusepath{stroke}%
\end{pgfscope}%
\begin{pgfscope}%
\pgfpathrectangle{\pgfqpoint{1.286132in}{0.839159in}}{\pgfqpoint{12.053712in}{5.967710in}}%
\pgfusepath{clip}%
\pgfsetbuttcap%
\pgfsetroundjoin%
\pgfsetlinewidth{1.505625pt}%
\definecolor{currentstroke}{rgb}{0.498039,0.498039,0.498039}%
\pgfsetstrokecolor{currentstroke}%
\pgfsetdash{}{0pt}%
\pgfpathmoveto{\pgfqpoint{2.055400in}{1.119915in}}%
\pgfpathlineto{\pgfqpoint{2.055400in}{1.119953in}}%
\pgfusepath{stroke}%
\end{pgfscope}%
\begin{pgfscope}%
\pgfpathrectangle{\pgfqpoint{1.286132in}{0.839159in}}{\pgfqpoint{12.053712in}{5.967710in}}%
\pgfusepath{clip}%
\pgfsetbuttcap%
\pgfsetroundjoin%
\pgfsetlinewidth{1.505625pt}%
\definecolor{currentstroke}{rgb}{0.498039,0.498039,0.498039}%
\pgfsetstrokecolor{currentstroke}%
\pgfsetdash{}{0pt}%
\pgfpathmoveto{\pgfqpoint{2.166086in}{1.120290in}}%
\pgfpathlineto{\pgfqpoint{2.166086in}{1.120563in}}%
\pgfusepath{stroke}%
\end{pgfscope}%
\begin{pgfscope}%
\pgfpathrectangle{\pgfqpoint{1.286132in}{0.839159in}}{\pgfqpoint{12.053712in}{5.967710in}}%
\pgfusepath{clip}%
\pgfsetbuttcap%
\pgfsetroundjoin%
\pgfsetlinewidth{1.505625pt}%
\definecolor{currentstroke}{rgb}{0.498039,0.498039,0.498039}%
\pgfsetstrokecolor{currentstroke}%
\pgfsetdash{}{0pt}%
\pgfpathmoveto{\pgfqpoint{2.276772in}{1.120810in}}%
\pgfpathlineto{\pgfqpoint{2.276772in}{1.121060in}}%
\pgfusepath{stroke}%
\end{pgfscope}%
\begin{pgfscope}%
\pgfpathrectangle{\pgfqpoint{1.286132in}{0.839159in}}{\pgfqpoint{12.053712in}{5.967710in}}%
\pgfusepath{clip}%
\pgfsetbuttcap%
\pgfsetroundjoin%
\pgfsetlinewidth{1.505625pt}%
\definecolor{currentstroke}{rgb}{0.498039,0.498039,0.498039}%
\pgfsetstrokecolor{currentstroke}%
\pgfsetdash{}{0pt}%
\pgfpathmoveto{\pgfqpoint{2.387458in}{1.119732in}}%
\pgfpathlineto{\pgfqpoint{2.387458in}{1.127230in}}%
\pgfusepath{stroke}%
\end{pgfscope}%
\begin{pgfscope}%
\pgfpathrectangle{\pgfqpoint{1.286132in}{0.839159in}}{\pgfqpoint{12.053712in}{5.967710in}}%
\pgfusepath{clip}%
\pgfsetbuttcap%
\pgfsetroundjoin%
\pgfsetlinewidth{1.505625pt}%
\definecolor{currentstroke}{rgb}{0.498039,0.498039,0.498039}%
\pgfsetstrokecolor{currentstroke}%
\pgfsetdash{}{0pt}%
\pgfpathmoveto{\pgfqpoint{2.498144in}{1.122034in}}%
\pgfpathlineto{\pgfqpoint{2.498144in}{1.123642in}}%
\pgfusepath{stroke}%
\end{pgfscope}%
\begin{pgfscope}%
\pgfpathrectangle{\pgfqpoint{1.286132in}{0.839159in}}{\pgfqpoint{12.053712in}{5.967710in}}%
\pgfusepath{clip}%
\pgfsetbuttcap%
\pgfsetroundjoin%
\pgfsetlinewidth{1.505625pt}%
\definecolor{currentstroke}{rgb}{0.498039,0.498039,0.498039}%
\pgfsetstrokecolor{currentstroke}%
\pgfsetdash{}{0pt}%
\pgfpathmoveto{\pgfqpoint{2.608830in}{1.123117in}}%
\pgfpathlineto{\pgfqpoint{2.608830in}{1.123700in}}%
\pgfusepath{stroke}%
\end{pgfscope}%
\begin{pgfscope}%
\pgfpathrectangle{\pgfqpoint{1.286132in}{0.839159in}}{\pgfqpoint{12.053712in}{5.967710in}}%
\pgfusepath{clip}%
\pgfsetbuttcap%
\pgfsetroundjoin%
\pgfsetlinewidth{1.505625pt}%
\definecolor{currentstroke}{rgb}{0.498039,0.498039,0.498039}%
\pgfsetstrokecolor{currentstroke}%
\pgfsetdash{}{0pt}%
\pgfpathmoveto{\pgfqpoint{2.719516in}{1.124351in}}%
\pgfpathlineto{\pgfqpoint{2.719516in}{1.125033in}}%
\pgfusepath{stroke}%
\end{pgfscope}%
\begin{pgfscope}%
\pgfpathrectangle{\pgfqpoint{1.286132in}{0.839159in}}{\pgfqpoint{12.053712in}{5.967710in}}%
\pgfusepath{clip}%
\pgfsetbuttcap%
\pgfsetroundjoin%
\pgfsetlinewidth{1.505625pt}%
\definecolor{currentstroke}{rgb}{0.498039,0.498039,0.498039}%
\pgfsetstrokecolor{currentstroke}%
\pgfsetdash{}{0pt}%
\pgfpathmoveto{\pgfqpoint{2.830202in}{1.125579in}}%
\pgfpathlineto{\pgfqpoint{2.830202in}{1.127888in}}%
\pgfusepath{stroke}%
\end{pgfscope}%
\begin{pgfscope}%
\pgfpathrectangle{\pgfqpoint{1.286132in}{0.839159in}}{\pgfqpoint{12.053712in}{5.967710in}}%
\pgfusepath{clip}%
\pgfsetbuttcap%
\pgfsetroundjoin%
\pgfsetlinewidth{1.505625pt}%
\definecolor{currentstroke}{rgb}{0.498039,0.498039,0.498039}%
\pgfsetstrokecolor{currentstroke}%
\pgfsetdash{}{0pt}%
\pgfpathmoveto{\pgfqpoint{2.940888in}{1.127033in}}%
\pgfpathlineto{\pgfqpoint{2.940888in}{1.128877in}}%
\pgfusepath{stroke}%
\end{pgfscope}%
\begin{pgfscope}%
\pgfpathrectangle{\pgfqpoint{1.286132in}{0.839159in}}{\pgfqpoint{12.053712in}{5.967710in}}%
\pgfusepath{clip}%
\pgfsetbuttcap%
\pgfsetroundjoin%
\pgfsetlinewidth{1.505625pt}%
\definecolor{currentstroke}{rgb}{0.498039,0.498039,0.498039}%
\pgfsetstrokecolor{currentstroke}%
\pgfsetdash{}{0pt}%
\pgfpathmoveto{\pgfqpoint{3.051574in}{1.129239in}}%
\pgfpathlineto{\pgfqpoint{3.051574in}{1.129583in}}%
\pgfusepath{stroke}%
\end{pgfscope}%
\begin{pgfscope}%
\pgfpathrectangle{\pgfqpoint{1.286132in}{0.839159in}}{\pgfqpoint{12.053712in}{5.967710in}}%
\pgfusepath{clip}%
\pgfsetbuttcap%
\pgfsetroundjoin%
\pgfsetlinewidth{1.505625pt}%
\definecolor{currentstroke}{rgb}{0.498039,0.498039,0.498039}%
\pgfsetstrokecolor{currentstroke}%
\pgfsetdash{}{0pt}%
\pgfpathmoveto{\pgfqpoint{3.162260in}{1.131618in}}%
\pgfpathlineto{\pgfqpoint{3.162260in}{1.132167in}}%
\pgfusepath{stroke}%
\end{pgfscope}%
\begin{pgfscope}%
\pgfpathrectangle{\pgfqpoint{1.286132in}{0.839159in}}{\pgfqpoint{12.053712in}{5.967710in}}%
\pgfusepath{clip}%
\pgfsetbuttcap%
\pgfsetroundjoin%
\pgfsetlinewidth{1.505625pt}%
\definecolor{currentstroke}{rgb}{0.498039,0.498039,0.498039}%
\pgfsetstrokecolor{currentstroke}%
\pgfsetdash{}{0pt}%
\pgfpathmoveto{\pgfqpoint{3.272946in}{1.133587in}}%
\pgfpathlineto{\pgfqpoint{3.272946in}{1.133983in}}%
\pgfusepath{stroke}%
\end{pgfscope}%
\begin{pgfscope}%
\pgfpathrectangle{\pgfqpoint{1.286132in}{0.839159in}}{\pgfqpoint{12.053712in}{5.967710in}}%
\pgfusepath{clip}%
\pgfsetbuttcap%
\pgfsetroundjoin%
\pgfsetlinewidth{1.505625pt}%
\definecolor{currentstroke}{rgb}{0.498039,0.498039,0.498039}%
\pgfsetstrokecolor{currentstroke}%
\pgfsetdash{}{0pt}%
\pgfpathmoveto{\pgfqpoint{3.383632in}{1.135951in}}%
\pgfpathlineto{\pgfqpoint{3.383632in}{1.138858in}}%
\pgfusepath{stroke}%
\end{pgfscope}%
\begin{pgfscope}%
\pgfpathrectangle{\pgfqpoint{1.286132in}{0.839159in}}{\pgfqpoint{12.053712in}{5.967710in}}%
\pgfusepath{clip}%
\pgfsetbuttcap%
\pgfsetroundjoin%
\pgfsetlinewidth{1.505625pt}%
\definecolor{currentstroke}{rgb}{0.498039,0.498039,0.498039}%
\pgfsetstrokecolor{currentstroke}%
\pgfsetdash{}{0pt}%
\pgfpathmoveto{\pgfqpoint{3.494319in}{1.139010in}}%
\pgfpathlineto{\pgfqpoint{3.494319in}{1.142289in}}%
\pgfusepath{stroke}%
\end{pgfscope}%
\begin{pgfscope}%
\pgfpathrectangle{\pgfqpoint{1.286132in}{0.839159in}}{\pgfqpoint{12.053712in}{5.967710in}}%
\pgfusepath{clip}%
\pgfsetbuttcap%
\pgfsetroundjoin%
\pgfsetlinewidth{1.505625pt}%
\definecolor{currentstroke}{rgb}{0.498039,0.498039,0.498039}%
\pgfsetstrokecolor{currentstroke}%
\pgfsetdash{}{0pt}%
\pgfpathmoveto{\pgfqpoint{3.605005in}{1.142144in}}%
\pgfpathlineto{\pgfqpoint{3.605005in}{1.149238in}}%
\pgfusepath{stroke}%
\end{pgfscope}%
\begin{pgfscope}%
\pgfpathrectangle{\pgfqpoint{1.286132in}{0.839159in}}{\pgfqpoint{12.053712in}{5.967710in}}%
\pgfusepath{clip}%
\pgfsetbuttcap%
\pgfsetroundjoin%
\pgfsetlinewidth{1.505625pt}%
\definecolor{currentstroke}{rgb}{0.498039,0.498039,0.498039}%
\pgfsetstrokecolor{currentstroke}%
\pgfsetdash{}{0pt}%
\pgfpathmoveto{\pgfqpoint{3.715691in}{1.146703in}}%
\pgfpathlineto{\pgfqpoint{3.715691in}{1.149915in}}%
\pgfusepath{stroke}%
\end{pgfscope}%
\begin{pgfscope}%
\pgfpathrectangle{\pgfqpoint{1.286132in}{0.839159in}}{\pgfqpoint{12.053712in}{5.967710in}}%
\pgfusepath{clip}%
\pgfsetbuttcap%
\pgfsetroundjoin%
\pgfsetlinewidth{1.505625pt}%
\definecolor{currentstroke}{rgb}{0.498039,0.498039,0.498039}%
\pgfsetstrokecolor{currentstroke}%
\pgfsetdash{}{0pt}%
\pgfpathmoveto{\pgfqpoint{3.826377in}{1.150483in}}%
\pgfpathlineto{\pgfqpoint{3.826377in}{1.154284in}}%
\pgfusepath{stroke}%
\end{pgfscope}%
\begin{pgfscope}%
\pgfpathrectangle{\pgfqpoint{1.286132in}{0.839159in}}{\pgfqpoint{12.053712in}{5.967710in}}%
\pgfusepath{clip}%
\pgfsetbuttcap%
\pgfsetroundjoin%
\pgfsetlinewidth{1.505625pt}%
\definecolor{currentstroke}{rgb}{0.498039,0.498039,0.498039}%
\pgfsetstrokecolor{currentstroke}%
\pgfsetdash{}{0pt}%
\pgfpathmoveto{\pgfqpoint{3.937063in}{1.156808in}}%
\pgfpathlineto{\pgfqpoint{3.937063in}{1.161808in}}%
\pgfusepath{stroke}%
\end{pgfscope}%
\begin{pgfscope}%
\pgfpathrectangle{\pgfqpoint{1.286132in}{0.839159in}}{\pgfqpoint{12.053712in}{5.967710in}}%
\pgfusepath{clip}%
\pgfsetbuttcap%
\pgfsetroundjoin%
\pgfsetlinewidth{1.505625pt}%
\definecolor{currentstroke}{rgb}{0.498039,0.498039,0.498039}%
\pgfsetstrokecolor{currentstroke}%
\pgfsetdash{}{0pt}%
\pgfpathmoveto{\pgfqpoint{4.047749in}{1.160662in}}%
\pgfpathlineto{\pgfqpoint{4.047749in}{1.162219in}}%
\pgfusepath{stroke}%
\end{pgfscope}%
\begin{pgfscope}%
\pgfpathrectangle{\pgfqpoint{1.286132in}{0.839159in}}{\pgfqpoint{12.053712in}{5.967710in}}%
\pgfusepath{clip}%
\pgfsetbuttcap%
\pgfsetroundjoin%
\pgfsetlinewidth{1.505625pt}%
\definecolor{currentstroke}{rgb}{0.498039,0.498039,0.498039}%
\pgfsetstrokecolor{currentstroke}%
\pgfsetdash{}{0pt}%
\pgfpathmoveto{\pgfqpoint{4.158435in}{1.167315in}}%
\pgfpathlineto{\pgfqpoint{4.158435in}{1.171074in}}%
\pgfusepath{stroke}%
\end{pgfscope}%
\begin{pgfscope}%
\pgfpathrectangle{\pgfqpoint{1.286132in}{0.839159in}}{\pgfqpoint{12.053712in}{5.967710in}}%
\pgfusepath{clip}%
\pgfsetbuttcap%
\pgfsetroundjoin%
\pgfsetlinewidth{1.505625pt}%
\definecolor{currentstroke}{rgb}{0.498039,0.498039,0.498039}%
\pgfsetstrokecolor{currentstroke}%
\pgfsetdash{}{0pt}%
\pgfpathmoveto{\pgfqpoint{4.269121in}{1.172418in}}%
\pgfpathlineto{\pgfqpoint{4.269121in}{1.179106in}}%
\pgfusepath{stroke}%
\end{pgfscope}%
\begin{pgfscope}%
\pgfpathrectangle{\pgfqpoint{1.286132in}{0.839159in}}{\pgfqpoint{12.053712in}{5.967710in}}%
\pgfusepath{clip}%
\pgfsetbuttcap%
\pgfsetroundjoin%
\pgfsetlinewidth{1.505625pt}%
\definecolor{currentstroke}{rgb}{0.498039,0.498039,0.498039}%
\pgfsetstrokecolor{currentstroke}%
\pgfsetdash{}{0pt}%
\pgfpathmoveto{\pgfqpoint{4.379807in}{1.180240in}}%
\pgfpathlineto{\pgfqpoint{4.379807in}{1.191494in}}%
\pgfusepath{stroke}%
\end{pgfscope}%
\begin{pgfscope}%
\pgfpathrectangle{\pgfqpoint{1.286132in}{0.839159in}}{\pgfqpoint{12.053712in}{5.967710in}}%
\pgfusepath{clip}%
\pgfsetbuttcap%
\pgfsetroundjoin%
\pgfsetlinewidth{1.505625pt}%
\definecolor{currentstroke}{rgb}{0.498039,0.498039,0.498039}%
\pgfsetstrokecolor{currentstroke}%
\pgfsetdash{}{0pt}%
\pgfpathmoveto{\pgfqpoint{4.490493in}{1.178475in}}%
\pgfpathlineto{\pgfqpoint{4.490493in}{1.208877in}}%
\pgfusepath{stroke}%
\end{pgfscope}%
\begin{pgfscope}%
\pgfpathrectangle{\pgfqpoint{1.286132in}{0.839159in}}{\pgfqpoint{12.053712in}{5.967710in}}%
\pgfusepath{clip}%
\pgfsetbuttcap%
\pgfsetroundjoin%
\pgfsetlinewidth{1.505625pt}%
\definecolor{currentstroke}{rgb}{0.498039,0.498039,0.498039}%
\pgfsetstrokecolor{currentstroke}%
\pgfsetdash{}{0pt}%
\pgfpathmoveto{\pgfqpoint{4.601179in}{1.193721in}}%
\pgfpathlineto{\pgfqpoint{4.601179in}{1.196714in}}%
\pgfusepath{stroke}%
\end{pgfscope}%
\begin{pgfscope}%
\pgfpathrectangle{\pgfqpoint{1.286132in}{0.839159in}}{\pgfqpoint{12.053712in}{5.967710in}}%
\pgfusepath{clip}%
\pgfsetbuttcap%
\pgfsetroundjoin%
\pgfsetlinewidth{1.505625pt}%
\definecolor{currentstroke}{rgb}{0.498039,0.498039,0.498039}%
\pgfsetstrokecolor{currentstroke}%
\pgfsetdash{}{0pt}%
\pgfpathmoveto{\pgfqpoint{4.711865in}{1.201475in}}%
\pgfpathlineto{\pgfqpoint{4.711865in}{1.208158in}}%
\pgfusepath{stroke}%
\end{pgfscope}%
\begin{pgfscope}%
\pgfpathrectangle{\pgfqpoint{1.286132in}{0.839159in}}{\pgfqpoint{12.053712in}{5.967710in}}%
\pgfusepath{clip}%
\pgfsetbuttcap%
\pgfsetroundjoin%
\pgfsetlinewidth{1.505625pt}%
\definecolor{currentstroke}{rgb}{0.498039,0.498039,0.498039}%
\pgfsetstrokecolor{currentstroke}%
\pgfsetdash{}{0pt}%
\pgfpathmoveto{\pgfqpoint{4.822551in}{1.214114in}}%
\pgfpathlineto{\pgfqpoint{4.822551in}{1.220352in}}%
\pgfusepath{stroke}%
\end{pgfscope}%
\begin{pgfscope}%
\pgfpathrectangle{\pgfqpoint{1.286132in}{0.839159in}}{\pgfqpoint{12.053712in}{5.967710in}}%
\pgfusepath{clip}%
\pgfsetbuttcap%
\pgfsetroundjoin%
\pgfsetlinewidth{1.505625pt}%
\definecolor{currentstroke}{rgb}{0.498039,0.498039,0.498039}%
\pgfsetstrokecolor{currentstroke}%
\pgfsetdash{}{0pt}%
\pgfpathmoveto{\pgfqpoint{4.933237in}{1.211206in}}%
\pgfpathlineto{\pgfqpoint{4.933237in}{1.263854in}}%
\pgfusepath{stroke}%
\end{pgfscope}%
\begin{pgfscope}%
\pgfpathrectangle{\pgfqpoint{1.286132in}{0.839159in}}{\pgfqpoint{12.053712in}{5.967710in}}%
\pgfusepath{clip}%
\pgfsetbuttcap%
\pgfsetroundjoin%
\pgfsetlinewidth{1.505625pt}%
\definecolor{currentstroke}{rgb}{0.498039,0.498039,0.498039}%
\pgfsetstrokecolor{currentstroke}%
\pgfsetdash{}{0pt}%
\pgfpathmoveto{\pgfqpoint{5.043923in}{1.229629in}}%
\pgfpathlineto{\pgfqpoint{5.043923in}{1.233147in}}%
\pgfusepath{stroke}%
\end{pgfscope}%
\begin{pgfscope}%
\pgfpathrectangle{\pgfqpoint{1.286132in}{0.839159in}}{\pgfqpoint{12.053712in}{5.967710in}}%
\pgfusepath{clip}%
\pgfsetbuttcap%
\pgfsetroundjoin%
\pgfsetlinewidth{1.505625pt}%
\definecolor{currentstroke}{rgb}{0.498039,0.498039,0.498039}%
\pgfsetstrokecolor{currentstroke}%
\pgfsetdash{}{0pt}%
\pgfpathmoveto{\pgfqpoint{5.154609in}{1.240809in}}%
\pgfpathlineto{\pgfqpoint{5.154609in}{1.243373in}}%
\pgfusepath{stroke}%
\end{pgfscope}%
\begin{pgfscope}%
\pgfpathrectangle{\pgfqpoint{1.286132in}{0.839159in}}{\pgfqpoint{12.053712in}{5.967710in}}%
\pgfusepath{clip}%
\pgfsetbuttcap%
\pgfsetroundjoin%
\pgfsetlinewidth{1.505625pt}%
\definecolor{currentstroke}{rgb}{0.498039,0.498039,0.498039}%
\pgfsetstrokecolor{currentstroke}%
\pgfsetdash{}{0pt}%
\pgfpathmoveto{\pgfqpoint{5.265296in}{1.254802in}}%
\pgfpathlineto{\pgfqpoint{5.265296in}{1.260754in}}%
\pgfusepath{stroke}%
\end{pgfscope}%
\begin{pgfscope}%
\pgfpathrectangle{\pgfqpoint{1.286132in}{0.839159in}}{\pgfqpoint{12.053712in}{5.967710in}}%
\pgfusepath{clip}%
\pgfsetbuttcap%
\pgfsetroundjoin%
\pgfsetlinewidth{1.505625pt}%
\definecolor{currentstroke}{rgb}{0.498039,0.498039,0.498039}%
\pgfsetstrokecolor{currentstroke}%
\pgfsetdash{}{0pt}%
\pgfpathmoveto{\pgfqpoint{5.375982in}{1.264442in}}%
\pgfpathlineto{\pgfqpoint{5.375982in}{1.277019in}}%
\pgfusepath{stroke}%
\end{pgfscope}%
\begin{pgfscope}%
\pgfpathrectangle{\pgfqpoint{1.286132in}{0.839159in}}{\pgfqpoint{12.053712in}{5.967710in}}%
\pgfusepath{clip}%
\pgfsetbuttcap%
\pgfsetroundjoin%
\pgfsetlinewidth{1.505625pt}%
\definecolor{currentstroke}{rgb}{0.498039,0.498039,0.498039}%
\pgfsetstrokecolor{currentstroke}%
\pgfsetdash{}{0pt}%
\pgfpathmoveto{\pgfqpoint{5.486668in}{1.278184in}}%
\pgfpathlineto{\pgfqpoint{5.486668in}{1.287076in}}%
\pgfusepath{stroke}%
\end{pgfscope}%
\begin{pgfscope}%
\pgfpathrectangle{\pgfqpoint{1.286132in}{0.839159in}}{\pgfqpoint{12.053712in}{5.967710in}}%
\pgfusepath{clip}%
\pgfsetbuttcap%
\pgfsetroundjoin%
\pgfsetlinewidth{1.505625pt}%
\definecolor{currentstroke}{rgb}{0.498039,0.498039,0.498039}%
\pgfsetstrokecolor{currentstroke}%
\pgfsetdash{}{0pt}%
\pgfpathmoveto{\pgfqpoint{5.597354in}{1.292182in}}%
\pgfpathlineto{\pgfqpoint{5.597354in}{1.299622in}}%
\pgfusepath{stroke}%
\end{pgfscope}%
\begin{pgfscope}%
\pgfpathrectangle{\pgfqpoint{1.286132in}{0.839159in}}{\pgfqpoint{12.053712in}{5.967710in}}%
\pgfusepath{clip}%
\pgfsetbuttcap%
\pgfsetroundjoin%
\pgfsetlinewidth{1.505625pt}%
\definecolor{currentstroke}{rgb}{0.498039,0.498039,0.498039}%
\pgfsetstrokecolor{currentstroke}%
\pgfsetdash{}{0pt}%
\pgfpathmoveto{\pgfqpoint{5.708040in}{1.391359in}}%
\pgfpathlineto{\pgfqpoint{5.708040in}{1.427533in}}%
\pgfusepath{stroke}%
\end{pgfscope}%
\begin{pgfscope}%
\pgfpathrectangle{\pgfqpoint{1.286132in}{0.839159in}}{\pgfqpoint{12.053712in}{5.967710in}}%
\pgfusepath{clip}%
\pgfsetbuttcap%
\pgfsetroundjoin%
\pgfsetlinewidth{1.505625pt}%
\definecolor{currentstroke}{rgb}{0.498039,0.498039,0.498039}%
\pgfsetstrokecolor{currentstroke}%
\pgfsetdash{}{0pt}%
\pgfpathmoveto{\pgfqpoint{5.818726in}{1.432965in}}%
\pgfpathlineto{\pgfqpoint{5.818726in}{1.464875in}}%
\pgfusepath{stroke}%
\end{pgfscope}%
\begin{pgfscope}%
\pgfpathrectangle{\pgfqpoint{1.286132in}{0.839159in}}{\pgfqpoint{12.053712in}{5.967710in}}%
\pgfusepath{clip}%
\pgfsetbuttcap%
\pgfsetroundjoin%
\pgfsetlinewidth{1.505625pt}%
\definecolor{currentstroke}{rgb}{0.498039,0.498039,0.498039}%
\pgfsetstrokecolor{currentstroke}%
\pgfsetdash{}{0pt}%
\pgfpathmoveto{\pgfqpoint{5.929412in}{1.373481in}}%
\pgfpathlineto{\pgfqpoint{5.929412in}{1.677804in}}%
\pgfusepath{stroke}%
\end{pgfscope}%
\begin{pgfscope}%
\pgfpathrectangle{\pgfqpoint{1.286132in}{0.839159in}}{\pgfqpoint{12.053712in}{5.967710in}}%
\pgfusepath{clip}%
\pgfsetbuttcap%
\pgfsetroundjoin%
\pgfsetlinewidth{1.505625pt}%
\definecolor{currentstroke}{rgb}{0.498039,0.498039,0.498039}%
\pgfsetstrokecolor{currentstroke}%
\pgfsetdash{}{0pt}%
\pgfpathmoveto{\pgfqpoint{6.040098in}{1.362905in}}%
\pgfpathlineto{\pgfqpoint{6.040098in}{1.371299in}}%
\pgfusepath{stroke}%
\end{pgfscope}%
\begin{pgfscope}%
\pgfpathrectangle{\pgfqpoint{1.286132in}{0.839159in}}{\pgfqpoint{12.053712in}{5.967710in}}%
\pgfusepath{clip}%
\pgfsetbuttcap%
\pgfsetroundjoin%
\pgfsetlinewidth{1.505625pt}%
\definecolor{currentstroke}{rgb}{0.498039,0.498039,0.498039}%
\pgfsetstrokecolor{currentstroke}%
\pgfsetdash{}{0pt}%
\pgfpathmoveto{\pgfqpoint{6.150784in}{1.376645in}}%
\pgfpathlineto{\pgfqpoint{6.150784in}{1.388269in}}%
\pgfusepath{stroke}%
\end{pgfscope}%
\begin{pgfscope}%
\pgfpathrectangle{\pgfqpoint{1.286132in}{0.839159in}}{\pgfqpoint{12.053712in}{5.967710in}}%
\pgfusepath{clip}%
\pgfsetbuttcap%
\pgfsetroundjoin%
\pgfsetlinewidth{1.505625pt}%
\definecolor{currentstroke}{rgb}{0.498039,0.498039,0.498039}%
\pgfsetstrokecolor{currentstroke}%
\pgfsetdash{}{0pt}%
\pgfpathmoveto{\pgfqpoint{6.261470in}{1.392225in}}%
\pgfpathlineto{\pgfqpoint{6.261470in}{1.444100in}}%
\pgfusepath{stroke}%
\end{pgfscope}%
\begin{pgfscope}%
\pgfpathrectangle{\pgfqpoint{1.286132in}{0.839159in}}{\pgfqpoint{12.053712in}{5.967710in}}%
\pgfusepath{clip}%
\pgfsetbuttcap%
\pgfsetroundjoin%
\pgfsetlinewidth{1.505625pt}%
\definecolor{currentstroke}{rgb}{0.498039,0.498039,0.498039}%
\pgfsetstrokecolor{currentstroke}%
\pgfsetdash{}{0pt}%
\pgfpathmoveto{\pgfqpoint{6.372156in}{1.414354in}}%
\pgfpathlineto{\pgfqpoint{6.372156in}{1.434951in}}%
\pgfusepath{stroke}%
\end{pgfscope}%
\begin{pgfscope}%
\pgfpathrectangle{\pgfqpoint{1.286132in}{0.839159in}}{\pgfqpoint{12.053712in}{5.967710in}}%
\pgfusepath{clip}%
\pgfsetbuttcap%
\pgfsetroundjoin%
\pgfsetlinewidth{1.505625pt}%
\definecolor{currentstroke}{rgb}{0.498039,0.498039,0.498039}%
\pgfsetstrokecolor{currentstroke}%
\pgfsetdash{}{0pt}%
\pgfpathmoveto{\pgfqpoint{6.482842in}{1.414618in}}%
\pgfpathlineto{\pgfqpoint{6.482842in}{1.532661in}}%
\pgfusepath{stroke}%
\end{pgfscope}%
\begin{pgfscope}%
\pgfpathrectangle{\pgfqpoint{1.286132in}{0.839159in}}{\pgfqpoint{12.053712in}{5.967710in}}%
\pgfusepath{clip}%
\pgfsetbuttcap%
\pgfsetroundjoin%
\pgfsetlinewidth{1.505625pt}%
\definecolor{currentstroke}{rgb}{0.498039,0.498039,0.498039}%
\pgfsetstrokecolor{currentstroke}%
\pgfsetdash{}{0pt}%
\pgfpathmoveto{\pgfqpoint{6.593528in}{1.463828in}}%
\pgfpathlineto{\pgfqpoint{6.593528in}{1.537541in}}%
\pgfusepath{stroke}%
\end{pgfscope}%
\begin{pgfscope}%
\pgfpathrectangle{\pgfqpoint{1.286132in}{0.839159in}}{\pgfqpoint{12.053712in}{5.967710in}}%
\pgfusepath{clip}%
\pgfsetbuttcap%
\pgfsetroundjoin%
\pgfsetlinewidth{1.505625pt}%
\definecolor{currentstroke}{rgb}{0.498039,0.498039,0.498039}%
\pgfsetstrokecolor{currentstroke}%
\pgfsetdash{}{0pt}%
\pgfpathmoveto{\pgfqpoint{6.704214in}{1.483251in}}%
\pgfpathlineto{\pgfqpoint{6.704214in}{1.519671in}}%
\pgfusepath{stroke}%
\end{pgfscope}%
\begin{pgfscope}%
\pgfpathrectangle{\pgfqpoint{1.286132in}{0.839159in}}{\pgfqpoint{12.053712in}{5.967710in}}%
\pgfusepath{clip}%
\pgfsetbuttcap%
\pgfsetroundjoin%
\pgfsetlinewidth{1.505625pt}%
\definecolor{currentstroke}{rgb}{0.498039,0.498039,0.498039}%
\pgfsetstrokecolor{currentstroke}%
\pgfsetdash{}{0pt}%
\pgfpathmoveto{\pgfqpoint{6.814900in}{1.503225in}}%
\pgfpathlineto{\pgfqpoint{6.814900in}{1.508786in}}%
\pgfusepath{stroke}%
\end{pgfscope}%
\begin{pgfscope}%
\pgfpathrectangle{\pgfqpoint{1.286132in}{0.839159in}}{\pgfqpoint{12.053712in}{5.967710in}}%
\pgfusepath{clip}%
\pgfsetbuttcap%
\pgfsetroundjoin%
\pgfsetlinewidth{1.505625pt}%
\definecolor{currentstroke}{rgb}{0.498039,0.498039,0.498039}%
\pgfsetstrokecolor{currentstroke}%
\pgfsetdash{}{0pt}%
\pgfpathmoveto{\pgfqpoint{6.925586in}{1.535058in}}%
\pgfpathlineto{\pgfqpoint{6.925586in}{1.551205in}}%
\pgfusepath{stroke}%
\end{pgfscope}%
\begin{pgfscope}%
\pgfpathrectangle{\pgfqpoint{1.286132in}{0.839159in}}{\pgfqpoint{12.053712in}{5.967710in}}%
\pgfusepath{clip}%
\pgfsetbuttcap%
\pgfsetroundjoin%
\pgfsetlinewidth{1.505625pt}%
\definecolor{currentstroke}{rgb}{0.498039,0.498039,0.498039}%
\pgfsetstrokecolor{currentstroke}%
\pgfsetdash{}{0pt}%
\pgfpathmoveto{\pgfqpoint{7.036272in}{1.547150in}}%
\pgfpathlineto{\pgfqpoint{7.036272in}{1.583728in}}%
\pgfusepath{stroke}%
\end{pgfscope}%
\begin{pgfscope}%
\pgfpathrectangle{\pgfqpoint{1.286132in}{0.839159in}}{\pgfqpoint{12.053712in}{5.967710in}}%
\pgfusepath{clip}%
\pgfsetbuttcap%
\pgfsetroundjoin%
\pgfsetlinewidth{1.505625pt}%
\definecolor{currentstroke}{rgb}{0.498039,0.498039,0.498039}%
\pgfsetstrokecolor{currentstroke}%
\pgfsetdash{}{0pt}%
\pgfpathmoveto{\pgfqpoint{7.146959in}{1.579024in}}%
\pgfpathlineto{\pgfqpoint{7.146959in}{1.589319in}}%
\pgfusepath{stroke}%
\end{pgfscope}%
\begin{pgfscope}%
\pgfpathrectangle{\pgfqpoint{1.286132in}{0.839159in}}{\pgfqpoint{12.053712in}{5.967710in}}%
\pgfusepath{clip}%
\pgfsetbuttcap%
\pgfsetroundjoin%
\pgfsetlinewidth{1.505625pt}%
\definecolor{currentstroke}{rgb}{0.498039,0.498039,0.498039}%
\pgfsetstrokecolor{currentstroke}%
\pgfsetdash{}{0pt}%
\pgfpathmoveto{\pgfqpoint{7.257645in}{1.586510in}}%
\pgfpathlineto{\pgfqpoint{7.257645in}{1.658832in}}%
\pgfusepath{stroke}%
\end{pgfscope}%
\begin{pgfscope}%
\pgfpathrectangle{\pgfqpoint{1.286132in}{0.839159in}}{\pgfqpoint{12.053712in}{5.967710in}}%
\pgfusepath{clip}%
\pgfsetbuttcap%
\pgfsetroundjoin%
\pgfsetlinewidth{1.505625pt}%
\definecolor{currentstroke}{rgb}{0.498039,0.498039,0.498039}%
\pgfsetstrokecolor{currentstroke}%
\pgfsetdash{}{0pt}%
\pgfpathmoveto{\pgfqpoint{7.368331in}{1.646593in}}%
\pgfpathlineto{\pgfqpoint{7.368331in}{1.655299in}}%
\pgfusepath{stroke}%
\end{pgfscope}%
\begin{pgfscope}%
\pgfpathrectangle{\pgfqpoint{1.286132in}{0.839159in}}{\pgfqpoint{12.053712in}{5.967710in}}%
\pgfusepath{clip}%
\pgfsetbuttcap%
\pgfsetroundjoin%
\pgfsetlinewidth{1.505625pt}%
\definecolor{currentstroke}{rgb}{0.498039,0.498039,0.498039}%
\pgfsetstrokecolor{currentstroke}%
\pgfsetdash{}{0pt}%
\pgfpathmoveto{\pgfqpoint{7.479017in}{1.617190in}}%
\pgfpathlineto{\pgfqpoint{7.479017in}{1.879232in}}%
\pgfusepath{stroke}%
\end{pgfscope}%
\begin{pgfscope}%
\pgfpathrectangle{\pgfqpoint{1.286132in}{0.839159in}}{\pgfqpoint{12.053712in}{5.967710in}}%
\pgfusepath{clip}%
\pgfsetbuttcap%
\pgfsetroundjoin%
\pgfsetlinewidth{1.505625pt}%
\definecolor{currentstroke}{rgb}{0.498039,0.498039,0.498039}%
\pgfsetstrokecolor{currentstroke}%
\pgfsetdash{}{0pt}%
\pgfpathmoveto{\pgfqpoint{7.589703in}{1.714995in}}%
\pgfpathlineto{\pgfqpoint{7.589703in}{1.993947in}}%
\pgfusepath{stroke}%
\end{pgfscope}%
\begin{pgfscope}%
\pgfpathrectangle{\pgfqpoint{1.286132in}{0.839159in}}{\pgfqpoint{12.053712in}{5.967710in}}%
\pgfusepath{clip}%
\pgfsetbuttcap%
\pgfsetroundjoin%
\pgfsetlinewidth{1.505625pt}%
\definecolor{currentstroke}{rgb}{0.498039,0.498039,0.498039}%
\pgfsetstrokecolor{currentstroke}%
\pgfsetdash{}{0pt}%
\pgfpathmoveto{\pgfqpoint{7.700389in}{1.709612in}}%
\pgfpathlineto{\pgfqpoint{7.700389in}{1.744459in}}%
\pgfusepath{stroke}%
\end{pgfscope}%
\begin{pgfscope}%
\pgfpathrectangle{\pgfqpoint{1.286132in}{0.839159in}}{\pgfqpoint{12.053712in}{5.967710in}}%
\pgfusepath{clip}%
\pgfsetbuttcap%
\pgfsetroundjoin%
\pgfsetlinewidth{1.505625pt}%
\definecolor{currentstroke}{rgb}{0.498039,0.498039,0.498039}%
\pgfsetstrokecolor{currentstroke}%
\pgfsetdash{}{0pt}%
\pgfpathmoveto{\pgfqpoint{7.811075in}{1.722752in}}%
\pgfpathlineto{\pgfqpoint{7.811075in}{1.745589in}}%
\pgfusepath{stroke}%
\end{pgfscope}%
\begin{pgfscope}%
\pgfpathrectangle{\pgfqpoint{1.286132in}{0.839159in}}{\pgfqpoint{12.053712in}{5.967710in}}%
\pgfusepath{clip}%
\pgfsetbuttcap%
\pgfsetroundjoin%
\pgfsetlinewidth{1.505625pt}%
\definecolor{currentstroke}{rgb}{0.498039,0.498039,0.498039}%
\pgfsetstrokecolor{currentstroke}%
\pgfsetdash{}{0pt}%
\pgfpathmoveto{\pgfqpoint{7.921761in}{1.762698in}}%
\pgfpathlineto{\pgfqpoint{7.921761in}{1.802720in}}%
\pgfusepath{stroke}%
\end{pgfscope}%
\begin{pgfscope}%
\pgfpathrectangle{\pgfqpoint{1.286132in}{0.839159in}}{\pgfqpoint{12.053712in}{5.967710in}}%
\pgfusepath{clip}%
\pgfsetbuttcap%
\pgfsetroundjoin%
\pgfsetlinewidth{1.505625pt}%
\definecolor{currentstroke}{rgb}{0.498039,0.498039,0.498039}%
\pgfsetstrokecolor{currentstroke}%
\pgfsetdash{}{0pt}%
\pgfpathmoveto{\pgfqpoint{8.032447in}{1.793213in}}%
\pgfpathlineto{\pgfqpoint{8.032447in}{1.806916in}}%
\pgfusepath{stroke}%
\end{pgfscope}%
\begin{pgfscope}%
\pgfpathrectangle{\pgfqpoint{1.286132in}{0.839159in}}{\pgfqpoint{12.053712in}{5.967710in}}%
\pgfusepath{clip}%
\pgfsetbuttcap%
\pgfsetroundjoin%
\pgfsetlinewidth{1.505625pt}%
\definecolor{currentstroke}{rgb}{0.498039,0.498039,0.498039}%
\pgfsetstrokecolor{currentstroke}%
\pgfsetdash{}{0pt}%
\pgfpathmoveto{\pgfqpoint{8.143133in}{1.831777in}}%
\pgfpathlineto{\pgfqpoint{8.143133in}{1.837647in}}%
\pgfusepath{stroke}%
\end{pgfscope}%
\begin{pgfscope}%
\pgfpathrectangle{\pgfqpoint{1.286132in}{0.839159in}}{\pgfqpoint{12.053712in}{5.967710in}}%
\pgfusepath{clip}%
\pgfsetbuttcap%
\pgfsetroundjoin%
\pgfsetlinewidth{1.505625pt}%
\definecolor{currentstroke}{rgb}{0.498039,0.498039,0.498039}%
\pgfsetstrokecolor{currentstroke}%
\pgfsetdash{}{0pt}%
\pgfpathmoveto{\pgfqpoint{8.253819in}{1.861908in}}%
\pgfpathlineto{\pgfqpoint{8.253819in}{1.894763in}}%
\pgfusepath{stroke}%
\end{pgfscope}%
\begin{pgfscope}%
\pgfpathrectangle{\pgfqpoint{1.286132in}{0.839159in}}{\pgfqpoint{12.053712in}{5.967710in}}%
\pgfusepath{clip}%
\pgfsetbuttcap%
\pgfsetroundjoin%
\pgfsetlinewidth{1.505625pt}%
\definecolor{currentstroke}{rgb}{0.498039,0.498039,0.498039}%
\pgfsetstrokecolor{currentstroke}%
\pgfsetdash{}{0pt}%
\pgfpathmoveto{\pgfqpoint{8.364505in}{1.900893in}}%
\pgfpathlineto{\pgfqpoint{8.364505in}{1.905672in}}%
\pgfusepath{stroke}%
\end{pgfscope}%
\begin{pgfscope}%
\pgfpathrectangle{\pgfqpoint{1.286132in}{0.839159in}}{\pgfqpoint{12.053712in}{5.967710in}}%
\pgfusepath{clip}%
\pgfsetbuttcap%
\pgfsetroundjoin%
\pgfsetlinewidth{1.505625pt}%
\definecolor{currentstroke}{rgb}{0.498039,0.498039,0.498039}%
\pgfsetstrokecolor{currentstroke}%
\pgfsetdash{}{0pt}%
\pgfpathmoveto{\pgfqpoint{8.475191in}{1.934086in}}%
\pgfpathlineto{\pgfqpoint{8.475191in}{1.951187in}}%
\pgfusepath{stroke}%
\end{pgfscope}%
\begin{pgfscope}%
\pgfpathrectangle{\pgfqpoint{1.286132in}{0.839159in}}{\pgfqpoint{12.053712in}{5.967710in}}%
\pgfusepath{clip}%
\pgfsetbuttcap%
\pgfsetroundjoin%
\pgfsetlinewidth{1.505625pt}%
\definecolor{currentstroke}{rgb}{0.498039,0.498039,0.498039}%
\pgfsetstrokecolor{currentstroke}%
\pgfsetdash{}{0pt}%
\pgfpathmoveto{\pgfqpoint{8.585877in}{1.970427in}}%
\pgfpathlineto{\pgfqpoint{8.585877in}{2.162641in}}%
\pgfusepath{stroke}%
\end{pgfscope}%
\begin{pgfscope}%
\pgfpathrectangle{\pgfqpoint{1.286132in}{0.839159in}}{\pgfqpoint{12.053712in}{5.967710in}}%
\pgfusepath{clip}%
\pgfsetbuttcap%
\pgfsetroundjoin%
\pgfsetlinewidth{1.505625pt}%
\definecolor{currentstroke}{rgb}{0.498039,0.498039,0.498039}%
\pgfsetstrokecolor{currentstroke}%
\pgfsetdash{}{0pt}%
\pgfpathmoveto{\pgfqpoint{8.696563in}{2.020227in}}%
\pgfpathlineto{\pgfqpoint{8.696563in}{2.035669in}}%
\pgfusepath{stroke}%
\end{pgfscope}%
\begin{pgfscope}%
\pgfpathrectangle{\pgfqpoint{1.286132in}{0.839159in}}{\pgfqpoint{12.053712in}{5.967710in}}%
\pgfusepath{clip}%
\pgfsetbuttcap%
\pgfsetroundjoin%
\pgfsetlinewidth{1.505625pt}%
\definecolor{currentstroke}{rgb}{0.498039,0.498039,0.498039}%
\pgfsetstrokecolor{currentstroke}%
\pgfsetdash{}{0pt}%
\pgfpathmoveto{\pgfqpoint{8.807249in}{2.060768in}}%
\pgfpathlineto{\pgfqpoint{8.807249in}{2.082834in}}%
\pgfusepath{stroke}%
\end{pgfscope}%
\begin{pgfscope}%
\pgfpathrectangle{\pgfqpoint{1.286132in}{0.839159in}}{\pgfqpoint{12.053712in}{5.967710in}}%
\pgfusepath{clip}%
\pgfsetbuttcap%
\pgfsetroundjoin%
\pgfsetlinewidth{1.505625pt}%
\definecolor{currentstroke}{rgb}{0.498039,0.498039,0.498039}%
\pgfsetstrokecolor{currentstroke}%
\pgfsetdash{}{0pt}%
\pgfpathmoveto{\pgfqpoint{8.917936in}{2.104367in}}%
\pgfpathlineto{\pgfqpoint{8.917936in}{2.110455in}}%
\pgfusepath{stroke}%
\end{pgfscope}%
\begin{pgfscope}%
\pgfpathrectangle{\pgfqpoint{1.286132in}{0.839159in}}{\pgfqpoint{12.053712in}{5.967710in}}%
\pgfusepath{clip}%
\pgfsetbuttcap%
\pgfsetroundjoin%
\pgfsetlinewidth{1.505625pt}%
\definecolor{currentstroke}{rgb}{0.498039,0.498039,0.498039}%
\pgfsetstrokecolor{currentstroke}%
\pgfsetdash{}{0pt}%
\pgfpathmoveto{\pgfqpoint{9.028622in}{2.151286in}}%
\pgfpathlineto{\pgfqpoint{9.028622in}{2.178458in}}%
\pgfusepath{stroke}%
\end{pgfscope}%
\begin{pgfscope}%
\pgfpathrectangle{\pgfqpoint{1.286132in}{0.839159in}}{\pgfqpoint{12.053712in}{5.967710in}}%
\pgfusepath{clip}%
\pgfsetbuttcap%
\pgfsetroundjoin%
\pgfsetlinewidth{1.505625pt}%
\definecolor{currentstroke}{rgb}{0.498039,0.498039,0.498039}%
\pgfsetstrokecolor{currentstroke}%
\pgfsetdash{}{0pt}%
\pgfpathmoveto{\pgfqpoint{9.139308in}{2.197248in}}%
\pgfpathlineto{\pgfqpoint{9.139308in}{2.216384in}}%
\pgfusepath{stroke}%
\end{pgfscope}%
\begin{pgfscope}%
\pgfpathrectangle{\pgfqpoint{1.286132in}{0.839159in}}{\pgfqpoint{12.053712in}{5.967710in}}%
\pgfusepath{clip}%
\pgfsetbuttcap%
\pgfsetroundjoin%
\pgfsetlinewidth{1.505625pt}%
\definecolor{currentstroke}{rgb}{0.498039,0.498039,0.498039}%
\pgfsetstrokecolor{currentstroke}%
\pgfsetdash{}{0pt}%
\pgfpathmoveto{\pgfqpoint{9.249994in}{2.247459in}}%
\pgfpathlineto{\pgfqpoint{9.249994in}{2.265105in}}%
\pgfusepath{stroke}%
\end{pgfscope}%
\begin{pgfscope}%
\pgfpathrectangle{\pgfqpoint{1.286132in}{0.839159in}}{\pgfqpoint{12.053712in}{5.967710in}}%
\pgfusepath{clip}%
\pgfsetbuttcap%
\pgfsetroundjoin%
\pgfsetlinewidth{1.505625pt}%
\definecolor{currentstroke}{rgb}{0.498039,0.498039,0.498039}%
\pgfsetstrokecolor{currentstroke}%
\pgfsetdash{}{0pt}%
\pgfpathmoveto{\pgfqpoint{9.360680in}{2.261571in}}%
\pgfpathlineto{\pgfqpoint{9.360680in}{2.554699in}}%
\pgfusepath{stroke}%
\end{pgfscope}%
\begin{pgfscope}%
\pgfpathrectangle{\pgfqpoint{1.286132in}{0.839159in}}{\pgfqpoint{12.053712in}{5.967710in}}%
\pgfusepath{clip}%
\pgfsetbuttcap%
\pgfsetroundjoin%
\pgfsetlinewidth{1.505625pt}%
\definecolor{currentstroke}{rgb}{0.498039,0.498039,0.498039}%
\pgfsetstrokecolor{currentstroke}%
\pgfsetdash{}{0pt}%
\pgfpathmoveto{\pgfqpoint{9.471366in}{2.326971in}}%
\pgfpathlineto{\pgfqpoint{9.471366in}{2.452727in}}%
\pgfusepath{stroke}%
\end{pgfscope}%
\begin{pgfscope}%
\pgfpathrectangle{\pgfqpoint{1.286132in}{0.839159in}}{\pgfqpoint{12.053712in}{5.967710in}}%
\pgfusepath{clip}%
\pgfsetbuttcap%
\pgfsetroundjoin%
\pgfsetlinewidth{1.505625pt}%
\definecolor{currentstroke}{rgb}{0.498039,0.498039,0.498039}%
\pgfsetstrokecolor{currentstroke}%
\pgfsetdash{}{0pt}%
\pgfpathmoveto{\pgfqpoint{9.582052in}{2.377725in}}%
\pgfpathlineto{\pgfqpoint{9.582052in}{2.430296in}}%
\pgfusepath{stroke}%
\end{pgfscope}%
\begin{pgfscope}%
\pgfpathrectangle{\pgfqpoint{1.286132in}{0.839159in}}{\pgfqpoint{12.053712in}{5.967710in}}%
\pgfusepath{clip}%
\pgfsetbuttcap%
\pgfsetroundjoin%
\pgfsetlinewidth{1.505625pt}%
\definecolor{currentstroke}{rgb}{0.498039,0.498039,0.498039}%
\pgfsetstrokecolor{currentstroke}%
\pgfsetdash{}{0pt}%
\pgfpathmoveto{\pgfqpoint{9.692738in}{2.458946in}}%
\pgfpathlineto{\pgfqpoint{9.692738in}{2.478163in}}%
\pgfusepath{stroke}%
\end{pgfscope}%
\begin{pgfscope}%
\pgfpathrectangle{\pgfqpoint{1.286132in}{0.839159in}}{\pgfqpoint{12.053712in}{5.967710in}}%
\pgfusepath{clip}%
\pgfsetbuttcap%
\pgfsetroundjoin%
\pgfsetlinewidth{1.505625pt}%
\definecolor{currentstroke}{rgb}{0.498039,0.498039,0.498039}%
\pgfsetstrokecolor{currentstroke}%
\pgfsetdash{}{0pt}%
\pgfpathmoveto{\pgfqpoint{9.803424in}{2.494046in}}%
\pgfpathlineto{\pgfqpoint{9.803424in}{2.579620in}}%
\pgfusepath{stroke}%
\end{pgfscope}%
\begin{pgfscope}%
\pgfpathrectangle{\pgfqpoint{1.286132in}{0.839159in}}{\pgfqpoint{12.053712in}{5.967710in}}%
\pgfusepath{clip}%
\pgfsetbuttcap%
\pgfsetroundjoin%
\pgfsetlinewidth{1.505625pt}%
\definecolor{currentstroke}{rgb}{0.498039,0.498039,0.498039}%
\pgfsetstrokecolor{currentstroke}%
\pgfsetdash{}{0pt}%
\pgfpathmoveto{\pgfqpoint{9.914110in}{2.569116in}}%
\pgfpathlineto{\pgfqpoint{9.914110in}{2.647269in}}%
\pgfusepath{stroke}%
\end{pgfscope}%
\begin{pgfscope}%
\pgfpathrectangle{\pgfqpoint{1.286132in}{0.839159in}}{\pgfqpoint{12.053712in}{5.967710in}}%
\pgfusepath{clip}%
\pgfsetbuttcap%
\pgfsetroundjoin%
\pgfsetlinewidth{1.505625pt}%
\definecolor{currentstroke}{rgb}{0.498039,0.498039,0.498039}%
\pgfsetstrokecolor{currentstroke}%
\pgfsetdash{}{0pt}%
\pgfpathmoveto{\pgfqpoint{10.024796in}{2.635635in}}%
\pgfpathlineto{\pgfqpoint{10.024796in}{2.663322in}}%
\pgfusepath{stroke}%
\end{pgfscope}%
\begin{pgfscope}%
\pgfpathrectangle{\pgfqpoint{1.286132in}{0.839159in}}{\pgfqpoint{12.053712in}{5.967710in}}%
\pgfusepath{clip}%
\pgfsetbuttcap%
\pgfsetroundjoin%
\pgfsetlinewidth{1.505625pt}%
\definecolor{currentstroke}{rgb}{0.498039,0.498039,0.498039}%
\pgfsetstrokecolor{currentstroke}%
\pgfsetdash{}{0pt}%
\pgfpathmoveto{\pgfqpoint{10.135482in}{2.694873in}}%
\pgfpathlineto{\pgfqpoint{10.135482in}{2.718820in}}%
\pgfusepath{stroke}%
\end{pgfscope}%
\begin{pgfscope}%
\pgfpathrectangle{\pgfqpoint{1.286132in}{0.839159in}}{\pgfqpoint{12.053712in}{5.967710in}}%
\pgfusepath{clip}%
\pgfsetbuttcap%
\pgfsetroundjoin%
\pgfsetlinewidth{1.505625pt}%
\definecolor{currentstroke}{rgb}{0.498039,0.498039,0.498039}%
\pgfsetstrokecolor{currentstroke}%
\pgfsetdash{}{0pt}%
\pgfpathmoveto{\pgfqpoint{10.246168in}{2.752758in}}%
\pgfpathlineto{\pgfqpoint{10.246168in}{2.772483in}}%
\pgfusepath{stroke}%
\end{pgfscope}%
\begin{pgfscope}%
\pgfpathrectangle{\pgfqpoint{1.286132in}{0.839159in}}{\pgfqpoint{12.053712in}{5.967710in}}%
\pgfusepath{clip}%
\pgfsetbuttcap%
\pgfsetroundjoin%
\pgfsetlinewidth{1.505625pt}%
\definecolor{currentstroke}{rgb}{0.498039,0.498039,0.498039}%
\pgfsetstrokecolor{currentstroke}%
\pgfsetdash{}{0pt}%
\pgfpathmoveto{\pgfqpoint{10.356854in}{2.819734in}}%
\pgfpathlineto{\pgfqpoint{10.356854in}{2.843616in}}%
\pgfusepath{stroke}%
\end{pgfscope}%
\begin{pgfscope}%
\pgfpathrectangle{\pgfqpoint{1.286132in}{0.839159in}}{\pgfqpoint{12.053712in}{5.967710in}}%
\pgfusepath{clip}%
\pgfsetbuttcap%
\pgfsetroundjoin%
\pgfsetlinewidth{1.505625pt}%
\definecolor{currentstroke}{rgb}{0.498039,0.498039,0.498039}%
\pgfsetstrokecolor{currentstroke}%
\pgfsetdash{}{0pt}%
\pgfpathmoveto{\pgfqpoint{10.467540in}{2.925195in}}%
\pgfpathlineto{\pgfqpoint{10.467540in}{3.070005in}}%
\pgfusepath{stroke}%
\end{pgfscope}%
\begin{pgfscope}%
\pgfpathrectangle{\pgfqpoint{1.286132in}{0.839159in}}{\pgfqpoint{12.053712in}{5.967710in}}%
\pgfusepath{clip}%
\pgfsetbuttcap%
\pgfsetroundjoin%
\pgfsetlinewidth{1.505625pt}%
\definecolor{currentstroke}{rgb}{0.498039,0.498039,0.498039}%
\pgfsetstrokecolor{currentstroke}%
\pgfsetdash{}{0pt}%
\pgfpathmoveto{\pgfqpoint{10.578226in}{2.972680in}}%
\pgfpathlineto{\pgfqpoint{10.578226in}{3.071355in}}%
\pgfusepath{stroke}%
\end{pgfscope}%
\begin{pgfscope}%
\pgfpathrectangle{\pgfqpoint{1.286132in}{0.839159in}}{\pgfqpoint{12.053712in}{5.967710in}}%
\pgfusepath{clip}%
\pgfsetbuttcap%
\pgfsetroundjoin%
\pgfsetlinewidth{1.505625pt}%
\definecolor{currentstroke}{rgb}{0.498039,0.498039,0.498039}%
\pgfsetstrokecolor{currentstroke}%
\pgfsetdash{}{0pt}%
\pgfpathmoveto{\pgfqpoint{10.688913in}{3.017374in}}%
\pgfpathlineto{\pgfqpoint{10.688913in}{3.036296in}}%
\pgfusepath{stroke}%
\end{pgfscope}%
\begin{pgfscope}%
\pgfpathrectangle{\pgfqpoint{1.286132in}{0.839159in}}{\pgfqpoint{12.053712in}{5.967710in}}%
\pgfusepath{clip}%
\pgfsetbuttcap%
\pgfsetroundjoin%
\pgfsetlinewidth{1.505625pt}%
\definecolor{currentstroke}{rgb}{0.498039,0.498039,0.498039}%
\pgfsetstrokecolor{currentstroke}%
\pgfsetdash{}{0pt}%
\pgfpathmoveto{\pgfqpoint{10.799599in}{3.088167in}}%
\pgfpathlineto{\pgfqpoint{10.799599in}{3.159043in}}%
\pgfusepath{stroke}%
\end{pgfscope}%
\begin{pgfscope}%
\pgfpathrectangle{\pgfqpoint{1.286132in}{0.839159in}}{\pgfqpoint{12.053712in}{5.967710in}}%
\pgfusepath{clip}%
\pgfsetbuttcap%
\pgfsetroundjoin%
\pgfsetlinewidth{1.505625pt}%
\definecolor{currentstroke}{rgb}{0.498039,0.498039,0.498039}%
\pgfsetstrokecolor{currentstroke}%
\pgfsetdash{}{0pt}%
\pgfpathmoveto{\pgfqpoint{10.910285in}{3.095593in}}%
\pgfpathlineto{\pgfqpoint{10.910285in}{3.367997in}}%
\pgfusepath{stroke}%
\end{pgfscope}%
\begin{pgfscope}%
\pgfpathrectangle{\pgfqpoint{1.286132in}{0.839159in}}{\pgfqpoint{12.053712in}{5.967710in}}%
\pgfusepath{clip}%
\pgfsetbuttcap%
\pgfsetroundjoin%
\pgfsetlinewidth{1.505625pt}%
\definecolor{currentstroke}{rgb}{0.498039,0.498039,0.498039}%
\pgfsetstrokecolor{currentstroke}%
\pgfsetdash{}{0pt}%
\pgfpathmoveto{\pgfqpoint{11.020971in}{3.222335in}}%
\pgfpathlineto{\pgfqpoint{11.020971in}{3.286749in}}%
\pgfusepath{stroke}%
\end{pgfscope}%
\begin{pgfscope}%
\pgfpathrectangle{\pgfqpoint{1.286132in}{0.839159in}}{\pgfqpoint{12.053712in}{5.967710in}}%
\pgfusepath{clip}%
\pgfsetbuttcap%
\pgfsetroundjoin%
\pgfsetlinewidth{1.505625pt}%
\definecolor{currentstroke}{rgb}{0.498039,0.498039,0.498039}%
\pgfsetstrokecolor{currentstroke}%
\pgfsetdash{}{0pt}%
\pgfpathmoveto{\pgfqpoint{11.131657in}{3.311119in}}%
\pgfpathlineto{\pgfqpoint{11.131657in}{3.321043in}}%
\pgfusepath{stroke}%
\end{pgfscope}%
\begin{pgfscope}%
\pgfpathrectangle{\pgfqpoint{1.286132in}{0.839159in}}{\pgfqpoint{12.053712in}{5.967710in}}%
\pgfusepath{clip}%
\pgfsetbuttcap%
\pgfsetroundjoin%
\pgfsetlinewidth{1.505625pt}%
\definecolor{currentstroke}{rgb}{0.498039,0.498039,0.498039}%
\pgfsetstrokecolor{currentstroke}%
\pgfsetdash{}{0pt}%
\pgfpathmoveto{\pgfqpoint{11.242343in}{3.242100in}}%
\pgfpathlineto{\pgfqpoint{11.242343in}{3.833067in}}%
\pgfusepath{stroke}%
\end{pgfscope}%
\begin{pgfscope}%
\pgfpathrectangle{\pgfqpoint{1.286132in}{0.839159in}}{\pgfqpoint{12.053712in}{5.967710in}}%
\pgfusepath{clip}%
\pgfsetbuttcap%
\pgfsetroundjoin%
\pgfsetlinewidth{1.505625pt}%
\definecolor{currentstroke}{rgb}{0.498039,0.498039,0.498039}%
\pgfsetstrokecolor{currentstroke}%
\pgfsetdash{}{0pt}%
\pgfpathmoveto{\pgfqpoint{11.353029in}{3.450038in}}%
\pgfpathlineto{\pgfqpoint{11.353029in}{3.482929in}}%
\pgfusepath{stroke}%
\end{pgfscope}%
\begin{pgfscope}%
\pgfpathrectangle{\pgfqpoint{1.286132in}{0.839159in}}{\pgfqpoint{12.053712in}{5.967710in}}%
\pgfusepath{clip}%
\pgfsetbuttcap%
\pgfsetroundjoin%
\pgfsetlinewidth{1.505625pt}%
\definecolor{currentstroke}{rgb}{0.498039,0.498039,0.498039}%
\pgfsetstrokecolor{currentstroke}%
\pgfsetdash{}{0pt}%
\pgfpathmoveto{\pgfqpoint{11.463715in}{3.544952in}}%
\pgfpathlineto{\pgfqpoint{11.463715in}{3.562818in}}%
\pgfusepath{stroke}%
\end{pgfscope}%
\begin{pgfscope}%
\pgfpathrectangle{\pgfqpoint{1.286132in}{0.839159in}}{\pgfqpoint{12.053712in}{5.967710in}}%
\pgfusepath{clip}%
\pgfsetbuttcap%
\pgfsetroundjoin%
\pgfsetlinewidth{1.505625pt}%
\definecolor{currentstroke}{rgb}{0.498039,0.498039,0.498039}%
\pgfsetstrokecolor{currentstroke}%
\pgfsetdash{}{0pt}%
\pgfpathmoveto{\pgfqpoint{11.574401in}{3.621780in}}%
\pgfpathlineto{\pgfqpoint{11.574401in}{3.701153in}}%
\pgfusepath{stroke}%
\end{pgfscope}%
\begin{pgfscope}%
\pgfpathrectangle{\pgfqpoint{1.286132in}{0.839159in}}{\pgfqpoint{12.053712in}{5.967710in}}%
\pgfusepath{clip}%
\pgfsetbuttcap%
\pgfsetroundjoin%
\pgfsetlinewidth{1.505625pt}%
\definecolor{currentstroke}{rgb}{0.498039,0.498039,0.498039}%
\pgfsetstrokecolor{currentstroke}%
\pgfsetdash{}{0pt}%
\pgfpathmoveto{\pgfqpoint{11.685087in}{3.722258in}}%
\pgfpathlineto{\pgfqpoint{11.685087in}{3.743115in}}%
\pgfusepath{stroke}%
\end{pgfscope}%
\begin{pgfscope}%
\pgfpathrectangle{\pgfqpoint{1.286132in}{0.839159in}}{\pgfqpoint{12.053712in}{5.967710in}}%
\pgfusepath{clip}%
\pgfsetbuttcap%
\pgfsetroundjoin%
\pgfsetlinewidth{1.505625pt}%
\definecolor{currentstroke}{rgb}{0.498039,0.498039,0.498039}%
\pgfsetstrokecolor{currentstroke}%
\pgfsetdash{}{0pt}%
\pgfpathmoveto{\pgfqpoint{11.795773in}{3.813765in}}%
\pgfpathlineto{\pgfqpoint{11.795773in}{3.871581in}}%
\pgfusepath{stroke}%
\end{pgfscope}%
\begin{pgfscope}%
\pgfpathrectangle{\pgfqpoint{1.286132in}{0.839159in}}{\pgfqpoint{12.053712in}{5.967710in}}%
\pgfusepath{clip}%
\pgfsetbuttcap%
\pgfsetroundjoin%
\pgfsetlinewidth{1.505625pt}%
\definecolor{currentstroke}{rgb}{0.498039,0.498039,0.498039}%
\pgfsetstrokecolor{currentstroke}%
\pgfsetdash{}{0pt}%
\pgfpathmoveto{\pgfqpoint{11.906459in}{3.881016in}}%
\pgfpathlineto{\pgfqpoint{11.906459in}{4.221154in}}%
\pgfusepath{stroke}%
\end{pgfscope}%
\begin{pgfscope}%
\pgfpathrectangle{\pgfqpoint{1.286132in}{0.839159in}}{\pgfqpoint{12.053712in}{5.967710in}}%
\pgfusepath{clip}%
\pgfsetbuttcap%
\pgfsetroundjoin%
\pgfsetlinewidth{1.505625pt}%
\definecolor{currentstroke}{rgb}{0.498039,0.498039,0.498039}%
\pgfsetstrokecolor{currentstroke}%
\pgfsetdash{}{0pt}%
\pgfpathmoveto{\pgfqpoint{12.017145in}{4.103684in}}%
\pgfpathlineto{\pgfqpoint{12.017145in}{4.194970in}}%
\pgfusepath{stroke}%
\end{pgfscope}%
\begin{pgfscope}%
\pgfpathrectangle{\pgfqpoint{1.286132in}{0.839159in}}{\pgfqpoint{12.053712in}{5.967710in}}%
\pgfusepath{clip}%
\pgfsetbuttcap%
\pgfsetroundjoin%
\pgfsetlinewidth{1.505625pt}%
\definecolor{currentstroke}{rgb}{0.498039,0.498039,0.498039}%
\pgfsetstrokecolor{currentstroke}%
\pgfsetdash{}{0pt}%
\pgfpathmoveto{\pgfqpoint{12.127831in}{4.321508in}}%
\pgfpathlineto{\pgfqpoint{12.127831in}{5.343252in}}%
\pgfusepath{stroke}%
\end{pgfscope}%
\begin{pgfscope}%
\pgfpathrectangle{\pgfqpoint{1.286132in}{0.839159in}}{\pgfqpoint{12.053712in}{5.967710in}}%
\pgfusepath{clip}%
\pgfsetbuttcap%
\pgfsetroundjoin%
\pgfsetlinewidth{1.505625pt}%
\definecolor{currentstroke}{rgb}{0.498039,0.498039,0.498039}%
\pgfsetstrokecolor{currentstroke}%
\pgfsetdash{}{0pt}%
\pgfpathmoveto{\pgfqpoint{12.238517in}{4.327094in}}%
\pgfpathlineto{\pgfqpoint{12.238517in}{4.609046in}}%
\pgfusepath{stroke}%
\end{pgfscope}%
\begin{pgfscope}%
\pgfpathrectangle{\pgfqpoint{1.286132in}{0.839159in}}{\pgfqpoint{12.053712in}{5.967710in}}%
\pgfusepath{clip}%
\pgfsetbuttcap%
\pgfsetroundjoin%
\pgfsetlinewidth{1.505625pt}%
\definecolor{currentstroke}{rgb}{0.498039,0.498039,0.498039}%
\pgfsetstrokecolor{currentstroke}%
\pgfsetdash{}{0pt}%
\pgfpathmoveto{\pgfqpoint{12.349203in}{4.408700in}}%
\pgfpathlineto{\pgfqpoint{12.349203in}{5.284530in}}%
\pgfusepath{stroke}%
\end{pgfscope}%
\begin{pgfscope}%
\pgfpathrectangle{\pgfqpoint{1.286132in}{0.839159in}}{\pgfqpoint{12.053712in}{5.967710in}}%
\pgfusepath{clip}%
\pgfsetbuttcap%
\pgfsetroundjoin%
\pgfsetlinewidth{1.505625pt}%
\definecolor{currentstroke}{rgb}{0.498039,0.498039,0.498039}%
\pgfsetstrokecolor{currentstroke}%
\pgfsetdash{}{0pt}%
\pgfpathmoveto{\pgfqpoint{12.459890in}{4.634725in}}%
\pgfpathlineto{\pgfqpoint{12.459890in}{5.025403in}}%
\pgfusepath{stroke}%
\end{pgfscope}%
\begin{pgfscope}%
\pgfpathrectangle{\pgfqpoint{1.286132in}{0.839159in}}{\pgfqpoint{12.053712in}{5.967710in}}%
\pgfusepath{clip}%
\pgfsetbuttcap%
\pgfsetroundjoin%
\pgfsetlinewidth{1.505625pt}%
\definecolor{currentstroke}{rgb}{0.498039,0.498039,0.498039}%
\pgfsetstrokecolor{currentstroke}%
\pgfsetdash{}{0pt}%
\pgfpathmoveto{\pgfqpoint{12.570576in}{4.667544in}}%
\pgfpathlineto{\pgfqpoint{12.570576in}{6.535610in}}%
\pgfusepath{stroke}%
\end{pgfscope}%
\begin{pgfscope}%
\pgfpathrectangle{\pgfqpoint{1.286132in}{0.839159in}}{\pgfqpoint{12.053712in}{5.967710in}}%
\pgfusepath{clip}%
\pgfsetbuttcap%
\pgfsetroundjoin%
\pgfsetlinewidth{1.505625pt}%
\definecolor{currentstroke}{rgb}{0.498039,0.498039,0.498039}%
\pgfsetstrokecolor{currentstroke}%
\pgfsetdash{}{0pt}%
\pgfpathmoveto{\pgfqpoint{12.681262in}{4.723218in}}%
\pgfpathlineto{\pgfqpoint{12.681262in}{5.331784in}}%
\pgfusepath{stroke}%
\end{pgfscope}%
\begin{pgfscope}%
\pgfpathrectangle{\pgfqpoint{1.286132in}{0.839159in}}{\pgfqpoint{12.053712in}{5.967710in}}%
\pgfusepath{clip}%
\pgfsetbuttcap%
\pgfsetroundjoin%
\pgfsetlinewidth{1.505625pt}%
\definecolor{currentstroke}{rgb}{0.498039,0.498039,0.498039}%
\pgfsetstrokecolor{currentstroke}%
\pgfsetdash{}{0pt}%
\pgfpathmoveto{\pgfqpoint{12.791948in}{4.418518in}}%
\pgfpathlineto{\pgfqpoint{12.791948in}{6.189484in}}%
\pgfusepath{stroke}%
\end{pgfscope}%
\begin{pgfscope}%
\pgfpathrectangle{\pgfqpoint{1.286132in}{0.839159in}}{\pgfqpoint{12.053712in}{5.967710in}}%
\pgfusepath{clip}%
\pgfsetbuttcap%
\pgfsetroundjoin%
\pgfsetlinewidth{1.505625pt}%
\definecolor{currentstroke}{rgb}{0.737255,0.741176,0.133333}%
\pgfsetstrokecolor{currentstroke}%
\pgfsetdash{}{0pt}%
\pgfpathmoveto{\pgfqpoint{1.834028in}{1.119274in}}%
\pgfpathlineto{\pgfqpoint{1.834028in}{1.119317in}}%
\pgfusepath{stroke}%
\end{pgfscope}%
\begin{pgfscope}%
\pgfpathrectangle{\pgfqpoint{1.286132in}{0.839159in}}{\pgfqpoint{12.053712in}{5.967710in}}%
\pgfusepath{clip}%
\pgfsetbuttcap%
\pgfsetroundjoin%
\pgfsetlinewidth{1.505625pt}%
\definecolor{currentstroke}{rgb}{0.737255,0.741176,0.133333}%
\pgfsetstrokecolor{currentstroke}%
\pgfsetdash{}{0pt}%
\pgfpathmoveto{\pgfqpoint{1.944714in}{1.119251in}}%
\pgfpathlineto{\pgfqpoint{1.944714in}{1.119257in}}%
\pgfusepath{stroke}%
\end{pgfscope}%
\begin{pgfscope}%
\pgfpathrectangle{\pgfqpoint{1.286132in}{0.839159in}}{\pgfqpoint{12.053712in}{5.967710in}}%
\pgfusepath{clip}%
\pgfsetbuttcap%
\pgfsetroundjoin%
\pgfsetlinewidth{1.505625pt}%
\definecolor{currentstroke}{rgb}{0.737255,0.741176,0.133333}%
\pgfsetstrokecolor{currentstroke}%
\pgfsetdash{}{0pt}%
\pgfpathmoveto{\pgfqpoint{2.055400in}{1.119316in}}%
\pgfpathlineto{\pgfqpoint{2.055400in}{1.119324in}}%
\pgfusepath{stroke}%
\end{pgfscope}%
\begin{pgfscope}%
\pgfpathrectangle{\pgfqpoint{1.286132in}{0.839159in}}{\pgfqpoint{12.053712in}{5.967710in}}%
\pgfusepath{clip}%
\pgfsetbuttcap%
\pgfsetroundjoin%
\pgfsetlinewidth{1.505625pt}%
\definecolor{currentstroke}{rgb}{0.737255,0.741176,0.133333}%
\pgfsetstrokecolor{currentstroke}%
\pgfsetdash{}{0pt}%
\pgfpathmoveto{\pgfqpoint{2.166086in}{1.119354in}}%
\pgfpathlineto{\pgfqpoint{2.166086in}{1.119462in}}%
\pgfusepath{stroke}%
\end{pgfscope}%
\begin{pgfscope}%
\pgfpathrectangle{\pgfqpoint{1.286132in}{0.839159in}}{\pgfqpoint{12.053712in}{5.967710in}}%
\pgfusepath{clip}%
\pgfsetbuttcap%
\pgfsetroundjoin%
\pgfsetlinewidth{1.505625pt}%
\definecolor{currentstroke}{rgb}{0.737255,0.741176,0.133333}%
\pgfsetstrokecolor{currentstroke}%
\pgfsetdash{}{0pt}%
\pgfpathmoveto{\pgfqpoint{2.276772in}{1.119410in}}%
\pgfpathlineto{\pgfqpoint{2.276772in}{1.119580in}}%
\pgfusepath{stroke}%
\end{pgfscope}%
\begin{pgfscope}%
\pgfpathrectangle{\pgfqpoint{1.286132in}{0.839159in}}{\pgfqpoint{12.053712in}{5.967710in}}%
\pgfusepath{clip}%
\pgfsetbuttcap%
\pgfsetroundjoin%
\pgfsetlinewidth{1.505625pt}%
\definecolor{currentstroke}{rgb}{0.737255,0.741176,0.133333}%
\pgfsetstrokecolor{currentstroke}%
\pgfsetdash{}{0pt}%
\pgfpathmoveto{\pgfqpoint{2.387458in}{1.119549in}}%
\pgfpathlineto{\pgfqpoint{2.387458in}{1.119857in}}%
\pgfusepath{stroke}%
\end{pgfscope}%
\begin{pgfscope}%
\pgfpathrectangle{\pgfqpoint{1.286132in}{0.839159in}}{\pgfqpoint{12.053712in}{5.967710in}}%
\pgfusepath{clip}%
\pgfsetbuttcap%
\pgfsetroundjoin%
\pgfsetlinewidth{1.505625pt}%
\definecolor{currentstroke}{rgb}{0.737255,0.741176,0.133333}%
\pgfsetstrokecolor{currentstroke}%
\pgfsetdash{}{0pt}%
\pgfpathmoveto{\pgfqpoint{2.498144in}{1.119567in}}%
\pgfpathlineto{\pgfqpoint{2.498144in}{1.119729in}}%
\pgfusepath{stroke}%
\end{pgfscope}%
\begin{pgfscope}%
\pgfpathrectangle{\pgfqpoint{1.286132in}{0.839159in}}{\pgfqpoint{12.053712in}{5.967710in}}%
\pgfusepath{clip}%
\pgfsetbuttcap%
\pgfsetroundjoin%
\pgfsetlinewidth{1.505625pt}%
\definecolor{currentstroke}{rgb}{0.737255,0.741176,0.133333}%
\pgfsetstrokecolor{currentstroke}%
\pgfsetdash{}{0pt}%
\pgfpathmoveto{\pgfqpoint{2.608830in}{1.119694in}}%
\pgfpathlineto{\pgfqpoint{2.608830in}{1.120028in}}%
\pgfusepath{stroke}%
\end{pgfscope}%
\begin{pgfscope}%
\pgfpathrectangle{\pgfqpoint{1.286132in}{0.839159in}}{\pgfqpoint{12.053712in}{5.967710in}}%
\pgfusepath{clip}%
\pgfsetbuttcap%
\pgfsetroundjoin%
\pgfsetlinewidth{1.505625pt}%
\definecolor{currentstroke}{rgb}{0.737255,0.741176,0.133333}%
\pgfsetstrokecolor{currentstroke}%
\pgfsetdash{}{0pt}%
\pgfpathmoveto{\pgfqpoint{2.719516in}{1.119619in}}%
\pgfpathlineto{\pgfqpoint{2.719516in}{1.119949in}}%
\pgfusepath{stroke}%
\end{pgfscope}%
\begin{pgfscope}%
\pgfpathrectangle{\pgfqpoint{1.286132in}{0.839159in}}{\pgfqpoint{12.053712in}{5.967710in}}%
\pgfusepath{clip}%
\pgfsetbuttcap%
\pgfsetroundjoin%
\pgfsetlinewidth{1.505625pt}%
\definecolor{currentstroke}{rgb}{0.737255,0.741176,0.133333}%
\pgfsetstrokecolor{currentstroke}%
\pgfsetdash{}{0pt}%
\pgfpathmoveto{\pgfqpoint{2.830202in}{1.119626in}}%
\pgfpathlineto{\pgfqpoint{2.830202in}{1.119988in}}%
\pgfusepath{stroke}%
\end{pgfscope}%
\begin{pgfscope}%
\pgfpathrectangle{\pgfqpoint{1.286132in}{0.839159in}}{\pgfqpoint{12.053712in}{5.967710in}}%
\pgfusepath{clip}%
\pgfsetbuttcap%
\pgfsetroundjoin%
\pgfsetlinewidth{1.505625pt}%
\definecolor{currentstroke}{rgb}{0.737255,0.741176,0.133333}%
\pgfsetstrokecolor{currentstroke}%
\pgfsetdash{}{0pt}%
\pgfpathmoveto{\pgfqpoint{2.940888in}{1.119840in}}%
\pgfpathlineto{\pgfqpoint{2.940888in}{1.120477in}}%
\pgfusepath{stroke}%
\end{pgfscope}%
\begin{pgfscope}%
\pgfpathrectangle{\pgfqpoint{1.286132in}{0.839159in}}{\pgfqpoint{12.053712in}{5.967710in}}%
\pgfusepath{clip}%
\pgfsetbuttcap%
\pgfsetroundjoin%
\pgfsetlinewidth{1.505625pt}%
\definecolor{currentstroke}{rgb}{0.737255,0.741176,0.133333}%
\pgfsetstrokecolor{currentstroke}%
\pgfsetdash{}{0pt}%
\pgfpathmoveto{\pgfqpoint{3.051574in}{1.119922in}}%
\pgfpathlineto{\pgfqpoint{3.051574in}{1.120556in}}%
\pgfusepath{stroke}%
\end{pgfscope}%
\begin{pgfscope}%
\pgfpathrectangle{\pgfqpoint{1.286132in}{0.839159in}}{\pgfqpoint{12.053712in}{5.967710in}}%
\pgfusepath{clip}%
\pgfsetbuttcap%
\pgfsetroundjoin%
\pgfsetlinewidth{1.505625pt}%
\definecolor{currentstroke}{rgb}{0.737255,0.741176,0.133333}%
\pgfsetstrokecolor{currentstroke}%
\pgfsetdash{}{0pt}%
\pgfpathmoveto{\pgfqpoint{3.162260in}{1.120115in}}%
\pgfpathlineto{\pgfqpoint{3.162260in}{1.120453in}}%
\pgfusepath{stroke}%
\end{pgfscope}%
\begin{pgfscope}%
\pgfpathrectangle{\pgfqpoint{1.286132in}{0.839159in}}{\pgfqpoint{12.053712in}{5.967710in}}%
\pgfusepath{clip}%
\pgfsetbuttcap%
\pgfsetroundjoin%
\pgfsetlinewidth{1.505625pt}%
\definecolor{currentstroke}{rgb}{0.737255,0.741176,0.133333}%
\pgfsetstrokecolor{currentstroke}%
\pgfsetdash{}{0pt}%
\pgfpathmoveto{\pgfqpoint{3.272946in}{1.119992in}}%
\pgfpathlineto{\pgfqpoint{3.272946in}{1.120604in}}%
\pgfusepath{stroke}%
\end{pgfscope}%
\begin{pgfscope}%
\pgfpathrectangle{\pgfqpoint{1.286132in}{0.839159in}}{\pgfqpoint{12.053712in}{5.967710in}}%
\pgfusepath{clip}%
\pgfsetbuttcap%
\pgfsetroundjoin%
\pgfsetlinewidth{1.505625pt}%
\definecolor{currentstroke}{rgb}{0.737255,0.741176,0.133333}%
\pgfsetstrokecolor{currentstroke}%
\pgfsetdash{}{0pt}%
\pgfpathmoveto{\pgfqpoint{3.383632in}{1.119948in}}%
\pgfpathlineto{\pgfqpoint{3.383632in}{1.120521in}}%
\pgfusepath{stroke}%
\end{pgfscope}%
\begin{pgfscope}%
\pgfpathrectangle{\pgfqpoint{1.286132in}{0.839159in}}{\pgfqpoint{12.053712in}{5.967710in}}%
\pgfusepath{clip}%
\pgfsetbuttcap%
\pgfsetroundjoin%
\pgfsetlinewidth{1.505625pt}%
\definecolor{currentstroke}{rgb}{0.737255,0.741176,0.133333}%
\pgfsetstrokecolor{currentstroke}%
\pgfsetdash{}{0pt}%
\pgfpathmoveto{\pgfqpoint{3.494319in}{1.120109in}}%
\pgfpathlineto{\pgfqpoint{3.494319in}{1.120433in}}%
\pgfusepath{stroke}%
\end{pgfscope}%
\begin{pgfscope}%
\pgfpathrectangle{\pgfqpoint{1.286132in}{0.839159in}}{\pgfqpoint{12.053712in}{5.967710in}}%
\pgfusepath{clip}%
\pgfsetbuttcap%
\pgfsetroundjoin%
\pgfsetlinewidth{1.505625pt}%
\definecolor{currentstroke}{rgb}{0.737255,0.741176,0.133333}%
\pgfsetstrokecolor{currentstroke}%
\pgfsetdash{}{0pt}%
\pgfpathmoveto{\pgfqpoint{3.605005in}{1.120134in}}%
\pgfpathlineto{\pgfqpoint{3.605005in}{1.120684in}}%
\pgfusepath{stroke}%
\end{pgfscope}%
\begin{pgfscope}%
\pgfpathrectangle{\pgfqpoint{1.286132in}{0.839159in}}{\pgfqpoint{12.053712in}{5.967710in}}%
\pgfusepath{clip}%
\pgfsetbuttcap%
\pgfsetroundjoin%
\pgfsetlinewidth{1.505625pt}%
\definecolor{currentstroke}{rgb}{0.737255,0.741176,0.133333}%
\pgfsetstrokecolor{currentstroke}%
\pgfsetdash{}{0pt}%
\pgfpathmoveto{\pgfqpoint{3.715691in}{1.120216in}}%
\pgfpathlineto{\pgfqpoint{3.715691in}{1.121116in}}%
\pgfusepath{stroke}%
\end{pgfscope}%
\begin{pgfscope}%
\pgfpathrectangle{\pgfqpoint{1.286132in}{0.839159in}}{\pgfqpoint{12.053712in}{5.967710in}}%
\pgfusepath{clip}%
\pgfsetbuttcap%
\pgfsetroundjoin%
\pgfsetlinewidth{1.505625pt}%
\definecolor{currentstroke}{rgb}{0.737255,0.741176,0.133333}%
\pgfsetstrokecolor{currentstroke}%
\pgfsetdash{}{0pt}%
\pgfpathmoveto{\pgfqpoint{3.826377in}{1.120228in}}%
\pgfpathlineto{\pgfqpoint{3.826377in}{1.121389in}}%
\pgfusepath{stroke}%
\end{pgfscope}%
\begin{pgfscope}%
\pgfpathrectangle{\pgfqpoint{1.286132in}{0.839159in}}{\pgfqpoint{12.053712in}{5.967710in}}%
\pgfusepath{clip}%
\pgfsetbuttcap%
\pgfsetroundjoin%
\pgfsetlinewidth{1.505625pt}%
\definecolor{currentstroke}{rgb}{0.737255,0.741176,0.133333}%
\pgfsetstrokecolor{currentstroke}%
\pgfsetdash{}{0pt}%
\pgfpathmoveto{\pgfqpoint{3.937063in}{1.120412in}}%
\pgfpathlineto{\pgfqpoint{3.937063in}{1.121076in}}%
\pgfusepath{stroke}%
\end{pgfscope}%
\begin{pgfscope}%
\pgfpathrectangle{\pgfqpoint{1.286132in}{0.839159in}}{\pgfqpoint{12.053712in}{5.967710in}}%
\pgfusepath{clip}%
\pgfsetbuttcap%
\pgfsetroundjoin%
\pgfsetlinewidth{1.505625pt}%
\definecolor{currentstroke}{rgb}{0.737255,0.741176,0.133333}%
\pgfsetstrokecolor{currentstroke}%
\pgfsetdash{}{0pt}%
\pgfpathmoveto{\pgfqpoint{4.047749in}{1.120615in}}%
\pgfpathlineto{\pgfqpoint{4.047749in}{1.120641in}}%
\pgfusepath{stroke}%
\end{pgfscope}%
\begin{pgfscope}%
\pgfpathrectangle{\pgfqpoint{1.286132in}{0.839159in}}{\pgfqpoint{12.053712in}{5.967710in}}%
\pgfusepath{clip}%
\pgfsetbuttcap%
\pgfsetroundjoin%
\pgfsetlinewidth{1.505625pt}%
\definecolor{currentstroke}{rgb}{0.737255,0.741176,0.133333}%
\pgfsetstrokecolor{currentstroke}%
\pgfsetdash{}{0pt}%
\pgfpathmoveto{\pgfqpoint{4.158435in}{1.120417in}}%
\pgfpathlineto{\pgfqpoint{4.158435in}{1.121142in}}%
\pgfusepath{stroke}%
\end{pgfscope}%
\begin{pgfscope}%
\pgfpathrectangle{\pgfqpoint{1.286132in}{0.839159in}}{\pgfqpoint{12.053712in}{5.967710in}}%
\pgfusepath{clip}%
\pgfsetbuttcap%
\pgfsetroundjoin%
\pgfsetlinewidth{1.505625pt}%
\definecolor{currentstroke}{rgb}{0.737255,0.741176,0.133333}%
\pgfsetstrokecolor{currentstroke}%
\pgfsetdash{}{0pt}%
\pgfpathmoveto{\pgfqpoint{4.269121in}{1.120884in}}%
\pgfpathlineto{\pgfqpoint{4.269121in}{1.121648in}}%
\pgfusepath{stroke}%
\end{pgfscope}%
\begin{pgfscope}%
\pgfpathrectangle{\pgfqpoint{1.286132in}{0.839159in}}{\pgfqpoint{12.053712in}{5.967710in}}%
\pgfusepath{clip}%
\pgfsetbuttcap%
\pgfsetroundjoin%
\pgfsetlinewidth{1.505625pt}%
\definecolor{currentstroke}{rgb}{0.737255,0.741176,0.133333}%
\pgfsetstrokecolor{currentstroke}%
\pgfsetdash{}{0pt}%
\pgfpathmoveto{\pgfqpoint{4.379807in}{1.120954in}}%
\pgfpathlineto{\pgfqpoint{4.379807in}{1.122578in}}%
\pgfusepath{stroke}%
\end{pgfscope}%
\begin{pgfscope}%
\pgfpathrectangle{\pgfqpoint{1.286132in}{0.839159in}}{\pgfqpoint{12.053712in}{5.967710in}}%
\pgfusepath{clip}%
\pgfsetbuttcap%
\pgfsetroundjoin%
\pgfsetlinewidth{1.505625pt}%
\definecolor{currentstroke}{rgb}{0.737255,0.741176,0.133333}%
\pgfsetstrokecolor{currentstroke}%
\pgfsetdash{}{0pt}%
\pgfpathmoveto{\pgfqpoint{4.490493in}{1.120758in}}%
\pgfpathlineto{\pgfqpoint{4.490493in}{1.121179in}}%
\pgfusepath{stroke}%
\end{pgfscope}%
\begin{pgfscope}%
\pgfpathrectangle{\pgfqpoint{1.286132in}{0.839159in}}{\pgfqpoint{12.053712in}{5.967710in}}%
\pgfusepath{clip}%
\pgfsetbuttcap%
\pgfsetroundjoin%
\pgfsetlinewidth{1.505625pt}%
\definecolor{currentstroke}{rgb}{0.737255,0.741176,0.133333}%
\pgfsetstrokecolor{currentstroke}%
\pgfsetdash{}{0pt}%
\pgfpathmoveto{\pgfqpoint{4.601179in}{1.119663in}}%
\pgfpathlineto{\pgfqpoint{4.601179in}{1.126417in}}%
\pgfusepath{stroke}%
\end{pgfscope}%
\begin{pgfscope}%
\pgfpathrectangle{\pgfqpoint{1.286132in}{0.839159in}}{\pgfqpoint{12.053712in}{5.967710in}}%
\pgfusepath{clip}%
\pgfsetbuttcap%
\pgfsetroundjoin%
\pgfsetlinewidth{1.505625pt}%
\definecolor{currentstroke}{rgb}{0.737255,0.741176,0.133333}%
\pgfsetstrokecolor{currentstroke}%
\pgfsetdash{}{0pt}%
\pgfpathmoveto{\pgfqpoint{4.711865in}{1.120704in}}%
\pgfpathlineto{\pgfqpoint{4.711865in}{1.121538in}}%
\pgfusepath{stroke}%
\end{pgfscope}%
\begin{pgfscope}%
\pgfpathrectangle{\pgfqpoint{1.286132in}{0.839159in}}{\pgfqpoint{12.053712in}{5.967710in}}%
\pgfusepath{clip}%
\pgfsetbuttcap%
\pgfsetroundjoin%
\pgfsetlinewidth{1.505625pt}%
\definecolor{currentstroke}{rgb}{0.737255,0.741176,0.133333}%
\pgfsetstrokecolor{currentstroke}%
\pgfsetdash{}{0pt}%
\pgfpathmoveto{\pgfqpoint{4.822551in}{1.121135in}}%
\pgfpathlineto{\pgfqpoint{4.822551in}{1.121678in}}%
\pgfusepath{stroke}%
\end{pgfscope}%
\begin{pgfscope}%
\pgfpathrectangle{\pgfqpoint{1.286132in}{0.839159in}}{\pgfqpoint{12.053712in}{5.967710in}}%
\pgfusepath{clip}%
\pgfsetbuttcap%
\pgfsetroundjoin%
\pgfsetlinewidth{1.505625pt}%
\definecolor{currentstroke}{rgb}{0.737255,0.741176,0.133333}%
\pgfsetstrokecolor{currentstroke}%
\pgfsetdash{}{0pt}%
\pgfpathmoveto{\pgfqpoint{4.933237in}{1.121361in}}%
\pgfpathlineto{\pgfqpoint{4.933237in}{1.122570in}}%
\pgfusepath{stroke}%
\end{pgfscope}%
\begin{pgfscope}%
\pgfpathrectangle{\pgfqpoint{1.286132in}{0.839159in}}{\pgfqpoint{12.053712in}{5.967710in}}%
\pgfusepath{clip}%
\pgfsetbuttcap%
\pgfsetroundjoin%
\pgfsetlinewidth{1.505625pt}%
\definecolor{currentstroke}{rgb}{0.737255,0.741176,0.133333}%
\pgfsetstrokecolor{currentstroke}%
\pgfsetdash{}{0pt}%
\pgfpathmoveto{\pgfqpoint{5.043923in}{1.121235in}}%
\pgfpathlineto{\pgfqpoint{5.043923in}{1.122701in}}%
\pgfusepath{stroke}%
\end{pgfscope}%
\begin{pgfscope}%
\pgfpathrectangle{\pgfqpoint{1.286132in}{0.839159in}}{\pgfqpoint{12.053712in}{5.967710in}}%
\pgfusepath{clip}%
\pgfsetbuttcap%
\pgfsetroundjoin%
\pgfsetlinewidth{1.505625pt}%
\definecolor{currentstroke}{rgb}{0.737255,0.741176,0.133333}%
\pgfsetstrokecolor{currentstroke}%
\pgfsetdash{}{0pt}%
\pgfpathmoveto{\pgfqpoint{5.154609in}{1.121338in}}%
\pgfpathlineto{\pgfqpoint{5.154609in}{1.122036in}}%
\pgfusepath{stroke}%
\end{pgfscope}%
\begin{pgfscope}%
\pgfpathrectangle{\pgfqpoint{1.286132in}{0.839159in}}{\pgfqpoint{12.053712in}{5.967710in}}%
\pgfusepath{clip}%
\pgfsetbuttcap%
\pgfsetroundjoin%
\pgfsetlinewidth{1.505625pt}%
\definecolor{currentstroke}{rgb}{0.737255,0.741176,0.133333}%
\pgfsetstrokecolor{currentstroke}%
\pgfsetdash{}{0pt}%
\pgfpathmoveto{\pgfqpoint{5.265296in}{1.121399in}}%
\pgfpathlineto{\pgfqpoint{5.265296in}{1.122126in}}%
\pgfusepath{stroke}%
\end{pgfscope}%
\begin{pgfscope}%
\pgfpathrectangle{\pgfqpoint{1.286132in}{0.839159in}}{\pgfqpoint{12.053712in}{5.967710in}}%
\pgfusepath{clip}%
\pgfsetbuttcap%
\pgfsetroundjoin%
\pgfsetlinewidth{1.505625pt}%
\definecolor{currentstroke}{rgb}{0.737255,0.741176,0.133333}%
\pgfsetstrokecolor{currentstroke}%
\pgfsetdash{}{0pt}%
\pgfpathmoveto{\pgfqpoint{5.375982in}{1.121631in}}%
\pgfpathlineto{\pgfqpoint{5.375982in}{1.123267in}}%
\pgfusepath{stroke}%
\end{pgfscope}%
\begin{pgfscope}%
\pgfpathrectangle{\pgfqpoint{1.286132in}{0.839159in}}{\pgfqpoint{12.053712in}{5.967710in}}%
\pgfusepath{clip}%
\pgfsetbuttcap%
\pgfsetroundjoin%
\pgfsetlinewidth{1.505625pt}%
\definecolor{currentstroke}{rgb}{0.737255,0.741176,0.133333}%
\pgfsetstrokecolor{currentstroke}%
\pgfsetdash{}{0pt}%
\pgfpathmoveto{\pgfqpoint{5.486668in}{1.120176in}}%
\pgfpathlineto{\pgfqpoint{5.486668in}{1.133639in}}%
\pgfusepath{stroke}%
\end{pgfscope}%
\begin{pgfscope}%
\pgfpathrectangle{\pgfqpoint{1.286132in}{0.839159in}}{\pgfqpoint{12.053712in}{5.967710in}}%
\pgfusepath{clip}%
\pgfsetbuttcap%
\pgfsetroundjoin%
\pgfsetlinewidth{1.505625pt}%
\definecolor{currentstroke}{rgb}{0.737255,0.741176,0.133333}%
\pgfsetstrokecolor{currentstroke}%
\pgfsetdash{}{0pt}%
\pgfpathmoveto{\pgfqpoint{5.597354in}{1.121611in}}%
\pgfpathlineto{\pgfqpoint{5.597354in}{1.122900in}}%
\pgfusepath{stroke}%
\end{pgfscope}%
\begin{pgfscope}%
\pgfpathrectangle{\pgfqpoint{1.286132in}{0.839159in}}{\pgfqpoint{12.053712in}{5.967710in}}%
\pgfusepath{clip}%
\pgfsetbuttcap%
\pgfsetroundjoin%
\pgfsetlinewidth{1.505625pt}%
\definecolor{currentstroke}{rgb}{0.737255,0.741176,0.133333}%
\pgfsetstrokecolor{currentstroke}%
\pgfsetdash{}{0pt}%
\pgfpathmoveto{\pgfqpoint{5.708040in}{1.122091in}}%
\pgfpathlineto{\pgfqpoint{5.708040in}{1.123780in}}%
\pgfusepath{stroke}%
\end{pgfscope}%
\begin{pgfscope}%
\pgfpathrectangle{\pgfqpoint{1.286132in}{0.839159in}}{\pgfqpoint{12.053712in}{5.967710in}}%
\pgfusepath{clip}%
\pgfsetbuttcap%
\pgfsetroundjoin%
\pgfsetlinewidth{1.505625pt}%
\definecolor{currentstroke}{rgb}{0.737255,0.741176,0.133333}%
\pgfsetstrokecolor{currentstroke}%
\pgfsetdash{}{0pt}%
\pgfpathmoveto{\pgfqpoint{5.818726in}{1.121681in}}%
\pgfpathlineto{\pgfqpoint{5.818726in}{1.122855in}}%
\pgfusepath{stroke}%
\end{pgfscope}%
\begin{pgfscope}%
\pgfpathrectangle{\pgfqpoint{1.286132in}{0.839159in}}{\pgfqpoint{12.053712in}{5.967710in}}%
\pgfusepath{clip}%
\pgfsetbuttcap%
\pgfsetroundjoin%
\pgfsetlinewidth{1.505625pt}%
\definecolor{currentstroke}{rgb}{0.737255,0.741176,0.133333}%
\pgfsetstrokecolor{currentstroke}%
\pgfsetdash{}{0pt}%
\pgfpathmoveto{\pgfqpoint{5.929412in}{1.122105in}}%
\pgfpathlineto{\pgfqpoint{5.929412in}{1.123282in}}%
\pgfusepath{stroke}%
\end{pgfscope}%
\begin{pgfscope}%
\pgfpathrectangle{\pgfqpoint{1.286132in}{0.839159in}}{\pgfqpoint{12.053712in}{5.967710in}}%
\pgfusepath{clip}%
\pgfsetbuttcap%
\pgfsetroundjoin%
\pgfsetlinewidth{1.505625pt}%
\definecolor{currentstroke}{rgb}{0.737255,0.741176,0.133333}%
\pgfsetstrokecolor{currentstroke}%
\pgfsetdash{}{0pt}%
\pgfpathmoveto{\pgfqpoint{6.040098in}{1.121151in}}%
\pgfpathlineto{\pgfqpoint{6.040098in}{1.125207in}}%
\pgfusepath{stroke}%
\end{pgfscope}%
\begin{pgfscope}%
\pgfpathrectangle{\pgfqpoint{1.286132in}{0.839159in}}{\pgfqpoint{12.053712in}{5.967710in}}%
\pgfusepath{clip}%
\pgfsetbuttcap%
\pgfsetroundjoin%
\pgfsetlinewidth{1.505625pt}%
\definecolor{currentstroke}{rgb}{0.737255,0.741176,0.133333}%
\pgfsetstrokecolor{currentstroke}%
\pgfsetdash{}{0pt}%
\pgfpathmoveto{\pgfqpoint{6.150784in}{1.122183in}}%
\pgfpathlineto{\pgfqpoint{6.150784in}{1.122812in}}%
\pgfusepath{stroke}%
\end{pgfscope}%
\begin{pgfscope}%
\pgfpathrectangle{\pgfqpoint{1.286132in}{0.839159in}}{\pgfqpoint{12.053712in}{5.967710in}}%
\pgfusepath{clip}%
\pgfsetbuttcap%
\pgfsetroundjoin%
\pgfsetlinewidth{1.505625pt}%
\definecolor{currentstroke}{rgb}{0.737255,0.741176,0.133333}%
\pgfsetstrokecolor{currentstroke}%
\pgfsetdash{}{0pt}%
\pgfpathmoveto{\pgfqpoint{6.261470in}{1.122235in}}%
\pgfpathlineto{\pgfqpoint{6.261470in}{1.123949in}}%
\pgfusepath{stroke}%
\end{pgfscope}%
\begin{pgfscope}%
\pgfpathrectangle{\pgfqpoint{1.286132in}{0.839159in}}{\pgfqpoint{12.053712in}{5.967710in}}%
\pgfusepath{clip}%
\pgfsetbuttcap%
\pgfsetroundjoin%
\pgfsetlinewidth{1.505625pt}%
\definecolor{currentstroke}{rgb}{0.737255,0.741176,0.133333}%
\pgfsetstrokecolor{currentstroke}%
\pgfsetdash{}{0pt}%
\pgfpathmoveto{\pgfqpoint{6.372156in}{1.121369in}}%
\pgfpathlineto{\pgfqpoint{6.372156in}{1.128009in}}%
\pgfusepath{stroke}%
\end{pgfscope}%
\begin{pgfscope}%
\pgfpathrectangle{\pgfqpoint{1.286132in}{0.839159in}}{\pgfqpoint{12.053712in}{5.967710in}}%
\pgfusepath{clip}%
\pgfsetbuttcap%
\pgfsetroundjoin%
\pgfsetlinewidth{1.505625pt}%
\definecolor{currentstroke}{rgb}{0.737255,0.741176,0.133333}%
\pgfsetstrokecolor{currentstroke}%
\pgfsetdash{}{0pt}%
\pgfpathmoveto{\pgfqpoint{6.482842in}{1.122450in}}%
\pgfpathlineto{\pgfqpoint{6.482842in}{1.123885in}}%
\pgfusepath{stroke}%
\end{pgfscope}%
\begin{pgfscope}%
\pgfpathrectangle{\pgfqpoint{1.286132in}{0.839159in}}{\pgfqpoint{12.053712in}{5.967710in}}%
\pgfusepath{clip}%
\pgfsetbuttcap%
\pgfsetroundjoin%
\pgfsetlinewidth{1.505625pt}%
\definecolor{currentstroke}{rgb}{0.737255,0.741176,0.133333}%
\pgfsetstrokecolor{currentstroke}%
\pgfsetdash{}{0pt}%
\pgfpathmoveto{\pgfqpoint{6.593528in}{1.122164in}}%
\pgfpathlineto{\pgfqpoint{6.593528in}{1.125293in}}%
\pgfusepath{stroke}%
\end{pgfscope}%
\begin{pgfscope}%
\pgfpathrectangle{\pgfqpoint{1.286132in}{0.839159in}}{\pgfqpoint{12.053712in}{5.967710in}}%
\pgfusepath{clip}%
\pgfsetbuttcap%
\pgfsetroundjoin%
\pgfsetlinewidth{1.505625pt}%
\definecolor{currentstroke}{rgb}{0.737255,0.741176,0.133333}%
\pgfsetstrokecolor{currentstroke}%
\pgfsetdash{}{0pt}%
\pgfpathmoveto{\pgfqpoint{6.704214in}{1.122255in}}%
\pgfpathlineto{\pgfqpoint{6.704214in}{1.123699in}}%
\pgfusepath{stroke}%
\end{pgfscope}%
\begin{pgfscope}%
\pgfpathrectangle{\pgfqpoint{1.286132in}{0.839159in}}{\pgfqpoint{12.053712in}{5.967710in}}%
\pgfusepath{clip}%
\pgfsetbuttcap%
\pgfsetroundjoin%
\pgfsetlinewidth{1.505625pt}%
\definecolor{currentstroke}{rgb}{0.737255,0.741176,0.133333}%
\pgfsetstrokecolor{currentstroke}%
\pgfsetdash{}{0pt}%
\pgfpathmoveto{\pgfqpoint{6.814900in}{1.122636in}}%
\pgfpathlineto{\pgfqpoint{6.814900in}{1.124032in}}%
\pgfusepath{stroke}%
\end{pgfscope}%
\begin{pgfscope}%
\pgfpathrectangle{\pgfqpoint{1.286132in}{0.839159in}}{\pgfqpoint{12.053712in}{5.967710in}}%
\pgfusepath{clip}%
\pgfsetbuttcap%
\pgfsetroundjoin%
\pgfsetlinewidth{1.505625pt}%
\definecolor{currentstroke}{rgb}{0.737255,0.741176,0.133333}%
\pgfsetstrokecolor{currentstroke}%
\pgfsetdash{}{0pt}%
\pgfpathmoveto{\pgfqpoint{6.925586in}{1.122385in}}%
\pgfpathlineto{\pgfqpoint{6.925586in}{1.123866in}}%
\pgfusepath{stroke}%
\end{pgfscope}%
\begin{pgfscope}%
\pgfpathrectangle{\pgfqpoint{1.286132in}{0.839159in}}{\pgfqpoint{12.053712in}{5.967710in}}%
\pgfusepath{clip}%
\pgfsetbuttcap%
\pgfsetroundjoin%
\pgfsetlinewidth{1.505625pt}%
\definecolor{currentstroke}{rgb}{0.737255,0.741176,0.133333}%
\pgfsetstrokecolor{currentstroke}%
\pgfsetdash{}{0pt}%
\pgfpathmoveto{\pgfqpoint{7.036272in}{1.122639in}}%
\pgfpathlineto{\pgfqpoint{7.036272in}{1.125486in}}%
\pgfusepath{stroke}%
\end{pgfscope}%
\begin{pgfscope}%
\pgfpathrectangle{\pgfqpoint{1.286132in}{0.839159in}}{\pgfqpoint{12.053712in}{5.967710in}}%
\pgfusepath{clip}%
\pgfsetbuttcap%
\pgfsetroundjoin%
\pgfsetlinewidth{1.505625pt}%
\definecolor{currentstroke}{rgb}{0.737255,0.741176,0.133333}%
\pgfsetstrokecolor{currentstroke}%
\pgfsetdash{}{0pt}%
\pgfpathmoveto{\pgfqpoint{7.146959in}{1.122727in}}%
\pgfpathlineto{\pgfqpoint{7.146959in}{1.125117in}}%
\pgfusepath{stroke}%
\end{pgfscope}%
\begin{pgfscope}%
\pgfpathrectangle{\pgfqpoint{1.286132in}{0.839159in}}{\pgfqpoint{12.053712in}{5.967710in}}%
\pgfusepath{clip}%
\pgfsetbuttcap%
\pgfsetroundjoin%
\pgfsetlinewidth{1.505625pt}%
\definecolor{currentstroke}{rgb}{0.737255,0.741176,0.133333}%
\pgfsetstrokecolor{currentstroke}%
\pgfsetdash{}{0pt}%
\pgfpathmoveto{\pgfqpoint{7.257645in}{1.123233in}}%
\pgfpathlineto{\pgfqpoint{7.257645in}{1.125166in}}%
\pgfusepath{stroke}%
\end{pgfscope}%
\begin{pgfscope}%
\pgfpathrectangle{\pgfqpoint{1.286132in}{0.839159in}}{\pgfqpoint{12.053712in}{5.967710in}}%
\pgfusepath{clip}%
\pgfsetbuttcap%
\pgfsetroundjoin%
\pgfsetlinewidth{1.505625pt}%
\definecolor{currentstroke}{rgb}{0.737255,0.741176,0.133333}%
\pgfsetstrokecolor{currentstroke}%
\pgfsetdash{}{0pt}%
\pgfpathmoveto{\pgfqpoint{7.368331in}{1.123459in}}%
\pgfpathlineto{\pgfqpoint{7.368331in}{1.123993in}}%
\pgfusepath{stroke}%
\end{pgfscope}%
\begin{pgfscope}%
\pgfpathrectangle{\pgfqpoint{1.286132in}{0.839159in}}{\pgfqpoint{12.053712in}{5.967710in}}%
\pgfusepath{clip}%
\pgfsetbuttcap%
\pgfsetroundjoin%
\pgfsetlinewidth{1.505625pt}%
\definecolor{currentstroke}{rgb}{0.737255,0.741176,0.133333}%
\pgfsetstrokecolor{currentstroke}%
\pgfsetdash{}{0pt}%
\pgfpathmoveto{\pgfqpoint{7.479017in}{1.123015in}}%
\pgfpathlineto{\pgfqpoint{7.479017in}{1.125129in}}%
\pgfusepath{stroke}%
\end{pgfscope}%
\begin{pgfscope}%
\pgfpathrectangle{\pgfqpoint{1.286132in}{0.839159in}}{\pgfqpoint{12.053712in}{5.967710in}}%
\pgfusepath{clip}%
\pgfsetbuttcap%
\pgfsetroundjoin%
\pgfsetlinewidth{1.505625pt}%
\definecolor{currentstroke}{rgb}{0.737255,0.741176,0.133333}%
\pgfsetstrokecolor{currentstroke}%
\pgfsetdash{}{0pt}%
\pgfpathmoveto{\pgfqpoint{7.589703in}{1.124363in}}%
\pgfpathlineto{\pgfqpoint{7.589703in}{1.126334in}}%
\pgfusepath{stroke}%
\end{pgfscope}%
\begin{pgfscope}%
\pgfpathrectangle{\pgfqpoint{1.286132in}{0.839159in}}{\pgfqpoint{12.053712in}{5.967710in}}%
\pgfusepath{clip}%
\pgfsetbuttcap%
\pgfsetroundjoin%
\pgfsetlinewidth{1.505625pt}%
\definecolor{currentstroke}{rgb}{0.737255,0.741176,0.133333}%
\pgfsetstrokecolor{currentstroke}%
\pgfsetdash{}{0pt}%
\pgfpathmoveto{\pgfqpoint{7.700389in}{1.122651in}}%
\pgfpathlineto{\pgfqpoint{7.700389in}{1.128227in}}%
\pgfusepath{stroke}%
\end{pgfscope}%
\begin{pgfscope}%
\pgfpathrectangle{\pgfqpoint{1.286132in}{0.839159in}}{\pgfqpoint{12.053712in}{5.967710in}}%
\pgfusepath{clip}%
\pgfsetbuttcap%
\pgfsetroundjoin%
\pgfsetlinewidth{1.505625pt}%
\definecolor{currentstroke}{rgb}{0.737255,0.741176,0.133333}%
\pgfsetstrokecolor{currentstroke}%
\pgfsetdash{}{0pt}%
\pgfpathmoveto{\pgfqpoint{7.811075in}{1.124405in}}%
\pgfpathlineto{\pgfqpoint{7.811075in}{1.126352in}}%
\pgfusepath{stroke}%
\end{pgfscope}%
\begin{pgfscope}%
\pgfpathrectangle{\pgfqpoint{1.286132in}{0.839159in}}{\pgfqpoint{12.053712in}{5.967710in}}%
\pgfusepath{clip}%
\pgfsetbuttcap%
\pgfsetroundjoin%
\pgfsetlinewidth{1.505625pt}%
\definecolor{currentstroke}{rgb}{0.737255,0.741176,0.133333}%
\pgfsetstrokecolor{currentstroke}%
\pgfsetdash{}{0pt}%
\pgfpathmoveto{\pgfqpoint{7.921761in}{1.123256in}}%
\pgfpathlineto{\pgfqpoint{7.921761in}{1.124977in}}%
\pgfusepath{stroke}%
\end{pgfscope}%
\begin{pgfscope}%
\pgfpathrectangle{\pgfqpoint{1.286132in}{0.839159in}}{\pgfqpoint{12.053712in}{5.967710in}}%
\pgfusepath{clip}%
\pgfsetbuttcap%
\pgfsetroundjoin%
\pgfsetlinewidth{1.505625pt}%
\definecolor{currentstroke}{rgb}{0.737255,0.741176,0.133333}%
\pgfsetstrokecolor{currentstroke}%
\pgfsetdash{}{0pt}%
\pgfpathmoveto{\pgfqpoint{8.032447in}{1.124101in}}%
\pgfpathlineto{\pgfqpoint{8.032447in}{1.126868in}}%
\pgfusepath{stroke}%
\end{pgfscope}%
\begin{pgfscope}%
\pgfpathrectangle{\pgfqpoint{1.286132in}{0.839159in}}{\pgfqpoint{12.053712in}{5.967710in}}%
\pgfusepath{clip}%
\pgfsetbuttcap%
\pgfsetroundjoin%
\pgfsetlinewidth{1.505625pt}%
\definecolor{currentstroke}{rgb}{0.737255,0.741176,0.133333}%
\pgfsetstrokecolor{currentstroke}%
\pgfsetdash{}{0pt}%
\pgfpathmoveto{\pgfqpoint{8.143133in}{1.124202in}}%
\pgfpathlineto{\pgfqpoint{8.143133in}{1.125318in}}%
\pgfusepath{stroke}%
\end{pgfscope}%
\begin{pgfscope}%
\pgfpathrectangle{\pgfqpoint{1.286132in}{0.839159in}}{\pgfqpoint{12.053712in}{5.967710in}}%
\pgfusepath{clip}%
\pgfsetbuttcap%
\pgfsetroundjoin%
\pgfsetlinewidth{1.505625pt}%
\definecolor{currentstroke}{rgb}{0.737255,0.741176,0.133333}%
\pgfsetstrokecolor{currentstroke}%
\pgfsetdash{}{0pt}%
\pgfpathmoveto{\pgfqpoint{8.253819in}{1.124273in}}%
\pgfpathlineto{\pgfqpoint{8.253819in}{1.126243in}}%
\pgfusepath{stroke}%
\end{pgfscope}%
\begin{pgfscope}%
\pgfpathrectangle{\pgfqpoint{1.286132in}{0.839159in}}{\pgfqpoint{12.053712in}{5.967710in}}%
\pgfusepath{clip}%
\pgfsetbuttcap%
\pgfsetroundjoin%
\pgfsetlinewidth{1.505625pt}%
\definecolor{currentstroke}{rgb}{0.737255,0.741176,0.133333}%
\pgfsetstrokecolor{currentstroke}%
\pgfsetdash{}{0pt}%
\pgfpathmoveto{\pgfqpoint{8.364505in}{1.124314in}}%
\pgfpathlineto{\pgfqpoint{8.364505in}{1.126500in}}%
\pgfusepath{stroke}%
\end{pgfscope}%
\begin{pgfscope}%
\pgfpathrectangle{\pgfqpoint{1.286132in}{0.839159in}}{\pgfqpoint{12.053712in}{5.967710in}}%
\pgfusepath{clip}%
\pgfsetbuttcap%
\pgfsetroundjoin%
\pgfsetlinewidth{1.505625pt}%
\definecolor{currentstroke}{rgb}{0.737255,0.741176,0.133333}%
\pgfsetstrokecolor{currentstroke}%
\pgfsetdash{}{0pt}%
\pgfpathmoveto{\pgfqpoint{8.475191in}{1.124979in}}%
\pgfpathlineto{\pgfqpoint{8.475191in}{1.127130in}}%
\pgfusepath{stroke}%
\end{pgfscope}%
\begin{pgfscope}%
\pgfpathrectangle{\pgfqpoint{1.286132in}{0.839159in}}{\pgfqpoint{12.053712in}{5.967710in}}%
\pgfusepath{clip}%
\pgfsetbuttcap%
\pgfsetroundjoin%
\pgfsetlinewidth{1.505625pt}%
\definecolor{currentstroke}{rgb}{0.737255,0.741176,0.133333}%
\pgfsetstrokecolor{currentstroke}%
\pgfsetdash{}{0pt}%
\pgfpathmoveto{\pgfqpoint{8.585877in}{1.123838in}}%
\pgfpathlineto{\pgfqpoint{8.585877in}{1.128755in}}%
\pgfusepath{stroke}%
\end{pgfscope}%
\begin{pgfscope}%
\pgfpathrectangle{\pgfqpoint{1.286132in}{0.839159in}}{\pgfqpoint{12.053712in}{5.967710in}}%
\pgfusepath{clip}%
\pgfsetbuttcap%
\pgfsetroundjoin%
\pgfsetlinewidth{1.505625pt}%
\definecolor{currentstroke}{rgb}{0.737255,0.741176,0.133333}%
\pgfsetstrokecolor{currentstroke}%
\pgfsetdash{}{0pt}%
\pgfpathmoveto{\pgfqpoint{8.696563in}{1.123291in}}%
\pgfpathlineto{\pgfqpoint{8.696563in}{1.129638in}}%
\pgfusepath{stroke}%
\end{pgfscope}%
\begin{pgfscope}%
\pgfpathrectangle{\pgfqpoint{1.286132in}{0.839159in}}{\pgfqpoint{12.053712in}{5.967710in}}%
\pgfusepath{clip}%
\pgfsetbuttcap%
\pgfsetroundjoin%
\pgfsetlinewidth{1.505625pt}%
\definecolor{currentstroke}{rgb}{0.737255,0.741176,0.133333}%
\pgfsetstrokecolor{currentstroke}%
\pgfsetdash{}{0pt}%
\pgfpathmoveto{\pgfqpoint{8.807249in}{1.110419in}}%
\pgfpathlineto{\pgfqpoint{8.807249in}{1.170369in}}%
\pgfusepath{stroke}%
\end{pgfscope}%
\begin{pgfscope}%
\pgfpathrectangle{\pgfqpoint{1.286132in}{0.839159in}}{\pgfqpoint{12.053712in}{5.967710in}}%
\pgfusepath{clip}%
\pgfsetbuttcap%
\pgfsetroundjoin%
\pgfsetlinewidth{1.505625pt}%
\definecolor{currentstroke}{rgb}{0.737255,0.741176,0.133333}%
\pgfsetstrokecolor{currentstroke}%
\pgfsetdash{}{0pt}%
\pgfpathmoveto{\pgfqpoint{8.917936in}{1.124693in}}%
\pgfpathlineto{\pgfqpoint{8.917936in}{1.124969in}}%
\pgfusepath{stroke}%
\end{pgfscope}%
\begin{pgfscope}%
\pgfpathrectangle{\pgfqpoint{1.286132in}{0.839159in}}{\pgfqpoint{12.053712in}{5.967710in}}%
\pgfusepath{clip}%
\pgfsetbuttcap%
\pgfsetroundjoin%
\pgfsetlinewidth{1.505625pt}%
\definecolor{currentstroke}{rgb}{0.737255,0.741176,0.133333}%
\pgfsetstrokecolor{currentstroke}%
\pgfsetdash{}{0pt}%
\pgfpathmoveto{\pgfqpoint{9.028622in}{1.124906in}}%
\pgfpathlineto{\pgfqpoint{9.028622in}{1.127752in}}%
\pgfusepath{stroke}%
\end{pgfscope}%
\begin{pgfscope}%
\pgfpathrectangle{\pgfqpoint{1.286132in}{0.839159in}}{\pgfqpoint{12.053712in}{5.967710in}}%
\pgfusepath{clip}%
\pgfsetbuttcap%
\pgfsetroundjoin%
\pgfsetlinewidth{1.505625pt}%
\definecolor{currentstroke}{rgb}{0.737255,0.741176,0.133333}%
\pgfsetstrokecolor{currentstroke}%
\pgfsetdash{}{0pt}%
\pgfpathmoveto{\pgfqpoint{9.139308in}{1.124297in}}%
\pgfpathlineto{\pgfqpoint{9.139308in}{1.127515in}}%
\pgfusepath{stroke}%
\end{pgfscope}%
\begin{pgfscope}%
\pgfpathrectangle{\pgfqpoint{1.286132in}{0.839159in}}{\pgfqpoint{12.053712in}{5.967710in}}%
\pgfusepath{clip}%
\pgfsetbuttcap%
\pgfsetroundjoin%
\pgfsetlinewidth{1.505625pt}%
\definecolor{currentstroke}{rgb}{0.737255,0.741176,0.133333}%
\pgfsetstrokecolor{currentstroke}%
\pgfsetdash{}{0pt}%
\pgfpathmoveto{\pgfqpoint{9.249994in}{1.124910in}}%
\pgfpathlineto{\pgfqpoint{9.249994in}{1.126303in}}%
\pgfusepath{stroke}%
\end{pgfscope}%
\begin{pgfscope}%
\pgfpathrectangle{\pgfqpoint{1.286132in}{0.839159in}}{\pgfqpoint{12.053712in}{5.967710in}}%
\pgfusepath{clip}%
\pgfsetbuttcap%
\pgfsetroundjoin%
\pgfsetlinewidth{1.505625pt}%
\definecolor{currentstroke}{rgb}{0.737255,0.741176,0.133333}%
\pgfsetstrokecolor{currentstroke}%
\pgfsetdash{}{0pt}%
\pgfpathmoveto{\pgfqpoint{9.360680in}{1.125014in}}%
\pgfpathlineto{\pgfqpoint{9.360680in}{1.126406in}}%
\pgfusepath{stroke}%
\end{pgfscope}%
\begin{pgfscope}%
\pgfpathrectangle{\pgfqpoint{1.286132in}{0.839159in}}{\pgfqpoint{12.053712in}{5.967710in}}%
\pgfusepath{clip}%
\pgfsetbuttcap%
\pgfsetroundjoin%
\pgfsetlinewidth{1.505625pt}%
\definecolor{currentstroke}{rgb}{0.737255,0.741176,0.133333}%
\pgfsetstrokecolor{currentstroke}%
\pgfsetdash{}{0pt}%
\pgfpathmoveto{\pgfqpoint{9.471366in}{1.123620in}}%
\pgfpathlineto{\pgfqpoint{9.471366in}{1.131467in}}%
\pgfusepath{stroke}%
\end{pgfscope}%
\begin{pgfscope}%
\pgfpathrectangle{\pgfqpoint{1.286132in}{0.839159in}}{\pgfqpoint{12.053712in}{5.967710in}}%
\pgfusepath{clip}%
\pgfsetbuttcap%
\pgfsetroundjoin%
\pgfsetlinewidth{1.505625pt}%
\definecolor{currentstroke}{rgb}{0.737255,0.741176,0.133333}%
\pgfsetstrokecolor{currentstroke}%
\pgfsetdash{}{0pt}%
\pgfpathmoveto{\pgfqpoint{9.582052in}{1.123807in}}%
\pgfpathlineto{\pgfqpoint{9.582052in}{1.128087in}}%
\pgfusepath{stroke}%
\end{pgfscope}%
\begin{pgfscope}%
\pgfpathrectangle{\pgfqpoint{1.286132in}{0.839159in}}{\pgfqpoint{12.053712in}{5.967710in}}%
\pgfusepath{clip}%
\pgfsetbuttcap%
\pgfsetroundjoin%
\pgfsetlinewidth{1.505625pt}%
\definecolor{currentstroke}{rgb}{0.737255,0.741176,0.133333}%
\pgfsetstrokecolor{currentstroke}%
\pgfsetdash{}{0pt}%
\pgfpathmoveto{\pgfqpoint{9.692738in}{1.124385in}}%
\pgfpathlineto{\pgfqpoint{9.692738in}{1.128256in}}%
\pgfusepath{stroke}%
\end{pgfscope}%
\begin{pgfscope}%
\pgfpathrectangle{\pgfqpoint{1.286132in}{0.839159in}}{\pgfqpoint{12.053712in}{5.967710in}}%
\pgfusepath{clip}%
\pgfsetbuttcap%
\pgfsetroundjoin%
\pgfsetlinewidth{1.505625pt}%
\definecolor{currentstroke}{rgb}{0.737255,0.741176,0.133333}%
\pgfsetstrokecolor{currentstroke}%
\pgfsetdash{}{0pt}%
\pgfpathmoveto{\pgfqpoint{9.803424in}{1.124613in}}%
\pgfpathlineto{\pgfqpoint{9.803424in}{1.127671in}}%
\pgfusepath{stroke}%
\end{pgfscope}%
\begin{pgfscope}%
\pgfpathrectangle{\pgfqpoint{1.286132in}{0.839159in}}{\pgfqpoint{12.053712in}{5.967710in}}%
\pgfusepath{clip}%
\pgfsetbuttcap%
\pgfsetroundjoin%
\pgfsetlinewidth{1.505625pt}%
\definecolor{currentstroke}{rgb}{0.737255,0.741176,0.133333}%
\pgfsetstrokecolor{currentstroke}%
\pgfsetdash{}{0pt}%
\pgfpathmoveto{\pgfqpoint{9.914110in}{1.125112in}}%
\pgfpathlineto{\pgfqpoint{9.914110in}{1.127472in}}%
\pgfusepath{stroke}%
\end{pgfscope}%
\begin{pgfscope}%
\pgfpathrectangle{\pgfqpoint{1.286132in}{0.839159in}}{\pgfqpoint{12.053712in}{5.967710in}}%
\pgfusepath{clip}%
\pgfsetbuttcap%
\pgfsetroundjoin%
\pgfsetlinewidth{1.505625pt}%
\definecolor{currentstroke}{rgb}{0.737255,0.741176,0.133333}%
\pgfsetstrokecolor{currentstroke}%
\pgfsetdash{}{0pt}%
\pgfpathmoveto{\pgfqpoint{10.024796in}{1.125694in}}%
\pgfpathlineto{\pgfqpoint{10.024796in}{1.129438in}}%
\pgfusepath{stroke}%
\end{pgfscope}%
\begin{pgfscope}%
\pgfpathrectangle{\pgfqpoint{1.286132in}{0.839159in}}{\pgfqpoint{12.053712in}{5.967710in}}%
\pgfusepath{clip}%
\pgfsetbuttcap%
\pgfsetroundjoin%
\pgfsetlinewidth{1.505625pt}%
\definecolor{currentstroke}{rgb}{0.737255,0.741176,0.133333}%
\pgfsetstrokecolor{currentstroke}%
\pgfsetdash{}{0pt}%
\pgfpathmoveto{\pgfqpoint{10.135482in}{1.125645in}}%
\pgfpathlineto{\pgfqpoint{10.135482in}{1.127144in}}%
\pgfusepath{stroke}%
\end{pgfscope}%
\begin{pgfscope}%
\pgfpathrectangle{\pgfqpoint{1.286132in}{0.839159in}}{\pgfqpoint{12.053712in}{5.967710in}}%
\pgfusepath{clip}%
\pgfsetbuttcap%
\pgfsetroundjoin%
\pgfsetlinewidth{1.505625pt}%
\definecolor{currentstroke}{rgb}{0.737255,0.741176,0.133333}%
\pgfsetstrokecolor{currentstroke}%
\pgfsetdash{}{0pt}%
\pgfpathmoveto{\pgfqpoint{10.246168in}{1.125789in}}%
\pgfpathlineto{\pgfqpoint{10.246168in}{1.127203in}}%
\pgfusepath{stroke}%
\end{pgfscope}%
\begin{pgfscope}%
\pgfpathrectangle{\pgfqpoint{1.286132in}{0.839159in}}{\pgfqpoint{12.053712in}{5.967710in}}%
\pgfusepath{clip}%
\pgfsetbuttcap%
\pgfsetroundjoin%
\pgfsetlinewidth{1.505625pt}%
\definecolor{currentstroke}{rgb}{0.737255,0.741176,0.133333}%
\pgfsetstrokecolor{currentstroke}%
\pgfsetdash{}{0pt}%
\pgfpathmoveto{\pgfqpoint{10.356854in}{1.126762in}}%
\pgfpathlineto{\pgfqpoint{10.356854in}{1.128570in}}%
\pgfusepath{stroke}%
\end{pgfscope}%
\begin{pgfscope}%
\pgfpathrectangle{\pgfqpoint{1.286132in}{0.839159in}}{\pgfqpoint{12.053712in}{5.967710in}}%
\pgfusepath{clip}%
\pgfsetbuttcap%
\pgfsetroundjoin%
\pgfsetlinewidth{1.505625pt}%
\definecolor{currentstroke}{rgb}{0.737255,0.741176,0.133333}%
\pgfsetstrokecolor{currentstroke}%
\pgfsetdash{}{0pt}%
\pgfpathmoveto{\pgfqpoint{10.467540in}{1.124793in}}%
\pgfpathlineto{\pgfqpoint{10.467540in}{1.128943in}}%
\pgfusepath{stroke}%
\end{pgfscope}%
\begin{pgfscope}%
\pgfpathrectangle{\pgfqpoint{1.286132in}{0.839159in}}{\pgfqpoint{12.053712in}{5.967710in}}%
\pgfusepath{clip}%
\pgfsetbuttcap%
\pgfsetroundjoin%
\pgfsetlinewidth{1.505625pt}%
\definecolor{currentstroke}{rgb}{0.737255,0.741176,0.133333}%
\pgfsetstrokecolor{currentstroke}%
\pgfsetdash{}{0pt}%
\pgfpathmoveto{\pgfqpoint{10.578226in}{1.126027in}}%
\pgfpathlineto{\pgfqpoint{10.578226in}{1.129285in}}%
\pgfusepath{stroke}%
\end{pgfscope}%
\begin{pgfscope}%
\pgfpathrectangle{\pgfqpoint{1.286132in}{0.839159in}}{\pgfqpoint{12.053712in}{5.967710in}}%
\pgfusepath{clip}%
\pgfsetbuttcap%
\pgfsetroundjoin%
\pgfsetlinewidth{1.505625pt}%
\definecolor{currentstroke}{rgb}{0.737255,0.741176,0.133333}%
\pgfsetstrokecolor{currentstroke}%
\pgfsetdash{}{0pt}%
\pgfpathmoveto{\pgfqpoint{10.688913in}{1.126152in}}%
\pgfpathlineto{\pgfqpoint{10.688913in}{1.129995in}}%
\pgfusepath{stroke}%
\end{pgfscope}%
\begin{pgfscope}%
\pgfpathrectangle{\pgfqpoint{1.286132in}{0.839159in}}{\pgfqpoint{12.053712in}{5.967710in}}%
\pgfusepath{clip}%
\pgfsetbuttcap%
\pgfsetroundjoin%
\pgfsetlinewidth{1.505625pt}%
\definecolor{currentstroke}{rgb}{0.737255,0.741176,0.133333}%
\pgfsetstrokecolor{currentstroke}%
\pgfsetdash{}{0pt}%
\pgfpathmoveto{\pgfqpoint{10.799599in}{1.126100in}}%
\pgfpathlineto{\pgfqpoint{10.799599in}{1.127836in}}%
\pgfusepath{stroke}%
\end{pgfscope}%
\begin{pgfscope}%
\pgfpathrectangle{\pgfqpoint{1.286132in}{0.839159in}}{\pgfqpoint{12.053712in}{5.967710in}}%
\pgfusepath{clip}%
\pgfsetbuttcap%
\pgfsetroundjoin%
\pgfsetlinewidth{1.505625pt}%
\definecolor{currentstroke}{rgb}{0.737255,0.741176,0.133333}%
\pgfsetstrokecolor{currentstroke}%
\pgfsetdash{}{0pt}%
\pgfpathmoveto{\pgfqpoint{10.910285in}{1.126866in}}%
\pgfpathlineto{\pgfqpoint{10.910285in}{1.130195in}}%
\pgfusepath{stroke}%
\end{pgfscope}%
\begin{pgfscope}%
\pgfpathrectangle{\pgfqpoint{1.286132in}{0.839159in}}{\pgfqpoint{12.053712in}{5.967710in}}%
\pgfusepath{clip}%
\pgfsetbuttcap%
\pgfsetroundjoin%
\pgfsetlinewidth{1.505625pt}%
\definecolor{currentstroke}{rgb}{0.737255,0.741176,0.133333}%
\pgfsetstrokecolor{currentstroke}%
\pgfsetdash{}{0pt}%
\pgfpathmoveto{\pgfqpoint{11.020971in}{1.125889in}}%
\pgfpathlineto{\pgfqpoint{11.020971in}{1.130646in}}%
\pgfusepath{stroke}%
\end{pgfscope}%
\begin{pgfscope}%
\pgfpathrectangle{\pgfqpoint{1.286132in}{0.839159in}}{\pgfqpoint{12.053712in}{5.967710in}}%
\pgfusepath{clip}%
\pgfsetbuttcap%
\pgfsetroundjoin%
\pgfsetlinewidth{1.505625pt}%
\definecolor{currentstroke}{rgb}{0.737255,0.741176,0.133333}%
\pgfsetstrokecolor{currentstroke}%
\pgfsetdash{}{0pt}%
\pgfpathmoveto{\pgfqpoint{11.131657in}{1.126066in}}%
\pgfpathlineto{\pgfqpoint{11.131657in}{1.129637in}}%
\pgfusepath{stroke}%
\end{pgfscope}%
\begin{pgfscope}%
\pgfpathrectangle{\pgfqpoint{1.286132in}{0.839159in}}{\pgfqpoint{12.053712in}{5.967710in}}%
\pgfusepath{clip}%
\pgfsetbuttcap%
\pgfsetroundjoin%
\pgfsetlinewidth{1.505625pt}%
\definecolor{currentstroke}{rgb}{0.737255,0.741176,0.133333}%
\pgfsetstrokecolor{currentstroke}%
\pgfsetdash{}{0pt}%
\pgfpathmoveto{\pgfqpoint{11.242343in}{1.126340in}}%
\pgfpathlineto{\pgfqpoint{11.242343in}{1.129184in}}%
\pgfusepath{stroke}%
\end{pgfscope}%
\begin{pgfscope}%
\pgfpathrectangle{\pgfqpoint{1.286132in}{0.839159in}}{\pgfqpoint{12.053712in}{5.967710in}}%
\pgfusepath{clip}%
\pgfsetbuttcap%
\pgfsetroundjoin%
\pgfsetlinewidth{1.505625pt}%
\definecolor{currentstroke}{rgb}{0.737255,0.741176,0.133333}%
\pgfsetstrokecolor{currentstroke}%
\pgfsetdash{}{0pt}%
\pgfpathmoveto{\pgfqpoint{11.353029in}{1.126101in}}%
\pgfpathlineto{\pgfqpoint{11.353029in}{1.130472in}}%
\pgfusepath{stroke}%
\end{pgfscope}%
\begin{pgfscope}%
\pgfpathrectangle{\pgfqpoint{1.286132in}{0.839159in}}{\pgfqpoint{12.053712in}{5.967710in}}%
\pgfusepath{clip}%
\pgfsetbuttcap%
\pgfsetroundjoin%
\pgfsetlinewidth{1.505625pt}%
\definecolor{currentstroke}{rgb}{0.737255,0.741176,0.133333}%
\pgfsetstrokecolor{currentstroke}%
\pgfsetdash{}{0pt}%
\pgfpathmoveto{\pgfqpoint{11.463715in}{1.126704in}}%
\pgfpathlineto{\pgfqpoint{11.463715in}{1.128494in}}%
\pgfusepath{stroke}%
\end{pgfscope}%
\begin{pgfscope}%
\pgfpathrectangle{\pgfqpoint{1.286132in}{0.839159in}}{\pgfqpoint{12.053712in}{5.967710in}}%
\pgfusepath{clip}%
\pgfsetbuttcap%
\pgfsetroundjoin%
\pgfsetlinewidth{1.505625pt}%
\definecolor{currentstroke}{rgb}{0.737255,0.741176,0.133333}%
\pgfsetstrokecolor{currentstroke}%
\pgfsetdash{}{0pt}%
\pgfpathmoveto{\pgfqpoint{11.574401in}{1.127662in}}%
\pgfpathlineto{\pgfqpoint{11.574401in}{1.130013in}}%
\pgfusepath{stroke}%
\end{pgfscope}%
\begin{pgfscope}%
\pgfpathrectangle{\pgfqpoint{1.286132in}{0.839159in}}{\pgfqpoint{12.053712in}{5.967710in}}%
\pgfusepath{clip}%
\pgfsetbuttcap%
\pgfsetroundjoin%
\pgfsetlinewidth{1.505625pt}%
\definecolor{currentstroke}{rgb}{0.737255,0.741176,0.133333}%
\pgfsetstrokecolor{currentstroke}%
\pgfsetdash{}{0pt}%
\pgfpathmoveto{\pgfqpoint{11.685087in}{1.126942in}}%
\pgfpathlineto{\pgfqpoint{11.685087in}{1.129693in}}%
\pgfusepath{stroke}%
\end{pgfscope}%
\begin{pgfscope}%
\pgfpathrectangle{\pgfqpoint{1.286132in}{0.839159in}}{\pgfqpoint{12.053712in}{5.967710in}}%
\pgfusepath{clip}%
\pgfsetbuttcap%
\pgfsetroundjoin%
\pgfsetlinewidth{1.505625pt}%
\definecolor{currentstroke}{rgb}{0.737255,0.741176,0.133333}%
\pgfsetstrokecolor{currentstroke}%
\pgfsetdash{}{0pt}%
\pgfpathmoveto{\pgfqpoint{11.795773in}{1.127142in}}%
\pgfpathlineto{\pgfqpoint{11.795773in}{1.127499in}}%
\pgfusepath{stroke}%
\end{pgfscope}%
\begin{pgfscope}%
\pgfpathrectangle{\pgfqpoint{1.286132in}{0.839159in}}{\pgfqpoint{12.053712in}{5.967710in}}%
\pgfusepath{clip}%
\pgfsetbuttcap%
\pgfsetroundjoin%
\pgfsetlinewidth{1.505625pt}%
\definecolor{currentstroke}{rgb}{0.737255,0.741176,0.133333}%
\pgfsetstrokecolor{currentstroke}%
\pgfsetdash{}{0pt}%
\pgfpathmoveto{\pgfqpoint{11.906459in}{1.127436in}}%
\pgfpathlineto{\pgfqpoint{11.906459in}{1.131040in}}%
\pgfusepath{stroke}%
\end{pgfscope}%
\begin{pgfscope}%
\pgfpathrectangle{\pgfqpoint{1.286132in}{0.839159in}}{\pgfqpoint{12.053712in}{5.967710in}}%
\pgfusepath{clip}%
\pgfsetbuttcap%
\pgfsetroundjoin%
\pgfsetlinewidth{1.505625pt}%
\definecolor{currentstroke}{rgb}{0.737255,0.741176,0.133333}%
\pgfsetstrokecolor{currentstroke}%
\pgfsetdash{}{0pt}%
\pgfpathmoveto{\pgfqpoint{12.017145in}{1.128230in}}%
\pgfpathlineto{\pgfqpoint{12.017145in}{1.130660in}}%
\pgfusepath{stroke}%
\end{pgfscope}%
\begin{pgfscope}%
\pgfpathrectangle{\pgfqpoint{1.286132in}{0.839159in}}{\pgfqpoint{12.053712in}{5.967710in}}%
\pgfusepath{clip}%
\pgfsetbuttcap%
\pgfsetroundjoin%
\pgfsetlinewidth{1.505625pt}%
\definecolor{currentstroke}{rgb}{0.737255,0.741176,0.133333}%
\pgfsetstrokecolor{currentstroke}%
\pgfsetdash{}{0pt}%
\pgfpathmoveto{\pgfqpoint{12.127831in}{1.128159in}}%
\pgfpathlineto{\pgfqpoint{12.127831in}{1.131765in}}%
\pgfusepath{stroke}%
\end{pgfscope}%
\begin{pgfscope}%
\pgfpathrectangle{\pgfqpoint{1.286132in}{0.839159in}}{\pgfqpoint{12.053712in}{5.967710in}}%
\pgfusepath{clip}%
\pgfsetbuttcap%
\pgfsetroundjoin%
\pgfsetlinewidth{1.505625pt}%
\definecolor{currentstroke}{rgb}{0.737255,0.741176,0.133333}%
\pgfsetstrokecolor{currentstroke}%
\pgfsetdash{}{0pt}%
\pgfpathmoveto{\pgfqpoint{12.238517in}{1.127095in}}%
\pgfpathlineto{\pgfqpoint{12.238517in}{1.130263in}}%
\pgfusepath{stroke}%
\end{pgfscope}%
\begin{pgfscope}%
\pgfpathrectangle{\pgfqpoint{1.286132in}{0.839159in}}{\pgfqpoint{12.053712in}{5.967710in}}%
\pgfusepath{clip}%
\pgfsetbuttcap%
\pgfsetroundjoin%
\pgfsetlinewidth{1.505625pt}%
\definecolor{currentstroke}{rgb}{0.737255,0.741176,0.133333}%
\pgfsetstrokecolor{currentstroke}%
\pgfsetdash{}{0pt}%
\pgfpathmoveto{\pgfqpoint{12.349203in}{1.127175in}}%
\pgfpathlineto{\pgfqpoint{12.349203in}{1.131066in}}%
\pgfusepath{stroke}%
\end{pgfscope}%
\begin{pgfscope}%
\pgfpathrectangle{\pgfqpoint{1.286132in}{0.839159in}}{\pgfqpoint{12.053712in}{5.967710in}}%
\pgfusepath{clip}%
\pgfsetbuttcap%
\pgfsetroundjoin%
\pgfsetlinewidth{1.505625pt}%
\definecolor{currentstroke}{rgb}{0.737255,0.741176,0.133333}%
\pgfsetstrokecolor{currentstroke}%
\pgfsetdash{}{0pt}%
\pgfpathmoveto{\pgfqpoint{12.459890in}{1.127233in}}%
\pgfpathlineto{\pgfqpoint{12.459890in}{1.130466in}}%
\pgfusepath{stroke}%
\end{pgfscope}%
\begin{pgfscope}%
\pgfpathrectangle{\pgfqpoint{1.286132in}{0.839159in}}{\pgfqpoint{12.053712in}{5.967710in}}%
\pgfusepath{clip}%
\pgfsetbuttcap%
\pgfsetroundjoin%
\pgfsetlinewidth{1.505625pt}%
\definecolor{currentstroke}{rgb}{0.737255,0.741176,0.133333}%
\pgfsetstrokecolor{currentstroke}%
\pgfsetdash{}{0pt}%
\pgfpathmoveto{\pgfqpoint{12.570576in}{1.120408in}}%
\pgfpathlineto{\pgfqpoint{12.570576in}{1.164026in}}%
\pgfusepath{stroke}%
\end{pgfscope}%
\begin{pgfscope}%
\pgfpathrectangle{\pgfqpoint{1.286132in}{0.839159in}}{\pgfqpoint{12.053712in}{5.967710in}}%
\pgfusepath{clip}%
\pgfsetbuttcap%
\pgfsetroundjoin%
\pgfsetlinewidth{1.505625pt}%
\definecolor{currentstroke}{rgb}{0.737255,0.741176,0.133333}%
\pgfsetstrokecolor{currentstroke}%
\pgfsetdash{}{0pt}%
\pgfpathmoveto{\pgfqpoint{12.681262in}{1.122174in}}%
\pgfpathlineto{\pgfqpoint{12.681262in}{1.152971in}}%
\pgfusepath{stroke}%
\end{pgfscope}%
\begin{pgfscope}%
\pgfpathrectangle{\pgfqpoint{1.286132in}{0.839159in}}{\pgfqpoint{12.053712in}{5.967710in}}%
\pgfusepath{clip}%
\pgfsetbuttcap%
\pgfsetroundjoin%
\pgfsetlinewidth{1.505625pt}%
\definecolor{currentstroke}{rgb}{0.737255,0.741176,0.133333}%
\pgfsetstrokecolor{currentstroke}%
\pgfsetdash{}{0pt}%
\pgfpathmoveto{\pgfqpoint{12.791948in}{1.127902in}}%
\pgfpathlineto{\pgfqpoint{12.791948in}{1.133138in}}%
\pgfusepath{stroke}%
\end{pgfscope}%
\begin{pgfscope}%
\pgfpathrectangle{\pgfqpoint{1.286132in}{0.839159in}}{\pgfqpoint{12.053712in}{5.967710in}}%
\pgfusepath{clip}%
\pgfsetbuttcap%
\pgfsetroundjoin%
\pgfsetlinewidth{1.505625pt}%
\definecolor{currentstroke}{rgb}{0.090196,0.745098,0.811765}%
\pgfsetstrokecolor{currentstroke}%
\pgfsetdash{}{0pt}%
\pgfpathmoveto{\pgfqpoint{1.834028in}{1.119210in}}%
\pgfpathlineto{\pgfqpoint{1.834028in}{1.119217in}}%
\pgfusepath{stroke}%
\end{pgfscope}%
\begin{pgfscope}%
\pgfpathrectangle{\pgfqpoint{1.286132in}{0.839159in}}{\pgfqpoint{12.053712in}{5.967710in}}%
\pgfusepath{clip}%
\pgfsetbuttcap%
\pgfsetroundjoin%
\pgfsetlinewidth{1.505625pt}%
\definecolor{currentstroke}{rgb}{0.090196,0.745098,0.811765}%
\pgfsetstrokecolor{currentstroke}%
\pgfsetdash{}{0pt}%
\pgfpathmoveto{\pgfqpoint{1.944714in}{1.119222in}}%
\pgfpathlineto{\pgfqpoint{1.944714in}{1.119225in}}%
\pgfusepath{stroke}%
\end{pgfscope}%
\begin{pgfscope}%
\pgfpathrectangle{\pgfqpoint{1.286132in}{0.839159in}}{\pgfqpoint{12.053712in}{5.967710in}}%
\pgfusepath{clip}%
\pgfsetbuttcap%
\pgfsetroundjoin%
\pgfsetlinewidth{1.505625pt}%
\definecolor{currentstroke}{rgb}{0.090196,0.745098,0.811765}%
\pgfsetstrokecolor{currentstroke}%
\pgfsetdash{}{0pt}%
\pgfpathmoveto{\pgfqpoint{2.055400in}{1.119258in}}%
\pgfpathlineto{\pgfqpoint{2.055400in}{1.119264in}}%
\pgfusepath{stroke}%
\end{pgfscope}%
\begin{pgfscope}%
\pgfpathrectangle{\pgfqpoint{1.286132in}{0.839159in}}{\pgfqpoint{12.053712in}{5.967710in}}%
\pgfusepath{clip}%
\pgfsetbuttcap%
\pgfsetroundjoin%
\pgfsetlinewidth{1.505625pt}%
\definecolor{currentstroke}{rgb}{0.090196,0.745098,0.811765}%
\pgfsetstrokecolor{currentstroke}%
\pgfsetdash{}{0pt}%
\pgfpathmoveto{\pgfqpoint{2.166086in}{1.119316in}}%
\pgfpathlineto{\pgfqpoint{2.166086in}{1.119446in}}%
\pgfusepath{stroke}%
\end{pgfscope}%
\begin{pgfscope}%
\pgfpathrectangle{\pgfqpoint{1.286132in}{0.839159in}}{\pgfqpoint{12.053712in}{5.967710in}}%
\pgfusepath{clip}%
\pgfsetbuttcap%
\pgfsetroundjoin%
\pgfsetlinewidth{1.505625pt}%
\definecolor{currentstroke}{rgb}{0.090196,0.745098,0.811765}%
\pgfsetstrokecolor{currentstroke}%
\pgfsetdash{}{0pt}%
\pgfpathmoveto{\pgfqpoint{2.276772in}{1.119439in}}%
\pgfpathlineto{\pgfqpoint{2.276772in}{1.119700in}}%
\pgfusepath{stroke}%
\end{pgfscope}%
\begin{pgfscope}%
\pgfpathrectangle{\pgfqpoint{1.286132in}{0.839159in}}{\pgfqpoint{12.053712in}{5.967710in}}%
\pgfusepath{clip}%
\pgfsetbuttcap%
\pgfsetroundjoin%
\pgfsetlinewidth{1.505625pt}%
\definecolor{currentstroke}{rgb}{0.090196,0.745098,0.811765}%
\pgfsetstrokecolor{currentstroke}%
\pgfsetdash{}{0pt}%
\pgfpathmoveto{\pgfqpoint{2.387458in}{1.119706in}}%
\pgfpathlineto{\pgfqpoint{2.387458in}{1.120095in}}%
\pgfusepath{stroke}%
\end{pgfscope}%
\begin{pgfscope}%
\pgfpathrectangle{\pgfqpoint{1.286132in}{0.839159in}}{\pgfqpoint{12.053712in}{5.967710in}}%
\pgfusepath{clip}%
\pgfsetbuttcap%
\pgfsetroundjoin%
\pgfsetlinewidth{1.505625pt}%
\definecolor{currentstroke}{rgb}{0.090196,0.745098,0.811765}%
\pgfsetstrokecolor{currentstroke}%
\pgfsetdash{}{0pt}%
\pgfpathmoveto{\pgfqpoint{2.498144in}{1.119929in}}%
\pgfpathlineto{\pgfqpoint{2.498144in}{1.120241in}}%
\pgfusepath{stroke}%
\end{pgfscope}%
\begin{pgfscope}%
\pgfpathrectangle{\pgfqpoint{1.286132in}{0.839159in}}{\pgfqpoint{12.053712in}{5.967710in}}%
\pgfusepath{clip}%
\pgfsetbuttcap%
\pgfsetroundjoin%
\pgfsetlinewidth{1.505625pt}%
\definecolor{currentstroke}{rgb}{0.090196,0.745098,0.811765}%
\pgfsetstrokecolor{currentstroke}%
\pgfsetdash{}{0pt}%
\pgfpathmoveto{\pgfqpoint{2.608830in}{1.120094in}}%
\pgfpathlineto{\pgfqpoint{2.608830in}{1.120468in}}%
\pgfusepath{stroke}%
\end{pgfscope}%
\begin{pgfscope}%
\pgfpathrectangle{\pgfqpoint{1.286132in}{0.839159in}}{\pgfqpoint{12.053712in}{5.967710in}}%
\pgfusepath{clip}%
\pgfsetbuttcap%
\pgfsetroundjoin%
\pgfsetlinewidth{1.505625pt}%
\definecolor{currentstroke}{rgb}{0.090196,0.745098,0.811765}%
\pgfsetstrokecolor{currentstroke}%
\pgfsetdash{}{0pt}%
\pgfpathmoveto{\pgfqpoint{2.719516in}{1.120523in}}%
\pgfpathlineto{\pgfqpoint{2.719516in}{1.121127in}}%
\pgfusepath{stroke}%
\end{pgfscope}%
\begin{pgfscope}%
\pgfpathrectangle{\pgfqpoint{1.286132in}{0.839159in}}{\pgfqpoint{12.053712in}{5.967710in}}%
\pgfusepath{clip}%
\pgfsetbuttcap%
\pgfsetroundjoin%
\pgfsetlinewidth{1.505625pt}%
\definecolor{currentstroke}{rgb}{0.090196,0.745098,0.811765}%
\pgfsetstrokecolor{currentstroke}%
\pgfsetdash{}{0pt}%
\pgfpathmoveto{\pgfqpoint{2.830202in}{1.121041in}}%
\pgfpathlineto{\pgfqpoint{2.830202in}{1.122572in}}%
\pgfusepath{stroke}%
\end{pgfscope}%
\begin{pgfscope}%
\pgfpathrectangle{\pgfqpoint{1.286132in}{0.839159in}}{\pgfqpoint{12.053712in}{5.967710in}}%
\pgfusepath{clip}%
\pgfsetbuttcap%
\pgfsetroundjoin%
\pgfsetlinewidth{1.505625pt}%
\definecolor{currentstroke}{rgb}{0.090196,0.745098,0.811765}%
\pgfsetstrokecolor{currentstroke}%
\pgfsetdash{}{0pt}%
\pgfpathmoveto{\pgfqpoint{2.940888in}{1.121342in}}%
\pgfpathlineto{\pgfqpoint{2.940888in}{1.123105in}}%
\pgfusepath{stroke}%
\end{pgfscope}%
\begin{pgfscope}%
\pgfpathrectangle{\pgfqpoint{1.286132in}{0.839159in}}{\pgfqpoint{12.053712in}{5.967710in}}%
\pgfusepath{clip}%
\pgfsetbuttcap%
\pgfsetroundjoin%
\pgfsetlinewidth{1.505625pt}%
\definecolor{currentstroke}{rgb}{0.090196,0.745098,0.811765}%
\pgfsetstrokecolor{currentstroke}%
\pgfsetdash{}{0pt}%
\pgfpathmoveto{\pgfqpoint{3.051574in}{1.122370in}}%
\pgfpathlineto{\pgfqpoint{3.051574in}{1.125022in}}%
\pgfusepath{stroke}%
\end{pgfscope}%
\begin{pgfscope}%
\pgfpathrectangle{\pgfqpoint{1.286132in}{0.839159in}}{\pgfqpoint{12.053712in}{5.967710in}}%
\pgfusepath{clip}%
\pgfsetbuttcap%
\pgfsetroundjoin%
\pgfsetlinewidth{1.505625pt}%
\definecolor{currentstroke}{rgb}{0.090196,0.745098,0.811765}%
\pgfsetstrokecolor{currentstroke}%
\pgfsetdash{}{0pt}%
\pgfpathmoveto{\pgfqpoint{3.162260in}{1.123050in}}%
\pgfpathlineto{\pgfqpoint{3.162260in}{1.125654in}}%
\pgfusepath{stroke}%
\end{pgfscope}%
\begin{pgfscope}%
\pgfpathrectangle{\pgfqpoint{1.286132in}{0.839159in}}{\pgfqpoint{12.053712in}{5.967710in}}%
\pgfusepath{clip}%
\pgfsetbuttcap%
\pgfsetroundjoin%
\pgfsetlinewidth{1.505625pt}%
\definecolor{currentstroke}{rgb}{0.090196,0.745098,0.811765}%
\pgfsetstrokecolor{currentstroke}%
\pgfsetdash{}{0pt}%
\pgfpathmoveto{\pgfqpoint{3.272946in}{1.122955in}}%
\pgfpathlineto{\pgfqpoint{3.272946in}{1.124840in}}%
\pgfusepath{stroke}%
\end{pgfscope}%
\begin{pgfscope}%
\pgfpathrectangle{\pgfqpoint{1.286132in}{0.839159in}}{\pgfqpoint{12.053712in}{5.967710in}}%
\pgfusepath{clip}%
\pgfsetbuttcap%
\pgfsetroundjoin%
\pgfsetlinewidth{1.505625pt}%
\definecolor{currentstroke}{rgb}{0.090196,0.745098,0.811765}%
\pgfsetstrokecolor{currentstroke}%
\pgfsetdash{}{0pt}%
\pgfpathmoveto{\pgfqpoint{3.383632in}{1.118966in}}%
\pgfpathlineto{\pgfqpoint{3.383632in}{1.144835in}}%
\pgfusepath{stroke}%
\end{pgfscope}%
\begin{pgfscope}%
\pgfpathrectangle{\pgfqpoint{1.286132in}{0.839159in}}{\pgfqpoint{12.053712in}{5.967710in}}%
\pgfusepath{clip}%
\pgfsetbuttcap%
\pgfsetroundjoin%
\pgfsetlinewidth{1.505625pt}%
\definecolor{currentstroke}{rgb}{0.090196,0.745098,0.811765}%
\pgfsetstrokecolor{currentstroke}%
\pgfsetdash{}{0pt}%
\pgfpathmoveto{\pgfqpoint{3.494319in}{1.125080in}}%
\pgfpathlineto{\pgfqpoint{3.494319in}{1.127852in}}%
\pgfusepath{stroke}%
\end{pgfscope}%
\begin{pgfscope}%
\pgfpathrectangle{\pgfqpoint{1.286132in}{0.839159in}}{\pgfqpoint{12.053712in}{5.967710in}}%
\pgfusepath{clip}%
\pgfsetbuttcap%
\pgfsetroundjoin%
\pgfsetlinewidth{1.505625pt}%
\definecolor{currentstroke}{rgb}{0.090196,0.745098,0.811765}%
\pgfsetstrokecolor{currentstroke}%
\pgfsetdash{}{0pt}%
\pgfpathmoveto{\pgfqpoint{3.605005in}{1.126658in}}%
\pgfpathlineto{\pgfqpoint{3.605005in}{1.132749in}}%
\pgfusepath{stroke}%
\end{pgfscope}%
\begin{pgfscope}%
\pgfpathrectangle{\pgfqpoint{1.286132in}{0.839159in}}{\pgfqpoint{12.053712in}{5.967710in}}%
\pgfusepath{clip}%
\pgfsetbuttcap%
\pgfsetroundjoin%
\pgfsetlinewidth{1.505625pt}%
\definecolor{currentstroke}{rgb}{0.090196,0.745098,0.811765}%
\pgfsetstrokecolor{currentstroke}%
\pgfsetdash{}{0pt}%
\pgfpathmoveto{\pgfqpoint{3.715691in}{1.128122in}}%
\pgfpathlineto{\pgfqpoint{3.715691in}{1.133200in}}%
\pgfusepath{stroke}%
\end{pgfscope}%
\begin{pgfscope}%
\pgfpathrectangle{\pgfqpoint{1.286132in}{0.839159in}}{\pgfqpoint{12.053712in}{5.967710in}}%
\pgfusepath{clip}%
\pgfsetbuttcap%
\pgfsetroundjoin%
\pgfsetlinewidth{1.505625pt}%
\definecolor{currentstroke}{rgb}{0.090196,0.745098,0.811765}%
\pgfsetstrokecolor{currentstroke}%
\pgfsetdash{}{0pt}%
\pgfpathmoveto{\pgfqpoint{3.826377in}{1.128260in}}%
\pgfpathlineto{\pgfqpoint{3.826377in}{1.135955in}}%
\pgfusepath{stroke}%
\end{pgfscope}%
\begin{pgfscope}%
\pgfpathrectangle{\pgfqpoint{1.286132in}{0.839159in}}{\pgfqpoint{12.053712in}{5.967710in}}%
\pgfusepath{clip}%
\pgfsetbuttcap%
\pgfsetroundjoin%
\pgfsetlinewidth{1.505625pt}%
\definecolor{currentstroke}{rgb}{0.090196,0.745098,0.811765}%
\pgfsetstrokecolor{currentstroke}%
\pgfsetdash{}{0pt}%
\pgfpathmoveto{\pgfqpoint{3.937063in}{1.130185in}}%
\pgfpathlineto{\pgfqpoint{3.937063in}{1.134194in}}%
\pgfusepath{stroke}%
\end{pgfscope}%
\begin{pgfscope}%
\pgfpathrectangle{\pgfqpoint{1.286132in}{0.839159in}}{\pgfqpoint{12.053712in}{5.967710in}}%
\pgfusepath{clip}%
\pgfsetbuttcap%
\pgfsetroundjoin%
\pgfsetlinewidth{1.505625pt}%
\definecolor{currentstroke}{rgb}{0.090196,0.745098,0.811765}%
\pgfsetstrokecolor{currentstroke}%
\pgfsetdash{}{0pt}%
\pgfpathmoveto{\pgfqpoint{4.047749in}{1.133810in}}%
\pgfpathlineto{\pgfqpoint{4.047749in}{1.138774in}}%
\pgfusepath{stroke}%
\end{pgfscope}%
\begin{pgfscope}%
\pgfpathrectangle{\pgfqpoint{1.286132in}{0.839159in}}{\pgfqpoint{12.053712in}{5.967710in}}%
\pgfusepath{clip}%
\pgfsetbuttcap%
\pgfsetroundjoin%
\pgfsetlinewidth{1.505625pt}%
\definecolor{currentstroke}{rgb}{0.090196,0.745098,0.811765}%
\pgfsetstrokecolor{currentstroke}%
\pgfsetdash{}{0pt}%
\pgfpathmoveto{\pgfqpoint{4.158435in}{1.126326in}}%
\pgfpathlineto{\pgfqpoint{4.158435in}{1.166078in}}%
\pgfusepath{stroke}%
\end{pgfscope}%
\begin{pgfscope}%
\pgfpathrectangle{\pgfqpoint{1.286132in}{0.839159in}}{\pgfqpoint{12.053712in}{5.967710in}}%
\pgfusepath{clip}%
\pgfsetbuttcap%
\pgfsetroundjoin%
\pgfsetlinewidth{1.505625pt}%
\definecolor{currentstroke}{rgb}{0.090196,0.745098,0.811765}%
\pgfsetstrokecolor{currentstroke}%
\pgfsetdash{}{0pt}%
\pgfpathmoveto{\pgfqpoint{4.269121in}{1.134181in}}%
\pgfpathlineto{\pgfqpoint{4.269121in}{1.144417in}}%
\pgfusepath{stroke}%
\end{pgfscope}%
\begin{pgfscope}%
\pgfpathrectangle{\pgfqpoint{1.286132in}{0.839159in}}{\pgfqpoint{12.053712in}{5.967710in}}%
\pgfusepath{clip}%
\pgfsetbuttcap%
\pgfsetroundjoin%
\pgfsetlinewidth{1.505625pt}%
\definecolor{currentstroke}{rgb}{0.090196,0.745098,0.811765}%
\pgfsetstrokecolor{currentstroke}%
\pgfsetdash{}{0pt}%
\pgfpathmoveto{\pgfqpoint{4.379807in}{1.134041in}}%
\pgfpathlineto{\pgfqpoint{4.379807in}{1.148323in}}%
\pgfusepath{stroke}%
\end{pgfscope}%
\begin{pgfscope}%
\pgfpathrectangle{\pgfqpoint{1.286132in}{0.839159in}}{\pgfqpoint{12.053712in}{5.967710in}}%
\pgfusepath{clip}%
\pgfsetbuttcap%
\pgfsetroundjoin%
\pgfsetlinewidth{1.505625pt}%
\definecolor{currentstroke}{rgb}{0.090196,0.745098,0.811765}%
\pgfsetstrokecolor{currentstroke}%
\pgfsetdash{}{0pt}%
\pgfpathmoveto{\pgfqpoint{4.490493in}{1.137695in}}%
\pgfpathlineto{\pgfqpoint{4.490493in}{1.144833in}}%
\pgfusepath{stroke}%
\end{pgfscope}%
\begin{pgfscope}%
\pgfpathrectangle{\pgfqpoint{1.286132in}{0.839159in}}{\pgfqpoint{12.053712in}{5.967710in}}%
\pgfusepath{clip}%
\pgfsetbuttcap%
\pgfsetroundjoin%
\pgfsetlinewidth{1.505625pt}%
\definecolor{currentstroke}{rgb}{0.090196,0.745098,0.811765}%
\pgfsetstrokecolor{currentstroke}%
\pgfsetdash{}{0pt}%
\pgfpathmoveto{\pgfqpoint{4.601179in}{1.143568in}}%
\pgfpathlineto{\pgfqpoint{4.601179in}{1.157488in}}%
\pgfusepath{stroke}%
\end{pgfscope}%
\begin{pgfscope}%
\pgfpathrectangle{\pgfqpoint{1.286132in}{0.839159in}}{\pgfqpoint{12.053712in}{5.967710in}}%
\pgfusepath{clip}%
\pgfsetbuttcap%
\pgfsetroundjoin%
\pgfsetlinewidth{1.505625pt}%
\definecolor{currentstroke}{rgb}{0.090196,0.745098,0.811765}%
\pgfsetstrokecolor{currentstroke}%
\pgfsetdash{}{0pt}%
\pgfpathmoveto{\pgfqpoint{4.711865in}{1.136391in}}%
\pgfpathlineto{\pgfqpoint{4.711865in}{1.148353in}}%
\pgfusepath{stroke}%
\end{pgfscope}%
\begin{pgfscope}%
\pgfpathrectangle{\pgfqpoint{1.286132in}{0.839159in}}{\pgfqpoint{12.053712in}{5.967710in}}%
\pgfusepath{clip}%
\pgfsetbuttcap%
\pgfsetroundjoin%
\pgfsetlinewidth{1.505625pt}%
\definecolor{currentstroke}{rgb}{0.090196,0.745098,0.811765}%
\pgfsetstrokecolor{currentstroke}%
\pgfsetdash{}{0pt}%
\pgfpathmoveto{\pgfqpoint{4.822551in}{1.144255in}}%
\pgfpathlineto{\pgfqpoint{4.822551in}{1.149494in}}%
\pgfusepath{stroke}%
\end{pgfscope}%
\begin{pgfscope}%
\pgfpathrectangle{\pgfqpoint{1.286132in}{0.839159in}}{\pgfqpoint{12.053712in}{5.967710in}}%
\pgfusepath{clip}%
\pgfsetbuttcap%
\pgfsetroundjoin%
\pgfsetlinewidth{1.505625pt}%
\definecolor{currentstroke}{rgb}{0.090196,0.745098,0.811765}%
\pgfsetstrokecolor{currentstroke}%
\pgfsetdash{}{0pt}%
\pgfpathmoveto{\pgfqpoint{4.933237in}{1.144491in}}%
\pgfpathlineto{\pgfqpoint{4.933237in}{1.156541in}}%
\pgfusepath{stroke}%
\end{pgfscope}%
\begin{pgfscope}%
\pgfpathrectangle{\pgfqpoint{1.286132in}{0.839159in}}{\pgfqpoint{12.053712in}{5.967710in}}%
\pgfusepath{clip}%
\pgfsetbuttcap%
\pgfsetroundjoin%
\pgfsetlinewidth{1.505625pt}%
\definecolor{currentstroke}{rgb}{0.090196,0.745098,0.811765}%
\pgfsetstrokecolor{currentstroke}%
\pgfsetdash{}{0pt}%
\pgfpathmoveto{\pgfqpoint{5.043923in}{1.152540in}}%
\pgfpathlineto{\pgfqpoint{5.043923in}{1.163667in}}%
\pgfusepath{stroke}%
\end{pgfscope}%
\begin{pgfscope}%
\pgfpathrectangle{\pgfqpoint{1.286132in}{0.839159in}}{\pgfqpoint{12.053712in}{5.967710in}}%
\pgfusepath{clip}%
\pgfsetbuttcap%
\pgfsetroundjoin%
\pgfsetlinewidth{1.505625pt}%
\definecolor{currentstroke}{rgb}{0.090196,0.745098,0.811765}%
\pgfsetstrokecolor{currentstroke}%
\pgfsetdash{}{0pt}%
\pgfpathmoveto{\pgfqpoint{5.154609in}{1.157701in}}%
\pgfpathlineto{\pgfqpoint{5.154609in}{1.167642in}}%
\pgfusepath{stroke}%
\end{pgfscope}%
\begin{pgfscope}%
\pgfpathrectangle{\pgfqpoint{1.286132in}{0.839159in}}{\pgfqpoint{12.053712in}{5.967710in}}%
\pgfusepath{clip}%
\pgfsetbuttcap%
\pgfsetroundjoin%
\pgfsetlinewidth{1.505625pt}%
\definecolor{currentstroke}{rgb}{0.090196,0.745098,0.811765}%
\pgfsetstrokecolor{currentstroke}%
\pgfsetdash{}{0pt}%
\pgfpathmoveto{\pgfqpoint{5.265296in}{1.160945in}}%
\pgfpathlineto{\pgfqpoint{5.265296in}{1.170817in}}%
\pgfusepath{stroke}%
\end{pgfscope}%
\begin{pgfscope}%
\pgfpathrectangle{\pgfqpoint{1.286132in}{0.839159in}}{\pgfqpoint{12.053712in}{5.967710in}}%
\pgfusepath{clip}%
\pgfsetbuttcap%
\pgfsetroundjoin%
\pgfsetlinewidth{1.505625pt}%
\definecolor{currentstroke}{rgb}{0.090196,0.745098,0.811765}%
\pgfsetstrokecolor{currentstroke}%
\pgfsetdash{}{0pt}%
\pgfpathmoveto{\pgfqpoint{5.375982in}{1.146125in}}%
\pgfpathlineto{\pgfqpoint{5.375982in}{1.183319in}}%
\pgfusepath{stroke}%
\end{pgfscope}%
\begin{pgfscope}%
\pgfpathrectangle{\pgfqpoint{1.286132in}{0.839159in}}{\pgfqpoint{12.053712in}{5.967710in}}%
\pgfusepath{clip}%
\pgfsetbuttcap%
\pgfsetroundjoin%
\pgfsetlinewidth{1.505625pt}%
\definecolor{currentstroke}{rgb}{0.090196,0.745098,0.811765}%
\pgfsetstrokecolor{currentstroke}%
\pgfsetdash{}{0pt}%
\pgfpathmoveto{\pgfqpoint{5.486668in}{1.179151in}}%
\pgfpathlineto{\pgfqpoint{5.486668in}{1.253118in}}%
\pgfusepath{stroke}%
\end{pgfscope}%
\begin{pgfscope}%
\pgfpathrectangle{\pgfqpoint{1.286132in}{0.839159in}}{\pgfqpoint{12.053712in}{5.967710in}}%
\pgfusepath{clip}%
\pgfsetbuttcap%
\pgfsetroundjoin%
\pgfsetlinewidth{1.505625pt}%
\definecolor{currentstroke}{rgb}{0.090196,0.745098,0.811765}%
\pgfsetstrokecolor{currentstroke}%
\pgfsetdash{}{0pt}%
\pgfpathmoveto{\pgfqpoint{5.597354in}{1.174417in}}%
\pgfpathlineto{\pgfqpoint{5.597354in}{1.187236in}}%
\pgfusepath{stroke}%
\end{pgfscope}%
\begin{pgfscope}%
\pgfpathrectangle{\pgfqpoint{1.286132in}{0.839159in}}{\pgfqpoint{12.053712in}{5.967710in}}%
\pgfusepath{clip}%
\pgfsetbuttcap%
\pgfsetroundjoin%
\pgfsetlinewidth{1.505625pt}%
\definecolor{currentstroke}{rgb}{0.090196,0.745098,0.811765}%
\pgfsetstrokecolor{currentstroke}%
\pgfsetdash{}{0pt}%
\pgfpathmoveto{\pgfqpoint{5.708040in}{1.172496in}}%
\pgfpathlineto{\pgfqpoint{5.708040in}{1.186761in}}%
\pgfusepath{stroke}%
\end{pgfscope}%
\begin{pgfscope}%
\pgfpathrectangle{\pgfqpoint{1.286132in}{0.839159in}}{\pgfqpoint{12.053712in}{5.967710in}}%
\pgfusepath{clip}%
\pgfsetbuttcap%
\pgfsetroundjoin%
\pgfsetlinewidth{1.505625pt}%
\definecolor{currentstroke}{rgb}{0.090196,0.745098,0.811765}%
\pgfsetstrokecolor{currentstroke}%
\pgfsetdash{}{0pt}%
\pgfpathmoveto{\pgfqpoint{5.818726in}{1.175801in}}%
\pgfpathlineto{\pgfqpoint{5.818726in}{1.187435in}}%
\pgfusepath{stroke}%
\end{pgfscope}%
\begin{pgfscope}%
\pgfpathrectangle{\pgfqpoint{1.286132in}{0.839159in}}{\pgfqpoint{12.053712in}{5.967710in}}%
\pgfusepath{clip}%
\pgfsetbuttcap%
\pgfsetroundjoin%
\pgfsetlinewidth{1.505625pt}%
\definecolor{currentstroke}{rgb}{0.090196,0.745098,0.811765}%
\pgfsetstrokecolor{currentstroke}%
\pgfsetdash{}{0pt}%
\pgfpathmoveto{\pgfqpoint{5.929412in}{1.178235in}}%
\pgfpathlineto{\pgfqpoint{5.929412in}{1.189538in}}%
\pgfusepath{stroke}%
\end{pgfscope}%
\begin{pgfscope}%
\pgfpathrectangle{\pgfqpoint{1.286132in}{0.839159in}}{\pgfqpoint{12.053712in}{5.967710in}}%
\pgfusepath{clip}%
\pgfsetbuttcap%
\pgfsetroundjoin%
\pgfsetlinewidth{1.505625pt}%
\definecolor{currentstroke}{rgb}{0.090196,0.745098,0.811765}%
\pgfsetstrokecolor{currentstroke}%
\pgfsetdash{}{0pt}%
\pgfpathmoveto{\pgfqpoint{6.040098in}{1.186814in}}%
\pgfpathlineto{\pgfqpoint{6.040098in}{1.206459in}}%
\pgfusepath{stroke}%
\end{pgfscope}%
\begin{pgfscope}%
\pgfpathrectangle{\pgfqpoint{1.286132in}{0.839159in}}{\pgfqpoint{12.053712in}{5.967710in}}%
\pgfusepath{clip}%
\pgfsetbuttcap%
\pgfsetroundjoin%
\pgfsetlinewidth{1.505625pt}%
\definecolor{currentstroke}{rgb}{0.090196,0.745098,0.811765}%
\pgfsetstrokecolor{currentstroke}%
\pgfsetdash{}{0pt}%
\pgfpathmoveto{\pgfqpoint{6.150784in}{1.185505in}}%
\pgfpathlineto{\pgfqpoint{6.150784in}{1.219524in}}%
\pgfusepath{stroke}%
\end{pgfscope}%
\begin{pgfscope}%
\pgfpathrectangle{\pgfqpoint{1.286132in}{0.839159in}}{\pgfqpoint{12.053712in}{5.967710in}}%
\pgfusepath{clip}%
\pgfsetbuttcap%
\pgfsetroundjoin%
\pgfsetlinewidth{1.505625pt}%
\definecolor{currentstroke}{rgb}{0.090196,0.745098,0.811765}%
\pgfsetstrokecolor{currentstroke}%
\pgfsetdash{}{0pt}%
\pgfpathmoveto{\pgfqpoint{6.261470in}{1.184659in}}%
\pgfpathlineto{\pgfqpoint{6.261470in}{1.212119in}}%
\pgfusepath{stroke}%
\end{pgfscope}%
\begin{pgfscope}%
\pgfpathrectangle{\pgfqpoint{1.286132in}{0.839159in}}{\pgfqpoint{12.053712in}{5.967710in}}%
\pgfusepath{clip}%
\pgfsetbuttcap%
\pgfsetroundjoin%
\pgfsetlinewidth{1.505625pt}%
\definecolor{currentstroke}{rgb}{0.090196,0.745098,0.811765}%
\pgfsetstrokecolor{currentstroke}%
\pgfsetdash{}{0pt}%
\pgfpathmoveto{\pgfqpoint{6.372156in}{1.183110in}}%
\pgfpathlineto{\pgfqpoint{6.372156in}{1.218220in}}%
\pgfusepath{stroke}%
\end{pgfscope}%
\begin{pgfscope}%
\pgfpathrectangle{\pgfqpoint{1.286132in}{0.839159in}}{\pgfqpoint{12.053712in}{5.967710in}}%
\pgfusepath{clip}%
\pgfsetbuttcap%
\pgfsetroundjoin%
\pgfsetlinewidth{1.505625pt}%
\definecolor{currentstroke}{rgb}{0.090196,0.745098,0.811765}%
\pgfsetstrokecolor{currentstroke}%
\pgfsetdash{}{0pt}%
\pgfpathmoveto{\pgfqpoint{6.482842in}{1.192755in}}%
\pgfpathlineto{\pgfqpoint{6.482842in}{1.230540in}}%
\pgfusepath{stroke}%
\end{pgfscope}%
\begin{pgfscope}%
\pgfpathrectangle{\pgfqpoint{1.286132in}{0.839159in}}{\pgfqpoint{12.053712in}{5.967710in}}%
\pgfusepath{clip}%
\pgfsetbuttcap%
\pgfsetroundjoin%
\pgfsetlinewidth{1.505625pt}%
\definecolor{currentstroke}{rgb}{0.090196,0.745098,0.811765}%
\pgfsetstrokecolor{currentstroke}%
\pgfsetdash{}{0pt}%
\pgfpathmoveto{\pgfqpoint{6.593528in}{1.210042in}}%
\pgfpathlineto{\pgfqpoint{6.593528in}{1.268205in}}%
\pgfusepath{stroke}%
\end{pgfscope}%
\begin{pgfscope}%
\pgfpathrectangle{\pgfqpoint{1.286132in}{0.839159in}}{\pgfqpoint{12.053712in}{5.967710in}}%
\pgfusepath{clip}%
\pgfsetbuttcap%
\pgfsetroundjoin%
\pgfsetlinewidth{1.505625pt}%
\definecolor{currentstroke}{rgb}{0.090196,0.745098,0.811765}%
\pgfsetstrokecolor{currentstroke}%
\pgfsetdash{}{0pt}%
\pgfpathmoveto{\pgfqpoint{6.704214in}{1.191899in}}%
\pgfpathlineto{\pgfqpoint{6.704214in}{1.230089in}}%
\pgfusepath{stroke}%
\end{pgfscope}%
\begin{pgfscope}%
\pgfpathrectangle{\pgfqpoint{1.286132in}{0.839159in}}{\pgfqpoint{12.053712in}{5.967710in}}%
\pgfusepath{clip}%
\pgfsetbuttcap%
\pgfsetroundjoin%
\pgfsetlinewidth{1.505625pt}%
\definecolor{currentstroke}{rgb}{0.090196,0.745098,0.811765}%
\pgfsetstrokecolor{currentstroke}%
\pgfsetdash{}{0pt}%
\pgfpathmoveto{\pgfqpoint{6.814900in}{1.200232in}}%
\pgfpathlineto{\pgfqpoint{6.814900in}{1.239520in}}%
\pgfusepath{stroke}%
\end{pgfscope}%
\begin{pgfscope}%
\pgfpathrectangle{\pgfqpoint{1.286132in}{0.839159in}}{\pgfqpoint{12.053712in}{5.967710in}}%
\pgfusepath{clip}%
\pgfsetbuttcap%
\pgfsetroundjoin%
\pgfsetlinewidth{1.505625pt}%
\definecolor{currentstroke}{rgb}{0.090196,0.745098,0.811765}%
\pgfsetstrokecolor{currentstroke}%
\pgfsetdash{}{0pt}%
\pgfpathmoveto{\pgfqpoint{6.925586in}{1.215271in}}%
\pgfpathlineto{\pgfqpoint{6.925586in}{1.234134in}}%
\pgfusepath{stroke}%
\end{pgfscope}%
\begin{pgfscope}%
\pgfpathrectangle{\pgfqpoint{1.286132in}{0.839159in}}{\pgfqpoint{12.053712in}{5.967710in}}%
\pgfusepath{clip}%
\pgfsetbuttcap%
\pgfsetroundjoin%
\pgfsetlinewidth{1.505625pt}%
\definecolor{currentstroke}{rgb}{0.090196,0.745098,0.811765}%
\pgfsetstrokecolor{currentstroke}%
\pgfsetdash{}{0pt}%
\pgfpathmoveto{\pgfqpoint{7.036272in}{1.212326in}}%
\pgfpathlineto{\pgfqpoint{7.036272in}{1.272973in}}%
\pgfusepath{stroke}%
\end{pgfscope}%
\begin{pgfscope}%
\pgfpathrectangle{\pgfqpoint{1.286132in}{0.839159in}}{\pgfqpoint{12.053712in}{5.967710in}}%
\pgfusepath{clip}%
\pgfsetbuttcap%
\pgfsetroundjoin%
\pgfsetlinewidth{1.505625pt}%
\definecolor{currentstroke}{rgb}{0.090196,0.745098,0.811765}%
\pgfsetstrokecolor{currentstroke}%
\pgfsetdash{}{0pt}%
\pgfpathmoveto{\pgfqpoint{7.146959in}{1.240169in}}%
\pgfpathlineto{\pgfqpoint{7.146959in}{1.272466in}}%
\pgfusepath{stroke}%
\end{pgfscope}%
\begin{pgfscope}%
\pgfpathrectangle{\pgfqpoint{1.286132in}{0.839159in}}{\pgfqpoint{12.053712in}{5.967710in}}%
\pgfusepath{clip}%
\pgfsetbuttcap%
\pgfsetroundjoin%
\pgfsetlinewidth{1.505625pt}%
\definecolor{currentstroke}{rgb}{0.090196,0.745098,0.811765}%
\pgfsetstrokecolor{currentstroke}%
\pgfsetdash{}{0pt}%
\pgfpathmoveto{\pgfqpoint{7.257645in}{1.201177in}}%
\pgfpathlineto{\pgfqpoint{7.257645in}{1.296069in}}%
\pgfusepath{stroke}%
\end{pgfscope}%
\begin{pgfscope}%
\pgfpathrectangle{\pgfqpoint{1.286132in}{0.839159in}}{\pgfqpoint{12.053712in}{5.967710in}}%
\pgfusepath{clip}%
\pgfsetbuttcap%
\pgfsetroundjoin%
\pgfsetlinewidth{1.505625pt}%
\definecolor{currentstroke}{rgb}{0.090196,0.745098,0.811765}%
\pgfsetstrokecolor{currentstroke}%
\pgfsetdash{}{0pt}%
\pgfpathmoveto{\pgfqpoint{7.368331in}{1.230211in}}%
\pgfpathlineto{\pgfqpoint{7.368331in}{1.253690in}}%
\pgfusepath{stroke}%
\end{pgfscope}%
\begin{pgfscope}%
\pgfpathrectangle{\pgfqpoint{1.286132in}{0.839159in}}{\pgfqpoint{12.053712in}{5.967710in}}%
\pgfusepath{clip}%
\pgfsetbuttcap%
\pgfsetroundjoin%
\pgfsetlinewidth{1.505625pt}%
\definecolor{currentstroke}{rgb}{0.090196,0.745098,0.811765}%
\pgfsetstrokecolor{currentstroke}%
\pgfsetdash{}{0pt}%
\pgfpathmoveto{\pgfqpoint{7.479017in}{1.243653in}}%
\pgfpathlineto{\pgfqpoint{7.479017in}{1.303980in}}%
\pgfusepath{stroke}%
\end{pgfscope}%
\begin{pgfscope}%
\pgfpathrectangle{\pgfqpoint{1.286132in}{0.839159in}}{\pgfqpoint{12.053712in}{5.967710in}}%
\pgfusepath{clip}%
\pgfsetbuttcap%
\pgfsetroundjoin%
\pgfsetlinewidth{1.505625pt}%
\definecolor{currentstroke}{rgb}{0.090196,0.745098,0.811765}%
\pgfsetstrokecolor{currentstroke}%
\pgfsetdash{}{0pt}%
\pgfpathmoveto{\pgfqpoint{7.589703in}{1.259639in}}%
\pgfpathlineto{\pgfqpoint{7.589703in}{1.313896in}}%
\pgfusepath{stroke}%
\end{pgfscope}%
\begin{pgfscope}%
\pgfpathrectangle{\pgfqpoint{1.286132in}{0.839159in}}{\pgfqpoint{12.053712in}{5.967710in}}%
\pgfusepath{clip}%
\pgfsetbuttcap%
\pgfsetroundjoin%
\pgfsetlinewidth{1.505625pt}%
\definecolor{currentstroke}{rgb}{0.090196,0.745098,0.811765}%
\pgfsetstrokecolor{currentstroke}%
\pgfsetdash{}{0pt}%
\pgfpathmoveto{\pgfqpoint{7.700389in}{1.268875in}}%
\pgfpathlineto{\pgfqpoint{7.700389in}{1.290439in}}%
\pgfusepath{stroke}%
\end{pgfscope}%
\begin{pgfscope}%
\pgfpathrectangle{\pgfqpoint{1.286132in}{0.839159in}}{\pgfqpoint{12.053712in}{5.967710in}}%
\pgfusepath{clip}%
\pgfsetbuttcap%
\pgfsetroundjoin%
\pgfsetlinewidth{1.505625pt}%
\definecolor{currentstroke}{rgb}{0.090196,0.745098,0.811765}%
\pgfsetstrokecolor{currentstroke}%
\pgfsetdash{}{0pt}%
\pgfpathmoveto{\pgfqpoint{7.811075in}{1.236383in}}%
\pgfpathlineto{\pgfqpoint{7.811075in}{1.273146in}}%
\pgfusepath{stroke}%
\end{pgfscope}%
\begin{pgfscope}%
\pgfpathrectangle{\pgfqpoint{1.286132in}{0.839159in}}{\pgfqpoint{12.053712in}{5.967710in}}%
\pgfusepath{clip}%
\pgfsetbuttcap%
\pgfsetroundjoin%
\pgfsetlinewidth{1.505625pt}%
\definecolor{currentstroke}{rgb}{0.090196,0.745098,0.811765}%
\pgfsetstrokecolor{currentstroke}%
\pgfsetdash{}{0pt}%
\pgfpathmoveto{\pgfqpoint{7.921761in}{1.281215in}}%
\pgfpathlineto{\pgfqpoint{7.921761in}{1.286548in}}%
\pgfusepath{stroke}%
\end{pgfscope}%
\begin{pgfscope}%
\pgfpathrectangle{\pgfqpoint{1.286132in}{0.839159in}}{\pgfqpoint{12.053712in}{5.967710in}}%
\pgfusepath{clip}%
\pgfsetbuttcap%
\pgfsetroundjoin%
\pgfsetlinewidth{1.505625pt}%
\definecolor{currentstroke}{rgb}{0.090196,0.745098,0.811765}%
\pgfsetstrokecolor{currentstroke}%
\pgfsetdash{}{0pt}%
\pgfpathmoveto{\pgfqpoint{8.032447in}{1.230336in}}%
\pgfpathlineto{\pgfqpoint{8.032447in}{1.290524in}}%
\pgfusepath{stroke}%
\end{pgfscope}%
\begin{pgfscope}%
\pgfpathrectangle{\pgfqpoint{1.286132in}{0.839159in}}{\pgfqpoint{12.053712in}{5.967710in}}%
\pgfusepath{clip}%
\pgfsetbuttcap%
\pgfsetroundjoin%
\pgfsetlinewidth{1.505625pt}%
\definecolor{currentstroke}{rgb}{0.090196,0.745098,0.811765}%
\pgfsetstrokecolor{currentstroke}%
\pgfsetdash{}{0pt}%
\pgfpathmoveto{\pgfqpoint{8.143133in}{1.244983in}}%
\pgfpathlineto{\pgfqpoint{8.143133in}{1.342149in}}%
\pgfusepath{stroke}%
\end{pgfscope}%
\begin{pgfscope}%
\pgfpathrectangle{\pgfqpoint{1.286132in}{0.839159in}}{\pgfqpoint{12.053712in}{5.967710in}}%
\pgfusepath{clip}%
\pgfsetbuttcap%
\pgfsetroundjoin%
\pgfsetlinewidth{1.505625pt}%
\definecolor{currentstroke}{rgb}{0.090196,0.745098,0.811765}%
\pgfsetstrokecolor{currentstroke}%
\pgfsetdash{}{0pt}%
\pgfpathmoveto{\pgfqpoint{8.253819in}{1.305805in}}%
\pgfpathlineto{\pgfqpoint{8.253819in}{1.352688in}}%
\pgfusepath{stroke}%
\end{pgfscope}%
\begin{pgfscope}%
\pgfpathrectangle{\pgfqpoint{1.286132in}{0.839159in}}{\pgfqpoint{12.053712in}{5.967710in}}%
\pgfusepath{clip}%
\pgfsetbuttcap%
\pgfsetroundjoin%
\pgfsetlinewidth{1.505625pt}%
\definecolor{currentstroke}{rgb}{0.090196,0.745098,0.811765}%
\pgfsetstrokecolor{currentstroke}%
\pgfsetdash{}{0pt}%
\pgfpathmoveto{\pgfqpoint{8.364505in}{1.303754in}}%
\pgfpathlineto{\pgfqpoint{8.364505in}{1.338705in}}%
\pgfusepath{stroke}%
\end{pgfscope}%
\begin{pgfscope}%
\pgfpathrectangle{\pgfqpoint{1.286132in}{0.839159in}}{\pgfqpoint{12.053712in}{5.967710in}}%
\pgfusepath{clip}%
\pgfsetbuttcap%
\pgfsetroundjoin%
\pgfsetlinewidth{1.505625pt}%
\definecolor{currentstroke}{rgb}{0.090196,0.745098,0.811765}%
\pgfsetstrokecolor{currentstroke}%
\pgfsetdash{}{0pt}%
\pgfpathmoveto{\pgfqpoint{8.475191in}{1.306482in}}%
\pgfpathlineto{\pgfqpoint{8.475191in}{1.335493in}}%
\pgfusepath{stroke}%
\end{pgfscope}%
\begin{pgfscope}%
\pgfpathrectangle{\pgfqpoint{1.286132in}{0.839159in}}{\pgfqpoint{12.053712in}{5.967710in}}%
\pgfusepath{clip}%
\pgfsetbuttcap%
\pgfsetroundjoin%
\pgfsetlinewidth{1.505625pt}%
\definecolor{currentstroke}{rgb}{0.090196,0.745098,0.811765}%
\pgfsetstrokecolor{currentstroke}%
\pgfsetdash{}{0pt}%
\pgfpathmoveto{\pgfqpoint{8.585877in}{1.252566in}}%
\pgfpathlineto{\pgfqpoint{8.585877in}{1.644855in}}%
\pgfusepath{stroke}%
\end{pgfscope}%
\begin{pgfscope}%
\pgfpathrectangle{\pgfqpoint{1.286132in}{0.839159in}}{\pgfqpoint{12.053712in}{5.967710in}}%
\pgfusepath{clip}%
\pgfsetbuttcap%
\pgfsetroundjoin%
\pgfsetlinewidth{1.505625pt}%
\definecolor{currentstroke}{rgb}{0.090196,0.745098,0.811765}%
\pgfsetstrokecolor{currentstroke}%
\pgfsetdash{}{0pt}%
\pgfpathmoveto{\pgfqpoint{8.696563in}{1.320904in}}%
\pgfpathlineto{\pgfqpoint{8.696563in}{1.421804in}}%
\pgfusepath{stroke}%
\end{pgfscope}%
\begin{pgfscope}%
\pgfpathrectangle{\pgfqpoint{1.286132in}{0.839159in}}{\pgfqpoint{12.053712in}{5.967710in}}%
\pgfusepath{clip}%
\pgfsetbuttcap%
\pgfsetroundjoin%
\pgfsetlinewidth{1.505625pt}%
\definecolor{currentstroke}{rgb}{0.090196,0.745098,0.811765}%
\pgfsetstrokecolor{currentstroke}%
\pgfsetdash{}{0pt}%
\pgfpathmoveto{\pgfqpoint{8.807249in}{1.312869in}}%
\pgfpathlineto{\pgfqpoint{8.807249in}{1.388718in}}%
\pgfusepath{stroke}%
\end{pgfscope}%
\begin{pgfscope}%
\pgfpathrectangle{\pgfqpoint{1.286132in}{0.839159in}}{\pgfqpoint{12.053712in}{5.967710in}}%
\pgfusepath{clip}%
\pgfsetbuttcap%
\pgfsetroundjoin%
\pgfsetlinewidth{1.505625pt}%
\definecolor{currentstroke}{rgb}{0.090196,0.745098,0.811765}%
\pgfsetstrokecolor{currentstroke}%
\pgfsetdash{}{0pt}%
\pgfpathmoveto{\pgfqpoint{8.917936in}{1.356126in}}%
\pgfpathlineto{\pgfqpoint{8.917936in}{1.440324in}}%
\pgfusepath{stroke}%
\end{pgfscope}%
\begin{pgfscope}%
\pgfpathrectangle{\pgfqpoint{1.286132in}{0.839159in}}{\pgfqpoint{12.053712in}{5.967710in}}%
\pgfusepath{clip}%
\pgfsetbuttcap%
\pgfsetroundjoin%
\pgfsetlinewidth{1.505625pt}%
\definecolor{currentstroke}{rgb}{0.090196,0.745098,0.811765}%
\pgfsetstrokecolor{currentstroke}%
\pgfsetdash{}{0pt}%
\pgfpathmoveto{\pgfqpoint{9.028622in}{1.278066in}}%
\pgfpathlineto{\pgfqpoint{9.028622in}{1.391709in}}%
\pgfusepath{stroke}%
\end{pgfscope}%
\begin{pgfscope}%
\pgfpathrectangle{\pgfqpoint{1.286132in}{0.839159in}}{\pgfqpoint{12.053712in}{5.967710in}}%
\pgfusepath{clip}%
\pgfsetbuttcap%
\pgfsetroundjoin%
\pgfsetlinewidth{1.505625pt}%
\definecolor{currentstroke}{rgb}{0.090196,0.745098,0.811765}%
\pgfsetstrokecolor{currentstroke}%
\pgfsetdash{}{0pt}%
\pgfpathmoveto{\pgfqpoint{9.139308in}{1.376269in}}%
\pgfpathlineto{\pgfqpoint{9.139308in}{1.433402in}}%
\pgfusepath{stroke}%
\end{pgfscope}%
\begin{pgfscope}%
\pgfpathrectangle{\pgfqpoint{1.286132in}{0.839159in}}{\pgfqpoint{12.053712in}{5.967710in}}%
\pgfusepath{clip}%
\pgfsetbuttcap%
\pgfsetroundjoin%
\pgfsetlinewidth{1.505625pt}%
\definecolor{currentstroke}{rgb}{0.090196,0.745098,0.811765}%
\pgfsetstrokecolor{currentstroke}%
\pgfsetdash{}{0pt}%
\pgfpathmoveto{\pgfqpoint{9.249994in}{1.360727in}}%
\pgfpathlineto{\pgfqpoint{9.249994in}{1.421154in}}%
\pgfusepath{stroke}%
\end{pgfscope}%
\begin{pgfscope}%
\pgfpathrectangle{\pgfqpoint{1.286132in}{0.839159in}}{\pgfqpoint{12.053712in}{5.967710in}}%
\pgfusepath{clip}%
\pgfsetbuttcap%
\pgfsetroundjoin%
\pgfsetlinewidth{1.505625pt}%
\definecolor{currentstroke}{rgb}{0.090196,0.745098,0.811765}%
\pgfsetstrokecolor{currentstroke}%
\pgfsetdash{}{0pt}%
\pgfpathmoveto{\pgfqpoint{9.360680in}{1.331946in}}%
\pgfpathlineto{\pgfqpoint{9.360680in}{1.459080in}}%
\pgfusepath{stroke}%
\end{pgfscope}%
\begin{pgfscope}%
\pgfpathrectangle{\pgfqpoint{1.286132in}{0.839159in}}{\pgfqpoint{12.053712in}{5.967710in}}%
\pgfusepath{clip}%
\pgfsetbuttcap%
\pgfsetroundjoin%
\pgfsetlinewidth{1.505625pt}%
\definecolor{currentstroke}{rgb}{0.090196,0.745098,0.811765}%
\pgfsetstrokecolor{currentstroke}%
\pgfsetdash{}{0pt}%
\pgfpathmoveto{\pgfqpoint{9.471366in}{1.354419in}}%
\pgfpathlineto{\pgfqpoint{9.471366in}{1.450025in}}%
\pgfusepath{stroke}%
\end{pgfscope}%
\begin{pgfscope}%
\pgfpathrectangle{\pgfqpoint{1.286132in}{0.839159in}}{\pgfqpoint{12.053712in}{5.967710in}}%
\pgfusepath{clip}%
\pgfsetbuttcap%
\pgfsetroundjoin%
\pgfsetlinewidth{1.505625pt}%
\definecolor{currentstroke}{rgb}{0.090196,0.745098,0.811765}%
\pgfsetstrokecolor{currentstroke}%
\pgfsetdash{}{0pt}%
\pgfpathmoveto{\pgfqpoint{9.582052in}{1.341002in}}%
\pgfpathlineto{\pgfqpoint{9.582052in}{1.416263in}}%
\pgfusepath{stroke}%
\end{pgfscope}%
\begin{pgfscope}%
\pgfpathrectangle{\pgfqpoint{1.286132in}{0.839159in}}{\pgfqpoint{12.053712in}{5.967710in}}%
\pgfusepath{clip}%
\pgfsetbuttcap%
\pgfsetroundjoin%
\pgfsetlinewidth{1.505625pt}%
\definecolor{currentstroke}{rgb}{0.090196,0.745098,0.811765}%
\pgfsetstrokecolor{currentstroke}%
\pgfsetdash{}{0pt}%
\pgfpathmoveto{\pgfqpoint{9.692738in}{1.346755in}}%
\pgfpathlineto{\pgfqpoint{9.692738in}{1.415586in}}%
\pgfusepath{stroke}%
\end{pgfscope}%
\begin{pgfscope}%
\pgfpathrectangle{\pgfqpoint{1.286132in}{0.839159in}}{\pgfqpoint{12.053712in}{5.967710in}}%
\pgfusepath{clip}%
\pgfsetbuttcap%
\pgfsetroundjoin%
\pgfsetlinewidth{1.505625pt}%
\definecolor{currentstroke}{rgb}{0.090196,0.745098,0.811765}%
\pgfsetstrokecolor{currentstroke}%
\pgfsetdash{}{0pt}%
\pgfpathmoveto{\pgfqpoint{9.803424in}{1.358955in}}%
\pgfpathlineto{\pgfqpoint{9.803424in}{1.466757in}}%
\pgfusepath{stroke}%
\end{pgfscope}%
\begin{pgfscope}%
\pgfpathrectangle{\pgfqpoint{1.286132in}{0.839159in}}{\pgfqpoint{12.053712in}{5.967710in}}%
\pgfusepath{clip}%
\pgfsetbuttcap%
\pgfsetroundjoin%
\pgfsetlinewidth{1.505625pt}%
\definecolor{currentstroke}{rgb}{0.090196,0.745098,0.811765}%
\pgfsetstrokecolor{currentstroke}%
\pgfsetdash{}{0pt}%
\pgfpathmoveto{\pgfqpoint{9.914110in}{1.374742in}}%
\pgfpathlineto{\pgfqpoint{9.914110in}{1.428648in}}%
\pgfusepath{stroke}%
\end{pgfscope}%
\begin{pgfscope}%
\pgfpathrectangle{\pgfqpoint{1.286132in}{0.839159in}}{\pgfqpoint{12.053712in}{5.967710in}}%
\pgfusepath{clip}%
\pgfsetbuttcap%
\pgfsetroundjoin%
\pgfsetlinewidth{1.505625pt}%
\definecolor{currentstroke}{rgb}{0.090196,0.745098,0.811765}%
\pgfsetstrokecolor{currentstroke}%
\pgfsetdash{}{0pt}%
\pgfpathmoveto{\pgfqpoint{10.024796in}{1.381053in}}%
\pgfpathlineto{\pgfqpoint{10.024796in}{1.494620in}}%
\pgfusepath{stroke}%
\end{pgfscope}%
\begin{pgfscope}%
\pgfpathrectangle{\pgfqpoint{1.286132in}{0.839159in}}{\pgfqpoint{12.053712in}{5.967710in}}%
\pgfusepath{clip}%
\pgfsetbuttcap%
\pgfsetroundjoin%
\pgfsetlinewidth{1.505625pt}%
\definecolor{currentstroke}{rgb}{0.090196,0.745098,0.811765}%
\pgfsetstrokecolor{currentstroke}%
\pgfsetdash{}{0pt}%
\pgfpathmoveto{\pgfqpoint{10.135482in}{1.367195in}}%
\pgfpathlineto{\pgfqpoint{10.135482in}{1.610176in}}%
\pgfusepath{stroke}%
\end{pgfscope}%
\begin{pgfscope}%
\pgfpathrectangle{\pgfqpoint{1.286132in}{0.839159in}}{\pgfqpoint{12.053712in}{5.967710in}}%
\pgfusepath{clip}%
\pgfsetbuttcap%
\pgfsetroundjoin%
\pgfsetlinewidth{1.505625pt}%
\definecolor{currentstroke}{rgb}{0.090196,0.745098,0.811765}%
\pgfsetstrokecolor{currentstroke}%
\pgfsetdash{}{0pt}%
\pgfpathmoveto{\pgfqpoint{10.246168in}{1.427603in}}%
\pgfpathlineto{\pgfqpoint{10.246168in}{1.545376in}}%
\pgfusepath{stroke}%
\end{pgfscope}%
\begin{pgfscope}%
\pgfpathrectangle{\pgfqpoint{1.286132in}{0.839159in}}{\pgfqpoint{12.053712in}{5.967710in}}%
\pgfusepath{clip}%
\pgfsetbuttcap%
\pgfsetroundjoin%
\pgfsetlinewidth{1.505625pt}%
\definecolor{currentstroke}{rgb}{0.090196,0.745098,0.811765}%
\pgfsetstrokecolor{currentstroke}%
\pgfsetdash{}{0pt}%
\pgfpathmoveto{\pgfqpoint{10.356854in}{1.408932in}}%
\pgfpathlineto{\pgfqpoint{10.356854in}{1.486248in}}%
\pgfusepath{stroke}%
\end{pgfscope}%
\begin{pgfscope}%
\pgfpathrectangle{\pgfqpoint{1.286132in}{0.839159in}}{\pgfqpoint{12.053712in}{5.967710in}}%
\pgfusepath{clip}%
\pgfsetbuttcap%
\pgfsetroundjoin%
\pgfsetlinewidth{1.505625pt}%
\definecolor{currentstroke}{rgb}{0.090196,0.745098,0.811765}%
\pgfsetstrokecolor{currentstroke}%
\pgfsetdash{}{0pt}%
\pgfpathmoveto{\pgfqpoint{10.467540in}{1.464939in}}%
\pgfpathlineto{\pgfqpoint{10.467540in}{1.519434in}}%
\pgfusepath{stroke}%
\end{pgfscope}%
\begin{pgfscope}%
\pgfpathrectangle{\pgfqpoint{1.286132in}{0.839159in}}{\pgfqpoint{12.053712in}{5.967710in}}%
\pgfusepath{clip}%
\pgfsetbuttcap%
\pgfsetroundjoin%
\pgfsetlinewidth{1.505625pt}%
\definecolor{currentstroke}{rgb}{0.090196,0.745098,0.811765}%
\pgfsetstrokecolor{currentstroke}%
\pgfsetdash{}{0pt}%
\pgfpathmoveto{\pgfqpoint{10.578226in}{1.445141in}}%
\pgfpathlineto{\pgfqpoint{10.578226in}{1.530884in}}%
\pgfusepath{stroke}%
\end{pgfscope}%
\begin{pgfscope}%
\pgfpathrectangle{\pgfqpoint{1.286132in}{0.839159in}}{\pgfqpoint{12.053712in}{5.967710in}}%
\pgfusepath{clip}%
\pgfsetbuttcap%
\pgfsetroundjoin%
\pgfsetlinewidth{1.505625pt}%
\definecolor{currentstroke}{rgb}{0.090196,0.745098,0.811765}%
\pgfsetstrokecolor{currentstroke}%
\pgfsetdash{}{0pt}%
\pgfpathmoveto{\pgfqpoint{10.688913in}{1.457470in}}%
\pgfpathlineto{\pgfqpoint{10.688913in}{1.577597in}}%
\pgfusepath{stroke}%
\end{pgfscope}%
\begin{pgfscope}%
\pgfpathrectangle{\pgfqpoint{1.286132in}{0.839159in}}{\pgfqpoint{12.053712in}{5.967710in}}%
\pgfusepath{clip}%
\pgfsetbuttcap%
\pgfsetroundjoin%
\pgfsetlinewidth{1.505625pt}%
\definecolor{currentstroke}{rgb}{0.090196,0.745098,0.811765}%
\pgfsetstrokecolor{currentstroke}%
\pgfsetdash{}{0pt}%
\pgfpathmoveto{\pgfqpoint{10.799599in}{1.466557in}}%
\pgfpathlineto{\pgfqpoint{10.799599in}{1.592726in}}%
\pgfusepath{stroke}%
\end{pgfscope}%
\begin{pgfscope}%
\pgfpathrectangle{\pgfqpoint{1.286132in}{0.839159in}}{\pgfqpoint{12.053712in}{5.967710in}}%
\pgfusepath{clip}%
\pgfsetbuttcap%
\pgfsetroundjoin%
\pgfsetlinewidth{1.505625pt}%
\definecolor{currentstroke}{rgb}{0.090196,0.745098,0.811765}%
\pgfsetstrokecolor{currentstroke}%
\pgfsetdash{}{0pt}%
\pgfpathmoveto{\pgfqpoint{10.910285in}{1.438978in}}%
\pgfpathlineto{\pgfqpoint{10.910285in}{1.569440in}}%
\pgfusepath{stroke}%
\end{pgfscope}%
\begin{pgfscope}%
\pgfpathrectangle{\pgfqpoint{1.286132in}{0.839159in}}{\pgfqpoint{12.053712in}{5.967710in}}%
\pgfusepath{clip}%
\pgfsetbuttcap%
\pgfsetroundjoin%
\pgfsetlinewidth{1.505625pt}%
\definecolor{currentstroke}{rgb}{0.090196,0.745098,0.811765}%
\pgfsetstrokecolor{currentstroke}%
\pgfsetdash{}{0pt}%
\pgfpathmoveto{\pgfqpoint{11.020971in}{1.401449in}}%
\pgfpathlineto{\pgfqpoint{11.020971in}{1.570261in}}%
\pgfusepath{stroke}%
\end{pgfscope}%
\begin{pgfscope}%
\pgfpathrectangle{\pgfqpoint{1.286132in}{0.839159in}}{\pgfqpoint{12.053712in}{5.967710in}}%
\pgfusepath{clip}%
\pgfsetbuttcap%
\pgfsetroundjoin%
\pgfsetlinewidth{1.505625pt}%
\definecolor{currentstroke}{rgb}{0.090196,0.745098,0.811765}%
\pgfsetstrokecolor{currentstroke}%
\pgfsetdash{}{0pt}%
\pgfpathmoveto{\pgfqpoint{11.131657in}{1.486771in}}%
\pgfpathlineto{\pgfqpoint{11.131657in}{1.531008in}}%
\pgfusepath{stroke}%
\end{pgfscope}%
\begin{pgfscope}%
\pgfpathrectangle{\pgfqpoint{1.286132in}{0.839159in}}{\pgfqpoint{12.053712in}{5.967710in}}%
\pgfusepath{clip}%
\pgfsetbuttcap%
\pgfsetroundjoin%
\pgfsetlinewidth{1.505625pt}%
\definecolor{currentstroke}{rgb}{0.090196,0.745098,0.811765}%
\pgfsetstrokecolor{currentstroke}%
\pgfsetdash{}{0pt}%
\pgfpathmoveto{\pgfqpoint{11.242343in}{1.518706in}}%
\pgfpathlineto{\pgfqpoint{11.242343in}{1.744513in}}%
\pgfusepath{stroke}%
\end{pgfscope}%
\begin{pgfscope}%
\pgfpathrectangle{\pgfqpoint{1.286132in}{0.839159in}}{\pgfqpoint{12.053712in}{5.967710in}}%
\pgfusepath{clip}%
\pgfsetbuttcap%
\pgfsetroundjoin%
\pgfsetlinewidth{1.505625pt}%
\definecolor{currentstroke}{rgb}{0.090196,0.745098,0.811765}%
\pgfsetstrokecolor{currentstroke}%
\pgfsetdash{}{0pt}%
\pgfpathmoveto{\pgfqpoint{11.353029in}{1.481947in}}%
\pgfpathlineto{\pgfqpoint{11.353029in}{1.700206in}}%
\pgfusepath{stroke}%
\end{pgfscope}%
\begin{pgfscope}%
\pgfpathrectangle{\pgfqpoint{1.286132in}{0.839159in}}{\pgfqpoint{12.053712in}{5.967710in}}%
\pgfusepath{clip}%
\pgfsetbuttcap%
\pgfsetroundjoin%
\pgfsetlinewidth{1.505625pt}%
\definecolor{currentstroke}{rgb}{0.090196,0.745098,0.811765}%
\pgfsetstrokecolor{currentstroke}%
\pgfsetdash{}{0pt}%
\pgfpathmoveto{\pgfqpoint{11.463715in}{1.531092in}}%
\pgfpathlineto{\pgfqpoint{11.463715in}{1.644996in}}%
\pgfusepath{stroke}%
\end{pgfscope}%
\begin{pgfscope}%
\pgfpathrectangle{\pgfqpoint{1.286132in}{0.839159in}}{\pgfqpoint{12.053712in}{5.967710in}}%
\pgfusepath{clip}%
\pgfsetbuttcap%
\pgfsetroundjoin%
\pgfsetlinewidth{1.505625pt}%
\definecolor{currentstroke}{rgb}{0.090196,0.745098,0.811765}%
\pgfsetstrokecolor{currentstroke}%
\pgfsetdash{}{0pt}%
\pgfpathmoveto{\pgfqpoint{11.574401in}{1.551582in}}%
\pgfpathlineto{\pgfqpoint{11.574401in}{1.682228in}}%
\pgfusepath{stroke}%
\end{pgfscope}%
\begin{pgfscope}%
\pgfpathrectangle{\pgfqpoint{1.286132in}{0.839159in}}{\pgfqpoint{12.053712in}{5.967710in}}%
\pgfusepath{clip}%
\pgfsetbuttcap%
\pgfsetroundjoin%
\pgfsetlinewidth{1.505625pt}%
\definecolor{currentstroke}{rgb}{0.090196,0.745098,0.811765}%
\pgfsetstrokecolor{currentstroke}%
\pgfsetdash{}{0pt}%
\pgfpathmoveto{\pgfqpoint{11.685087in}{1.498302in}}%
\pgfpathlineto{\pgfqpoint{11.685087in}{1.641702in}}%
\pgfusepath{stroke}%
\end{pgfscope}%
\begin{pgfscope}%
\pgfpathrectangle{\pgfqpoint{1.286132in}{0.839159in}}{\pgfqpoint{12.053712in}{5.967710in}}%
\pgfusepath{clip}%
\pgfsetbuttcap%
\pgfsetroundjoin%
\pgfsetlinewidth{1.505625pt}%
\definecolor{currentstroke}{rgb}{0.090196,0.745098,0.811765}%
\pgfsetstrokecolor{currentstroke}%
\pgfsetdash{}{0pt}%
\pgfpathmoveto{\pgfqpoint{11.795773in}{1.465215in}}%
\pgfpathlineto{\pgfqpoint{11.795773in}{1.690412in}}%
\pgfusepath{stroke}%
\end{pgfscope}%
\begin{pgfscope}%
\pgfpathrectangle{\pgfqpoint{1.286132in}{0.839159in}}{\pgfqpoint{12.053712in}{5.967710in}}%
\pgfusepath{clip}%
\pgfsetbuttcap%
\pgfsetroundjoin%
\pgfsetlinewidth{1.505625pt}%
\definecolor{currentstroke}{rgb}{0.090196,0.745098,0.811765}%
\pgfsetstrokecolor{currentstroke}%
\pgfsetdash{}{0pt}%
\pgfpathmoveto{\pgfqpoint{11.906459in}{1.567292in}}%
\pgfpathlineto{\pgfqpoint{11.906459in}{1.839625in}}%
\pgfusepath{stroke}%
\end{pgfscope}%
\begin{pgfscope}%
\pgfpathrectangle{\pgfqpoint{1.286132in}{0.839159in}}{\pgfqpoint{12.053712in}{5.967710in}}%
\pgfusepath{clip}%
\pgfsetbuttcap%
\pgfsetroundjoin%
\pgfsetlinewidth{1.505625pt}%
\definecolor{currentstroke}{rgb}{0.090196,0.745098,0.811765}%
\pgfsetstrokecolor{currentstroke}%
\pgfsetdash{}{0pt}%
\pgfpathmoveto{\pgfqpoint{12.017145in}{1.675291in}}%
\pgfpathlineto{\pgfqpoint{12.017145in}{1.767058in}}%
\pgfusepath{stroke}%
\end{pgfscope}%
\begin{pgfscope}%
\pgfpathrectangle{\pgfqpoint{1.286132in}{0.839159in}}{\pgfqpoint{12.053712in}{5.967710in}}%
\pgfusepath{clip}%
\pgfsetbuttcap%
\pgfsetroundjoin%
\pgfsetlinewidth{1.505625pt}%
\definecolor{currentstroke}{rgb}{0.090196,0.745098,0.811765}%
\pgfsetstrokecolor{currentstroke}%
\pgfsetdash{}{0pt}%
\pgfpathmoveto{\pgfqpoint{12.127831in}{1.669828in}}%
\pgfpathlineto{\pgfqpoint{12.127831in}{1.742391in}}%
\pgfusepath{stroke}%
\end{pgfscope}%
\begin{pgfscope}%
\pgfpathrectangle{\pgfqpoint{1.286132in}{0.839159in}}{\pgfqpoint{12.053712in}{5.967710in}}%
\pgfusepath{clip}%
\pgfsetbuttcap%
\pgfsetroundjoin%
\pgfsetlinewidth{1.505625pt}%
\definecolor{currentstroke}{rgb}{0.090196,0.745098,0.811765}%
\pgfsetstrokecolor{currentstroke}%
\pgfsetdash{}{0pt}%
\pgfpathmoveto{\pgfqpoint{12.238517in}{1.559164in}}%
\pgfpathlineto{\pgfqpoint{12.238517in}{1.768329in}}%
\pgfusepath{stroke}%
\end{pgfscope}%
\begin{pgfscope}%
\pgfpathrectangle{\pgfqpoint{1.286132in}{0.839159in}}{\pgfqpoint{12.053712in}{5.967710in}}%
\pgfusepath{clip}%
\pgfsetbuttcap%
\pgfsetroundjoin%
\pgfsetlinewidth{1.505625pt}%
\definecolor{currentstroke}{rgb}{0.090196,0.745098,0.811765}%
\pgfsetstrokecolor{currentstroke}%
\pgfsetdash{}{0pt}%
\pgfpathmoveto{\pgfqpoint{12.349203in}{1.643267in}}%
\pgfpathlineto{\pgfqpoint{12.349203in}{1.766980in}}%
\pgfusepath{stroke}%
\end{pgfscope}%
\begin{pgfscope}%
\pgfpathrectangle{\pgfqpoint{1.286132in}{0.839159in}}{\pgfqpoint{12.053712in}{5.967710in}}%
\pgfusepath{clip}%
\pgfsetbuttcap%
\pgfsetroundjoin%
\pgfsetlinewidth{1.505625pt}%
\definecolor{currentstroke}{rgb}{0.090196,0.745098,0.811765}%
\pgfsetstrokecolor{currentstroke}%
\pgfsetdash{}{0pt}%
\pgfpathmoveto{\pgfqpoint{12.459890in}{1.635423in}}%
\pgfpathlineto{\pgfqpoint{12.459890in}{1.819416in}}%
\pgfusepath{stroke}%
\end{pgfscope}%
\begin{pgfscope}%
\pgfpathrectangle{\pgfqpoint{1.286132in}{0.839159in}}{\pgfqpoint{12.053712in}{5.967710in}}%
\pgfusepath{clip}%
\pgfsetbuttcap%
\pgfsetroundjoin%
\pgfsetlinewidth{1.505625pt}%
\definecolor{currentstroke}{rgb}{0.090196,0.745098,0.811765}%
\pgfsetstrokecolor{currentstroke}%
\pgfsetdash{}{0pt}%
\pgfpathmoveto{\pgfqpoint{12.570576in}{1.763203in}}%
\pgfpathlineto{\pgfqpoint{12.570576in}{1.942818in}}%
\pgfusepath{stroke}%
\end{pgfscope}%
\begin{pgfscope}%
\pgfpathrectangle{\pgfqpoint{1.286132in}{0.839159in}}{\pgfqpoint{12.053712in}{5.967710in}}%
\pgfusepath{clip}%
\pgfsetbuttcap%
\pgfsetroundjoin%
\pgfsetlinewidth{1.505625pt}%
\definecolor{currentstroke}{rgb}{0.090196,0.745098,0.811765}%
\pgfsetstrokecolor{currentstroke}%
\pgfsetdash{}{0pt}%
\pgfpathmoveto{\pgfqpoint{12.681262in}{1.725846in}}%
\pgfpathlineto{\pgfqpoint{12.681262in}{1.932145in}}%
\pgfusepath{stroke}%
\end{pgfscope}%
\begin{pgfscope}%
\pgfpathrectangle{\pgfqpoint{1.286132in}{0.839159in}}{\pgfqpoint{12.053712in}{5.967710in}}%
\pgfusepath{clip}%
\pgfsetbuttcap%
\pgfsetroundjoin%
\pgfsetlinewidth{1.505625pt}%
\definecolor{currentstroke}{rgb}{0.090196,0.745098,0.811765}%
\pgfsetstrokecolor{currentstroke}%
\pgfsetdash{}{0pt}%
\pgfpathmoveto{\pgfqpoint{12.791948in}{1.704209in}}%
\pgfpathlineto{\pgfqpoint{12.791948in}{1.866698in}}%
\pgfusepath{stroke}%
\end{pgfscope}%
\begin{pgfscope}%
\pgfpathrectangle{\pgfqpoint{1.286132in}{0.839159in}}{\pgfqpoint{12.053712in}{5.967710in}}%
\pgfusepath{clip}%
\pgfsetbuttcap%
\pgfsetroundjoin%
\pgfsetlinewidth{1.505625pt}%
\definecolor{currentstroke}{rgb}{0.121569,0.466667,0.705882}%
\pgfsetstrokecolor{currentstroke}%
\pgfsetdash{}{0pt}%
\pgfpathmoveto{\pgfqpoint{1.834028in}{1.119234in}}%
\pgfpathlineto{\pgfqpoint{1.834028in}{1.119246in}}%
\pgfusepath{stroke}%
\end{pgfscope}%
\begin{pgfscope}%
\pgfpathrectangle{\pgfqpoint{1.286132in}{0.839159in}}{\pgfqpoint{12.053712in}{5.967710in}}%
\pgfusepath{clip}%
\pgfsetbuttcap%
\pgfsetroundjoin%
\pgfsetlinewidth{1.505625pt}%
\definecolor{currentstroke}{rgb}{0.121569,0.466667,0.705882}%
\pgfsetstrokecolor{currentstroke}%
\pgfsetdash{}{0pt}%
\pgfpathmoveto{\pgfqpoint{1.944714in}{1.119336in}}%
\pgfpathlineto{\pgfqpoint{1.944714in}{1.119350in}}%
\pgfusepath{stroke}%
\end{pgfscope}%
\begin{pgfscope}%
\pgfpathrectangle{\pgfqpoint{1.286132in}{0.839159in}}{\pgfqpoint{12.053712in}{5.967710in}}%
\pgfusepath{clip}%
\pgfsetbuttcap%
\pgfsetroundjoin%
\pgfsetlinewidth{1.505625pt}%
\definecolor{currentstroke}{rgb}{0.121569,0.466667,0.705882}%
\pgfsetstrokecolor{currentstroke}%
\pgfsetdash{}{0pt}%
\pgfpathmoveto{\pgfqpoint{2.055400in}{1.119389in}}%
\pgfpathlineto{\pgfqpoint{2.055400in}{1.119413in}}%
\pgfusepath{stroke}%
\end{pgfscope}%
\begin{pgfscope}%
\pgfpathrectangle{\pgfqpoint{1.286132in}{0.839159in}}{\pgfqpoint{12.053712in}{5.967710in}}%
\pgfusepath{clip}%
\pgfsetbuttcap%
\pgfsetroundjoin%
\pgfsetlinewidth{1.505625pt}%
\definecolor{currentstroke}{rgb}{0.121569,0.466667,0.705882}%
\pgfsetstrokecolor{currentstroke}%
\pgfsetdash{}{0pt}%
\pgfpathmoveto{\pgfqpoint{2.166086in}{1.119441in}}%
\pgfpathlineto{\pgfqpoint{2.166086in}{1.119455in}}%
\pgfusepath{stroke}%
\end{pgfscope}%
\begin{pgfscope}%
\pgfpathrectangle{\pgfqpoint{1.286132in}{0.839159in}}{\pgfqpoint{12.053712in}{5.967710in}}%
\pgfusepath{clip}%
\pgfsetbuttcap%
\pgfsetroundjoin%
\pgfsetlinewidth{1.505625pt}%
\definecolor{currentstroke}{rgb}{0.121569,0.466667,0.705882}%
\pgfsetstrokecolor{currentstroke}%
\pgfsetdash{}{0pt}%
\pgfpathmoveto{\pgfqpoint{2.276772in}{1.119512in}}%
\pgfpathlineto{\pgfqpoint{2.276772in}{1.119530in}}%
\pgfusepath{stroke}%
\end{pgfscope}%
\begin{pgfscope}%
\pgfpathrectangle{\pgfqpoint{1.286132in}{0.839159in}}{\pgfqpoint{12.053712in}{5.967710in}}%
\pgfusepath{clip}%
\pgfsetbuttcap%
\pgfsetroundjoin%
\pgfsetlinewidth{1.505625pt}%
\definecolor{currentstroke}{rgb}{0.121569,0.466667,0.705882}%
\pgfsetstrokecolor{currentstroke}%
\pgfsetdash{}{0pt}%
\pgfpathmoveto{\pgfqpoint{2.387458in}{1.113613in}}%
\pgfpathlineto{\pgfqpoint{2.387458in}{1.137632in}}%
\pgfusepath{stroke}%
\end{pgfscope}%
\begin{pgfscope}%
\pgfpathrectangle{\pgfqpoint{1.286132in}{0.839159in}}{\pgfqpoint{12.053712in}{5.967710in}}%
\pgfusepath{clip}%
\pgfsetbuttcap%
\pgfsetroundjoin%
\pgfsetlinewidth{1.505625pt}%
\definecolor{currentstroke}{rgb}{0.121569,0.466667,0.705882}%
\pgfsetstrokecolor{currentstroke}%
\pgfsetdash{}{0pt}%
\pgfpathmoveto{\pgfqpoint{2.498144in}{1.119640in}}%
\pgfpathlineto{\pgfqpoint{2.498144in}{1.119686in}}%
\pgfusepath{stroke}%
\end{pgfscope}%
\begin{pgfscope}%
\pgfpathrectangle{\pgfqpoint{1.286132in}{0.839159in}}{\pgfqpoint{12.053712in}{5.967710in}}%
\pgfusepath{clip}%
\pgfsetbuttcap%
\pgfsetroundjoin%
\pgfsetlinewidth{1.505625pt}%
\definecolor{currentstroke}{rgb}{0.121569,0.466667,0.705882}%
\pgfsetstrokecolor{currentstroke}%
\pgfsetdash{}{0pt}%
\pgfpathmoveto{\pgfqpoint{2.608830in}{1.119736in}}%
\pgfpathlineto{\pgfqpoint{2.608830in}{1.119758in}}%
\pgfusepath{stroke}%
\end{pgfscope}%
\begin{pgfscope}%
\pgfpathrectangle{\pgfqpoint{1.286132in}{0.839159in}}{\pgfqpoint{12.053712in}{5.967710in}}%
\pgfusepath{clip}%
\pgfsetbuttcap%
\pgfsetroundjoin%
\pgfsetlinewidth{1.505625pt}%
\definecolor{currentstroke}{rgb}{0.121569,0.466667,0.705882}%
\pgfsetstrokecolor{currentstroke}%
\pgfsetdash{}{0pt}%
\pgfpathmoveto{\pgfqpoint{2.719516in}{1.119787in}}%
\pgfpathlineto{\pgfqpoint{2.719516in}{1.119875in}}%
\pgfusepath{stroke}%
\end{pgfscope}%
\begin{pgfscope}%
\pgfpathrectangle{\pgfqpoint{1.286132in}{0.839159in}}{\pgfqpoint{12.053712in}{5.967710in}}%
\pgfusepath{clip}%
\pgfsetbuttcap%
\pgfsetroundjoin%
\pgfsetlinewidth{1.505625pt}%
\definecolor{currentstroke}{rgb}{0.121569,0.466667,0.705882}%
\pgfsetstrokecolor{currentstroke}%
\pgfsetdash{}{0pt}%
\pgfpathmoveto{\pgfqpoint{2.830202in}{1.119884in}}%
\pgfpathlineto{\pgfqpoint{2.830202in}{1.119989in}}%
\pgfusepath{stroke}%
\end{pgfscope}%
\begin{pgfscope}%
\pgfpathrectangle{\pgfqpoint{1.286132in}{0.839159in}}{\pgfqpoint{12.053712in}{5.967710in}}%
\pgfusepath{clip}%
\pgfsetbuttcap%
\pgfsetroundjoin%
\pgfsetlinewidth{1.505625pt}%
\definecolor{currentstroke}{rgb}{0.121569,0.466667,0.705882}%
\pgfsetstrokecolor{currentstroke}%
\pgfsetdash{}{0pt}%
\pgfpathmoveto{\pgfqpoint{2.940888in}{1.119992in}}%
\pgfpathlineto{\pgfqpoint{2.940888in}{1.120021in}}%
\pgfusepath{stroke}%
\end{pgfscope}%
\begin{pgfscope}%
\pgfpathrectangle{\pgfqpoint{1.286132in}{0.839159in}}{\pgfqpoint{12.053712in}{5.967710in}}%
\pgfusepath{clip}%
\pgfsetbuttcap%
\pgfsetroundjoin%
\pgfsetlinewidth{1.505625pt}%
\definecolor{currentstroke}{rgb}{0.121569,0.466667,0.705882}%
\pgfsetstrokecolor{currentstroke}%
\pgfsetdash{}{0pt}%
\pgfpathmoveto{\pgfqpoint{3.051574in}{1.120074in}}%
\pgfpathlineto{\pgfqpoint{3.051574in}{1.120116in}}%
\pgfusepath{stroke}%
\end{pgfscope}%
\begin{pgfscope}%
\pgfpathrectangle{\pgfqpoint{1.286132in}{0.839159in}}{\pgfqpoint{12.053712in}{5.967710in}}%
\pgfusepath{clip}%
\pgfsetbuttcap%
\pgfsetroundjoin%
\pgfsetlinewidth{1.505625pt}%
\definecolor{currentstroke}{rgb}{0.121569,0.466667,0.705882}%
\pgfsetstrokecolor{currentstroke}%
\pgfsetdash{}{0pt}%
\pgfpathmoveto{\pgfqpoint{3.162260in}{1.120119in}}%
\pgfpathlineto{\pgfqpoint{3.162260in}{1.120274in}}%
\pgfusepath{stroke}%
\end{pgfscope}%
\begin{pgfscope}%
\pgfpathrectangle{\pgfqpoint{1.286132in}{0.839159in}}{\pgfqpoint{12.053712in}{5.967710in}}%
\pgfusepath{clip}%
\pgfsetbuttcap%
\pgfsetroundjoin%
\pgfsetlinewidth{1.505625pt}%
\definecolor{currentstroke}{rgb}{0.121569,0.466667,0.705882}%
\pgfsetstrokecolor{currentstroke}%
\pgfsetdash{}{0pt}%
\pgfpathmoveto{\pgfqpoint{3.272946in}{1.120312in}}%
\pgfpathlineto{\pgfqpoint{3.272946in}{1.120773in}}%
\pgfusepath{stroke}%
\end{pgfscope}%
\begin{pgfscope}%
\pgfpathrectangle{\pgfqpoint{1.286132in}{0.839159in}}{\pgfqpoint{12.053712in}{5.967710in}}%
\pgfusepath{clip}%
\pgfsetbuttcap%
\pgfsetroundjoin%
\pgfsetlinewidth{1.505625pt}%
\definecolor{currentstroke}{rgb}{0.121569,0.466667,0.705882}%
\pgfsetstrokecolor{currentstroke}%
\pgfsetdash{}{0pt}%
\pgfpathmoveto{\pgfqpoint{3.383632in}{1.120326in}}%
\pgfpathlineto{\pgfqpoint{3.383632in}{1.120516in}}%
\pgfusepath{stroke}%
\end{pgfscope}%
\begin{pgfscope}%
\pgfpathrectangle{\pgfqpoint{1.286132in}{0.839159in}}{\pgfqpoint{12.053712in}{5.967710in}}%
\pgfusepath{clip}%
\pgfsetbuttcap%
\pgfsetroundjoin%
\pgfsetlinewidth{1.505625pt}%
\definecolor{currentstroke}{rgb}{0.121569,0.466667,0.705882}%
\pgfsetstrokecolor{currentstroke}%
\pgfsetdash{}{0pt}%
\pgfpathmoveto{\pgfqpoint{3.494319in}{1.120533in}}%
\pgfpathlineto{\pgfqpoint{3.494319in}{1.120711in}}%
\pgfusepath{stroke}%
\end{pgfscope}%
\begin{pgfscope}%
\pgfpathrectangle{\pgfqpoint{1.286132in}{0.839159in}}{\pgfqpoint{12.053712in}{5.967710in}}%
\pgfusepath{clip}%
\pgfsetbuttcap%
\pgfsetroundjoin%
\pgfsetlinewidth{1.505625pt}%
\definecolor{currentstroke}{rgb}{0.121569,0.466667,0.705882}%
\pgfsetstrokecolor{currentstroke}%
\pgfsetdash{}{0pt}%
\pgfpathmoveto{\pgfqpoint{3.605005in}{1.120525in}}%
\pgfpathlineto{\pgfqpoint{3.605005in}{1.120789in}}%
\pgfusepath{stroke}%
\end{pgfscope}%
\begin{pgfscope}%
\pgfpathrectangle{\pgfqpoint{1.286132in}{0.839159in}}{\pgfqpoint{12.053712in}{5.967710in}}%
\pgfusepath{clip}%
\pgfsetbuttcap%
\pgfsetroundjoin%
\pgfsetlinewidth{1.505625pt}%
\definecolor{currentstroke}{rgb}{0.121569,0.466667,0.705882}%
\pgfsetstrokecolor{currentstroke}%
\pgfsetdash{}{0pt}%
\pgfpathmoveto{\pgfqpoint{3.715691in}{1.120703in}}%
\pgfpathlineto{\pgfqpoint{3.715691in}{1.121006in}}%
\pgfusepath{stroke}%
\end{pgfscope}%
\begin{pgfscope}%
\pgfpathrectangle{\pgfqpoint{1.286132in}{0.839159in}}{\pgfqpoint{12.053712in}{5.967710in}}%
\pgfusepath{clip}%
\pgfsetbuttcap%
\pgfsetroundjoin%
\pgfsetlinewidth{1.505625pt}%
\definecolor{currentstroke}{rgb}{0.121569,0.466667,0.705882}%
\pgfsetstrokecolor{currentstroke}%
\pgfsetdash{}{0pt}%
\pgfpathmoveto{\pgfqpoint{3.826377in}{1.120836in}}%
\pgfpathlineto{\pgfqpoint{3.826377in}{1.120913in}}%
\pgfusepath{stroke}%
\end{pgfscope}%
\begin{pgfscope}%
\pgfpathrectangle{\pgfqpoint{1.286132in}{0.839159in}}{\pgfqpoint{12.053712in}{5.967710in}}%
\pgfusepath{clip}%
\pgfsetbuttcap%
\pgfsetroundjoin%
\pgfsetlinewidth{1.505625pt}%
\definecolor{currentstroke}{rgb}{0.121569,0.466667,0.705882}%
\pgfsetstrokecolor{currentstroke}%
\pgfsetdash{}{0pt}%
\pgfpathmoveto{\pgfqpoint{3.937063in}{1.120406in}}%
\pgfpathlineto{\pgfqpoint{3.937063in}{1.123201in}}%
\pgfusepath{stroke}%
\end{pgfscope}%
\begin{pgfscope}%
\pgfpathrectangle{\pgfqpoint{1.286132in}{0.839159in}}{\pgfqpoint{12.053712in}{5.967710in}}%
\pgfusepath{clip}%
\pgfsetbuttcap%
\pgfsetroundjoin%
\pgfsetlinewidth{1.505625pt}%
\definecolor{currentstroke}{rgb}{0.121569,0.466667,0.705882}%
\pgfsetstrokecolor{currentstroke}%
\pgfsetdash{}{0pt}%
\pgfpathmoveto{\pgfqpoint{4.047749in}{1.121039in}}%
\pgfpathlineto{\pgfqpoint{4.047749in}{1.121451in}}%
\pgfusepath{stroke}%
\end{pgfscope}%
\begin{pgfscope}%
\pgfpathrectangle{\pgfqpoint{1.286132in}{0.839159in}}{\pgfqpoint{12.053712in}{5.967710in}}%
\pgfusepath{clip}%
\pgfsetbuttcap%
\pgfsetroundjoin%
\pgfsetlinewidth{1.505625pt}%
\definecolor{currentstroke}{rgb}{0.121569,0.466667,0.705882}%
\pgfsetstrokecolor{currentstroke}%
\pgfsetdash{}{0pt}%
\pgfpathmoveto{\pgfqpoint{4.158435in}{1.121291in}}%
\pgfpathlineto{\pgfqpoint{4.158435in}{1.121808in}}%
\pgfusepath{stroke}%
\end{pgfscope}%
\begin{pgfscope}%
\pgfpathrectangle{\pgfqpoint{1.286132in}{0.839159in}}{\pgfqpoint{12.053712in}{5.967710in}}%
\pgfusepath{clip}%
\pgfsetbuttcap%
\pgfsetroundjoin%
\pgfsetlinewidth{1.505625pt}%
\definecolor{currentstroke}{rgb}{0.121569,0.466667,0.705882}%
\pgfsetstrokecolor{currentstroke}%
\pgfsetdash{}{0pt}%
\pgfpathmoveto{\pgfqpoint{4.269121in}{1.121381in}}%
\pgfpathlineto{\pgfqpoint{4.269121in}{1.121512in}}%
\pgfusepath{stroke}%
\end{pgfscope}%
\begin{pgfscope}%
\pgfpathrectangle{\pgfqpoint{1.286132in}{0.839159in}}{\pgfqpoint{12.053712in}{5.967710in}}%
\pgfusepath{clip}%
\pgfsetbuttcap%
\pgfsetroundjoin%
\pgfsetlinewidth{1.505625pt}%
\definecolor{currentstroke}{rgb}{0.121569,0.466667,0.705882}%
\pgfsetstrokecolor{currentstroke}%
\pgfsetdash{}{0pt}%
\pgfpathmoveto{\pgfqpoint{4.379807in}{1.121620in}}%
\pgfpathlineto{\pgfqpoint{4.379807in}{1.121917in}}%
\pgfusepath{stroke}%
\end{pgfscope}%
\begin{pgfscope}%
\pgfpathrectangle{\pgfqpoint{1.286132in}{0.839159in}}{\pgfqpoint{12.053712in}{5.967710in}}%
\pgfusepath{clip}%
\pgfsetbuttcap%
\pgfsetroundjoin%
\pgfsetlinewidth{1.505625pt}%
\definecolor{currentstroke}{rgb}{0.121569,0.466667,0.705882}%
\pgfsetstrokecolor{currentstroke}%
\pgfsetdash{}{0pt}%
\pgfpathmoveto{\pgfqpoint{4.490493in}{1.121677in}}%
\pgfpathlineto{\pgfqpoint{4.490493in}{1.121922in}}%
\pgfusepath{stroke}%
\end{pgfscope}%
\begin{pgfscope}%
\pgfpathrectangle{\pgfqpoint{1.286132in}{0.839159in}}{\pgfqpoint{12.053712in}{5.967710in}}%
\pgfusepath{clip}%
\pgfsetbuttcap%
\pgfsetroundjoin%
\pgfsetlinewidth{1.505625pt}%
\definecolor{currentstroke}{rgb}{0.121569,0.466667,0.705882}%
\pgfsetstrokecolor{currentstroke}%
\pgfsetdash{}{0pt}%
\pgfpathmoveto{\pgfqpoint{4.601179in}{1.121904in}}%
\pgfpathlineto{\pgfqpoint{4.601179in}{1.122242in}}%
\pgfusepath{stroke}%
\end{pgfscope}%
\begin{pgfscope}%
\pgfpathrectangle{\pgfqpoint{1.286132in}{0.839159in}}{\pgfqpoint{12.053712in}{5.967710in}}%
\pgfusepath{clip}%
\pgfsetbuttcap%
\pgfsetroundjoin%
\pgfsetlinewidth{1.505625pt}%
\definecolor{currentstroke}{rgb}{0.121569,0.466667,0.705882}%
\pgfsetstrokecolor{currentstroke}%
\pgfsetdash{}{0pt}%
\pgfpathmoveto{\pgfqpoint{4.711865in}{1.121925in}}%
\pgfpathlineto{\pgfqpoint{4.711865in}{1.122504in}}%
\pgfusepath{stroke}%
\end{pgfscope}%
\begin{pgfscope}%
\pgfpathrectangle{\pgfqpoint{1.286132in}{0.839159in}}{\pgfqpoint{12.053712in}{5.967710in}}%
\pgfusepath{clip}%
\pgfsetbuttcap%
\pgfsetroundjoin%
\pgfsetlinewidth{1.505625pt}%
\definecolor{currentstroke}{rgb}{0.121569,0.466667,0.705882}%
\pgfsetstrokecolor{currentstroke}%
\pgfsetdash{}{0pt}%
\pgfpathmoveto{\pgfqpoint{4.822551in}{1.122132in}}%
\pgfpathlineto{\pgfqpoint{4.822551in}{1.122709in}}%
\pgfusepath{stroke}%
\end{pgfscope}%
\begin{pgfscope}%
\pgfpathrectangle{\pgfqpoint{1.286132in}{0.839159in}}{\pgfqpoint{12.053712in}{5.967710in}}%
\pgfusepath{clip}%
\pgfsetbuttcap%
\pgfsetroundjoin%
\pgfsetlinewidth{1.505625pt}%
\definecolor{currentstroke}{rgb}{0.121569,0.466667,0.705882}%
\pgfsetstrokecolor{currentstroke}%
\pgfsetdash{}{0pt}%
\pgfpathmoveto{\pgfqpoint{4.933237in}{1.122244in}}%
\pgfpathlineto{\pgfqpoint{4.933237in}{1.122748in}}%
\pgfusepath{stroke}%
\end{pgfscope}%
\begin{pgfscope}%
\pgfpathrectangle{\pgfqpoint{1.286132in}{0.839159in}}{\pgfqpoint{12.053712in}{5.967710in}}%
\pgfusepath{clip}%
\pgfsetbuttcap%
\pgfsetroundjoin%
\pgfsetlinewidth{1.505625pt}%
\definecolor{currentstroke}{rgb}{0.121569,0.466667,0.705882}%
\pgfsetstrokecolor{currentstroke}%
\pgfsetdash{}{0pt}%
\pgfpathmoveto{\pgfqpoint{5.043923in}{1.122507in}}%
\pgfpathlineto{\pgfqpoint{5.043923in}{1.122651in}}%
\pgfusepath{stroke}%
\end{pgfscope}%
\begin{pgfscope}%
\pgfpathrectangle{\pgfqpoint{1.286132in}{0.839159in}}{\pgfqpoint{12.053712in}{5.967710in}}%
\pgfusepath{clip}%
\pgfsetbuttcap%
\pgfsetroundjoin%
\pgfsetlinewidth{1.505625pt}%
\definecolor{currentstroke}{rgb}{0.121569,0.466667,0.705882}%
\pgfsetstrokecolor{currentstroke}%
\pgfsetdash{}{0pt}%
\pgfpathmoveto{\pgfqpoint{5.154609in}{1.122530in}}%
\pgfpathlineto{\pgfqpoint{5.154609in}{1.123943in}}%
\pgfusepath{stroke}%
\end{pgfscope}%
\begin{pgfscope}%
\pgfpathrectangle{\pgfqpoint{1.286132in}{0.839159in}}{\pgfqpoint{12.053712in}{5.967710in}}%
\pgfusepath{clip}%
\pgfsetbuttcap%
\pgfsetroundjoin%
\pgfsetlinewidth{1.505625pt}%
\definecolor{currentstroke}{rgb}{0.121569,0.466667,0.705882}%
\pgfsetstrokecolor{currentstroke}%
\pgfsetdash{}{0pt}%
\pgfpathmoveto{\pgfqpoint{5.265296in}{1.122818in}}%
\pgfpathlineto{\pgfqpoint{5.265296in}{1.122914in}}%
\pgfusepath{stroke}%
\end{pgfscope}%
\begin{pgfscope}%
\pgfpathrectangle{\pgfqpoint{1.286132in}{0.839159in}}{\pgfqpoint{12.053712in}{5.967710in}}%
\pgfusepath{clip}%
\pgfsetbuttcap%
\pgfsetroundjoin%
\pgfsetlinewidth{1.505625pt}%
\definecolor{currentstroke}{rgb}{0.121569,0.466667,0.705882}%
\pgfsetstrokecolor{currentstroke}%
\pgfsetdash{}{0pt}%
\pgfpathmoveto{\pgfqpoint{5.375982in}{1.123069in}}%
\pgfpathlineto{\pgfqpoint{5.375982in}{1.123170in}}%
\pgfusepath{stroke}%
\end{pgfscope}%
\begin{pgfscope}%
\pgfpathrectangle{\pgfqpoint{1.286132in}{0.839159in}}{\pgfqpoint{12.053712in}{5.967710in}}%
\pgfusepath{clip}%
\pgfsetbuttcap%
\pgfsetroundjoin%
\pgfsetlinewidth{1.505625pt}%
\definecolor{currentstroke}{rgb}{0.121569,0.466667,0.705882}%
\pgfsetstrokecolor{currentstroke}%
\pgfsetdash{}{0pt}%
\pgfpathmoveto{\pgfqpoint{5.486668in}{1.122890in}}%
\pgfpathlineto{\pgfqpoint{5.486668in}{1.124483in}}%
\pgfusepath{stroke}%
\end{pgfscope}%
\begin{pgfscope}%
\pgfpathrectangle{\pgfqpoint{1.286132in}{0.839159in}}{\pgfqpoint{12.053712in}{5.967710in}}%
\pgfusepath{clip}%
\pgfsetbuttcap%
\pgfsetroundjoin%
\pgfsetlinewidth{1.505625pt}%
\definecolor{currentstroke}{rgb}{0.121569,0.466667,0.705882}%
\pgfsetstrokecolor{currentstroke}%
\pgfsetdash{}{0pt}%
\pgfpathmoveto{\pgfqpoint{5.597354in}{1.123417in}}%
\pgfpathlineto{\pgfqpoint{5.597354in}{1.123579in}}%
\pgfusepath{stroke}%
\end{pgfscope}%
\begin{pgfscope}%
\pgfpathrectangle{\pgfqpoint{1.286132in}{0.839159in}}{\pgfqpoint{12.053712in}{5.967710in}}%
\pgfusepath{clip}%
\pgfsetbuttcap%
\pgfsetroundjoin%
\pgfsetlinewidth{1.505625pt}%
\definecolor{currentstroke}{rgb}{0.121569,0.466667,0.705882}%
\pgfsetstrokecolor{currentstroke}%
\pgfsetdash{}{0pt}%
\pgfpathmoveto{\pgfqpoint{5.708040in}{1.123666in}}%
\pgfpathlineto{\pgfqpoint{5.708040in}{1.123898in}}%
\pgfusepath{stroke}%
\end{pgfscope}%
\begin{pgfscope}%
\pgfpathrectangle{\pgfqpoint{1.286132in}{0.839159in}}{\pgfqpoint{12.053712in}{5.967710in}}%
\pgfusepath{clip}%
\pgfsetbuttcap%
\pgfsetroundjoin%
\pgfsetlinewidth{1.505625pt}%
\definecolor{currentstroke}{rgb}{0.121569,0.466667,0.705882}%
\pgfsetstrokecolor{currentstroke}%
\pgfsetdash{}{0pt}%
\pgfpathmoveto{\pgfqpoint{5.818726in}{1.123815in}}%
\pgfpathlineto{\pgfqpoint{5.818726in}{1.124018in}}%
\pgfusepath{stroke}%
\end{pgfscope}%
\begin{pgfscope}%
\pgfpathrectangle{\pgfqpoint{1.286132in}{0.839159in}}{\pgfqpoint{12.053712in}{5.967710in}}%
\pgfusepath{clip}%
\pgfsetbuttcap%
\pgfsetroundjoin%
\pgfsetlinewidth{1.505625pt}%
\definecolor{currentstroke}{rgb}{0.121569,0.466667,0.705882}%
\pgfsetstrokecolor{currentstroke}%
\pgfsetdash{}{0pt}%
\pgfpathmoveto{\pgfqpoint{5.929412in}{1.124175in}}%
\pgfpathlineto{\pgfqpoint{5.929412in}{1.124588in}}%
\pgfusepath{stroke}%
\end{pgfscope}%
\begin{pgfscope}%
\pgfpathrectangle{\pgfqpoint{1.286132in}{0.839159in}}{\pgfqpoint{12.053712in}{5.967710in}}%
\pgfusepath{clip}%
\pgfsetbuttcap%
\pgfsetroundjoin%
\pgfsetlinewidth{1.505625pt}%
\definecolor{currentstroke}{rgb}{0.121569,0.466667,0.705882}%
\pgfsetstrokecolor{currentstroke}%
\pgfsetdash{}{0pt}%
\pgfpathmoveto{\pgfqpoint{6.040098in}{1.124279in}}%
\pgfpathlineto{\pgfqpoint{6.040098in}{1.124585in}}%
\pgfusepath{stroke}%
\end{pgfscope}%
\begin{pgfscope}%
\pgfpathrectangle{\pgfqpoint{1.286132in}{0.839159in}}{\pgfqpoint{12.053712in}{5.967710in}}%
\pgfusepath{clip}%
\pgfsetbuttcap%
\pgfsetroundjoin%
\pgfsetlinewidth{1.505625pt}%
\definecolor{currentstroke}{rgb}{0.121569,0.466667,0.705882}%
\pgfsetstrokecolor{currentstroke}%
\pgfsetdash{}{0pt}%
\pgfpathmoveto{\pgfqpoint{6.150784in}{1.124405in}}%
\pgfpathlineto{\pgfqpoint{6.150784in}{1.125001in}}%
\pgfusepath{stroke}%
\end{pgfscope}%
\begin{pgfscope}%
\pgfpathrectangle{\pgfqpoint{1.286132in}{0.839159in}}{\pgfqpoint{12.053712in}{5.967710in}}%
\pgfusepath{clip}%
\pgfsetbuttcap%
\pgfsetroundjoin%
\pgfsetlinewidth{1.505625pt}%
\definecolor{currentstroke}{rgb}{0.121569,0.466667,0.705882}%
\pgfsetstrokecolor{currentstroke}%
\pgfsetdash{}{0pt}%
\pgfpathmoveto{\pgfqpoint{6.261470in}{1.124706in}}%
\pgfpathlineto{\pgfqpoint{6.261470in}{1.124751in}}%
\pgfusepath{stroke}%
\end{pgfscope}%
\begin{pgfscope}%
\pgfpathrectangle{\pgfqpoint{1.286132in}{0.839159in}}{\pgfqpoint{12.053712in}{5.967710in}}%
\pgfusepath{clip}%
\pgfsetbuttcap%
\pgfsetroundjoin%
\pgfsetlinewidth{1.505625pt}%
\definecolor{currentstroke}{rgb}{0.121569,0.466667,0.705882}%
\pgfsetstrokecolor{currentstroke}%
\pgfsetdash{}{0pt}%
\pgfpathmoveto{\pgfqpoint{6.372156in}{1.124873in}}%
\pgfpathlineto{\pgfqpoint{6.372156in}{1.125140in}}%
\pgfusepath{stroke}%
\end{pgfscope}%
\begin{pgfscope}%
\pgfpathrectangle{\pgfqpoint{1.286132in}{0.839159in}}{\pgfqpoint{12.053712in}{5.967710in}}%
\pgfusepath{clip}%
\pgfsetbuttcap%
\pgfsetroundjoin%
\pgfsetlinewidth{1.505625pt}%
\definecolor{currentstroke}{rgb}{0.121569,0.466667,0.705882}%
\pgfsetstrokecolor{currentstroke}%
\pgfsetdash{}{0pt}%
\pgfpathmoveto{\pgfqpoint{6.482842in}{1.125130in}}%
\pgfpathlineto{\pgfqpoint{6.482842in}{1.125372in}}%
\pgfusepath{stroke}%
\end{pgfscope}%
\begin{pgfscope}%
\pgfpathrectangle{\pgfqpoint{1.286132in}{0.839159in}}{\pgfqpoint{12.053712in}{5.967710in}}%
\pgfusepath{clip}%
\pgfsetbuttcap%
\pgfsetroundjoin%
\pgfsetlinewidth{1.505625pt}%
\definecolor{currentstroke}{rgb}{0.121569,0.466667,0.705882}%
\pgfsetstrokecolor{currentstroke}%
\pgfsetdash{}{0pt}%
\pgfpathmoveto{\pgfqpoint{6.593528in}{1.125298in}}%
\pgfpathlineto{\pgfqpoint{6.593528in}{1.125850in}}%
\pgfusepath{stroke}%
\end{pgfscope}%
\begin{pgfscope}%
\pgfpathrectangle{\pgfqpoint{1.286132in}{0.839159in}}{\pgfqpoint{12.053712in}{5.967710in}}%
\pgfusepath{clip}%
\pgfsetbuttcap%
\pgfsetroundjoin%
\pgfsetlinewidth{1.505625pt}%
\definecolor{currentstroke}{rgb}{0.121569,0.466667,0.705882}%
\pgfsetstrokecolor{currentstroke}%
\pgfsetdash{}{0pt}%
\pgfpathmoveto{\pgfqpoint{6.704214in}{1.125564in}}%
\pgfpathlineto{\pgfqpoint{6.704214in}{1.125724in}}%
\pgfusepath{stroke}%
\end{pgfscope}%
\begin{pgfscope}%
\pgfpathrectangle{\pgfqpoint{1.286132in}{0.839159in}}{\pgfqpoint{12.053712in}{5.967710in}}%
\pgfusepath{clip}%
\pgfsetbuttcap%
\pgfsetroundjoin%
\pgfsetlinewidth{1.505625pt}%
\definecolor{currentstroke}{rgb}{0.121569,0.466667,0.705882}%
\pgfsetstrokecolor{currentstroke}%
\pgfsetdash{}{0pt}%
\pgfpathmoveto{\pgfqpoint{6.814900in}{1.125939in}}%
\pgfpathlineto{\pgfqpoint{6.814900in}{1.128079in}}%
\pgfusepath{stroke}%
\end{pgfscope}%
\begin{pgfscope}%
\pgfpathrectangle{\pgfqpoint{1.286132in}{0.839159in}}{\pgfqpoint{12.053712in}{5.967710in}}%
\pgfusepath{clip}%
\pgfsetbuttcap%
\pgfsetroundjoin%
\pgfsetlinewidth{1.505625pt}%
\definecolor{currentstroke}{rgb}{0.121569,0.466667,0.705882}%
\pgfsetstrokecolor{currentstroke}%
\pgfsetdash{}{0pt}%
\pgfpathmoveto{\pgfqpoint{6.925586in}{1.125783in}}%
\pgfpathlineto{\pgfqpoint{6.925586in}{1.127222in}}%
\pgfusepath{stroke}%
\end{pgfscope}%
\begin{pgfscope}%
\pgfpathrectangle{\pgfqpoint{1.286132in}{0.839159in}}{\pgfqpoint{12.053712in}{5.967710in}}%
\pgfusepath{clip}%
\pgfsetbuttcap%
\pgfsetroundjoin%
\pgfsetlinewidth{1.505625pt}%
\definecolor{currentstroke}{rgb}{0.121569,0.466667,0.705882}%
\pgfsetstrokecolor{currentstroke}%
\pgfsetdash{}{0pt}%
\pgfpathmoveto{\pgfqpoint{7.036272in}{1.126279in}}%
\pgfpathlineto{\pgfqpoint{7.036272in}{1.126701in}}%
\pgfusepath{stroke}%
\end{pgfscope}%
\begin{pgfscope}%
\pgfpathrectangle{\pgfqpoint{1.286132in}{0.839159in}}{\pgfqpoint{12.053712in}{5.967710in}}%
\pgfusepath{clip}%
\pgfsetbuttcap%
\pgfsetroundjoin%
\pgfsetlinewidth{1.505625pt}%
\definecolor{currentstroke}{rgb}{0.121569,0.466667,0.705882}%
\pgfsetstrokecolor{currentstroke}%
\pgfsetdash{}{0pt}%
\pgfpathmoveto{\pgfqpoint{7.146959in}{1.126694in}}%
\pgfpathlineto{\pgfqpoint{7.146959in}{1.126850in}}%
\pgfusepath{stroke}%
\end{pgfscope}%
\begin{pgfscope}%
\pgfpathrectangle{\pgfqpoint{1.286132in}{0.839159in}}{\pgfqpoint{12.053712in}{5.967710in}}%
\pgfusepath{clip}%
\pgfsetbuttcap%
\pgfsetroundjoin%
\pgfsetlinewidth{1.505625pt}%
\definecolor{currentstroke}{rgb}{0.121569,0.466667,0.705882}%
\pgfsetstrokecolor{currentstroke}%
\pgfsetdash{}{0pt}%
\pgfpathmoveto{\pgfqpoint{7.257645in}{1.126893in}}%
\pgfpathlineto{\pgfqpoint{7.257645in}{1.127389in}}%
\pgfusepath{stroke}%
\end{pgfscope}%
\begin{pgfscope}%
\pgfpathrectangle{\pgfqpoint{1.286132in}{0.839159in}}{\pgfqpoint{12.053712in}{5.967710in}}%
\pgfusepath{clip}%
\pgfsetbuttcap%
\pgfsetroundjoin%
\pgfsetlinewidth{1.505625pt}%
\definecolor{currentstroke}{rgb}{0.121569,0.466667,0.705882}%
\pgfsetstrokecolor{currentstroke}%
\pgfsetdash{}{0pt}%
\pgfpathmoveto{\pgfqpoint{7.368331in}{1.127210in}}%
\pgfpathlineto{\pgfqpoint{7.368331in}{1.127475in}}%
\pgfusepath{stroke}%
\end{pgfscope}%
\begin{pgfscope}%
\pgfpathrectangle{\pgfqpoint{1.286132in}{0.839159in}}{\pgfqpoint{12.053712in}{5.967710in}}%
\pgfusepath{clip}%
\pgfsetbuttcap%
\pgfsetroundjoin%
\pgfsetlinewidth{1.505625pt}%
\definecolor{currentstroke}{rgb}{0.121569,0.466667,0.705882}%
\pgfsetstrokecolor{currentstroke}%
\pgfsetdash{}{0pt}%
\pgfpathmoveto{\pgfqpoint{7.479017in}{1.127298in}}%
\pgfpathlineto{\pgfqpoint{7.479017in}{1.128015in}}%
\pgfusepath{stroke}%
\end{pgfscope}%
\begin{pgfscope}%
\pgfpathrectangle{\pgfqpoint{1.286132in}{0.839159in}}{\pgfqpoint{12.053712in}{5.967710in}}%
\pgfusepath{clip}%
\pgfsetbuttcap%
\pgfsetroundjoin%
\pgfsetlinewidth{1.505625pt}%
\definecolor{currentstroke}{rgb}{0.121569,0.466667,0.705882}%
\pgfsetstrokecolor{currentstroke}%
\pgfsetdash{}{0pt}%
\pgfpathmoveto{\pgfqpoint{7.589703in}{1.127641in}}%
\pgfpathlineto{\pgfqpoint{7.589703in}{1.129320in}}%
\pgfusepath{stroke}%
\end{pgfscope}%
\begin{pgfscope}%
\pgfpathrectangle{\pgfqpoint{1.286132in}{0.839159in}}{\pgfqpoint{12.053712in}{5.967710in}}%
\pgfusepath{clip}%
\pgfsetbuttcap%
\pgfsetroundjoin%
\pgfsetlinewidth{1.505625pt}%
\definecolor{currentstroke}{rgb}{0.121569,0.466667,0.705882}%
\pgfsetstrokecolor{currentstroke}%
\pgfsetdash{}{0pt}%
\pgfpathmoveto{\pgfqpoint{7.700389in}{1.127846in}}%
\pgfpathlineto{\pgfqpoint{7.700389in}{1.127937in}}%
\pgfusepath{stroke}%
\end{pgfscope}%
\begin{pgfscope}%
\pgfpathrectangle{\pgfqpoint{1.286132in}{0.839159in}}{\pgfqpoint{12.053712in}{5.967710in}}%
\pgfusepath{clip}%
\pgfsetbuttcap%
\pgfsetroundjoin%
\pgfsetlinewidth{1.505625pt}%
\definecolor{currentstroke}{rgb}{0.121569,0.466667,0.705882}%
\pgfsetstrokecolor{currentstroke}%
\pgfsetdash{}{0pt}%
\pgfpathmoveto{\pgfqpoint{7.811075in}{1.128141in}}%
\pgfpathlineto{\pgfqpoint{7.811075in}{1.128874in}}%
\pgfusepath{stroke}%
\end{pgfscope}%
\begin{pgfscope}%
\pgfpathrectangle{\pgfqpoint{1.286132in}{0.839159in}}{\pgfqpoint{12.053712in}{5.967710in}}%
\pgfusepath{clip}%
\pgfsetbuttcap%
\pgfsetroundjoin%
\pgfsetlinewidth{1.505625pt}%
\definecolor{currentstroke}{rgb}{0.121569,0.466667,0.705882}%
\pgfsetstrokecolor{currentstroke}%
\pgfsetdash{}{0pt}%
\pgfpathmoveto{\pgfqpoint{7.921761in}{1.128415in}}%
\pgfpathlineto{\pgfqpoint{7.921761in}{1.128913in}}%
\pgfusepath{stroke}%
\end{pgfscope}%
\begin{pgfscope}%
\pgfpathrectangle{\pgfqpoint{1.286132in}{0.839159in}}{\pgfqpoint{12.053712in}{5.967710in}}%
\pgfusepath{clip}%
\pgfsetbuttcap%
\pgfsetroundjoin%
\pgfsetlinewidth{1.505625pt}%
\definecolor{currentstroke}{rgb}{0.121569,0.466667,0.705882}%
\pgfsetstrokecolor{currentstroke}%
\pgfsetdash{}{0pt}%
\pgfpathmoveto{\pgfqpoint{8.032447in}{1.128592in}}%
\pgfpathlineto{\pgfqpoint{8.032447in}{1.129704in}}%
\pgfusepath{stroke}%
\end{pgfscope}%
\begin{pgfscope}%
\pgfpathrectangle{\pgfqpoint{1.286132in}{0.839159in}}{\pgfqpoint{12.053712in}{5.967710in}}%
\pgfusepath{clip}%
\pgfsetbuttcap%
\pgfsetroundjoin%
\pgfsetlinewidth{1.505625pt}%
\definecolor{currentstroke}{rgb}{0.121569,0.466667,0.705882}%
\pgfsetstrokecolor{currentstroke}%
\pgfsetdash{}{0pt}%
\pgfpathmoveto{\pgfqpoint{8.143133in}{1.128951in}}%
\pgfpathlineto{\pgfqpoint{8.143133in}{1.129464in}}%
\pgfusepath{stroke}%
\end{pgfscope}%
\begin{pgfscope}%
\pgfpathrectangle{\pgfqpoint{1.286132in}{0.839159in}}{\pgfqpoint{12.053712in}{5.967710in}}%
\pgfusepath{clip}%
\pgfsetbuttcap%
\pgfsetroundjoin%
\pgfsetlinewidth{1.505625pt}%
\definecolor{currentstroke}{rgb}{0.121569,0.466667,0.705882}%
\pgfsetstrokecolor{currentstroke}%
\pgfsetdash{}{0pt}%
\pgfpathmoveto{\pgfqpoint{8.253819in}{1.129328in}}%
\pgfpathlineto{\pgfqpoint{8.253819in}{1.129860in}}%
\pgfusepath{stroke}%
\end{pgfscope}%
\begin{pgfscope}%
\pgfpathrectangle{\pgfqpoint{1.286132in}{0.839159in}}{\pgfqpoint{12.053712in}{5.967710in}}%
\pgfusepath{clip}%
\pgfsetbuttcap%
\pgfsetroundjoin%
\pgfsetlinewidth{1.505625pt}%
\definecolor{currentstroke}{rgb}{0.121569,0.466667,0.705882}%
\pgfsetstrokecolor{currentstroke}%
\pgfsetdash{}{0pt}%
\pgfpathmoveto{\pgfqpoint{8.364505in}{1.129898in}}%
\pgfpathlineto{\pgfqpoint{8.364505in}{1.129998in}}%
\pgfusepath{stroke}%
\end{pgfscope}%
\begin{pgfscope}%
\pgfpathrectangle{\pgfqpoint{1.286132in}{0.839159in}}{\pgfqpoint{12.053712in}{5.967710in}}%
\pgfusepath{clip}%
\pgfsetbuttcap%
\pgfsetroundjoin%
\pgfsetlinewidth{1.505625pt}%
\definecolor{currentstroke}{rgb}{0.121569,0.466667,0.705882}%
\pgfsetstrokecolor{currentstroke}%
\pgfsetdash{}{0pt}%
\pgfpathmoveto{\pgfqpoint{8.475191in}{1.130223in}}%
\pgfpathlineto{\pgfqpoint{8.475191in}{1.131827in}}%
\pgfusepath{stroke}%
\end{pgfscope}%
\begin{pgfscope}%
\pgfpathrectangle{\pgfqpoint{1.286132in}{0.839159in}}{\pgfqpoint{12.053712in}{5.967710in}}%
\pgfusepath{clip}%
\pgfsetbuttcap%
\pgfsetroundjoin%
\pgfsetlinewidth{1.505625pt}%
\definecolor{currentstroke}{rgb}{0.121569,0.466667,0.705882}%
\pgfsetstrokecolor{currentstroke}%
\pgfsetdash{}{0pt}%
\pgfpathmoveto{\pgfqpoint{8.585877in}{1.130480in}}%
\pgfpathlineto{\pgfqpoint{8.585877in}{1.130770in}}%
\pgfusepath{stroke}%
\end{pgfscope}%
\begin{pgfscope}%
\pgfpathrectangle{\pgfqpoint{1.286132in}{0.839159in}}{\pgfqpoint{12.053712in}{5.967710in}}%
\pgfusepath{clip}%
\pgfsetbuttcap%
\pgfsetroundjoin%
\pgfsetlinewidth{1.505625pt}%
\definecolor{currentstroke}{rgb}{0.121569,0.466667,0.705882}%
\pgfsetstrokecolor{currentstroke}%
\pgfsetdash{}{0pt}%
\pgfpathmoveto{\pgfqpoint{8.696563in}{1.130840in}}%
\pgfpathlineto{\pgfqpoint{8.696563in}{1.131242in}}%
\pgfusepath{stroke}%
\end{pgfscope}%
\begin{pgfscope}%
\pgfpathrectangle{\pgfqpoint{1.286132in}{0.839159in}}{\pgfqpoint{12.053712in}{5.967710in}}%
\pgfusepath{clip}%
\pgfsetbuttcap%
\pgfsetroundjoin%
\pgfsetlinewidth{1.505625pt}%
\definecolor{currentstroke}{rgb}{0.121569,0.466667,0.705882}%
\pgfsetstrokecolor{currentstroke}%
\pgfsetdash{}{0pt}%
\pgfpathmoveto{\pgfqpoint{8.807249in}{1.130996in}}%
\pgfpathlineto{\pgfqpoint{8.807249in}{1.131752in}}%
\pgfusepath{stroke}%
\end{pgfscope}%
\begin{pgfscope}%
\pgfpathrectangle{\pgfqpoint{1.286132in}{0.839159in}}{\pgfqpoint{12.053712in}{5.967710in}}%
\pgfusepath{clip}%
\pgfsetbuttcap%
\pgfsetroundjoin%
\pgfsetlinewidth{1.505625pt}%
\definecolor{currentstroke}{rgb}{0.121569,0.466667,0.705882}%
\pgfsetstrokecolor{currentstroke}%
\pgfsetdash{}{0pt}%
\pgfpathmoveto{\pgfqpoint{8.917936in}{1.131089in}}%
\pgfpathlineto{\pgfqpoint{8.917936in}{1.131796in}}%
\pgfusepath{stroke}%
\end{pgfscope}%
\begin{pgfscope}%
\pgfpathrectangle{\pgfqpoint{1.286132in}{0.839159in}}{\pgfqpoint{12.053712in}{5.967710in}}%
\pgfusepath{clip}%
\pgfsetbuttcap%
\pgfsetroundjoin%
\pgfsetlinewidth{1.505625pt}%
\definecolor{currentstroke}{rgb}{0.121569,0.466667,0.705882}%
\pgfsetstrokecolor{currentstroke}%
\pgfsetdash{}{0pt}%
\pgfpathmoveto{\pgfqpoint{9.028622in}{1.131490in}}%
\pgfpathlineto{\pgfqpoint{9.028622in}{1.131935in}}%
\pgfusepath{stroke}%
\end{pgfscope}%
\begin{pgfscope}%
\pgfpathrectangle{\pgfqpoint{1.286132in}{0.839159in}}{\pgfqpoint{12.053712in}{5.967710in}}%
\pgfusepath{clip}%
\pgfsetbuttcap%
\pgfsetroundjoin%
\pgfsetlinewidth{1.505625pt}%
\definecolor{currentstroke}{rgb}{0.121569,0.466667,0.705882}%
\pgfsetstrokecolor{currentstroke}%
\pgfsetdash{}{0pt}%
\pgfpathmoveto{\pgfqpoint{9.139308in}{1.131947in}}%
\pgfpathlineto{\pgfqpoint{9.139308in}{1.132495in}}%
\pgfusepath{stroke}%
\end{pgfscope}%
\begin{pgfscope}%
\pgfpathrectangle{\pgfqpoint{1.286132in}{0.839159in}}{\pgfqpoint{12.053712in}{5.967710in}}%
\pgfusepath{clip}%
\pgfsetbuttcap%
\pgfsetroundjoin%
\pgfsetlinewidth{1.505625pt}%
\definecolor{currentstroke}{rgb}{0.121569,0.466667,0.705882}%
\pgfsetstrokecolor{currentstroke}%
\pgfsetdash{}{0pt}%
\pgfpathmoveto{\pgfqpoint{9.249994in}{1.132177in}}%
\pgfpathlineto{\pgfqpoint{9.249994in}{1.132980in}}%
\pgfusepath{stroke}%
\end{pgfscope}%
\begin{pgfscope}%
\pgfpathrectangle{\pgfqpoint{1.286132in}{0.839159in}}{\pgfqpoint{12.053712in}{5.967710in}}%
\pgfusepath{clip}%
\pgfsetbuttcap%
\pgfsetroundjoin%
\pgfsetlinewidth{1.505625pt}%
\definecolor{currentstroke}{rgb}{0.121569,0.466667,0.705882}%
\pgfsetstrokecolor{currentstroke}%
\pgfsetdash{}{0pt}%
\pgfpathmoveto{\pgfqpoint{9.360680in}{1.132495in}}%
\pgfpathlineto{\pgfqpoint{9.360680in}{1.132756in}}%
\pgfusepath{stroke}%
\end{pgfscope}%
\begin{pgfscope}%
\pgfpathrectangle{\pgfqpoint{1.286132in}{0.839159in}}{\pgfqpoint{12.053712in}{5.967710in}}%
\pgfusepath{clip}%
\pgfsetbuttcap%
\pgfsetroundjoin%
\pgfsetlinewidth{1.505625pt}%
\definecolor{currentstroke}{rgb}{0.121569,0.466667,0.705882}%
\pgfsetstrokecolor{currentstroke}%
\pgfsetdash{}{0pt}%
\pgfpathmoveto{\pgfqpoint{9.471366in}{1.132919in}}%
\pgfpathlineto{\pgfqpoint{9.471366in}{1.133704in}}%
\pgfusepath{stroke}%
\end{pgfscope}%
\begin{pgfscope}%
\pgfpathrectangle{\pgfqpoint{1.286132in}{0.839159in}}{\pgfqpoint{12.053712in}{5.967710in}}%
\pgfusepath{clip}%
\pgfsetbuttcap%
\pgfsetroundjoin%
\pgfsetlinewidth{1.505625pt}%
\definecolor{currentstroke}{rgb}{0.121569,0.466667,0.705882}%
\pgfsetstrokecolor{currentstroke}%
\pgfsetdash{}{0pt}%
\pgfpathmoveto{\pgfqpoint{9.582052in}{1.133175in}}%
\pgfpathlineto{\pgfqpoint{9.582052in}{1.133728in}}%
\pgfusepath{stroke}%
\end{pgfscope}%
\begin{pgfscope}%
\pgfpathrectangle{\pgfqpoint{1.286132in}{0.839159in}}{\pgfqpoint{12.053712in}{5.967710in}}%
\pgfusepath{clip}%
\pgfsetbuttcap%
\pgfsetroundjoin%
\pgfsetlinewidth{1.505625pt}%
\definecolor{currentstroke}{rgb}{0.121569,0.466667,0.705882}%
\pgfsetstrokecolor{currentstroke}%
\pgfsetdash{}{0pt}%
\pgfpathmoveto{\pgfqpoint{9.692738in}{1.133670in}}%
\pgfpathlineto{\pgfqpoint{9.692738in}{1.134321in}}%
\pgfusepath{stroke}%
\end{pgfscope}%
\begin{pgfscope}%
\pgfpathrectangle{\pgfqpoint{1.286132in}{0.839159in}}{\pgfqpoint{12.053712in}{5.967710in}}%
\pgfusepath{clip}%
\pgfsetbuttcap%
\pgfsetroundjoin%
\pgfsetlinewidth{1.505625pt}%
\definecolor{currentstroke}{rgb}{0.121569,0.466667,0.705882}%
\pgfsetstrokecolor{currentstroke}%
\pgfsetdash{}{0pt}%
\pgfpathmoveto{\pgfqpoint{9.803424in}{1.133948in}}%
\pgfpathlineto{\pgfqpoint{9.803424in}{1.134257in}}%
\pgfusepath{stroke}%
\end{pgfscope}%
\begin{pgfscope}%
\pgfpathrectangle{\pgfqpoint{1.286132in}{0.839159in}}{\pgfqpoint{12.053712in}{5.967710in}}%
\pgfusepath{clip}%
\pgfsetbuttcap%
\pgfsetroundjoin%
\pgfsetlinewidth{1.505625pt}%
\definecolor{currentstroke}{rgb}{0.121569,0.466667,0.705882}%
\pgfsetstrokecolor{currentstroke}%
\pgfsetdash{}{0pt}%
\pgfpathmoveto{\pgfqpoint{9.914110in}{1.126774in}}%
\pgfpathlineto{\pgfqpoint{9.914110in}{1.158454in}}%
\pgfusepath{stroke}%
\end{pgfscope}%
\begin{pgfscope}%
\pgfpathrectangle{\pgfqpoint{1.286132in}{0.839159in}}{\pgfqpoint{12.053712in}{5.967710in}}%
\pgfusepath{clip}%
\pgfsetbuttcap%
\pgfsetroundjoin%
\pgfsetlinewidth{1.505625pt}%
\definecolor{currentstroke}{rgb}{0.121569,0.466667,0.705882}%
\pgfsetstrokecolor{currentstroke}%
\pgfsetdash{}{0pt}%
\pgfpathmoveto{\pgfqpoint{10.024796in}{1.134897in}}%
\pgfpathlineto{\pgfqpoint{10.024796in}{1.135433in}}%
\pgfusepath{stroke}%
\end{pgfscope}%
\begin{pgfscope}%
\pgfpathrectangle{\pgfqpoint{1.286132in}{0.839159in}}{\pgfqpoint{12.053712in}{5.967710in}}%
\pgfusepath{clip}%
\pgfsetbuttcap%
\pgfsetroundjoin%
\pgfsetlinewidth{1.505625pt}%
\definecolor{currentstroke}{rgb}{0.121569,0.466667,0.705882}%
\pgfsetstrokecolor{currentstroke}%
\pgfsetdash{}{0pt}%
\pgfpathmoveto{\pgfqpoint{10.135482in}{1.135273in}}%
\pgfpathlineto{\pgfqpoint{10.135482in}{1.135479in}}%
\pgfusepath{stroke}%
\end{pgfscope}%
\begin{pgfscope}%
\pgfpathrectangle{\pgfqpoint{1.286132in}{0.839159in}}{\pgfqpoint{12.053712in}{5.967710in}}%
\pgfusepath{clip}%
\pgfsetbuttcap%
\pgfsetroundjoin%
\pgfsetlinewidth{1.505625pt}%
\definecolor{currentstroke}{rgb}{0.121569,0.466667,0.705882}%
\pgfsetstrokecolor{currentstroke}%
\pgfsetdash{}{0pt}%
\pgfpathmoveto{\pgfqpoint{10.246168in}{1.135626in}}%
\pgfpathlineto{\pgfqpoint{10.246168in}{1.136000in}}%
\pgfusepath{stroke}%
\end{pgfscope}%
\begin{pgfscope}%
\pgfpathrectangle{\pgfqpoint{1.286132in}{0.839159in}}{\pgfqpoint{12.053712in}{5.967710in}}%
\pgfusepath{clip}%
\pgfsetbuttcap%
\pgfsetroundjoin%
\pgfsetlinewidth{1.505625pt}%
\definecolor{currentstroke}{rgb}{0.121569,0.466667,0.705882}%
\pgfsetstrokecolor{currentstroke}%
\pgfsetdash{}{0pt}%
\pgfpathmoveto{\pgfqpoint{10.356854in}{1.136061in}}%
\pgfpathlineto{\pgfqpoint{10.356854in}{1.136774in}}%
\pgfusepath{stroke}%
\end{pgfscope}%
\begin{pgfscope}%
\pgfpathrectangle{\pgfqpoint{1.286132in}{0.839159in}}{\pgfqpoint{12.053712in}{5.967710in}}%
\pgfusepath{clip}%
\pgfsetbuttcap%
\pgfsetroundjoin%
\pgfsetlinewidth{1.505625pt}%
\definecolor{currentstroke}{rgb}{0.121569,0.466667,0.705882}%
\pgfsetstrokecolor{currentstroke}%
\pgfsetdash{}{0pt}%
\pgfpathmoveto{\pgfqpoint{10.467540in}{1.133903in}}%
\pgfpathlineto{\pgfqpoint{10.467540in}{1.144864in}}%
\pgfusepath{stroke}%
\end{pgfscope}%
\begin{pgfscope}%
\pgfpathrectangle{\pgfqpoint{1.286132in}{0.839159in}}{\pgfqpoint{12.053712in}{5.967710in}}%
\pgfusepath{clip}%
\pgfsetbuttcap%
\pgfsetroundjoin%
\pgfsetlinewidth{1.505625pt}%
\definecolor{currentstroke}{rgb}{0.121569,0.466667,0.705882}%
\pgfsetstrokecolor{currentstroke}%
\pgfsetdash{}{0pt}%
\pgfpathmoveto{\pgfqpoint{10.578226in}{1.136830in}}%
\pgfpathlineto{\pgfqpoint{10.578226in}{1.136992in}}%
\pgfusepath{stroke}%
\end{pgfscope}%
\begin{pgfscope}%
\pgfpathrectangle{\pgfqpoint{1.286132in}{0.839159in}}{\pgfqpoint{12.053712in}{5.967710in}}%
\pgfusepath{clip}%
\pgfsetbuttcap%
\pgfsetroundjoin%
\pgfsetlinewidth{1.505625pt}%
\definecolor{currentstroke}{rgb}{0.121569,0.466667,0.705882}%
\pgfsetstrokecolor{currentstroke}%
\pgfsetdash{}{0pt}%
\pgfpathmoveto{\pgfqpoint{10.688913in}{1.137257in}}%
\pgfpathlineto{\pgfqpoint{10.688913in}{1.137581in}}%
\pgfusepath{stroke}%
\end{pgfscope}%
\begin{pgfscope}%
\pgfpathrectangle{\pgfqpoint{1.286132in}{0.839159in}}{\pgfqpoint{12.053712in}{5.967710in}}%
\pgfusepath{clip}%
\pgfsetbuttcap%
\pgfsetroundjoin%
\pgfsetlinewidth{1.505625pt}%
\definecolor{currentstroke}{rgb}{0.121569,0.466667,0.705882}%
\pgfsetstrokecolor{currentstroke}%
\pgfsetdash{}{0pt}%
\pgfpathmoveto{\pgfqpoint{10.799599in}{1.137713in}}%
\pgfpathlineto{\pgfqpoint{10.799599in}{1.138135in}}%
\pgfusepath{stroke}%
\end{pgfscope}%
\begin{pgfscope}%
\pgfpathrectangle{\pgfqpoint{1.286132in}{0.839159in}}{\pgfqpoint{12.053712in}{5.967710in}}%
\pgfusepath{clip}%
\pgfsetbuttcap%
\pgfsetroundjoin%
\pgfsetlinewidth{1.505625pt}%
\definecolor{currentstroke}{rgb}{0.121569,0.466667,0.705882}%
\pgfsetstrokecolor{currentstroke}%
\pgfsetdash{}{0pt}%
\pgfpathmoveto{\pgfqpoint{10.910285in}{1.138267in}}%
\pgfpathlineto{\pgfqpoint{10.910285in}{1.138966in}}%
\pgfusepath{stroke}%
\end{pgfscope}%
\begin{pgfscope}%
\pgfpathrectangle{\pgfqpoint{1.286132in}{0.839159in}}{\pgfqpoint{12.053712in}{5.967710in}}%
\pgfusepath{clip}%
\pgfsetbuttcap%
\pgfsetroundjoin%
\pgfsetlinewidth{1.505625pt}%
\definecolor{currentstroke}{rgb}{0.121569,0.466667,0.705882}%
\pgfsetstrokecolor{currentstroke}%
\pgfsetdash{}{0pt}%
\pgfpathmoveto{\pgfqpoint{11.020971in}{1.138520in}}%
\pgfpathlineto{\pgfqpoint{11.020971in}{1.139451in}}%
\pgfusepath{stroke}%
\end{pgfscope}%
\begin{pgfscope}%
\pgfpathrectangle{\pgfqpoint{1.286132in}{0.839159in}}{\pgfqpoint{12.053712in}{5.967710in}}%
\pgfusepath{clip}%
\pgfsetbuttcap%
\pgfsetroundjoin%
\pgfsetlinewidth{1.505625pt}%
\definecolor{currentstroke}{rgb}{0.121569,0.466667,0.705882}%
\pgfsetstrokecolor{currentstroke}%
\pgfsetdash{}{0pt}%
\pgfpathmoveto{\pgfqpoint{11.131657in}{1.139085in}}%
\pgfpathlineto{\pgfqpoint{11.131657in}{1.139721in}}%
\pgfusepath{stroke}%
\end{pgfscope}%
\begin{pgfscope}%
\pgfpathrectangle{\pgfqpoint{1.286132in}{0.839159in}}{\pgfqpoint{12.053712in}{5.967710in}}%
\pgfusepath{clip}%
\pgfsetbuttcap%
\pgfsetroundjoin%
\pgfsetlinewidth{1.505625pt}%
\definecolor{currentstroke}{rgb}{0.121569,0.466667,0.705882}%
\pgfsetstrokecolor{currentstroke}%
\pgfsetdash{}{0pt}%
\pgfpathmoveto{\pgfqpoint{11.242343in}{1.139416in}}%
\pgfpathlineto{\pgfqpoint{11.242343in}{1.140014in}}%
\pgfusepath{stroke}%
\end{pgfscope}%
\begin{pgfscope}%
\pgfpathrectangle{\pgfqpoint{1.286132in}{0.839159in}}{\pgfqpoint{12.053712in}{5.967710in}}%
\pgfusepath{clip}%
\pgfsetbuttcap%
\pgfsetroundjoin%
\pgfsetlinewidth{1.505625pt}%
\definecolor{currentstroke}{rgb}{0.121569,0.466667,0.705882}%
\pgfsetstrokecolor{currentstroke}%
\pgfsetdash{}{0pt}%
\pgfpathmoveto{\pgfqpoint{11.353029in}{1.139765in}}%
\pgfpathlineto{\pgfqpoint{11.353029in}{1.140318in}}%
\pgfusepath{stroke}%
\end{pgfscope}%
\begin{pgfscope}%
\pgfpathrectangle{\pgfqpoint{1.286132in}{0.839159in}}{\pgfqpoint{12.053712in}{5.967710in}}%
\pgfusepath{clip}%
\pgfsetbuttcap%
\pgfsetroundjoin%
\pgfsetlinewidth{1.505625pt}%
\definecolor{currentstroke}{rgb}{0.121569,0.466667,0.705882}%
\pgfsetstrokecolor{currentstroke}%
\pgfsetdash{}{0pt}%
\pgfpathmoveto{\pgfqpoint{11.463715in}{1.140193in}}%
\pgfpathlineto{\pgfqpoint{11.463715in}{1.140571in}}%
\pgfusepath{stroke}%
\end{pgfscope}%
\begin{pgfscope}%
\pgfpathrectangle{\pgfqpoint{1.286132in}{0.839159in}}{\pgfqpoint{12.053712in}{5.967710in}}%
\pgfusepath{clip}%
\pgfsetbuttcap%
\pgfsetroundjoin%
\pgfsetlinewidth{1.505625pt}%
\definecolor{currentstroke}{rgb}{0.121569,0.466667,0.705882}%
\pgfsetstrokecolor{currentstroke}%
\pgfsetdash{}{0pt}%
\pgfpathmoveto{\pgfqpoint{11.574401in}{1.140563in}}%
\pgfpathlineto{\pgfqpoint{11.574401in}{1.141246in}}%
\pgfusepath{stroke}%
\end{pgfscope}%
\begin{pgfscope}%
\pgfpathrectangle{\pgfqpoint{1.286132in}{0.839159in}}{\pgfqpoint{12.053712in}{5.967710in}}%
\pgfusepath{clip}%
\pgfsetbuttcap%
\pgfsetroundjoin%
\pgfsetlinewidth{1.505625pt}%
\definecolor{currentstroke}{rgb}{0.121569,0.466667,0.705882}%
\pgfsetstrokecolor{currentstroke}%
\pgfsetdash{}{0pt}%
\pgfpathmoveto{\pgfqpoint{11.685087in}{1.141152in}}%
\pgfpathlineto{\pgfqpoint{11.685087in}{1.141659in}}%
\pgfusepath{stroke}%
\end{pgfscope}%
\begin{pgfscope}%
\pgfpathrectangle{\pgfqpoint{1.286132in}{0.839159in}}{\pgfqpoint{12.053712in}{5.967710in}}%
\pgfusepath{clip}%
\pgfsetbuttcap%
\pgfsetroundjoin%
\pgfsetlinewidth{1.505625pt}%
\definecolor{currentstroke}{rgb}{0.121569,0.466667,0.705882}%
\pgfsetstrokecolor{currentstroke}%
\pgfsetdash{}{0pt}%
\pgfpathmoveto{\pgfqpoint{11.795773in}{1.141614in}}%
\pgfpathlineto{\pgfqpoint{11.795773in}{1.141929in}}%
\pgfusepath{stroke}%
\end{pgfscope}%
\begin{pgfscope}%
\pgfpathrectangle{\pgfqpoint{1.286132in}{0.839159in}}{\pgfqpoint{12.053712in}{5.967710in}}%
\pgfusepath{clip}%
\pgfsetbuttcap%
\pgfsetroundjoin%
\pgfsetlinewidth{1.505625pt}%
\definecolor{currentstroke}{rgb}{0.121569,0.466667,0.705882}%
\pgfsetstrokecolor{currentstroke}%
\pgfsetdash{}{0pt}%
\pgfpathmoveto{\pgfqpoint{11.906459in}{1.142105in}}%
\pgfpathlineto{\pgfqpoint{11.906459in}{1.142787in}}%
\pgfusepath{stroke}%
\end{pgfscope}%
\begin{pgfscope}%
\pgfpathrectangle{\pgfqpoint{1.286132in}{0.839159in}}{\pgfqpoint{12.053712in}{5.967710in}}%
\pgfusepath{clip}%
\pgfsetbuttcap%
\pgfsetroundjoin%
\pgfsetlinewidth{1.505625pt}%
\definecolor{currentstroke}{rgb}{0.121569,0.466667,0.705882}%
\pgfsetstrokecolor{currentstroke}%
\pgfsetdash{}{0pt}%
\pgfpathmoveto{\pgfqpoint{12.017145in}{1.142719in}}%
\pgfpathlineto{\pgfqpoint{12.017145in}{1.143028in}}%
\pgfusepath{stroke}%
\end{pgfscope}%
\begin{pgfscope}%
\pgfpathrectangle{\pgfqpoint{1.286132in}{0.839159in}}{\pgfqpoint{12.053712in}{5.967710in}}%
\pgfusepath{clip}%
\pgfsetbuttcap%
\pgfsetroundjoin%
\pgfsetlinewidth{1.505625pt}%
\definecolor{currentstroke}{rgb}{0.121569,0.466667,0.705882}%
\pgfsetstrokecolor{currentstroke}%
\pgfsetdash{}{0pt}%
\pgfpathmoveto{\pgfqpoint{12.127831in}{1.143163in}}%
\pgfpathlineto{\pgfqpoint{12.127831in}{1.143402in}}%
\pgfusepath{stroke}%
\end{pgfscope}%
\begin{pgfscope}%
\pgfpathrectangle{\pgfqpoint{1.286132in}{0.839159in}}{\pgfqpoint{12.053712in}{5.967710in}}%
\pgfusepath{clip}%
\pgfsetbuttcap%
\pgfsetroundjoin%
\pgfsetlinewidth{1.505625pt}%
\definecolor{currentstroke}{rgb}{0.121569,0.466667,0.705882}%
\pgfsetstrokecolor{currentstroke}%
\pgfsetdash{}{0pt}%
\pgfpathmoveto{\pgfqpoint{12.238517in}{1.143618in}}%
\pgfpathlineto{\pgfqpoint{12.238517in}{1.143959in}}%
\pgfusepath{stroke}%
\end{pgfscope}%
\begin{pgfscope}%
\pgfpathrectangle{\pgfqpoint{1.286132in}{0.839159in}}{\pgfqpoint{12.053712in}{5.967710in}}%
\pgfusepath{clip}%
\pgfsetbuttcap%
\pgfsetroundjoin%
\pgfsetlinewidth{1.505625pt}%
\definecolor{currentstroke}{rgb}{0.121569,0.466667,0.705882}%
\pgfsetstrokecolor{currentstroke}%
\pgfsetdash{}{0pt}%
\pgfpathmoveto{\pgfqpoint{12.349203in}{1.143894in}}%
\pgfpathlineto{\pgfqpoint{12.349203in}{1.144177in}}%
\pgfusepath{stroke}%
\end{pgfscope}%
\begin{pgfscope}%
\pgfpathrectangle{\pgfqpoint{1.286132in}{0.839159in}}{\pgfqpoint{12.053712in}{5.967710in}}%
\pgfusepath{clip}%
\pgfsetbuttcap%
\pgfsetroundjoin%
\pgfsetlinewidth{1.505625pt}%
\definecolor{currentstroke}{rgb}{0.121569,0.466667,0.705882}%
\pgfsetstrokecolor{currentstroke}%
\pgfsetdash{}{0pt}%
\pgfpathmoveto{\pgfqpoint{12.459890in}{1.144814in}}%
\pgfpathlineto{\pgfqpoint{12.459890in}{1.145983in}}%
\pgfusepath{stroke}%
\end{pgfscope}%
\begin{pgfscope}%
\pgfpathrectangle{\pgfqpoint{1.286132in}{0.839159in}}{\pgfqpoint{12.053712in}{5.967710in}}%
\pgfusepath{clip}%
\pgfsetbuttcap%
\pgfsetroundjoin%
\pgfsetlinewidth{1.505625pt}%
\definecolor{currentstroke}{rgb}{0.121569,0.466667,0.705882}%
\pgfsetstrokecolor{currentstroke}%
\pgfsetdash{}{0pt}%
\pgfpathmoveto{\pgfqpoint{12.570576in}{1.140483in}}%
\pgfpathlineto{\pgfqpoint{12.570576in}{1.164212in}}%
\pgfusepath{stroke}%
\end{pgfscope}%
\begin{pgfscope}%
\pgfpathrectangle{\pgfqpoint{1.286132in}{0.839159in}}{\pgfqpoint{12.053712in}{5.967710in}}%
\pgfusepath{clip}%
\pgfsetbuttcap%
\pgfsetroundjoin%
\pgfsetlinewidth{1.505625pt}%
\definecolor{currentstroke}{rgb}{0.121569,0.466667,0.705882}%
\pgfsetstrokecolor{currentstroke}%
\pgfsetdash{}{0pt}%
\pgfpathmoveto{\pgfqpoint{12.681262in}{1.140295in}}%
\pgfpathlineto{\pgfqpoint{12.681262in}{1.172726in}}%
\pgfusepath{stroke}%
\end{pgfscope}%
\begin{pgfscope}%
\pgfpathrectangle{\pgfqpoint{1.286132in}{0.839159in}}{\pgfqpoint{12.053712in}{5.967710in}}%
\pgfusepath{clip}%
\pgfsetbuttcap%
\pgfsetroundjoin%
\pgfsetlinewidth{1.505625pt}%
\definecolor{currentstroke}{rgb}{0.121569,0.466667,0.705882}%
\pgfsetstrokecolor{currentstroke}%
\pgfsetdash{}{0pt}%
\pgfpathmoveto{\pgfqpoint{12.791948in}{1.146674in}}%
\pgfpathlineto{\pgfqpoint{12.791948in}{1.150057in}}%
\pgfusepath{stroke}%
\end{pgfscope}%
\begin{pgfscope}%
\pgfpathrectangle{\pgfqpoint{1.286132in}{0.839159in}}{\pgfqpoint{12.053712in}{5.967710in}}%
\pgfusepath{clip}%
\pgfsetbuttcap%
\pgfsetroundjoin%
\pgfsetlinewidth{1.505625pt}%
\definecolor{currentstroke}{rgb}{1.000000,0.498039,0.054902}%
\pgfsetstrokecolor{currentstroke}%
\pgfsetdash{}{0pt}%
\pgfpathmoveto{\pgfqpoint{1.834028in}{1.119254in}}%
\pgfpathlineto{\pgfqpoint{1.834028in}{1.119265in}}%
\pgfusepath{stroke}%
\end{pgfscope}%
\begin{pgfscope}%
\pgfpathrectangle{\pgfqpoint{1.286132in}{0.839159in}}{\pgfqpoint{12.053712in}{5.967710in}}%
\pgfusepath{clip}%
\pgfsetbuttcap%
\pgfsetroundjoin%
\pgfsetlinewidth{1.505625pt}%
\definecolor{currentstroke}{rgb}{1.000000,0.498039,0.054902}%
\pgfsetstrokecolor{currentstroke}%
\pgfsetdash{}{0pt}%
\pgfpathmoveto{\pgfqpoint{1.944714in}{1.119575in}}%
\pgfpathlineto{\pgfqpoint{1.944714in}{1.119627in}}%
\pgfusepath{stroke}%
\end{pgfscope}%
\begin{pgfscope}%
\pgfpathrectangle{\pgfqpoint{1.286132in}{0.839159in}}{\pgfqpoint{12.053712in}{5.967710in}}%
\pgfusepath{clip}%
\pgfsetbuttcap%
\pgfsetroundjoin%
\pgfsetlinewidth{1.505625pt}%
\definecolor{currentstroke}{rgb}{1.000000,0.498039,0.054902}%
\pgfsetstrokecolor{currentstroke}%
\pgfsetdash{}{0pt}%
\pgfpathmoveto{\pgfqpoint{2.055400in}{1.119615in}}%
\pgfpathlineto{\pgfqpoint{2.055400in}{1.120844in}}%
\pgfusepath{stroke}%
\end{pgfscope}%
\begin{pgfscope}%
\pgfpathrectangle{\pgfqpoint{1.286132in}{0.839159in}}{\pgfqpoint{12.053712in}{5.967710in}}%
\pgfusepath{clip}%
\pgfsetbuttcap%
\pgfsetroundjoin%
\pgfsetlinewidth{1.505625pt}%
\definecolor{currentstroke}{rgb}{1.000000,0.498039,0.054902}%
\pgfsetstrokecolor{currentstroke}%
\pgfsetdash{}{0pt}%
\pgfpathmoveto{\pgfqpoint{2.166086in}{1.120273in}}%
\pgfpathlineto{\pgfqpoint{2.166086in}{1.120294in}}%
\pgfusepath{stroke}%
\end{pgfscope}%
\begin{pgfscope}%
\pgfpathrectangle{\pgfqpoint{1.286132in}{0.839159in}}{\pgfqpoint{12.053712in}{5.967710in}}%
\pgfusepath{clip}%
\pgfsetbuttcap%
\pgfsetroundjoin%
\pgfsetlinewidth{1.505625pt}%
\definecolor{currentstroke}{rgb}{1.000000,0.498039,0.054902}%
\pgfsetstrokecolor{currentstroke}%
\pgfsetdash{}{0pt}%
\pgfpathmoveto{\pgfqpoint{2.276772in}{1.120819in}}%
\pgfpathlineto{\pgfqpoint{2.276772in}{1.120933in}}%
\pgfusepath{stroke}%
\end{pgfscope}%
\begin{pgfscope}%
\pgfpathrectangle{\pgfqpoint{1.286132in}{0.839159in}}{\pgfqpoint{12.053712in}{5.967710in}}%
\pgfusepath{clip}%
\pgfsetbuttcap%
\pgfsetroundjoin%
\pgfsetlinewidth{1.505625pt}%
\definecolor{currentstroke}{rgb}{1.000000,0.498039,0.054902}%
\pgfsetstrokecolor{currentstroke}%
\pgfsetdash{}{0pt}%
\pgfpathmoveto{\pgfqpoint{2.387458in}{1.121508in}}%
\pgfpathlineto{\pgfqpoint{2.387458in}{1.122097in}}%
\pgfusepath{stroke}%
\end{pgfscope}%
\begin{pgfscope}%
\pgfpathrectangle{\pgfqpoint{1.286132in}{0.839159in}}{\pgfqpoint{12.053712in}{5.967710in}}%
\pgfusepath{clip}%
\pgfsetbuttcap%
\pgfsetroundjoin%
\pgfsetlinewidth{1.505625pt}%
\definecolor{currentstroke}{rgb}{1.000000,0.498039,0.054902}%
\pgfsetstrokecolor{currentstroke}%
\pgfsetdash{}{0pt}%
\pgfpathmoveto{\pgfqpoint{2.498144in}{1.118197in}}%
\pgfpathlineto{\pgfqpoint{2.498144in}{1.134730in}}%
\pgfusepath{stroke}%
\end{pgfscope}%
\begin{pgfscope}%
\pgfpathrectangle{\pgfqpoint{1.286132in}{0.839159in}}{\pgfqpoint{12.053712in}{5.967710in}}%
\pgfusepath{clip}%
\pgfsetbuttcap%
\pgfsetroundjoin%
\pgfsetlinewidth{1.505625pt}%
\definecolor{currentstroke}{rgb}{1.000000,0.498039,0.054902}%
\pgfsetstrokecolor{currentstroke}%
\pgfsetdash{}{0pt}%
\pgfpathmoveto{\pgfqpoint{2.608830in}{1.123121in}}%
\pgfpathlineto{\pgfqpoint{2.608830in}{1.124056in}}%
\pgfusepath{stroke}%
\end{pgfscope}%
\begin{pgfscope}%
\pgfpathrectangle{\pgfqpoint{1.286132in}{0.839159in}}{\pgfqpoint{12.053712in}{5.967710in}}%
\pgfusepath{clip}%
\pgfsetbuttcap%
\pgfsetroundjoin%
\pgfsetlinewidth{1.505625pt}%
\definecolor{currentstroke}{rgb}{1.000000,0.498039,0.054902}%
\pgfsetstrokecolor{currentstroke}%
\pgfsetdash{}{0pt}%
\pgfpathmoveto{\pgfqpoint{2.719516in}{1.124273in}}%
\pgfpathlineto{\pgfqpoint{2.719516in}{1.125558in}}%
\pgfusepath{stroke}%
\end{pgfscope}%
\begin{pgfscope}%
\pgfpathrectangle{\pgfqpoint{1.286132in}{0.839159in}}{\pgfqpoint{12.053712in}{5.967710in}}%
\pgfusepath{clip}%
\pgfsetbuttcap%
\pgfsetroundjoin%
\pgfsetlinewidth{1.505625pt}%
\definecolor{currentstroke}{rgb}{1.000000,0.498039,0.054902}%
\pgfsetstrokecolor{currentstroke}%
\pgfsetdash{}{0pt}%
\pgfpathmoveto{\pgfqpoint{2.830202in}{1.125913in}}%
\pgfpathlineto{\pgfqpoint{2.830202in}{1.126960in}}%
\pgfusepath{stroke}%
\end{pgfscope}%
\begin{pgfscope}%
\pgfpathrectangle{\pgfqpoint{1.286132in}{0.839159in}}{\pgfqpoint{12.053712in}{5.967710in}}%
\pgfusepath{clip}%
\pgfsetbuttcap%
\pgfsetroundjoin%
\pgfsetlinewidth{1.505625pt}%
\definecolor{currentstroke}{rgb}{1.000000,0.498039,0.054902}%
\pgfsetstrokecolor{currentstroke}%
\pgfsetdash{}{0pt}%
\pgfpathmoveto{\pgfqpoint{2.940888in}{1.127160in}}%
\pgfpathlineto{\pgfqpoint{2.940888in}{1.132943in}}%
\pgfusepath{stroke}%
\end{pgfscope}%
\begin{pgfscope}%
\pgfpathrectangle{\pgfqpoint{1.286132in}{0.839159in}}{\pgfqpoint{12.053712in}{5.967710in}}%
\pgfusepath{clip}%
\pgfsetbuttcap%
\pgfsetroundjoin%
\pgfsetlinewidth{1.505625pt}%
\definecolor{currentstroke}{rgb}{1.000000,0.498039,0.054902}%
\pgfsetstrokecolor{currentstroke}%
\pgfsetdash{}{0pt}%
\pgfpathmoveto{\pgfqpoint{3.051574in}{1.129210in}}%
\pgfpathlineto{\pgfqpoint{3.051574in}{1.129719in}}%
\pgfusepath{stroke}%
\end{pgfscope}%
\begin{pgfscope}%
\pgfpathrectangle{\pgfqpoint{1.286132in}{0.839159in}}{\pgfqpoint{12.053712in}{5.967710in}}%
\pgfusepath{clip}%
\pgfsetbuttcap%
\pgfsetroundjoin%
\pgfsetlinewidth{1.505625pt}%
\definecolor{currentstroke}{rgb}{1.000000,0.498039,0.054902}%
\pgfsetstrokecolor{currentstroke}%
\pgfsetdash{}{0pt}%
\pgfpathmoveto{\pgfqpoint{3.162260in}{1.131257in}}%
\pgfpathlineto{\pgfqpoint{3.162260in}{1.132720in}}%
\pgfusepath{stroke}%
\end{pgfscope}%
\begin{pgfscope}%
\pgfpathrectangle{\pgfqpoint{1.286132in}{0.839159in}}{\pgfqpoint{12.053712in}{5.967710in}}%
\pgfusepath{clip}%
\pgfsetbuttcap%
\pgfsetroundjoin%
\pgfsetlinewidth{1.505625pt}%
\definecolor{currentstroke}{rgb}{1.000000,0.498039,0.054902}%
\pgfsetstrokecolor{currentstroke}%
\pgfsetdash{}{0pt}%
\pgfpathmoveto{\pgfqpoint{3.272946in}{1.128716in}}%
\pgfpathlineto{\pgfqpoint{3.272946in}{1.157294in}}%
\pgfusepath{stroke}%
\end{pgfscope}%
\begin{pgfscope}%
\pgfpathrectangle{\pgfqpoint{1.286132in}{0.839159in}}{\pgfqpoint{12.053712in}{5.967710in}}%
\pgfusepath{clip}%
\pgfsetbuttcap%
\pgfsetroundjoin%
\pgfsetlinewidth{1.505625pt}%
\definecolor{currentstroke}{rgb}{1.000000,0.498039,0.054902}%
\pgfsetstrokecolor{currentstroke}%
\pgfsetdash{}{0pt}%
\pgfpathmoveto{\pgfqpoint{3.383632in}{1.136667in}}%
\pgfpathlineto{\pgfqpoint{3.383632in}{1.137236in}}%
\pgfusepath{stroke}%
\end{pgfscope}%
\begin{pgfscope}%
\pgfpathrectangle{\pgfqpoint{1.286132in}{0.839159in}}{\pgfqpoint{12.053712in}{5.967710in}}%
\pgfusepath{clip}%
\pgfsetbuttcap%
\pgfsetroundjoin%
\pgfsetlinewidth{1.505625pt}%
\definecolor{currentstroke}{rgb}{1.000000,0.498039,0.054902}%
\pgfsetstrokecolor{currentstroke}%
\pgfsetdash{}{0pt}%
\pgfpathmoveto{\pgfqpoint{3.494319in}{1.139553in}}%
\pgfpathlineto{\pgfqpoint{3.494319in}{1.141314in}}%
\pgfusepath{stroke}%
\end{pgfscope}%
\begin{pgfscope}%
\pgfpathrectangle{\pgfqpoint{1.286132in}{0.839159in}}{\pgfqpoint{12.053712in}{5.967710in}}%
\pgfusepath{clip}%
\pgfsetbuttcap%
\pgfsetroundjoin%
\pgfsetlinewidth{1.505625pt}%
\definecolor{currentstroke}{rgb}{1.000000,0.498039,0.054902}%
\pgfsetstrokecolor{currentstroke}%
\pgfsetdash{}{0pt}%
\pgfpathmoveto{\pgfqpoint{3.605005in}{1.139788in}}%
\pgfpathlineto{\pgfqpoint{3.605005in}{1.156811in}}%
\pgfusepath{stroke}%
\end{pgfscope}%
\begin{pgfscope}%
\pgfpathrectangle{\pgfqpoint{1.286132in}{0.839159in}}{\pgfqpoint{12.053712in}{5.967710in}}%
\pgfusepath{clip}%
\pgfsetbuttcap%
\pgfsetroundjoin%
\pgfsetlinewidth{1.505625pt}%
\definecolor{currentstroke}{rgb}{1.000000,0.498039,0.054902}%
\pgfsetstrokecolor{currentstroke}%
\pgfsetdash{}{0pt}%
\pgfpathmoveto{\pgfqpoint{3.715691in}{1.144391in}}%
\pgfpathlineto{\pgfqpoint{3.715691in}{1.158691in}}%
\pgfusepath{stroke}%
\end{pgfscope}%
\begin{pgfscope}%
\pgfpathrectangle{\pgfqpoint{1.286132in}{0.839159in}}{\pgfqpoint{12.053712in}{5.967710in}}%
\pgfusepath{clip}%
\pgfsetbuttcap%
\pgfsetroundjoin%
\pgfsetlinewidth{1.505625pt}%
\definecolor{currentstroke}{rgb}{1.000000,0.498039,0.054902}%
\pgfsetstrokecolor{currentstroke}%
\pgfsetdash{}{0pt}%
\pgfpathmoveto{\pgfqpoint{3.826377in}{1.150105in}}%
\pgfpathlineto{\pgfqpoint{3.826377in}{1.161339in}}%
\pgfusepath{stroke}%
\end{pgfscope}%
\begin{pgfscope}%
\pgfpathrectangle{\pgfqpoint{1.286132in}{0.839159in}}{\pgfqpoint{12.053712in}{5.967710in}}%
\pgfusepath{clip}%
\pgfsetbuttcap%
\pgfsetroundjoin%
\pgfsetlinewidth{1.505625pt}%
\definecolor{currentstroke}{rgb}{1.000000,0.498039,0.054902}%
\pgfsetstrokecolor{currentstroke}%
\pgfsetdash{}{0pt}%
\pgfpathmoveto{\pgfqpoint{3.937063in}{1.154438in}}%
\pgfpathlineto{\pgfqpoint{3.937063in}{1.162313in}}%
\pgfusepath{stroke}%
\end{pgfscope}%
\begin{pgfscope}%
\pgfpathrectangle{\pgfqpoint{1.286132in}{0.839159in}}{\pgfqpoint{12.053712in}{5.967710in}}%
\pgfusepath{clip}%
\pgfsetbuttcap%
\pgfsetroundjoin%
\pgfsetlinewidth{1.505625pt}%
\definecolor{currentstroke}{rgb}{1.000000,0.498039,0.054902}%
\pgfsetstrokecolor{currentstroke}%
\pgfsetdash{}{0pt}%
\pgfpathmoveto{\pgfqpoint{4.047749in}{1.161600in}}%
\pgfpathlineto{\pgfqpoint{4.047749in}{1.165598in}}%
\pgfusepath{stroke}%
\end{pgfscope}%
\begin{pgfscope}%
\pgfpathrectangle{\pgfqpoint{1.286132in}{0.839159in}}{\pgfqpoint{12.053712in}{5.967710in}}%
\pgfusepath{clip}%
\pgfsetbuttcap%
\pgfsetroundjoin%
\pgfsetlinewidth{1.505625pt}%
\definecolor{currentstroke}{rgb}{1.000000,0.498039,0.054902}%
\pgfsetstrokecolor{currentstroke}%
\pgfsetdash{}{0pt}%
\pgfpathmoveto{\pgfqpoint{4.158435in}{1.166639in}}%
\pgfpathlineto{\pgfqpoint{4.158435in}{1.172335in}}%
\pgfusepath{stroke}%
\end{pgfscope}%
\begin{pgfscope}%
\pgfpathrectangle{\pgfqpoint{1.286132in}{0.839159in}}{\pgfqpoint{12.053712in}{5.967710in}}%
\pgfusepath{clip}%
\pgfsetbuttcap%
\pgfsetroundjoin%
\pgfsetlinewidth{1.505625pt}%
\definecolor{currentstroke}{rgb}{1.000000,0.498039,0.054902}%
\pgfsetstrokecolor{currentstroke}%
\pgfsetdash{}{0pt}%
\pgfpathmoveto{\pgfqpoint{4.269121in}{1.173314in}}%
\pgfpathlineto{\pgfqpoint{4.269121in}{1.182047in}}%
\pgfusepath{stroke}%
\end{pgfscope}%
\begin{pgfscope}%
\pgfpathrectangle{\pgfqpoint{1.286132in}{0.839159in}}{\pgfqpoint{12.053712in}{5.967710in}}%
\pgfusepath{clip}%
\pgfsetbuttcap%
\pgfsetroundjoin%
\pgfsetlinewidth{1.505625pt}%
\definecolor{currentstroke}{rgb}{1.000000,0.498039,0.054902}%
\pgfsetstrokecolor{currentstroke}%
\pgfsetdash{}{0pt}%
\pgfpathmoveto{\pgfqpoint{4.379807in}{1.180655in}}%
\pgfpathlineto{\pgfqpoint{4.379807in}{1.182519in}}%
\pgfusepath{stroke}%
\end{pgfscope}%
\begin{pgfscope}%
\pgfpathrectangle{\pgfqpoint{1.286132in}{0.839159in}}{\pgfqpoint{12.053712in}{5.967710in}}%
\pgfusepath{clip}%
\pgfsetbuttcap%
\pgfsetroundjoin%
\pgfsetlinewidth{1.505625pt}%
\definecolor{currentstroke}{rgb}{1.000000,0.498039,0.054902}%
\pgfsetstrokecolor{currentstroke}%
\pgfsetdash{}{0pt}%
\pgfpathmoveto{\pgfqpoint{4.490493in}{1.184619in}}%
\pgfpathlineto{\pgfqpoint{4.490493in}{1.192458in}}%
\pgfusepath{stroke}%
\end{pgfscope}%
\begin{pgfscope}%
\pgfpathrectangle{\pgfqpoint{1.286132in}{0.839159in}}{\pgfqpoint{12.053712in}{5.967710in}}%
\pgfusepath{clip}%
\pgfsetbuttcap%
\pgfsetroundjoin%
\pgfsetlinewidth{1.505625pt}%
\definecolor{currentstroke}{rgb}{1.000000,0.498039,0.054902}%
\pgfsetstrokecolor{currentstroke}%
\pgfsetdash{}{0pt}%
\pgfpathmoveto{\pgfqpoint{4.601179in}{1.194442in}}%
\pgfpathlineto{\pgfqpoint{4.601179in}{1.198489in}}%
\pgfusepath{stroke}%
\end{pgfscope}%
\begin{pgfscope}%
\pgfpathrectangle{\pgfqpoint{1.286132in}{0.839159in}}{\pgfqpoint{12.053712in}{5.967710in}}%
\pgfusepath{clip}%
\pgfsetbuttcap%
\pgfsetroundjoin%
\pgfsetlinewidth{1.505625pt}%
\definecolor{currentstroke}{rgb}{1.000000,0.498039,0.054902}%
\pgfsetstrokecolor{currentstroke}%
\pgfsetdash{}{0pt}%
\pgfpathmoveto{\pgfqpoint{4.711865in}{1.204404in}}%
\pgfpathlineto{\pgfqpoint{4.711865in}{1.213542in}}%
\pgfusepath{stroke}%
\end{pgfscope}%
\begin{pgfscope}%
\pgfpathrectangle{\pgfqpoint{1.286132in}{0.839159in}}{\pgfqpoint{12.053712in}{5.967710in}}%
\pgfusepath{clip}%
\pgfsetbuttcap%
\pgfsetroundjoin%
\pgfsetlinewidth{1.505625pt}%
\definecolor{currentstroke}{rgb}{1.000000,0.498039,0.054902}%
\pgfsetstrokecolor{currentstroke}%
\pgfsetdash{}{0pt}%
\pgfpathmoveto{\pgfqpoint{4.822551in}{1.216187in}}%
\pgfpathlineto{\pgfqpoint{4.822551in}{1.220507in}}%
\pgfusepath{stroke}%
\end{pgfscope}%
\begin{pgfscope}%
\pgfpathrectangle{\pgfqpoint{1.286132in}{0.839159in}}{\pgfqpoint{12.053712in}{5.967710in}}%
\pgfusepath{clip}%
\pgfsetbuttcap%
\pgfsetroundjoin%
\pgfsetlinewidth{1.505625pt}%
\definecolor{currentstroke}{rgb}{1.000000,0.498039,0.054902}%
\pgfsetstrokecolor{currentstroke}%
\pgfsetdash{}{0pt}%
\pgfpathmoveto{\pgfqpoint{4.933237in}{1.220803in}}%
\pgfpathlineto{\pgfqpoint{4.933237in}{1.229421in}}%
\pgfusepath{stroke}%
\end{pgfscope}%
\begin{pgfscope}%
\pgfpathrectangle{\pgfqpoint{1.286132in}{0.839159in}}{\pgfqpoint{12.053712in}{5.967710in}}%
\pgfusepath{clip}%
\pgfsetbuttcap%
\pgfsetroundjoin%
\pgfsetlinewidth{1.505625pt}%
\definecolor{currentstroke}{rgb}{1.000000,0.498039,0.054902}%
\pgfsetstrokecolor{currentstroke}%
\pgfsetdash{}{0pt}%
\pgfpathmoveto{\pgfqpoint{5.043923in}{1.234413in}}%
\pgfpathlineto{\pgfqpoint{5.043923in}{1.245645in}}%
\pgfusepath{stroke}%
\end{pgfscope}%
\begin{pgfscope}%
\pgfpathrectangle{\pgfqpoint{1.286132in}{0.839159in}}{\pgfqpoint{12.053712in}{5.967710in}}%
\pgfusepath{clip}%
\pgfsetbuttcap%
\pgfsetroundjoin%
\pgfsetlinewidth{1.505625pt}%
\definecolor{currentstroke}{rgb}{1.000000,0.498039,0.054902}%
\pgfsetstrokecolor{currentstroke}%
\pgfsetdash{}{0pt}%
\pgfpathmoveto{\pgfqpoint{5.154609in}{1.243984in}}%
\pgfpathlineto{\pgfqpoint{5.154609in}{1.251204in}}%
\pgfusepath{stroke}%
\end{pgfscope}%
\begin{pgfscope}%
\pgfpathrectangle{\pgfqpoint{1.286132in}{0.839159in}}{\pgfqpoint{12.053712in}{5.967710in}}%
\pgfusepath{clip}%
\pgfsetbuttcap%
\pgfsetroundjoin%
\pgfsetlinewidth{1.505625pt}%
\definecolor{currentstroke}{rgb}{1.000000,0.498039,0.054902}%
\pgfsetstrokecolor{currentstroke}%
\pgfsetdash{}{0pt}%
\pgfpathmoveto{\pgfqpoint{5.265296in}{1.256427in}}%
\pgfpathlineto{\pgfqpoint{5.265296in}{1.307442in}}%
\pgfusepath{stroke}%
\end{pgfscope}%
\begin{pgfscope}%
\pgfpathrectangle{\pgfqpoint{1.286132in}{0.839159in}}{\pgfqpoint{12.053712in}{5.967710in}}%
\pgfusepath{clip}%
\pgfsetbuttcap%
\pgfsetroundjoin%
\pgfsetlinewidth{1.505625pt}%
\definecolor{currentstroke}{rgb}{1.000000,0.498039,0.054902}%
\pgfsetstrokecolor{currentstroke}%
\pgfsetdash{}{0pt}%
\pgfpathmoveto{\pgfqpoint{5.375982in}{1.273493in}}%
\pgfpathlineto{\pgfqpoint{5.375982in}{1.310145in}}%
\pgfusepath{stroke}%
\end{pgfscope}%
\begin{pgfscope}%
\pgfpathrectangle{\pgfqpoint{1.286132in}{0.839159in}}{\pgfqpoint{12.053712in}{5.967710in}}%
\pgfusepath{clip}%
\pgfsetbuttcap%
\pgfsetroundjoin%
\pgfsetlinewidth{1.505625pt}%
\definecolor{currentstroke}{rgb}{1.000000,0.498039,0.054902}%
\pgfsetstrokecolor{currentstroke}%
\pgfsetdash{}{0pt}%
\pgfpathmoveto{\pgfqpoint{5.486668in}{1.345775in}}%
\pgfpathlineto{\pgfqpoint{5.486668in}{1.547342in}}%
\pgfusepath{stroke}%
\end{pgfscope}%
\begin{pgfscope}%
\pgfpathrectangle{\pgfqpoint{1.286132in}{0.839159in}}{\pgfqpoint{12.053712in}{5.967710in}}%
\pgfusepath{clip}%
\pgfsetbuttcap%
\pgfsetroundjoin%
\pgfsetlinewidth{1.505625pt}%
\definecolor{currentstroke}{rgb}{1.000000,0.498039,0.054902}%
\pgfsetstrokecolor{currentstroke}%
\pgfsetdash{}{0pt}%
\pgfpathmoveto{\pgfqpoint{5.597354in}{1.291931in}}%
\pgfpathlineto{\pgfqpoint{5.597354in}{1.300862in}}%
\pgfusepath{stroke}%
\end{pgfscope}%
\begin{pgfscope}%
\pgfpathrectangle{\pgfqpoint{1.286132in}{0.839159in}}{\pgfqpoint{12.053712in}{5.967710in}}%
\pgfusepath{clip}%
\pgfsetbuttcap%
\pgfsetroundjoin%
\pgfsetlinewidth{1.505625pt}%
\definecolor{currentstroke}{rgb}{1.000000,0.498039,0.054902}%
\pgfsetstrokecolor{currentstroke}%
\pgfsetdash{}{0pt}%
\pgfpathmoveto{\pgfqpoint{5.708040in}{1.315738in}}%
\pgfpathlineto{\pgfqpoint{5.708040in}{1.321180in}}%
\pgfusepath{stroke}%
\end{pgfscope}%
\begin{pgfscope}%
\pgfpathrectangle{\pgfqpoint{1.286132in}{0.839159in}}{\pgfqpoint{12.053712in}{5.967710in}}%
\pgfusepath{clip}%
\pgfsetbuttcap%
\pgfsetroundjoin%
\pgfsetlinewidth{1.505625pt}%
\definecolor{currentstroke}{rgb}{1.000000,0.498039,0.054902}%
\pgfsetstrokecolor{currentstroke}%
\pgfsetdash{}{0pt}%
\pgfpathmoveto{\pgfqpoint{5.818726in}{1.320435in}}%
\pgfpathlineto{\pgfqpoint{5.818726in}{1.446956in}}%
\pgfusepath{stroke}%
\end{pgfscope}%
\begin{pgfscope}%
\pgfpathrectangle{\pgfqpoint{1.286132in}{0.839159in}}{\pgfqpoint{12.053712in}{5.967710in}}%
\pgfusepath{clip}%
\pgfsetbuttcap%
\pgfsetroundjoin%
\pgfsetlinewidth{1.505625pt}%
\definecolor{currentstroke}{rgb}{1.000000,0.498039,0.054902}%
\pgfsetstrokecolor{currentstroke}%
\pgfsetdash{}{0pt}%
\pgfpathmoveto{\pgfqpoint{5.929412in}{1.346043in}}%
\pgfpathlineto{\pgfqpoint{5.929412in}{1.446391in}}%
\pgfusepath{stroke}%
\end{pgfscope}%
\begin{pgfscope}%
\pgfpathrectangle{\pgfqpoint{1.286132in}{0.839159in}}{\pgfqpoint{12.053712in}{5.967710in}}%
\pgfusepath{clip}%
\pgfsetbuttcap%
\pgfsetroundjoin%
\pgfsetlinewidth{1.505625pt}%
\definecolor{currentstroke}{rgb}{1.000000,0.498039,0.054902}%
\pgfsetstrokecolor{currentstroke}%
\pgfsetdash{}{0pt}%
\pgfpathmoveto{\pgfqpoint{6.040098in}{1.358268in}}%
\pgfpathlineto{\pgfqpoint{6.040098in}{1.372203in}}%
\pgfusepath{stroke}%
\end{pgfscope}%
\begin{pgfscope}%
\pgfpathrectangle{\pgfqpoint{1.286132in}{0.839159in}}{\pgfqpoint{12.053712in}{5.967710in}}%
\pgfusepath{clip}%
\pgfsetbuttcap%
\pgfsetroundjoin%
\pgfsetlinewidth{1.505625pt}%
\definecolor{currentstroke}{rgb}{1.000000,0.498039,0.054902}%
\pgfsetstrokecolor{currentstroke}%
\pgfsetdash{}{0pt}%
\pgfpathmoveto{\pgfqpoint{6.150784in}{1.379922in}}%
\pgfpathlineto{\pgfqpoint{6.150784in}{1.421774in}}%
\pgfusepath{stroke}%
\end{pgfscope}%
\begin{pgfscope}%
\pgfpathrectangle{\pgfqpoint{1.286132in}{0.839159in}}{\pgfqpoint{12.053712in}{5.967710in}}%
\pgfusepath{clip}%
\pgfsetbuttcap%
\pgfsetroundjoin%
\pgfsetlinewidth{1.505625pt}%
\definecolor{currentstroke}{rgb}{1.000000,0.498039,0.054902}%
\pgfsetstrokecolor{currentstroke}%
\pgfsetdash{}{0pt}%
\pgfpathmoveto{\pgfqpoint{6.261470in}{1.388976in}}%
\pgfpathlineto{\pgfqpoint{6.261470in}{1.401171in}}%
\pgfusepath{stroke}%
\end{pgfscope}%
\begin{pgfscope}%
\pgfpathrectangle{\pgfqpoint{1.286132in}{0.839159in}}{\pgfqpoint{12.053712in}{5.967710in}}%
\pgfusepath{clip}%
\pgfsetbuttcap%
\pgfsetroundjoin%
\pgfsetlinewidth{1.505625pt}%
\definecolor{currentstroke}{rgb}{1.000000,0.498039,0.054902}%
\pgfsetstrokecolor{currentstroke}%
\pgfsetdash{}{0pt}%
\pgfpathmoveto{\pgfqpoint{6.372156in}{1.381921in}}%
\pgfpathlineto{\pgfqpoint{6.372156in}{1.532248in}}%
\pgfusepath{stroke}%
\end{pgfscope}%
\begin{pgfscope}%
\pgfpathrectangle{\pgfqpoint{1.286132in}{0.839159in}}{\pgfqpoint{12.053712in}{5.967710in}}%
\pgfusepath{clip}%
\pgfsetbuttcap%
\pgfsetroundjoin%
\pgfsetlinewidth{1.505625pt}%
\definecolor{currentstroke}{rgb}{1.000000,0.498039,0.054902}%
\pgfsetstrokecolor{currentstroke}%
\pgfsetdash{}{0pt}%
\pgfpathmoveto{\pgfqpoint{6.482842in}{1.426410in}}%
\pgfpathlineto{\pgfqpoint{6.482842in}{1.464173in}}%
\pgfusepath{stroke}%
\end{pgfscope}%
\begin{pgfscope}%
\pgfpathrectangle{\pgfqpoint{1.286132in}{0.839159in}}{\pgfqpoint{12.053712in}{5.967710in}}%
\pgfusepath{clip}%
\pgfsetbuttcap%
\pgfsetroundjoin%
\pgfsetlinewidth{1.505625pt}%
\definecolor{currentstroke}{rgb}{1.000000,0.498039,0.054902}%
\pgfsetstrokecolor{currentstroke}%
\pgfsetdash{}{0pt}%
\pgfpathmoveto{\pgfqpoint{6.593528in}{1.464115in}}%
\pgfpathlineto{\pgfqpoint{6.593528in}{1.470556in}}%
\pgfusepath{stroke}%
\end{pgfscope}%
\begin{pgfscope}%
\pgfpathrectangle{\pgfqpoint{1.286132in}{0.839159in}}{\pgfqpoint{12.053712in}{5.967710in}}%
\pgfusepath{clip}%
\pgfsetbuttcap%
\pgfsetroundjoin%
\pgfsetlinewidth{1.505625pt}%
\definecolor{currentstroke}{rgb}{1.000000,0.498039,0.054902}%
\pgfsetstrokecolor{currentstroke}%
\pgfsetdash{}{0pt}%
\pgfpathmoveto{\pgfqpoint{6.704214in}{1.484846in}}%
\pgfpathlineto{\pgfqpoint{6.704214in}{1.490516in}}%
\pgfusepath{stroke}%
\end{pgfscope}%
\begin{pgfscope}%
\pgfpathrectangle{\pgfqpoint{1.286132in}{0.839159in}}{\pgfqpoint{12.053712in}{5.967710in}}%
\pgfusepath{clip}%
\pgfsetbuttcap%
\pgfsetroundjoin%
\pgfsetlinewidth{1.505625pt}%
\definecolor{currentstroke}{rgb}{1.000000,0.498039,0.054902}%
\pgfsetstrokecolor{currentstroke}%
\pgfsetdash{}{0pt}%
\pgfpathmoveto{\pgfqpoint{6.814900in}{1.500078in}}%
\pgfpathlineto{\pgfqpoint{6.814900in}{1.518787in}}%
\pgfusepath{stroke}%
\end{pgfscope}%
\begin{pgfscope}%
\pgfpathrectangle{\pgfqpoint{1.286132in}{0.839159in}}{\pgfqpoint{12.053712in}{5.967710in}}%
\pgfusepath{clip}%
\pgfsetbuttcap%
\pgfsetroundjoin%
\pgfsetlinewidth{1.505625pt}%
\definecolor{currentstroke}{rgb}{1.000000,0.498039,0.054902}%
\pgfsetstrokecolor{currentstroke}%
\pgfsetdash{}{0pt}%
\pgfpathmoveto{\pgfqpoint{6.925586in}{1.531671in}}%
\pgfpathlineto{\pgfqpoint{6.925586in}{1.537492in}}%
\pgfusepath{stroke}%
\end{pgfscope}%
\begin{pgfscope}%
\pgfpathrectangle{\pgfqpoint{1.286132in}{0.839159in}}{\pgfqpoint{12.053712in}{5.967710in}}%
\pgfusepath{clip}%
\pgfsetbuttcap%
\pgfsetroundjoin%
\pgfsetlinewidth{1.505625pt}%
\definecolor{currentstroke}{rgb}{1.000000,0.498039,0.054902}%
\pgfsetstrokecolor{currentstroke}%
\pgfsetdash{}{0pt}%
\pgfpathmoveto{\pgfqpoint{7.036272in}{1.522258in}}%
\pgfpathlineto{\pgfqpoint{7.036272in}{1.670847in}}%
\pgfusepath{stroke}%
\end{pgfscope}%
\begin{pgfscope}%
\pgfpathrectangle{\pgfqpoint{1.286132in}{0.839159in}}{\pgfqpoint{12.053712in}{5.967710in}}%
\pgfusepath{clip}%
\pgfsetbuttcap%
\pgfsetroundjoin%
\pgfsetlinewidth{1.505625pt}%
\definecolor{currentstroke}{rgb}{1.000000,0.498039,0.054902}%
\pgfsetstrokecolor{currentstroke}%
\pgfsetdash{}{0pt}%
\pgfpathmoveto{\pgfqpoint{7.146959in}{1.593398in}}%
\pgfpathlineto{\pgfqpoint{7.146959in}{1.634106in}}%
\pgfusepath{stroke}%
\end{pgfscope}%
\begin{pgfscope}%
\pgfpathrectangle{\pgfqpoint{1.286132in}{0.839159in}}{\pgfqpoint{12.053712in}{5.967710in}}%
\pgfusepath{clip}%
\pgfsetbuttcap%
\pgfsetroundjoin%
\pgfsetlinewidth{1.505625pt}%
\definecolor{currentstroke}{rgb}{1.000000,0.498039,0.054902}%
\pgfsetstrokecolor{currentstroke}%
\pgfsetdash{}{0pt}%
\pgfpathmoveto{\pgfqpoint{7.257645in}{1.611564in}}%
\pgfpathlineto{\pgfqpoint{7.257645in}{1.637031in}}%
\pgfusepath{stroke}%
\end{pgfscope}%
\begin{pgfscope}%
\pgfpathrectangle{\pgfqpoint{1.286132in}{0.839159in}}{\pgfqpoint{12.053712in}{5.967710in}}%
\pgfusepath{clip}%
\pgfsetbuttcap%
\pgfsetroundjoin%
\pgfsetlinewidth{1.505625pt}%
\definecolor{currentstroke}{rgb}{1.000000,0.498039,0.054902}%
\pgfsetstrokecolor{currentstroke}%
\pgfsetdash{}{0pt}%
\pgfpathmoveto{\pgfqpoint{7.368331in}{1.606726in}}%
\pgfpathlineto{\pgfqpoint{7.368331in}{1.635686in}}%
\pgfusepath{stroke}%
\end{pgfscope}%
\begin{pgfscope}%
\pgfpathrectangle{\pgfqpoint{1.286132in}{0.839159in}}{\pgfqpoint{12.053712in}{5.967710in}}%
\pgfusepath{clip}%
\pgfsetbuttcap%
\pgfsetroundjoin%
\pgfsetlinewidth{1.505625pt}%
\definecolor{currentstroke}{rgb}{1.000000,0.498039,0.054902}%
\pgfsetstrokecolor{currentstroke}%
\pgfsetdash{}{0pt}%
\pgfpathmoveto{\pgfqpoint{7.479017in}{1.560795in}}%
\pgfpathlineto{\pgfqpoint{7.479017in}{2.111884in}}%
\pgfusepath{stroke}%
\end{pgfscope}%
\begin{pgfscope}%
\pgfpathrectangle{\pgfqpoint{1.286132in}{0.839159in}}{\pgfqpoint{12.053712in}{5.967710in}}%
\pgfusepath{clip}%
\pgfsetbuttcap%
\pgfsetroundjoin%
\pgfsetlinewidth{1.505625pt}%
\definecolor{currentstroke}{rgb}{1.000000,0.498039,0.054902}%
\pgfsetstrokecolor{currentstroke}%
\pgfsetdash{}{0pt}%
\pgfpathmoveto{\pgfqpoint{7.589703in}{1.665239in}}%
\pgfpathlineto{\pgfqpoint{7.589703in}{1.689119in}}%
\pgfusepath{stroke}%
\end{pgfscope}%
\begin{pgfscope}%
\pgfpathrectangle{\pgfqpoint{1.286132in}{0.839159in}}{\pgfqpoint{12.053712in}{5.967710in}}%
\pgfusepath{clip}%
\pgfsetbuttcap%
\pgfsetroundjoin%
\pgfsetlinewidth{1.505625pt}%
\definecolor{currentstroke}{rgb}{1.000000,0.498039,0.054902}%
\pgfsetstrokecolor{currentstroke}%
\pgfsetdash{}{0pt}%
\pgfpathmoveto{\pgfqpoint{7.700389in}{1.688215in}}%
\pgfpathlineto{\pgfqpoint{7.700389in}{1.701797in}}%
\pgfusepath{stroke}%
\end{pgfscope}%
\begin{pgfscope}%
\pgfpathrectangle{\pgfqpoint{1.286132in}{0.839159in}}{\pgfqpoint{12.053712in}{5.967710in}}%
\pgfusepath{clip}%
\pgfsetbuttcap%
\pgfsetroundjoin%
\pgfsetlinewidth{1.505625pt}%
\definecolor{currentstroke}{rgb}{1.000000,0.498039,0.054902}%
\pgfsetstrokecolor{currentstroke}%
\pgfsetdash{}{0pt}%
\pgfpathmoveto{\pgfqpoint{7.811075in}{1.725347in}}%
\pgfpathlineto{\pgfqpoint{7.811075in}{1.739283in}}%
\pgfusepath{stroke}%
\end{pgfscope}%
\begin{pgfscope}%
\pgfpathrectangle{\pgfqpoint{1.286132in}{0.839159in}}{\pgfqpoint{12.053712in}{5.967710in}}%
\pgfusepath{clip}%
\pgfsetbuttcap%
\pgfsetroundjoin%
\pgfsetlinewidth{1.505625pt}%
\definecolor{currentstroke}{rgb}{1.000000,0.498039,0.054902}%
\pgfsetstrokecolor{currentstroke}%
\pgfsetdash{}{0pt}%
\pgfpathmoveto{\pgfqpoint{7.921761in}{1.740975in}}%
\pgfpathlineto{\pgfqpoint{7.921761in}{1.813263in}}%
\pgfusepath{stroke}%
\end{pgfscope}%
\begin{pgfscope}%
\pgfpathrectangle{\pgfqpoint{1.286132in}{0.839159in}}{\pgfqpoint{12.053712in}{5.967710in}}%
\pgfusepath{clip}%
\pgfsetbuttcap%
\pgfsetroundjoin%
\pgfsetlinewidth{1.505625pt}%
\definecolor{currentstroke}{rgb}{1.000000,0.498039,0.054902}%
\pgfsetstrokecolor{currentstroke}%
\pgfsetdash{}{0pt}%
\pgfpathmoveto{\pgfqpoint{8.032447in}{1.785663in}}%
\pgfpathlineto{\pgfqpoint{8.032447in}{1.793456in}}%
\pgfusepath{stroke}%
\end{pgfscope}%
\begin{pgfscope}%
\pgfpathrectangle{\pgfqpoint{1.286132in}{0.839159in}}{\pgfqpoint{12.053712in}{5.967710in}}%
\pgfusepath{clip}%
\pgfsetbuttcap%
\pgfsetroundjoin%
\pgfsetlinewidth{1.505625pt}%
\definecolor{currentstroke}{rgb}{1.000000,0.498039,0.054902}%
\pgfsetstrokecolor{currentstroke}%
\pgfsetdash{}{0pt}%
\pgfpathmoveto{\pgfqpoint{8.143133in}{1.822400in}}%
\pgfpathlineto{\pgfqpoint{8.143133in}{1.836554in}}%
\pgfusepath{stroke}%
\end{pgfscope}%
\begin{pgfscope}%
\pgfpathrectangle{\pgfqpoint{1.286132in}{0.839159in}}{\pgfqpoint{12.053712in}{5.967710in}}%
\pgfusepath{clip}%
\pgfsetbuttcap%
\pgfsetroundjoin%
\pgfsetlinewidth{1.505625pt}%
\definecolor{currentstroke}{rgb}{1.000000,0.498039,0.054902}%
\pgfsetstrokecolor{currentstroke}%
\pgfsetdash{}{0pt}%
\pgfpathmoveto{\pgfqpoint{8.253819in}{1.855566in}}%
\pgfpathlineto{\pgfqpoint{8.253819in}{2.051000in}}%
\pgfusepath{stroke}%
\end{pgfscope}%
\begin{pgfscope}%
\pgfpathrectangle{\pgfqpoint{1.286132in}{0.839159in}}{\pgfqpoint{12.053712in}{5.967710in}}%
\pgfusepath{clip}%
\pgfsetbuttcap%
\pgfsetroundjoin%
\pgfsetlinewidth{1.505625pt}%
\definecolor{currentstroke}{rgb}{1.000000,0.498039,0.054902}%
\pgfsetstrokecolor{currentstroke}%
\pgfsetdash{}{0pt}%
\pgfpathmoveto{\pgfqpoint{8.364505in}{1.941203in}}%
\pgfpathlineto{\pgfqpoint{8.364505in}{1.969885in}}%
\pgfusepath{stroke}%
\end{pgfscope}%
\begin{pgfscope}%
\pgfpathrectangle{\pgfqpoint{1.286132in}{0.839159in}}{\pgfqpoint{12.053712in}{5.967710in}}%
\pgfusepath{clip}%
\pgfsetbuttcap%
\pgfsetroundjoin%
\pgfsetlinewidth{1.505625pt}%
\definecolor{currentstroke}{rgb}{1.000000,0.498039,0.054902}%
\pgfsetstrokecolor{currentstroke}%
\pgfsetdash{}{0pt}%
\pgfpathmoveto{\pgfqpoint{8.475191in}{1.985568in}}%
\pgfpathlineto{\pgfqpoint{8.475191in}{2.299415in}}%
\pgfusepath{stroke}%
\end{pgfscope}%
\begin{pgfscope}%
\pgfpathrectangle{\pgfqpoint{1.286132in}{0.839159in}}{\pgfqpoint{12.053712in}{5.967710in}}%
\pgfusepath{clip}%
\pgfsetbuttcap%
\pgfsetroundjoin%
\pgfsetlinewidth{1.505625pt}%
\definecolor{currentstroke}{rgb}{1.000000,0.498039,0.054902}%
\pgfsetstrokecolor{currentstroke}%
\pgfsetdash{}{0pt}%
\pgfpathmoveto{\pgfqpoint{8.585877in}{1.926991in}}%
\pgfpathlineto{\pgfqpoint{8.585877in}{2.314570in}}%
\pgfusepath{stroke}%
\end{pgfscope}%
\begin{pgfscope}%
\pgfpathrectangle{\pgfqpoint{1.286132in}{0.839159in}}{\pgfqpoint{12.053712in}{5.967710in}}%
\pgfusepath{clip}%
\pgfsetbuttcap%
\pgfsetroundjoin%
\pgfsetlinewidth{1.505625pt}%
\definecolor{currentstroke}{rgb}{1.000000,0.498039,0.054902}%
\pgfsetstrokecolor{currentstroke}%
\pgfsetdash{}{0pt}%
\pgfpathmoveto{\pgfqpoint{8.696563in}{2.088558in}}%
\pgfpathlineto{\pgfqpoint{8.696563in}{2.155716in}}%
\pgfusepath{stroke}%
\end{pgfscope}%
\begin{pgfscope}%
\pgfpathrectangle{\pgfqpoint{1.286132in}{0.839159in}}{\pgfqpoint{12.053712in}{5.967710in}}%
\pgfusepath{clip}%
\pgfsetbuttcap%
\pgfsetroundjoin%
\pgfsetlinewidth{1.505625pt}%
\definecolor{currentstroke}{rgb}{1.000000,0.498039,0.054902}%
\pgfsetstrokecolor{currentstroke}%
\pgfsetdash{}{0pt}%
\pgfpathmoveto{\pgfqpoint{8.807249in}{2.057109in}}%
\pgfpathlineto{\pgfqpoint{8.807249in}{2.095926in}}%
\pgfusepath{stroke}%
\end{pgfscope}%
\begin{pgfscope}%
\pgfpathrectangle{\pgfqpoint{1.286132in}{0.839159in}}{\pgfqpoint{12.053712in}{5.967710in}}%
\pgfusepath{clip}%
\pgfsetbuttcap%
\pgfsetroundjoin%
\pgfsetlinewidth{1.505625pt}%
\definecolor{currentstroke}{rgb}{1.000000,0.498039,0.054902}%
\pgfsetstrokecolor{currentstroke}%
\pgfsetdash{}{0pt}%
\pgfpathmoveto{\pgfqpoint{8.917936in}{2.059676in}}%
\pgfpathlineto{\pgfqpoint{8.917936in}{2.185966in}}%
\pgfusepath{stroke}%
\end{pgfscope}%
\begin{pgfscope}%
\pgfpathrectangle{\pgfqpoint{1.286132in}{0.839159in}}{\pgfqpoint{12.053712in}{5.967710in}}%
\pgfusepath{clip}%
\pgfsetbuttcap%
\pgfsetroundjoin%
\pgfsetlinewidth{1.505625pt}%
\definecolor{currentstroke}{rgb}{1.000000,0.498039,0.054902}%
\pgfsetstrokecolor{currentstroke}%
\pgfsetdash{}{0pt}%
\pgfpathmoveto{\pgfqpoint{9.028622in}{2.131616in}}%
\pgfpathlineto{\pgfqpoint{9.028622in}{2.143359in}}%
\pgfusepath{stroke}%
\end{pgfscope}%
\begin{pgfscope}%
\pgfpathrectangle{\pgfqpoint{1.286132in}{0.839159in}}{\pgfqpoint{12.053712in}{5.967710in}}%
\pgfusepath{clip}%
\pgfsetbuttcap%
\pgfsetroundjoin%
\pgfsetlinewidth{1.505625pt}%
\definecolor{currentstroke}{rgb}{1.000000,0.498039,0.054902}%
\pgfsetstrokecolor{currentstroke}%
\pgfsetdash{}{0pt}%
\pgfpathmoveto{\pgfqpoint{9.139308in}{2.175326in}}%
\pgfpathlineto{\pgfqpoint{9.139308in}{2.178889in}}%
\pgfusepath{stroke}%
\end{pgfscope}%
\begin{pgfscope}%
\pgfpathrectangle{\pgfqpoint{1.286132in}{0.839159in}}{\pgfqpoint{12.053712in}{5.967710in}}%
\pgfusepath{clip}%
\pgfsetbuttcap%
\pgfsetroundjoin%
\pgfsetlinewidth{1.505625pt}%
\definecolor{currentstroke}{rgb}{1.000000,0.498039,0.054902}%
\pgfsetstrokecolor{currentstroke}%
\pgfsetdash{}{0pt}%
\pgfpathmoveto{\pgfqpoint{9.249994in}{2.226100in}}%
\pgfpathlineto{\pgfqpoint{9.249994in}{2.324908in}}%
\pgfusepath{stroke}%
\end{pgfscope}%
\begin{pgfscope}%
\pgfpathrectangle{\pgfqpoint{1.286132in}{0.839159in}}{\pgfqpoint{12.053712in}{5.967710in}}%
\pgfusepath{clip}%
\pgfsetbuttcap%
\pgfsetroundjoin%
\pgfsetlinewidth{1.505625pt}%
\definecolor{currentstroke}{rgb}{1.000000,0.498039,0.054902}%
\pgfsetstrokecolor{currentstroke}%
\pgfsetdash{}{0pt}%
\pgfpathmoveto{\pgfqpoint{9.360680in}{2.269714in}}%
\pgfpathlineto{\pgfqpoint{9.360680in}{2.291986in}}%
\pgfusepath{stroke}%
\end{pgfscope}%
\begin{pgfscope}%
\pgfpathrectangle{\pgfqpoint{1.286132in}{0.839159in}}{\pgfqpoint{12.053712in}{5.967710in}}%
\pgfusepath{clip}%
\pgfsetbuttcap%
\pgfsetroundjoin%
\pgfsetlinewidth{1.505625pt}%
\definecolor{currentstroke}{rgb}{1.000000,0.498039,0.054902}%
\pgfsetstrokecolor{currentstroke}%
\pgfsetdash{}{0pt}%
\pgfpathmoveto{\pgfqpoint{9.471366in}{2.317876in}}%
\pgfpathlineto{\pgfqpoint{9.471366in}{2.331974in}}%
\pgfusepath{stroke}%
\end{pgfscope}%
\begin{pgfscope}%
\pgfpathrectangle{\pgfqpoint{1.286132in}{0.839159in}}{\pgfqpoint{12.053712in}{5.967710in}}%
\pgfusepath{clip}%
\pgfsetbuttcap%
\pgfsetroundjoin%
\pgfsetlinewidth{1.505625pt}%
\definecolor{currentstroke}{rgb}{1.000000,0.498039,0.054902}%
\pgfsetstrokecolor{currentstroke}%
\pgfsetdash{}{0pt}%
\pgfpathmoveto{\pgfqpoint{9.582052in}{2.373454in}}%
\pgfpathlineto{\pgfqpoint{9.582052in}{2.380799in}}%
\pgfusepath{stroke}%
\end{pgfscope}%
\begin{pgfscope}%
\pgfpathrectangle{\pgfqpoint{1.286132in}{0.839159in}}{\pgfqpoint{12.053712in}{5.967710in}}%
\pgfusepath{clip}%
\pgfsetbuttcap%
\pgfsetroundjoin%
\pgfsetlinewidth{1.505625pt}%
\definecolor{currentstroke}{rgb}{1.000000,0.498039,0.054902}%
\pgfsetstrokecolor{currentstroke}%
\pgfsetdash{}{0pt}%
\pgfpathmoveto{\pgfqpoint{9.692738in}{2.426989in}}%
\pgfpathlineto{\pgfqpoint{9.692738in}{2.430474in}}%
\pgfusepath{stroke}%
\end{pgfscope}%
\begin{pgfscope}%
\pgfpathrectangle{\pgfqpoint{1.286132in}{0.839159in}}{\pgfqpoint{12.053712in}{5.967710in}}%
\pgfusepath{clip}%
\pgfsetbuttcap%
\pgfsetroundjoin%
\pgfsetlinewidth{1.505625pt}%
\definecolor{currentstroke}{rgb}{1.000000,0.498039,0.054902}%
\pgfsetstrokecolor{currentstroke}%
\pgfsetdash{}{0pt}%
\pgfpathmoveto{\pgfqpoint{9.803424in}{2.480647in}}%
\pgfpathlineto{\pgfqpoint{9.803424in}{2.499883in}}%
\pgfusepath{stroke}%
\end{pgfscope}%
\begin{pgfscope}%
\pgfpathrectangle{\pgfqpoint{1.286132in}{0.839159in}}{\pgfqpoint{12.053712in}{5.967710in}}%
\pgfusepath{clip}%
\pgfsetbuttcap%
\pgfsetroundjoin%
\pgfsetlinewidth{1.505625pt}%
\definecolor{currentstroke}{rgb}{1.000000,0.498039,0.054902}%
\pgfsetstrokecolor{currentstroke}%
\pgfsetdash{}{0pt}%
\pgfpathmoveto{\pgfqpoint{9.914110in}{2.540928in}}%
\pgfpathlineto{\pgfqpoint{9.914110in}{2.618161in}}%
\pgfusepath{stroke}%
\end{pgfscope}%
\begin{pgfscope}%
\pgfpathrectangle{\pgfqpoint{1.286132in}{0.839159in}}{\pgfqpoint{12.053712in}{5.967710in}}%
\pgfusepath{clip}%
\pgfsetbuttcap%
\pgfsetroundjoin%
\pgfsetlinewidth{1.505625pt}%
\definecolor{currentstroke}{rgb}{1.000000,0.498039,0.054902}%
\pgfsetstrokecolor{currentstroke}%
\pgfsetdash{}{0pt}%
\pgfpathmoveto{\pgfqpoint{10.024796in}{2.590613in}}%
\pgfpathlineto{\pgfqpoint{10.024796in}{2.604134in}}%
\pgfusepath{stroke}%
\end{pgfscope}%
\begin{pgfscope}%
\pgfpathrectangle{\pgfqpoint{1.286132in}{0.839159in}}{\pgfqpoint{12.053712in}{5.967710in}}%
\pgfusepath{clip}%
\pgfsetbuttcap%
\pgfsetroundjoin%
\pgfsetlinewidth{1.505625pt}%
\definecolor{currentstroke}{rgb}{1.000000,0.498039,0.054902}%
\pgfsetstrokecolor{currentstroke}%
\pgfsetdash{}{0pt}%
\pgfpathmoveto{\pgfqpoint{10.135482in}{2.648247in}}%
\pgfpathlineto{\pgfqpoint{10.135482in}{2.671900in}}%
\pgfusepath{stroke}%
\end{pgfscope}%
\begin{pgfscope}%
\pgfpathrectangle{\pgfqpoint{1.286132in}{0.839159in}}{\pgfqpoint{12.053712in}{5.967710in}}%
\pgfusepath{clip}%
\pgfsetbuttcap%
\pgfsetroundjoin%
\pgfsetlinewidth{1.505625pt}%
\definecolor{currentstroke}{rgb}{1.000000,0.498039,0.054902}%
\pgfsetstrokecolor{currentstroke}%
\pgfsetdash{}{0pt}%
\pgfpathmoveto{\pgfqpoint{10.246168in}{2.705901in}}%
\pgfpathlineto{\pgfqpoint{10.246168in}{2.735276in}}%
\pgfusepath{stroke}%
\end{pgfscope}%
\begin{pgfscope}%
\pgfpathrectangle{\pgfqpoint{1.286132in}{0.839159in}}{\pgfqpoint{12.053712in}{5.967710in}}%
\pgfusepath{clip}%
\pgfsetbuttcap%
\pgfsetroundjoin%
\pgfsetlinewidth{1.505625pt}%
\definecolor{currentstroke}{rgb}{1.000000,0.498039,0.054902}%
\pgfsetstrokecolor{currentstroke}%
\pgfsetdash{}{0pt}%
\pgfpathmoveto{\pgfqpoint{10.356854in}{2.761861in}}%
\pgfpathlineto{\pgfqpoint{10.356854in}{2.846307in}}%
\pgfusepath{stroke}%
\end{pgfscope}%
\begin{pgfscope}%
\pgfpathrectangle{\pgfqpoint{1.286132in}{0.839159in}}{\pgfqpoint{12.053712in}{5.967710in}}%
\pgfusepath{clip}%
\pgfsetbuttcap%
\pgfsetroundjoin%
\pgfsetlinewidth{1.505625pt}%
\definecolor{currentstroke}{rgb}{1.000000,0.498039,0.054902}%
\pgfsetstrokecolor{currentstroke}%
\pgfsetdash{}{0pt}%
\pgfpathmoveto{\pgfqpoint{10.467540in}{2.829910in}}%
\pgfpathlineto{\pgfqpoint{10.467540in}{2.899375in}}%
\pgfusepath{stroke}%
\end{pgfscope}%
\begin{pgfscope}%
\pgfpathrectangle{\pgfqpoint{1.286132in}{0.839159in}}{\pgfqpoint{12.053712in}{5.967710in}}%
\pgfusepath{clip}%
\pgfsetbuttcap%
\pgfsetroundjoin%
\pgfsetlinewidth{1.505625pt}%
\definecolor{currentstroke}{rgb}{1.000000,0.498039,0.054902}%
\pgfsetstrokecolor{currentstroke}%
\pgfsetdash{}{0pt}%
\pgfpathmoveto{\pgfqpoint{10.578226in}{2.909343in}}%
\pgfpathlineto{\pgfqpoint{10.578226in}{2.927922in}}%
\pgfusepath{stroke}%
\end{pgfscope}%
\begin{pgfscope}%
\pgfpathrectangle{\pgfqpoint{1.286132in}{0.839159in}}{\pgfqpoint{12.053712in}{5.967710in}}%
\pgfusepath{clip}%
\pgfsetbuttcap%
\pgfsetroundjoin%
\pgfsetlinewidth{1.505625pt}%
\definecolor{currentstroke}{rgb}{1.000000,0.498039,0.054902}%
\pgfsetstrokecolor{currentstroke}%
\pgfsetdash{}{0pt}%
\pgfpathmoveto{\pgfqpoint{10.688913in}{2.969483in}}%
\pgfpathlineto{\pgfqpoint{10.688913in}{2.982832in}}%
\pgfusepath{stroke}%
\end{pgfscope}%
\begin{pgfscope}%
\pgfpathrectangle{\pgfqpoint{1.286132in}{0.839159in}}{\pgfqpoint{12.053712in}{5.967710in}}%
\pgfusepath{clip}%
\pgfsetbuttcap%
\pgfsetroundjoin%
\pgfsetlinewidth{1.505625pt}%
\definecolor{currentstroke}{rgb}{1.000000,0.498039,0.054902}%
\pgfsetstrokecolor{currentstroke}%
\pgfsetdash{}{0pt}%
\pgfpathmoveto{\pgfqpoint{10.799599in}{2.977945in}}%
\pgfpathlineto{\pgfqpoint{10.799599in}{3.283829in}}%
\pgfusepath{stroke}%
\end{pgfscope}%
\begin{pgfscope}%
\pgfpathrectangle{\pgfqpoint{1.286132in}{0.839159in}}{\pgfqpoint{12.053712in}{5.967710in}}%
\pgfusepath{clip}%
\pgfsetbuttcap%
\pgfsetroundjoin%
\pgfsetlinewidth{1.505625pt}%
\definecolor{currentstroke}{rgb}{1.000000,0.498039,0.054902}%
\pgfsetstrokecolor{currentstroke}%
\pgfsetdash{}{0pt}%
\pgfpathmoveto{\pgfqpoint{10.910285in}{3.119831in}}%
\pgfpathlineto{\pgfqpoint{10.910285in}{3.135405in}}%
\pgfusepath{stroke}%
\end{pgfscope}%
\begin{pgfscope}%
\pgfpathrectangle{\pgfqpoint{1.286132in}{0.839159in}}{\pgfqpoint{12.053712in}{5.967710in}}%
\pgfusepath{clip}%
\pgfsetbuttcap%
\pgfsetroundjoin%
\pgfsetlinewidth{1.505625pt}%
\definecolor{currentstroke}{rgb}{1.000000,0.498039,0.054902}%
\pgfsetstrokecolor{currentstroke}%
\pgfsetdash{}{0pt}%
\pgfpathmoveto{\pgfqpoint{11.020971in}{3.185607in}}%
\pgfpathlineto{\pgfqpoint{11.020971in}{3.190935in}}%
\pgfusepath{stroke}%
\end{pgfscope}%
\begin{pgfscope}%
\pgfpathrectangle{\pgfqpoint{1.286132in}{0.839159in}}{\pgfqpoint{12.053712in}{5.967710in}}%
\pgfusepath{clip}%
\pgfsetbuttcap%
\pgfsetroundjoin%
\pgfsetlinewidth{1.505625pt}%
\definecolor{currentstroke}{rgb}{1.000000,0.498039,0.054902}%
\pgfsetstrokecolor{currentstroke}%
\pgfsetdash{}{0pt}%
\pgfpathmoveto{\pgfqpoint{11.131657in}{3.248022in}}%
\pgfpathlineto{\pgfqpoint{11.131657in}{3.283242in}}%
\pgfusepath{stroke}%
\end{pgfscope}%
\begin{pgfscope}%
\pgfpathrectangle{\pgfqpoint{1.286132in}{0.839159in}}{\pgfqpoint{12.053712in}{5.967710in}}%
\pgfusepath{clip}%
\pgfsetbuttcap%
\pgfsetroundjoin%
\pgfsetlinewidth{1.505625pt}%
\definecolor{currentstroke}{rgb}{1.000000,0.498039,0.054902}%
\pgfsetstrokecolor{currentstroke}%
\pgfsetdash{}{0pt}%
\pgfpathmoveto{\pgfqpoint{11.242343in}{3.261539in}}%
\pgfpathlineto{\pgfqpoint{11.242343in}{3.644262in}}%
\pgfusepath{stroke}%
\end{pgfscope}%
\begin{pgfscope}%
\pgfpathrectangle{\pgfqpoint{1.286132in}{0.839159in}}{\pgfqpoint{12.053712in}{5.967710in}}%
\pgfusepath{clip}%
\pgfsetbuttcap%
\pgfsetroundjoin%
\pgfsetlinewidth{1.505625pt}%
\definecolor{currentstroke}{rgb}{1.000000,0.498039,0.054902}%
\pgfsetstrokecolor{currentstroke}%
\pgfsetdash{}{0pt}%
\pgfpathmoveto{\pgfqpoint{11.353029in}{3.413400in}}%
\pgfpathlineto{\pgfqpoint{11.353029in}{3.431797in}}%
\pgfusepath{stroke}%
\end{pgfscope}%
\begin{pgfscope}%
\pgfpathrectangle{\pgfqpoint{1.286132in}{0.839159in}}{\pgfqpoint{12.053712in}{5.967710in}}%
\pgfusepath{clip}%
\pgfsetbuttcap%
\pgfsetroundjoin%
\pgfsetlinewidth{1.505625pt}%
\definecolor{currentstroke}{rgb}{1.000000,0.498039,0.054902}%
\pgfsetstrokecolor{currentstroke}%
\pgfsetdash{}{0pt}%
\pgfpathmoveto{\pgfqpoint{11.463715in}{3.491751in}}%
\pgfpathlineto{\pgfqpoint{11.463715in}{3.509252in}}%
\pgfusepath{stroke}%
\end{pgfscope}%
\begin{pgfscope}%
\pgfpathrectangle{\pgfqpoint{1.286132in}{0.839159in}}{\pgfqpoint{12.053712in}{5.967710in}}%
\pgfusepath{clip}%
\pgfsetbuttcap%
\pgfsetroundjoin%
\pgfsetlinewidth{1.505625pt}%
\definecolor{currentstroke}{rgb}{1.000000,0.498039,0.054902}%
\pgfsetstrokecolor{currentstroke}%
\pgfsetdash{}{0pt}%
\pgfpathmoveto{\pgfqpoint{11.574401in}{3.512358in}}%
\pgfpathlineto{\pgfqpoint{11.574401in}{3.783010in}}%
\pgfusepath{stroke}%
\end{pgfscope}%
\begin{pgfscope}%
\pgfpathrectangle{\pgfqpoint{1.286132in}{0.839159in}}{\pgfqpoint{12.053712in}{5.967710in}}%
\pgfusepath{clip}%
\pgfsetbuttcap%
\pgfsetroundjoin%
\pgfsetlinewidth{1.505625pt}%
\definecolor{currentstroke}{rgb}{1.000000,0.498039,0.054902}%
\pgfsetstrokecolor{currentstroke}%
\pgfsetdash{}{0pt}%
\pgfpathmoveto{\pgfqpoint{11.685087in}{3.648793in}}%
\pgfpathlineto{\pgfqpoint{11.685087in}{3.665903in}}%
\pgfusepath{stroke}%
\end{pgfscope}%
\begin{pgfscope}%
\pgfpathrectangle{\pgfqpoint{1.286132in}{0.839159in}}{\pgfqpoint{12.053712in}{5.967710in}}%
\pgfusepath{clip}%
\pgfsetbuttcap%
\pgfsetroundjoin%
\pgfsetlinewidth{1.505625pt}%
\definecolor{currentstroke}{rgb}{1.000000,0.498039,0.054902}%
\pgfsetstrokecolor{currentstroke}%
\pgfsetdash{}{0pt}%
\pgfpathmoveto{\pgfqpoint{11.795773in}{3.740186in}}%
\pgfpathlineto{\pgfqpoint{11.795773in}{3.757138in}}%
\pgfusepath{stroke}%
\end{pgfscope}%
\begin{pgfscope}%
\pgfpathrectangle{\pgfqpoint{1.286132in}{0.839159in}}{\pgfqpoint{12.053712in}{5.967710in}}%
\pgfusepath{clip}%
\pgfsetbuttcap%
\pgfsetroundjoin%
\pgfsetlinewidth{1.505625pt}%
\definecolor{currentstroke}{rgb}{1.000000,0.498039,0.054902}%
\pgfsetstrokecolor{currentstroke}%
\pgfsetdash{}{0pt}%
\pgfpathmoveto{\pgfqpoint{11.906459in}{3.811599in}}%
\pgfpathlineto{\pgfqpoint{11.906459in}{3.997036in}}%
\pgfusepath{stroke}%
\end{pgfscope}%
\begin{pgfscope}%
\pgfpathrectangle{\pgfqpoint{1.286132in}{0.839159in}}{\pgfqpoint{12.053712in}{5.967710in}}%
\pgfusepath{clip}%
\pgfsetbuttcap%
\pgfsetroundjoin%
\pgfsetlinewidth{1.505625pt}%
\definecolor{currentstroke}{rgb}{1.000000,0.498039,0.054902}%
\pgfsetstrokecolor{currentstroke}%
\pgfsetdash{}{0pt}%
\pgfpathmoveto{\pgfqpoint{12.017145in}{3.933983in}}%
\pgfpathlineto{\pgfqpoint{12.017145in}{3.993304in}}%
\pgfusepath{stroke}%
\end{pgfscope}%
\begin{pgfscope}%
\pgfpathrectangle{\pgfqpoint{1.286132in}{0.839159in}}{\pgfqpoint{12.053712in}{5.967710in}}%
\pgfusepath{clip}%
\pgfsetbuttcap%
\pgfsetroundjoin%
\pgfsetlinewidth{1.505625pt}%
\definecolor{currentstroke}{rgb}{1.000000,0.498039,0.054902}%
\pgfsetstrokecolor{currentstroke}%
\pgfsetdash{}{0pt}%
\pgfpathmoveto{\pgfqpoint{12.127831in}{4.013827in}}%
\pgfpathlineto{\pgfqpoint{12.127831in}{4.437731in}}%
\pgfusepath{stroke}%
\end{pgfscope}%
\begin{pgfscope}%
\pgfpathrectangle{\pgfqpoint{1.286132in}{0.839159in}}{\pgfqpoint{12.053712in}{5.967710in}}%
\pgfusepath{clip}%
\pgfsetbuttcap%
\pgfsetroundjoin%
\pgfsetlinewidth{1.505625pt}%
\definecolor{currentstroke}{rgb}{1.000000,0.498039,0.054902}%
\pgfsetstrokecolor{currentstroke}%
\pgfsetdash{}{0pt}%
\pgfpathmoveto{\pgfqpoint{12.238517in}{4.095145in}}%
\pgfpathlineto{\pgfqpoint{12.238517in}{4.274773in}}%
\pgfusepath{stroke}%
\end{pgfscope}%
\begin{pgfscope}%
\pgfpathrectangle{\pgfqpoint{1.286132in}{0.839159in}}{\pgfqpoint{12.053712in}{5.967710in}}%
\pgfusepath{clip}%
\pgfsetbuttcap%
\pgfsetroundjoin%
\pgfsetlinewidth{1.505625pt}%
\definecolor{currentstroke}{rgb}{1.000000,0.498039,0.054902}%
\pgfsetstrokecolor{currentstroke}%
\pgfsetdash{}{0pt}%
\pgfpathmoveto{\pgfqpoint{12.349203in}{4.188394in}}%
\pgfpathlineto{\pgfqpoint{12.349203in}{4.225607in}}%
\pgfusepath{stroke}%
\end{pgfscope}%
\begin{pgfscope}%
\pgfpathrectangle{\pgfqpoint{1.286132in}{0.839159in}}{\pgfqpoint{12.053712in}{5.967710in}}%
\pgfusepath{clip}%
\pgfsetbuttcap%
\pgfsetroundjoin%
\pgfsetlinewidth{1.505625pt}%
\definecolor{currentstroke}{rgb}{1.000000,0.498039,0.054902}%
\pgfsetstrokecolor{currentstroke}%
\pgfsetdash{}{0pt}%
\pgfpathmoveto{\pgfqpoint{12.459890in}{4.561412in}}%
\pgfpathlineto{\pgfqpoint{12.459890in}{5.201006in}}%
\pgfusepath{stroke}%
\end{pgfscope}%
\begin{pgfscope}%
\pgfpathrectangle{\pgfqpoint{1.286132in}{0.839159in}}{\pgfqpoint{12.053712in}{5.967710in}}%
\pgfusepath{clip}%
\pgfsetbuttcap%
\pgfsetroundjoin%
\pgfsetlinewidth{1.505625pt}%
\definecolor{currentstroke}{rgb}{1.000000,0.498039,0.054902}%
\pgfsetstrokecolor{currentstroke}%
\pgfsetdash{}{0pt}%
\pgfpathmoveto{\pgfqpoint{12.570576in}{4.967308in}}%
\pgfpathlineto{\pgfqpoint{12.570576in}{5.093113in}}%
\pgfusepath{stroke}%
\end{pgfscope}%
\begin{pgfscope}%
\pgfpathrectangle{\pgfqpoint{1.286132in}{0.839159in}}{\pgfqpoint{12.053712in}{5.967710in}}%
\pgfusepath{clip}%
\pgfsetbuttcap%
\pgfsetroundjoin%
\pgfsetlinewidth{1.505625pt}%
\definecolor{currentstroke}{rgb}{1.000000,0.498039,0.054902}%
\pgfsetstrokecolor{currentstroke}%
\pgfsetdash{}{0pt}%
\pgfpathmoveto{\pgfqpoint{12.681262in}{4.934303in}}%
\pgfpathlineto{\pgfqpoint{12.681262in}{5.885952in}}%
\pgfusepath{stroke}%
\end{pgfscope}%
\begin{pgfscope}%
\pgfpathrectangle{\pgfqpoint{1.286132in}{0.839159in}}{\pgfqpoint{12.053712in}{5.967710in}}%
\pgfusepath{clip}%
\pgfsetbuttcap%
\pgfsetroundjoin%
\pgfsetlinewidth{1.505625pt}%
\definecolor{currentstroke}{rgb}{1.000000,0.498039,0.054902}%
\pgfsetstrokecolor{currentstroke}%
\pgfsetdash{}{0pt}%
\pgfpathmoveto{\pgfqpoint{12.791948in}{4.922052in}}%
\pgfpathlineto{\pgfqpoint{12.791948in}{5.116709in}}%
\pgfusepath{stroke}%
\end{pgfscope}%
\begin{pgfscope}%
\pgfpathrectangle{\pgfqpoint{1.286132in}{0.839159in}}{\pgfqpoint{12.053712in}{5.967710in}}%
\pgfusepath{clip}%
\pgfsetrectcap%
\pgfsetroundjoin%
\pgfsetlinewidth{1.505625pt}%
\definecolor{currentstroke}{rgb}{0.121569,0.466667,0.705882}%
\pgfsetstrokecolor{currentstroke}%
\pgfsetdash{}{0pt}%
\pgfpathmoveto{\pgfqpoint{1.834028in}{1.119284in}}%
\pgfpathlineto{\pgfqpoint{1.944714in}{1.119257in}}%
\pgfpathlineto{\pgfqpoint{2.055400in}{1.119326in}}%
\pgfpathlineto{\pgfqpoint{2.166086in}{1.119435in}}%
\pgfpathlineto{\pgfqpoint{2.276772in}{1.119507in}}%
\pgfpathlineto{\pgfqpoint{2.387458in}{1.119642in}}%
\pgfpathlineto{\pgfqpoint{2.498144in}{1.119693in}}%
\pgfpathlineto{\pgfqpoint{2.608830in}{1.120017in}}%
\pgfpathlineto{\pgfqpoint{2.719516in}{1.119925in}}%
\pgfpathlineto{\pgfqpoint{2.830202in}{1.119902in}}%
\pgfpathlineto{\pgfqpoint{2.940888in}{1.119918in}}%
\pgfpathlineto{\pgfqpoint{3.051574in}{1.120297in}}%
\pgfpathlineto{\pgfqpoint{3.162260in}{1.120298in}}%
\pgfpathlineto{\pgfqpoint{3.272946in}{1.120227in}}%
\pgfpathlineto{\pgfqpoint{3.383632in}{1.121442in}}%
\pgfpathlineto{\pgfqpoint{3.494319in}{1.120347in}}%
\pgfpathlineto{\pgfqpoint{3.605005in}{1.122077in}}%
\pgfpathlineto{\pgfqpoint{3.715691in}{1.120650in}}%
\pgfpathlineto{\pgfqpoint{3.826377in}{1.120715in}}%
\pgfpathlineto{\pgfqpoint{3.937063in}{1.120611in}}%
\pgfpathlineto{\pgfqpoint{4.047749in}{1.120893in}}%
\pgfpathlineto{\pgfqpoint{4.158435in}{1.120834in}}%
\pgfpathlineto{\pgfqpoint{4.269121in}{1.121578in}}%
\pgfpathlineto{\pgfqpoint{4.379807in}{1.121020in}}%
\pgfpathlineto{\pgfqpoint{4.490493in}{1.121350in}}%
\pgfpathlineto{\pgfqpoint{4.601179in}{1.122059in}}%
\pgfpathlineto{\pgfqpoint{4.711865in}{1.122048in}}%
\pgfpathlineto{\pgfqpoint{4.822551in}{1.122014in}}%
\pgfpathlineto{\pgfqpoint{4.933237in}{1.121867in}}%
\pgfpathlineto{\pgfqpoint{5.043923in}{1.122084in}}%
\pgfpathlineto{\pgfqpoint{5.154609in}{1.122030in}}%
\pgfpathlineto{\pgfqpoint{5.265296in}{1.123056in}}%
\pgfpathlineto{\pgfqpoint{5.375982in}{1.122855in}}%
\pgfpathlineto{\pgfqpoint{5.486668in}{1.122828in}}%
\pgfpathlineto{\pgfqpoint{5.597354in}{1.122430in}}%
\pgfpathlineto{\pgfqpoint{5.708040in}{1.128718in}}%
\pgfpathlineto{\pgfqpoint{5.818726in}{1.122202in}}%
\pgfpathlineto{\pgfqpoint{5.929412in}{1.123365in}}%
\pgfpathlineto{\pgfqpoint{6.040098in}{1.123190in}}%
\pgfpathlineto{\pgfqpoint{6.150784in}{1.123786in}}%
\pgfpathlineto{\pgfqpoint{6.261470in}{1.122733in}}%
\pgfpathlineto{\pgfqpoint{6.372156in}{1.124035in}}%
\pgfpathlineto{\pgfqpoint{6.482842in}{1.123422in}}%
\pgfpathlineto{\pgfqpoint{6.593528in}{1.123750in}}%
\pgfpathlineto{\pgfqpoint{6.704214in}{1.123984in}}%
\pgfpathlineto{\pgfqpoint{6.814900in}{1.123681in}}%
\pgfpathlineto{\pgfqpoint{6.925586in}{1.123420in}}%
\pgfpathlineto{\pgfqpoint{7.036272in}{1.123855in}}%
\pgfpathlineto{\pgfqpoint{7.146959in}{1.123901in}}%
\pgfpathlineto{\pgfqpoint{7.257645in}{1.124977in}}%
\pgfpathlineto{\pgfqpoint{7.368331in}{1.126331in}}%
\pgfpathlineto{\pgfqpoint{7.479017in}{1.125369in}}%
\pgfpathlineto{\pgfqpoint{7.589703in}{1.125757in}}%
\pgfpathlineto{\pgfqpoint{7.700389in}{1.125019in}}%
\pgfpathlineto{\pgfqpoint{7.811075in}{1.125301in}}%
\pgfpathlineto{\pgfqpoint{7.921761in}{1.124921in}}%
\pgfpathlineto{\pgfqpoint{8.032447in}{1.124581in}}%
\pgfpathlineto{\pgfqpoint{8.143133in}{1.124910in}}%
\pgfpathlineto{\pgfqpoint{8.253819in}{1.124782in}}%
\pgfpathlineto{\pgfqpoint{8.364505in}{1.125546in}}%
\pgfpathlineto{\pgfqpoint{8.475191in}{1.126094in}}%
\pgfpathlineto{\pgfqpoint{8.585877in}{1.125924in}}%
\pgfpathlineto{\pgfqpoint{8.696563in}{1.125050in}}%
\pgfpathlineto{\pgfqpoint{8.807249in}{1.127339in}}%
\pgfpathlineto{\pgfqpoint{8.917936in}{1.126073in}}%
\pgfpathlineto{\pgfqpoint{9.028622in}{1.126352in}}%
\pgfpathlineto{\pgfqpoint{9.139308in}{1.125842in}}%
\pgfpathlineto{\pgfqpoint{9.249994in}{1.126889in}}%
\pgfpathlineto{\pgfqpoint{9.360680in}{1.126270in}}%
\pgfpathlineto{\pgfqpoint{9.471366in}{1.127105in}}%
\pgfpathlineto{\pgfqpoint{9.582052in}{1.128153in}}%
\pgfpathlineto{\pgfqpoint{9.692738in}{1.127444in}}%
\pgfpathlineto{\pgfqpoint{9.803424in}{1.126881in}}%
\pgfpathlineto{\pgfqpoint{9.914110in}{1.127907in}}%
\pgfpathlineto{\pgfqpoint{10.024796in}{1.127394in}}%
\pgfpathlineto{\pgfqpoint{10.135482in}{1.127286in}}%
\pgfpathlineto{\pgfqpoint{10.246168in}{1.128636in}}%
\pgfpathlineto{\pgfqpoint{10.356854in}{1.127281in}}%
\pgfpathlineto{\pgfqpoint{10.467540in}{1.127095in}}%
\pgfpathlineto{\pgfqpoint{10.578226in}{1.129063in}}%
\pgfpathlineto{\pgfqpoint{10.688913in}{1.128177in}}%
\pgfpathlineto{\pgfqpoint{10.799599in}{1.129679in}}%
\pgfpathlineto{\pgfqpoint{10.910285in}{1.128514in}}%
\pgfpathlineto{\pgfqpoint{11.020971in}{1.131048in}}%
\pgfpathlineto{\pgfqpoint{11.131657in}{1.128050in}}%
\pgfpathlineto{\pgfqpoint{11.242343in}{1.129586in}}%
\pgfpathlineto{\pgfqpoint{11.353029in}{1.128258in}}%
\pgfpathlineto{\pgfqpoint{11.463715in}{1.128846in}}%
\pgfpathlineto{\pgfqpoint{11.574401in}{1.129889in}}%
\pgfpathlineto{\pgfqpoint{11.685087in}{1.130449in}}%
\pgfpathlineto{\pgfqpoint{11.795773in}{1.130179in}}%
\pgfpathlineto{\pgfqpoint{11.906459in}{1.133156in}}%
\pgfpathlineto{\pgfqpoint{12.017145in}{1.145498in}}%
\pgfpathlineto{\pgfqpoint{12.127831in}{1.130442in}}%
\pgfpathlineto{\pgfqpoint{12.238517in}{1.129386in}}%
\pgfpathlineto{\pgfqpoint{12.349203in}{1.129613in}}%
\pgfpathlineto{\pgfqpoint{12.459890in}{1.131977in}}%
\pgfpathlineto{\pgfqpoint{12.570576in}{1.131238in}}%
\pgfpathlineto{\pgfqpoint{12.681262in}{1.130341in}}%
\pgfpathlineto{\pgfqpoint{12.791948in}{1.129137in}}%
\pgfusepath{stroke}%
\end{pgfscope}%
\begin{pgfscope}%
\pgfpathrectangle{\pgfqpoint{1.286132in}{0.839159in}}{\pgfqpoint{12.053712in}{5.967710in}}%
\pgfusepath{clip}%
\pgfsetrectcap%
\pgfsetroundjoin%
\pgfsetlinewidth{1.505625pt}%
\definecolor{currentstroke}{rgb}{1.000000,0.498039,0.054902}%
\pgfsetstrokecolor{currentstroke}%
\pgfsetdash{}{0pt}%
\pgfpathmoveto{\pgfqpoint{1.834028in}{1.119213in}}%
\pgfpathlineto{\pgfqpoint{1.944714in}{1.119221in}}%
\pgfpathlineto{\pgfqpoint{2.055400in}{1.119259in}}%
\pgfpathlineto{\pgfqpoint{2.166086in}{1.119417in}}%
\pgfpathlineto{\pgfqpoint{2.276772in}{1.119582in}}%
\pgfpathlineto{\pgfqpoint{2.387458in}{1.119815in}}%
\pgfpathlineto{\pgfqpoint{2.498144in}{1.120189in}}%
\pgfpathlineto{\pgfqpoint{2.608830in}{1.120526in}}%
\pgfpathlineto{\pgfqpoint{2.719516in}{1.121147in}}%
\pgfpathlineto{\pgfqpoint{2.830202in}{1.121457in}}%
\pgfpathlineto{\pgfqpoint{2.940888in}{1.127921in}}%
\pgfpathlineto{\pgfqpoint{3.051574in}{1.123293in}}%
\pgfpathlineto{\pgfqpoint{3.162260in}{1.123897in}}%
\pgfpathlineto{\pgfqpoint{3.272946in}{1.128755in}}%
\pgfpathlineto{\pgfqpoint{3.383632in}{1.125984in}}%
\pgfpathlineto{\pgfqpoint{3.494319in}{1.127689in}}%
\pgfpathlineto{\pgfqpoint{3.605005in}{1.127551in}}%
\pgfpathlineto{\pgfqpoint{3.715691in}{1.129210in}}%
\pgfpathlineto{\pgfqpoint{3.826377in}{1.131380in}}%
\pgfpathlineto{\pgfqpoint{3.937063in}{1.131688in}}%
\pgfpathlineto{\pgfqpoint{4.047749in}{1.130656in}}%
\pgfpathlineto{\pgfqpoint{4.158435in}{1.132491in}}%
\pgfpathlineto{\pgfqpoint{4.269121in}{1.136057in}}%
\pgfpathlineto{\pgfqpoint{4.379807in}{1.139049in}}%
\pgfpathlineto{\pgfqpoint{4.490493in}{1.142250in}}%
\pgfpathlineto{\pgfqpoint{4.601179in}{1.145236in}}%
\pgfpathlineto{\pgfqpoint{4.711865in}{1.145065in}}%
\pgfpathlineto{\pgfqpoint{4.822551in}{1.145247in}}%
\pgfpathlineto{\pgfqpoint{4.933237in}{1.150502in}}%
\pgfpathlineto{\pgfqpoint{5.043923in}{1.156747in}}%
\pgfpathlineto{\pgfqpoint{5.154609in}{1.158511in}}%
\pgfpathlineto{\pgfqpoint{5.265296in}{1.154750in}}%
\pgfpathlineto{\pgfqpoint{5.375982in}{1.156518in}}%
\pgfpathlineto{\pgfqpoint{5.486668in}{1.179932in}}%
\pgfpathlineto{\pgfqpoint{5.597354in}{1.184282in}}%
\pgfpathlineto{\pgfqpoint{5.708040in}{1.244235in}}%
\pgfpathlineto{\pgfqpoint{5.818726in}{1.195593in}}%
\pgfpathlineto{\pgfqpoint{5.929412in}{1.179133in}}%
\pgfpathlineto{\pgfqpoint{6.040098in}{1.181653in}}%
\pgfpathlineto{\pgfqpoint{6.150784in}{1.182328in}}%
\pgfpathlineto{\pgfqpoint{6.261470in}{1.189105in}}%
\pgfpathlineto{\pgfqpoint{6.372156in}{1.195059in}}%
\pgfpathlineto{\pgfqpoint{6.482842in}{1.205717in}}%
\pgfpathlineto{\pgfqpoint{6.593528in}{1.192503in}}%
\pgfpathlineto{\pgfqpoint{6.704214in}{1.198180in}}%
\pgfpathlineto{\pgfqpoint{6.814900in}{1.198741in}}%
\pgfpathlineto{\pgfqpoint{6.925586in}{1.206804in}}%
\pgfpathlineto{\pgfqpoint{7.036272in}{1.217863in}}%
\pgfpathlineto{\pgfqpoint{7.146959in}{1.215017in}}%
\pgfpathlineto{\pgfqpoint{7.257645in}{1.231934in}}%
\pgfpathlineto{\pgfqpoint{7.368331in}{1.229013in}}%
\pgfpathlineto{\pgfqpoint{7.479017in}{1.232361in}}%
\pgfpathlineto{\pgfqpoint{7.589703in}{1.278316in}}%
\pgfpathlineto{\pgfqpoint{7.700389in}{1.232703in}}%
\pgfpathlineto{\pgfqpoint{7.811075in}{1.245254in}}%
\pgfpathlineto{\pgfqpoint{7.921761in}{1.262076in}}%
\pgfpathlineto{\pgfqpoint{8.032447in}{1.254803in}}%
\pgfpathlineto{\pgfqpoint{8.143133in}{1.254324in}}%
\pgfpathlineto{\pgfqpoint{8.253819in}{1.276024in}}%
\pgfpathlineto{\pgfqpoint{8.364505in}{1.278704in}}%
\pgfpathlineto{\pgfqpoint{8.475191in}{1.283813in}}%
\pgfpathlineto{\pgfqpoint{8.585877in}{1.275519in}}%
\pgfpathlineto{\pgfqpoint{8.696563in}{1.295590in}}%
\pgfpathlineto{\pgfqpoint{8.807249in}{1.294960in}}%
\pgfpathlineto{\pgfqpoint{8.917936in}{1.328563in}}%
\pgfpathlineto{\pgfqpoint{9.028622in}{1.338672in}}%
\pgfpathlineto{\pgfqpoint{9.139308in}{1.357993in}}%
\pgfpathlineto{\pgfqpoint{9.249994in}{1.323421in}}%
\pgfpathlineto{\pgfqpoint{9.360680in}{1.322530in}}%
\pgfpathlineto{\pgfqpoint{9.471366in}{1.362048in}}%
\pgfpathlineto{\pgfqpoint{9.582052in}{1.334326in}}%
\pgfpathlineto{\pgfqpoint{9.692738in}{1.386812in}}%
\pgfpathlineto{\pgfqpoint{9.803424in}{1.358807in}}%
\pgfpathlineto{\pgfqpoint{9.914110in}{1.412815in}}%
\pgfpathlineto{\pgfqpoint{10.024796in}{1.391431in}}%
\pgfpathlineto{\pgfqpoint{10.135482in}{1.397015in}}%
\pgfpathlineto{\pgfqpoint{10.246168in}{1.362021in}}%
\pgfpathlineto{\pgfqpoint{10.356854in}{1.424636in}}%
\pgfpathlineto{\pgfqpoint{10.467540in}{1.433384in}}%
\pgfpathlineto{\pgfqpoint{10.578226in}{1.478509in}}%
\pgfpathlineto{\pgfqpoint{10.688913in}{1.424573in}}%
\pgfpathlineto{\pgfqpoint{10.799599in}{1.444362in}}%
\pgfpathlineto{\pgfqpoint{10.910285in}{1.494556in}}%
\pgfpathlineto{\pgfqpoint{11.020971in}{1.549160in}}%
\pgfpathlineto{\pgfqpoint{11.131657in}{1.450769in}}%
\pgfpathlineto{\pgfqpoint{11.242343in}{1.462752in}}%
\pgfpathlineto{\pgfqpoint{11.353029in}{1.566345in}}%
\pgfpathlineto{\pgfqpoint{11.463715in}{1.564747in}}%
\pgfpathlineto{\pgfqpoint{11.574401in}{1.486996in}}%
\pgfpathlineto{\pgfqpoint{11.685087in}{1.530304in}}%
\pgfpathlineto{\pgfqpoint{11.795773in}{1.577210in}}%
\pgfpathlineto{\pgfqpoint{11.906459in}{1.576308in}}%
\pgfpathlineto{\pgfqpoint{12.017145in}{1.588542in}}%
\pgfpathlineto{\pgfqpoint{12.127831in}{1.514731in}}%
\pgfpathlineto{\pgfqpoint{12.238517in}{1.580056in}}%
\pgfpathlineto{\pgfqpoint{12.349203in}{1.600517in}}%
\pgfpathlineto{\pgfqpoint{12.459890in}{1.612825in}}%
\pgfpathlineto{\pgfqpoint{12.570576in}{1.656175in}}%
\pgfpathlineto{\pgfqpoint{12.681262in}{1.683376in}}%
\pgfpathlineto{\pgfqpoint{12.791948in}{1.726141in}}%
\pgfusepath{stroke}%
\end{pgfscope}%
\begin{pgfscope}%
\pgfpathrectangle{\pgfqpoint{1.286132in}{0.839159in}}{\pgfqpoint{12.053712in}{5.967710in}}%
\pgfusepath{clip}%
\pgfsetrectcap%
\pgfsetroundjoin%
\pgfsetlinewidth{1.505625pt}%
\definecolor{currentstroke}{rgb}{0.172549,0.627451,0.172549}%
\pgfsetstrokecolor{currentstroke}%
\pgfsetdash{}{0pt}%
\pgfpathmoveto{\pgfqpoint{1.834028in}{1.119240in}}%
\pgfpathlineto{\pgfqpoint{1.944714in}{1.119341in}}%
\pgfpathlineto{\pgfqpoint{2.055400in}{1.119400in}}%
\pgfpathlineto{\pgfqpoint{2.166086in}{1.119448in}}%
\pgfpathlineto{\pgfqpoint{2.276772in}{1.119517in}}%
\pgfpathlineto{\pgfqpoint{2.387458in}{1.119584in}}%
\pgfpathlineto{\pgfqpoint{2.498144in}{1.119656in}}%
\pgfpathlineto{\pgfqpoint{2.608830in}{1.119716in}}%
\pgfpathlineto{\pgfqpoint{2.719516in}{1.119801in}}%
\pgfpathlineto{\pgfqpoint{2.830202in}{1.119886in}}%
\pgfpathlineto{\pgfqpoint{2.940888in}{1.120032in}}%
\pgfpathlineto{\pgfqpoint{3.051574in}{1.120081in}}%
\pgfpathlineto{\pgfqpoint{3.162260in}{1.120235in}}%
\pgfpathlineto{\pgfqpoint{3.272946in}{1.120322in}}%
\pgfpathlineto{\pgfqpoint{3.383632in}{1.120430in}}%
\pgfpathlineto{\pgfqpoint{3.494319in}{1.120550in}}%
\pgfpathlineto{\pgfqpoint{3.605005in}{1.120814in}}%
\pgfpathlineto{\pgfqpoint{3.715691in}{1.120850in}}%
\pgfpathlineto{\pgfqpoint{3.826377in}{1.121007in}}%
\pgfpathlineto{\pgfqpoint{3.937063in}{1.121090in}}%
\pgfpathlineto{\pgfqpoint{4.047749in}{1.121197in}}%
\pgfpathlineto{\pgfqpoint{4.158435in}{1.121491in}}%
\pgfpathlineto{\pgfqpoint{4.269121in}{1.121480in}}%
\pgfpathlineto{\pgfqpoint{4.379807in}{1.121807in}}%
\pgfpathlineto{\pgfqpoint{4.490493in}{1.121790in}}%
\pgfpathlineto{\pgfqpoint{4.601179in}{1.122276in}}%
\pgfpathlineto{\pgfqpoint{4.711865in}{1.122331in}}%
\pgfpathlineto{\pgfqpoint{4.822551in}{1.122364in}}%
\pgfpathlineto{\pgfqpoint{4.933237in}{1.122797in}}%
\pgfpathlineto{\pgfqpoint{5.043923in}{1.122570in}}%
\pgfpathlineto{\pgfqpoint{5.154609in}{1.122948in}}%
\pgfpathlineto{\pgfqpoint{5.265296in}{1.122940in}}%
\pgfpathlineto{\pgfqpoint{5.375982in}{1.130303in}}%
\pgfpathlineto{\pgfqpoint{5.486668in}{1.123688in}}%
\pgfpathlineto{\pgfqpoint{5.597354in}{1.126283in}}%
\pgfpathlineto{\pgfqpoint{5.708040in}{1.126298in}}%
\pgfpathlineto{\pgfqpoint{5.818726in}{1.124144in}}%
\pgfpathlineto{\pgfqpoint{5.929412in}{1.124390in}}%
\pgfpathlineto{\pgfqpoint{6.040098in}{1.124965in}}%
\pgfpathlineto{\pgfqpoint{6.150784in}{1.125011in}}%
\pgfpathlineto{\pgfqpoint{6.261470in}{1.125743in}}%
\pgfpathlineto{\pgfqpoint{6.372156in}{1.124901in}}%
\pgfpathlineto{\pgfqpoint{6.482842in}{1.125187in}}%
\pgfpathlineto{\pgfqpoint{6.593528in}{1.125582in}}%
\pgfpathlineto{\pgfqpoint{6.704214in}{1.125566in}}%
\pgfpathlineto{\pgfqpoint{6.814900in}{1.126088in}}%
\pgfpathlineto{\pgfqpoint{6.925586in}{1.126461in}}%
\pgfpathlineto{\pgfqpoint{7.036272in}{1.126588in}}%
\pgfpathlineto{\pgfqpoint{7.146959in}{1.126810in}}%
\pgfpathlineto{\pgfqpoint{7.257645in}{1.127386in}}%
\pgfpathlineto{\pgfqpoint{7.368331in}{1.127582in}}%
\pgfpathlineto{\pgfqpoint{7.479017in}{1.127971in}}%
\pgfpathlineto{\pgfqpoint{7.589703in}{1.142772in}}%
\pgfpathlineto{\pgfqpoint{7.700389in}{1.129217in}}%
\pgfpathlineto{\pgfqpoint{7.811075in}{1.128798in}}%
\pgfpathlineto{\pgfqpoint{7.921761in}{1.129152in}}%
\pgfpathlineto{\pgfqpoint{8.032447in}{1.129531in}}%
\pgfpathlineto{\pgfqpoint{8.143133in}{1.129787in}}%
\pgfpathlineto{\pgfqpoint{8.253819in}{1.130219in}}%
\pgfpathlineto{\pgfqpoint{8.364505in}{1.130735in}}%
\pgfpathlineto{\pgfqpoint{8.475191in}{1.130987in}}%
\pgfpathlineto{\pgfqpoint{8.585877in}{1.131157in}}%
\pgfpathlineto{\pgfqpoint{8.696563in}{1.131442in}}%
\pgfpathlineto{\pgfqpoint{8.807249in}{1.131879in}}%
\pgfpathlineto{\pgfqpoint{8.917936in}{1.132042in}}%
\pgfpathlineto{\pgfqpoint{9.028622in}{1.132503in}}%
\pgfpathlineto{\pgfqpoint{9.139308in}{1.132452in}}%
\pgfpathlineto{\pgfqpoint{9.249994in}{1.133154in}}%
\pgfpathlineto{\pgfqpoint{9.360680in}{1.133287in}}%
\pgfpathlineto{\pgfqpoint{9.471366in}{1.133699in}}%
\pgfpathlineto{\pgfqpoint{9.582052in}{1.134003in}}%
\pgfpathlineto{\pgfqpoint{9.692738in}{1.134762in}}%
\pgfpathlineto{\pgfqpoint{9.803424in}{1.134568in}}%
\pgfpathlineto{\pgfqpoint{9.914110in}{1.135326in}}%
\pgfpathlineto{\pgfqpoint{10.024796in}{1.135690in}}%
\pgfpathlineto{\pgfqpoint{10.135482in}{1.137001in}}%
\pgfpathlineto{\pgfqpoint{10.246168in}{1.136364in}}%
\pgfpathlineto{\pgfqpoint{10.356854in}{1.136669in}}%
\pgfpathlineto{\pgfqpoint{10.467540in}{1.137750in}}%
\pgfpathlineto{\pgfqpoint{10.578226in}{1.138003in}}%
\pgfpathlineto{\pgfqpoint{10.688913in}{1.138734in}}%
\pgfpathlineto{\pgfqpoint{10.799599in}{1.138718in}}%
\pgfpathlineto{\pgfqpoint{10.910285in}{1.139414in}}%
\pgfpathlineto{\pgfqpoint{11.020971in}{1.140328in}}%
\pgfpathlineto{\pgfqpoint{11.131657in}{1.139989in}}%
\pgfpathlineto{\pgfqpoint{11.242343in}{1.141077in}}%
\pgfpathlineto{\pgfqpoint{11.353029in}{1.151930in}}%
\pgfpathlineto{\pgfqpoint{11.463715in}{1.141053in}}%
\pgfpathlineto{\pgfqpoint{11.574401in}{1.141816in}}%
\pgfpathlineto{\pgfqpoint{11.685087in}{1.145432in}}%
\pgfpathlineto{\pgfqpoint{11.795773in}{1.143190in}}%
\pgfpathlineto{\pgfqpoint{11.906459in}{1.143496in}}%
\pgfpathlineto{\pgfqpoint{12.017145in}{1.145773in}}%
\pgfpathlineto{\pgfqpoint{12.127831in}{1.145317in}}%
\pgfpathlineto{\pgfqpoint{12.238517in}{1.145600in}}%
\pgfpathlineto{\pgfqpoint{12.349203in}{1.145284in}}%
\pgfpathlineto{\pgfqpoint{12.459890in}{1.146445in}}%
\pgfpathlineto{\pgfqpoint{12.570576in}{1.146828in}}%
\pgfpathlineto{\pgfqpoint{12.681262in}{1.148292in}}%
\pgfpathlineto{\pgfqpoint{12.791948in}{1.149163in}}%
\pgfusepath{stroke}%
\end{pgfscope}%
\begin{pgfscope}%
\pgfpathrectangle{\pgfqpoint{1.286132in}{0.839159in}}{\pgfqpoint{12.053712in}{5.967710in}}%
\pgfusepath{clip}%
\pgfsetrectcap%
\pgfsetroundjoin%
\pgfsetlinewidth{1.505625pt}%
\definecolor{currentstroke}{rgb}{0.839216,0.152941,0.156863}%
\pgfsetstrokecolor{currentstroke}%
\pgfsetdash{}{0pt}%
\pgfpathmoveto{\pgfqpoint{1.834028in}{1.119261in}}%
\pgfpathlineto{\pgfqpoint{1.944714in}{1.119612in}}%
\pgfpathlineto{\pgfqpoint{2.055400in}{1.119931in}}%
\pgfpathlineto{\pgfqpoint{2.166086in}{1.120473in}}%
\pgfpathlineto{\pgfqpoint{2.276772in}{1.120891in}}%
\pgfpathlineto{\pgfqpoint{2.387458in}{1.121603in}}%
\pgfpathlineto{\pgfqpoint{2.498144in}{1.122538in}}%
\pgfpathlineto{\pgfqpoint{2.608830in}{1.123491in}}%
\pgfpathlineto{\pgfqpoint{2.719516in}{1.124620in}}%
\pgfpathlineto{\pgfqpoint{2.830202in}{1.126423in}}%
\pgfpathlineto{\pgfqpoint{2.940888in}{1.136398in}}%
\pgfpathlineto{\pgfqpoint{3.051574in}{1.129629in}}%
\pgfpathlineto{\pgfqpoint{3.162260in}{1.139428in}}%
\pgfpathlineto{\pgfqpoint{3.272946in}{1.134349in}}%
\pgfpathlineto{\pgfqpoint{3.383632in}{1.140103in}}%
\pgfpathlineto{\pgfqpoint{3.494319in}{1.140095in}}%
\pgfpathlineto{\pgfqpoint{3.605005in}{1.146042in}}%
\pgfpathlineto{\pgfqpoint{3.715691in}{1.148956in}}%
\pgfpathlineto{\pgfqpoint{3.826377in}{1.154614in}}%
\pgfpathlineto{\pgfqpoint{3.937063in}{1.157789in}}%
\pgfpathlineto{\pgfqpoint{4.047749in}{1.167611in}}%
\pgfpathlineto{\pgfqpoint{4.158435in}{1.171265in}}%
\pgfpathlineto{\pgfqpoint{4.269121in}{1.174679in}}%
\pgfpathlineto{\pgfqpoint{4.379807in}{1.192990in}}%
\pgfpathlineto{\pgfqpoint{4.490493in}{1.196633in}}%
\pgfpathlineto{\pgfqpoint{4.601179in}{1.202147in}}%
\pgfpathlineto{\pgfqpoint{4.711865in}{1.212125in}}%
\pgfpathlineto{\pgfqpoint{4.822551in}{1.220618in}}%
\pgfpathlineto{\pgfqpoint{4.933237in}{1.232965in}}%
\pgfpathlineto{\pgfqpoint{5.043923in}{1.239531in}}%
\pgfpathlineto{\pgfqpoint{5.154609in}{1.251256in}}%
\pgfpathlineto{\pgfqpoint{5.265296in}{1.272316in}}%
\pgfpathlineto{\pgfqpoint{5.375982in}{1.282854in}}%
\pgfpathlineto{\pgfqpoint{5.486668in}{1.377328in}}%
\pgfpathlineto{\pgfqpoint{5.597354in}{1.369025in}}%
\pgfpathlineto{\pgfqpoint{5.708040in}{1.549296in}}%
\pgfpathlineto{\pgfqpoint{5.818726in}{1.384781in}}%
\pgfpathlineto{\pgfqpoint{5.929412in}{1.374021in}}%
\pgfpathlineto{\pgfqpoint{6.040098in}{1.377196in}}%
\pgfpathlineto{\pgfqpoint{6.150784in}{1.391142in}}%
\pgfpathlineto{\pgfqpoint{6.261470in}{1.430244in}}%
\pgfpathlineto{\pgfqpoint{6.372156in}{1.422405in}}%
\pgfpathlineto{\pgfqpoint{6.482842in}{1.444028in}}%
\pgfpathlineto{\pgfqpoint{6.593528in}{1.462573in}}%
\pgfpathlineto{\pgfqpoint{6.704214in}{1.491706in}}%
\pgfpathlineto{\pgfqpoint{6.814900in}{1.503187in}}%
\pgfpathlineto{\pgfqpoint{6.925586in}{1.529828in}}%
\pgfpathlineto{\pgfqpoint{7.036272in}{1.557589in}}%
\pgfpathlineto{\pgfqpoint{7.146959in}{1.585319in}}%
\pgfpathlineto{\pgfqpoint{7.257645in}{1.611856in}}%
\pgfpathlineto{\pgfqpoint{7.368331in}{1.641659in}}%
\pgfpathlineto{\pgfqpoint{7.479017in}{1.658897in}}%
\pgfpathlineto{\pgfqpoint{7.589703in}{1.765570in}}%
\pgfpathlineto{\pgfqpoint{7.700389in}{1.752210in}}%
\pgfpathlineto{\pgfqpoint{7.811075in}{1.757415in}}%
\pgfpathlineto{\pgfqpoint{7.921761in}{1.792071in}}%
\pgfpathlineto{\pgfqpoint{8.032447in}{1.834586in}}%
\pgfpathlineto{\pgfqpoint{8.143133in}{1.871249in}}%
\pgfpathlineto{\pgfqpoint{8.253819in}{1.911423in}}%
\pgfpathlineto{\pgfqpoint{8.364505in}{1.935753in}}%
\pgfpathlineto{\pgfqpoint{8.475191in}{1.968650in}}%
\pgfpathlineto{\pgfqpoint{8.585877in}{2.017895in}}%
\pgfpathlineto{\pgfqpoint{8.696563in}{2.120257in}}%
\pgfpathlineto{\pgfqpoint{8.807249in}{2.101392in}}%
\pgfpathlineto{\pgfqpoint{8.917936in}{2.158177in}}%
\pgfpathlineto{\pgfqpoint{9.028622in}{2.200467in}}%
\pgfpathlineto{\pgfqpoint{9.139308in}{2.237090in}}%
\pgfpathlineto{\pgfqpoint{9.249994in}{2.290845in}}%
\pgfpathlineto{\pgfqpoint{9.360680in}{2.324517in}}%
\pgfpathlineto{\pgfqpoint{9.471366in}{2.520116in}}%
\pgfpathlineto{\pgfqpoint{9.582052in}{2.464251in}}%
\pgfpathlineto{\pgfqpoint{9.692738in}{2.483894in}}%
\pgfpathlineto{\pgfqpoint{9.803424in}{2.555419in}}%
\pgfpathlineto{\pgfqpoint{9.914110in}{2.630109in}}%
\pgfpathlineto{\pgfqpoint{10.024796in}{2.804450in}}%
\pgfpathlineto{\pgfqpoint{10.135482in}{2.749395in}}%
\pgfpathlineto{\pgfqpoint{10.246168in}{2.778765in}}%
\pgfpathlineto{\pgfqpoint{10.356854in}{2.887670in}}%
\pgfpathlineto{\pgfqpoint{10.467540in}{2.978360in}}%
\pgfpathlineto{\pgfqpoint{10.578226in}{3.004230in}}%
\pgfpathlineto{\pgfqpoint{10.688913in}{3.079061in}}%
\pgfpathlineto{\pgfqpoint{10.799599in}{3.137375in}}%
\pgfpathlineto{\pgfqpoint{10.910285in}{3.286635in}}%
\pgfpathlineto{\pgfqpoint{11.020971in}{3.313805in}}%
\pgfpathlineto{\pgfqpoint{11.131657in}{3.363434in}}%
\pgfpathlineto{\pgfqpoint{11.242343in}{3.465947in}}%
\pgfpathlineto{\pgfqpoint{11.353029in}{3.585475in}}%
\pgfpathlineto{\pgfqpoint{11.463715in}{3.588849in}}%
\pgfpathlineto{\pgfqpoint{11.574401in}{3.749749in}}%
\pgfpathlineto{\pgfqpoint{11.685087in}{3.878791in}}%
\pgfpathlineto{\pgfqpoint{11.795773in}{3.968433in}}%
\pgfpathlineto{\pgfqpoint{11.906459in}{3.982039in}}%
\pgfpathlineto{\pgfqpoint{12.017145in}{4.161532in}}%
\pgfpathlineto{\pgfqpoint{12.127831in}{4.179587in}}%
\pgfpathlineto{\pgfqpoint{12.238517in}{4.333542in}}%
\pgfpathlineto{\pgfqpoint{12.349203in}{4.456108in}}%
\pgfpathlineto{\pgfqpoint{12.459890in}{4.542699in}}%
\pgfpathlineto{\pgfqpoint{12.570576in}{4.601491in}}%
\pgfpathlineto{\pgfqpoint{12.681262in}{4.644594in}}%
\pgfpathlineto{\pgfqpoint{12.791948in}{4.872869in}}%
\pgfusepath{stroke}%
\end{pgfscope}%
\begin{pgfscope}%
\pgfpathrectangle{\pgfqpoint{1.286132in}{0.839159in}}{\pgfqpoint{12.053712in}{5.967710in}}%
\pgfusepath{clip}%
\pgfsetrectcap%
\pgfsetroundjoin%
\pgfsetlinewidth{1.505625pt}%
\definecolor{currentstroke}{rgb}{0.580392,0.403922,0.741176}%
\pgfsetstrokecolor{currentstroke}%
\pgfsetdash{}{0pt}%
\pgfpathmoveto{\pgfqpoint{1.834028in}{1.119288in}}%
\pgfpathlineto{\pgfqpoint{1.944714in}{1.119257in}}%
\pgfpathlineto{\pgfqpoint{2.055400in}{1.119326in}}%
\pgfpathlineto{\pgfqpoint{2.166086in}{1.119388in}}%
\pgfpathlineto{\pgfqpoint{2.276772in}{1.119570in}}%
\pgfpathlineto{\pgfqpoint{2.387458in}{1.119567in}}%
\pgfpathlineto{\pgfqpoint{2.498144in}{1.119627in}}%
\pgfpathlineto{\pgfqpoint{2.608830in}{1.119734in}}%
\pgfpathlineto{\pgfqpoint{2.719516in}{1.119779in}}%
\pgfpathlineto{\pgfqpoint{2.830202in}{1.119936in}}%
\pgfpathlineto{\pgfqpoint{2.940888in}{1.119905in}}%
\pgfpathlineto{\pgfqpoint{3.051574in}{1.120409in}}%
\pgfpathlineto{\pgfqpoint{3.162260in}{1.120094in}}%
\pgfpathlineto{\pgfqpoint{3.272946in}{1.120308in}}%
\pgfpathlineto{\pgfqpoint{3.383632in}{1.120455in}}%
\pgfpathlineto{\pgfqpoint{3.494319in}{1.120309in}}%
\pgfpathlineto{\pgfqpoint{3.605005in}{1.120318in}}%
\pgfpathlineto{\pgfqpoint{3.715691in}{1.120642in}}%
\pgfpathlineto{\pgfqpoint{3.826377in}{1.120930in}}%
\pgfpathlineto{\pgfqpoint{3.937063in}{1.120661in}}%
\pgfpathlineto{\pgfqpoint{4.047749in}{1.129374in}}%
\pgfpathlineto{\pgfqpoint{4.158435in}{1.121200in}}%
\pgfpathlineto{\pgfqpoint{4.269121in}{1.121334in}}%
\pgfpathlineto{\pgfqpoint{4.379807in}{1.121220in}}%
\pgfpathlineto{\pgfqpoint{4.490493in}{1.121334in}}%
\pgfpathlineto{\pgfqpoint{4.601179in}{1.121151in}}%
\pgfpathlineto{\pgfqpoint{4.711865in}{1.121367in}}%
\pgfpathlineto{\pgfqpoint{4.822551in}{1.121677in}}%
\pgfpathlineto{\pgfqpoint{4.933237in}{1.122013in}}%
\pgfpathlineto{\pgfqpoint{5.043923in}{1.121831in}}%
\pgfpathlineto{\pgfqpoint{5.154609in}{1.121943in}}%
\pgfpathlineto{\pgfqpoint{5.265296in}{1.121834in}}%
\pgfpathlineto{\pgfqpoint{5.375982in}{1.122614in}}%
\pgfpathlineto{\pgfqpoint{5.486668in}{1.122185in}}%
\pgfpathlineto{\pgfqpoint{5.597354in}{1.122388in}}%
\pgfpathlineto{\pgfqpoint{5.708040in}{1.122645in}}%
\pgfpathlineto{\pgfqpoint{5.818726in}{1.122098in}}%
\pgfpathlineto{\pgfqpoint{5.929412in}{1.123851in}}%
\pgfpathlineto{\pgfqpoint{6.040098in}{1.122501in}}%
\pgfpathlineto{\pgfqpoint{6.150784in}{1.122139in}}%
\pgfpathlineto{\pgfqpoint{6.261470in}{1.122092in}}%
\pgfpathlineto{\pgfqpoint{6.372156in}{1.123607in}}%
\pgfpathlineto{\pgfqpoint{6.482842in}{1.123529in}}%
\pgfpathlineto{\pgfqpoint{6.593528in}{1.122991in}}%
\pgfpathlineto{\pgfqpoint{6.704214in}{1.122623in}}%
\pgfpathlineto{\pgfqpoint{6.814900in}{1.123594in}}%
\pgfpathlineto{\pgfqpoint{6.925586in}{1.124387in}}%
\pgfpathlineto{\pgfqpoint{7.036272in}{1.123031in}}%
\pgfpathlineto{\pgfqpoint{7.146959in}{1.123992in}}%
\pgfpathlineto{\pgfqpoint{7.257645in}{1.124111in}}%
\pgfpathlineto{\pgfqpoint{7.368331in}{1.125148in}}%
\pgfpathlineto{\pgfqpoint{7.479017in}{1.124222in}}%
\pgfpathlineto{\pgfqpoint{7.589703in}{1.124065in}}%
\pgfpathlineto{\pgfqpoint{7.700389in}{1.124076in}}%
\pgfpathlineto{\pgfqpoint{7.811075in}{1.124660in}}%
\pgfpathlineto{\pgfqpoint{7.921761in}{1.124556in}}%
\pgfpathlineto{\pgfqpoint{8.032447in}{1.124572in}}%
\pgfpathlineto{\pgfqpoint{8.143133in}{1.124962in}}%
\pgfpathlineto{\pgfqpoint{8.253819in}{1.124438in}}%
\pgfpathlineto{\pgfqpoint{8.364505in}{1.126144in}}%
\pgfpathlineto{\pgfqpoint{8.475191in}{1.125740in}}%
\pgfpathlineto{\pgfqpoint{8.585877in}{1.125726in}}%
\pgfpathlineto{\pgfqpoint{8.696563in}{1.127137in}}%
\pgfpathlineto{\pgfqpoint{8.807249in}{1.125601in}}%
\pgfpathlineto{\pgfqpoint{8.917936in}{1.126668in}}%
\pgfpathlineto{\pgfqpoint{9.028622in}{1.126884in}}%
\pgfpathlineto{\pgfqpoint{9.139308in}{1.124768in}}%
\pgfpathlineto{\pgfqpoint{9.249994in}{1.126242in}}%
\pgfpathlineto{\pgfqpoint{9.360680in}{1.128611in}}%
\pgfpathlineto{\pgfqpoint{9.471366in}{1.126507in}}%
\pgfpathlineto{\pgfqpoint{9.582052in}{1.125425in}}%
\pgfpathlineto{\pgfqpoint{9.692738in}{1.126707in}}%
\pgfpathlineto{\pgfqpoint{9.803424in}{1.129219in}}%
\pgfpathlineto{\pgfqpoint{9.914110in}{1.126142in}}%
\pgfpathlineto{\pgfqpoint{10.024796in}{1.127379in}}%
\pgfpathlineto{\pgfqpoint{10.135482in}{1.127464in}}%
\pgfpathlineto{\pgfqpoint{10.246168in}{1.126614in}}%
\pgfpathlineto{\pgfqpoint{10.356854in}{1.127307in}}%
\pgfpathlineto{\pgfqpoint{10.467540in}{1.126805in}}%
\pgfpathlineto{\pgfqpoint{10.578226in}{1.128110in}}%
\pgfpathlineto{\pgfqpoint{10.688913in}{1.128564in}}%
\pgfpathlineto{\pgfqpoint{10.799599in}{1.127861in}}%
\pgfpathlineto{\pgfqpoint{10.910285in}{1.128867in}}%
\pgfpathlineto{\pgfqpoint{11.020971in}{1.128515in}}%
\pgfpathlineto{\pgfqpoint{11.131657in}{1.128348in}}%
\pgfpathlineto{\pgfqpoint{11.242343in}{1.132701in}}%
\pgfpathlineto{\pgfqpoint{11.353029in}{1.128634in}}%
\pgfpathlineto{\pgfqpoint{11.463715in}{1.127599in}}%
\pgfpathlineto{\pgfqpoint{11.574401in}{1.129531in}}%
\pgfpathlineto{\pgfqpoint{11.685087in}{1.129747in}}%
\pgfpathlineto{\pgfqpoint{11.795773in}{1.131596in}}%
\pgfpathlineto{\pgfqpoint{11.906459in}{1.130442in}}%
\pgfpathlineto{\pgfqpoint{12.017145in}{1.130063in}}%
\pgfpathlineto{\pgfqpoint{12.127831in}{1.145852in}}%
\pgfpathlineto{\pgfqpoint{12.238517in}{1.129361in}}%
\pgfpathlineto{\pgfqpoint{12.349203in}{1.131006in}}%
\pgfpathlineto{\pgfqpoint{12.459890in}{1.133356in}}%
\pgfpathlineto{\pgfqpoint{12.570576in}{1.129680in}}%
\pgfpathlineto{\pgfqpoint{12.681262in}{1.129634in}}%
\pgfpathlineto{\pgfqpoint{12.791948in}{1.130317in}}%
\pgfusepath{stroke}%
\end{pgfscope}%
\begin{pgfscope}%
\pgfpathrectangle{\pgfqpoint{1.286132in}{0.839159in}}{\pgfqpoint{12.053712in}{5.967710in}}%
\pgfusepath{clip}%
\pgfsetrectcap%
\pgfsetroundjoin%
\pgfsetlinewidth{1.505625pt}%
\definecolor{currentstroke}{rgb}{0.549020,0.337255,0.294118}%
\pgfsetstrokecolor{currentstroke}%
\pgfsetdash{}{0pt}%
\pgfpathmoveto{\pgfqpoint{1.834028in}{1.119216in}}%
\pgfpathlineto{\pgfqpoint{1.944714in}{1.119225in}}%
\pgfpathlineto{\pgfqpoint{2.055400in}{1.119263in}}%
\pgfpathlineto{\pgfqpoint{2.166086in}{1.119358in}}%
\pgfpathlineto{\pgfqpoint{2.276772in}{1.119602in}}%
\pgfpathlineto{\pgfqpoint{2.387458in}{1.119735in}}%
\pgfpathlineto{\pgfqpoint{2.498144in}{1.120334in}}%
\pgfpathlineto{\pgfqpoint{2.608830in}{1.122468in}}%
\pgfpathlineto{\pgfqpoint{2.719516in}{1.120974in}}%
\pgfpathlineto{\pgfqpoint{2.830202in}{1.122122in}}%
\pgfpathlineto{\pgfqpoint{2.940888in}{1.122425in}}%
\pgfpathlineto{\pgfqpoint{3.051574in}{1.122881in}}%
\pgfpathlineto{\pgfqpoint{3.162260in}{1.123788in}}%
\pgfpathlineto{\pgfqpoint{3.272946in}{1.124272in}}%
\pgfpathlineto{\pgfqpoint{3.383632in}{1.126586in}}%
\pgfpathlineto{\pgfqpoint{3.494319in}{1.126694in}}%
\pgfpathlineto{\pgfqpoint{3.605005in}{1.129559in}}%
\pgfpathlineto{\pgfqpoint{3.715691in}{1.128904in}}%
\pgfpathlineto{\pgfqpoint{3.826377in}{1.132616in}}%
\pgfpathlineto{\pgfqpoint{3.937063in}{1.133371in}}%
\pgfpathlineto{\pgfqpoint{4.047749in}{1.136403in}}%
\pgfpathlineto{\pgfqpoint{4.158435in}{1.134665in}}%
\pgfpathlineto{\pgfqpoint{4.269121in}{1.138840in}}%
\pgfpathlineto{\pgfqpoint{4.379807in}{1.139962in}}%
\pgfpathlineto{\pgfqpoint{4.490493in}{1.141751in}}%
\pgfpathlineto{\pgfqpoint{4.601179in}{1.145043in}}%
\pgfpathlineto{\pgfqpoint{4.711865in}{1.145496in}}%
\pgfpathlineto{\pgfqpoint{4.822551in}{1.152907in}}%
\pgfpathlineto{\pgfqpoint{4.933237in}{1.150826in}}%
\pgfpathlineto{\pgfqpoint{5.043923in}{1.149854in}}%
\pgfpathlineto{\pgfqpoint{5.154609in}{1.155063in}}%
\pgfpathlineto{\pgfqpoint{5.265296in}{1.166470in}}%
\pgfpathlineto{\pgfqpoint{5.375982in}{1.164659in}}%
\pgfpathlineto{\pgfqpoint{5.486668in}{1.173754in}}%
\pgfpathlineto{\pgfqpoint{5.597354in}{1.160250in}}%
\pgfpathlineto{\pgfqpoint{5.708040in}{1.198326in}}%
\pgfpathlineto{\pgfqpoint{5.818726in}{1.213801in}}%
\pgfpathlineto{\pgfqpoint{5.929412in}{1.240834in}}%
\pgfpathlineto{\pgfqpoint{6.040098in}{1.193975in}}%
\pgfpathlineto{\pgfqpoint{6.150784in}{1.197854in}}%
\pgfpathlineto{\pgfqpoint{6.261470in}{1.199605in}}%
\pgfpathlineto{\pgfqpoint{6.372156in}{1.216938in}}%
\pgfpathlineto{\pgfqpoint{6.482842in}{1.211481in}}%
\pgfpathlineto{\pgfqpoint{6.593528in}{1.228115in}}%
\pgfpathlineto{\pgfqpoint{6.704214in}{1.214266in}}%
\pgfpathlineto{\pgfqpoint{6.814900in}{1.225681in}}%
\pgfpathlineto{\pgfqpoint{6.925586in}{1.228362in}}%
\pgfpathlineto{\pgfqpoint{7.036272in}{1.240546in}}%
\pgfpathlineto{\pgfqpoint{7.146959in}{1.250655in}}%
\pgfpathlineto{\pgfqpoint{7.257645in}{1.254552in}}%
\pgfpathlineto{\pgfqpoint{7.368331in}{1.246359in}}%
\pgfpathlineto{\pgfqpoint{7.479017in}{1.255416in}}%
\pgfpathlineto{\pgfqpoint{7.589703in}{1.318261in}}%
\pgfpathlineto{\pgfqpoint{7.700389in}{1.262488in}}%
\pgfpathlineto{\pgfqpoint{7.811075in}{1.284536in}}%
\pgfpathlineto{\pgfqpoint{7.921761in}{1.283991in}}%
\pgfpathlineto{\pgfqpoint{8.032447in}{1.279005in}}%
\pgfpathlineto{\pgfqpoint{8.143133in}{1.297207in}}%
\pgfpathlineto{\pgfqpoint{8.253819in}{1.286371in}}%
\pgfpathlineto{\pgfqpoint{8.364505in}{1.318258in}}%
\pgfpathlineto{\pgfqpoint{8.475191in}{1.317717in}}%
\pgfpathlineto{\pgfqpoint{8.585877in}{1.353869in}}%
\pgfpathlineto{\pgfqpoint{8.696563in}{1.386690in}}%
\pgfpathlineto{\pgfqpoint{8.807249in}{1.388802in}}%
\pgfpathlineto{\pgfqpoint{8.917936in}{1.346466in}}%
\pgfpathlineto{\pgfqpoint{9.028622in}{1.391274in}}%
\pgfpathlineto{\pgfqpoint{9.139308in}{1.378841in}}%
\pgfpathlineto{\pgfqpoint{9.249994in}{1.376316in}}%
\pgfpathlineto{\pgfqpoint{9.360680in}{1.423266in}}%
\pgfpathlineto{\pgfqpoint{9.471366in}{1.433946in}}%
\pgfpathlineto{\pgfqpoint{9.582052in}{1.407114in}}%
\pgfpathlineto{\pgfqpoint{9.692738in}{1.390779in}}%
\pgfpathlineto{\pgfqpoint{9.803424in}{1.455554in}}%
\pgfpathlineto{\pgfqpoint{9.914110in}{1.527848in}}%
\pgfpathlineto{\pgfqpoint{10.024796in}{1.484942in}}%
\pgfpathlineto{\pgfqpoint{10.135482in}{1.475568in}}%
\pgfpathlineto{\pgfqpoint{10.246168in}{1.518180in}}%
\pgfpathlineto{\pgfqpoint{10.356854in}{1.483995in}}%
\pgfpathlineto{\pgfqpoint{10.467540in}{1.513502in}}%
\pgfpathlineto{\pgfqpoint{10.578226in}{1.534070in}}%
\pgfpathlineto{\pgfqpoint{10.688913in}{1.482075in}}%
\pgfpathlineto{\pgfqpoint{10.799599in}{1.517193in}}%
\pgfpathlineto{\pgfqpoint{10.910285in}{1.626930in}}%
\pgfpathlineto{\pgfqpoint{11.020971in}{1.516491in}}%
\pgfpathlineto{\pgfqpoint{11.131657in}{1.583421in}}%
\pgfpathlineto{\pgfqpoint{11.242343in}{1.576504in}}%
\pgfpathlineto{\pgfqpoint{11.353029in}{1.670180in}}%
\pgfpathlineto{\pgfqpoint{11.463715in}{1.590066in}}%
\pgfpathlineto{\pgfqpoint{11.574401in}{1.650283in}}%
\pgfpathlineto{\pgfqpoint{11.685087in}{1.646720in}}%
\pgfpathlineto{\pgfqpoint{11.795773in}{1.667512in}}%
\pgfpathlineto{\pgfqpoint{11.906459in}{1.698975in}}%
\pgfpathlineto{\pgfqpoint{12.017145in}{1.675086in}}%
\pgfpathlineto{\pgfqpoint{12.127831in}{1.801558in}}%
\pgfpathlineto{\pgfqpoint{12.238517in}{1.762547in}}%
\pgfpathlineto{\pgfqpoint{12.349203in}{1.855978in}}%
\pgfpathlineto{\pgfqpoint{12.459890in}{1.859868in}}%
\pgfpathlineto{\pgfqpoint{12.570576in}{2.000065in}}%
\pgfpathlineto{\pgfqpoint{12.681262in}{1.900086in}}%
\pgfpathlineto{\pgfqpoint{12.791948in}{1.826738in}}%
\pgfusepath{stroke}%
\end{pgfscope}%
\begin{pgfscope}%
\pgfpathrectangle{\pgfqpoint{1.286132in}{0.839159in}}{\pgfqpoint{12.053712in}{5.967710in}}%
\pgfusepath{clip}%
\pgfsetrectcap%
\pgfsetroundjoin%
\pgfsetlinewidth{1.505625pt}%
\definecolor{currentstroke}{rgb}{0.890196,0.466667,0.760784}%
\pgfsetstrokecolor{currentstroke}%
\pgfsetdash{}{0pt}%
\pgfpathmoveto{\pgfqpoint{1.834028in}{1.119245in}}%
\pgfpathlineto{\pgfqpoint{1.944714in}{1.119348in}}%
\pgfpathlineto{\pgfqpoint{2.055400in}{1.119404in}}%
\pgfpathlineto{\pgfqpoint{2.166086in}{1.119451in}}%
\pgfpathlineto{\pgfqpoint{2.276772in}{1.119527in}}%
\pgfpathlineto{\pgfqpoint{2.387458in}{1.119588in}}%
\pgfpathlineto{\pgfqpoint{2.498144in}{1.119737in}}%
\pgfpathlineto{\pgfqpoint{2.608830in}{1.119743in}}%
\pgfpathlineto{\pgfqpoint{2.719516in}{1.119846in}}%
\pgfpathlineto{\pgfqpoint{2.830202in}{1.120000in}}%
\pgfpathlineto{\pgfqpoint{2.940888in}{1.119980in}}%
\pgfpathlineto{\pgfqpoint{3.051574in}{1.120130in}}%
\pgfpathlineto{\pgfqpoint{3.162260in}{1.120334in}}%
\pgfpathlineto{\pgfqpoint{3.272946in}{1.120506in}}%
\pgfpathlineto{\pgfqpoint{3.383632in}{1.120418in}}%
\pgfpathlineto{\pgfqpoint{3.494319in}{1.120578in}}%
\pgfpathlineto{\pgfqpoint{3.605005in}{1.120929in}}%
\pgfpathlineto{\pgfqpoint{3.715691in}{1.120779in}}%
\pgfpathlineto{\pgfqpoint{3.826377in}{1.121256in}}%
\pgfpathlineto{\pgfqpoint{3.937063in}{1.121186in}}%
\pgfpathlineto{\pgfqpoint{4.047749in}{1.121676in}}%
\pgfpathlineto{\pgfqpoint{4.158435in}{1.121501in}}%
\pgfpathlineto{\pgfqpoint{4.269121in}{1.121474in}}%
\pgfpathlineto{\pgfqpoint{4.379807in}{1.121925in}}%
\pgfpathlineto{\pgfqpoint{4.490493in}{1.121955in}}%
\pgfpathlineto{\pgfqpoint{4.601179in}{1.122046in}}%
\pgfpathlineto{\pgfqpoint{4.711865in}{1.122708in}}%
\pgfpathlineto{\pgfqpoint{4.822551in}{1.122611in}}%
\pgfpathlineto{\pgfqpoint{4.933237in}{1.122518in}}%
\pgfpathlineto{\pgfqpoint{5.043923in}{1.122530in}}%
\pgfpathlineto{\pgfqpoint{5.154609in}{1.123575in}}%
\pgfpathlineto{\pgfqpoint{5.265296in}{1.122926in}}%
\pgfpathlineto{\pgfqpoint{5.375982in}{1.123152in}}%
\pgfpathlineto{\pgfqpoint{5.486668in}{1.123423in}}%
\pgfpathlineto{\pgfqpoint{5.597354in}{1.123982in}}%
\pgfpathlineto{\pgfqpoint{5.708040in}{1.132979in}}%
\pgfpathlineto{\pgfqpoint{5.818726in}{1.123964in}}%
\pgfpathlineto{\pgfqpoint{5.929412in}{1.124744in}}%
\pgfpathlineto{\pgfqpoint{6.040098in}{1.124601in}}%
\pgfpathlineto{\pgfqpoint{6.150784in}{1.125029in}}%
\pgfpathlineto{\pgfqpoint{6.261470in}{1.125146in}}%
\pgfpathlineto{\pgfqpoint{6.372156in}{1.125726in}}%
\pgfpathlineto{\pgfqpoint{6.482842in}{1.125285in}}%
\pgfpathlineto{\pgfqpoint{6.593528in}{1.125500in}}%
\pgfpathlineto{\pgfqpoint{6.704214in}{1.125848in}}%
\pgfpathlineto{\pgfqpoint{6.814900in}{1.126638in}}%
\pgfpathlineto{\pgfqpoint{6.925586in}{1.126644in}}%
\pgfpathlineto{\pgfqpoint{7.036272in}{1.127109in}}%
\pgfpathlineto{\pgfqpoint{7.146959in}{1.126861in}}%
\pgfpathlineto{\pgfqpoint{7.257645in}{1.127120in}}%
\pgfpathlineto{\pgfqpoint{7.368331in}{1.128863in}}%
\pgfpathlineto{\pgfqpoint{7.479017in}{1.128360in}}%
\pgfpathlineto{\pgfqpoint{7.589703in}{1.137391in}}%
\pgfpathlineto{\pgfqpoint{7.700389in}{1.129072in}}%
\pgfpathlineto{\pgfqpoint{7.811075in}{1.128767in}}%
\pgfpathlineto{\pgfqpoint{7.921761in}{1.129304in}}%
\pgfpathlineto{\pgfqpoint{8.032447in}{1.129099in}}%
\pgfpathlineto{\pgfqpoint{8.143133in}{1.129296in}}%
\pgfpathlineto{\pgfqpoint{8.253819in}{1.129411in}}%
\pgfpathlineto{\pgfqpoint{8.364505in}{1.130119in}}%
\pgfpathlineto{\pgfqpoint{8.475191in}{1.130089in}}%
\pgfpathlineto{\pgfqpoint{8.585877in}{1.130757in}}%
\pgfpathlineto{\pgfqpoint{8.696563in}{1.131214in}}%
\pgfpathlineto{\pgfqpoint{8.807249in}{1.131483in}}%
\pgfpathlineto{\pgfqpoint{8.917936in}{1.131662in}}%
\pgfpathlineto{\pgfqpoint{9.028622in}{1.132916in}}%
\pgfpathlineto{\pgfqpoint{9.139308in}{1.132887in}}%
\pgfpathlineto{\pgfqpoint{9.249994in}{1.132768in}}%
\pgfpathlineto{\pgfqpoint{9.360680in}{1.133188in}}%
\pgfpathlineto{\pgfqpoint{9.471366in}{1.133572in}}%
\pgfpathlineto{\pgfqpoint{9.582052in}{1.134630in}}%
\pgfpathlineto{\pgfqpoint{9.692738in}{1.134707in}}%
\pgfpathlineto{\pgfqpoint{9.803424in}{1.134473in}}%
\pgfpathlineto{\pgfqpoint{9.914110in}{1.148095in}}%
\pgfpathlineto{\pgfqpoint{10.024796in}{1.135877in}}%
\pgfpathlineto{\pgfqpoint{10.135482in}{1.136594in}}%
\pgfpathlineto{\pgfqpoint{10.246168in}{1.136682in}}%
\pgfpathlineto{\pgfqpoint{10.356854in}{1.137199in}}%
\pgfpathlineto{\pgfqpoint{10.467540in}{1.143908in}}%
\pgfpathlineto{\pgfqpoint{10.578226in}{1.138048in}}%
\pgfpathlineto{\pgfqpoint{10.688913in}{1.137886in}}%
\pgfpathlineto{\pgfqpoint{10.799599in}{1.138705in}}%
\pgfpathlineto{\pgfqpoint{10.910285in}{1.139429in}}%
\pgfpathlineto{\pgfqpoint{11.020971in}{1.140916in}}%
\pgfpathlineto{\pgfqpoint{11.131657in}{1.139372in}}%
\pgfpathlineto{\pgfqpoint{11.242343in}{1.139969in}}%
\pgfpathlineto{\pgfqpoint{11.353029in}{1.140833in}}%
\pgfpathlineto{\pgfqpoint{11.463715in}{1.140748in}}%
\pgfpathlineto{\pgfqpoint{11.574401in}{1.141348in}}%
\pgfpathlineto{\pgfqpoint{11.685087in}{1.141752in}}%
\pgfpathlineto{\pgfqpoint{11.795773in}{1.142127in}}%
\pgfpathlineto{\pgfqpoint{11.906459in}{1.143580in}}%
\pgfpathlineto{\pgfqpoint{12.017145in}{1.144985in}}%
\pgfpathlineto{\pgfqpoint{12.127831in}{1.150877in}}%
\pgfpathlineto{\pgfqpoint{12.238517in}{1.146201in}}%
\pgfpathlineto{\pgfqpoint{12.349203in}{1.146680in}}%
\pgfpathlineto{\pgfqpoint{12.459890in}{1.148317in}}%
\pgfpathlineto{\pgfqpoint{12.570576in}{1.178073in}}%
\pgfpathlineto{\pgfqpoint{12.681262in}{1.146925in}}%
\pgfpathlineto{\pgfqpoint{12.791948in}{1.147399in}}%
\pgfusepath{stroke}%
\end{pgfscope}%
\begin{pgfscope}%
\pgfpathrectangle{\pgfqpoint{1.286132in}{0.839159in}}{\pgfqpoint{12.053712in}{5.967710in}}%
\pgfusepath{clip}%
\pgfsetrectcap%
\pgfsetroundjoin%
\pgfsetlinewidth{1.505625pt}%
\definecolor{currentstroke}{rgb}{0.498039,0.498039,0.498039}%
\pgfsetstrokecolor{currentstroke}%
\pgfsetdash{}{0pt}%
\pgfpathmoveto{\pgfqpoint{1.834028in}{1.119265in}}%
\pgfpathlineto{\pgfqpoint{1.944714in}{1.119631in}}%
\pgfpathlineto{\pgfqpoint{2.055400in}{1.119934in}}%
\pgfpathlineto{\pgfqpoint{2.166086in}{1.120427in}}%
\pgfpathlineto{\pgfqpoint{2.276772in}{1.120935in}}%
\pgfpathlineto{\pgfqpoint{2.387458in}{1.123481in}}%
\pgfpathlineto{\pgfqpoint{2.498144in}{1.122838in}}%
\pgfpathlineto{\pgfqpoint{2.608830in}{1.123409in}}%
\pgfpathlineto{\pgfqpoint{2.719516in}{1.124692in}}%
\pgfpathlineto{\pgfqpoint{2.830202in}{1.126734in}}%
\pgfpathlineto{\pgfqpoint{2.940888in}{1.127955in}}%
\pgfpathlineto{\pgfqpoint{3.051574in}{1.129411in}}%
\pgfpathlineto{\pgfqpoint{3.162260in}{1.131892in}}%
\pgfpathlineto{\pgfqpoint{3.272946in}{1.133785in}}%
\pgfpathlineto{\pgfqpoint{3.383632in}{1.137405in}}%
\pgfpathlineto{\pgfqpoint{3.494319in}{1.140649in}}%
\pgfpathlineto{\pgfqpoint{3.605005in}{1.145691in}}%
\pgfpathlineto{\pgfqpoint{3.715691in}{1.148309in}}%
\pgfpathlineto{\pgfqpoint{3.826377in}{1.152384in}}%
\pgfpathlineto{\pgfqpoint{3.937063in}{1.159308in}}%
\pgfpathlineto{\pgfqpoint{4.047749in}{1.161440in}}%
\pgfpathlineto{\pgfqpoint{4.158435in}{1.169195in}}%
\pgfpathlineto{\pgfqpoint{4.269121in}{1.175762in}}%
\pgfpathlineto{\pgfqpoint{4.379807in}{1.185867in}}%
\pgfpathlineto{\pgfqpoint{4.490493in}{1.193676in}}%
\pgfpathlineto{\pgfqpoint{4.601179in}{1.195218in}}%
\pgfpathlineto{\pgfqpoint{4.711865in}{1.204817in}}%
\pgfpathlineto{\pgfqpoint{4.822551in}{1.217233in}}%
\pgfpathlineto{\pgfqpoint{4.933237in}{1.237530in}}%
\pgfpathlineto{\pgfqpoint{5.043923in}{1.231388in}}%
\pgfpathlineto{\pgfqpoint{5.154609in}{1.242091in}}%
\pgfpathlineto{\pgfqpoint{5.265296in}{1.257778in}}%
\pgfpathlineto{\pgfqpoint{5.375982in}{1.270731in}}%
\pgfpathlineto{\pgfqpoint{5.486668in}{1.282630in}}%
\pgfpathlineto{\pgfqpoint{5.597354in}{1.295902in}}%
\pgfpathlineto{\pgfqpoint{5.708040in}{1.409446in}}%
\pgfpathlineto{\pgfqpoint{5.818726in}{1.448920in}}%
\pgfpathlineto{\pgfqpoint{5.929412in}{1.525643in}}%
\pgfpathlineto{\pgfqpoint{6.040098in}{1.367102in}}%
\pgfpathlineto{\pgfqpoint{6.150784in}{1.382457in}}%
\pgfpathlineto{\pgfqpoint{6.261470in}{1.418162in}}%
\pgfpathlineto{\pgfqpoint{6.372156in}{1.424653in}}%
\pgfpathlineto{\pgfqpoint{6.482842in}{1.473639in}}%
\pgfpathlineto{\pgfqpoint{6.593528in}{1.500685in}}%
\pgfpathlineto{\pgfqpoint{6.704214in}{1.501461in}}%
\pgfpathlineto{\pgfqpoint{6.814900in}{1.506005in}}%
\pgfpathlineto{\pgfqpoint{6.925586in}{1.543131in}}%
\pgfpathlineto{\pgfqpoint{7.036272in}{1.565439in}}%
\pgfpathlineto{\pgfqpoint{7.146959in}{1.584172in}}%
\pgfpathlineto{\pgfqpoint{7.257645in}{1.622671in}}%
\pgfpathlineto{\pgfqpoint{7.368331in}{1.650946in}}%
\pgfpathlineto{\pgfqpoint{7.479017in}{1.748211in}}%
\pgfpathlineto{\pgfqpoint{7.589703in}{1.854471in}}%
\pgfpathlineto{\pgfqpoint{7.700389in}{1.727036in}}%
\pgfpathlineto{\pgfqpoint{7.811075in}{1.734171in}}%
\pgfpathlineto{\pgfqpoint{7.921761in}{1.782709in}}%
\pgfpathlineto{\pgfqpoint{8.032447in}{1.800064in}}%
\pgfpathlineto{\pgfqpoint{8.143133in}{1.834712in}}%
\pgfpathlineto{\pgfqpoint{8.253819in}{1.878335in}}%
\pgfpathlineto{\pgfqpoint{8.364505in}{1.903283in}}%
\pgfpathlineto{\pgfqpoint{8.475191in}{1.942637in}}%
\pgfpathlineto{\pgfqpoint{8.585877in}{2.066534in}}%
\pgfpathlineto{\pgfqpoint{8.696563in}{2.027948in}}%
\pgfpathlineto{\pgfqpoint{8.807249in}{2.071801in}}%
\pgfpathlineto{\pgfqpoint{8.917936in}{2.107411in}}%
\pgfpathlineto{\pgfqpoint{9.028622in}{2.164872in}}%
\pgfpathlineto{\pgfqpoint{9.139308in}{2.206816in}}%
\pgfpathlineto{\pgfqpoint{9.249994in}{2.256282in}}%
\pgfpathlineto{\pgfqpoint{9.360680in}{2.408135in}}%
\pgfpathlineto{\pgfqpoint{9.471366in}{2.389849in}}%
\pgfpathlineto{\pgfqpoint{9.582052in}{2.404011in}}%
\pgfpathlineto{\pgfqpoint{9.692738in}{2.468555in}}%
\pgfpathlineto{\pgfqpoint{9.803424in}{2.536833in}}%
\pgfpathlineto{\pgfqpoint{9.914110in}{2.608193in}}%
\pgfpathlineto{\pgfqpoint{10.024796in}{2.649478in}}%
\pgfpathlineto{\pgfqpoint{10.135482in}{2.706846in}}%
\pgfpathlineto{\pgfqpoint{10.246168in}{2.762620in}}%
\pgfpathlineto{\pgfqpoint{10.356854in}{2.831675in}}%
\pgfpathlineto{\pgfqpoint{10.467540in}{2.997600in}}%
\pgfpathlineto{\pgfqpoint{10.578226in}{3.022018in}}%
\pgfpathlineto{\pgfqpoint{10.688913in}{3.026835in}}%
\pgfpathlineto{\pgfqpoint{10.799599in}{3.123605in}}%
\pgfpathlineto{\pgfqpoint{10.910285in}{3.231795in}}%
\pgfpathlineto{\pgfqpoint{11.020971in}{3.254542in}}%
\pgfpathlineto{\pgfqpoint{11.131657in}{3.316081in}}%
\pgfpathlineto{\pgfqpoint{11.242343in}{3.537583in}}%
\pgfpathlineto{\pgfqpoint{11.353029in}{3.466483in}}%
\pgfpathlineto{\pgfqpoint{11.463715in}{3.553885in}}%
\pgfpathlineto{\pgfqpoint{11.574401in}{3.661466in}}%
\pgfpathlineto{\pgfqpoint{11.685087in}{3.732687in}}%
\pgfpathlineto{\pgfqpoint{11.795773in}{3.842673in}}%
\pgfpathlineto{\pgfqpoint{11.906459in}{4.051085in}}%
\pgfpathlineto{\pgfqpoint{12.017145in}{4.149327in}}%
\pgfpathlineto{\pgfqpoint{12.127831in}{4.832380in}}%
\pgfpathlineto{\pgfqpoint{12.238517in}{4.468070in}}%
\pgfpathlineto{\pgfqpoint{12.349203in}{4.846615in}}%
\pgfpathlineto{\pgfqpoint{12.459890in}{4.830064in}}%
\pgfpathlineto{\pgfqpoint{12.570576in}{5.601577in}}%
\pgfpathlineto{\pgfqpoint{12.681262in}{5.027501in}}%
\pgfpathlineto{\pgfqpoint{12.791948in}{5.304001in}}%
\pgfusepath{stroke}%
\end{pgfscope}%
\begin{pgfscope}%
\pgfpathrectangle{\pgfqpoint{1.286132in}{0.839159in}}{\pgfqpoint{12.053712in}{5.967710in}}%
\pgfusepath{clip}%
\pgfsetrectcap%
\pgfsetroundjoin%
\pgfsetlinewidth{1.505625pt}%
\definecolor{currentstroke}{rgb}{0.737255,0.741176,0.133333}%
\pgfsetstrokecolor{currentstroke}%
\pgfsetdash{}{0pt}%
\pgfpathmoveto{\pgfqpoint{1.834028in}{1.119295in}}%
\pgfpathlineto{\pgfqpoint{1.944714in}{1.119254in}}%
\pgfpathlineto{\pgfqpoint{2.055400in}{1.119320in}}%
\pgfpathlineto{\pgfqpoint{2.166086in}{1.119408in}}%
\pgfpathlineto{\pgfqpoint{2.276772in}{1.119495in}}%
\pgfpathlineto{\pgfqpoint{2.387458in}{1.119703in}}%
\pgfpathlineto{\pgfqpoint{2.498144in}{1.119648in}}%
\pgfpathlineto{\pgfqpoint{2.608830in}{1.119861in}}%
\pgfpathlineto{\pgfqpoint{2.719516in}{1.119784in}}%
\pgfpathlineto{\pgfqpoint{2.830202in}{1.119807in}}%
\pgfpathlineto{\pgfqpoint{2.940888in}{1.120159in}}%
\pgfpathlineto{\pgfqpoint{3.051574in}{1.120239in}}%
\pgfpathlineto{\pgfqpoint{3.162260in}{1.120284in}}%
\pgfpathlineto{\pgfqpoint{3.272946in}{1.120298in}}%
\pgfpathlineto{\pgfqpoint{3.383632in}{1.120235in}}%
\pgfpathlineto{\pgfqpoint{3.494319in}{1.120271in}}%
\pgfpathlineto{\pgfqpoint{3.605005in}{1.120409in}}%
\pgfpathlineto{\pgfqpoint{3.715691in}{1.120666in}}%
\pgfpathlineto{\pgfqpoint{3.826377in}{1.120809in}}%
\pgfpathlineto{\pgfqpoint{3.937063in}{1.120744in}}%
\pgfpathlineto{\pgfqpoint{4.047749in}{1.120628in}}%
\pgfpathlineto{\pgfqpoint{4.158435in}{1.120780in}}%
\pgfpathlineto{\pgfqpoint{4.269121in}{1.121266in}}%
\pgfpathlineto{\pgfqpoint{4.379807in}{1.121766in}}%
\pgfpathlineto{\pgfqpoint{4.490493in}{1.120968in}}%
\pgfpathlineto{\pgfqpoint{4.601179in}{1.123040in}}%
\pgfpathlineto{\pgfqpoint{4.711865in}{1.121121in}}%
\pgfpathlineto{\pgfqpoint{4.822551in}{1.121407in}}%
\pgfpathlineto{\pgfqpoint{4.933237in}{1.121965in}}%
\pgfpathlineto{\pgfqpoint{5.043923in}{1.121968in}}%
\pgfpathlineto{\pgfqpoint{5.154609in}{1.121687in}}%
\pgfpathlineto{\pgfqpoint{5.265296in}{1.121762in}}%
\pgfpathlineto{\pgfqpoint{5.375982in}{1.122449in}}%
\pgfpathlineto{\pgfqpoint{5.486668in}{1.126907in}}%
\pgfpathlineto{\pgfqpoint{5.597354in}{1.122255in}}%
\pgfpathlineto{\pgfqpoint{5.708040in}{1.122935in}}%
\pgfpathlineto{\pgfqpoint{5.818726in}{1.122268in}}%
\pgfpathlineto{\pgfqpoint{5.929412in}{1.122693in}}%
\pgfpathlineto{\pgfqpoint{6.040098in}{1.123179in}}%
\pgfpathlineto{\pgfqpoint{6.150784in}{1.122498in}}%
\pgfpathlineto{\pgfqpoint{6.261470in}{1.123092in}}%
\pgfpathlineto{\pgfqpoint{6.372156in}{1.124689in}}%
\pgfpathlineto{\pgfqpoint{6.482842in}{1.123168in}}%
\pgfpathlineto{\pgfqpoint{6.593528in}{1.123729in}}%
\pgfpathlineto{\pgfqpoint{6.704214in}{1.122977in}}%
\pgfpathlineto{\pgfqpoint{6.814900in}{1.123334in}}%
\pgfpathlineto{\pgfqpoint{6.925586in}{1.123126in}}%
\pgfpathlineto{\pgfqpoint{7.036272in}{1.124062in}}%
\pgfpathlineto{\pgfqpoint{7.146959in}{1.123922in}}%
\pgfpathlineto{\pgfqpoint{7.257645in}{1.124200in}}%
\pgfpathlineto{\pgfqpoint{7.368331in}{1.123726in}}%
\pgfpathlineto{\pgfqpoint{7.479017in}{1.124072in}}%
\pgfpathlineto{\pgfqpoint{7.589703in}{1.125348in}}%
\pgfpathlineto{\pgfqpoint{7.700389in}{1.125439in}}%
\pgfpathlineto{\pgfqpoint{7.811075in}{1.125379in}}%
\pgfpathlineto{\pgfqpoint{7.921761in}{1.124117in}}%
\pgfpathlineto{\pgfqpoint{8.032447in}{1.125485in}}%
\pgfpathlineto{\pgfqpoint{8.143133in}{1.124760in}}%
\pgfpathlineto{\pgfqpoint{8.253819in}{1.125258in}}%
\pgfpathlineto{\pgfqpoint{8.364505in}{1.125407in}}%
\pgfpathlineto{\pgfqpoint{8.475191in}{1.126055in}}%
\pgfpathlineto{\pgfqpoint{8.585877in}{1.126296in}}%
\pgfpathlineto{\pgfqpoint{8.696563in}{1.126464in}}%
\pgfpathlineto{\pgfqpoint{8.807249in}{1.140394in}}%
\pgfpathlineto{\pgfqpoint{8.917936in}{1.124831in}}%
\pgfpathlineto{\pgfqpoint{9.028622in}{1.126329in}}%
\pgfpathlineto{\pgfqpoint{9.139308in}{1.125906in}}%
\pgfpathlineto{\pgfqpoint{9.249994in}{1.125606in}}%
\pgfpathlineto{\pgfqpoint{9.360680in}{1.125710in}}%
\pgfpathlineto{\pgfqpoint{9.471366in}{1.127543in}}%
\pgfpathlineto{\pgfqpoint{9.582052in}{1.125947in}}%
\pgfpathlineto{\pgfqpoint{9.692738in}{1.126320in}}%
\pgfpathlineto{\pgfqpoint{9.803424in}{1.126142in}}%
\pgfpathlineto{\pgfqpoint{9.914110in}{1.126292in}}%
\pgfpathlineto{\pgfqpoint{10.024796in}{1.127566in}}%
\pgfpathlineto{\pgfqpoint{10.135482in}{1.126395in}}%
\pgfpathlineto{\pgfqpoint{10.246168in}{1.126496in}}%
\pgfpathlineto{\pgfqpoint{10.356854in}{1.127666in}}%
\pgfpathlineto{\pgfqpoint{10.467540in}{1.126868in}}%
\pgfpathlineto{\pgfqpoint{10.578226in}{1.127656in}}%
\pgfpathlineto{\pgfqpoint{10.688913in}{1.128073in}}%
\pgfpathlineto{\pgfqpoint{10.799599in}{1.126968in}}%
\pgfpathlineto{\pgfqpoint{10.910285in}{1.128531in}}%
\pgfpathlineto{\pgfqpoint{11.020971in}{1.128267in}}%
\pgfpathlineto{\pgfqpoint{11.131657in}{1.127852in}}%
\pgfpathlineto{\pgfqpoint{11.242343in}{1.127762in}}%
\pgfpathlineto{\pgfqpoint{11.353029in}{1.128286in}}%
\pgfpathlineto{\pgfqpoint{11.463715in}{1.127599in}}%
\pgfpathlineto{\pgfqpoint{11.574401in}{1.128838in}}%
\pgfpathlineto{\pgfqpoint{11.685087in}{1.128318in}}%
\pgfpathlineto{\pgfqpoint{11.795773in}{1.127320in}}%
\pgfpathlineto{\pgfqpoint{11.906459in}{1.129238in}}%
\pgfpathlineto{\pgfqpoint{12.017145in}{1.129445in}}%
\pgfpathlineto{\pgfqpoint{12.127831in}{1.129962in}}%
\pgfpathlineto{\pgfqpoint{12.238517in}{1.128679in}}%
\pgfpathlineto{\pgfqpoint{12.349203in}{1.129120in}}%
\pgfpathlineto{\pgfqpoint{12.459890in}{1.128849in}}%
\pgfpathlineto{\pgfqpoint{12.570576in}{1.142217in}}%
\pgfpathlineto{\pgfqpoint{12.681262in}{1.137573in}}%
\pgfpathlineto{\pgfqpoint{12.791948in}{1.130520in}}%
\pgfusepath{stroke}%
\end{pgfscope}%
\begin{pgfscope}%
\pgfpathrectangle{\pgfqpoint{1.286132in}{0.839159in}}{\pgfqpoint{12.053712in}{5.967710in}}%
\pgfusepath{clip}%
\pgfsetrectcap%
\pgfsetroundjoin%
\pgfsetlinewidth{1.505625pt}%
\definecolor{currentstroke}{rgb}{0.090196,0.745098,0.811765}%
\pgfsetstrokecolor{currentstroke}%
\pgfsetdash{}{0pt}%
\pgfpathmoveto{\pgfqpoint{1.834028in}{1.119213in}}%
\pgfpathlineto{\pgfqpoint{1.944714in}{1.119224in}}%
\pgfpathlineto{\pgfqpoint{2.055400in}{1.119261in}}%
\pgfpathlineto{\pgfqpoint{2.166086in}{1.119381in}}%
\pgfpathlineto{\pgfqpoint{2.276772in}{1.119569in}}%
\pgfpathlineto{\pgfqpoint{2.387458in}{1.119900in}}%
\pgfpathlineto{\pgfqpoint{2.498144in}{1.120085in}}%
\pgfpathlineto{\pgfqpoint{2.608830in}{1.120281in}}%
\pgfpathlineto{\pgfqpoint{2.719516in}{1.120825in}}%
\pgfpathlineto{\pgfqpoint{2.830202in}{1.121806in}}%
\pgfpathlineto{\pgfqpoint{2.940888in}{1.122223in}}%
\pgfpathlineto{\pgfqpoint{3.051574in}{1.123696in}}%
\pgfpathlineto{\pgfqpoint{3.162260in}{1.124352in}}%
\pgfpathlineto{\pgfqpoint{3.272946in}{1.123897in}}%
\pgfpathlineto{\pgfqpoint{3.383632in}{1.131900in}}%
\pgfpathlineto{\pgfqpoint{3.494319in}{1.126466in}}%
\pgfpathlineto{\pgfqpoint{3.605005in}{1.129704in}}%
\pgfpathlineto{\pgfqpoint{3.715691in}{1.130661in}}%
\pgfpathlineto{\pgfqpoint{3.826377in}{1.132107in}}%
\pgfpathlineto{\pgfqpoint{3.937063in}{1.132190in}}%
\pgfpathlineto{\pgfqpoint{4.047749in}{1.136292in}}%
\pgfpathlineto{\pgfqpoint{4.158435in}{1.146202in}}%
\pgfpathlineto{\pgfqpoint{4.269121in}{1.139299in}}%
\pgfpathlineto{\pgfqpoint{4.379807in}{1.141182in}}%
\pgfpathlineto{\pgfqpoint{4.490493in}{1.141264in}}%
\pgfpathlineto{\pgfqpoint{4.601179in}{1.150528in}}%
\pgfpathlineto{\pgfqpoint{4.711865in}{1.142372in}}%
\pgfpathlineto{\pgfqpoint{4.822551in}{1.146874in}}%
\pgfpathlineto{\pgfqpoint{4.933237in}{1.150516in}}%
\pgfpathlineto{\pgfqpoint{5.043923in}{1.158104in}}%
\pgfpathlineto{\pgfqpoint{5.154609in}{1.162671in}}%
\pgfpathlineto{\pgfqpoint{5.265296in}{1.165881in}}%
\pgfpathlineto{\pgfqpoint{5.375982in}{1.164722in}}%
\pgfpathlineto{\pgfqpoint{5.486668in}{1.216134in}}%
\pgfpathlineto{\pgfqpoint{5.597354in}{1.180826in}}%
\pgfpathlineto{\pgfqpoint{5.708040in}{1.179628in}}%
\pgfpathlineto{\pgfqpoint{5.818726in}{1.181618in}}%
\pgfpathlineto{\pgfqpoint{5.929412in}{1.183886in}}%
\pgfpathlineto{\pgfqpoint{6.040098in}{1.196637in}}%
\pgfpathlineto{\pgfqpoint{6.150784in}{1.202515in}}%
\pgfpathlineto{\pgfqpoint{6.261470in}{1.198389in}}%
\pgfpathlineto{\pgfqpoint{6.372156in}{1.200665in}}%
\pgfpathlineto{\pgfqpoint{6.482842in}{1.211647in}}%
\pgfpathlineto{\pgfqpoint{6.593528in}{1.239124in}}%
\pgfpathlineto{\pgfqpoint{6.704214in}{1.210994in}}%
\pgfpathlineto{\pgfqpoint{6.814900in}{1.219876in}}%
\pgfpathlineto{\pgfqpoint{6.925586in}{1.224702in}}%
\pgfpathlineto{\pgfqpoint{7.036272in}{1.242649in}}%
\pgfpathlineto{\pgfqpoint{7.146959in}{1.256317in}}%
\pgfpathlineto{\pgfqpoint{7.257645in}{1.248623in}}%
\pgfpathlineto{\pgfqpoint{7.368331in}{1.241951in}}%
\pgfpathlineto{\pgfqpoint{7.479017in}{1.273817in}}%
\pgfpathlineto{\pgfqpoint{7.589703in}{1.286767in}}%
\pgfpathlineto{\pgfqpoint{7.700389in}{1.279657in}}%
\pgfpathlineto{\pgfqpoint{7.811075in}{1.254765in}}%
\pgfpathlineto{\pgfqpoint{7.921761in}{1.283881in}}%
\pgfpathlineto{\pgfqpoint{8.032447in}{1.260430in}}%
\pgfpathlineto{\pgfqpoint{8.143133in}{1.293566in}}%
\pgfpathlineto{\pgfqpoint{8.253819in}{1.329246in}}%
\pgfpathlineto{\pgfqpoint{8.364505in}{1.321230in}}%
\pgfpathlineto{\pgfqpoint{8.475191in}{1.320987in}}%
\pgfpathlineto{\pgfqpoint{8.585877in}{1.448711in}}%
\pgfpathlineto{\pgfqpoint{8.696563in}{1.371354in}}%
\pgfpathlineto{\pgfqpoint{8.807249in}{1.350793in}}%
\pgfpathlineto{\pgfqpoint{8.917936in}{1.398225in}}%
\pgfpathlineto{\pgfqpoint{9.028622in}{1.334888in}}%
\pgfpathlineto{\pgfqpoint{9.139308in}{1.404836in}}%
\pgfpathlineto{\pgfqpoint{9.249994in}{1.390941in}}%
\pgfpathlineto{\pgfqpoint{9.360680in}{1.395513in}}%
\pgfpathlineto{\pgfqpoint{9.471366in}{1.402222in}}%
\pgfpathlineto{\pgfqpoint{9.582052in}{1.378632in}}%
\pgfpathlineto{\pgfqpoint{9.692738in}{1.381171in}}%
\pgfpathlineto{\pgfqpoint{9.803424in}{1.412856in}}%
\pgfpathlineto{\pgfqpoint{9.914110in}{1.401695in}}%
\pgfpathlineto{\pgfqpoint{10.024796in}{1.437836in}}%
\pgfpathlineto{\pgfqpoint{10.135482in}{1.488685in}}%
\pgfpathlineto{\pgfqpoint{10.246168in}{1.486489in}}%
\pgfpathlineto{\pgfqpoint{10.356854in}{1.447590in}}%
\pgfpathlineto{\pgfqpoint{10.467540in}{1.492187in}}%
\pgfpathlineto{\pgfqpoint{10.578226in}{1.488013in}}%
\pgfpathlineto{\pgfqpoint{10.688913in}{1.517534in}}%
\pgfpathlineto{\pgfqpoint{10.799599in}{1.529641in}}%
\pgfpathlineto{\pgfqpoint{10.910285in}{1.504209in}}%
\pgfpathlineto{\pgfqpoint{11.020971in}{1.485855in}}%
\pgfpathlineto{\pgfqpoint{11.131657in}{1.508889in}}%
\pgfpathlineto{\pgfqpoint{11.242343in}{1.631609in}}%
\pgfpathlineto{\pgfqpoint{11.353029in}{1.591077in}}%
\pgfpathlineto{\pgfqpoint{11.463715in}{1.588044in}}%
\pgfpathlineto{\pgfqpoint{11.574401in}{1.616905in}}%
\pgfpathlineto{\pgfqpoint{11.685087in}{1.570002in}}%
\pgfpathlineto{\pgfqpoint{11.795773in}{1.577813in}}%
\pgfpathlineto{\pgfqpoint{11.906459in}{1.703458in}}%
\pgfpathlineto{\pgfqpoint{12.017145in}{1.721174in}}%
\pgfpathlineto{\pgfqpoint{12.127831in}{1.706109in}}%
\pgfpathlineto{\pgfqpoint{12.238517in}{1.663746in}}%
\pgfpathlineto{\pgfqpoint{12.349203in}{1.705124in}}%
\pgfpathlineto{\pgfqpoint{12.459890in}{1.727420in}}%
\pgfpathlineto{\pgfqpoint{12.570576in}{1.853010in}}%
\pgfpathlineto{\pgfqpoint{12.681262in}{1.828996in}}%
\pgfpathlineto{\pgfqpoint{12.791948in}{1.785454in}}%
\pgfusepath{stroke}%
\end{pgfscope}%
\begin{pgfscope}%
\pgfpathrectangle{\pgfqpoint{1.286132in}{0.839159in}}{\pgfqpoint{12.053712in}{5.967710in}}%
\pgfusepath{clip}%
\pgfsetrectcap%
\pgfsetroundjoin%
\pgfsetlinewidth{1.505625pt}%
\definecolor{currentstroke}{rgb}{0.121569,0.466667,0.705882}%
\pgfsetstrokecolor{currentstroke}%
\pgfsetdash{}{0pt}%
\pgfpathmoveto{\pgfqpoint{1.834028in}{1.119240in}}%
\pgfpathlineto{\pgfqpoint{1.944714in}{1.119343in}}%
\pgfpathlineto{\pgfqpoint{2.055400in}{1.119401in}}%
\pgfpathlineto{\pgfqpoint{2.166086in}{1.119448in}}%
\pgfpathlineto{\pgfqpoint{2.276772in}{1.119521in}}%
\pgfpathlineto{\pgfqpoint{2.387458in}{1.125622in}}%
\pgfpathlineto{\pgfqpoint{2.498144in}{1.119663in}}%
\pgfpathlineto{\pgfqpoint{2.608830in}{1.119747in}}%
\pgfpathlineto{\pgfqpoint{2.719516in}{1.119831in}}%
\pgfpathlineto{\pgfqpoint{2.830202in}{1.119936in}}%
\pgfpathlineto{\pgfqpoint{2.940888in}{1.120006in}}%
\pgfpathlineto{\pgfqpoint{3.051574in}{1.120095in}}%
\pgfpathlineto{\pgfqpoint{3.162260in}{1.120196in}}%
\pgfpathlineto{\pgfqpoint{3.272946in}{1.120542in}}%
\pgfpathlineto{\pgfqpoint{3.383632in}{1.120421in}}%
\pgfpathlineto{\pgfqpoint{3.494319in}{1.120622in}}%
\pgfpathlineto{\pgfqpoint{3.605005in}{1.120657in}}%
\pgfpathlineto{\pgfqpoint{3.715691in}{1.120855in}}%
\pgfpathlineto{\pgfqpoint{3.826377in}{1.120874in}}%
\pgfpathlineto{\pgfqpoint{3.937063in}{1.121804in}}%
\pgfpathlineto{\pgfqpoint{4.047749in}{1.121245in}}%
\pgfpathlineto{\pgfqpoint{4.158435in}{1.121549in}}%
\pgfpathlineto{\pgfqpoint{4.269121in}{1.121447in}}%
\pgfpathlineto{\pgfqpoint{4.379807in}{1.121768in}}%
\pgfpathlineto{\pgfqpoint{4.490493in}{1.121800in}}%
\pgfpathlineto{\pgfqpoint{4.601179in}{1.122073in}}%
\pgfpathlineto{\pgfqpoint{4.711865in}{1.122214in}}%
\pgfpathlineto{\pgfqpoint{4.822551in}{1.122421in}}%
\pgfpathlineto{\pgfqpoint{4.933237in}{1.122496in}}%
\pgfpathlineto{\pgfqpoint{5.043923in}{1.122579in}}%
\pgfpathlineto{\pgfqpoint{5.154609in}{1.123236in}}%
\pgfpathlineto{\pgfqpoint{5.265296in}{1.122866in}}%
\pgfpathlineto{\pgfqpoint{5.375982in}{1.123120in}}%
\pgfpathlineto{\pgfqpoint{5.486668in}{1.123687in}}%
\pgfpathlineto{\pgfqpoint{5.597354in}{1.123498in}}%
\pgfpathlineto{\pgfqpoint{5.708040in}{1.123782in}}%
\pgfpathlineto{\pgfqpoint{5.818726in}{1.123917in}}%
\pgfpathlineto{\pgfqpoint{5.929412in}{1.124382in}}%
\pgfpathlineto{\pgfqpoint{6.040098in}{1.124432in}}%
\pgfpathlineto{\pgfqpoint{6.150784in}{1.124703in}}%
\pgfpathlineto{\pgfqpoint{6.261470in}{1.124728in}}%
\pgfpathlineto{\pgfqpoint{6.372156in}{1.125006in}}%
\pgfpathlineto{\pgfqpoint{6.482842in}{1.125251in}}%
\pgfpathlineto{\pgfqpoint{6.593528in}{1.125574in}}%
\pgfpathlineto{\pgfqpoint{6.704214in}{1.125644in}}%
\pgfpathlineto{\pgfqpoint{6.814900in}{1.127009in}}%
\pgfpathlineto{\pgfqpoint{6.925586in}{1.126503in}}%
\pgfpathlineto{\pgfqpoint{7.036272in}{1.126490in}}%
\pgfpathlineto{\pgfqpoint{7.146959in}{1.126772in}}%
\pgfpathlineto{\pgfqpoint{7.257645in}{1.127141in}}%
\pgfpathlineto{\pgfqpoint{7.368331in}{1.127343in}}%
\pgfpathlineto{\pgfqpoint{7.479017in}{1.127656in}}%
\pgfpathlineto{\pgfqpoint{7.589703in}{1.128481in}}%
\pgfpathlineto{\pgfqpoint{7.700389in}{1.127891in}}%
\pgfpathlineto{\pgfqpoint{7.811075in}{1.128507in}}%
\pgfpathlineto{\pgfqpoint{7.921761in}{1.128664in}}%
\pgfpathlineto{\pgfqpoint{8.032447in}{1.129148in}}%
\pgfpathlineto{\pgfqpoint{8.143133in}{1.129208in}}%
\pgfpathlineto{\pgfqpoint{8.253819in}{1.129594in}}%
\pgfpathlineto{\pgfqpoint{8.364505in}{1.129948in}}%
\pgfpathlineto{\pgfqpoint{8.475191in}{1.131025in}}%
\pgfpathlineto{\pgfqpoint{8.585877in}{1.130625in}}%
\pgfpathlineto{\pgfqpoint{8.696563in}{1.131041in}}%
\pgfpathlineto{\pgfqpoint{8.807249in}{1.131374in}}%
\pgfpathlineto{\pgfqpoint{8.917936in}{1.131443in}}%
\pgfpathlineto{\pgfqpoint{9.028622in}{1.131713in}}%
\pgfpathlineto{\pgfqpoint{9.139308in}{1.132221in}}%
\pgfpathlineto{\pgfqpoint{9.249994in}{1.132578in}}%
\pgfpathlineto{\pgfqpoint{9.360680in}{1.132626in}}%
\pgfpathlineto{\pgfqpoint{9.471366in}{1.133311in}}%
\pgfpathlineto{\pgfqpoint{9.582052in}{1.133452in}}%
\pgfpathlineto{\pgfqpoint{9.692738in}{1.133995in}}%
\pgfpathlineto{\pgfqpoint{9.803424in}{1.134103in}}%
\pgfpathlineto{\pgfqpoint{9.914110in}{1.142614in}}%
\pgfpathlineto{\pgfqpoint{10.024796in}{1.135165in}}%
\pgfpathlineto{\pgfqpoint{10.135482in}{1.135376in}}%
\pgfpathlineto{\pgfqpoint{10.246168in}{1.135813in}}%
\pgfpathlineto{\pgfqpoint{10.356854in}{1.136417in}}%
\pgfpathlineto{\pgfqpoint{10.467540in}{1.139383in}}%
\pgfpathlineto{\pgfqpoint{10.578226in}{1.136911in}}%
\pgfpathlineto{\pgfqpoint{10.688913in}{1.137419in}}%
\pgfpathlineto{\pgfqpoint{10.799599in}{1.137924in}}%
\pgfpathlineto{\pgfqpoint{10.910285in}{1.138616in}}%
\pgfpathlineto{\pgfqpoint{11.020971in}{1.138985in}}%
\pgfpathlineto{\pgfqpoint{11.131657in}{1.139403in}}%
\pgfpathlineto{\pgfqpoint{11.242343in}{1.139715in}}%
\pgfpathlineto{\pgfqpoint{11.353029in}{1.140041in}}%
\pgfpathlineto{\pgfqpoint{11.463715in}{1.140382in}}%
\pgfpathlineto{\pgfqpoint{11.574401in}{1.140905in}}%
\pgfpathlineto{\pgfqpoint{11.685087in}{1.141405in}}%
\pgfpathlineto{\pgfqpoint{11.795773in}{1.141771in}}%
\pgfpathlineto{\pgfqpoint{11.906459in}{1.142446in}}%
\pgfpathlineto{\pgfqpoint{12.017145in}{1.142873in}}%
\pgfpathlineto{\pgfqpoint{12.127831in}{1.143283in}}%
\pgfpathlineto{\pgfqpoint{12.238517in}{1.143788in}}%
\pgfpathlineto{\pgfqpoint{12.349203in}{1.144036in}}%
\pgfpathlineto{\pgfqpoint{12.459890in}{1.145398in}}%
\pgfpathlineto{\pgfqpoint{12.570576in}{1.152348in}}%
\pgfpathlineto{\pgfqpoint{12.681262in}{1.156510in}}%
\pgfpathlineto{\pgfqpoint{12.791948in}{1.148366in}}%
\pgfusepath{stroke}%
\end{pgfscope}%
\begin{pgfscope}%
\pgfpathrectangle{\pgfqpoint{1.286132in}{0.839159in}}{\pgfqpoint{12.053712in}{5.967710in}}%
\pgfusepath{clip}%
\pgfsetrectcap%
\pgfsetroundjoin%
\pgfsetlinewidth{1.505625pt}%
\definecolor{currentstroke}{rgb}{1.000000,0.498039,0.054902}%
\pgfsetstrokecolor{currentstroke}%
\pgfsetdash{}{0pt}%
\pgfpathmoveto{\pgfqpoint{1.834028in}{1.119259in}}%
\pgfpathlineto{\pgfqpoint{1.944714in}{1.119601in}}%
\pgfpathlineto{\pgfqpoint{2.055400in}{1.120230in}}%
\pgfpathlineto{\pgfqpoint{2.166086in}{1.120284in}}%
\pgfpathlineto{\pgfqpoint{2.276772in}{1.120876in}}%
\pgfpathlineto{\pgfqpoint{2.387458in}{1.121802in}}%
\pgfpathlineto{\pgfqpoint{2.498144in}{1.126464in}}%
\pgfpathlineto{\pgfqpoint{2.608830in}{1.123589in}}%
\pgfpathlineto{\pgfqpoint{2.719516in}{1.124916in}}%
\pgfpathlineto{\pgfqpoint{2.830202in}{1.126436in}}%
\pgfpathlineto{\pgfqpoint{2.940888in}{1.130052in}}%
\pgfpathlineto{\pgfqpoint{3.051574in}{1.129464in}}%
\pgfpathlineto{\pgfqpoint{3.162260in}{1.131988in}}%
\pgfpathlineto{\pgfqpoint{3.272946in}{1.143005in}}%
\pgfpathlineto{\pgfqpoint{3.383632in}{1.136952in}}%
\pgfpathlineto{\pgfqpoint{3.494319in}{1.140434in}}%
\pgfpathlineto{\pgfqpoint{3.605005in}{1.148300in}}%
\pgfpathlineto{\pgfqpoint{3.715691in}{1.151541in}}%
\pgfpathlineto{\pgfqpoint{3.826377in}{1.155722in}}%
\pgfpathlineto{\pgfqpoint{3.937063in}{1.158376in}}%
\pgfpathlineto{\pgfqpoint{4.047749in}{1.163599in}}%
\pgfpathlineto{\pgfqpoint{4.158435in}{1.169487in}}%
\pgfpathlineto{\pgfqpoint{4.269121in}{1.177680in}}%
\pgfpathlineto{\pgfqpoint{4.379807in}{1.181587in}}%
\pgfpathlineto{\pgfqpoint{4.490493in}{1.188539in}}%
\pgfpathlineto{\pgfqpoint{4.601179in}{1.196466in}}%
\pgfpathlineto{\pgfqpoint{4.711865in}{1.208973in}}%
\pgfpathlineto{\pgfqpoint{4.822551in}{1.218347in}}%
\pgfpathlineto{\pgfqpoint{4.933237in}{1.225112in}}%
\pgfpathlineto{\pgfqpoint{5.043923in}{1.240029in}}%
\pgfpathlineto{\pgfqpoint{5.154609in}{1.247594in}}%
\pgfpathlineto{\pgfqpoint{5.265296in}{1.281935in}}%
\pgfpathlineto{\pgfqpoint{5.375982in}{1.291819in}}%
\pgfpathlineto{\pgfqpoint{5.486668in}{1.446558in}}%
\pgfpathlineto{\pgfqpoint{5.597354in}{1.296396in}}%
\pgfpathlineto{\pgfqpoint{5.708040in}{1.318459in}}%
\pgfpathlineto{\pgfqpoint{5.818726in}{1.383696in}}%
\pgfpathlineto{\pgfqpoint{5.929412in}{1.396217in}}%
\pgfpathlineto{\pgfqpoint{6.040098in}{1.365236in}}%
\pgfpathlineto{\pgfqpoint{6.150784in}{1.400848in}}%
\pgfpathlineto{\pgfqpoint{6.261470in}{1.395074in}}%
\pgfpathlineto{\pgfqpoint{6.372156in}{1.457084in}}%
\pgfpathlineto{\pgfqpoint{6.482842in}{1.445291in}}%
\pgfpathlineto{\pgfqpoint{6.593528in}{1.467335in}}%
\pgfpathlineto{\pgfqpoint{6.704214in}{1.487681in}}%
\pgfpathlineto{\pgfqpoint{6.814900in}{1.509433in}}%
\pgfpathlineto{\pgfqpoint{6.925586in}{1.534582in}}%
\pgfpathlineto{\pgfqpoint{7.036272in}{1.596553in}}%
\pgfpathlineto{\pgfqpoint{7.146959in}{1.613752in}}%
\pgfpathlineto{\pgfqpoint{7.257645in}{1.624298in}}%
\pgfpathlineto{\pgfqpoint{7.368331in}{1.621206in}}%
\pgfpathlineto{\pgfqpoint{7.479017in}{1.836339in}}%
\pgfpathlineto{\pgfqpoint{7.589703in}{1.677179in}}%
\pgfpathlineto{\pgfqpoint{7.700389in}{1.695006in}}%
\pgfpathlineto{\pgfqpoint{7.811075in}{1.732315in}}%
\pgfpathlineto{\pgfqpoint{7.921761in}{1.777119in}}%
\pgfpathlineto{\pgfqpoint{8.032447in}{1.789559in}}%
\pgfpathlineto{\pgfqpoint{8.143133in}{1.829477in}}%
\pgfpathlineto{\pgfqpoint{8.253819in}{1.953283in}}%
\pgfpathlineto{\pgfqpoint{8.364505in}{1.955544in}}%
\pgfpathlineto{\pgfqpoint{8.475191in}{2.142492in}}%
\pgfpathlineto{\pgfqpoint{8.585877in}{2.120781in}}%
\pgfpathlineto{\pgfqpoint{8.696563in}{2.122137in}}%
\pgfpathlineto{\pgfqpoint{8.807249in}{2.076518in}}%
\pgfpathlineto{\pgfqpoint{8.917936in}{2.122821in}}%
\pgfpathlineto{\pgfqpoint{9.028622in}{2.137488in}}%
\pgfpathlineto{\pgfqpoint{9.139308in}{2.177108in}}%
\pgfpathlineto{\pgfqpoint{9.249994in}{2.275504in}}%
\pgfpathlineto{\pgfqpoint{9.360680in}{2.280850in}}%
\pgfpathlineto{\pgfqpoint{9.471366in}{2.324925in}}%
\pgfpathlineto{\pgfqpoint{9.582052in}{2.377126in}}%
\pgfpathlineto{\pgfqpoint{9.692738in}{2.428731in}}%
\pgfpathlineto{\pgfqpoint{9.803424in}{2.490265in}}%
\pgfpathlineto{\pgfqpoint{9.914110in}{2.579545in}}%
\pgfpathlineto{\pgfqpoint{10.024796in}{2.597373in}}%
\pgfpathlineto{\pgfqpoint{10.135482in}{2.660074in}}%
\pgfpathlineto{\pgfqpoint{10.246168in}{2.720589in}}%
\pgfpathlineto{\pgfqpoint{10.356854in}{2.804084in}}%
\pgfpathlineto{\pgfqpoint{10.467540in}{2.864643in}}%
\pgfpathlineto{\pgfqpoint{10.578226in}{2.918632in}}%
\pgfpathlineto{\pgfqpoint{10.688913in}{2.976157in}}%
\pgfpathlineto{\pgfqpoint{10.799599in}{3.130887in}}%
\pgfpathlineto{\pgfqpoint{10.910285in}{3.127618in}}%
\pgfpathlineto{\pgfqpoint{11.020971in}{3.188271in}}%
\pgfpathlineto{\pgfqpoint{11.131657in}{3.265632in}}%
\pgfpathlineto{\pgfqpoint{11.242343in}{3.452900in}}%
\pgfpathlineto{\pgfqpoint{11.353029in}{3.422599in}}%
\pgfpathlineto{\pgfqpoint{11.463715in}{3.500501in}}%
\pgfpathlineto{\pgfqpoint{11.574401in}{3.647684in}}%
\pgfpathlineto{\pgfqpoint{11.685087in}{3.657348in}}%
\pgfpathlineto{\pgfqpoint{11.795773in}{3.748662in}}%
\pgfpathlineto{\pgfqpoint{11.906459in}{3.904318in}}%
\pgfpathlineto{\pgfqpoint{12.017145in}{3.963644in}}%
\pgfpathlineto{\pgfqpoint{12.127831in}{4.225779in}}%
\pgfpathlineto{\pgfqpoint{12.238517in}{4.184959in}}%
\pgfpathlineto{\pgfqpoint{12.349203in}{4.207001in}}%
\pgfpathlineto{\pgfqpoint{12.459890in}{4.881209in}}%
\pgfpathlineto{\pgfqpoint{12.570576in}{5.030210in}}%
\pgfpathlineto{\pgfqpoint{12.681262in}{5.410128in}}%
\pgfpathlineto{\pgfqpoint{12.791948in}{5.019381in}}%
\pgfusepath{stroke}%
\end{pgfscope}%
\begin{pgfscope}%
\pgfsetrectcap%
\pgfsetmiterjoin%
\pgfsetlinewidth{0.803000pt}%
\definecolor{currentstroke}{rgb}{0.000000,0.000000,0.000000}%
\pgfsetstrokecolor{currentstroke}%
\pgfsetdash{}{0pt}%
\pgfpathmoveto{\pgfqpoint{1.286132in}{0.839159in}}%
\pgfpathlineto{\pgfqpoint{1.286132in}{6.806869in}}%
\pgfusepath{stroke}%
\end{pgfscope}%
\begin{pgfscope}%
\pgfsetrectcap%
\pgfsetmiterjoin%
\pgfsetlinewidth{0.803000pt}%
\definecolor{currentstroke}{rgb}{0.000000,0.000000,0.000000}%
\pgfsetstrokecolor{currentstroke}%
\pgfsetdash{}{0pt}%
\pgfpathmoveto{\pgfqpoint{13.339844in}{0.839159in}}%
\pgfpathlineto{\pgfqpoint{13.339844in}{6.806869in}}%
\pgfusepath{stroke}%
\end{pgfscope}%
\begin{pgfscope}%
\pgfsetrectcap%
\pgfsetmiterjoin%
\pgfsetlinewidth{0.803000pt}%
\definecolor{currentstroke}{rgb}{0.000000,0.000000,0.000000}%
\pgfsetstrokecolor{currentstroke}%
\pgfsetdash{}{0pt}%
\pgfpathmoveto{\pgfqpoint{1.286132in}{0.839159in}}%
\pgfpathlineto{\pgfqpoint{13.339844in}{0.839159in}}%
\pgfusepath{stroke}%
\end{pgfscope}%
\begin{pgfscope}%
\pgfsetrectcap%
\pgfsetmiterjoin%
\pgfsetlinewidth{0.803000pt}%
\definecolor{currentstroke}{rgb}{0.000000,0.000000,0.000000}%
\pgfsetstrokecolor{currentstroke}%
\pgfsetdash{}{0pt}%
\pgfpathmoveto{\pgfqpoint{1.286132in}{6.806869in}}%
\pgfpathlineto{\pgfqpoint{13.339844in}{6.806869in}}%
\pgfusepath{stroke}%
\end{pgfscope}%
\begin{pgfscope}%
\definecolor{textcolor}{rgb}{0.000000,0.000000,0.000000}%
\pgfsetstrokecolor{textcolor}%
\pgfsetfillcolor{textcolor}%
\pgftext[x=7.312988in,y=6.890203in,,base]{\color{textcolor}\rmfamily\fontsize{12.000000}{14.400000}\selectfont Performance of heuristics}%
\end{pgfscope}%
\begin{pgfscope}%
\pgfsetbuttcap%
\pgfsetmiterjoin%
\definecolor{currentfill}{rgb}{1.000000,1.000000,1.000000}%
\pgfsetfillcolor{currentfill}%
\pgfsetfillopacity{0.800000}%
\pgfsetlinewidth{1.003750pt}%
\definecolor{currentstroke}{rgb}{0.800000,0.800000,0.800000}%
\pgfsetstrokecolor{currentstroke}%
\pgfsetstrokeopacity{0.800000}%
\pgfsetdash{}{0pt}%
\pgfpathmoveto{\pgfqpoint{1.383354in}{4.243571in}}%
\pgfpathlineto{\pgfqpoint{2.752684in}{4.243571in}}%
\pgfpathquadraticcurveto{\pgfqpoint{2.780462in}{4.243571in}}{\pgfqpoint{2.780462in}{4.271349in}}%
\pgfpathlineto{\pgfqpoint{2.780462in}{6.709647in}}%
\pgfpathquadraticcurveto{\pgfqpoint{2.780462in}{6.737425in}}{\pgfqpoint{2.752684in}{6.737425in}}%
\pgfpathlineto{\pgfqpoint{1.383354in}{6.737425in}}%
\pgfpathquadraticcurveto{\pgfqpoint{1.355576in}{6.737425in}}{\pgfqpoint{1.355576in}{6.709647in}}%
\pgfpathlineto{\pgfqpoint{1.355576in}{4.271349in}}%
\pgfpathquadraticcurveto{\pgfqpoint{1.355576in}{4.243571in}}{\pgfqpoint{1.383354in}{4.243571in}}%
\pgfpathclose%
\pgfusepath{stroke,fill}%
\end{pgfscope}%
\begin{pgfscope}%
\pgfsetbuttcap%
\pgfsetroundjoin%
\pgfsetlinewidth{1.505625pt}%
\definecolor{currentstroke}{rgb}{0.121569,0.466667,0.705882}%
\pgfsetstrokecolor{currentstroke}%
\pgfsetdash{}{0pt}%
\pgfpathmoveto{\pgfqpoint{1.550020in}{6.555513in}}%
\pgfpathlineto{\pgfqpoint{1.550020in}{6.694402in}}%
\pgfusepath{stroke}%
\end{pgfscope}%
\begin{pgfscope}%
\pgfsetrectcap%
\pgfsetroundjoin%
\pgfsetlinewidth{1.505625pt}%
\definecolor{currentstroke}{rgb}{0.121569,0.466667,0.705882}%
\pgfsetstrokecolor{currentstroke}%
\pgfsetdash{}{0pt}%
\pgfpathmoveto{\pgfqpoint{1.411132in}{6.624957in}}%
\pgfpathlineto{\pgfqpoint{1.688909in}{6.624957in}}%
\pgfusepath{stroke}%
\end{pgfscope}%
\begin{pgfscope}%
\definecolor{textcolor}{rgb}{0.000000,0.000000,0.000000}%
\pgfsetstrokecolor{textcolor}%
\pgfsetfillcolor{textcolor}%
\pgftext[x=1.800020in,y=6.576346in,left,base]{\color{textcolor}\rmfamily\fontsize{10.000000}{12.000000}\selectfont ER-swap}%
\end{pgfscope}%
\begin{pgfscope}%
\pgfsetbuttcap%
\pgfsetroundjoin%
\pgfsetlinewidth{1.505625pt}%
\definecolor{currentstroke}{rgb}{1.000000,0.498039,0.054902}%
\pgfsetstrokecolor{currentstroke}%
\pgfsetdash{}{0pt}%
\pgfpathmoveto{\pgfqpoint{1.550020in}{6.351656in}}%
\pgfpathlineto{\pgfqpoint{1.550020in}{6.490544in}}%
\pgfusepath{stroke}%
\end{pgfscope}%
\begin{pgfscope}%
\pgfsetrectcap%
\pgfsetroundjoin%
\pgfsetlinewidth{1.505625pt}%
\definecolor{currentstroke}{rgb}{1.000000,0.498039,0.054902}%
\pgfsetstrokecolor{currentstroke}%
\pgfsetdash{}{0pt}%
\pgfpathmoveto{\pgfqpoint{1.411132in}{6.421100in}}%
\pgfpathlineto{\pgfqpoint{1.688909in}{6.421100in}}%
\pgfusepath{stroke}%
\end{pgfscope}%
\begin{pgfscope}%
\definecolor{textcolor}{rgb}{0.000000,0.000000,0.000000}%
\pgfsetstrokecolor{textcolor}%
\pgfsetfillcolor{textcolor}%
\pgftext[x=1.800020in,y=6.372489in,left,base]{\color{textcolor}\rmfamily\fontsize{10.000000}{12.000000}\selectfont ER-twoopt}%
\end{pgfscope}%
\begin{pgfscope}%
\pgfsetbuttcap%
\pgfsetroundjoin%
\pgfsetlinewidth{1.505625pt}%
\definecolor{currentstroke}{rgb}{0.172549,0.627451,0.172549}%
\pgfsetstrokecolor{currentstroke}%
\pgfsetdash{}{0pt}%
\pgfpathmoveto{\pgfqpoint{1.550020in}{6.147798in}}%
\pgfpathlineto{\pgfqpoint{1.550020in}{6.286687in}}%
\pgfusepath{stroke}%
\end{pgfscope}%
\begin{pgfscope}%
\pgfsetrectcap%
\pgfsetroundjoin%
\pgfsetlinewidth{1.505625pt}%
\definecolor{currentstroke}{rgb}{0.172549,0.627451,0.172549}%
\pgfsetstrokecolor{currentstroke}%
\pgfsetdash{}{0pt}%
\pgfpathmoveto{\pgfqpoint{1.411132in}{6.217243in}}%
\pgfpathlineto{\pgfqpoint{1.688909in}{6.217243in}}%
\pgfusepath{stroke}%
\end{pgfscope}%
\begin{pgfscope}%
\definecolor{textcolor}{rgb}{0.000000,0.000000,0.000000}%
\pgfsetstrokecolor{textcolor}%
\pgfsetfillcolor{textcolor}%
\pgftext[x=1.800020in,y=6.168632in,left,base]{\color{textcolor}\rmfamily\fontsize{10.000000}{12.000000}\selectfont ER-greedy}%
\end{pgfscope}%
\begin{pgfscope}%
\pgfsetbuttcap%
\pgfsetroundjoin%
\pgfsetlinewidth{1.505625pt}%
\definecolor{currentstroke}{rgb}{0.839216,0.152941,0.156863}%
\pgfsetstrokecolor{currentstroke}%
\pgfsetdash{}{0pt}%
\pgfpathmoveto{\pgfqpoint{1.550020in}{5.941974in}}%
\pgfpathlineto{\pgfqpoint{1.550020in}{6.080863in}}%
\pgfusepath{stroke}%
\end{pgfscope}%
\begin{pgfscope}%
\pgfsetrectcap%
\pgfsetroundjoin%
\pgfsetlinewidth{1.505625pt}%
\definecolor{currentstroke}{rgb}{0.839216,0.152941,0.156863}%
\pgfsetstrokecolor{currentstroke}%
\pgfsetdash{}{0pt}%
\pgfpathmoveto{\pgfqpoint{1.411132in}{6.011419in}}%
\pgfpathlineto{\pgfqpoint{1.688909in}{6.011419in}}%
\pgfusepath{stroke}%
\end{pgfscope}%
\begin{pgfscope}%
\definecolor{textcolor}{rgb}{0.000000,0.000000,0.000000}%
\pgfsetstrokecolor{textcolor}%
\pgfsetfillcolor{textcolor}%
\pgftext[x=1.800020in,y=5.962808in,left,base]{\color{textcolor}\rmfamily\fontsize{10.000000}{12.000000}\selectfont ER-custom}%
\end{pgfscope}%
\begin{pgfscope}%
\pgfsetbuttcap%
\pgfsetroundjoin%
\pgfsetlinewidth{1.505625pt}%
\definecolor{currentstroke}{rgb}{0.580392,0.403922,0.741176}%
\pgfsetstrokecolor{currentstroke}%
\pgfsetdash{}{0pt}%
\pgfpathmoveto{\pgfqpoint{1.550020in}{5.738117in}}%
\pgfpathlineto{\pgfqpoint{1.550020in}{5.877006in}}%
\pgfusepath{stroke}%
\end{pgfscope}%
\begin{pgfscope}%
\pgfsetrectcap%
\pgfsetroundjoin%
\pgfsetlinewidth{1.505625pt}%
\definecolor{currentstroke}{rgb}{0.580392,0.403922,0.741176}%
\pgfsetstrokecolor{currentstroke}%
\pgfsetdash{}{0pt}%
\pgfpathmoveto{\pgfqpoint{1.411132in}{5.807562in}}%
\pgfpathlineto{\pgfqpoint{1.688909in}{5.807562in}}%
\pgfusepath{stroke}%
\end{pgfscope}%
\begin{pgfscope}%
\definecolor{textcolor}{rgb}{0.000000,0.000000,0.000000}%
\pgfsetstrokecolor{textcolor}%
\pgfsetfillcolor{textcolor}%
\pgftext[x=1.800020in,y=5.758951in,left,base]{\color{textcolor}\rmfamily\fontsize{10.000000}{12.000000}\selectfont MR-swap}%
\end{pgfscope}%
\begin{pgfscope}%
\pgfsetbuttcap%
\pgfsetroundjoin%
\pgfsetlinewidth{1.505625pt}%
\definecolor{currentstroke}{rgb}{0.549020,0.337255,0.294118}%
\pgfsetstrokecolor{currentstroke}%
\pgfsetdash{}{0pt}%
\pgfpathmoveto{\pgfqpoint{1.550020in}{5.534260in}}%
\pgfpathlineto{\pgfqpoint{1.550020in}{5.673149in}}%
\pgfusepath{stroke}%
\end{pgfscope}%
\begin{pgfscope}%
\pgfsetrectcap%
\pgfsetroundjoin%
\pgfsetlinewidth{1.505625pt}%
\definecolor{currentstroke}{rgb}{0.549020,0.337255,0.294118}%
\pgfsetstrokecolor{currentstroke}%
\pgfsetdash{}{0pt}%
\pgfpathmoveto{\pgfqpoint{1.411132in}{5.603704in}}%
\pgfpathlineto{\pgfqpoint{1.688909in}{5.603704in}}%
\pgfusepath{stroke}%
\end{pgfscope}%
\begin{pgfscope}%
\definecolor{textcolor}{rgb}{0.000000,0.000000,0.000000}%
\pgfsetstrokecolor{textcolor}%
\pgfsetfillcolor{textcolor}%
\pgftext[x=1.800020in,y=5.555093in,left,base]{\color{textcolor}\rmfamily\fontsize{10.000000}{12.000000}\selectfont MR-twoopt}%
\end{pgfscope}%
\begin{pgfscope}%
\pgfsetbuttcap%
\pgfsetroundjoin%
\pgfsetlinewidth{1.505625pt}%
\definecolor{currentstroke}{rgb}{0.890196,0.466667,0.760784}%
\pgfsetstrokecolor{currentstroke}%
\pgfsetdash{}{0pt}%
\pgfpathmoveto{\pgfqpoint{1.550020in}{5.330403in}}%
\pgfpathlineto{\pgfqpoint{1.550020in}{5.469292in}}%
\pgfusepath{stroke}%
\end{pgfscope}%
\begin{pgfscope}%
\pgfsetrectcap%
\pgfsetroundjoin%
\pgfsetlinewidth{1.505625pt}%
\definecolor{currentstroke}{rgb}{0.890196,0.466667,0.760784}%
\pgfsetstrokecolor{currentstroke}%
\pgfsetdash{}{0pt}%
\pgfpathmoveto{\pgfqpoint{1.411132in}{5.399847in}}%
\pgfpathlineto{\pgfqpoint{1.688909in}{5.399847in}}%
\pgfusepath{stroke}%
\end{pgfscope}%
\begin{pgfscope}%
\definecolor{textcolor}{rgb}{0.000000,0.000000,0.000000}%
\pgfsetstrokecolor{textcolor}%
\pgfsetfillcolor{textcolor}%
\pgftext[x=1.800020in,y=5.351236in,left,base]{\color{textcolor}\rmfamily\fontsize{10.000000}{12.000000}\selectfont MR-greedy}%
\end{pgfscope}%
\begin{pgfscope}%
\pgfsetbuttcap%
\pgfsetroundjoin%
\pgfsetlinewidth{1.505625pt}%
\definecolor{currentstroke}{rgb}{0.498039,0.498039,0.498039}%
\pgfsetstrokecolor{currentstroke}%
\pgfsetdash{}{0pt}%
\pgfpathmoveto{\pgfqpoint{1.550020in}{5.124579in}}%
\pgfpathlineto{\pgfqpoint{1.550020in}{5.263468in}}%
\pgfusepath{stroke}%
\end{pgfscope}%
\begin{pgfscope}%
\pgfsetrectcap%
\pgfsetroundjoin%
\pgfsetlinewidth{1.505625pt}%
\definecolor{currentstroke}{rgb}{0.498039,0.498039,0.498039}%
\pgfsetstrokecolor{currentstroke}%
\pgfsetdash{}{0pt}%
\pgfpathmoveto{\pgfqpoint{1.411132in}{5.194023in}}%
\pgfpathlineto{\pgfqpoint{1.688909in}{5.194023in}}%
\pgfusepath{stroke}%
\end{pgfscope}%
\begin{pgfscope}%
\definecolor{textcolor}{rgb}{0.000000,0.000000,0.000000}%
\pgfsetstrokecolor{textcolor}%
\pgfsetfillcolor{textcolor}%
\pgftext[x=1.800020in,y=5.145412in,left,base]{\color{textcolor}\rmfamily\fontsize{10.000000}{12.000000}\selectfont MR-custom}%
\end{pgfscope}%
\begin{pgfscope}%
\pgfsetbuttcap%
\pgfsetroundjoin%
\pgfsetlinewidth{1.505625pt}%
\definecolor{currentstroke}{rgb}{0.737255,0.741176,0.133333}%
\pgfsetstrokecolor{currentstroke}%
\pgfsetdash{}{0pt}%
\pgfpathmoveto{\pgfqpoint{1.550020in}{4.920722in}}%
\pgfpathlineto{\pgfqpoint{1.550020in}{5.059611in}}%
\pgfusepath{stroke}%
\end{pgfscope}%
\begin{pgfscope}%
\pgfsetrectcap%
\pgfsetroundjoin%
\pgfsetlinewidth{1.505625pt}%
\definecolor{currentstroke}{rgb}{0.737255,0.741176,0.133333}%
\pgfsetstrokecolor{currentstroke}%
\pgfsetdash{}{0pt}%
\pgfpathmoveto{\pgfqpoint{1.411132in}{4.990166in}}%
\pgfpathlineto{\pgfqpoint{1.688909in}{4.990166in}}%
\pgfusepath{stroke}%
\end{pgfscope}%
\begin{pgfscope}%
\definecolor{textcolor}{rgb}{0.000000,0.000000,0.000000}%
\pgfsetstrokecolor{textcolor}%
\pgfsetfillcolor{textcolor}%
\pgftext[x=1.800020in,y=4.941555in,left,base]{\color{textcolor}\rmfamily\fontsize{10.000000}{12.000000}\selectfont NMR-swap}%
\end{pgfscope}%
\begin{pgfscope}%
\pgfsetbuttcap%
\pgfsetroundjoin%
\pgfsetlinewidth{1.505625pt}%
\definecolor{currentstroke}{rgb}{0.090196,0.745098,0.811765}%
\pgfsetstrokecolor{currentstroke}%
\pgfsetdash{}{0pt}%
\pgfpathmoveto{\pgfqpoint{1.550020in}{4.716864in}}%
\pgfpathlineto{\pgfqpoint{1.550020in}{4.855753in}}%
\pgfusepath{stroke}%
\end{pgfscope}%
\begin{pgfscope}%
\pgfsetrectcap%
\pgfsetroundjoin%
\pgfsetlinewidth{1.505625pt}%
\definecolor{currentstroke}{rgb}{0.090196,0.745098,0.811765}%
\pgfsetstrokecolor{currentstroke}%
\pgfsetdash{}{0pt}%
\pgfpathmoveto{\pgfqpoint{1.411132in}{4.786309in}}%
\pgfpathlineto{\pgfqpoint{1.688909in}{4.786309in}}%
\pgfusepath{stroke}%
\end{pgfscope}%
\begin{pgfscope}%
\definecolor{textcolor}{rgb}{0.000000,0.000000,0.000000}%
\pgfsetstrokecolor{textcolor}%
\pgfsetfillcolor{textcolor}%
\pgftext[x=1.800020in,y=4.737698in,left,base]{\color{textcolor}\rmfamily\fontsize{10.000000}{12.000000}\selectfont NMR-twoopt}%
\end{pgfscope}%
\begin{pgfscope}%
\pgfsetbuttcap%
\pgfsetroundjoin%
\pgfsetlinewidth{1.505625pt}%
\definecolor{currentstroke}{rgb}{0.121569,0.466667,0.705882}%
\pgfsetstrokecolor{currentstroke}%
\pgfsetdash{}{0pt}%
\pgfpathmoveto{\pgfqpoint{1.550020in}{4.513007in}}%
\pgfpathlineto{\pgfqpoint{1.550020in}{4.651896in}}%
\pgfusepath{stroke}%
\end{pgfscope}%
\begin{pgfscope}%
\pgfsetrectcap%
\pgfsetroundjoin%
\pgfsetlinewidth{1.505625pt}%
\definecolor{currentstroke}{rgb}{0.121569,0.466667,0.705882}%
\pgfsetstrokecolor{currentstroke}%
\pgfsetdash{}{0pt}%
\pgfpathmoveto{\pgfqpoint{1.411132in}{4.582452in}}%
\pgfpathlineto{\pgfqpoint{1.688909in}{4.582452in}}%
\pgfusepath{stroke}%
\end{pgfscope}%
\begin{pgfscope}%
\definecolor{textcolor}{rgb}{0.000000,0.000000,0.000000}%
\pgfsetstrokecolor{textcolor}%
\pgfsetfillcolor{textcolor}%
\pgftext[x=1.800020in,y=4.533841in,left,base]{\color{textcolor}\rmfamily\fontsize{10.000000}{12.000000}\selectfont NMR-greedy}%
\end{pgfscope}%
\begin{pgfscope}%
\pgfsetbuttcap%
\pgfsetroundjoin%
\pgfsetlinewidth{1.505625pt}%
\definecolor{currentstroke}{rgb}{1.000000,0.498039,0.054902}%
\pgfsetstrokecolor{currentstroke}%
\pgfsetdash{}{0pt}%
\pgfpathmoveto{\pgfqpoint{1.550020in}{4.307183in}}%
\pgfpathlineto{\pgfqpoint{1.550020in}{4.446072in}}%
\pgfusepath{stroke}%
\end{pgfscope}%
\begin{pgfscope}%
\pgfsetrectcap%
\pgfsetroundjoin%
\pgfsetlinewidth{1.505625pt}%
\definecolor{currentstroke}{rgb}{1.000000,0.498039,0.054902}%
\pgfsetstrokecolor{currentstroke}%
\pgfsetdash{}{0pt}%
\pgfpathmoveto{\pgfqpoint{1.411132in}{4.376628in}}%
\pgfpathlineto{\pgfqpoint{1.688909in}{4.376628in}}%
\pgfusepath{stroke}%
\end{pgfscope}%
\begin{pgfscope}%
\definecolor{textcolor}{rgb}{0.000000,0.000000,0.000000}%
\pgfsetstrokecolor{textcolor}%
\pgfsetfillcolor{textcolor}%
\pgftext[x=1.800020in,y=4.328017in,left,base]{\color{textcolor}\rmfamily\fontsize{10.000000}{12.000000}\selectfont NMR-custom}%
\end{pgfscope}%
\end{pgfpicture}%
\makeatother%
\endgroup%
}
\end{figure}

\begin{thebibliography}{}
\bibitem{2-opt} 
Matthias,Heiko,Berthold: Worst Case and Probabilistic Analysis of the 2-Opt Algorithm for the TSP [2013],
\\\texttt{http://www-cs-faculty.stanford.edu/\~{}uno/abcde.html}
\bibitem{nearest} 
Weihuang,Jeffrey: Investigating TSP Heuristics for Location-Based Services [2017]
,
\\\texttt{
https://link.springer.com/article/10.1007/s41019-016-0030-0}
\end{thebibliography}

\end{document}